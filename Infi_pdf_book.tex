\documentclass{tstextbook}

\usepackage{amsmath}
\usepackage{amssymb}
\usepackage{graphicx}
\usepackage{hyperref}
\usepackage{xcolor}

\begin{document}

\title{Example Document}
\author{HTML2LaTeX Converter}
\maketitle

\Chapter{פונקציות של משתנה ממשי}

\section{תכונות של מספרים ממשיים}

\begin{symbolize}
עבור \(A, B\subseteq \mathbb{R}\) נסמן \(A\leq B\) בשביל להגיד:
$$\forall a \in A\quad \forall b \in B\qquad a\leq b$$
כאשר אם \(c \in \mathbb{R}\) נסמן \(A\leq c\) בשביל להגיד:
$$\forall a \in A\quad a\leq c$$

\end{symbolize}
\begin{definition}[אקסיומת השלמות]
תהיו \(A,B \subseteq \mathbb{R}\) לא ריקות כך ש-\(A\leq B\). אזי קיים \(c \in \mathbb{R}\) כך שמתקיים \(A\leq c \leq B\).

\end{definition}
\begin{definition}[קבוצה חסומה מלעיל]
קבוצה \(A\subseteq \mathbb{R}\) נקראת חסומה מלעיל אם קיים \(c \in \mathbb{R}\) כך שמתקיים \(A\leq c\).

\end{definition}
\begin{definition}[חסם מלעיל]
תהי \(A\subseteq \mathbb{R}\) קבוצה. אזי \(c \in \mathbb{R}\) נקרא חסם מלעיל של \(A\) אם מתקיים:
$$\forall a \in A\quad  a \leq c$$

\end{definition}
\begin{proposition}
לכל קבוצה חסומה מלעיל קיים חסם מלעיל \(c \in \mathbb{R}\).

\end{proposition}
\begin{definition}[חסם עליון]
החסם מלעיל הקטן ביותר. כלומר אם:
$$A=\left\{  c \in \mathbb{R}\mid c \text{ is upper bound}  \right\}$$
אז \(m \in \mathbb{R}\) יהיה החסם העליון אם \(m=\min A\). מסומן \(m=\sup(A)\).

\end{definition}
\begin{proposition}
יחידות החסם העליון

\end{proposition}
\begin{proof}
נניח בשלילה שקיימים 2 חסמים עליונים \(x,y\) כך ש-\(x\neq y\). כיוון ש-\(y\) חסם עליון הוא בפרט חסם מלעיל ולכן מקיים \(x\leq y\). כיוון ש-\(x\) חסם עליון הוא בפרט חסם מלעיל ומקיים \(y\leq x\). וקיבלנו כי \(x=y\) בסתירה. ולכן קיים חסם עליון יחיד.

\end{proof}
\begin{theorem}[עקרון החסם העליון]
לכל קבוצה \(A\subseteq \mathbb{R}\) חסומה מלעיל קיים חסם עליון.

\end{theorem}
\begin{proof}
  \begin{enumerate}
    \item יהי \(L\subseteq \mathbb{R}\) קבוצה חסומה מלעיל. נסמן ב-\(U\) את קבוצת החסמים מלעיל של \(L\). 


    \item קבוצה זו אינה ריקה כיוון ש-\(L\) חסומה מלעיל והיא גם אינה חסומה מלעיל בעצמה כיוון שמתקיים \(M+1 \in U\). 


    \item לפי אקסיומת השלמות כיוון שמתקיים \(L\leq U\) קיים \(c \in \mathbb{R}\) כך ש-\(L\leq c\leq U\). 


    \item נקבל כי \(L\leq c\) ולכן חסם מלעיל של \(L\). וגם כיוון ש-\(c\leq U\) הוא החסם מלעיל המינימלי אשר לפי ההגדרה \(c=\sup L\). 


  \end{enumerate}
\end{proof}
\begin{proposition}[טענות שקולות לחסם העליון]
הטענות הבאות שקולות:

  \begin{enumerate}
    \item \(\beta=\sup A\). 


    \item \(\beta\) חסם מלעיל וכן כל \(\beta > x \in\mathbb{R}\) קיים \(x<a \leq \beta\)


    \item \(\beta\) חסם מלעיל ולכל \(\varepsilon>0\) קיים \(\alpha \in A\) כך ש-\(\beta-\varepsilon < a\leq \beta\). 


  \end{enumerate}
\end{proposition}
\begin{proof}
  \begin{itemize}
    \item נראה \(1\implies 2\):
נניח בשלילה כי 2 לא מתקיים. אז \(\beta\) הוא חסם מלעיל וקיים \(x \in \mathbb{R}\) כך ש-\(x<\beta\) וגם לא קיים \(a\) כך ש-\(x<a\). כלומר לכל \(a\) מתקיים \(a<x\). ולכן \(x\) חסם מלעיל, כיוון שקטן מהחסם העליון, בסתירה.
    \item נראה \(2\implies 3\):
נסמן \(x=\beta-\varepsilon\). מתקיים \(\beta-\varepsilon<\beta\) ולכן לפי 2 מתקיים:
$$\beta-\varepsilon<a\leq \beta$$
    \item נראה \(3\implies 1\):
נניח בשלילה כי \(\beta\) לא החסם העליון. ולכן כיוון ש-\(A\) לא ריקה קיים עבורה חסם עליון לפי משפט החסם העליון. נסמן \(x = \sup A\). נסמן \(\varepsilon=\beta-x\)(חיובי כי \(\beta>0\)) ונקבל:
$$\exists a \in A\quad x<a\leq \beta$$
בסתירה לכך ש-\(x\) החסם העליון, ולכן נקבל כי \(\beta\) החסם העליון.
  \end{itemize}
\end{proof}
\section{פונקציות}

\begin{definition}[פונקציה של משתנה ממשי]
פונקציה \(F:D\to E\) עם \(D\subseteq \mathbb{R}\) ו- \(E\subseteq \mathbb{R}\). זאת אומרת \(f\) מתאימה לכל \(x \in D\) מספר יחיד.

\end{definition}
\begin{definition}[גרף של פונקציה]
$$\Gamma_{f}=\left\{  (x,y)\mid x \in D\land y=f(x)  \right\}=\left\{  (x,f(x))\mid x \in D  \right\}$$

\end{definition}
\begin{remark}
לפעמים מציינים את \(f\) ע"י נוסחה בלבד בלי לציין תחומים. במקרים אלו נפעל ע"י המוסכמות הבאות:

  \begin{enumerate}
    \item מוסכמת התחום המקסימלית - ניקח את \(D\subseteq \mathbb{R}\) להיות הקבוצה הגדולה ביותר שעבורה \(f(x)\) מוגדרת היטב 


    \item מוסכמת הטווח המקסימלית - ניקח את \(E=\mathbb{R}\). 


  \end{enumerate}
\end{remark}
\begin{definition}[קטע פתוח וסגור]
  \begin{enumerate}
    \item קטע פתוח - קבוצה מהצורה הבאה: 
$$(a,b)\equiv\left\{  x \in \mathbb{R}\mid a<x<b  \right\}$$


    \item קטע סגור - קבוצה מהצורה הבאה: 
$$[a,b]\equiv \left\{  x \in \mathbb{R}  \mid a\leq x \leq b\right\}$$


    \item קטע חצי פתוח - קבוצה מהצורה \((a,b]\) או \([a,b)\) כאשר: 
$$\left( a,b]=\left\{  x \in \mathbb{R}\mid a<x\leq b  \right\}\qquad [a,b \right)=\left\{  x \in \mathbb{R}\mid a\leq x< b  \right\}$$


    \item קרן פתוחה - קטע פתוח שאחד הגבולות שלו הוא אינסוף: 
$$\left( -\infty,a \right)\equiv \left\{  x \in \mathbb{R} \mid x< a \right\}\qquad \left( a,\infty \right)\equiv\left\{  x \in \mathbb{R}\mid x> a  \right\}$$


    \item קרן סגורה - קטע חצי סגור שאחד הגבולות שלו הוא אינסוף. החלק עם האינסוף פתוח כי אנחנו לא כוללים את "אינסוף": 
$$\left[ a,\infty)=\left\{ x \in \mathbb{R}\mid a\leq x   \right\}\qquad (-\infty,b \right]=\left\{  x \in \mathbb{R}\mid x\leq b  \right\}$$


  \end{enumerate}
\end{definition}
\begin{definition}[סביבה של נקודה]
יהי \(x_{0} \in \mathbb{R}\). סביבה של \(x_{0}\) היא תת קבוצה של \(\mathbb{R}\) מהצורה \((x_{0}-h,x_{0}+h)\) לאיזשהו \(0<h \in \mathbb{R}\)

\end{definition}
\begin{definition}[סביבה מנוקבת של נקודה]
יהי \(x_{0} \in \mathbb{R}\). סביבה מנוקבת של \(x_{0}\) היא תת קבוצה של \(\mathbb{R}\) מהצורה:
$$(x_{0}-h,x_{0}+h)\setminus \{ x_{0} \}=(x_{0}-h, x_{0})\cup (x_{0},x_{0}+h)$$

\end{definition}
\begin{definition}[סביבה חד צדדית של נקודה]
  \begin{enumerate}
    \item סביבה ימינית של \(x_{0}\) היא קבוצה מהצורה \([x_{0},x_{0}+h)\). 


    \item סביבה ימנית מנוקבת של \(x_{0}\) היא קבוצה מהצורה \((x_{0},x_{0}+h)\)


    \item סביבה שמאלית של \(x_{0}\) היא קבוצה מהצורה \((x_{0}-h,x_{0}]\). 


    \item סביבה שמאלית מנוקבת של \(x_{0}\) היא קבוצה מהצורה \((x_{0}-h,x_{0})\)


  \end{enumerate}
\end{definition}
\begin{definition}[פונקציה חסומה מלעיל]
פונקציה \(f:D\to E\) נקראת חסומה מלעיל אם קיים \(M \in \mathbb{R}\) כך שלכל \(x \in D\) מתקיים:
$$f(x)\leq M$$

\end{definition}
\begin{definition}[פונקציה חסומה מלרע]
פונקציה \(f:D\to E\) נקראת חסומה מלעיל אם קיים \(M \in \mathbb{R}\) כך שלכל \(x \in D\) מתקיים:
$$f(x)\geq M$$

\end{definition}
\begin{definition}[פונקציה חסומה]
פונקציה \(f:D\to E\) נקראת חסומה גם למעיל וגם מלרע. כלומר קיימים \(M_{1},M_{2}\in \mathbb{R}\) כך שלכל \(x \in D\) מתקיים:
$$M_{1}\leq x \leq M_{2}$$

\end{definition}
\begin{corollary}
פונקציה חסומה אם"ם קיים \(M \in \mathbb{R}\) כך שלכל \(x \in D\) מתקיים \(|x|<M\).

\end{corollary}
\begin{definition}[פונקציות מוטוניות מונוטונית]
יהי \(f:D\to\mathbb{R}\). יהיו \(x_{1},x_{2} \in D\).

  \begin{enumerate}
    \item פונקציה נקראת מונטונית עולה אם מתקיים: 
$$x_{1}< x_{2}\implies f(x_{1})\leq f(x_{2})$$


    \item פונקציה נקראת מונטונית עולה ממש אם מתקיים: 
$$x_{1}<x_{2}\implies f(x_{1})<f(x_{2})$$


    \item פונקציה נקראת מונטונית יורדת אם מתקיים: 
$$x_{1}<x_{2}\implies f(x_{2})\leq f(x_{1})$$


    \item פונקציה נקראת מונטונית יורדת ממש אם מתקיים: 
$$x_{1}<x_{2}\implies f(x_{2})<f(x_{1})$$


  \end{enumerate}
\end{definition}
\begin{corollary}
  \begin{enumerate}
    \item אם פונקציה היא מונוטונית עולה ממש אז היא בפרט מונוטנית עולה. 


    \item אם פונקציה היא מונטונית יורדת ממש אז היא בפרט מונטונית יורדת. 


    \item אם פונקציה היא קבועה אז היא גם מונוטונית יורדת וגם מונוטרית עולה, אך בכל מקרה אחר לא ייתכן שיקיים את שניהם. 


  \end{enumerate}
\end{corollary}
\section{גבולות של פונקציות}

\begin{definition}[גבול של פונקציה ממשית בנקודה]
יהי \(x_{0} \in \mathbb{R}\), ותהי \(f:D\to E\) פונקציה ממשית המוגדרת בסביבה מנוקבת של \(x_{0}\). נאמר ש-\(L\) הוא הגבול של \(f\) ב-\(x_{0}\) אם מתקיים:
$$\forall \varepsilon>0\quad \exists \delta > 0\quad  \forall x \in D\quad \left( 0<\lvert x-x_{0} \rvert <\delta\implies \lvert f(x)-L \rvert <\varepsilon \right)$$

\end{definition}
\begin{proposition}[יחידות הגבול]
אם מתקיים:
$$(1)\lim_{ x \to x_{0} } f(x)=L_{1}\qquad (2)\lim_{ x \to x_{0} } f(x)=L_{2}$$
אז \(L_{1}=L_{2}\)

\end{proposition}
\begin{proof}
נניח בשלילה כי \(L_{2}\neq L_{1}\). נניח בלי הגבלת הכלליות כי \(L_{2} > L_{1}\). מגבול \((1)\) נקבל כי עבור \(\varepsilon=\frac{L_{2}-L_{1}}{2}>0\) קיים \(\delta_{1}\) כך ש:
$$0<\lvert x-x_{0} \rvert <\delta_{1}\implies \lvert f(x)-L_{1} \rvert < \frac{L_{2}-L_{1}}{2}$$
מגבול \((2)\) נקבל כי עבור \(\varepsilon=\frac{L_{2}-L_{1}}{2}\) כי:
$$0<\lvert x - x_{0} \rvert <\delta_{1}\implies \lvert f(x)-L_{2} \rvert < \frac{L_{2}-L_{1}}{2}$$
ולכן אם ניקח \(\lvert x-x_{0} \rvert<\min\left( \delta_{1},\delta_{2} \right)\) נקבל כי:
$$\frac{L_{1}+L_{2}}{2}<f(x)<\frac{3L_{2}-L_{1}}{2}\;\;\land \;\;\frac{3L_{1}-L_{2}}{2}<f(x)<\frac{L_{2}+L_{1}}{2}$$
ומטרנזטיביות נקבל כי \(f(x)<f(x)\) בסתירה, ולכן \(L_{1}=L_{2}\).

\end{proof}
\begin{proposition}[אפיון היינה לגבולות]
תהי \(f:D\to \mathbb{R}\) פונקציה ממשית. יהי \(x_{0} \in \mathbb{R}\). אם כל סדרה \((x_{n})\) המקיימת:

  \begin{enumerate}
    \item לכל \(n \in \mathbb{N}\) מתקיים \(x_{n}\in D\). 


    \item לכל \(n \in N\) מתקיים \(x_{n} \neq x_{0}\). 


    \item הגבול של הסדרה שואפת ל-\(x_{0}\). 
מקיימת \(f(x_{n})\xrightarrow{n\to \infty}f(x_{0})\) אם"ם מתקיים \(\lim_{ x \to x_{0} }f(x)=L\)


  \end{enumerate}
\end{proposition}
\begin{itemize}
  \item נראה ראשית את הכיוון \(\impliedby\). כלומר נניח כי כל סדרה \((x_{n})\) המקיימת את תנאים \(1,2,3\) תקיים \(f(x_{n})\to f(x_{0})\).
נניח בשלילה שלא מקיים \(\lim_{ x \to x_{0} }f(x)=L\). כלומר:
$$(i)\quad \exists\varepsilon>0 \quad \forall \delta> 0 \quad \exists x \in D\quad  0 <\lvert x-x_{0} \rvert <\delta \land \lvert f(x)-L \rvert \geq \varepsilon$$
נסמן את ה-\(\varepsilon\) הזה(אשר קיומו מובטח מהפסוק \((i)\)) ב-\(\varepsilon'\). נבנה סדרה \((a_n)_{n=1}^\infty\) באופן הבא:
האיבר \(a_{n}\) יהיה ה-\(x\) שמקבלית מ-\((i)\) עבור \(\varepsilon=\varepsilon'\). ו-\(\delta=\frac{1}{n}\). מתקיים:
$$0<\lvert a_{n}-x_{0} \rvert <\frac{1}{n}\implies x_{0}\neq a_{n}\land -\frac{1}{n}+x_{0}<a_{n}<\frac{1}{n}+x_{0}$$
כאשר נשים לב כי ממשפט הכריך \((a_{n})\) שואף ל-\(x_{0}\) כאשר \(n\to \infty\). וכן כיוון ש-\((i)\) מגדיר \(x \in D\) אז גם:
$$\forall n \in \mathbb{N}\quad a_{n}\in D$$
ולכן הסדרה \((a_n)_{n=1}^\infty\) מקיימת את תנאים \(1,2,3\) ולכן מקיימת \(\lim_{ n \to \infty }f(a_{n})=L\) אבל לפי \((i)\) מתקיים \(\lvert f(x)-L \rvert\geq \varepsilon\) בסתירה.
  \item כעת נראה את הכיוון \(\implies\). נניח כי \(\lim_{ x \to x_{0} }f(x)=L\). יהי \((x_{n})\) סדרה המקיימת \(1,2,3\). לפי הגדרת הגבול מתקיים:
$$(*)\quad \forall \varepsilon>0\quad \exists \delta>0\quad \forall x \in D\quad 0<\lvert x-x_{0} \rvert <\delta\implies \lvert f(x)-L \rvert <\varepsilon$$
וכן מתנאי 3 מתקיים:
$$(* *)\quad \forall\varepsilon>0\quad \exists N \in \mathbb{N}\quad  \forall n \in \mathbb{N}\quad n>N\implies \lvert x_{n}-x_{0} \rvert <\varepsilon$$
יהי \(\varepsilon> 0\). מ-\((*)\) נקבל \(\delta\) כך שלכל \(x \in D\) מתקיים:
$$\lvert x_{n}-x_{0} \rvert <\delta\implies \lvert f(x_{n}) -f(x_{0})\rvert <\varepsilon$$
מ-\((* *)\) עבור \(\varepsilon = \delta\) נקבל \(N \in \mathbb{N}\) כך שלכל \(n> N\) מתקיים \(\lvert x_{n}-x_{0} \rvert<\delta\). כיוון שלפי תנאי 2 מתקיים כי לכל \(n \in \mathbb{N}\) מתקיים \(x_{n}\neq x_{0}\) גם מתקיים \(0<\lvert x_{n}-x_{0} \rvert<\delta\) אבל מ-\((*)\) זה גורר \(\lvert f(x_{n})-L \rvert<\varepsilon\) ולכן \(\lim_{ n \to \infty }x_{n}=L\).
\end{itemize}
\begin{definition}[גבול חד צדדי של פונקציה בנקודה]
יהי \(x_{0} \in \mathbb{R}\), ו-\(f:D\to E\) פונקציה ממשית.

  \begin{enumerate}
    \item קיים גבול ימיני אם קיימת סביבה ימנית מנוקבת של \(x_{0}\) כך שמתקיים: 
$$\forall \varepsilon > 0\quad \exists \delta>0\quad \forall x \in D \quad 0<x-x_{0}<\delta\implies \lvert f(x)-L \rvert <\varepsilon$$


    \item קיים גבול שמאלי אם קיימת שביבה שמאלית מנוקבת של \(x_{0}\) כך שמתקיים: 
$$\forall \varepsilon > 0\quad \exists \delta>0\quad \forall x \in D \quad 0<x_{0}-x<\delta\implies \lvert f(x)-L \rvert <\varepsilon$$


  \end{enumerate}
\end{definition}
\begin{proposition}[יחידות הגבול החד צצדי]
  \begin{enumerate}
    \item אם קיים גבול ימני ב-\(x_{0}\) אז הוא יחיד 


    \item אם קיים גבול ימני ב-\(x_{0}\) אז הוא יחיד 


  \end{enumerate}
\end{proposition}
\begin{symbolize}
  \begin{enumerate}
    \item אם \(L\) הוא הגבול של \(f\) ב-\(x_{0}\) נסמן \(\underset{ x \to x_{0} }{\lim }f(x)=L\)


    \item אם \(L\) הוא הגבול הימיני של \(f\) ב-\(x_{0}\) נסמן \(\underset{ x \to x_{0}^{+} }{\lim }f(x)=L\)


    \item אם \(L\) הוא הגבול השמאלי של \(f\) ב-\(x_{0}\) נסמן \(\underset{ x \to x_{0}^{-} }{\lim }f(x)=L\)


  \end{enumerate}
\end{symbolize}
\begin{proposition}
תהי \(f:D\to \mathbb{R}\) פונקציה המוגדרת בסביבה מנוקבת של \(x_{0}\). אזי \(f\) בעלת גבול ב-\(x_{0}\) אם"ם 3 התנאים הבאים מתקיים:

  \begin{enumerate}
    \item הגבול הימיני מוגדר ב-\(x_{0}\). 


    \item הגבול השמאלי מוגדר ב-\(x_{0}\). 


    \item מתקיים \(\underset{ x \to x_{0}^{+} }{\lim }f(x)=\underset{ x \to x_{0}^{-} }{\lim }f(x)\) 
ובמקרה זה מתקיים:
$$\lim_{ x \to x_{0} } f(x)=\lim_{ x \to x_{0}^{+} }f(x)=\lim_{ x \to x_{0}^{-} }  f(x)$$


  \end{enumerate}
\end{proposition}
\begin{theorem}[אפיון היינה לגבולות חד צדדיים]
תהי \(f:D\to \mathbb{R}\) פונקציה ממשית. יהי \(x_{0} \in \mathbb{R}\). אם כל סדרה \((x_{n})\) המקיימת:

  \begin{enumerate}
    \item לכל \(n \in \mathbb{N}\) מתקיים \(x_{n}\in D\). 


    \item לכל \(n \in N\) מתקיים \(x_{n} > x_{0}\). 


    \item הגבול של הסדרה שואפת ל-\(x_{0}\). 
מקיימת \(f(x_{n})\xrightarrow{n\to \infty}f(x_{0})\) אם"ם מתקיים \(\lim_{ x \to x_{0}^{+} }f(x)=L\).
כאשר עבור גבול שמאלי תנאי 2 יהיה \(x_{n}<x_{0}\).


  \end{enumerate}
\end{theorem}
\section{גבולות במובן הרחב}

\begin{definition}[שאיפה לאינסוף בנקודה]
תהי \(f:D\to \mathbb{R}\) פונקציה המוגדרת בסביבה מנוקבת של \(x_{0} \in \mathbb{R}\). 

  \begin{enumerate}
    \item נאמר כי \(f\) שואפת לאינסוף בנקודה \(x_{0}\) אם מתקיים: 
$$\forall M \in \mathbb{R}\quad \exists \delta > 0\quad  \forall x \in D\quad 0<\lvert x-x_{0} \rvert <\delta\implies f(x) > M$$
. נאמר כי \(f\) שואפת למינוס אינסוף בנקודה \(x_{0}\) אם מתקיים:
$$\forall M \in \mathbb{R}\quad \exists \delta > 0\quad  \forall x \in D\quad 0<\lvert x-x_{0} \rvert <\delta\implies f(x) < M$$
  \end{enumerate}
\end{definition}
\begin{definition}[שאיפה לערך באינסוף]
  \begin{enumerate}
    \item נאמר ש-\(f:\left( a,\infty \right)\to \mathbb{R}\) שואף ל-\(L\) באינסוף אם מתקיים: 
$$\forall\varepsilon>0\quad \exists N \in \mathbb{R}\quad \forall x \in \mathrm{Dom}(f)\quad  x > N\implies \lvert f(x)-L \rvert <\varepsilon$$
ומסומן \(\underset{ x \to \infty }{\lim }f(x)=L\).


    \item נאמר ש-\(f:\left( -\infty,a \right)\to \mathbb{R}\) שואף ל-\(L\) במינוס אינסוף אם מתקיים: 
$$\forall\varepsilon>0\quad \exists N \in \mathbb{R}\quad \forall x \in \mathrm{Dom}(f)\quad  x < N\implies \lvert f(x)-L \rvert <\varepsilon$$
ומסומן \(\underset{ x \to -\infty }{\lim }f(x)=L\)


  \end{enumerate}
\end{definition}
\begin{definition}[שאיפה לאינסוף באינסוף]
  \begin{enumerate}
    \item נאמר ש-\(f:\left( a,\infty \right)\to \mathbb{R}\) שואף לאינסוף באינוסף אם מתקיים: 
$$\forall M  \in \mathbb{R}\quad \exists N \in \mathbb{R}\quad  \forall x \in \mathrm{Dom}(f)\quad  x > N\implies f(x)>M$$
כאשר נסמן זאת ב-\(\underset{ x \to \infty }{\lim }f(x)=\infty\)


    \item נאמר ש-\(f:\left(-\infty,a \right)\to \mathbb{R}\) שואף לאינסוף במינוס אינסוף אם מתקיים: 
$$\forall M  \in \mathbb{R}\quad \exists N \in \mathbb{R}\quad  \forall x \in \mathrm{Dom}(f)\quad  x < N\implies f(x)>M$$
כאשר נסמן זאת ב-\(\underset{ x \to -\infty }{\lim }f(x)=\infty\)


    \item נאמר ש-\(f:\left( a,\infty \right)\to \mathbb{R}\) שואף למינוס אינסוף באינוסף אם מתקיים: 
$$\forall M  \in \mathbb{R}\quad \exists N \in \mathbb{R}\quad  \forall x \in \mathrm{Dom}(f)\quad  x > N\implies f(x)<M$$
כאשר נסמן זאת ב-\(\underset{ x \to \infty }{\lim }f(x)=-\infty\)


    \item נאמר ש-\(f:\left(-\infty,a \right)\to \mathbb{R}\) שואף למינוס אינסוף במינוס אינסוף אם מתקיים: 
$$\forall M  \in \mathbb{R}\quad \exists N \in \mathbb{R}\quad  \forall x \in \mathrm{Dom}(f)\quad  x < N\implies f(x)<M$$
כאשר נסמן זאת ב-\(\underset{ x \to -\infty }{\lim }f(x)=-\infty\)


  \end{enumerate}
\end{definition}
\Chapter{פונקציות רציפות}

\section{הגדרה של פונקציה רציפה}

\begin{definition}[רציפות בנקודה]
תהי \(f:D\to \mathbb{R}\) פונקציה ו-\(x_{0} \in \mathbb{R}\). נאמר כי \(f\) רציפה ב-\(x_{0}\) אם מתקיים:

  \begin{enumerate}
    \item הפונקציה \(f\) מוגדרת בסביבה מלאה של \(x_{0}\). 


    \item קיים ל-\(f\) גבול ב-\(x_{0}\). 


    \item הגבול שווה לערך בהקודה. כלומר: 
$$\lim_{ x \to x_{0} } f(x)=f(x_{0})$$


  \end{enumerate}
\end{definition}
\begin{remark}
אנחנו הרבה פעמים משתמשים בכיוון השני, אם פונקציה היא רציפה, אז זה מאפשר למצוא את הגבול ממש בקלות! פשוט מציבים את הנקודה, וזה יהיה הגבול.

\end{remark}
\begin{definition}[פונקציה רציפה בתחום]
פונקציה \(f\) נקראת רציפה בתחום \(D\) אם לכל \(x \in D\) הפונקציה רציפה ב-\(x\). אם רציפה בכל תחום הגדרתה נאמר שהיא פונקציה רציפה.

\end{definition}
\begin{proposition}[אפיון היינה לרציפות]
תהי \(f:D\to \mathbb{R}\) פונקציה ממשית. יהי \(x_{0} \in \mathbb{R}\). אם כל סדרה \((x_{n})\) המקיימת:

  \begin{enumerate}
    \item לכל \(n \in \mathbb{N}\) מתקיים \(x_{n}\in D\). 


    \item הגבול של הסדרה שואפת ל-\(x_{0}\). 
מקיימת \(f(x_{n})\xrightarrow{n\to \infty}f(x_{0})\) אם"ם מתקיים \(\lim_{ x \to x_{0} }f(x)=L\)


  \end{enumerate}
\end{proposition}
\begin{theorem}[אריתמטיקה של פונקציות רציפות]
יהיו \(g,f\) פונקציות רציפות ב-\(x_{0} \in \mathbb{R}\). 

  \begin{enumerate}
    \item הסכום \(f+g\) יהיה רציף ב-\(x_{0}\)


    \item המכפלה \(f\cdot g\) תהיה רציפה ב-\(x_{0}\). 


    \item אם בנוסף \(0\neq g(x_{0})\) אז \(\frac{1}{g}\) רציפה ב-\(x_{0}\), וכן במקרה זה גם \(\frac{f}{g}\) רציפה ב-\(x_{0}\). 


    \item אם \(\lambda \in \mathbb{R}\) אז \(\lambda f\) רציפה ב-\(x_{0}\). 


    \item ההפרש \(f-g\) רציפה ב-\(x_{0}\). 


  \end{enumerate}
\end{theorem}
\begin{proof}
נובע ישירות מאריתמטיקה של גבולות של פונקציות.

\end{proof}
\begin{definition}[רציפות חד צצדית]
תהי \(f:D\to \mathbb{R}\) ו-\(x_{0} \in \mathbb{R}\).

  \begin{enumerate}
    \item אם מתקיים \(\underset{ x \to x_{0}^{+} }{\lim }f(x)=f(x_{0})\) אז נאמר כי רציף מימין. 


    \item אם מתקיים \(\underset{ x \to x_{0}^{-} }{\lim }f(x)=f(x_{0})\) אז נאמר כי רציף משמאל. 


  \end{enumerate}
\end{definition}
\begin{proposition}
פונקציה רציפה בנקודה אם"ם רציפה גם משמאל וגם מימין.

\end{proposition}
\begin{theorem}[הרכבה של פונקציות רציפות]
  \begin{enumerate}
    \item תהי \(f\) פונקציה המוגדרת בסיבה מנוקבת של \(x_{0}\) כך ש-\(\underset{ x \to x_{0} }{\lim }f(x)=y_{0}\). תהי \(g\) רציפה ב-\(y_{0}\). אזי \(g\circ f\) בעלת גבול ב-\(x_{0}\) ומתקיים: 
$$\lim_{ x \to x_{0} } g(f(x))=\lim_{ x \to x_{0} } \left( g \circ  f \right)(x)=g(y_{0})$$


    \item אם בנוסף \(f\)  רציפה ב-\(x_{0}\) אז מתקיים \(g \circ f\) רציפה ב-\(x_{0}\). 


  \end{enumerate}
\end{theorem}
\begin{proof}
  \begin{enumerate}
    \item יהי \(\varepsilon>0\). תהי \(g\) רציפה ב-\(y_{0}\). לכן קיים \(\delta_{1}>0\) כך שלכל \(y \in \mathrm{Dom}(g)\) מתקיים: 
$$0<\lvert x-x_{0} \rvert <\delta\implies \lvert g(y)-g(y_{0}) \rvert <\varepsilon$$
עכשיו \(\underset{ x \to x_{0} }{\lim }f(x)=y\) נותן \(\delta>0\) כך שלכל \(x \in \mathrm{Dom}\left( g \circ f \right)\) מתקיים:
\begin{gather*}0<\lvert x-x_{0} \rvert <\delta\implies \lvert f(x)-y_{0} \rvert <\delta_{1}\implies \lvert g(f(x))-y_{0} \rvert <\delta_{1} \implies \\\implies \lvert g(f(x))-g(y_{0}) \rvert <\varepsilon \implies \lim_{ x \to x_{0} } \left( g\circ f \right)(x_{0})=g(y_{0})
\end{gather*}


    \item כעת נתון כי \(f\) רציפה ב-\(x_{0}\). לכן \(\underset{ x \to x_{0} }{\lim }f(x)=f(x_{0})=y\). נקבל כי: 
$$\lim_{ x \to x_{0} } g \circ f(x)=g(f(x_{0}))=\left( g\circ f \right)(x_{0})$$
זאת אומרת \(g\circ f\) רציפה ב-\(x_{0}\).


  \end{enumerate}
\end{proof}
\section{משפטים על פונקציות רציפות}

\begin{theorem}[ערך הביניים]
יהי \(f:[a,b]\to \mathbb{R}\) פונקציה רציפה. לכל \(\lambda \in \left[ \min(f(a),f(b)),\max(f(a),f(b)) \right]\) קיים \(c \in [a,b]\) כך ש-\(f(c)=\lambda\).

\end{theorem}
\begin{proof}
  \begin{enumerate}
    \item נניח בלי הגבלת הכלליות כי \(f(a)\leq f(b)\). נגדיר: 
$$g(x)=f(x)-\lambda \qquad S=\left\{  x\mid g(x)\leq 0  \right\}$$


    \item אם \(f(a)=\lambda\) או \(f(b)=\lambda\) סיימנו. אחרת \(f(a)<\lambda\) וגם \(f(b)>\lambda\), ולכן \(g(a)<0\) וגם \(g(b)>0\). 


    \item הקבוצה \(S\) חסומה כיוון ש-\(a\leq x \leq b\) ולכן כיוון שלא ריקה(\(a \in S\)) לפי משפט החסם העליון \(\sup S\) מוגדר. נסמן \(c=\sup S\)


    \item אם \(0<g(c)\) כיוון שהפונקציה רציפה קיימת סביבה מלאה \(A=(c-h,c+h)\) כך ש-\(A\geq 0\). עבור למשל \(x' = c- \frac{h}{2}\in A\) נקבל \(x'< c\) וגם \(0\leq g(x')\) בסתירה לכך ש-\(c\) החסם התחתון. לכן \(g(c)\leq 0\). 


    \item אם \(0> g(c)\) כיוון שרציף קיימת סביבה מלאה \(A=(c-h,c+h)\) כך ש-\(\forall x \in A\quad g(x)<0\). עבור למשל \(g\left( c+\frac{h}{2} \right)<0\) נקבל \(c+\frac{h}{2}\in S\) בסתירה לכך ש-\(c\) החסם העליון ולכן \(g(c)\geq 0\). 


    \item משלבים 4+5 נקבל כי \(g(c)=0\) ולכן \(f(c)=\lambda\). 


  \end{enumerate}
\end{proof}
\begin{theorem}[ווירשטראס הראשון]
כל פונקציה \(f:[a,b]\to \mathbb{R}\) רציפה בקטע סדור היא חסומה

\end{theorem}
\begin{proof}
  \begin{enumerate}
    \item נניח בשלילה כי \(f(x)\) לא חסומה. לכן: 
$$(i)\quad \forall M \in \mathbb{R}\quad \exists x \in [a,b]\quad f(x)>M$$


    \item לכן קיימת סדרה ששואפת לאינסוף - יהי \(n \in \mathbb{N}\). נגדיר \(a_{n}\) בתור ה-\(x\) המתקבל מ-\((i)\) עבור \(M=n\). ממשפט הפרוסה כיוון ש \(n\to \infty\) נקבל \(f(a_{n})\to \infty\). 


    \item כיוון שלכל \(n \in \mathbb{N}\) מתקיים \(a_{n}\in [a,b]\) מתקיים ש-\((a_{n})\) חסומה. לכן מבולצנו ווירשטאס נקבל תת סדרה \((a_{n_{k}})\) אשר מתכנסת. נגדיר \(\lim_{ n \to \infty }a_{n_{k}}=x_{0}\)


    \item מרציפות נקבל \(\lim_{ n \to \infty }f(a_{n_{k}})=f(x_{0})\) כאשר מירושה נקבל \(\lim_{ k \to \infty }f(a_{n_{k}})=\infty\) בסתירה, ולכן חסומה 


  \end{enumerate}
\end{proof}
\begin{theorem}[ווירשטראס השני]
אם \(f\) מוגדר ורציף בקטע סגור \([a,b]\) אז מקבל ערך מינימלי ומקסימלי.

\end{theorem}
\begin{proof}
  \begin{enumerate}
    \item כיוון ש-\(f\) רציפה ב-\([a,b]\) לפי ווירשטראס הראשון נקבל כי \(f(x)\) חסומה בתחום, ולכן \(M=\sup(f(x))\) מוגדר. 


    \item קיימת סדרה \((x_{n})\) המתכנסת ל-\(M\). זאת כיוון שמאפיון הסופרמום: 
$$\forall \varepsilon > 0\quad \exists x \in [a,b]\quad M-\varepsilon < f(x)$$
כעת יהי \(n \in \mathbb{N}\). נגדיר את \(x_{n}\) בתור ה-\(x\) שמתקבל עבור \(\varepsilon=\frac{1}{n}\). מסנדוויץ כיוון שמתקיים:
$$\sup M-\frac{1}{n}< f(x_{n})<\sup M+\frac{1}{n}$$
נקבל \(f(x_{n})\to M\).


    \item מבולצנו ווירשטראס קיימת תת סדרה \(x_{n_{k}}\) אשר מתכנסת - \(x_{n}\) חסומה ע"י \([a,b]\) ומתקיים \(\lim_{ k \to \infty }x_{n_{k}}=d\). 


    \item מרציפות נקבל \(f(x_{n_{k}})=f(d)\). וכן כיוון שמתקיים: 
$$\forall k \in \mathbb{N}\quad  M- \frac{1}{k}\leq M-\frac{1}{n_{k}}<f(x_{n_{k}})\leq M$$
ונקבל ממשפט הערך כי \(f(x_{n})\to M\) ולכן \(f(d)=M\) וקיים מקסימום.


  \end{enumerate}
\end{proof}
\section{תכונות של פונקציות רציפות ומונטוניות}

\begin{definition}[מיון נקודות אי רציפות]
תהי \(f\) פונקציה המוגדרת בסביבה מלאה של \(x_{0}\). אזי:

  \begin{enumerate}
    \item אם \(\underset{ x \to x_{0} }{\lim }f(x)\) קיים אבל הוא שונה מ-\(x_{0}\) נאמר ש-\(f\) בעלת אי רציפות סליקה 


    \item אם \(\underset{ x \to x_{0}^{+} }{\lim }f(x)\) ו-\(\underset{ x \to x_{0}^{-} }{\lim }f(x)\) קיימים אך שונים נאמר שקיים אי רציפות מהסוג הראשון. 


    \item אם לפתוח אחד מהגבולות החד צדדיות לא קיימות נאמר שקיים אי רציפות מהסוג השני, או אי רציפות עיקרית. 


  \end{enumerate}
\end{definition}
\begin{definition}[המשכה רציפה]
אם לפונקציה \(f:D\to \mathbb{R}\) יש אי רציפות סליקה ב-\(x_{0}\), ניתן להגדיר פונקציה חדשה כך ש-\(h:D\to \mathbb{R}\) כך:
$$\forall x \neq x_{0}\quad h(x)=f(x)\qquad h(x_{0})=\lim_{ x \to x_{0} } f(x)$$
כאשר פונקציה זו תהיה רציפה.

\end{definition}
\begin{definition}[מקטע]
קטע פתוח, סגור או קרן.

\end{definition}
\begin{proposition}
  \begin{enumerate}
    \item אם \(f\) רציפה אז התמונה של \(f\) תהיה מקטע. 


    \item אם \(f\) מונטונית ממש אז \(f\) תהיה חד חד ערכית. 


    \item אם \(f\) מונטונית והתמונה של \(f\) מקטע אז \(f\) רציפה. 


  \end{enumerate}
\end{proposition}
\begin{proposition}
תהי \(I\subseteq \mathbb{R}\) מקטע, ו-\(f:I\to \mathbb{R}\) מונטונית עולה ממש, ורציפה ב-\(I\). יהיו \(a,b \in \mathbb{R}\) עם \(a< b\).

  \begin{enumerate}
    \item אם \(I=[a,b]\) אז \(\mathrm{Im}f=[f(a),f(b)]\)


    \item אם \(I=(a,b]\) אז \(\mathrm{Im}f=[ f(a),\underset{ x \to b^{-} }{\lim }f(x) )\)


    \item אם \(I=\left( a,\infty \right)\) אז \(\mathrm{Im}f=\left( \underset{ x \to a^{+} }{\lim }f(x),\underset{ x \to \infty }{\lim }f(x) \right)\) 
כאשר כל המקרים מתקיימים גם עבור הצד שני של התחום.


  \end{enumerate}
\end{proposition}
\begin{proposition}
כל פונקציה מונוטונית ממש היא חד חד ערכית.

\end{proposition}
\begin{proposition}
לכל פונקציה מונוטונית עולה וחסומה יש גבול חד צדדי

\end{proposition}
\begin{proof}
תהי \(f\) פונקציה מונוטונית עולה וחסומה.

  \begin{enumerate}
    \item נגדיר: 
$$A=\left\{  f(x)\mid x \in (a,b)  \right\}$$
כאשר נשים לב כי \(A\) לא ריקה כיוון ש-\(f\left( \frac{a+b}{2} \right)\in A\), וכן \(A\) חסומה כיוון ש-\(f(x)\) חסום.


    \item יהי \(\varepsilon> 0\). מאפיון החסם העליון נקבל \(x_{1} \in [a,b]\) אשר יקיים: 
$$\sup A-\varepsilon < f(x_{1})\leq \sup  A$$


    \item נבחר \(\delta=b-x_{1}\). כעת לכל \(x \in (a,b)\) מתקיים: 
$$0<b-x<\delta\implies -x<-x_{1}\implies x< x_{1}$$
וכיוון שמונוטונית עולה \(f(x_{1})<f(x)\) וכיוון שחסומה \(f(x)<\sup A\) ולכן:
$$\sup  A-\varepsilon < f(x_{1}) < f(x)< \sup  A< \sup  A+\varepsilon\implies-\varepsilon<f(x)-\sup  A<\varepsilon$$
וכעת \(\left\lvert  f(x)-\sup A  \right\rvert<\varepsilon\) וקיבלנו כי \(\underset{ x \to b^{-} }{\lim }f(x)=\sup A\).


  \end{enumerate}
\end{proof}
\section{רציפות במידה שווה}

\begin{definition}[רציפות במידה שווה]
$$\forall \varepsilon>0 \;\;\exists\delta>0 \;\;\forall x_1, x_2 \;\;|x_2-x_1|<\delta \Rightarrow |f(x_2)-f(x_1)|<\varepsilon$$

\end{definition}
\begin{proposition}[תנאי לאי רציפות במידה שווה]
יהי \(f:D\to \mathbb{R}\) פונקציה אם קיימות שתי סדרות \((\hat{x_n})_{n=1}^\infty, (x_n)_{n=1}^\infty\) המקיימות:

  \begin{enumerate}
    \item לכל \(n \in \mathbb{N}\) מתקיים \(x_{n},\hat{x}_{n}\in D\)


    \item הפרש שואף ל-0: 
$$\lim_{n\to\infty} (x_n-\hat{x_n})=0$$


    \item מתקיים: 
 $$\forall n\in\mathbb N \qquad |f(x_n)-f(\hat{x_n})|\geq \varepsilon_0$$
 אזי הפונקציה לא רציפה במידה שווה. כלומר קיימות שתי סדרות שמתקרבות אחת לשנייה אבל המרחק בין ערכי ה-\(y\) שלהם לא יהיו קטנים כרצונינו.


  \end{enumerate}
\end{proposition}
\textbf{טענה} משפט קנטור
 אם פונקציה רציפה בקטע סגור אז \(f\) רציפה במידה שווה.

\begin{proof}
נניח בשלילה ש-\(f\) רציפה בקטע סגור אך אינה רציפה במידה שווה. לכן קיים \(\epsilon_0 > 0\) ושתי סדרות \((b_n), (a_n)\) כך ש-\(\lim_{n\to\infty}(b_n-a_n) = 0\) וגם:
$$\forall n\in\mathbb N\quad |f(b_n)-f(a_n)|\geq \epsilon_0$$
לפי בולצנו ווירשטרס קיימת ל-\((a_n)\) תת סדרה \((a_{n_k})\) מתכנסת. נסמן \(\lim_{k\to\infty} a_{n_k}=L\).
מאריתמטיקה של גבולות מתקיים:
$$b_{n_k} = a_{n_k}+(b_{n_k}- a_{n_k}) \Rightarrow \lim  b_{n_k} = L+0=L$$
וכן מרציפות מתקיים:
$$\lim_{k\rightarrow \infty} f(b_{n_k})=f(L)\qquad \lim_{k\rightarrow \infty} f(a_{n_k})=f(L)$$
ולכן:
$$\lim_{k\rightarrow \infty} (b_{n_k}-a_{n_k})=L-L=0$$
וגם $$\lim f(a_{n_k}) - f(b_{n_k})=0$$
בסתירה לזה ש-$$\forall n\in\mathbb N\quad |f(b_n)-f(a_n)|\geq \epsilon_0$$\textbf{הערה}
טענה זו נכונה גם עבור קטע פתוח אם קיימים הגבולות החד צדיים.

\end{proof}
\Chapter{נגזרת של פונקציה במשתנה אחד}

\section{הגדרת הנגזרת}

\begin{definition}[נגזרת בנקודה]
תהי \(f(x):A\to \mathbb{R}\) פונקציה כאשר \(A\subseteq \mathbb{R}\). אזי \(f\) גזירה בנקודה \(x_{0}\) אם מוגדרת בסביבה מלאה ומתקיים:
$$f'(x_{0})= \lim_{ x \to x_{0} } \frac{f(x)-f(x_{0})}{x-x_{0}}$$

\end{definition}
\begin{corollary}
ניתן להגדיר \(h=x_{0}-x\) ולקבל:
$$f'(x_{0})=\lim_{ h \to 0 } \frac{f(x_{0}+h)-f(x_{0})}{h}$$
מהרכבה של גבולות.

\end{corollary}
\begin{symbolize}
לעיתים מסמנים הנגזרת של \(f\) לפי \(x\) בתור \(\frac{\mathrm{d} f}{\mathrm{d} x}\)

\end{symbolize}
\begin{example}
תהי \(f:\mathbb{R}\to \mathbb{R}\) פונקציה המוגדרת \(f(x)=c\) כאשר \(c \in \mathbb{R}\). נניח כי הנגזרת תהיה \(f'(x_{0})=0\) לכל \(x_{0} \in \mathbb{R}\). יהי \(x_{0} \in \mathbb{R}\). מתקיים:
$$f'(x_{0})= \lim_{ x \to x_{0} } \frac{f(x)-f(x_{0})}{x-x_{0}} = \lim_{ x \to x_{0} } \frac{c-c}{x-x_{0}}=0$$
ולכן הנגזרת היא אפס.

\end{example}
\begin{definition}[קירוב לינארי בנקודה]
אם קיים קבוע \(M\) וסביבה של \(x_{0}\) כך שלכל \(x\) בסביבה מתקיים:
$$M$$

\end{definition}
\begin{proposition}[קירוב לינארי]
אם \(f'(x)\) גזירה בנקודה \(x_{0}\) כאשר \(y_{0}=f(x_{0})\) אז הקירוב הלינארי יהיה:
$$y-y_{0}=f'(x)(x-x_{0})$$

\end{proposition}
\begin{proposition}
פונקציה גזירה אם"ם קיים קירוב לינארי.

\end{proposition}
\begin{proof}
הכיוון של אם הפונקציה גזירה קיים קירוב לינארי הוא מיידי. נניח כעת כי קיים קירוב לינארי. כלומר קיים קבוע \(M\) כך

\end{proof}
\begin{proposition}
אם פונקציה \(f\) גזירה בנקודה אזי רציפה בנקודה.

\end{proposition}
\begin{proof}
מתקיים:
$$f(x)=(f(x)-f(x_{0}))+f(x_{0})=\frac{f(x)-f(x_{0})}{x-x_{0}}(x-x_{0})+f(x_{0})\xrightarrow{x\to x_{0}} f(x_{0})$$
ולכן רציף בנקודה.

\end{proof}
\begin{definition}[גזירות חד כיוונית]
  \begin{enumerate}
    \item גזירות מימין: 
$$f'_{+}(x_{0})=\lim_{ x \to x_{0}^{+} } \frac{f(x)-f(x_{0})}{x-x_{0}}$$


    \item גזירות משמאל: 
$$f'_{-}(x_{0})_{-}=\lim_{ x \to x_{0} } \frac{f(x)-f(x_{0})}{x-x_{0}}$$


  \end{enumerate}
\end{definition}
\begin{proposition}
  \begin{enumerate}
    \item שיוויון בין הנגזרת החד צדדיות גורר גזירות בנקודה 


    \item גזירות חד צדדית גורר רציפות חד צדדית. 


  \end{enumerate}
\end{proposition}
\begin{definition}[פונקציה גזירה בתחום]
  \begin{enumerate}
    \item אם פונקציה היא גזירה בנקודה לנקודה נקודה בתחום ניתן להגיד שהיא גזירה בתחום. 


    \item אם פונקציה היא גזירה בכל תחום הגדרתה אנחנו אומרים שהפונקציה היא גזירה. 


  \end{enumerate}
\end{definition}
\begin{definition}[פונקציה ליפשיצית]
פונקציה \(f:D\to \mathbb{R}\) עבורה קיים קבוע \(M\) כך שלכל \(x_{1},x_{2} \in D\) מתקיים:
$$\lvert f(x_{1})-f(x_{2}) \rvert \leq M|x_{1}-x_{2}|$$
כאשר \(M\) נקרא הקבוע ליפשיץ.

\end{definition}
\begin{proposition}
אם הנגזרת של פונקציה היא חסומה אז הפונקציה היא ליפשיצית

\end{proposition}
\begin{proof}
נניח כי מתקיים \(\lvert f'(x) \rvert\leq M\) לכל \(x \in D\). יהיו \(x,y \in I\). ממשפט לגרנג' מתקיים:
$$f(y)-f(x)=f'(c)(y-x)\leq M(y - x)$$
ולכן לישיצית עם קבוע ליפשיץ \(M\).

\end{proof}
\section{חוקי גזירה}

\begin{proposition}[אריתמטיקה של נגזרות]
נניח \(f,g\) מוגדרות בסביבה מלאה של \(x_{0}\) וגזירות. אזי:

  \begin{enumerate}
    \item הסכום \(f+g\) גזיר ומתקיים: 
$$(f+g)'(x_{0})= f'(x_{0})+g'(x_{0})$$


    \item המכפלה \(f\cdot g\) גזירה ב-\(x_{0}\) ומתקיים: 
$$\left( f\cdot g \right)'(x_{0})=f'(x_{0})g(x_{0})+f(x_{0})g'(x_{0})$$


    \item עבור \(\lambda \in \mathbb{R}\) מתקיים \(\lambda f\) גזירה וכן: 
$$\left( \lambda f \right)'(x_{0})=\lambda f'(x_{0})$$


    \item ההפרש \(f-g\) גזיר ומתקיים: 
$$(f-g)'(x_{0})= f'(x_{0})-g'(x_{0})$$


    \item אם \(g(x_{0})\neq 0\) אזי \(\frac{1}{g}\) גזירה ומתקיים: 
$$\left( \frac{1}{g} \right)'(x_{0})=-\frac{g'(x_{0})}{g(x_{0})^{2}}$$


    \item אם \(g(x_{0})\neq 0\) אז \(\frac{f(x)}{g(x)}\) גזיר ב-\(x_{0}\) ומתקיים: 
$$\left( \frac{f}{g} \right)'(x_{0})= \frac{f'(x_{0})g(x_{0})-g'(x_{0})f(x_{0})}{g(x_{0})^{2}}$$


  \end{enumerate}
\end{proposition}
\begin{proof}
  \begin{enumerate}
    \item מתקיים: 
\begin{gather*}\frac{(f+g)(x)-(f+g)(x_{0})}{x-x_{0}}= \frac{f(x)+g(x)-f(x_{0})+g(x_{0})}{x-x_{0}}=  \\=\frac{f(x)-f(x_{0})}{x-x_{0}}+ \frac{g(x)-g(x_{0})}{x-x_{0}}\xrightarrow{x\to x_{0}}f'(x_{0})+g'(x_{0})
\end{gather*}


    \item מתקיים: 
\begin{gather*}\frac{(fg)(x)-(fg)(x_{0})}{x-x_{0}}=\frac{f(x)g(x)-f(x_{0})g(x_{0})}{x-x_{0}}=  \\=\frac{f(x)g(x)-f(x_{0})g(x)+f(x_{0})g(x)-f(x_{0})g(x_{0})}{x-x_{0}}= \\=g(x)\left( \frac{f(x)-f(x_{0})}{x-x_{0}} \right)+f(x_{0})\left( \frac{g(x)-g(x_{0})}{x-x_{0}} \right)\xrightarrow{x\to x_{0}} g(x_{0})f'(x_{0})+ f'(x_{0})g(x_{0})
\end{gather*}


    \item אם \(g(x_{0})\) מוגדרת בסביבה מנוקבת של \(x_{0}\) אזי \(\frac{1}{g}(x_{0})\) מוגדרת בסביבה מנוקבת של \(x_{0}\), ומתקיים: 
\begin{gather*}\frac{\left( \frac{1}{g} \right)(x)-\left( \frac{1}{g} \right)(x_{0})}{x-x_{0}}= \frac{\frac{1}{g(x)}-\frac{1}{g(x_{0})}}{x-x_{0}}=\frac{\frac{g(x_{0})-g(x)}{g(x)g(x_{0})}}{x-x_{0}}=  \\=\frac{-(g(x)-g(x_{0}))}{(x-x_{0})g(x)g(x_{0})}=-\frac{1}{g(x)g(x_{0})}\left( \frac{g(x)-g(x_{0})}{x-x_{0}} \right)\xrightarrow{x\to x_{0}} -\frac{g'(x_{0})}{g(x_{0})^{2}}
\end{gather*}


  \end{enumerate}
\end{proof}
\begin{theorem}[כלל השרשרת]
אם \(f\) גזירה ב-\(x_{0}\) ו-\(g\) גזירה ב-\(y_{0}=f(x_{0})\) אז \(g\circ f\) גזירה ב-\(x_{0}\) ומתקיים:
$$\left( g\circ f \right)'(x_{0})=g'(f(x_{0}))f'(x_{0})$$

\end{theorem}
\begin{theorem}[נגזרת של פונקציה הופכית]
אם פונקציה היא חח"ע ועל אז קיימת הופכית \(f^{-1}\)
אם מתקיים \(f\) גזירה בנקודה \(x_0\), \(f'(x_0) \neq 0\) וגם \(f^{-1}\) רציפה בנקודה \(y_0=f(x_0)\) אז מתקיים ש-\(f^{-1}\) גזירה בנקודה \(y_0\) ומתקיים:
$$(f^{-1})'(y_0) = \frac{1}{f'(x_0)}=\frac{1}{f'(f^{-1} (y_0))}$$

\end{theorem}
ממשפט זה נובע שאם קיימת נקודה שבה השיפוע 0, אז הנגזרת של ההופעית לא מוגדרת בנקודה.
\textbf{הוכחה}
אם \(y\neq y_{0}\) מתקיים:
$$f^{-1}(y)\neq f^{-1}(y_{0})$$
כי \(f^{-1}\) חח"ע. כעת:
$$f^{-1}(y_{0})=\frac{x-x_{0}}{y-x_{0}}$$

\section{משפטי גזירה}

\begin{theorem}[פרמה]
אם קיימת נקודת קיצון מקומית \(x_0\) והפונקציה גזירה בה אז \(f'(x_0)=0\)

\end{theorem}
\begin{proof}
  \begin{enumerate}
    \item נניח \(x_{0}\) מקסימום מקומי. לכן נקודה פנימית. 


    \item קיים \(\delta>0\) כך ש-\(f(x)\leq f(x_{0})\) עבור כל \(x\) ב-\(\left( x_{0}-\delta,x_{0}+\delta \right)\). \(f\) גזירה ב-\(x_{0}\) ומתקיים: 
$$f'_{-}(x_{0})=f'(x_{0})=f'_{+}(x_{0})$$


    \item לכל \(x \in \left( x_{0},x_{0}+\delta \right)\) מתקיים: 
$$\frac{f(x)-f(x_{0})}{x-x_{0}}\leq 0$$
כיוון ש:
$$f(x)-f(x_{0})\leq 0\quad x-x_{0}\leq 0$$


    \item אי שיוויון חלש בין פונקציות גורר אי שיוויון חלש בין גבולות, ונקבל \(f'(x_{0})\leq 0\)


    \item בעופן דומה עבור \(x \in \left( x_{0}-\delta,x_{0} \right)\) נקבל \(f'(x_{0})\geq 0\). 


    \item סה"כ קיבלנו כי \(f'(x_{0})=0\). 


  \end{enumerate}
\end{proof}
\begin{corollary}
מהוכחה ראינו כי:

  \begin{enumerate}
    \item אם פונקציה עולה אז \(0\leq f'(x_{0})\). 


    \item אם פונקציה יורדת אז \(0\geq f'(x_{0})\). 


  \end{enumerate}
\end{corollary}
\begin{theorem}[רול]
אם פונקציה \(f:[a,b] \rightarrow \mathbb R\):
 רציפה בקטע הסגור \([a,b]\), גזירה בקטע הפתוח \((a,b)\), מקיימת \(f(a)=f(b)\)
אז קיימת נקודה \(c\in [a,b]\) המקיימת \(f'(c)=0\)

\end{theorem}
\begin{proof}
  \begin{enumerate}
    \item הפונקציה \(f\) רציפה בקטע הסגור \([a,b]\) ולכן מווירשטראס \(f\) חסומה בקטע וקיימים \(c_{1},c_{2} \in [a,b]\) נקודות מקסימום ומינימום כך שמתקיים: 
$$\forall x \in [a,b]\quad f(c_{1})\leq f(x)\leq f(c_{2})$$


    \item אם \(c_{1}=c_{2}\) אז מתקיים: 
$$\forall x \in[a,b]\quad f(c_{1})\leq f(x)\leq f(c_{1})\implies f(x)=f(c_{1})$$
והפונקציה תהיה קבועה


    \item אם \(c_{1}\neq c_{2}\) אז \((a,b)\neq(c_{1},c_{2})\) כי אחרת \(f(b)\neq f(a)\) בסתירה לתנאי 3. לכן \(a<c_{1}<b\) או \(a<c_{2}<b\) ולכן לפתחות אחת מבין \(c_{1},c_{2}\) נקודות פנימית, לכן נקודת קיצון מקומית. 


    \item מתנאי 2 הפונקציה תהיה גזירה בנקודת הקיצון הזו ולכן תקיים ממשפט פרמה \(f'(c_{1})=0\) או \(f'(c_{2})=0\). 


  \end{enumerate}
\end{proof}
\begin{theorem}[לגרנג']
אם פונקציה \(f:[a,b] \rightarrow \mathbb R\):
 רציפה בקטע הסגור \([a,b]\), גזירה בקטע הפתוח \((a,b)\)
אז קיימת נקודה \(c\in (a,b)\) המקיימת 
$$f'(c)=\frac{f(b)-f(a)}{b-a}$$

\end{theorem}
\begin{proof}
  \begin{enumerate}
    \item נגדיר \(\ell:\mathbb{R}\to \mathbb{R}\) הישר המשיק שמחבר בין \((a,f(a))\) ו-\((b,f(b))\). מתקיים: 
$$\ell(x)=\frac{f(b)-f(a)}{b-a}(x-a)+f(a)$$


    \item נגדיר \(h(x)=f(x)-\ell(x)\) כאשר \(h\) רציפה ב-\([a,b]\) וגזירה ב-\((a,b)\) מאריתמטיקה של גזירות ורציפות. 


    \item נגזור את \(h(x)\) ונקבל: 
$$h'(x)=f'(x)-\ell'(x)=f'(x)- \frac{f(b)-f(a)}{b-a}$$


    \item בנוסף מתקיים \(h(a)=h(b)=0\) ולכן מקיים את התנאים של משפט רול ולכן קיים נקודה \(c \in (a,b)\) כך שמתקיים: 
$$h'(c)=0\implies f'(c)=\frac{f(b)-f(a)}{b-a}$$


  \end{enumerate}
\end{proof}
\begin{theorem}[קושי]
אם פונקציות \(f,g\) רציפות בקטע הסגור \([a,b]\) וגזירות בקטע הפתוח \((a,b)\) אז קיים \(c\in (a,b)\) המקיים $$f'(c)\left(g(b)-g(a)\right) = g'(c)\left(f(b) - f(a)\right)$$
ובפרט אם \(g'(x) \neq 0\) לכל \(x\in (a,b)\) אז מתקיים $$\frac{f'(c)}{g'(c)} = \frac{f(b) - f(a)}{g(b) - g(a)}$$

\end{theorem}
\begin{theorem}[דרבו]
אם פונקציה היא גזירה ב-\(a,b\) ובעלת נגזרת ימנית ב-\(a\) ונגזרת שמאלית ב-\(b\) אז לכל \(f'_{+}(a) < \lambda < f'_{-}(b)\) קיים \(c\) כך ש-\(f'(c)=\lambda\)

\end{theorem}
\begin{proof}
  \begin{enumerate}
    \item נגדיר \(g(x)=f(x)-\lambda x\). נשים לב כי \(g(x)\) רציפה כסכום של פונקציות רציפות. 


    \item מהמשפט השני של ווירשטראס קיים \(c \in [a,b]\) כך ש: 
$$\forall x \in [a,b]\quad  g(c)\leq g(x)$$


    \item מהנתון נובע כי: 
$$g'_{+}(a)=f'_{+}(a)-\lambda<0\implies \lim_{ x \to a } \frac{g(x)-g(a)}{x-a}<0$$


    \item לכן קיימת סביבה מנוקבת ימנית של \(a\) כך ש-\(\frac{g(x)-g(a)}{x-a}< 0\). 


    \item באופן דומה, כיוון ש-\(b\neq c \neq a\) מתקיים: 
$$g'_{-}(b)=f'_{-}(b)-\lambda>0$$
כלומר קיימת סביבה שבה \(g(x)<g(b)\)


    \item כלומר \(c\) נקודה פנימית. ולכן מקסימום או מינימום מקומי ומקיימת: 
$$g'(c)=0\implies f'(c)-\lambda=0\implies f'(c)=\lambda$$


  \end{enumerate}
\end{proof}
\begin{proposition}
לפונקציה גזירה יש רק אי רציפויות מהסוג השני.

\end{proposition}
\begin{proof}
נזכר כי אם \(\underset{ x \to a^{+} }{\lim }f'(x)=L\) כאשר \(f\) רציפה ב-\(a\) אז \(f'_{+}(a)=L\). נניח בשלילה שקיימת נקודת אי רציפות שאינה מהסוג השני, ולכן שתי הגבולות החד צדדיות של הנגזרת קיימות. נסמן
$$\lim_{ x \to a^{+} } f'(x)=L\qquad \lim_{ x \to a^{-} } f'(x)=K$$
כיוון ש-\(f\) גזירה ב-\(a\) אז מתקיים:
$$\lim_{ x \to a^{-} } f'(x)=\lim_{ x \to a^{+} } f'(x)=f'(a)$$
ולכן \(K=L\) והפונקציה רציפה, בסתירה.

\end{proof}
\section{משפט טיילור וקירובים}

\begin{definition}[פולינום טיילור    ]
תהי \(f\) פונקציה. הפולינום טיילור מסדר \(n\) סביב \(a\) יהיה:
$$P(n,f,a)=\sum_{k=0}^n \frac{f^{(k)}(a)}{k!}(x-a)^k$$$$R(n,f,a) = f(x) - P(n, f, a)$$

\end{definition}
\begin{proposition}
פולינום טיילור מסדר \(n\) של הפונקציה \(f\) סביב הנקודה \(x_{0}\) הוא הפולינום היחיד שמקיים:
$$\lim_{x\rightarrow c} \frac{R(n,f,a)}{(x-a)^n} = 0$$

\end{proposition}
\begin{theorem}[השארית של פולינום טיילור]
קיים \(c\in [a,x]\), כלומר בין הסביבה שבסביבה מפתחים את הפולינום למטרה שאנחנו רוצים למצוא את השארית ממנו המקיים:

\end{theorem}
\begin{enumerate}
  \item שארית לגרנג': 
 $$R_n = \frac{f^{(n+1)}(c)}{(n+1)!}(x-a)^n$$


  \item שארית קושי: 
 $$R_n = \frac{f^{(n+1)(c)}}{n!}(x-c)^k(n-a)$$


\end{enumerate}
\section{משפט להופיטל}

\begin{theorem}[לופיטל]
אם קיימים \(f,g\) כך ש-\(f,g\) גזירות בסביבה ומתקיים:
    $$(1) \lim_{x\to c} f(x) =\lim_{x\to c} g(x) = 0$$
וקיימת סביבה מנוקבת שבה \(g(x)\neq 0\) אז:
    $$\lim_{x\rightarrow c} \frac{f(x)}{g(x)} = \lim_{x\rightarrow c} \frac{f'(x)}{g'(x)}$$

\end{theorem}
\begin{remark}
קיים גם גרסה על שאיפה לאינסוף ושל ביטויים מהצורה \(\frac{\infty}{\infty}\)

\end{remark}
\Chapter{אינטגרל של פונקציה במשתנה אחד}

\section{אינטגרל לפי דרבו ורימן}

\begin{definition}[חלוקה של קטע]
קבוצה סופית \(P\) של נקודות בקטע \([a,b]\) הכוללות גם את \(a\) וגם את \(b\).

\end{definition}
\begin{definition}[קוטר חלוקה]
הגודל של ההפרש הגדול ביותר בין 2 נקודות בחלוקה.

\end{definition}
\begin{definition}[סכום דרבו]
חסומה \(f(x)\) וחלוקה \(P=\{x_0, ... , x_n\}\). עבור \(1\leq i\leq n\) נגדיר:
$$M_i = \underset{x_{i-1}\leq x\leq x_i}{\sup}f(x)\qquad U(f,P)=\sum_{i=1}^{n} M_i (x_i - x_{i-1})$$$$m_i = \underset{x_{i-1}\leq x\leq x_i}{\inf}f(x)\qquad L(f,P)=\sum_{i=1}^{n} m_i (x_i - x_{i-1}) $$

\end{definition}
\begin{definition}[תנודה של חלוקה]
$$\omega_i = M_i - m_i$$

\end{definition}
\begin{definition}[קבוצת חלוקה]
$$\mathcal{U} = \left\{U(f,P) \mid P \text{ is partition } \right\}\qquad \mathcal{L} = \left\{L(f,P) \mid P \text{ is partition } \right\}$$

\end{definition}
\begin{definition}[אינטגרל עליון ותחתון]
מוגדרים בצורה הבאה:
$$\overline{\int_a^b} f(x) dx = \sup(\mathcal{U})\qquad \underline{\int_a^b} f(x)dx=\inf(\mathcal{L})$$

\end{definition}
\begin{definition}[אינטגרביליות לפי דרבו]
פונקציה חסומה \(f:D\to \mathbb{R}\) נקראת אינטגרבילית לפי דרבו אם האינטגרל התחתון שווה לאינטגרל העליון.

\end{definition}
\begin{proposition}[אי שיוויון בין האינטגרלים]
$$m(b-a)\leq \mathcal{L}\leq \underline{\int_a^b} f(x)dx \leq \overline{\int_a^b} f(x) dx \leq \mathcal{U}\leq M(b-a)$$

\end{proposition}
\begin{proposition}[למת החתכים עבור קבוצות]
אם \(U,L\) הם קבוצות המקיימות \(U\geq L\) אז הביטויים הבאים שקולים: 

  \begin{enumerate}
    \item קיים \(L\leq c\leq U\) יחיד. 


    \item מתקיים \(\sup L = \inf U\). 


    \item לכל \(\varepsilon>0\) קיים \(u\in U\quad l\in L\) כך ש-\(u-l<\varepsilon\)


    \item קיימות סדרות \((u_n)_{n=1}^{\infty}\) של איברים ב-\(U\) ו-\((l_n)_{n=1}^\infty\) של איברים ב-\(L\) המקיימות \(\underset{n\rightarrow\infty}{\lim }u_n - l_n = 0\)


  \end{enumerate}
\end{proposition}
\begin{proposition}[למת החתכים עבור סכומי דרבו]
  \begin{enumerate}
    \item קיים מספר \(c\) יחיד המקיים שלכל שתי חלוקות \(P_1,P_2\) מתקיים \(L(f,P_1)\leq c \leq U(f,P_2)\). 


    \item הפונקציה \(f\) אינטגרבילית. 


    \item לכל \(\varepsilon>0\) קיימים 2 חלוקות \(P_1, P_2\) שמקיימות \(U(f,P_1)-L(f,P_2)<\varepsilon\)


    \item קיימת סדרה של חלוקות \((P_n)_{n=1}^{\infty}\) אשר מקיימת  \(\underset{n\rightarrow \infty}{\lim} \left( U(f,P_n) - L(f,P_n)\right) = 0\)


  \end{enumerate}
\end{proposition}
\begin{proposition}[תנאי דרבו לאינטגרביליות]
פונקציה \(f\) אינטגרבילית אם"ם לכל \(\varepsilon>0\) קיימת חלוקה \(P\) כך ש- \(U(f,P)-L(f,P)<\varepsilon\)

\end{proposition}
\begin{proposition}
אם \(f\) אינטגרבילית ב-\(\left[ \alpha,\beta \right]\subseteq(a,b)\) לכל \(\alpha,\beta\) ו-\(f\) חסומה אז \(f\) אינטגרבילית ב-\([a,b]\).

\end{proposition}
\begin{proof}
הפונקציה \(f\) חסומה ולכן קיים \(M \in \mathbb{R}\) כך ש:
$$\forall x \in [a,b]\quad \lvert f(x) \rvert <M$$
יהי \(\varepsilon>0\). נבחר \(\alpha,\beta \in \mathbb{R}\) המקיימים \(b-\beta<\frac{\varepsilon}{8M}\), ו-\(\alpha-a<\frac{\varepsilon}{8M}\). לפי הנתון \(f\) אינטגרבילית ב-\(\left[ \alpha,\beta \right]\) ולכן קיימת חלוקה \(Q\) כך שמתקיים:
$$U(f,Q)-L(f,Q)<\frac{\varepsilon}{2}$$
נגדיר חלוקה \(P=Q\cup \{ a,b \}\). מתקיים:
\begin{gather*}U(f,P)-L(f,P)=\left( \sup_{\left[ a,\alpha \right]}(f)-\inf _{\left[ a,\alpha \right]}(f) \right)\left( \alpha-a \right)+U(f,Q)-L(f,Q)\left( \sup_{\left[ \beta,b \right]}(f)-\inf _{\left[ \beta,b \right]}(f) \right)\left( b-\beta \right) \leq \\\leq 2M\cdot \frac{\varepsilon}{8M}+\frac{\varepsilon}{2} +2M\cdot \frac{\varepsilon}{8M}=\frac{\varepsilon}{4}+\frac{\varepsilon}{2}+\frac{\varepsilon}{4}=\varepsilon
\end{gather*}
ומקיים את תנאי דרבו לאינטגרביליות.

\end{proof}
\begin{definition}[סכום רימן]
נגדיר חלוקה \(P=\{x_0,...,x_n\}\) של \([a, b]\) וקבוצה של נקודות \(P^*=\{c_1,...,c_n \}\) המקיימות \(x_{i-1}\leq c_i\leq x_i\). סכום רימן יהיה מוגדר:
$$S(f, P, P^*)=\sum_{i=1}^n f(c_i)(x_i-x_{i-1})$$

\end{definition}
\begin{definition}[אינטגרביליות לפי רימן]
קיים \(I\in R\) כך שלכל \(\varepsilon>0\) קיים \(\delta>0\) כך שעבור חלוקה המקיימת \(\Delta P < \delta\) ולכל בחירה של נקודות \(P^*\) מתקיים \(|S(f,P,P^*)-I|<\varepsilon\)

\end{definition}
\begin{proposition}
פונקציה אינטגרבילית לפי רימן היא חסומה.

\end{proposition}
\begin{proof}
  \begin{enumerate}
    \item נניח בשלילה שלא חסום. נגדיר חלוקה עם \(\Delta P<\delta\). עבור \(\varepsilon=1\) הסכום רימן יקיים: 
$$I-1<S(f,P,P^{*})<I+1$$


    \item חייב להיות מההנחה בשלילה קטע בחלוקה עבורו לא חסום. בקטע זה קיים נקודה שגודלה מכל שאר האיברים בסכום, בסתירה לתנאי רימן. 


  \end{enumerate}
\end{proof}
\begin{remark}
נשים לב כי עבור דרבו זו הייתה דרישה בהגדרה, כאשר כאן זה נובע מההגדרה.

\end{remark}
\begin{theorem}[שקילות בין אינטגרל דרבו ורימן]
אינטגרביליות לפי רימן שקולה לאינטגרביליות לפי דרבו.

\end{theorem}
נראה אינטגרביליות לפי רימן גוררת אינטגרביליות לפי דרבו.

\begin{proof}
יהי \(\varepsilon> 0\). מתנאי רימן קיים \(0< \delta\) כך שעבור חלוקה \(P=\left\{  x_{0},\dots,x_{n}  \right\}\) כך ש-\(\Delta P<\delta\) לכל חלוקת נקודות \(P^{*}\) מתקיים:
$$\lvert S(f,P,P^{*}) - I\rvert <\frac{\varepsilon}{4}$$
נסתכל על הקטע חלוקה \([x_{i-1},x_{i}]\). נגדיר:
$$M_{i}=\sup _{x \in [x_{i-1},x_{i}]} f(x)\qquad m_{i}=\inf _{x \in[x_{i-1},x_{i}]}f(x)$$
אנו יודעים כי מזהות של סופרמום כי מתקיים:
$$\sup \left\{  x-y \mid x,y \in A\right\}=\sup A-\inf A$$
לכן:
$$\omega_{i}(f,P)=M_{i}-m_{i}=\sup _{x \in[x_{i-1},x_{i}]} (f(x)-f(y))$$
בפרט \(\frac{1}{2}(M_{i}-m_{i})\) לא חסם עליון וקיימים \(c_{i},\hat{c}_{i}\) כך ש:
$$\frac{1}{2}(M_{i}-m_{i})\leq \lvert f(c_{i})-f\left( \hat{c}_{i} \right) \rvert $$
נסמן \(P^{*}=\left\{  c_{1},\dots, c_{n}  \right\}\) ו-\(\widehat{P^{*}}=\left\{  \hat{c}_{1},\dots,\hat{c}_{n}  \right\}\) הנקודות עבור כל חלוקה. כעת מתקיים:
\begin{gather*}U(f,P)-L(f,P)=\sum (M_{i}-m_{i})(x_{i-1}-x_{i})\leq \\\leq \sum_{i=1}^{n} 2\left\lvert  f(c_{i})-f\left( \hat{c}_{i} \right)  \right\rvert (x_{i-1}-x_{i})= \\=\left\lvert  2S(f,P,P^{*})-2S\left( f,P,\widehat{P^{*}}  \right)  \right\rvert = \frac{2\varepsilon}{4}+\frac{2\varepsilon}{4}=\varepsilon
\end{gather*}

\end{proof}
\section{תכונות האינטגרל המסויים}

\begin{proposition}
אם \(f,g\) פונקציות אינטגרביליות, מתקיים:

\end{proposition}
\begin{enumerate}
  \item לינאריות:\\
$$\int f\pm g=\int f \pm \int g\qquad \int c f =c\int f$$


  \item המכפלה \(f\cdot g\) גם תהיה פונקציה אינטגרבילית. 


  \item אם קיים \(C>0\) כך ש-\(|g|\geq C\) אז \(\frac{f}{g}\) אינטגרבילית. 


  \item אם \(f(x)>0\) ב-\(x\in[a,b]\) אז \(\int f >0\)


  \item מונוטוניות: \(f\geq g\Rightarrow \int f \geq \int g\)


  \item ערך מוחלט: \(\left| \int f(x) \right|\leq \int\left| f(x)\right|\)


  \item ירושה: אם פונקציה אינטגרבילת בתחום אז תהיה אינטגרבילת בכל תת תחום 


  \item אינווריטנטיות להזזה אופקית:\\
$$\int_{a-c}^{b-c} f(x+c) = \int_a^b f$$


  \item הומותטיה:\\
$$\int_{a/m}^{b/m} f(mx)=\frac{1}{m}\int_a^b f(x) dx$$


  \item אדיטיביות של תחום: 
$$\int_a^b f = \int_a^c f + \int_c^b f$$


\end{enumerate}
\begin{proposition}[דברים הגוררים אינטגרביליות]
אם \(f:[a,b]\to \mathbb{R}\) אז הביטיים הבאים גוררים אינטגרביליות:

\end{proposition}
\begin{enumerate}
  \item מונוטוניות. 


  \item רציפות. 


  \item ליפשיציות 


  \item מכפלה של פונקציות אינטגרביליות 


\end{enumerate}
\begin{proposition}
אם \(f:[a,b]\rightarrow \mathbb R\) מונוטונית אז היא אינטגרבילית

\end{proposition}
\begin{proof}
  \begin{enumerate}
    \item נניח \(f\) מונוטונית עולה ב-\([a,b]\). 


    \item הפונקציה \(f\) חסומה כי \(\forall x \quad f(a)\leq f(x)\leq f(b)\). 


    \item יהי \(\varepsilon>0\). נבחר חלוקה \(P=\{x_0,...,x_n\}\) כך ש-\(\Delta P < \frac{\varepsilon}{f(b)-f(a)}\)


    \item כעת מתקיים \(M_i=f(x_i)\quad m_i=f(x_{i-1})\) ממונוטוניות 


    \item נקבל:\\
\begin{gather*}U(f,P)-L(f,P)=\sum_{i=1}^n (M_i-m_i)(x_i-x_{i-1})  \\\leq\Delta P\sum_{i=1}^n (f(x_i)-f(x_{i-1}))=\Delta P(f(b)-f(a))\leq \varepsilon
\end{gather*}
ולכן אינטגרבילי מתנאי דרבו לאינטגרביליות.


  \end{enumerate}
\end{proof}
\begin{proposition}[רציפות גורר אינטגרבילות]
אם פונקציה רציפה בקטע סגור אז היא אינטגרביליות הקטע.

\end{proposition}
\begin{proof}
  \begin{enumerate}
    \item ממשפט קנטור כיוון שרציפה בקטע סוגר נקבל כי רציפה במ"ש בקטע. לכן: 
$$\forall \varepsilon>0 \quad \exists \delta> 0\quad \forall x,y \in[a,b]\quad \lvert x-y \rvert <\delta\implies \lvert f(x)-f(y) \rvert <\varepsilon$$


    \item יהי \(\varepsilon>0\). נציב בביטוי של רציפות במידה שווה \(\varepsilon_{1}=\frac{\varepsilon}{b-a}\). נקבל \(\delta> 0\) כך ש: 
$$\forall x,y \in [a,b]\quad \lvert x-y \rvert <\delta\implies f(x)-f(y)<\frac{\varepsilon}{b-a}$$


    \item נגדיר חלוקה \(P=\left\{  \alpha=x_{0},\dots,x_{n}=\beta  \right\}\) כך ש-\(\Delta P<\delta\). 


    \item כיוון שרציף ב-\([a,b]\) גם רציף ב-\([x_{i-1},x_{i}]\) לכן מווירשטראס השני נקבל כי קיים \(c_{i}\) עבורו \(f(c_{i})\) מקסימלי ב-\([x_{i-1},x_{i}]\). 


    \item מתקיים: 
$$\lvert d_{i}-c_{i} \rvert \leq x_{i}-x_{i-1}<\delta\implies \lvert f(d_{i})-f(c_{i}) \rvert <\frac{\varepsilon}{b-a}\implies f(d_{i})-f(c_{i})<\frac{\varepsilon}{b-a}$$


    \item נסתכל על הסכומי דרבו: 
\begin{gather*}L(f,P)=\sum_{i=1}^{n} m_{i}(x_{i}-x_{i-1})=\sum_{i=1}^{n} f(d_{i})(x_{i}-x_{i-1})  \\U(f,P)=\sum_{i=1}^{n} M_{i}(x_{i}-x_{i-1})=\sum_{i=1}^{n} f(c_{i})(x_{i}-x_{i-1}) \\U(f,P)-L(f,P)=\sum_{i=1}^{n} (f(c_{i})-f(d_{i}))(x_{i}-x_{i-1})\leq \frac{\varepsilon}{b-a}(b-a)=\varepsilon
\end{gather*}
ולכן אינטגבילי מתנאי דרבו לאינטגרביליות.


  \end{enumerate}
\end{proof}
\section{המשפט היסודי של האינפי}

\begin{theorem}[היסודי של האינפי]
עבור פונקציה \(f:A\to \mathbb{R}\) נקבל כי:

  \begin{enumerate}
    \item אם פונקציה \(f\) היא \underline{אינטגרבילית} אז \(\tilde F(x)=\int_a^x f\) רציפה. 


    \item אם פונקציה \(f\) היא \underline{רציפה} אז \(\tilde F(x)=\int_a^x f(t) dt\) גזירה ומתקיים \(f = \tilde F'\). 


  \end{enumerate}
\end{theorem}
\begin{proof}
  \begin{enumerate}
    \item נראה כי \(\lim_{x\to x_0} F(x) = F(x_0)\). מתקיים: 
$$            |F(x)-F(x_0)|=\left|\int_a^x f-\int_a^{x_0} f \right|=\left|\int_{x_0}^x f\right|\leq  \int_{x_0}^x |f| \leq \int_{x_0}^x M = M(x-x_0)
$$
יהי \(\varepsilon >0\). נגדיר \(\delta = \frac{\varepsilon}{M}\). כעת לכל \(|x-x_0|<\delta\) מתקיים:
$$|F(x)-F(x_0)|\leq M(x-x_0)<\varepsilon$$
  \end{enumerate}
\end{proof}
לפני שנוכיח את החלק השני נוכיח משפט אחר:

\begin{theorem}[ערך הממוצע האינטגרלי]
אם \(f\) פונקציה רציפה ב- \([a,b]\), אז קיימת נקודה \(c\in (a,b)\) המקיימת: 
$$f(c)=\frac{1}{b-a}\int_a^b f$$

\end{theorem}
\begin{proof}
מהמשפט היסודי הפונקציה \(\tilde F=\int_a^x f(t) dt\) גזירה.
לכן ניתן להפעיל את לגרנג' ונקבל:
$$\frac{1}{b-a}\int_a^b f(t) dt = \frac{\tilde{F}(b)-\tilde{F}(a)}{b-a}=\tilde{F'}(c)=f(c)$$

\end{proof}
כעת ניתן לחזור להוכיח את החלק השני של המשפט היסודי:

\begin{proof}
  \begin{enumerate}
    \item נראה שגזיר בכל \(x_0\) לפי הגדרה:\\
$$\frac{F(x)-F(x_0)}{x-x_0}=\frac{\int_a^x f -\int_a^{x_0} f}{x-x_0}=\frac{\int_{x_0}^x f}{x-x_0}$$
כעת נשתמש בערך הממוצע האינטגרלי כיוון ש-\(f\) רציפה וניקח את הגבול כאשר \(x\rightarrow x_0\):
$$\lim_{x\rightarrow x_0} \frac{\int_{x_0}^x f}{x-x_0} = \lim_{x\rightarrow x_0} \frac{f(C(x))(x-x_0))}{x-x_0}=\lim_{x\rightarrow x_0} f(C(x))$$
וכיוון ש-\(f\) רציפה ו-\(x_0 \leq C(x)\leq x\) אזי:
$$\underset{x\rightarrow x_0}{\lim} f(C(x))=f(x_0)$$
  \end{enumerate}
\end{proof}
\begin{proposition}[הנוסחה היסודית]
אם \(f\) אינטגרבילית ובעלת פונקציה קדומה \(F\) אז מתקיים: 
$$\int_a^b F' = F(b)-F(a)$$

\end{proposition}
\begin{proof}
תהי \(P=\left\{  x_{0},\dots,x_{n}  \right\}\) חלוקה כלשהי של \([a,b]\). כיוון ש-\(F\) פונקציה קדומה, \(F\) גזירה, ומקיימת את תנאי משפט לגרנג' בגל קטע \([x_{i-1},x_{i}]\). כלומר עבור \(i=1,\dots,n\) קיים \(\hat{c}_{i}\) המקיימת:
$$F(x_{i})-F(x_{i-1})=F'\left( \hat{c}_{i} \right)(x_{i}-x_{i-1})=f\left( \hat{c}_{i} \right)(x_{i}-x_{i-1})$$
נסמן \(P^{*}=\left\{  \hat{c}_{1},\dots,\hat{c}_{n}  \right\}\) וכעת מתקיים:
$$S(f,P,P^{*})=\sum_{i=1}^{n} f\left( \hat{c}_{i} \right)(x_{i}-x_{i-1})=\sum_{i=1}^{n} (F(x_{i})-F(x_{i-1}))=F(b)-F(a)$$
וכיוון שמתקיים:
$$L(f,P)\leq \int_{a}^{b} f(t) \, dt\leq U(f,P) $$
המספר(נזכור כי האינטגרל המסויים זה מספר) היחיד שמקיים זאת מלמת החתכים זה יהיה ערך האינטגרל. ולכן:
$$\int_{a}^{b}f(t)\;\mathrm{d}t=F(b)-F(a)$$

\end{proof}
\section{שיטות אינטגרציה}

\begin{proposition}[אינטגרלים אלמנטריים]
מהמשפט היסודי של האינפי ניתן להגיע לאינטגרלים הבאים:
\begin{gather*}\int x^{n}\,d x={\frac{x^{n+1}}{n+1}}+C,\quad{\mathrm{if~}}n\neq-1 \\\int{x}^{{-{{1}}}}{\left.{d}{x}\right.}={\ln{{\left|{x}\right|}}}+{C}\\\int{e}^{{{x}}}{\left.{d}{x}\right.}={e}^{{{x}}}+{C}\\\int\sin{{x}}{\left.{d}{x}\right.}=-\cos{{x}}+{C}\\\int\cos{{x}}{\left.{d}{x}\right.}={\sin{{x}}}+{C}\\\int{{\sec}^{{{2}}}{x}}{\left.{d}{x}\right.}={\tan{{x}}}+{C}\\\int{\sec{{x}}}{\tan{{x}}}{\left.{d}{x}\right.}={\sec{{x}}}+{C}\\\int{\frac{{{1}}}{{{1}+{x}^{{{2}}}}}}{\left.{d}{x}\right.}={\arctan{{x}}}+{C}\\\int{\frac{{{1}}}{{\sqrt{{{1}-{x}^{{{2}}}}}}}}=\arcsin x+C
\end{gather*}

\end{proposition}
\begin{proposition}[אינטגרציה ע"י הצבה]
עבור אינטגרל מסויים מתקיים:
$$\int_{a}^{b}f(g(x))\cdot g^{\prime}(x)\,d x=\int_{g(a)}^{g(b)}f(u)\,d u$$
כאשר עבור אינטגרל לא מסויים נקבל את אותו הדבר ללא הגבולות.

\end{proposition}
\begin{proof}
מכלל השרשרת וההגדרה של הפונקציה הקדומה מתקיים:
$$(F\circ g)^{\prime}(x)=F^{\prime}(g(x))\cdot g^{\prime}(x)=f(g(x))\cdot g^{\prime}(x)$$
כאשר שימוש במשפט היסודי של האינפי נותן לנו:
\begin{gather*}{{\int_{a}^{b}f(g(x))\cdot g^{\prime}(x)\,d x=\int_{a}^{b}(F\circ g)^{\prime}(x)\,d x}} {{=(F\circ g)(b)-(F\circ g)(a)}}\\ {{=F(g(b))-F(g(a))}} {{=\int_{g(a)}^{g(b)}f(u)\,d u}}\end{gather*}

\end{proof}
\begin{proposition}[אינטגרציה בחלקים]
מתקיים:
\begin{gather*}\int_{a}^{b}u(x)v^{\prime}(x)\,dx=\left[u(x)v(x)\right]_{a}^{b}-\int_{a}^{b}u^{\prime}(x)v(x)\,dx=\\=u(b)v(b)-u(a)v(a)-\int_{a}^{b}u^{\prime}(x)v(x)\,dx 
\end{gather*}
כאשר אם נסמן \(u(x)=u\) ו-\(du=u'(x)dx\) ניתן לכתוב בצורה הבאה:
$$\int u\,d v\;=\;u v-\int v\,d u$$

\end{proposition}
\begin{proof}
אנו יודעים כי הנגזרת מכפלה של שתי פונקציות גזירות תהיה:
$$\left(u(x)v(x)\right)^{\prime}=u^{\prime}(x)v(x)+u(x)v^{\prime}(x)$$
כאשר אם נבצע אינטגרציה על שתי האגפים נקבל:
$$\int\left(u(x)v(x)\right)^{\prime}d x=\int u^{\prime}(x)v(x)\,d x+\int u(x)v^{\prime}(x)\,d x$$
כאשר נשים לב כי האינטגרל של הנגזרת יתן את הפונקציה:
$$u(x)v(x)=\int u^{\prime}(x)v(x)\,d x+\int u(x)v^{\prime}(x)\,d x$$
נעביר אגפים ונקבל:
$$\int u(x)v^{\prime}(x)\,d x=u(x)v(x)-\int u^{\prime}(x)v(x)\,d x$$

\end{proof}
\section{אינטגרלים לא אמיתיים}

\begin{definition}[אינטגרל לא אמיתי]
אינטגרל אשר שהתחום שלו אינו חסום ולכן כביכול לא אינטגרבילי בתחום שלו, אך במקרים מסויימים נגדיר ערך לאינטגרל. ניתן לחלק ל-2 סוגים:

\end{definition}
\begin{enumerate}
  \item אינטגרל שהתחום שלו הוא אין סופי - כלומר הגבולות של האינטגרל כוללים \(\pm \infty\). 


  \item אינטגרל עם תחום סופי שהפונקציה אינה חסומה בתחום שלו. 
בשתי המקרים ניתן להציב \(t\) לפתור אינטגרל ממשי ולקחת את הגבול.


\end{enumerate}
\begin{example}
עבור אינטגרל של הפונקציה החסומה \(f:\left[ a,\infty \right)\to \mathbb{R}\) האינטגרל הלא אמיתי בתחום יהיה מוגדר:
$$\int_a^\infty f = \lim_{t\rightarrow \infty} \int_a^t f$$

\end{example}
\begin{remark}
לא תמיד האינטגרל הלא אמיתי קיים. הוא קיים רק אם הגבול מתכנס.

\end{remark}
\begin{proposition}[מבחן ההשוואה]
  \begin{enumerate}
    \item אם \(g<f<h\) פונקציות אינטגרביליות כאשר \(\int g, \int h\) מתכנסים אז גם \(\int f\) מתכנס. 


    \item אם \(0<f<h\) כאשר \(\int f\) מתבדר אז \(\int h\) מתבדר. 


  \end{enumerate}
\end{proposition}
\begin{proposition}[מבחן ההשוואה הגבולי]
אם \(\frac{f}{g} = L > 0\) אז \(\int f, \int g\) מתכנסים ומתבדרים יחדיו

\end{proposition}
\begin{definition}[התכנסות בהחלט]
אם \(\int |a_n|\) מתכנס אז \(\int a_n\) מתכנס

\end{definition}
\begin{proposition}[מבחן דירכלה]
אם \(f,g\) פונקציות כך ש-\(f\) יורדת ומתקיים \(\lim_{x\rightarrow \infty} f = 0\) ו-\(f', g\) אינטגרביליות אז \(\int (f\cdot g)\) מתכנס.

\end{proposition}
\begin{proposition}[אינטגרליים מתכנסים ידועים]
\end{proposition}
\Chapter{מרחבים מטרים}

\section{הגדרה של מרחב מטרי}

ננסה להכליל את הנושא של גבול למרחב כללי(כלומר לא רק ב-\(\mathbb{R}\)). הגדרת הגבול הידועה אומרת כי אם:
$$\forall\varepsilon>0\quad\exists\delta>0\quad\mathrm{such\,that}\quad\forall x\,:\,0<\left|x-x_{0}\right|<\delta\qquad\left|f(x)-y_{0}\right|<\varepsilon.$$
אז ניתן להגיד שהגבול כש-\(x\to x_{0}\)  של \(f(x)\) הוא \(y_{0}\). ניתן להגדיר פונקציית מרחק \(d(x,y)=|x-y|\) וניתן כעת לכתוב את הגדרת הגבול באופן דומה:
$$\forall\varepsilon>0\quad\exists\delta>0\quad\mathrm{such\,that}\quad\forall x\,:\,0<d(x,x_{0})<\delta\qquad d(f(x),y_{0})<\varepsilon$$
כדי שההגדרה תהיה הגיונית, נגדרש כי הטווח של \(d\) תהיה \(\mathbb{R}\), איך אין דרישה שהתחום יהיה \(\mathbb{R}\).

\begin{definition}[מטריקה]
עבור קבוצה לא ריקה \(X\) פונקציה \(d:X\times X\to \mathbb{R}\) נקראת מטריקה אם מקיימת:

  \begin{enumerate}
    \item סימטרייה - \(d(x,y)=d(y,x)\)


    \item חיוביות - \(d(x,y)\geq 0\) כאשר יש שיוויון אם"ם \(x=y\). 


    \item אי שיוויון המשולש. לכל \(x,y,z\in X\) מתקיים: 
$$d(x,z)\leq d(x,y)+d(y,z)$$


  \end{enumerate}
\end{definition}
\begin{remark}
ניתן בעקרון לדרוש בתכונה \(2\) רק שיש שיוויון אם"ם \(x=y\) כי במקרה זה מאי שיוויון המשולש עבור \((x,y,x)\) נקבל:
$$d(x,x)\leq d(x,y)+d(x,y)\implies 0\leq 2d(x,y)\implies 0\leq d(x,y)$$

\end{remark}
\begin{proposition}[הכללה של אי שיוויון המשולש למספר ערכים]
$$d(x_{1},x_{n})\leq\sum_{j=1}^{n-1}d(x_{j},x_{j+1})$$

\end{proposition}
\begin{proposition}[אי שיוויון המשולש ההפוך]
$$d(x,y)\geq|d(x,z)-d(y,z)|$$

\end{proposition}
\begin{definition}[מרחב מטרי]
השילוב של הקבוצה \(X\) ביחד עם מטריקה \(d:X\times X\to \mathbb{R}\) יוצר מרחב מטרי \((X,d)\)

\end{definition}
\begin{proposition}[תת קבוצה יוצר מרחב מטרי]
יהי \((X,d)\) מרחב מטרי, ו-\(Y\subseteq X\) ביחד עם הצמצום של \(d\) ל-\(Y\times Y\) יוצר מרחב מטרי. 

\end{proposition}
\begin{proof}
כיוון ש-\(y_{1},y_{2}\in Y\) נקודות ב-\(X\) אז מקיימים את כל הדרישות של מטריקה ביחיד עם \(d_{Y\times Y}\).

\end{proof}
\begin{definition}[תת מרחב מטרי]
עבור \((X,d)\) מרחב מטרי, ו-\(Y\subseteq X\) תת קבוצה, אז \(\left( Y,d_{Y\times Y}\right)\) הוא תת מרחב מטרי של \((X,d)\). \(d_{Y\times Y}\) נקרא המטריקה המושרת על \(Y\) מ-\(X\).

\end{definition}
\begin{definition}[מטריקה על מכפלה ישירה]
יהיו \((X,d_{X}),\left( Y,d_{Y} \right)\) מרחבים מטרים. ניתן להגדיר מרחב מכפלה \(\left( X\times Y,\rho \right)\) במספר דרכים: 
$$\rho((x_{1},y_{1}),(x_{2},y_{2}))=\begin{cases}d_{X}(x_{1},x_{2})+d_{Y}(y_{1},y_{2}) \\\sqrt{d_{X}(x_{1},x_{2})^{2}+d_{Y}(y_{1},y_{2})^{2}} \\\max\{(d_{X}(x_{1},x_{2}),d_{Y}(y_{1},y_{2})\}
\end{cases}$$

\end{definition}
\begin{definition}[קבוצה חסומה]
קבוצה \(A\subseteq X\) תקרא חסומה ב-\(X\) אם קיים כדור פתוח ב-\(X\) שמכיל את \(A\). כלומר אם קיים \(x \in X\) כך שעבורו קיים \(r>0\) שמקיים \(A\subseteq \mathcal{B}_{r}(x)\).

\end{definition}
\begin{lemma}
אם \(A\subseteq X\) חסומה אז לכל \(a\in X\) קיים \(r>0\) עבורו \(A\subseteq B_{r}(a)\). 

\end{lemma}
\begin{definition}[פסודו מטריקה]
כמו מטריקה אך לא בהכרח מתקיים הדרישה
$$x\neq y\implies d(x,y)\neq 0$$
כלומר יתכן פה מצב שנקבל כי המרחק בין שתי נקודות שונות היא 0. 
הקטע בפסודו מטריקה זה שניתן להגדיר בעזרתה מטריקה. כל מה שנדרש הוא להסתכל על מחלקות השקילות של האיברים שהמרחק ביניהם 0 - כלומר \(x\sim y\) אם"ם \(d(x,y)=0\).
כעת נסתכל על מטריקה חדשה \(\bar{d}\) של המרחק בין נציגים.

\end{definition}
\section{מרחבים נורמים}

\begin{definition}[נורמה]
מרחב וקטורי \(X\) אשר מעל \(\mathbb{R}\) או \(\mathbb{C}\) נקרא מרחב נורמי אם קיים נורמה \(||\cdot||:X\to \mathbb{R}\) המקיימת:

\end{definition}
\begin{enumerate}
  \item חיוביות - \(||x||\geq 0\) כאשר יש שיוויון אם"ם \(x=0\)


  \item הומוגניות - לכל \(x \in X\) ו-\(a\in \mathbb{R}\) מתקיים \(||ax||=a||x||\)


  \item אי שיוויון המשולש - \(||x+y||\leq||x||+||y||\)


\end{enumerate}
\begin{definition}[מטריקה מושרת מנורמה]
בהנתן מרחב נורמי ניתן להגדיר מטריקה על המרחב ע"י:
$$d(x,y)=\left\|x-y\right\|$$

\end{definition}
ולכן למעשה כל נורמה יוצרת מטריקה, כאשר תהיה בעלת תכונות נוספות שלא בהכרח יש למטריקה כללית. למשל:
$$\begin{array}{c}{{d(x+z,y+z)=d(x,y)}}\qquad  {{d\left( \lambda x,\lambda y \right)=|\lambda|\,d(x,y)}}\end{array}$$
כאשר למעשה אם מתקיים שתי התכונות האלו, נקבל כי המטריקה שלנו היא מטריקה שמושרת מאיזושהי נורמה.

\section{גבולות}

\begin{definition}[סדרה מתכנסת]
סדרה \(\{ x_n\}_{n=0}^{\infty}\) נקראת מתכנסת במרחב מטרי \((X,d)\) אם קיים \(x\) כך ש:
$$\forall\varepsilon>0\quad\exists N\in\mathbb{N}\quad s u c h\;t h a t\quad\forall n>N\qquad d(x_{n},x)<\varepsilon.$$

\end{definition}
\begin{proposition}
אם סדרה מכנסת אז הגבול שלה הוא יחיד.

\end{proposition}
\begin{proof}
נניח בשלילה כי \(x_{n}\to x\) וגם \(x_{n}\to y\) כאשר \(x\neq y\). כעת עבור \(\varepsilon>0\) קיים \(n\) מספיק גדול כך ש:
$$d(x_{n},x)<\varepsilon \qquad d(x_{n},y)<\varepsilon$$
ולכן מאי שיוויון המשולש נקבל:
$$d(x,y)\leq d(x,x_{n})+d(x_{n},y)<2\varepsilon$$
וכיוון שניתן לבחור \(\varepsilon>0\) קטן כרצונינו מתקיים \(d(x,y)=0\) ולכן \(x=y\).

\end{proof}
\begin{proposition}
אם סדרה מתכנסת כל התתי סדרות שלה גם כן מתכנסות ולאותו הגבול

\end{proposition}
\begin{proposition}
סדרה \(\{ x_n\}_{n=0}^{\infty}\) במרחב מטרי \((X,d)\) מתכנסת ל-\(x\) אם"ם לכל \(r>0\) הסדרה תהיה בסופו של דבר חסומה ב-\(B_{r}(x)\). כלומר קיים \(N\) כך שלכל \(n>N\) מתקיים \(x_{n}\in B_{r}(x)\).

\end{proposition}
\begin{proposition}
אם \(x_{n}\to x\), ו-\(y_{n}\to y\), אז \(d(x_{n},y_{n})\to d(x,y)\)

\end{proposition}
\begin{corollary}
במרחב מטרי, \(x_{n}\to x\) גורר \(||x_{n}||\to ||x||\)

\end{corollary}
\begin{remark}
הכיוון ההפוך אינו נכון.

\end{remark}
\begin{proposition}
אם \(x_{n}\to x\) סדרה מתכנסת, אז כל תת סדרה גם כן תהיה סדרה מתכנסת, ותתכנס ל-\(x\).

\end{proposition}
\begin{proposition}
התכנסות של המטריקה האוקלידית, שקול להתכנסות איבר איבר:
$$\operatorname*{lim}_{n\to\infty}x_{n}=x \iff\operatorname*{lim}_{n\to\infty}x_{n}^{j}=x^{j}\quad f o r\,j=1,\ldots,k.$$

\end{proposition}
\begin{proposition}
נניח \(\{ x_n\}_{n=0}^{\infty}\) סדרה במרחב מטרי \((X,d)\) כך שקיים \(\varepsilon>0\) כך ש:
$$\forall i\in \mathbb{N}\quad d\left( x_{i},x_{i+1} \right)\geq\varepsilon$$
אז הסדרה לא מתכנסת.

\end{proposition}
\begin{remark}
הכיוון השני לא נכון.

\end{remark}
\section{כדורים}

\begin{definition}[ספרה]
אוסף כל האיברים ב-\(X\) שנמצאים במרחק \(r>0\) מ-\(a\in X\):
$${\mathcal{S}}_{r}(a)=\left\{x\in\mathbb{X}\,:\,d(x,a)=r\right\}.$$

\end{definition}
\begin{definition}[כדור פתוח]
אוסף כל האיברים ב-\(X\) שנמצאים במרחק קטן מ-\(r>0\) מ-\(a\in X\):
$${\mathcal{B}}_{r}(a)=\left\{x\in\mathbb{X}\,:\,d(x,a)<r\right\}.$$

\end{definition}
\begin{definition}[כדור סגור]
אוסף כל האיברים ב-\(X\) שנמצאים במרחק קטן או שווה ל-\(r>0\) מ-\(a\in X\):
$${\hat{\mathcal{B}}}_{r}(a)=\left\{x\in\mathbb{X}\,:\,d(x,a)\leq r\right\}.$$

\end{definition}
כאשר אנחנו במטריקת \(\mathbb{R}^n\) זה מזכיר איך שאינטויטיבית נראים כדורים. במטריקה הדיסקרטית למשל, נקבל:
$$\mathcal{B}_{r}(a)=\left\{\begin{array}{l l}{{\{a\}}}&{{r\leq1}}\\ {\:{X}}&{{r>1.}}\end{array}\right.$$

\subsection{תכונות של כדורים}

\begin{lemma}
לכל \(x,y\in X\), ולכל \(r>0\), נקבל:
$$x\in{\mathcal{B}}_{r}(y)\iff y\in{\mathcal{B}}_{r}(x).$$

\end{lemma}
\begin{lemma}
יהי \(x \in X\), ו-\(y\in \mathcal{B}_{r}(x)\) לכל \(r>0\), אז \(x=y\)

\end{lemma}
\begin{lemma}
יהי \(x \in X\) ו-\(y\in \mathcal{B}_{r}(x)\) כאשר נתון \(s>0\), מתקיים:
$$d(x,y)+s\leq r\implies \mathcal{B}_{s}(y)\subseteq  \mathcal{B}_{r}(x)$$

\end{lemma}
\begin{proposition}
בהנתן \((X,d)\) מרחב מטרי ו-\(Y\subseteq X\) תת מרחב מטרי, כאשר \(B^X_{r},B_{r}^Y\) הם הכדורים במרחבים המטרים המתאימים, עם מטריקות \(d_{X},d_{Y}\), אז מתקיים:
$$\mathcal{B}_{r}^{Y}(y)=Y\cap\mathcal{B}_{r}^{\mathbb{X}}(y).$$

\end{proposition}
\begin{proof}
נראה מההגדרה ששתי הקבוצות שוות:
$$\begin{array}{l l}{{{\mathcal{B}}_{r}^{Y}(y)=\{z\in Y\,:\,d_{Y}(z,y)<r\}}}\\ {{\begin{array}{l}{{=\{z\in Y\,:\,d_{X}(z,y)<r\}}}\\ {{=\{z\in X\,:\,z\in Y,\,\ d_{X}(z,y)<r\}}}\\ {{\,=\{z\in X\,:\,z\in Y\}\cap\{z\in X\,:\,d_{X}(z,y)<r\}}}\\ {{\,=Y\cap{\mathcal{B}}_{r}^{\mathbb{X}}(y).}}\end{array}}}\end{array}$$

\end{proof}
\begin{definition}[קבוצה חסומה]
תהנתן מרחב מטרי \((X,d)\), קבוצה \(A\subseteq X\) נקראת חסומה אם ניתן לחסום אותה בכדור פתוח. כלומר:
$$\exists a\in\mathbb{X}\qquad A\subseteq{\mathcal{B}}_{r}(a)$$

\end{definition}
\begin{lemma}
אם \(A\subseteq X\) חסום, אז:
$$\forall b\in\mathbb{X}\quad\exists\rho>0\quad s u c h\;t h a t\quad A\subset{\mathcal{B}}_{\rho}(b).$$

\end{lemma}
\begin{lemma}
כל כדור פתוח סיביב \(a\) מכיל כדור סגור סביב \(a\)

\end{lemma}
זה נובע מכך שמוכל בכדור סגור ברדיוס למשל \(\frac{r}{2}\).

\section{יחסים בין מרחבים מטרים}

\begin{definition}[הומאומורפיזם]
אם קיימת פונקציה \(f:X\to Y\) אשר חח"ע ועל ו-\(f\),\(f^{-1}\) רציפות אז \((X,d)\) ו-\((Y,\rho)\) הם הומאומורפים

\end{definition}
\begin{definition}[שקילות]
אם קיימת פונקציה \(f:X\to Y\) חחע ועל וקיים \(0<C_{1},C_{2}\) כך ש:
$$C_{1}d(x_{1},x_{2})\leq \rho(f(x_{1}),f(x_{2}))\leq C_{2}d(x_{1}x_{2})
$$

\end{definition}
\begin{remark}
לא משנה מה מראים ששקול כיוון שניתן לחלק החלק הימיני ב-\(C_{2}\) ואת השמאלי ב-\(C_{1}\) ולקבל:
$$\frac{1}{C_{2}}\rho(f(x_{1}),f(x_{2}))\leq d(x_{1},x_{2})\leq \frac{1}{C_{1}}\rho(f(x_{1}),f(x_{2}))$$

\end{remark}
\begin{example}
נראה כי הקטעים \(\left( 0,\infty \right)\), \((0,1)\) עם המטריקה המושרת מהמטריקה הסטנדרטית על \(\mathbb{R}\) אינם מרחבים מטרים שקולים. נניח בשלילה כי יש שקילות. לכן קיים פונקציה \(f:(0,1)\to\left( 0,\infty \right)\) המקיימת עבור \(C_{1},C_{2}>0\):
$$C_{1}|x-y|\leq|f(x)-f(y)|\leq C_{2}|x-y|$$
כאשר נשים לב כי \(|x-y|<1\) ולכן \(|f(x)-f(y)|<C_{2}\). לכל \(x,y\in(0,1)\). אך \(f\) צריכה להיות על, ובפרט צריכים להיות \(s,t\in(0,1)\) כך ש-\(f(t)=2C_{2}+1,f(s)=1\) ולכן עבור ערכים אלו מהשקילות מתקיים:
$$|f(s)-f(t)|=|2C_{2}+1-1|=2C_{2}>C_{2}$$
בסתירה לזה ש-\(|f(x)-f(y)|<C_{2}\).

\end{example}
\begin{definition}[מטריקות שקולות]
כאשר שני מטריקות הם שקולות ביחיד עם פונקציית הזהות, המטריקות נקראות מטריקות שקולות. כלומר אם קיימים \(C_{1},C_{2}>0\) כך ש:
$$C_{1}d(x_{1},x_{2})\leq \rho(x_{1},x_{2})\leq C_{2}d(x_{1},x_{2})$$

\end{definition}
\begin{definition}[איזומטריה]
אם קיימת פונקציה \(f:X\to Y\) אשר חח"ע ועל וגם מתקיים:
$$\rho(f(x),f(y))=d(x,y)$$

\end{definition}
כאשר כמו עם שקילות, ניתן להגדיר מטריקות איזומטריות אם הם איזומטריות ביחס לפונקציית הזהות, שזה בפועל אומר שהם אותו הדבר עד כדי שם.

\begin{proposition}
איזומטריה \(\impliedby\) שקילות \(\impliedby\) רציפות

\end{proposition}
\begin{definition}[פונקציה ליפשיץ\רציפה ליפשיץ]
פונקציה \(f:X_{d}\to Y_{\rho}\) כך שקיים \(L\geq0\) כך ש-\(\rho(f(x),f(y))\leq Ld(x,y)\). \(L\) נקרא קבוע לפישיץ של \(f\). מהתנאי אוטומטית נובע רציפות. פונקציה שקילות היא ליפשיצית.

\end{definition}
\begin{proposition}
כל שתי נורמות \(|| \cdot ||\), \(|| \cdot ||'\) על \(\mathbb{R}^n\) הם שקולות, ובפרט משרות מטריקות שקולות.
כלומר קיים \(0<C_{1}\leq C_{2}\) כך ש:
$$C_{1}\|x\|\leq\|x\|^{\prime}\leq C_{2}\|x\|$$

\end{proposition}
\begin{proof}
מספיק להראות כי כל נורמה שקולה לנורמה \(||\cdot||_{1}\). יהי \(||\cdot||\) נורמה כלשהי על \(\mathbb{R} ^n\). מתקיים:
$$||x||=  \left\lVert  \sum_{i} e_{i}x_{i}  \right\rVert = \sum_{i}\lvert x_{i} \rvert \lVert e_{i} \rVert \leq \sum_{i}\lvert x_{i} \rvert \max \lVert e_{i} \rVert =\max\lVert e_{i} \rVert \sum_{i}\lvert x_{i} \rvert =C_{2}\lVert x \rVert_{1} $$
כעת נסתכל על הפונקציה \(f:S^{n-1}\to\mathbb{R}\) (כאשר \(S^{n-1}\) זה ספרת היחידה לפי נורמה \(\lVert \cdot \rVert_{1}\)) המוגדרת ע"י \(f(x)=\lVert x \rVert\). זוהי פונקציה רציפה כצמצום של פונקציה רציפה, וכן \(S^{n-1}\) קומפקטי, לכן פונקציה זו משיגה מינימום ומקסימום. נסמן את במינימום ב-\(m\). נשים לב כי המינימום אינו נמצא ב-0 כיוון שאם היה נמצא ב-0 הינו מקבלים כי \(\lVert x_{0} \rVert=0\) כלומר \(x_{0}=0\) בסתירה לכך שנמצא על ספרת היחידה. לכן \(m>0\). כמו כן מתקיים:
$$\left\|x\right\|=\|x\|_{1}\left\|{\frac{x}{\|x\|_{1}}}\right\|=\|x\|_{1}f{\big(}x/\|x\|_{1}{\big)}\geq\|x\|_{1}f{\big(}x_{0}{\big)}=m\|x\|_{1},$$
וזה משלים את האי שיוויון השני עבור \(C_{1}=m\).

\end{proof}
\begin{proposition}
כל מרחב נורמי נוצר סופית ממימד \(m\) איזומטרית ל-\(\mathbb{R}^m\)  עם נורמה כלשהי.

\end{proposition}
\begin{proof}
יהי \(a_{i}\) בסיס ל-\(V\). נסתכל על ההעתקה הלינארית \(T:V\to \mathbb{R}^m\) הנתון ע"י:
$$T\left(\sum_{i=1}^{m}x_{i}a_{i}\right)=\big(x_{1},\ldots,x_{m}\big)$$
העתקה זו היא הפיכה ובפרט הומאומורפיזם. עבור \(x \in \mathbb{R}^m\) נגדיר:
$$\lVert x \rVert ^{\prime}=\|T^{-1}(x)\|.$$
זוהי נורמה כיוון שבברור \(\lVert x \rVert'\geq 0\)  כאשר יש שיוויון אם"ם \(T^{-1}(x)=x=0\). מתקיים הומוגניות כיוון ש:
$$\|\alpha x\|^{\prime}=\|T^{-1}(\alpha x)\|=\|\alpha\,T^{-1}(x)\|=|\alpha|\|T^{-1}(x)\|=|\alpha|\|x\|_{m}.$$
ולכן קיבלנו ש-\(\lVert \cdot \rVert'\) היא נורמה על \(\mathbb{R}^m\). כעת \(T\) איזומטרייה כיוון שלכל \(u,v\in V\) מתקיים:
$$\left\|u- v \right\|=\left\|T^{-1}T(u-v)\right\|=\left\|T(u-v)\right\|^{\prime}=\|T u-Tv\|^{\prime}.$$

\end{proof}
\section{פונקציות רציפות}

\begin{definition}[רציפות]
פונקציה \(f:X\to Y\) נקראת רציפה ב-\(x \in X\) אם לכל \(\varepsilon>0\) קיים \(\delta>0\) כך שלכל \(z\in B_{\delta}(x)\) מתקיים \(f(z)\in B_{\varepsilon}(f(x))\).

\end{definition}
נשים לב כי זה מתלכד עם ההגדרת \(\varepsilon-\delta\) של רציפות של פונקציות ממשיות. כמו כן, כמו שהיה איפיון של רציפות בעזרת סדרות בפונקציות \(\mathbb{R}\to\mathbb{R}\), גם פה יש הגדרה שקולה לרציפות:

\begin{proposition}[איפיון היינה]
פונקציה \(f:(X,d)\to\left( Y,\rho \right)\) רציפה אם לכל סדרה \(x_{n}\in X\) מתקיים:
$$\operatorname*{lim}_{n\to\infty}x_{n}=a\implies\operatorname*{lim}_{n\to\infty}f(x_{n})=f(a)$$

\end{proposition}
\begin{proof}
נראה ראשית כי ההגדרה הראשונה שקולה לשנייה. כלומר נניח כי מתקיים:
$$\forall\varepsilon>0\quad\exists\delta>0\qquad x\in{\mathcal{B}}_{\delta}(a)\implies f(x)\in{\mathcal{B}}_{\varepsilon}(f(a)).$$
תהי \(x_{n}\) סדרה המתכנסת ל-\(a\). לכן מהגדרת הגבול נקבל:
$$\forall\varepsilon>0\quad\exists N\in\mathbb{N}\quad\mathrm{such\;that}\quad\forall n>N\quad x_{n}\in\mathcal{B}_{\varepsilon}(a).$$
אם נציב בהגדרה של הרציפות נקבל:
$$\forall\varepsilon>0\quad\exists N\in\mathbb{N}\quad{\mathrm{such~that}}\quad\forall n>N\quad f(x_{n})\in{\mathcal{B}}_{\varepsilon}(f(a)),$$
וקיבלנו \(f(x_{n})\to f(a)\). כעת נניח את ההגדרה השנייה ונראה את ההגדרה הראשונה. כלומר נניח \(x_{n}\to a\) גורר \(f(x_{n})\to a\). נניח בשלילה \(f\) לא רציף ב-\(a\). לכן:
$$\exists\varepsilon>0\quad\forall n\in\mathbb{N}\quad\exists x_{n}:d(x_{n},a)<{\frac{1}{n}}\quad\mathrm{and}\quad\rho(f(x_{n}),f(a))\geq\varepsilon.$$
כאשר נשים לב כי הסדרה \(x_{n}\) מתכנסת ל-\(a\), אך הסדרה \(f(x_{n})\) לא מתכנסת ל-\(f(a)\), בסתירה להנחה.

\end{proof}
\begin{example}
  \begin{enumerate}
    \item כל פונקציה קבועה היא רציפה 


    \item אם \(d\) פונקציה מרחק של מרחב מטרי \(X\) ו-\(a\) נקודה שרירותית, אז \(f:x\mapsto d(x,a)\) פונקציה רציפה. זה כיוון שעבור \(\varepsilon=\delta\) נקבל \(d(x,z)<\delta\) ולכן\\
$$\left|f(x)-f(z)\right|=\left|d(x,a)-d(z,a)\right|\leq d(x,z)<\varepsilon.$$


    \item כל פונקציה המוגדרת על המרחב הדיסקרטי שהטווח שלה הוא מרחב מטרי היא רציפה. ניתן לקח 


  \end{enumerate}
\end{example}
\begin{proposition}
יהיו \((X,d),\left( Y,\rho \right)\) מרחבים מטרים. תהי פונקציה \(f:(X,d)\to\left( Y,\rho \right)\). מתקיים
- פונקציה \(f\) רציפה אם"ם לכל \(U\subseteq Y\) פתוחה, \(f^{-1}(U)\subseteq X\) פתוחה.
- פונקציה \(f\) רציפה אם"ם לכל \(U\subseteq Y\) סגורה, \(f^{-1}(U)\subseteq X\) סגורה.

\end{proposition}
\begin{proposition}
הרכבה של פונקציות רציפות היא רציפה

\end{proposition}
\begin{proposition}
יהיו \(f,g:X\to\mathbb{R}\) רציפות. אזי \(f\cdot g,f\pm g\) רציפות וכן \(\frac{f}{g}\) רציף כל עוד \(g\) לא מתאפסת.

\end{proposition}
\begin{proposition}
יהי \(V\) מרחב וקטורי סוף מימדי, ויהי \(W\) מרחב נורמי(לאו דווקא סוף מימדי). אזי כל העתקה לינארית \(T:V\to W\) היא רציפה(לפי כל הנורמות על \(V\), שהם שקולות)

\end{proposition}
\begin{proposition}
כל אופרטור לינארי חסום הוא רציף.

\end{proposition}
\begin{proposition}
כל נורמה היא פונקציה רציפה.

\end{proposition}
\section{קומפקטיות}

\begin{definition}[קומפקטיות סדרתית]
מרחב מטרי \((X,d)\) יקרא קומפקטי אם לכל סדרה יש תת סדרה מתכנסת לאיבר בקבוצה.

\end{definition}
\textbf{דוגמאות:}
- קטעים סגורים ב-\(\mathbb{R}\)
- קבוצות סופיות, עם כל מטריקה
- איחוד סופי של קבוצות קומפקטיות

\subsection{תכונות של מרחבים קומפקטים}

\begin{proposition}[ערכים אקסטרימלים]
כל פונקציה רציפה ממרחב מטרי קומפקטי משיגה מינימום ומקסימום

\end{proposition}
\begin{proposition}
פונקציות רציפות הפועלות על תתי מרחבים מטרים קומפקטים הולכות לתתי מרחבים קומפקטים כלומר אם \(K\subseteq X\) קומפקטי ו-\(f:X\to Y\) אז \(f(K)\) קומפקטי.

\end{proposition}
\begin{proposition}
מרחב מטרי קומפקטי הוא סגור וחסום

\end{proposition}
\begin{proof}
יהי \(K\) תת מרחב מטרי קומפקטי.
סגירות: נניח בשלילה שקיים סדרה של איברים מ-\(K\) שמתכנס לאיבר שלא ב-\(K\). לפי הקומפקטיות, לסדרה זו תהיה תת סדרה שתכנסת לאיבר מ-\(K\). אבל כל התתי סדרות מתכנסים לאותו איבר, בסתירה.
חסימות: נניח בשלילה ש-\(K\) לא חסום. ויהי \(x_{0}\in K\). כיוון ש-\(K\) לא חסום קיים סדרה \(\{ x_n\}_{n=0}^{\infty}\)  כך ש-\(d(x_{n},x_{0})>n\). כיוון ש-\(K\) קומפקטית סדרית, קיים לסדרה זו תת סדרה \(\{ x_{n_{k}}\}_{k=0}^{\infty}\)  המתכנסת ל-\(x \in K\) לכל איבר בתת סדרה זו מתקיים:
$$n_{k}<d(x_{0},x_{n_{k}})\leq d(x_{0},x)+d(x,x_{n_{k}}).$$
כיוון ש-\(x_{n_{k}}\to x\) נקבל סתירה כי האגף הימיני חסום והשמאלי מתבדר(\(n_{k}\) מונוטונית עולה ממש)

\end{proof}
\begin{proposition}
אם \((X,d)\) מרחבים מטרים קומפקטים אז כל תת קבוצה סגורה של \(K\subseteq X\) היא קומפקטית.

\end{proposition}
\begin{proof}
אם תת קבוצה היא סגורה היא מכילה את כל נקודות הגבול שלה, לכן אם יש סדרה של איברים ב-\(K\) מהקומפקטיות של \(X\) יש לה תת סדרה מתכנסת ב-\(X\). מהסגירות של \(K\) נקבל כי הגבול נמצא ב-\(K\). ולכן קומפקטית 

\end{proof}
\begin{corollary}
אם \((X,d)\) מרחב מטרי קומפקטי, ו-\(\left( Y,\rho \right)\) מרחב מטרי כללי, אז אם \(f:X\to Y\) רציף, אז לכל קבוצה סגורה \(C\subseteq X\) נקבל \(f(C)\) סגור.

\end{corollary}
\begin{proof}
כיוון ש-\(X\) קומפקטית, גם \(C\) קומקטית כיוון שסגורה מהטענה הקודמת. \(f(C)\) היא לכן קומפקטית כיוון ש-\(f\) רציפה, ולכן סגורה וחסומה. 

\end{proof}
\begin{proposition}
איחוד סופי של קבוצות קומפקטיות הוא קומפקטי

\end{proposition}
\begin{proof}
בהנתן סדרה \(x_{n}\in \bigcup_{i}K_{i}=K\) אנו יודעים כי כיוון שזהו איחוד סופי חייב להיות \(K_{i}\) שבתוכו הסדרה מופיעה אינסוף פעמים. לכן קיים תת סדרה ב-\(K_{i}\) ומקומפקטיות לתת סדרה זו קיימת תת סדרה מתכנסת ב-\(K_{i}\). לכן לסדרה \(x_{n}\) יש תת סדרה המתכנסת ב-\(K_{i}\subseteq K\) ולכן קומפקטי

\end{proof}
\begin{definition}[רציפות במידה שווה]
יהיו \((X,d), \left( Y,\rho \right)\) מרחבים מטרים. פונקציה \(f:(X,d)\to \left( Y,\rho \right)\) נקרא רציף במידה שווה אם לכל \(\varepsilon>0\) קיים \(\delta>0\) כך שלכל \(x,y\in X\) מתקיים:
$$d(x,y)<\delta\implies \rho(f(x),f(y))<\varepsilon$$

\end{definition}
\begin{proposition}
יהיו \((X,d),\left( Y,\rho \right)\) מרחבים מטרים כאשר \((X,d)\)  קומפקטי, אז אם \(f: X\to Y\) רציף אז רציף במידה שווה

\end{proposition}
\begin{proof}
נניח \(f\) רציף אך לא רציף במידה שווה, לכן קיים \(\varepsilon>0\) כך שלכל \(\delta>\) קיימים \(x,y\in X\) כך ש-\(d(x,y)<\delta\), אבל \(\rho(f(x),f(y))\geq\varepsilon\). 
יהי \(\varepsilon>0\). נבחר \(\delta_{n}=\frac{1}{n}\). וכאשר  \(x_{n},y_{n}\in X\)  הם כך ש-\(d(x_{n},y_{n})<\delta_{n}=\frac{1}{n}\). לכן מתקיים:
$$d(x_{n},y_{n})<{\frac{1}{n}},\qquad\rho(f(x_{n}),f(y_{n}))\geq\varepsilon.$$
כיוון ש-\(X\) קומפקטי קיים לסדרה \(x_{n}\) תת סדרה מתכנסת \(x_{n_{k}}\) המתכנסת ל-\(x\). כאשר אנו יודעים כי \(y_{n_{k}}\) היא תת סדרה מתכנסת של \(y_{n}\). קיים אבל גם ל-\(y_{n_{k}}\) תת סדרה מתכנסת \(y_{n_{k_{j}}}\) מקומפקטיות ל-\(y\). נזכור כי \(x_{n_{k_{j}}}\to x\) מקומפקטיות.  כעת מהגדרה של גבול בעזרת סדרות נקבל:
$$d(x,y)=\operatorname*{lim}_{j\to\infty}d(x_{n_{k_{j}}},y_{n_{k_{j}}})\leq\operatorname*{lim}_{j\to\infty}{\frac{1}{n_{k_{j}}}}=0,$$
לכן \(x=y\). מצד שני כיוון ש-\(f\) רציפה נקבל:
$$\lim_{j\to\infty}\rho\bigl(f\bigl(x_{n_{k_{j}}}\bigr),f\bigl(x\bigr)\bigr)=0 \qquad \lim_{j\to\infty}\rho\bigl(f(y_{n_{k_{j}}}),f(x)\bigr)=0$$
ולכן נקבל כעת כי:
$$\operatorname*{lim sup}_{j\to\infty}\rho(f(x_{n_{i_{j}}}),f(y_{n_{i_{j}}}))\leq\operatorname*{lim sup}_{j\to\infty}\left(\rho(f(x_{n_{i_{j}}}),f(x))+\rho(f(x),f(y_{n_{i_{j}}}))\right)=0,$$
בסתירה לכך ש-\(\rho(f(x_{n_{k_{j}}}),f(y_{n_{k_{j}}}))\geq\varepsilon\) לכל \(j\).

\end{proof}
\begin{remark}
זה מתאים למשפט קנטור ב-\(\mathbb{R}\) שאומר שפונקציה רציפה בקטע סגור רציפה במ"ש.

\end{remark}
\begin{theorem}[בולצנו וירשטראס]
תת קבוצה \(K\subseteq\left( \mathbb{R} ^n,||\cdot||_{2} \right)\) היא קומפקטית אם"ם \(K\) חסומה וסגורה

\end{theorem}
\begin{proof}
אם הכיוון שכל \(K\) קומפקית חייבת להיות סגורה וחסומה אנחנו מכירים. כעת נניח כי \(K\) סגורה וחסומה. כיוון שחסומה, קיים $$Q=[a_{1},b_{1}]\times\ldots\times[a_{n},b_{n}]$$
כך ש-\(K\) מוכלת בתוכה. כיוון ש-\(K\) סגורה מספיק להראות כי \(Q\) קומפקטית. נשים לב כי כל רכיב של הווקטור \(Q\) איזומורפי ל-\(\mathbb{R}\). נסכל על סדרה כללית של איברים ב-\(Q\). נסתכל ראשית רק על הרכיב הראשון. מבולצנו ווירשטרס ב-\(\mathbb{R}\) אנו יודעים כי יש תת סדרה מתכנסת \(x_{1}^{k_{m}}\). כעת נעבור לרכיב השני. נשים לב כי \(x_{2}^{k_{m}}\) היא סדרה ברכיב השני לכן יש לו תת סדרה מתכנסת \(x_{2}^{k_{m_{j}}}\). נמשיך ככה הלאה ל-\(n\) רכיבים עד שנקבל תת סדרה \(x^{q}\) המתכנסת לפי כל הרכיבים, ולכן קומפקטית.

\end{proof}
קיימת אף גרסה חזרה יותר של טענה זו:

\begin{proposition}
יהי \(\left( V,||\cdot|| \right)\) מרחב נורמי נוצר סופית. כל קבוצה סגורה וחסומה היא קומפקטית

\end{proposition}
\begin{proof}
זה נובע ישירות מכך שכל מרחב נורמי נוצר סופית איזומטרי ל-\(\mathbb{R}^n\) עם נורמה כלשהי, ולכן כיוון שהטענה הקודמת מתקיימת עבור \(\mathbb{R}^n\), תתקיים עבור מרחב נורמי כללי.

\end{proof}
\section{שלמות}

\begin{definition}[סדרת קושי]
יהי \((X,d)\) מרחב מטרי, \(\{ x_n\}_{n=0}^{\infty}\)  נקראת סדרת קושי אם לכל \(\varepsilon>0\) קיים \(N\) כך שלכל \(n,m>N\) מתקיים \(d(x_{n},x_{m})<\varepsilon\) .

\end{definition}
\begin{definition}[מרחב מטרי שלם]
מקיים שכל סדרת קושי מתכנס לאיבר במרחב מטרי.

\end{definition}
\begin{proposition}
יהי \((X,d), \left( Y,\rho \right)\) מרחבים מטרים שקולים. \(X\) שלם אם"ם \(Y\) שלם.

\end{proposition}
\begin{proof}
יהי \(f:X\to Y\) העתקה חח"ע ועל כך שעבור \(c_{2}>c_{1}>0\) מתקיים לכל \(x,y\in X\):
$$c_{1}d(x,y)\leq\rho(f(x),f(y))\leq c_{2}d(x,y).$$
נניח \(Y\) שלם, ו-\(x_{n}\) סדרת קושי ב-\(X\). מהאי שיוויון הימיני הקבל כי \(f(x_{n})\) היא סדרת קושי ב-\(Y\), ולכן מתכנסת ל-\(y\in Y\). כיוון ש-\(f\) על, קיים \(x \in X\) כך ש-\(f(x)=y\). ולכן נקבל מהאי שיוויון השמאלי כעת כי \(f(x_{n})\to y\), גורר \(x_{n}\to x\).

\end{proof}
\begin{theorem}[השלמה למרחב מטרי]
יהי מרחב מטרי \((X,d)\), קיים \((\hat{X},\hat{d})\)  מרחב מטרי שלם והעתקה \(i:X\to \hat{X}\) כך ש:

  \begin{enumerate}
    \item מתקיים \(i:X\to i(X)\) איזומטריה.  


    \item התמונה \(i(x)\) צפופה ב-\(\hat{X}\).(\(\overline{i(x)}=\hat{X}\)). 
כאשר \((\hat{X},\hat{d})\) יחידים עד כדי איזומטריה.


  \end{enumerate}
\end{theorem}
\begin{definition}[העתקה מכווצת]
יהא \((X,d)\) מרחב מטרי. פונקציה \(f:X\to X\) נקראת מכווצת אם קיים \(0\leq \lambda<1\) כך שלכל \(x\),\(y\):
$$d(f(x),f(y))\leq \lambda d(x,y)$$

\end{definition}
\begin{theorem}[ ההעתקה המכווצת]
יהא \((X,d)\) מרחב מטרי שלם. ותהא \(f:X\to X\) מכווצת, אזי יש ל-\(f\) נקודת שבת יחידה. כלומר יש \(x \in X\) יחיד כך ש-\(f(x)=x\).

\end{theorem}
\begin{proof}
יהי \(x_{0}\in X\). נגדיר באופן רקורסיבי את \(x_{n}\) ע"י \(x_{0}=x_{0}\quad  x_{n}=f(x_{n})\)
ראשית נראה כי זוהי סדרת קושי. נשים לב כי ממהנחה מתקיים:
$$d(x_{n+1},x_{n})=d(f(x_{n}),f(x_{n-1}))\leq\lambda\,d(x_{n},x_{n-1}),$$
וכן מאינדוקציה נקבל כי \(d(x_{n+1},x_{n})\leq \lambda^nd(x_{1},x_{0})\). כעת עבור \(n>m\geq N\) נקבל:
$${{d(x_{n},x_{m})\leq\displaystyle\sum_{k=m}^{n-1}d(x_{k+1},x_{k})\leq\displaystyle\sum_{k=m}^{n-1}\lambda^{k}d(x_{1},x_{0})}}{{\leq\displaystyle\sum_{k=N}^{\infty}\lambda^{k}d(x_{1},x_{0})=\displaystyle\frac{\lambda^{N}}{1-\lambda}d(x_{1},x_{0}).}}$$
כעת, בהנתן \(\varepsilon>0\), נבחר \(N\) כך ש-\(\frac{\lambda^N}{1-\lambda}d(x_{1},x_{0})<\varepsilon\) ולכל \(n,m>N\) נקבל \(d(x_{n},x_{m})<\varepsilon\). כיוון ש-\(X\) שלם, נקבל כי \(x_{n}\) מתכנס. נסמן את הגבול ב-\(x\). הנקודה \(x\) היא אכן נקודת שבת. כיוון ש-\(f\) רציף(והמטריקה רציפה) אנחנו מקבלים:
$$d(x,f(x))=\operatorname*{lim}_{n\to\infty}d(x_{n},f(x_{n}))=\operatorname*{lim}_{n\to\infty}d(x_{n},x_{n+1})=0$$
ולכן \(x=f(x)\) וזוהי אכן נקודת שבת. כעת נדרש להראות רק יחידות.
נניח בשלילה שיש שני נקודות שבת \(x\neq y\) המקיימות \(f(x)=x,f(y)=y\). מתקיים מהנתון:
$$d(x,y)=d(f(x),f(y))\leq\lambda\,d(x,y)\implies d(x,y)\left( 1-\lambda \right)\leq 0$$
וקיבלנו כי \(d(x,y)\leq 0\) ומאי שליליות המטריקה נקבל כי \(d(x,y)=0\) ולכן \(x=y\) בסתירה.

\end{proof}
\begin{theorem}[העתקה כמעט מכווצת]
יהי \((X,d)\) מרחב מטרי שלם וקומפקטי, \(f:X\to X\) המקיימת:
$$d(f(x),f(y))< d(x,y)$$
אז יש ל-\(f\) נקודת שבת יחידה

\end{theorem}
\begin{remark}
ההבדל בין משפט ההעתקה המכווצת למשפט ההעתקה הכמעט מכווצת(שם לא רשמי) הוא שבהעתקה הכמעט מכווצת לא נדרש \(\lambda\) ויש אי שיוויון ממש אבל בתמורה לכך צריך לדרוש גם קומפקטיות.

\end{remark}
\begin{proof}
נראה שמהקומפקטיות נובע שההעתקה מכווצת ממש. 
נניח בשלילה שקיים סדרה כך שלכל \(n\in\mathbb{N}\) יש \((x_{n},y_{n})\) כך ש-
$$.d(f(x_{n}),f(y_{n}))\geq\left( 1-{\textstyle{\frac{1}{n}}} \right)d(x_{n},y_{n})$$
מאחר ו-\(X\) קומפקטית, יש תת סדרה מתכנסת. נעבור לאחת כזו ונניח שמתכנסת ל-\((x,y)\).
לכן מרציפות ומהנחת השלילה נקבל:
$$d(f(x),f(y))=\operatorname*{lim}_{n\to\infty}d(f(x_{n}),f(y_{n}))\leq\operatorname*{lim}_{n\to\infty}(1-{\frac{1}{n}})d(x_{n},y_{n})=d(x,y)$$
לכן \(d(f(x),f(y))=d(x,y)\) בסתירה להיות \(f\) כמעט מכווצת. לכן כל סדרה מקיימת:
$$d(f(x),f(y))<(1-{\textstyle{\frac{1}{n}}})d(x,y)$$
ולכן \(f\) באמת מכווצת, וקיים נקודת שבת יחידה.

\end{proof}
\section{אוסף מטריקות}

\textbf{המטריקה הדיסקרטית}
קיים עבור כל קבוצה \(X\). ניתן להגדיר פונקציית מרחק באופן הבא:
$$\forall x,y\in X:\,d(x,y)={\left\{\begin{array}{l l}{1}&{x\neq y}\\ {0}&{x=y}\end{array}\right.}$$
- סדרה מתכנסת במטריקה הדיסקרטית אם"ם היא קבועה החל ממקום מסויים.
- כל קבוצה במטריקה הדיסקרטית היא גם סגורה וגם פתוחה.

\textbf{מטריקה \(p\) אדית}
ניתן להגדיר פונקציה \(|\cdot|_{p}:\mathbb{Q}\to \mathbb{R}_{\geq 0}\), שתקבל איבר מהצורה רציונאלי מהצורה \(p^n \frac{a}{b}\) כאשר \(a,b\) זרים ל-\(p\), ותחזיר את \(p^{-n}\). מקיימת \(|a\cdot b|_{p}=|a|_{p}\cdot |b|_{p}\). וכן מקיימת את אי שיוויון המשולש החזק:
$$|x+y|_{p}\leq \max \{ |x|_{p},|y|_{p} \}$$
זו תהיה מטריקה, אך לא תהיה מושרת מנורמה.

\subsection{מרחבים נורמים}

\textbf{נורמת sup}
מוגדר במרחב הפונקציות. הנורמה מוגדרת:
$$\left\Vert{f}\right\Vert_{\infty}=\operatorname*{sup}_{[0,1]}\left\vert{f}\right\vert=\operatorname*{sup}_{x\in[0,1]}\left\vert{f}\left(x\right)\right\vert$$
כאשר בקטע סגור קומפקטי ולכן מקבל את ערכו המקסימלי. ולכן \(||f||_{\infty}=\max|f(x)|\)

\textbf{נורמת \(p\)}$$\|(x_{1},...x_{d})\|_{p}=\left(\sum_{i=1}^{d}|x_{i}|^{p}\right)^{\frac{1}{p}}$$
כאשר במקרה הגבולי:
$$\left\Vert(x_{1},...x_{d})\right\Vert_{\infty}=\operatorname*{max}_{i}\left\vert x_{i}\right\vert$$

המרחב \(l^p\):
לכל \(1\leq p<\infty\) ו-\(d\in\mathbb{N}\cup \left\{  \infty  \right\}\) ניתן להסתכל על אוסף הסדרות(סופיות או אינסופיות) הבא:
$$\ell^{p}\left(\mathbb{R}^{d}\right)=\left\{\left(x_{n}\right)_{n=1}^{d}\ :\ \sum|x_{i}|^{p}<\infty\right\}\subseteq\mathbb{R}^{d}$$
כאשר הסדרה סופית, למעשה כל סדרה ניתנת ליצוג ע"י וקטור, ונקבל בדיוק את \(\mathbb{R}^d\). על האוסף הזה ניתן להגדיר את נורמה:
$$||x||_{p}=\left(\sum_{i=1}^{d}|x_{i}|^{p}\right)^{\frac{1}{p}}$$

\Chapter{נגזרות של פונקציה מרובת משתנים}

\section{נגזרות}

אנחנו כרגע מתייחסים לפונקציות מהצורה \(f:\mathbb{R}^k\to\mathbb{R}^m\). ניתן לכתוב פונקציה כזו כאוסף של פונקציות \(f_{i}:\mathbb{R}\to\mathbb{R}^m\). כלומר ניתן לכתוב:
$$f(x)=\begin{pmatrix}{{f_{1}\left(x_{1},\ldots,x_{k}\right)}}\\ {{\ldots}}\\ {{f_{m}\!\left(x_{1},\ldots,x_{k}\right)}}
\end{pmatrix} \qquad  f_{\alpha}=\left\langle f,e_{\alpha}\right\rangle$$

\begin{definition}[דיפרנציאבליות]
פונקציה \(f:\mathbb{R}^k\to\mathbb{R}^m\) נקראת גזירה/דיפרציאבלית ב-\(a\) אם קיימת העתקה הלינארית \(T:\mathbb{R}^k\to\mathbb{R}^m\) המקיימת:
$$\operatorname*{lim}_{v\to0}{\frac{f(a+v)-f(a)-T(v)}{||v||}}=0_{\mathbb{R}^m}$$

\end{definition}
\begin{definition}[הנגזרת בנקודה]
אם פונקציה היא דיפרנציאבילית, אז ההעתקה היחידה \(Df_{a}(v)\) המקיימת:
$$\operatorname*{lim}_{v\to0}{\frac{f(a+v)-f(a)-Df_{a}(v)}{||v||}}=0_{\mathbb{R}^m}$$
נקראת הנגזרת בנקודה \(a\). 

\end{definition}
נשים לב כי כיוון שאנחנו במטריקת \(l_{p}\), התכנסות איבר איבר לאפס שקול להתכנסות של הנורמה, לכן הגדרה שקולה לגזירות היא:
$$\lim_{ v \to 0 } \frac{||f(a+v)-f(a)-T(v)||_{\mathbb{R}^m}}{||v||_{\mathbb{R}^k}}=0$$\textbf{הגדרה} פונקציית הנגזרת
תהי \(A\subseteq \mathbb{R}^k\) פתוחה, ותהי \(f:A\to\mathbb{R}^m\) גזירה בכל נקודה \(x \in A\). אזי \(Df:A\to \mathrm{Hom}\left( \mathbb{R}^k,\mathbb{R}^m \right)\) הומגדרת ע"י \(x\to (Df)_{x}\) נקראת פונקציית הנגזרת.

\begin{proposition}
אם הפונקציה היא גזירה, הנגזרת היא יחידה.

\end{proposition}
\begin{proof}
נניח \(T,S\) העתקות לינארית, אשר מקיימות:
$$\operatorname*{lim}_{v\to0}{\frac{f(x+v)-f(x)-Tv}{\|v\|_{\mathbb{R}^{k}}}}=0\qquad\operatorname*{lim}_{v\to0}{\frac{f(x+v)-f(x)-Sv}{\|v\|_{\mathbb{R}^{k}}}}=0.$$
עבור שינוי קטן \(v\in \mathbb{R}^n\) מתקיים:
$$(S-T) v =(f(x+ v )-f(x)-T v )-(f(x+ v )-f(x)-S v )\,.$$
לפי אי שיוויון המשולש:
$$\operatorname*{lim}_{ v \to0}{\frac{\|(S-T) v \|_{\mathbb{R}^{m}}}{\| v \|_{\mathbb{R}^{k}}}}\leq\operatorname*{lim}_{ v \to0}{\frac{\|f(x+ v )-f(x)-T v \|_{\mathbb{R}^{m}}}{\| v \|_{\mathbb{R}^{k}}}}+\operatorname*{lim}_{ v \to0}{\frac{\|f(x+ v )-f(x)-S v \|_{\mathbb{R}^{m}}}{\| v \|_{\mathbb{R}^{k}}}}=0.$$
מהומוגניות הנורמה:
$$\operatorname*{lim}_{ v \to0}\left\|\left(S-T\right)\left({\frac{ v }{\| v \|_{\mathbb{R}^{k}}}}\right)\right\|_{\mathbb{R}^{m}}=0.$$
הגבול הזה נכון לכל סדרה \(v\to 0\). ניקח \(v=te_{j}\) ונשאיף \(t\to 0\) ונקבל לכל \(j\):
$$(S-T)(e_{j})=0,$$
ולכן \(S=T\).

\end{proof}
\begin{definition}[נגזרת חלקית]
$$\partial_{j}f(x)=\operatorname*{lim}_{t\rightarrow0}{\frac{f(x+t e_{j})-f(x)}{t}}.$$
כאשר ניתן גם לפרק את \(f\) ולכתוב נגזרת חלקית של רכיב של \(f\):
$$\partial_{j}f_{i}(x)=\operatorname*{lim}_{t\rightarrow0}{\frac{f_{i}(x+t e_{j})-f_{i}(x)}{t}}\qquad i=1,\ldots,m\quad j=1,\ldots,k.$$

\end{definition}
\begin{proposition}[נגזרת כיוונת]
יהי \(A\subseteq \mathbb{R}^k\) ו-\(f:A\to\mathbb{R}^m\) גזירה בנקודה פנימית. לכן לכל \(v\in \mathbb{R}^k\) מתקיים:
$$(D f)_{x}( v )=\operatorname*{lim}_{t\rightarrow0}{\frac{f(x+t v )-f(x)}{t}}.$$

\end{proposition}
כאשר למעשה ניתן לפרש את \((Df)_{x}(v)\) בתור הקצב שינוי של \(f\) כשזזים מ-\(x\) בכיוון של \(v\).

\begin{proposition}
אם פונקציה היא גזירה ב-\(a\) אז כל הנגזרות הכיווניות קיימות, ומתקיים:
$$\left(D f\right)_{a}\left(v\right)=\operatorname*{lim}_{t\rightarrow0}{\frac{f\left(a+t v\right)-f\left(a\right)}{t}}$$

\end{proposition}
\begin{corollary}
יהי \(A\subseteq \mathbb{R}^k\) ו-\(f:A\to\mathbb{R}^m\) גזיר בנקודה פנימית \(x \in A\) אז כל הנגזרות החלקיות קיימות ומתקיים:\(\partial_{j}f(x)=(D f)_{x}(e_{j}).\)

\end{corollary}
\begin{proof}
לפי הגדרה, \(f\) רציפה ב-\(x\), וכעת לכל \(v\neq 0\) מתקיים:
$$\operatorname*{lim}_{t\to0}{\frac{f(x+t v )-f(x)-(D f)_{x}(t v )}{\|t v \|_{\mathbb{R}^{k}}}}=0.$$
וכעת מלינאריות של הנגזרת וההומוגניות של הנורמה נקבל:
$$\operatorname*{lim}_{t\to0}{\frac{f(x+t v)-f(x)-t\left(D f\right)_{x}(v)}{t}}=\operatorname*{lim}_{t\to0}{\frac{f(x+t v)-f(x)-\left(D f\right)_{x}(t v)}{\|t v\|_{\mathbb{R}^{k}}}}{\frac{|t|\|v\|_{\mathbb{R}^{k}}}{t}}=0,$$
כיוון ש-\(\frac{f(x+t v )\!-\!f(x)\!-\!(D f)_{x}(t v )}{\|t v \|_{\mathbb{R}^{k}}}\) שואף ל-0 ו-\(\frac{|t|\|v\|_{\mathbb{R}^{k}}}{t}\) חסום אז נקבל כי הגבול שואף ל-0.

\end{proof}
\begin{proposition}[ההצגה בעזרת מטריצה]
כיוון שהנגזרת היא העתקה לינארית אז ניתן להציג אותה בעזרת מטריצה בצורה הבא:
$$(D f)_x=\left(\begin{array}{lllll}\partial_1 f(x) & \mid & \ldots & \mid & \partial_k f(x)\end{array}\right)=\left(\begin{array}{cccc}\partial_1 f_1(x) & \ldots & \partial_k f_1(x) \\\vdots & \ddots & \vdots \\\partial_1 f_m(x) & \ldots & \partial_k f_m(x)
\end{array}\right)$$

\end{proposition}
\begin{proposition}
אם פונקציה היא גזירה בנקודה, אז היא גם רציפה בה

\end{proposition}
\begin{proof}
נסמן את הנגזרת בנקודה ב-\(T\). נקבל מהגדרת הנגזרת:
$$\operatorname*{lim}_{ v \to0}{\frac{f\left( x+ v  \right)-f(x)-T\left(  v  \right)}{\| v \|_{\mathbb{R}^{k}}}}=0_{\mathbb{R}^k}$$
ולכן המונה שואף ל-0. מתקיים \(\lim_{ v \to 0 }T(v)=0\) ולכן מאריתמטיקה של גבולות נקבל:
$$\lim_{ v \to 0 } f(x+v)-f(x)=0$$
ולכן רציף.

\end{proof}
\begin{proposition}
הנגזרת הוא אופרטור לינארי. כלומר אם \(f,g:A\to\mathbb{R}m\) פונקציות גזירות אז \(h=\alpha f+\beta g\) גזירה וכן מתקיים \((Dh)_{x}=\alpha (Df)_{x}+\beta (Dg)_{x}\).

\end{proposition}
\begin{proposition}
יהי \(A\subseteq \mathbb{R}^k\) קבוצה פתוחה, ויהי \(f:A\to\mathbb{R}^m\). נסמן את הרכיבים \(f_{1},\dots,f_{m}\) פונקציות מ-\(f_{i}:A\to\mathbb{R}\). אזי \(f\) דיפרנציאבילי אם"ם לכל \(1\leq i\leq m\) נקבל \(f_{i}\) דיפרנציאבילית.

\end{proposition}
\begin{proof}
אם \(f\) דיפרנציאבילי אז קיים העתקה לינארית \(T\) כך ש-
$$\operatorname*{lim}_{ v \to0}{\frac{f\left( x+ v  \right)-f(x)-T\left(  v  \right)}{\| v \|_{\mathbb{R}^{k}}}}=0_{\mathbb{R}^k}$$
זהו גבול וקטורי, זה אומר שנדרש שכל אחד מהרכיבים של \(Tv\) אשר אשר נסמן ב-\(T_{i}v\) מתכנס ל-0. כלומר מתקיים:
$$\operatorname*{lim}_{ v \to0}{\frac{f_{i}(x+ v )-f_{i}(x)-(T( v ))_{i}}{\| v \|_{\mathbb{R}^{k}}}}=0.$$
וזה למעשה אומר כי \(T(v)_{i}\) זה הנגזרת של \(f_{i}\). הכיוון השני מושג באופן דומה, אם כל אחד מהגבולות האלה מתכנסות ל-0, אז גם הגבול של \(Tv\) מתכנס ל-0 ולכן \(T\) הנגזרת.

\end{proof}
\begin{theorem}[תנאי מספיק לדיפרנציאביליות]
אם \(A\subseteq \mathbb{R}^k\) קבוצה פתוחה ו-\(f:A\to\mathbb{R}^m\), אז אם כל ה-\(k\) נגזרות הכיווניות של \(f\) קיימות בסביבה של \(x \in\mathbb{R}^k\) וגם רציפות ב-\(x\), אז \(f\) גזיר ב-\(x\).

\end{theorem}
\begin{proof}
מהטענה הקודמת מספיק להראות כי עבור \(1\leq i\leq m\) מתקיים. לכן מספיק להראות שקיים העתקה לינארית \(T\in \mathrm{Hom}\left( \mathbb{R}^k,\mathbb{R} \right)\) כך ש:
$$\operatorname*{lim}_{ v \to0}{\frac{f\left( x+ v  \right)-f(x)-T\left(  v  \right)}{\| v \|_{\mathbb{R}^{k}}}}=0_{\mathbb{R}^k}$$
אנחנו ודעים כי אם \(T\) קיים אז מתקיים:
$$T=\left(\partial_{1}f(x),\ldots,\partial_{k}f(x)\right).$$
ולכן צריך להראות כי:
$$\operatorname*{lim}_{ v \to0}{\frac{1}{\| v \|_{\mathbb{R}^{k}}}}\left(f(x+ v )-f(x)-\sum_{j=1}^{k}\partial_{j}f(x) v _{j}\right)=0.$$
כיוון ש-\(f\) יש נגזרות כיוויניות רציפות, נרצה להשתמש במשפט ערך הביניים על כל רכיב. נכתוב את השינוי ב-\(f\) בתור סכום טלסקופי:
\begin{gather*}{f{\big(}x+ v {\big)}-f{\big(}x{\big)}=f{\big(}x+ v _{1}e_{1}{\big)}-f{\big(}x{\big)}}\\ {+f{\big(}x+ v _{1}e_{1}+ v _{2}e_{2}{\big)}-f{\big(}x+ v _{1}e_{1}{\big)}}\\ {+\ldots}\\ {+f{\big(}x+ v {\big)}-f{\big(}x+ v _{1}e_{1}+ v _{2}e_{2}+\cdots+ v _{k-1}e_{k-1}{\big)}.}\end{gather*}
ונקבל:
$$f(x+ v )-f(x)=\sum_{j=1}^{k}\left(f\left(x+\sum_{i=1}^{j} v _{i}e_{i}\right)-f\left(x+\sum_{i=1}^{j-1} v _{i}e_{i}\right)\right).$$
נסתכל על האיבר ה-\(j\) בסכום ונגדיר:
$$g_{j}(t)=f\left(x+\sum_{i=1}^{j-1} v _{i}e_{i}+t e_{j}\right),$$
כך שהאיבר ה-\(j\) בסכום שווה ל-\(g_{j}(v_{j})-g_{j}(0)\) ומתקיים:
$$g_{j}^{\prime}(t)=\operatorname*{lim}_{h\to0}{\frac{g_{j}(t+h)-g_{j}(t)}{h}}=\operatorname*{lim}_{h\to0}{\frac{f\left(x+\sum_{i=1}^{j-1} v _{i}e_{i}+t e_{j}+h e_{j}\right)-f\left(x+\sum_{i=1}^{j-1} v _{i}e_{i}+t e_{j}\right)}{h}}$$
כעת לפי משפט ערך הביניים(החד מימדי) קיים \(\theta_{j}\in(0,1)\) כך ש:
$$g_{j}( v _{j})-g_{j}(0)=g_{j}^{\prime}(\theta_{j} v _{j}) v _{j}.$$
ונקבל סה"כ:
$$f(x+ v )-f(x)=\sum_{j=1}^{k}g_{j}( v _{j})-g_{j}(0)=\sum_{j=1}^{k}\partial_{j}f\left(x+\sum_{i=1}^{j-1} v _{i}e_{i}+\theta_{j} v _{j}e_{j}\right) v _{j}.$$
ולכן:
$${\frac{1}{\|v\|_{\mathbb{R}^{k}}}}\left(f(x+ v )-f(x)-\sum_{j=1}^{k}\partial_{j}f(x) v _{j}\right)=\sum_{j=1}^{k}\left(\left[\partial_{j}f\left(x+\sum_{i=1}^{j-1} v _{i}e_{i}+\theta_{j} v _{j}e_{j}\right)-\partial_{j}f(x)\right]{\frac{ v _{j}}{\|v\|_{\mathbb{R}^{k}}}}\right).$$
נשים לב כי \(\frac{v_{j}}{||v||_{\mathbb{R}_{k}}}\) חסום. כאשר מהרציפות של הנגזרת נקבל:
$$\partial_{j}f\left(x+\sum_{i=1}^{j-1} v _{i}e_{i}+\theta_{j} v _{j}e_{j}\right)-\partial_{j}f(x)\xrightarrow{v\to 0} 0$$
ולכן נקבל כי הביטוי כולו שואף לאפס, ולכן דיפרנציאבילי.

\end{proof}
\begin{definition}[גזיר ברציפות]
אם \(A\subseteq \mathbb{R}^k\) קבוצה פתוחה ו-\(f:A\to\mathbb{R}^m\) פונקציה גזירה ב-\(A\). פונקציה \(f\) היא גזירה ברציפות ב-\(x\) אם ההעתקה \(Df:A\to \mathrm{Hom}\left( \mathbb{R}^k,\mathbb{R}^m \right)\) רציף ב-\(x\). פונקציה היא גזירה ברציפות אם בכל נקודה היא גזירה ברציפות.

\end{definition}
נזכור כי העתקה לינארית היא רציפה אם"ם היא חסומה, לכן כדי להראות שפונקציה גזירה נדרש להראות שחסום בכל נקודה, כלומר נדרש שכל הנגזרות הכיווניות יהיו רציפות(אם נגזרת כיוונית כלשהי לא רציפה בנקודה אז לא חסומה)

\begin{corollary}
פונקציה היא גזירה ברציפות אם"ם כל הנגזרות הכיווניות שלה רציפות.

\end{corollary}
\section{פונקציות רב מימדיות}

\subsection{מסילות}

\begin{definition}[מסילה]
פונקציה מהצורה \(\gamma:[a,b]\to X\) כאשר \(X\) היא פונקציה רציפה נקראת מסילה.

\end{definition}
\begin{definition}[קשירה מסילתית]
עבור מסילה \(\gamma:[a,b]\to X\) תת קבוצה \(A\subseteq X\) היא קשירה מסילתית אם לכל \(x,y\in A\) קיים מסלול \(\gamma:[a,b]\to A\) כך ש-\(\gamma(a)=x,\gamma(b)=y\)

\end{definition}
\begin{theorem}[הערך הביניים]
אם \(A\subseteq X\) קשירה מסילתית ו-\(f:A\to\mathbb{R}\) רציפה, אז לכל \(x,y\in X\) ו-\(L\in[f(x),f(y)]\) קיים \(z\in A\) כך ש-\(f(z)=L\)

\end{theorem}
\begin{proof}
$$[a,b]\xrightarrow{\gamma}A\xrightarrow{f}\mathbb{R}$$
ונשים לב כי \(g=\gamma \circ f:[a,b]\to\mathbb{R}\) פונקציה רציפה כהרכבה של פונקציות רציפות. וכן מתקיים \(g(a)=f(x),g(b)=f(y)\) ופונקציה זו מקיימת את משפט ערך הביניים החד מימדי.

\end{proof}
\textbf{סימון - נגזרת של מסילה:}$$\left( D\gamma \right)_{t}:=\gamma^{\prime}(t)=\begin{pmatrix}\gamma_{1}'(t) \\\vdots \\\gamma_{m}'(t)
\end{pmatrix}$$

\begin{remark}
במקרה הספציפי של מסילה הנורמה האופרטורית מקיימת:
$$\|(D\gamma)_{t}\|_{\mathrm{op}_{1,m}}=\operatorname*{max}\{\|(D\gamma)_{t}(1)\|_{\mathbb{R}^{m}},\|(D\gamma)_{t}(-1)\|_{\mathbb{R}^{m}}\}=\|\gamma^{\prime}(t)\|_{\mathbb{R}^{m}}.$$

\end{remark}
\begin{lemma}
יהי \(\gamma:[0,1]\to\mathbb{R}^m\) מסילה גזירה כך שמתקיים:
$$\operatorname*{sup}_{0\leq t\leq1}\|(D\gamma)_{t}\|_{\mathrm{op}_{1,m}}=M$$
ולכן נקבל:
$$\|\gamma(1)-\gamma(0)\|_{\mathbb{R}^{m}}\leq M$$

\end{lemma}
הלמה הזאת היא המקרה הפרטי של המשפט הכללי הבא:
\textbf{משפט}
יהי \(A\subseteq \mathbb{R}^k\) קבוצה פתוחה. תהי \(f:A\to\mathbb{R}^m\) פונקציה גזירה, ו-\(a,b\) כך ש:
$$[a,b]=\{t b+(1-t)a\colon0\leq t\leq1\}\subset A$$
כמו כן נניח כי
$$\operatorname*{sup}_{c\in[a,b]}\|(D f)_{c}\|_{\mathrm{op}_{k,m}}=M.$$
אזי נקבל:
$$\|f(b)-f(a)\|_{\mathbb{R}^{m}}\leq M\,\|b-a\|_{\mathbb{R}^{k}}.$$

\subsection{פונקציות סקלריות}

כעת נתעסק עם פונקציות \(f:A\to \mathbb{R}\) כאשר \(A\subseteq \mathbb{R}^k\). פונקציות אלו חשובות במיוחד כיוון שניתן לפרק כל פונקציה \(f:\mathbb{R}^k\to\mathbb{R}^m\) לאוסף פונקציות סקלריות \(f_{i}:\mathbb{R}^k\to\mathbb{R}\).

\begin{definition}[גרדיאנט]
עבור פונקציה מהצורה \(f:\mathbb{R}^n\to\mathbb{R}\) נגדיר \(\bar{\nabla}f\in\mathbb{R}^m\) בתור הווקטור היחיד שמקיים לכל \(v\in \mathbb{R}^n\):
$$D f_{x}(v)=\langle\nabla f(x),v\rangle$$

\end{definition}
\begin{proposition}
ניתן להציג את הגרדיאנט בצורה וקטורית:
$$\nabla f\left(x\right)=\left(\begin{array}{c}{{\partial_{1}f\left(x\right)}}\\ {{\vdots}}\\ {{\partial_{k}f\left(x\right)}}\end{array}\right)=\left(D f\right)_{x}^{T}$$

\end{proposition}
נשים לב כי אם \(f:A\to \mathbb{R}^k\) דיפרנציאבלית אז \(x\to \bar{\nabla}f(x)\) היא פונקציה:
$$\bar{\nabla} f:A\to\mathbb{R}^k$$\textbf{פירוש גאומטרי:}
ננסה לחפש את הוקטור יחידה שעבורו הקצב שינוי המקסימלי. כלומר נחפש \(\left\lVert  \hat{v}  \right\rVert=1\) כך ש-\((Df)_{x}\left( \hat{v} \right)\) מקסימלי. כאשר ראינו שזה מתי ש-\(\left\langle  \bar{\nabla}f, \hat{v}  \right\rangle\) מקסימלי. מאי שיוויון קושי שוורץ נקבל כי:
$$\left\langle  \bar{\nabla} f, \hat{v}  \right\rangle \leq \left\lVert  \bar{\nabla} f  \right\rVert \left\lVert  \hat{v}  \right\rVert =\left\lVert  \bar{\nabla} f  \right\rVert $$
כאשר הערך המקסימלי, כלומר השיוויון מתקבל כאשר תלויים לינארים, כלומר \(\hat{v}=\frac{\bar{\nabla}f}{\left\lVert  \bar{\nabla}f  \right\rVert}\). ולכן הכיוון של הגרדינט הוא הכיוון של הגידול המקסימלי, וכן מתקיים \(\left\lVert  \bar{\nabla}f(a)  \right\rVert=\lVert Df_{a} \rVert_{op}\).

\begin{proposition}
יהי \(f:A\to\mathbb{R}\) פונקציה גזירה. נניח שב-\(a\) מתקבל ל-\(f\) מקסימום באיזושהי סביבה \(U\) של \(a\). הראו כי \(\bar{\nabla}f(a)=0\).

\end{proposition}
\begin{proof}
כיוון שמקבלת מקסימום, קיים סביבה \(A\) של \(a\) שעבורה לכל \(x \in A\) מתקיים \(f(x)\leq f(a)\). נסתכל על הנגזרת הכיוונית ה-\(i\). כיוון ש-\(f\) גזירה מתקיים מצד אחד:
$$\partial_{i} f|_{a}=\lim_{ h^+ \to 0 } \frac{f(a+he_{i})-f(a)}{h}\leq 0$$
כאשר מצד שני:
$$\partial_{i} f|_{a}=\lim_{ h^- \to 0 } \frac{f(a+he_{i})-f(a)}{h}\geq 0$$
ולכן נקבל \(\partial_{i}f|_{a}=0\) ולכן \(\bar{\nabla}f(a)=0\).

\end{proof}
\begin{theorem}[ערך הממוצע]
יהי \(A\subseteq \mathbb{R}^k\) קבוצה פתוחה, \(a,b\in A\) כך שהקטע המחבר ביניהם הוא:
$$[a,b]:=\{a+t(b-a)\::\:t\in[0,1]\}\subset A.$$
אם \(f:A\to\mathbb{R}\) גזירה, קיים \(\theta \in(0,1)\) כך ש:
$$f(b)-f(a)=(D f)_{a+\theta(b-a)}(b-a).$$

\end{theorem}
\begin{proof}
נרצה להשתמש במשפט ערך הממוצע החד מימדי. נסתכל על המסילה \(\gamma:[0,1]\to A\) הנתונה ע"י $$\gamma(t)=a+t(b-a),$$ ופנוקציה \(g:I\to\mathbb{R}\) הנתונה ע"י \(g(t)=f\left( \gamma(t) \right)\). הפונקציה \(g\) היא גזירה כך ש:
$$g^{\prime}(t)=(D f)_{\gamma(t)}\circ(D\gamma)_{t}=(D f)_{\gamma(t)}(b-a).$$
וממשפט ערך ההמוצע החד מימדי נקבל \(\theta \in(0,1)\) כך ש:
$$g(1)-g(0)=g^{\prime}(\theta)=(D f)_{a+\theta(b-a)}(b-a)$$

\end{proof}
זה למעשה אומר כי בין 2 נקודות על הגרף הקיים נקודה שהשיפוע שלו שווה לשיפוע הישר שמחבר בין הנקודות

\section{נגזרות מסדר גבוה}

נרצה להכליל את המושג של נגזרת לסדרים גבוההים יותר. ראשית נרצה למצוא מה מקבלים כשלקחים את הנגזרת של הנגזרת.
\textbf{הגדרה} נגזרת שנייה בנקודה
יהי \(f:A\to\mathbb{R}^m\) פונקציה גזירה כאשר \(A\subseteq \mathbb{R}^k\) קבוצה פתוחה. הפונקציה תקרא גזירה פעמיים בנקודה \(x\) אם קיים הטנסור \(D^2f(v):\mathrm{Hom}\left( \mathbb{R}^k,\mathrm{Hom}\left( \mathbb{R}^m,\mathbb{R}^m \right) \right)\) (יקרא הנגזרת השנייה) אשר מקיים:
$$ \lim\limits_{ v \to0}\frac{(Df)_{x+ v }-(Df)_x-(D^2f)_x\left(  v  \right)}{\| v \|_{\mathbb{R}^k}}=0_{\mathrm{Hom}\left( \mathbb{R}^k,\mathbb{R}^m \right)}$$

זהו בדיוק ההגדרה של הנגזרת עבור הפונקציית הנגזרת. נשים לב כי זהו גבול מטריציוני, כלומר נדרש שכל איבר במטריצה יתכנס לגבול. נשים לב כי הנגזרת השנייה מקבלת וקטור ומחזירה מטריצה, כאשר המטריצה הזאת תקבל וקטור ותחזיר וקטור אחר.

\begin{remark}
הנגזרת השנייה לעיתים מכונת הסיאן

\end{remark}
\begin{definition}[נגזרת חלקית מסדר שני]
$$\partial_{i}\partial_{j}\left(f\left(x\right)\right)=\lim_{ t \to 0 } \frac{\partial_{j}f\left(x+t e_{i}\right)-\partial_{j}f\left(x\right)}{t}$$

\end{definition}
זה למעשה מציאה של הנגזרת הכיווני של הפונקציה \(\partial_{j}f(x)\).

לכן בפועל נקבל כי:
$$\left(D^{2}f\right)\left(v\right)\left(w\right)=\sum_{i=1}^{k}\sum_{j=1}^{k}\partial_{i}\partial_{j}f\left(x\right)v_{i}w_{j}$$

\begin{proposition}[תנאי מספיק לדיפרנציביליות של הנגזרת השנייה]
תהי \(f:A\to\mathbb{R}^m\) כך ש-\(A\subseteq \mathbb{R}^k\) פתוחה. אם לכל \(j=1,\dots,k\) הפונקציה \(\partial_{j}f:A\to\mathbb{R}^m\) קיימת, ולכל \(i=1,\dots,k\) הנגזרת החלקית \(\partial_{i}\partial _j(f)\), אז \(f\) דיפרנציאבילית פעמיים ב-\(x\).

\end{proposition}
\begin{proposition}
הנגזרת השנייה היא תבנית בילינארית. כלומר מתקיים:
$$\begin{array}{c}{{\left(D^{2}f\right)_{x}\left(\alpha v_{1}+\beta v_{2}\right)\left(w\right)=\alpha\left(D^{2}f\right)_{x}\left(v_{1}\right)\left(w\right)+\beta\left(D^{2}f\right)_{x}\left(v_{2}\right)\left(w\right)}}\\ {{\left(D^{2}f\right)_{x}\left(v\right)\left(\alpha w_{1}+\beta w_{2}\right)=\alpha\left(D^{2}f\right)_{x}\left(v\right)\left(w_{1}\right)+\beta\left(D^{2}f\right)_{x}\left(v\right)\left(w_{2}\right)}}\end{array}$$

\end{proposition}
\begin{proof}
זוהי מסקנה מיידית מכך ש- \(v\mapsto (D^2f)_{x}(v)\) לינארי ו-\(w\mapsto (D^2f)_{x}(v)\) לינארי

\end{proof}
כיוון שזהו תבנית לינארית נסמן \((D^2f)_{x}(w,v)=(D^2f)_{x}(w)(v)\).

\textbf{משפט קלרו}
אם \(f_{xy}\) ו-\(f_{yx}\) רציפות אז \(f_{xy}=f_{yx}\).

נראה עם הוכחה יפה בעזרת פוביני
\textbf{הוכחה}
נניח כי קיים מלבן \(R=[a,b]\times[c,d]\) עבורה מוגדרת ורציפה \(f_{xy},f_{yx}\). כעת, כיוון שהפונקציות רציפות הם אינטגרביליות, ומתקיים:
\begin{gather*}\int_c^d \!\!\!\int_{a}^b f_{yx}(x,y)dxdy=\int_{c}^d \!\!\!\int_{a}^b(f_{y}(x,y))_{x}dxdy=\int_{c}^d f_{y}(b,y)-f_{y}(a,y)\;dy \\=f(b,d)-f(b,c)-f(a,d)+f(a,c)  
\end{gather*}
כאשר עבור \(f_{xy}\) נשתמש בפוביני(ניתן כיוון שרציפות) כדי להחליף את \(dxdy\) ל-\(dydx\) ונקבל:
\begin{gather*}\int_c^d \!\!\!\int_{a}^b f_{xy}(x,y)dxdy=\int_{c}^d \!\!\!\int_{a}^b(f_{x}(x,y))_{y}dxdy= \int_{a}^b \int_{c}^d (f_{x}(x,y))_{y} dy\;dx\\\int_{c}^d f_{x}(x,d)-f_{y}(x,c)\;dy=f(b,d)-f(b,c)-f(a,d)+f(a,c)
\end{gather*}
וכעת קיבלנו:
$$\int_c^d \!\!\!\int_{a}^b \!\!f_{yx}\;dxdy=\int_c^d \!\!\!\int_{a}^b \!\!f_{xy}\;dxdy\implies \int_c^d \!\!\!\int_{a}^b \!f_{yx}-f_{xy}\;dxdy=0$$
לכן מספיק להראות כי \(f_{yx}-f_{xy}\) זהותית אפס. נניח בשלילה שלא זהותית אפס, לכן קיים נקודה שאינה אפס, ומרציפות יש לנקודה זו סביבה שאינה אפס. כיוון שזה נכון עבור כל מלבן, זה נכון עבור מלבן שנמצא בסביבה הזו, ולכן עבור מלבן זה יהיה נפח שאינו אפס, בסתירה.

\begin{example}
נסתכל על הפונקציה:
$$f(x,y)=\begin{cases}\frac{xy(x^2-y^2)}{x^2+y^2} & (x,y)\neq(0,0) \\0 & (x,y)=(0,0)
\end{cases}$$
נראה \(f_{xy}\neq f_{yx}\). מתקיים ב-\((0,0)\):
$$f_{x}(0,0)=\lim_{ h \to 0 } \frac{f(h,0)-f(0,0)}{h}=\lim_{ h \to 0 } \frac{0-0}{h}=0$$$$f_{y}(0,0)=\lim_{ h \to 0 } \frac{f(0,h)-f(0,0)}{h}=0$$
כעת נחשב את הנגזרות השניות:
$$f_{xy}(0,0)=(f_{x})_{y}(0,0)=\lim_{ h \to 0 } \frac{f_{x}(0,h)-\cancelto{ 0 }{ f_{x}(0,0) }}{h}$$
נחשב לשם כך את \(f_{x}(0,h)\) בעזרת חוקי גזירה ונקבל:
$$f_{x}(x,y)= \frac{x^4y+4x^2y^3-y^5}{(x^2+y^2)^2}\implies f_{x}(0,h)=-\frac{h^5}{h^4}=-h$$
ולכן נקבל \(f_{xy}(0,0)=-1\). כעת נחשב את \(f_{yx}(0,0)\), ראשית נחשב את \(f_{y}(x,y)\) בנקודה שונה מ-\((0,0)\) ונקבל:
$$f_{y}(x,y)= \frac{x^5-4x^3y^2-xy^4}{(x^2+y^2)^2}\implies f(h,0)=h$$
ומזה ניתן לקבל:
$$f_{yx}(0,0)=(f_{y})_{x}(0,0)=\lim_{ h \to 0 } \frac{f_{y}(h,0)-\cancelto{ 0 }{ f_{y}(0,0) }}{h}=1\neq f_{xy}=-1$$

\end{example}
\begin{definition}[נגזרת מסדר n]
פונקציה \(f\) גזירה מסדר \(n\) אם היא גזירה מסדר \(n-1\) והנגזרת \(D^{n-1}f\) היא גזירה. $$(D^{n}f)_{x}\in\operatorname{Hom}(\mathbb{R}^{k},\operatorname{Hom}(\mathbb{R}^{k},\dots,\operatorname{Hom}(\mathbb{R}^{k},\mathbb{R}^{m}))),$$
זוהי תבנית מולטילינארית.

\end{definition}
\section{טיילור רב מימדי}

\begin{theorem}[טיילור]
תהי \(A\subseteq \mathbb{R}^k\) פתוחה, \(f:A\to\mathbb{R}^m\) דיפרנציאבילית \(p\) פעמים ברציפות ב-\(x \in A\). אזי, לכל \(v\) קטן מספיק כך ש-\(x+v\in A\) נוכל לכתוב:
$$\!\!\!\!f\left(x+v\right)\!=\!f\left(x\right)\!+\!\left(D f\right)_{x}\left(v\right)\!+\!\frac12\left(D^{2}f\right)_{x}\left(v,v\right),\ldots,\frac1{p!}\left(D^{(p)}f\right)_{x}\left(v,\ldots,v\right)\!+\!R_{p}\left(v\right)_{x}\left(v,\ldots,v\right)$$
כאשר \(R_{p}(v)\) היא פונקציית השארית המקיימת:
$$\operatorname*{lim}_{\nu\to0}{\frac{R_{p}(\nu)}{\|\nu\|_{\mathbb{R}^{k}}^{p}}}=0.$$

\end{theorem}
\begin{theorem}[שארית לגרנג']
אם הפונקציה \(f\) גזירה \(p+1\) פעמים, פונקציית השארית מקיימת 
$$R_{p}(\nu)=\frac{1}{(p+1)!}({D}^{p+1}f)_{x+\theta\nu}(\nu,\nu,\ldots,\nu)$$
עבור \(\theta \in(0,1)\) כלשהו

\end{theorem}
\section{סיווג נקודות קיצון}

\begin{definition}[חיובית בהחלט]
תבנית בילינארית סימטרית \(T\in \mathrm{Hom}\left( \mathbb{R}^k ,\mathrm{Hom}\left( \mathbb{R}^m , \mathbb{R}^m\right)\right)\) נקראת חיובית בהחלט אם:
$$\forall v \in \mathbb{R}^k\qquad T(v,v)>0$$

\end{definition}
\begin{definition}[חיובית למחצה]
תבנית בילינארית סימטרית \(T\in \mathrm{Hom}\left( \mathbb{R}^k ,\mathrm{Hom}\left( \mathbb{R}^m , \mathbb{R}^m\right)\right)\) נקראת חיובית למצה אם:
$$\forall v \in \mathbb{R}^k\qquad T(v,v)\geq0$$

\end{definition}
\begin{proposition}[תנאים שקולים לחיוביות בהחלט]
מטריצה \(S\) של תבנית בילינארית סימטרית תהיה חיובית בהחלט אם כל אחד מהתנאים הבאים מתקיים:

  \begin{enumerate}
    \item כל הערכים העצמיים חיוביים ממש - \(\lambda>0\)


    \item כל הערכים על האלכסון בצורה המדורגת חיוביים ממש 


    \item אם מסתכלים על כל ה-\(n\) מרובעים שמכילים את המספר הקודקוד הימיני העליון אז הדטרמיננטה של כולם חיוביים ממש 


    \item קיים \(A\) כך ש-\(S=A^TA\) כאשר ל-\(A\) יש עמודות בלתי תלויות(אם לא משתמשים בתנאי על העמודות נקבל שיכול להיות חיובי למחצה) 


    \item מתקיים \(x^T S x>0\) לכל \(x\) פרט ל-\(x=0\) (שוב התנאי האחרון בשביל המקרה שחיובי למחצה) 


  \end{enumerate}
\end{proposition}
\begin{remark}
תנאים 1,2,3 שקולים לתבנית חיובית למחצה אם מסתכלים גם על המקרה של אי שיוויון חלש. כלומר המקרה שהם חיוביים אך לא ממש.

\end{remark}
קיימים הגדרות ותנאי שקולים עבור המקרה של תבנית בילינארית סימטרית שלילית בהחלטת ושלילית למחצה.

\begin{definition}[נקודת מינימום]
עבור פונקציה \(f:A\to\mathbb{R}\) כאשר \(A\subseteq \mathbb{R}^k\) פתוחה, נקודה \(x \in A\) תהיה נקודת מינימום אם קיים קבוצה \(B\subseteq A\) כך שלכל \(b\in B\) מתקיים \(f(b)\geq f(x)\).

\end{definition}
\begin{definition}[נקודת מקסימום]
עבור פונקציה \(f:A\to\mathbb{R}\) כאשר \(A\subseteq \mathbb{R}^k\) פתוחה, נקודה \(x \in A\) תהיה נקודת מקסימום אם קיים קבוצה \(B\subseteq A\) כך שלכל \(b\in B\) מתקיים \(f(b)\leq f(x)\).

\end{definition}
\begin{definition}[נקודה קריטית]
נקודה שמאפסת את הנגזרת. כלומר נקודה \(x\) המקיימת \((Df)_{x}=0\).

\end{definition}
\begin{theorem}
יהי \(f:A\to \mathbb{R}\) כאשר \(A\subseteq \mathbb{R}^k\) פתוחה, וגזירה פעמיים ברציפות. תהי \(A\) נקודה קריטית, אזי אם \(D^2f\) רציפה ב-\(x\), מתקיים:

  \begin{enumerate}
    \item אם \((D^2f)_{x}\) חיובית בהחלט אז \(x\) נקודת מינימום. 


    \item אם \(x\) נקודת מינימום אז \((D^2f)_{x}\) חיובית למחצה. 


  \end{enumerate}
\end{theorem}
עבור מקסימום מקומי נקבל משפט זהה עבור תבנית בילינארית שלילית בהחלט ולמחצה.

עבור מטריצה \(2\times 2\) ניתן להשתמש בזה ש-\(\det A=\lambda_{1}\lambda_{2}\) ו-\(\mathrm{Tr}(A)=\lambda_{1}+\lambda_{2}\) כדי להסיק את הסימן של הערכים העצמיים ללא חישוב.

\textbf{תבנית כללית למציאת נקודות קיצון:}
עבור פונקציה \(f:\mathbb{R}^n\to\mathbb{R}\) נבצע את השלבים הבאים

\begin{enumerate}
  \item נחשב את הנגזרות החלקיות מסדר שני. אם לא כולם קיימות, הפונקציה לא גזירה ברציפות פעמיים ולא ניתן להמשיך בתהליך. 


  \item נוכיח שהנגזרות החלקיות מסדר שני רציפות, ונסיק כי \(f\) גזירה פעמיים ברציפות 


  \item לחפש נקודות קריטיות - נקודות שמאפסות את הנגזרת. 


  \item בכל אחד מהנקודות האלה, יש למצוא את הערכים העצמיים של ההסיאן 


  \item אם ישנו ערך עצמי 0, לא ניתן לסווג את הנקודה בתהליך זה. 


  \item אם כל הערכים העצמיים חיוביים, אז הנקודה היא מינימום, אם כולם שליליים אז מקסימום, אם ישנם ערכים עם סימן מעורב, אז הנקודה היא אוכף. 


\end{enumerate}
\section{כלל השרשרת}

\begin{proposition}[כלל השרשרת]
אם \(A\subseteq \mathbb{R}^k\) ו-\(B\subseteq \mathbb{R}^m\) תחומים פתוחים ו-\(f:A\to B,g:B\to\mathbb{R}^n\), כאשר \(f\) גזירה ב-\(x \in A\) ו-\(g\) גזיר ב-\(f(x)\in B\) אז \(g\circ f\) גזיר ב-\(x\) והנגזרת נתונה ע"י:
$$\underbrace{(D(g\circ f))_{x}}_{\mathbb{R}^{k}\to\mathbb{R}^{n}}=\underbrace{(D g)_{f(x)}}_{\mathbb{R}^{m}\to\mathbb{R}^{n}}\circ\underbrace{(D f)_{x}}_{\mathbb{R}^{k}\to\mathbb{R}^{m}}$$

\end{proposition}
\begin{proof}
נסתכל על הסתות קטנות \(u\in\mathbb{R}^n,v\in\mathbb{R}^m\). נגדיר את הפונקציות שגיאה של הקירוב הלינארי:
$$\begin{array}{l c r}{{r(u)=f(x+u)-f(x)-(D f)_{x}(u)}}\\ {{s( v )=g(f(x)+ v )-g(f(x))-(D g)_{f(x)}( v )}}\end{array}$$
כאשר אנו יודעים כי לפי הגדרת הנגזרת הפונקציות השגיאה שואפות ל-0, כאשר כיוון שזוהי מטריקה אוקלידית הם שואפות ל-0 איבר איבר ולכן גם הנורמה שואפת ל-0 ומתקיים:
$$\operatorname*{lim}_{u\to0}{\frac{\|r(u)\|_{\mathbb{R}^{m}}}{\|u\|_{\mathbb{R}^{k}}}}=0\qquad{\mathrm{~and~}}\qquad\operatorname*{lim}_{ v \to0}{\frac{\|s( v )\|_{\mathbb{R}^{n}}}{\left\| v \right\|_{\mathbb{R}^{m}}}}=0$$
כעת נציב \(v=(Df)_{x}(u)+r(u)\) בהגדרה של \(s\) ונקבל:
$$\begin{array}{r l}{s((D f)_{x}(u)+r(u))=g{\big(}f(x)+(D f)_{x}(u)+r(u){\big)}} {-g(f(x))} {-(D g)_{f(x)}{\big(}(D f)_{x}(u)+r(u){\big)}}\end{array}$$
כאשר נעביר אגפים ונקבל:
$$\begin{array}{r l}{g(f(x+u))-g(f(x))=(D g)_{f(x)}((D f)_{x}(u))} {+(D g)_{f(x)}(r(u))} {+s((D f)_{x}(u)+r(u))}\end{array}$$
נעביר לאגף אחד את כל הביטויים שתלויים ב-\(r(u)\) ונחלק את המשוואה ב-\(||u||_{\mathbb{R}^k}\) ונקבל:
$${{\frac{g\circ f(x+u)-g\circ f(x)-(D g)_{f(x)}\circ(D f)_{x}(u)}{\|u\|_{\mathbb{R}^{k}}}=(D g)_{f(x)}{\frac{r(u)}{\|u\|_{\mathbb{R}^{k}}}}}} {{+\,{\frac{s((D f)_{x}(u)+r(u))}{\|u\|_{\mathbb{R}^{k}}}}.}}$$
נשים לב כי באגף שמאל למעשה יש לנו את הגדרת הנגזרת. נדרש להראות שמתכנס ל-0 ונקבל את המבוקש. ניקח נורמות של שתי האגפים ונשתמש באי שיוויון המשולש:
$${\frac{\|g\circ f(x+u)-g\circ f(x)-(D g)_{f(x)}\circ(D f)_{x}(u)\|_{\mathbb{R}^{n}}}{\|u\|_{\mathbb{R}^{k}}}}\leq\|(D g)_{f(x)}\|_{m,n}{\frac{\|r(u)\|_{\mathbb{R}^{m}}}{\|u\|_{\mathbb{R}^{k}}}}+\,{\frac{\|s((D f)_{x}(u)+r(u))\|_{\mathbb{R}^{n}}}{\|u\|_{\mathbb{R}^{k}}}}.$$
מספיק להראות כי אגף ימין שואף ל-0 כאשר \(u\to_{0}\). כלומר מספיק להראות:
$$\operatorname*{lim}_{u\to0}\underbrace{\frac{\|s((D f)_{x}(u)+r(u))\|_{\mathbb{R}^{n}}}{\|u\|_{\mathbb{R}^{k}}}}_{\stackrel{\mathrm{def}}{=}\psi(u)}=0$$
כעת נכפיל ונחלק את \(\psi\) ונקבל:
$$\psi(u)= \begin{cases}\frac{\left\|s\left((D f)_x(u)+r(u)\right)\right\|_{\mathbb{R}^n}}{\left\|(D f)_x(u)+r(u)\right\|_{\mathbb{R}^m}} \frac{\left\|(D f)_x(u)+r(u)\right\|_{\mathbb{R}^m}}{\|u\|_{\mathbb{R}^k}} & (D f)_x(u)+r(u) \neq 0 \\ 0 & \text { otherwise. }\end{cases}$$
לפי אי שיוויון המשולש נקבל:
$$\psi(u) \leq \begin{cases}\frac{\left\|s\left((D f)_x(u)+r(u)\right)\right\|_{\mathbb{R}^n}}{\left\|(D f)_x(u)+r(u)\right\|_{\mathbb{R}^m}}\left(\left\|(D f)_x\right\|_{\mathrm{op}_{k, m}}+\frac{\|r(u)\|_{\mathbb{R}^m}}{\|u\|_{\mathbb{R}^k}}\right) & (D f)_x(u)+r(u) \neq 0 \\ 0 & \text { otherwise. }\end{cases}$$
כעת כאשר \(u\to 0\) נקבל \((Df)_{x}(u)+r(u)\to 0\) ולכן לפי ההגדרה של \(s\) הגורם הראשון מתאפס כאשר הגורם שני חסום, ולכן \(\lim_{ u \to 0 }\psi(u)=0\).

\end{proof}
\begin{example}
נסתכל על הפונקציות
$$f(x,y)=(x^{2}+y,\sin y,\cos y)\qquad\qquad g(r,s,t)=(e^{s+t},r)$$
ההרכבה תתן לנו העתקה \(g\circ f:\mathbb{R}^2\to\mathbb{R}^2\) הנתונה ע"י:
$$(g\circ f)(x,y)=g(x^{2}+y,\sin y,\cos y)=(e^{\sin y+\cos y},x^{2}+y)$$
ונקבל כי הדיפרנציאלים יתנו לנו:
$$D(g\circ f)_{(x,y)}=\left(\begin{array}{c c}{{0}}&{{(\cos y-\sin y)e^{\sin y+\cos y}}}\\ {{2x}}&{{1}}\end{array}\right)$$
כאשר עבור כל אחד מהפונקציות נקבל:
$$D g_{(s,r,t)}=\left(\begin{array}{c c c}{{0}}&{{e^{s+t}}}&{{e^{s+t}}}\\ {{1}}&{{0}}&{{0}}\end{array}\right)\qquad D f_{(x,y)}=\left(\begin{array}{c c}{{2x}}&{{1}}\\ {{0}}&{{\cos y}}\\ {{0}}&{{-\sin y}}\end{array}\right)$$
ואכן מתקיים:
$$\left(\begin{array}{c c c}{{0}}&{{e^{\sin y+\cos y}}}&{{e^{\sin y+\cos y}}}\\ {{1}}&{{0}}&{{0}}\end{array}\right)\left(\begin{array}{c c c}{{2x}}&{{1}}\\ {{0}}&{{\cos y}}\\ {{0}}&{{-\sin y}}\end{array}\right)=\left(\begin{array}{c c c}{{0}}&{{(\cos y-\sin y)e^{\sin y+\cos y}}}\\ {{2x}}&{{1}}\end{array}\right)$$

\end{example}
אם נסתכל על המטריצות המייצגות נקבל $$\left(D(g\circ f)_{a}\right)_{i j}=\sum_{l=1}^{m}\left(D g_{f(a)}\right)_{i l}\left(D f_{a}\right)_{l j}$$
כאשר האיברים של המטריצה המייצגת יהיו:
$$(\partial_{j}(g\circ f)_{i}(a))_{i j}=\left(\sum_{l=1}^{m}\partial_{l}g_{i}(f(a))\cdot\partial_{j}f_{l}(a)\right)$$
כאשר נשים לב כי אכן במקרה \(1\times 1\) נקבל:
$$(g\circ f)^{\prime}(a)=g^{\prime}(f(a))\cdot f^{\prime}(a)$$

\begin{example}
יהי \(\gamma:\mathbb{R}\to\mathbb{R}^k\) ו-\(g:\mathbb{R}^k\to\mathbb{R}\) נקבל \(h=\gamma \circ g:\mathbb{R}\to\mathbb{R}\) כלומר:
$$\mathbb{R}\ {\stackrel{\gamma}{\to}}\ \mathbb{R}^{m}\ {\stackrel{g}{\to}}\ \mathbb{R}$$
אם נכתוב \(\gamma=\left( \gamma_{1},\dots,\gamma_{m} \right)\) נקבל:
$$h=(g\circ\gamma)(t)=g(\gamma_{1}(t),...,\gamma_{m}(t))$$
כאשר אם נגזור לפי כלל השרשרת נקבל:
$$h^{\prime}(t)=\left( D g_{\gamma(t)} \right)\circ\left( D\gamma_{t} \right)(e_{1})=\sum_{l=1}^{m}\frac{\partial g}{\partial x_{l}}\left( \gamma(t) \right)\cdot\gamma_{l}^{\prime}(t)=\left\langle  \bar{\nabla} g(f(t)), \gamma'(t)  \right\rangle $$

\end{example}
\begin{proposition}[נגזרת של מכפלה פנימית]
בהנתן פונקציות גזירות \(f:\mathbb{R}^m\to \mathbb{R}^n\) ו-\(g:\mathbb{R}^k\to\mathbb{R}^n\), נקבל:
$$D(f,g)_{(\underline{{a}},\underline{{b}})}(\underline{{v}}_{1},\underline{{v}}_{2})=\langle g(\underline{{b}}),D f_{\underline{{a}}}(\underline{{v}}_{1})\rangle+\langle f({\underline{{a}}}),D g_{\underline{{b}}}(\underline{{v}}_{2})\rangle.$$

\end{proposition}
\begin{proof}
נגדיר פונקציה של המכפלה הפנימית הסטנדרטית \(\left\langle  \cdot, \cdot  \right\rangle:\mathbb{R}^{2n}\to\mathbb{R}\) המוגדרת:
$$(x_{1},...,x_{n},y_{1},...,y_{n})\mapsto\sum x_{i}y_{i}$$
נשים לב כי:
$${\frac{\partial\left( \sum x_{i}y_{i} \right)}{\partial x_{i}}}=y_{i} \qquad  {\frac{\partial(\sum x_{i}y_{i})}{\partial y_{i}}}=x_{i}$$
ולכן נקבל:
$$.D(\langle\cdot,\cdot\rangle)_{({\underline{{x}}},{\underline{{y}}})}=\left(y_{1},...,y_{n}\quad{\Big|}~x_{1}\cdot...,x_{n}~\right)=\left(~{\underline{{y}}}~{\Big|}~{\underline{{x}}}~\right)$$
כעת נניח  \(f:\mathbb{R}^m\to \mathbb{R}^n\) ו-\(g:\mathbb{R}^k\to\mathbb{R}^n\) פונקציות גזירות. ניתן להגדיר \(h:\mathbb{R}^m\times \mathbb{R}^k\to \mathbb{R}\) ע"י \(h\left( \underline{a},\underline{b} \right)=\left\langle  f\left( \underline{a} \right), g\left( \underline{b} \right)  \right\rangle\). ניתן לראות את \(h\) כהרכבה של \(\left\langle  \cdot, \cdot  \right\rangle\) על הפונקציה \(\left( f\times g \right):\mathbb{R}^m\times \mathbb{R}^k\to\mathbb{R}^{2n}\) המוגדרת \(\left( f\times g \right)\left( \underline{a},\underline{b} \right)=\left( f\left( \underline{a} \right),g\left( \underline{b} \right) \right)\).
כעת ניתן לחשב את הנגזרת של \(h\) בעזרת כלל השרשרת. נקבל:
$$D h_{(\underline{{{a}}},\underline{{{b}}})}(v)=D(\langle\cdot,\cdot\rangle)_{(f(\underline{{{a}}}),g(\underline{{{b}}}))}\circ D(f\times g)_{(\underline{{{a}}},\underline{{{b}}})}(v)$$
כאשר ניתן לחשב את הנגזרת של \(f\times g\) בעזרת מטריצת בלוקים:
$$ D(f\times g)_{(\underline{a},\underline{b})}(v)=\left(\begin{array}{c|c}Df_{\underline{a}}&0\\\hline0&Dg_{\underline{b}}\\\end{array}\right)\left(\frac{v_{1}}{\underline{v_{2}}}\right)=\left(\frac{Df_{\underline{a}}(\underline{v_{1}})}{Dg_{\underline{b}}(\underline{v_{2}})}\right)$$
ונקבל בסך הכל:
$$ D\langle f,g\rangle_{(\underline{a},\underline{b})}(\underline{v_1},\underline{v_2})=\left(\begin{array}{c|c}g(\underline{b})&f(\underline{a})\end{array}\right)\left(\frac{Df_{\underline{a}}(v_1)}{Dg_{\underline{b}}(\underline{v_2})}\right)=\langle g(\underline{b}),Df_{\underline{a}}(\underline{v_1})\rangle+\langle f(\underline{a}),Dg_{\underline{b}}(\underline{v_2})\rangle $$

\end{proof}
\begin{example}
כאשר \(m=k=n=1\) נקבל:
$$D(f g)_{(a,b)}(v_{1},v_{2})=g(b)\cdot D f_{a}(v_{1})+f(a)\cdot D g_{b}(v_{2})$$
כאשר אם נרכיב על המסילה \(\gamma:\mathbb{R}\to\mathbb{R}^2\) הנתונה ע"י \(t\mapsto(t,t)\) נקבל את הפונקציה:
$$p:\mathbb{R}\to\mathbb{R}\qquad p(t)=f(t)g(t)$$
אשר מקיימת:
$$p^{\prime}(t)=D p_{t}(1)=D(f g)_{(t,t)}\circ D\gamma_{t}(1)=D(f g)_{(t,t)}\left(\begin{array}{l}{{1}}\\ {{1}}\end{array}\right)=g(t)f^{\prime}(t)+f(t)g^{\prime}(t)$$
וקיבלנו את כלל המכפלה

\end{example}
\begin{proposition}[נגזרת של פונקציה הופכית]
תהי \(f:\mathbb{R}^k\to\mathbb{R}^k\) פונקציה הפיכה. יהי \(a\in \mathbb{R}^k\) כך ש-\(f\) גזירה ב-\(a\), ו-\(f^{-1}\) גזירה ב-\(b=f(a)\). מתקיים:
$$\left(D f\right)_{a}^{-1}=\left(D\left(f^{-1}\right)\right)_{b}$$

\end{proposition}
\begin{proof}
אנו יודעים כי \(f\circ f^{-1}=Id\) ולכן מכלל השרשרת נקבל:
$$(Df^{-1})_{f(a)}\circ Df_{a}=D(Id)_{a}=Id$$
כאשר השתמשנו בזה ש-\(D(Id)_{a}=Id\)(נובע ישירות מההגדרה) וכעת כיוון שזהו כפל מטריצות ניתן להכפיל במטריצה ההופכית ולקבל:
$$D(f^{-1})_{f(a)}=(D f_{a})^{-1}$$

\end{proof}
\section{משפט ההעתקה הפתוחה}

תזכורת: עבור \(A\in \mathrm{Hom}\left( \mathbb{R}^k,\mathbb{R}^m \right)\) מרחב השורות זה המרחב הנוצר מהבסיס כשלוקחים את השורות כבסיס, ומרחב העמודות זה המרחב הנוצר כשלוקחים את העמודות כבסיס. המימד של מרחב השורות ושל מרחב העמודות שווה, לכן:
$$\mathrm{rank}(A)\leq\operatorname*{min}(m,k).$$
מטריצה נקראת דרגה מלאה אם היא בגודל המקסימלי שיכולה להיות בהתאם לגודל שלה(כלומר באי שיוויון יש שיוויון). במקרה זה, היא תהיה על אם \(m\leq k\) וחח"ע אם \(m\geq k\).

\begin{lemma}
אם \(m\leq k\) ו-\(A\in\mathrm{Hom}\left( \mathbb{R}^k,\mathbb{R}^m \right)\) הוא מדרגה מלאה(כלומר במקרה שלנו, על) קיים מטריצה \(T\in\mathrm{Hom}\left( \mathbb{R}^m,\mathbb{R}^k \right)\) כך ש-\(AT\in\mathrm{Hom}\left( \mathbb{R}^m,\mathbb{R}^m \right)\) הפיכה.

\end{lemma}
\begin{proof}
כיוון ש-\(A\) מדרגה \(m\) אז הגרעין שלו ממימד \(k-m\). כלומר ל-\(\mathbb{R}^k\) יש תת מרחב \(m\) מימדי שאורתוגונאלי ל-\(\ker A\).
לסיים

\end{proof}
\begin{definition}[העתקה פתוחה]
יהיו \((X,d_{x}),(Y,d_{Y})\) מרחבים מטרים. העתקה \(f:X\to Y\) נקראת העתקה פתוחה
אם \(A\subseteq X\) פתוח ב-\(X\) גורר \(f(A)\) פתוח ב-\(Y\).

\end{definition}
\begin{remark}
נשים לה הגדרה זו דומה מאוד להגדרה של רציפות של פונקציה, רק שהגרירה הולכת לכיוון ההפוך לכן אם \(f:X\to Y\) הפיכה, ו-\(f^{-1}:Y\to X\) רציפה, אז \(f:X\to Y\) היא העתקה פתוחה.

\end{remark}
\begin{example}
$$f:\mathbb{R}\to \mathbb{R}^3\quad x\mapsto x^3\qquad  \text{ החותפ הקתעה}$$$$f:\mathbb{R}\to\mathbb{R}\quad x\mapsto x^2 \qquad \text{החותפ הקתעה אל}$$
כיוון ש-\(A=(-2,2)\) הולך ל-\(f(A)=[0,4)\).

\end{example}
\begin{theorem}[משפט ההעתקה הפתוחה]
יהי \(m\leq k\), \(A\subseteq \mathbb{R}^k\) קבוצה פתוחה, ו-\(f\in C^1\left( A;\mathbb{R}^m \right)\) כך ש-\((Df)_{x}\) מדרגה מלאה לכל \(x \in A\). אזי \(f\) העתקה פתוחה.

\end{theorem}
\section{משפט הפונקציה ההפוכה}

\begin{theorem}[הפונקציה ההפוכה]
תהי \(U\subseteq \mathbb{R}^n\) ו-\(f:U\to\mathbb{R}^n\) פונקציה גזירה ותהי \(a\in U\) כך ש-\(\det(Df(a))\neq 0\) אז:

  \begin{enumerate}
    \item קיימת סביבה \(V\subseteq U\) של \(a\) וסביבה \(W\) של \(f(a)\)


    \item קיימת פונקציה \(f:W\to V\) כך ש: 
$$\forall x \in V\quad g(f(x))=x\qquad \forall y\in W\quad f(g(y))=y$$


    \item הפונקציה \(g\) גזירה ומקיימת: 
$$Dg(f(a))=[Df(a)]^{-1}$$


  \end{enumerate}
\end{theorem}
\begin{example}
נתון הפונקציה \(f:\mathbb{R}^2\to\mathbb{R}^2\) המוגדרת 
$$f\begin{pmatrix}x_{1}\\x_{2}\end{pmatrix}=\begin{pmatrix}x_{1}+e^{x_{2}} \\x_{2}+e^{x_{1}}
\end{pmatrix}$$
נמצא האם קיים היפוך מקומי עבור כל נקודה ב-\(\mathbb{R}^2\)
ראשית נמצא את הנגזרת:
$$Df\begin{pmatrix}x_{1}\\ x_{2}\end{pmatrix}=\begin{pmatrix}1 & e^{x_{2}} \\e^{x_{1}} & 1
\end{pmatrix}\implies \det Df=1-e^{x_{1}+x_{2}}$$
כאשר הדטרמיננטה לא מתאפסת כל עוד \(x_{1}+x_{2}\neq 0\), ולכן במקרה בוודאות תהיה פונקציה הפוכה. ייתכן אבל שיש נקודות שעבורן \(x_{1}=-x_{2}\) וקיים היפוך מקומי.

\end{example}
\section{כופלי לגרנג'}

\begin{theorem}[כופלי לגרנג']
תהי \(B\subseteq \mathbb{R}^k\) פתוחה. \(n+1\leq k\). 
$$\begin{array}{c}{{f,g_{1},\ldots,g_{n}\in C^{1}{\bigl(}B;\mathbb{R}{\bigr)},}}\\ {{{}}}\\ {{A=\left\{x\in B:\ g_{1}{\bigl(}x{\bigr)}=\cdots=g_{n}{\bigl(}x{\bigr)}=0\right\}.}}\end{array}$$
נגדיר \(F:B\to\mathbb{R}^{n+1}\) ע"י:
$$F(x)=\left(\begin{array}{c}{{g_{1}\left(x\right)}}\\ {{\vdots}}\\ {{g_{n}\left(x\right)}}\\ {{f(x)}}\end{array}\right).$$
כעת, אם \(x_{0}\in A\) קיצון מקומי של \(f|_{A}\) אזי \((DF)_{x_{0}}\) הוא לא מדרגה מלאה(כלומר מדרגה קטנה מ-\(n+1\)). כלומר \((Dg_{1})_{x_{0}},\dots,(Dg_{n})_{x_{0}},(Df)_{x_{0}}\) הם תלויים לינארית, וקיימים \(\lambda_{1},\dots,\lambda_{n}\) כך ש:
$$\left(D f\right)_{x_{0}}=\sum_{i=1}^{n}\lambda_{i}\left(D g_{i}\right)_{x_{0}}$$

\end{theorem}
\begin{proof}
נניח בלי הגבלת הכלליות \(x_{0}\) היא נקודת מקסימום מקומית. כלומר קיימת סביבה \(U\subseteq B\) של \(x_{0}\) כך ש:
$$f(x_{0})=\max _{x \in A \cap U} f(x)$$
אנחנו רוצים להוכיח שהדרגה של \((DF(x))_{x_{0}}\) קטנה מ-\(n+1\). נניח בשלילה שמדרגה \(n+1\). לכן ממשפט ההעתקה הפתוחה קיים סביבה פתוחה \(U\subseteq B\) של \(x_{0}\) כך ש:
$$F(x_{0})=(0,\ldots,0,f(x_{0}))^{T}$$
תהיה נקודה פנימית של \(F(U)\) ב-\(\mathbb{R}^{n+1}\). במקרה זה קיים \(\varepsilon>0\) כך ש-\(\left( 0,\dots,0,f(x_{0})+\varepsilon \right)^T \in F(U)\). כלומר קיים \(x_{+}\in U\) כך ש-\(F(x_{+})=\left( 0,\dots,0,f(x_{0})+\varepsilon \right)^T\). מזה נקבל ש \(x_{+}\in U \cap A\) וגם \(f(x_{+})=f(x_{0})+\varepsilon\), בסתירה לכך ש-\(x_{0}\) מקסימום מקומי. 

\end{proof}
\begin{proposition}[כופלי לגרנג' בפונקציה סקלרית]
תהי \(B\subseteq \mathbb{R}^k\) קבוצה פתוחה, ויהיו \(f,g:B\to\mathbb{R}\) פונקציות גזירות ברציפות.
תהי \(A=\left\{  x \in B\mid g(x)=0  \right\}\) ונניח כי \(\bar{\nabla}g(x) \neq 0\) לכל \(x \in A\). אזי אם \(x\) מינימום או מקסימום מקומי של \(f|_{A}\) קיימת \(\lambda \in \mathbb{R}\) כך ש-\(\bar{\nabla}f(x)=\lambda \bar{\nabla}g(x)\).

\end{proposition}
\section{פונקציות קמורות}

\begin{definition}[הצירוף הקמור]
צירוף לינארי כך שסכום כל המקדמים יהיה 1.

\end{definition}
\begin{definition}[פונקציה קמורה]
פונקציה \(f:\mathbb{R}^{n}\to \mathbb{R}\) המקיימת לכל \(x,y \in \mathbb{R}^{n}\):
$$f((1-t)x+ty)\leq (1-t)f(x)+tf(y)$$
עבור \(0\leq t \leq 1\) כאשר \((1-t)f(x)+tf(y)\) יהיה הצירוף הקמור.

\end{definition}
\begin{proposition}
עבור פונקציה \(f:\mathbb{R}\to \mathbb{R}\) הביטויים הבאים שקולים:

  \begin{enumerate}
    \item הפונקציה \(f\) קמורה. 


    \item הנגזרת ראשונה מקיימת: 
$$f(y)\geq f(x)+f'(x)(y-x)$$
כלומר הפונקציה תהיה תמיד מעל המשיק המחבר בין שתי נקודות.


    \item הנגזרת השנייה מקיימת: 
$$f''(x)\geq 0$$


  \end{enumerate}
\end{proposition}
\begin{definition}[פונקציה קמורה ממש]
פונקציה \(f:\mathbb{R}^{n}\to \mathbb{R}\) אשר לכל \(x\neq y \in \mathbb{R}^{n}\) ו-\(0<\lambda<1\) מקיימת:
$$f{\big(}\lambda x+(1-\lambda)y{\big)}<\lambda f(x)+(1-\lambda)f(y)$$

\end{definition}
\begin{remark}
עבור פונקציה קמורה הערך של הפונקציה תמיד תהיה קטנה או שווה לערך של כל משיק, כאשר עבור פונקציה קמורה ממש הערך של הפונקציה תהיה קטנה ממש מהערך של המשיק בכל נקודה.

\end{remark}
\begin{definition}[קבוצה קמורה]
קבוצה \(C\subseteq \mathbb{R}^{n}\) נקראת קמורה אם לכל נקודות \(x,y \in {C}\) ו-\(0\leq \lambda \leq 1\) מתקיים:
$$\lambda x+(1-\lambda)y\in C$$
כלומר אם שתי נקודות הם בקבוצה נדרש כי גם הקו המחבר ביניהם יהיה בקבוצה.

\end{definition}
\begin{definition}[אפיגרף]
עבור פונקציה \(f:\mathbb{R}^{n}\to \mathbb{R}\) האפיגרף יהיה תת קבוצה של \(\mathbb{R}^{n}\times \mathbb{R}\) המכיל את כל הנקודות \((x,t)\) כך ש-\(t\geq f(x)\).

\end{definition}
\begin{proposition}
פונקציה היא קומרה אם"ם האפיגרף היא קבוצה קמורה.

\end{proposition}
\begin{reminder}
העתקה אפינית זוהי פונקציה מהצורה:
$$f(x)=a^{T}x+b$$

\end{reminder}
\begin{proposition}[תכונות של פונקציות קמורות]
  \begin{enumerate}
    \item אם הפונקיות \(f,g:\mathbb{R}^{n}\to \mathbb{R}\) הם קמורות אזי גם \(f+g\) תהיה קמורה. 


    \item אם \(f:\mathbb{R}^{n}\to \mathbb{R}\) קמורה ו-\(\lambda \geq 0\) אזי גם \(\lambda f\) תהיה קמורה. 


    \item כל פונקציה לינארית או אפינית תנינ קמורה. 


    \item אם גם \(f\) וגם \(-f\) פונקציות קמורות אזי \(f\) היא העתקה אפינית. 


    \item אם \(f,g\) היא פונקציה קמורה אזי \(h\) המוגדרת על ידי \(\max\{ f(x),g(x) \}\) היא גם תהיה קמורה. 


  \end{enumerate}
\end{proposition}
\section{גזירות}

\begin{proposition}
יהי \(f:\mathbb{R}^{n}\to \mathbb{R}\) פונקציה גזירה. אזי \(f\) היא קמורה אם"ם לכל \(x,y \in \mathbb{R}\) מתקיים:
$$f(y)\geq f(x)+\nabla f(x)^{T}(y-x)$$

\end{proposition}
\begin{proof}
נניח כי \(f\) היא קמורה. יהי \(x\neq y \in \mathbb{R}^{n}\). הקמירות של \(f\) נותנת לנו כי:
$$f{\big(}(x+y)/2{\big)}\leq{\frac{1}{2}}f(x)+{\frac{1}{2}}f(y).$$
כעת נסמן \(h\equiv y-x\). ניתן לכתוב את האי שיוויון בצורה הבאה:
$$f(x+h/2)\leq{\frac{1}{2}}f(x)+{\frac{1}{2}}f(x+h)$$
כאשר ניתן לפשט ולקבל:
$$f(x+h)-f(x)\geq{\frac{f(x+h/2)-f(x)}{1/2}}$$
כעת ניתן להשתמש בדרישת הקמירות במקום על \(x\) ו-\(y=x+h\) על \(x\) ו-\(x+\frac{h}{2}\) ולקבל:
$$f(x+h)-f(x)\geq{\frac{f(x+h/2)-f(x)}{1/2}}\geq{\frac{f(x+h/4)-f(x)}{1/4}},$$
או באופן כללי ניתן להמשיך באופן זה על חזקות של 2 ולקבל לכל \(k \in \mathbb{N}\):
$$f(x+h)-f(x)\geq{\frac{f(x+2^{-k}h)-f(x)}{2^{-k}}}$$
נזכור כעת את ההגדרה של הנגזרת הכיוונית בכיוון \(h\)$$\partial_{h} f\equiv \lim_{ t \to 0 } \frac{1}{t}(f(x+th)-f(x)) = \bar{\nabla} f(x)^{T}h$$
כעת אם ניקח את הגבול \(k\to \infty\) באי שייוון נקבל:
$$f(x+h)-f(x)\geq\operatorname*{lim}_{k\to\infty}{\frac{f(x+2^{-k}h)-f(x)}{2^{-k}}}=\partial_{h}f=\nabla f(x)^{T}h$$
כאשר אם נציב חזרה את \(h=y-x\) נקבל את האי שיוויון המבוקש. 
עבור הכיוון השני נניח כי מתקיים האי שיוויון לכל \(x, y \in \mathbb{R}^{n}\). בנוסף נניח \(w,z \in \mathbb{R}^{n}\) ו-\(0\leq \lambda \leq 1\) ונסמן:
$$x:= \lambda w+\left( 1-\lambda \right) z$$
כאשר כעת האי שיוויון נותן לנו:
$$\begin{array}{c}{(i)\quad {f(w)\geq f(x)+\nabla f(x)^{T}(w-x),}}\\ (ii)\quad \;{{f(z)\geq f(x)+\nabla f(x)^{T}(z-x).}}\end{array}$$
כמו כן נשים לב כי:
$$w-x=(1-\lambda)(w-z)\qquad{\mathrm{~and~}}\qquad z-x=\lambda(z-w).$$
ולכן אם נכפיל את \((i)\) ב-\(\lambda\) ואת \((ii)\) ב-\(\left( 1-\lambda \right)\) ונחבר נקבל:
$$f(y)>f(x)+\nabla f(x)^{T}(y-x)$$$$f(y)\geq f(x)+\nabla f(x)^{T}(y-x)=f(x)$$$$g(y):=f(y)-\nabla f(x)^{T}(y-x)$$$$\nabla g(y)=\nabla f(y)-\nabla f(x) \qquad \bar{\nabla}^2 g(y)=\bar{\nabla}^2 f(y)$$$$f(y)=f(x)+\nabla f(x)^{T}(y-x)+{\frac{1}{2}}(y-x)^{T}\nabla^{2}f(x+t(y-x))(y-x)$$$$f(y)\geq f(x)+\nabla f(x)^{T}(y-x)$$$$\left\{x\,\in\,\mathbb{R}\,:\,f^{\prime\prime}(x)\,>\,0\right\}$$

\end{proof}
\Chapter{אינטגרל של פונקציה מרובת משתנים}

\section{אינטגרציה על תיבות}

\begin{definition}[תיבה]
תיבה היא קבוצה מהצורה:
$$A=[a_{1},b_{1}]\times[a_{2},b_{2}]\times\dots\times[a_{k},b_{k}],$$
כאשר הנפח שלו מוגדר:
$$V(A)=\prod_{i=1}^{k}(b_{i}-a_{i}).$$
כאשר הרוחב של \(A\):
$$\max_{1\leq i\leq k}(b_{i}-a_{i})$$

\end{definition}
כיוון שאנו מתעסקים עם תיבות, נוח לעיתים להשתמש בנורמת האינסוף
$$||x||_{\infty}=\operatorname*{max}\left\{\left|x_{1}\right|,\ldots,\left|x_{k}\right|\right\}.$$
כיוון שכדור סגור בנורמה זו היא התיבה \([-1,1]\).

\begin{definition}[חלוקה]
תהא \(A=[a_{1},b_{1}]\times\dots \times[a_{k},b_{k}]\) תיבה. חלוקה \(P\) של \(A\) היא \(P=P_{1}\times P_{2}\times\dots \times P_{k}\)
כאשר \(P_{j}\) היא חלוקה של הקטע \([a_{j},b_{j}]\). כלומר:
$$P_{j}=\left\{  t_{j}^0,\dots,t_{j}^{m_{j}}  \right\}$$
כאשר $$a_{j}=t_{j}^{0}<t_{j}^{1}<\cdots<t_{j}^{m_{j}}=b_{j}.$$ חלוקה כזו משרה חלוקה של התיבה \(A\) לתיבות:
$$C_{i_{1},\dots,i_{k}}=[t_{1}^{i_{1}-1},t_{1}^{i_{1}}]\times\cdots\times[t_{k}^{i_{k}-1},t_{k}^{i_{k}}].$$
כאשר הנפח המקורי \(A\) מקיים:
$$V(A)=\sum_{i_{1},...,i_{k}}V(C_{i_{1},...,i_{k}}).$$

\end{definition}
\begin{definition}[סכום תחתון]
בהנתן פונקציה \(f\) וחלוקה \(P=A_{1}\times\dots \times A_{r}\). נגדיר \(m_{i}=\inf_{x \in A_{i}}f(x)\). הסכום התחתון של \(f\) ביחס ל-\(P\) יהיה:
$$s(f,P)=\sum_{i=1}^{r}m_{i}\,V(A_{i})$$

\end{definition}
\begin{definition}[סכום עליון]
בהנתן פונקציה \(f\) וחלוקה \(P=A_{1}\times\dots \times A_{r}\). נגדיר \(M_{i}=\sup_{x \in A_{i}}f(x)\). הסכום התחתון של \(f\) ביחס ל-\(P\) יהיה:
$$S(f,P)=\sum_{i=1}^{r}M_{i}\,V(A_{i})$$

\end{definition}
\begin{definition}[עידון של חלוקה]
חלוקה \(P'\) המקיימת \(P\subseteq P'\) נקראת עידון

\end{definition}
\begin{proposition}
אם \(P'\) עידון של חלוקה מתקיים:
$$s(f,P)\leq s(f,P^{\prime})\leq S\left(f,P^{\prime}\right)\leq S\left(f,P\right).$$

\end{proposition}
\begin{definition}[אינטגרל תחתון]
$$\underline{\int}_{A}f(x) \;\mathrm{d}x=\operatorname*{sup}\left\{s(f,P):\,a l l\,p a r t i t i o n s\,P\right\}.$$

\end{definition}
\begin{definition}[אינטגרל עליון]
$${\overline{{\int}_{A}}}f(x)\,\mathrm{d} x=\operatorname*{inf}\left\{S(f,P):\,a l l\,p a r t i t i o n s\,P\right\}.$$

\end{definition}
\begin{definition}[אינטגרבליות לפי רימן]
פונקציה שהאינטגרל העליון שווה לאינטגרל התחתון נקראת אינטגרבילית רימן, כלומר אם:
$$\underline{\int_{A}} f(x)\;\mathrm{d}x=\overline{\int}_{A}f(x)\mathrm{d}x$$
ובמקרה זה נקרא לביטוי זה האינטגרל ונסמן:
$$\int_{A}f(x)\,d x\quad \text{או}\quad\int_{A}f\left( x_{1},\ldots,x_{k} \right)\,d x_{1}d x_{2}\ldots d x_{k}$$

\end{definition}
\begin{remark}
פונקציה אינטגרבילית צריכה להיות חסומה, אחרת אין בכלל משמעות לסכום עליון ותחתון.

\end{remark}
\begin{proposition}[תכונות בסיסיות של אינטגרלים]
  \begin{enumerate}
    \item סכום של אינטגרלים - אם \(f,g\) אינטגרביליים אז \(f+g\)  אינטגרבילים, ומתקיים: 
$$\int_{A}(f+g)(x)\,d x=\int_{A}f(x)\,d x+\int_{A}g(x)\,d x.$$


    \item כפל בסקלר - אם \(f\) אינטגרבילי אז \(cf\) אינטגרבילי ומתקיים: 
$$\int_{A}(c f)(x)\,d x=c\int_{A}f(x)\,d x.$$


    \item אינטגרל על יחידה - פונקציית היחידה היא אינטגרבילית, ומקיימת: 
$$\int_{A}1\,d x=V(A).$$


    \item חיוביות - אם \(f(x)\geq 0\) לכל \(x \in A\) אז מתקיים: 
$$\int_{A}f(x)\,d x\geq0.$$


    \item מונוטוניות - אם \(f(x)\leq g(x)\) לכל \(x\) אז מתקיים: 
$$\int_{A}f(x)\,d x\leq\int_{A}g(x)\,d x.$$


    \item אם \(|f(x)|\leq M\) לכל \(x\) אז: 
$$\left|\int_{A}f(x)\,d x\right|\leq M\,V(A).$$


  \end{enumerate}
\end{proposition}
\begin{proposition}
יהיו \(A,B\subseteq \mathbb{R}^k\) תיבות כך ש-\(A^{\circ},B^{\circ}\) זרים ו-\(A\cup B\) תיבה. פונקציה \(f: A\cup B\to\mathbb{R}\) אינטגרבילית על \(A\) ועל \(B\) אם"ם אינטגרבילית על \(A\cup B\), ובמקרה זה מתקיים:
$$\int_{A\cup B}f\,d x=\int_{A}f\,d x+\int_{B}f\,d x.$$

\end{proposition}
\begin{proof}
נניח כי \(f\) אינטגרבילית ב-\(A\) וב-\(B\). יהיו \(P_{A},P_{B}\) חלוקות של \(A\) ושל \(B\). נניח... לסיים עמוד 139.

\end{proof}
\begin{proposition}
יהי \(A\subseteq \mathbb{R}^k\) תיבה, ו-\(f:A\to\mathbb{R}\) פונקציה רציפה, אזי \(f\) אינטגרבילית ב-\(A\).

\end{proposition}
\begin{proof}
כיוון ש-\(A\) קומפקטית, פונקציה \(f\) היא חסומה ורציפה במידה שווה ב-\(A\).
יהי \(\varepsilon>0\). מרציפות במידה שווה נקבל \(\delta>0\) כך ש:
$$\|x-y\|\leq\delta\implies|f(x)-f(y)|\leq{\frac{\varepsilon}{V(A)}}.$$
יהי \(P\) חלוקה של \(A\) כך שקוטר החלוקה של כל תת תיבא קטן מ-\(\delta\). לכן:
$$S\left(f,P\right)-s(f,P)=\sum_{i}(M_{i}-m_{i})\,V(A_{i})\leq \sum_{i}\delta V(A)=\varepsilon$$
וכעת נקבל:
$${\overline{{\int_{A}}}}f\,d x-\underline{\int}_{A}f\,d x\leq\varepsilon$$
ולכן אינטגרבילי.

\end{proof}
\section{מידה אפס}

\begin{definition}[מידה אפס]
קבוצה \(A\subseteq \mathbb{R}^k\) נקראת ממידה אפס אם לכל \(\varepsilon>0\) קיימים אוסף בן מנייה של תיבות \((B_{i})_{i\in\mathbb{N}}\)  כך ש-\(A\subseteq \bigcup_{i}B_{i}\) ו-\(\sum_{i}V(B_{i})<\varepsilon\).

\end{definition}
כלומר בצורה פשוטה, ניתן לכסות את הקבוצה ע"י תיבות, וניתן להקטין את התיבות כך שהנפח יקטן ככל שנרצה.

\begin{lemma}
איחוד סופי של קבוצות ממידה אפס הוא ממידה אפס

\end{lemma}
\begin{proof}
להשלים

\end{proof}
\begin{example}
  \begin{enumerate}
    \item יחידון(קבוצה עם איבר יחיד, נקודה) הא ממידה אפס, ולכן גם כל קבוצה בת מנייה. 


    \item קבוצה אם פנים שלא ריק הוא לא ממידה אפס, כיוון שאם \(a\in A^{\circ}\) אז קיים תיבה \(B\) כך ש- \(B\subseteq A\), ולכן מכסה את \(A\) ע"י קופסאות \((B_{i})_{i}\) שמקיימים \(\sum_{i}V(B_{i})\geq V(B)\).  


    \item הישר \(\mathbb{R}\times \{ 0 \}\subseteq \mathbb{R}^2\) הוא ממידה אפס, כיוון שמתקיים \(\mathbb{R}\times \{ 0 \}=\bigcup_{i\in\mathbb{Z}}[i,i+1]\times \{ 0 \}\) וכל אחד מהאיברים באיחוד הוא ממידה אפס כיוון שניתן לכסות אותו ע"י התיבה \([i,i+1]\times\left[ 0,\varepsilon \right]\) לכל \(\varepsilon>0\). 


    \item אם \(B\subseteq \mathbb{R}^k\), אז \(\partial B\) הוא ממידה אפס 


  \end{enumerate}
\end{example}
\begin{theorem}
יהי \((X,d)\) מרחב מטרי, אז \(A\subseteq X\) קומפקטי אם"ם לכל אוסף של קבוצות פתוחות \(\left\{  U_{\alpha}  \right\}_{\alpha \in I}\) כך ש-\(A\subseteq \bigcup_{\alpha \in I}U_{\alpha}\) קיים אוסף סופי של \(U_{\alpha_{1}},\dots,U_{\alpha_{m}}\) כך ש-\(A\subseteq \bigcup_{i=1}^mU_{\alpha_{i}}\).

\end{theorem}
אוסף כזה נקרא כיסוי פתוח.

\begin{theorem}
אם \(A\subseteq \mathbb{R}^k\) הוא קומפקטי ממידה אפס, אז לכל \(\varepsilon>0\) קיים כיסוי סופי של תיבות שיכסו את \(A\), וניתן לקחת אותם להיות תיבות פתוחות.

\end{theorem}
\begin{proof}
לכל \(\varepsilon>0\) קיימים תיבות \((B_{i})_{i\in\mathbb{N}}\) כך ש- \(A\subseteq \bigcup_{i}B_{i}\) ו-\(\sum_{i}V(B_{i})<\varepsilon /2\). לכל \(i\) ניקח תיבה \(B'\) כך ש- \(B_{i}\subseteq(B_{i}')^{\circ}\), ו- \(V(B'_{i})<2V(B_{i})\). כעת בשתמש במשפט על האוסף \(\left( (B_{i}')^{\circ} \right)_{i\in\mathbb{N}}\). 

\end{proof}
\begin{proposition}
  \begin{enumerate}
    \item יהי \(B_{1},\dots,B_{n}\) אוסף של תיבות. קיימים תיבות \(C_{1},\dots,C_{m}\) כך שהפנים שלהם זר, כך ש-\(\bigcup_{i=1}^nB_{i}=\bigcup_{j=1}^mC_{m}\). זה נכון לכל אוסף בר מנייה של תיבות. 


    \item כל תיבה \(B\subseteq \mathbb{R}^k\) ניתן לחלוקה לתיבות עם יחס בין הצלע הארוכה לקצרה של לכל היותר 2. 


    \item אם \(B \subseteq \mathbb{R}^k\) תיבה ככה שהיחס בין הצלע הארוכה לקצרה היא לכל היותר 2, אז קיים קובייה \(B\subseteq C\) כך ש-\(V(C)\leq 2^kV(B)\)


  \end{enumerate}
\end{proposition}
\begin{corollary}
אם \(A\subseteq \mathbb{R}^k\) ממידה אפס, אז לכל \(\varepsilon>0\) ניתן לכסות את \(A\) ע"י קוביות(מהצורה \([a,b]^k\)) \((C_{i})_{i\in\mathbb{N}}\), כך ש-\(\sum_{i}V(C_{i})<\varepsilon\). 

\end{corollary}
\begin{theorem}[לבג ויטאלי]
יהי \(A\subseteq \mathbb{R}^k\) תיבה כך ש-\(f:A\to\mathbb{R}\) פונקציה חסומה. \(f\) תהיה אינטגרבילית אם"ם נקודות האי רציפות הם ממידה אפס.

\end{theorem}
\begin{remark}
זה אומר כי אם קבוצה היא לא ממידה אפס, אז מכילה נקודות רציפות.

\end{remark}
\begin{proposition}
אם \(f_{1},\dots f_{m}:A\to\mathbb{R}\) פונקציות אינטגרביליות, ו-\(\phi:\mathbb{R}^m\to\mathbb{R}\) פונקציות רציפות, אז:
$$\int_{A}\phi\bigl(f_{1}(x),\ldots,f_{m}(x)\bigr)\,d x$$
כאשר בפרט מכפלה סופית של פונקציות אינטגרביליות היא אינטגרבילית

\end{proposition}
\begin{proof}
נקודות האי רציפות של \(\phi\) זה האיחוד של הנקודות האי רציפות של \(f_{i}\).

\end{proof}
\begin{proposition}
אם קבוצה היא קומפקטיות וממידה אפס אז הנפח שלה הוא אפס

\end{proposition}
\begin{definition}
יהי \((X,d),\left( Y,\rho \right)\) מרחבים מטרים. עבור \(f:X\to Y\) נגדיר 
$$\omega_{f}(x)=\lim_{ \delta \to 0 } \mathrm{diam}\left( f\left( B_{\delta}(x) \right) \right)\qquad F_{\varepsilon}=\left\{ x \in X \mid 
\omega_{f}(x)\geq \varepsilon  \right\}$$
כאשר נזכור כי:
$$\mathrm{diam(A)}=\sup_{y_{1},y_{2}\in A} \rho\left(y_{1},y_{2} \right)\quad A\subseteq Y$$

\end{definition}
נשים לב כי אם \(x\) נקודת רציפות:
\begin{gather*} \forall \varepsilon >0 \quad \exists \delta>0\quad \forall x'\in B_{\delta}(x)\quad f(x')\in B_{\varepsilon}(f(x))\implies  \\\forall\varepsilon>0 \quad \exists \delta>0\quad \forall x',x''\in B_{\delta}(x)\quad \rho(f(x'),f(x''))\implies \\
\omega_{f}(x)\leq \mathrm{diam}\left( f\left( B_{\delta}(x) \right) \right)<2\varepsilon\end{gather*}
כלומר \(x\) נקודת רציפות גורר \(\omega_{f}(x)=0\). באופן דומה, הגרירה ההפוכה גם נכונה, ולכן קבוצת נקודות האי רציפות יהיו שוות לקבוצה:
$$\left\{  x \in X \mid \omega_{f}(x)>0  \right\}=\bigcup_{n\in\mathbb{N}}\left\{  x \in X \mid \omega_{f}(x)\geq \frac{1}{n}  \right\}=\bigcup_{n \in \mathbb{N}}F_{\frac{1}{n}}$$

\begin{proposition}[ניסוח שקול למשפט לבג]
עבור \(A\subseteq \mathbb{R}^k\) קבוצה פתוחה, פונקציה \(f:A\to\mathbb{R}\) חסומה היא אינטגרבילית אם"ם \(F_{\frac{1}{n}}\) ממידה אפס לכל \(n\).

\end{proposition}
\section{נפח של קבוצות}

\begin{definition}[פונקציה מציינת]
מחזירה 1 עם הנקודה נמצאת בקבוצה \(S\) ו-0 אם לא. כלומר:
$$\chi_{S}\left(x\right)=\left\{{\begin{array}{l l}{1}&{\;x\in S}\\ {0}&{\;x\notin S.}\end{array}}\right.$$

\end{definition}
לפעמים מסמנים את הפונקציה ב- \(1_{S}\). 

\begin{definition}[נפח של קבוצה]
הנפח של קבוצה \(S\) תהיה:
$$V(S)=\int_{A}\!\chi_{S}(x)\,d x$$
אם האינטגרל הזה מוגדר הקבוצה נקראת בעלת נפח

\end{definition}
\begin{proposition}
קבוצה \(A\) בעלת נפח אם"ם \(\partial A\) ממידה אפס.

\end{proposition}
\begin{proof}
נזכור כי \(1_{A}\) אינטגרבילית אם"ם קבוצת נקודות האי רציפות הם ממידה אפס. נקודות האי רציפות של \(1_{A}\) הן בדיוק הערכים שבהן מחליפה בין 0 ל-1, כלומר השפה של \(A\). לכן \(A\) בעלת נפח אם"ם \(\partial A\) ממידה אפס.

\end{proof}
\begin{proposition}
תהי \(B\subseteq \mathbb{R}^k\) תיבה. אזי $$V(B)=\int_{B}1_{B}d x$$

\end{proposition}
\begin{proof}
יהי \(\varepsilon>0\). קיימת חלוקה \(P\) כך ש:
$$\underline{{{s}}}(1_{B},P)-\epsilon<\int\limits_{B}1_{B}d x<\overline{{{S}}}(1_{B},P)+\epsilon$$
מהגדרת הסכום העליון והתחתון מתקיים:
$$\begin{array}{l}{{\underline{{s}}(1_{B},P)=\displaystyle\sum_{j=1}^{r}\operatorname*{inf}_{x\in A_{j}}1_{B}(x)\cdot V(B_{j})=\displaystyle\sum_{j=1}^{r}V(B_{j})=V(B)}}\\ {{\overline{{S}}(1_{B},P)=\displaystyle\sum_{j=1}^{r}\operatorname*{sup}_{x\in A_{j}}1_{B}(x)\cdot V(B_{j})=\displaystyle\sum_{j=1}^{r}V(B_{j})=V(B)}}\end{array}$$
ולכן:
$$V(B)-\epsilon<\int_{B}1_{B}d x<V(B)+\epsilon$$
זה נכון לכל \(\varepsilon>\) ולכן \(V(B)=\int_{B}1_{B}\)

\end{proof}
\begin{proposition}
יהיו \(A,B,C\subseteq \mathbb{R}^k\) קבוצות בעלות נפח. אזי \(C\cap D, C \cup D\) בעלות נפח ומתקיים:
$$V\left(C\cup D\right)=V\left(C\right)+V\left(D\right)-V\left(C\cap D\right)$$

\end{proposition}
\begin{proposition}
תהי \(S\subseteq \mathbb{R}^k\) קבוצה קומפקטית, ו-\(f\in C\left( S;\mathbb{R} \right)\)(גזיר ברציפות ב-\(S\) ורציף ב-\(\mathbb{R}\)), אזי הגרף של \(f\):
$$\left\{\left(x,f\left(x\right)\right):x\in S\right\}\subset\mathbb{R}^{k+1}$$
הוא ממידה 0 ב-\(\mathbb{R}^{k+1}\).

\end{proposition}
\begin{proposition}
  \begin{enumerate}
    \item תהי \(f:\mathbb{R}^k\to\mathbb{R}^k\) פונקציה לינארית. אם \(N\subseteq \mathbb{R}^k\) ממידה 0 אזי \(f(N)\) ממידה אפס. 


    \item תהי \(f:\mathbb{R}^k\to\mathbb{R}^k\) פונקציה ליפשיצית. אם \(N\subseteq \mathbb{R}^k\) ממידה אפס, אזי \(f(N)\) ממידה אפס. 


    \item תהי \(f:\mathbb{R}^k\to\mathbb{R}^k\) פונקציה גזירה ברציפות. אם \(K\subseteq \mathbb{R}^n\) קבוצה קומפקטית ממידה אפס, \(f(K)\) תהיה ממידה אפס. 


  \end{enumerate}
\end{proposition}
\begin{proposition}
קבוצה \(A\subseteq \mathbb{R}^n\) בעלת נפח אם"ם \(A^\circ,\bar{A}\) בעלות נפח.

\end{proposition}
\begin{proposition}
קבוצה \(A\subseteq \mathbb{R}^n\) בעלת נפח ו-\(f:\bar{A}\to \mathbb{R}\) חסומה. אזי \(f\) אינטגרבילית על \(\bar{A}\) אם"ם \(f\) אינטגרבילית על \(A\) אם"ם \(f\) אינטגרבילית על \(A^\circ\) ומתקיים:
$$\int_{A^\circ }f(x)=\int_{A}f(x)=\int_{\bar{A}}f(x)$$

\end{proposition}
\begin{proposition}
תהי \(B\subseteq \mathbb{R}^k\) תיבה סגורה ו-\(\left\{  B_{\alpha}  \right\}_{\alpha=1}^\infty\)  אוסף בן מנייה של תיבות פתוחות כך ש-\(B\subseteq \bigcup_{\alpha=1}^\infty B_{\alpha}\) אז מתקיים:
$$V(B)\le\sum_{\alpha=1}^{\infty}V(B_{\alpha})$$

\end{proposition}
\begin{proposition}
קבוצה פתוחה חסומה לא בהכרח בעלת נפח. כלומר השפה שלה לא בהכרח ממידה אפס.

\end{proposition}
\begin{example}
קבוצת קנטור. תהי \(q_{n}\) סדרה של כל המספרים הרציונאלים. נגדיר:
$$U=\bigcup_{n=1}^\infty B\left( q_{n}, \frac{1}{2^n} \right)= \bigcup_{n=1}^\infty \left( q_{n}- \frac{1}{2^n},q_{n}+\frac{1}{2^n} \right)$$
נשים לב כי \(U\) פתוחה, וכמו כן \(U\) צפופה ב-\(\mathbb{R}\). וכן \(\mathbb{Q}\subseteq U\). מתקיים:
$$U=\bar{U}\setminus  U^\circ  = \mathbb{R} \setminus U= U^\circ =A $$
נניח בשלילה ש-\(A\) ממידה אפס. נשים לב כי:
$$[1,4]\subseteq(0,5)\cap\left( U\cup A \right) \subseteq\left( (0,5)\cap U \right)\cup \left( A\cap (0,5)\right)$$
נגדיר \(U'=(0,5)\cap U\) קבוצה פתוחה כחיתוך סופי של קבוצות פתוחות. כעת:
$$\partial U' \subseteq \left( A \cap(0,5) \right)\cup \{ 0,5 \}$$
כאשר הוספנו את הנקודות קצה כדי שיכיל את נקודות הגבול. וקיבלנו איחוד של קבוצות ממידה אפס, ולכן השפה ממידה אפס. ולכן \(U'\) בעלת נפח. כעת:
\begin{gather*} V\left( U'\cap A \right)\leq V\left( (0,5)\cap \bigcup_{n=1}^\infty\left( q_{n}-\frac{1}{2^n},q_{n}+\frac{1}{2^n}\cup A \right) \right)\leq  \\
\leq \sum_{n=1}^\infty V\left( \left( q_{n}-\frac{1}{2^n},q_{n}+\frac{1}{2^n} \right) \right) +V(A_{n}[0,5])=\sum_{n=1}^\infty \frac{2}{2^n}+0=2\end{gather*}
אבל \(V([1,4])=3\) והגענו לסתירה.

\end{example}
\begin{proposition}[תנאי שקול לאינטגרביליות פונקציות סקלאריות]
עבור \(A\subseteq \mathbb{R}^k\) פתוחה ופונקציה חיובית \(f:\left[ 0,\infty \right]\to A\) אם הביטוי:
$$\operatorname*{sup}\left\lbrace\int_{K}f:K\in U\quad {\text{בעלת נפח וקומפקטית}}\right\rbrace$$
סופי, אז הפוקציה \(f\) אינטגרבילית, והאינטגרל שווה לסופרמום. 

\end{proposition}
\begin{proposition}[סדרת מיצוי]
תהי \(A\subseteq \mathbb{R}^k\) קבוצה פתוחה, אזי יש סדרת קבוצות קומפקטיות בעלות נפח \(C_{N}\) כך ש-\(C_{N}\subset C^\circ_{N+1}\) ו-\(\sum_{N=1}^\infty C_{N}=A\). סדרה זו נקראת סדרת מיצוי.

\end{proposition}
\section{אינטגרלים לא אמיתיים}

\begin{definition}[אינטגרל לא אמיתי של פונקציה חיובית]
תהא \(A\subseteq \mathbb{R}^k\) קבוצה עם שפה ממידה אפס ותהא \(f:A\to\left[ 0,\infty \right]\) פונקציה שקבוצת נקודות האי רציפות שלה ממידה אפס. נאמר ש-\(f\) אינטגרבילית על \(A\) אם:
$$c=\operatorname*{sup}\left\lbrace\int_{K}f:K\in U\quad {\text{בעלת נפח וקומפקטית}}\right\rbrace $$
סופי. ובמקרה זה האינטגרל שווה ל-\(c\).

\end{definition}
\begin{proposition}
במקרה ש-\(A\) בעלת נפח ו-\(f\) חסומה, אז ההגדרה הזו שקולה ללהגדרה הרגילה.

\end{proposition}
\begin{proof}
כיוון ש-\(f\) אי שלילית, \(\int_{A}f(x)dx\geq c\). מצד שני, ראינו כי
$$\int_{A}f(x)dx=\lim_{ i \to \infty }\int_{S_{i}}f(x)\;\mathrm{d}x\leq c$$
כאשר \(S_{i}\subseteq A^\circ\) איחוד סופי של תיבות.

\end{proof}
\begin{definition}[אינטגרל לא אמיתי של פונקציה כללית]
עבור פונקציה \(f:\mathbb{R}\to A\) כללית, נגדיר \(f_{+}=\max\{ f,0 \}\) ו-\(f_{-}=\max\{ -f,0 \}\) כאשר הפונקציה אינטגרבילית אם"ם \(f_{-},f_{+}\) אינטגרביליות, והאינטגרל יהיה שווה ל:
$$\int_{A}f=\int_{A}f_{+}-\int_{A}f_{-}$$

\end{definition}
\begin{definition}[סדרת מיצוי]
נאמר כי \(A_{n}\subseteq A\) סדרה ממצה אם \(A_{1}\subseteq A_{2} \subseteq \dots \subseteq A\) ולכל קבוצה \(B\subseteq A\) קומפקטית, נקבל \(B \subseteq A_{n}\) החל ממקום מסוים. 

\end{definition}
\begin{proposition}
יהי \((A_{n})\) סדרה ממצה. אם \(f:A\to \mathbb{R}\) אינטגרבילי במובן אינטגרל לא אמיתי, אז:
$$\lim_{ n \to \infty } \int_{A_{n}}f(x)\;\mathrm{d}x=\int_{A} \;f(x)\;\mathrm{d}x$$

\end{proposition}
\begin{remark}
אם \(A_{n}\) בעלת נפח ו-\(f|_{A_{n}}\) חסומה, אז צד שמאל של המשוואה הוא אינטגרל אמיתי.

\end{remark}
\begin{proposition}
פונקציה \(f\) אינטגרבילית על תחום \(A\) אם"ם קיים מיצוי קומפקטי \(D_{n}\) של \(A\) כך ש-
$$\operatorname*{lim}_{n\to\infty}\int_{D_{n}}|f|=L\in\mathbb{R}$$

\end{proposition}
\section{משפט פוביני}

\textbf{אינטואיציה} אם מחלקים את התיבות לקוביות וסוכמים אותם, ניתן לסכום אותם לפי השורות או לפי העמודות, וכיוון שכמות התיבות היא סופית נצפה לקבל אותו דבר.

\begin{theorem}[פוביני]
יהיו \(A\subseteq \mathbb{R}^k,B\subseteq \mathbb{R}^m\) תיבות סגורות ו-\(f:A\times B\to\mathbb{R}\) פונקציה אינטגרבילית. לכל \(x  \in A\) ו-\(y\in B\) נגדיר:
$$f_{x}(y):=f(x,y),\;f_{y}(x):=f(x,y)$$
אם לכל \(x \in A,y \in B\) האינטגרלים$$\int_{B}f_{x}(y)d y,\;\int_{A}f_{y}(x)d x$$
קיימים אז:
$$\int_{A\times B}f(x,y)d x d y=\int_{A}\left( \int_{B}f_{x}(y)d y \right)d x=\int_{A}\left( \int_{B}f_{y}(x)d x \right)d y$$

\end{theorem}
זה מאפשר לנו לבצע אינטגרציה לכל משתנה בנפרד, ולהחליף סדר אינטגרציה.

\begin{proof}
כיוון ש-\(f\) אינטגרבילית ב-\(A\times B\) לכל \(\varepsilon>0\) קיימת חלוקה \(P\) כך ש-
$$S\left(f,P\right)-s(f,P)\leq\varepsilon.$$
כאשר כל חלוקה \(P\) של \(A\times B\) היא מכפלה של חלוקות \(P_{A},P_{B}\) של תיבות \(A\) ושל תיבות \(B\). נסמן:
$$G(y)=\overline{{{\int_{A}}}}f(x,y)\,d x\qquad{\mathrm{and}}\qquad H(y)=\underline{{{\int_{A}}}}f(x,y)\,d x.$$
כעת:
$$\begin{aligned} {{s ( f, P )}} & {{} {{} {} {} {{}=} \sum_{i} \sum_{j} V ( A_{i} ) V ( B_{j} ) \operatorname* {i n f}_{( x, y ) \in A_{i} \times B_{j}} f ( x, y )}} \\ {{}} & {{} {{} {{}=\sum_{j} V ( B_{j} ) \sum_{i} V ( A_{i} ) \operatorname* {i n f}_{( x, y ) \in A_{i} \times B_{j}} f ( x, y )}}} \\ {{}} & {{} {{} {{} \leq\sum_{j} V ( B_{j} ) \operatorname* {i n f}_{y \in B_{j}} \sum_{j} V ( A_{i} ) \operatorname* {i n f}_{x \in A_{i}} f ( x, y )}}} \\ {{}} & {{} {{} {{} \leq\sum_{j} V ( B_{j} ) \operatorname* {i n f}_{y \in B_{j}} H ( y )=s ( H, P_{B} ).}}} \\ \end{aligned}
$$
ובאופן דומה נקבל:
$$S\left(f,P\right)\geq S\left(G,P_{B}\right).$$
כעת נוכל להסיק כי:
$$\begin{aligned}\int_{A\times B}f(x,y) dxdy-\varepsilon & \leq s(f,P)\leq s(H,P_{B}) \\&\leq\underline{\int}_{B}H(y) dy\leq\overline{\int}_{B}H(y) dy \\&\leq S\left(H,P_{B}\right)\leq S\left(G,P_{B}\right)\leq S\left(f,P\right) \\&\leq\int_{A\times B}f(x,y) dxdy+\varepsilon.
\end{aligned}$$
כיוון שנכון לכל \(\varepsilon\) נקבל כי:
$$\int_{A\times B}f(x,y)\,d x d y=\int_{B}H(y)\,d y=\int_{B}{\underline{{\int_{A}}}}f(x,y)\,d x\,d y.$$
ובאופן דומה, נקבל:
$$\int_{A\times B}f(x,y)\,d x d y=\int_{B}G(y)\,d y=\int_{B}{\overline{{\int_{A}}}}f(x,y)\,d x\,d y.$$

\end{proof}
משפט זה מאוד חשוב כיוון שהופך אינטגרל על קבוצה לאוסף אינטגרלים חד מימדיים.

\begin{example}
ננסה למצוא נפח של גוף סיבוב. יהי \(f:[a,b]\to\left[ 0,\infty \right]\) פונקציה רציפה. וכן נגדיר:
$$C=\left\{(x,y,z)\in\mathbb{R}^{3}:\;x\in[a,b],\;y^{2}+z^{2}\leq f^{2}(x)\right\}.$$
נשים לב כי \(C\) קבוצה בעלת נפח. כעת מפוביני מתקיים:
$$V(C)=\int_{a}^{b}\left(\int\chi_{C}(x,y,z)\,d y d z\right)\,d x=\int_{a}^{b}\pi f^{2}(x)\,d x,$$
כיוון שהאינטגרל הפנימי הוא נשטח של דיסטה ברדיוס \(f(x)\). 

\end{example}
\begin{corollary}
יהיו \(g_{0},g_{1}:[a,b]\to \mathbb{R}\) פונקציות רציפות כך ש-\(g_{0}\leq g_{1}\) ותהי:
$$D=\left\{\left(x,y\right)\in\mathbb{R}^{2}:x\in\left[a,b\right],y\in\left[g_{0}\left(x\right),g_{1}\left(x\right)\right]\right\}$$
אזי \(D\) היא קבוצת בעלת נפח ולכל פונקציה רציפה \(f:D\to\mathbb{R}\) נקבל:
$$\int_{D}f\left(x,y\right)d x d y=\int_{a}^{b}\left(\int_{g_{0}\left(x\right)}^{g_{1}\left(x\right)}f\left(x,y\right)d y\right)d x$$

\end{corollary}
\begin{corollary}
יהיו \(h_{0},h_{1}:D\to\mathbb{R}\) פונקציות רציפות כך ש-\(h_{0}\leq h_{1}\) ותהי:
$$E=\left\{\left(x,y,z\right)\in\mathbb{R}^{2}:x\in\left[a,b\right],y\in\left[g_{0}\left(x\right),g_{1}\left(x\right)\right],z\in\left[h_{0}\left(x,y\right),h_{1}\left(x,y\right)\right]\right\}$$
אזי \(E\) קבוצת בעלת נפח ולכל פונקציה רציפה \(f:E\to\mathbb{R}\) נקבל:
$$\int_{E}f\left(x,y,z\right)d x d y d z=\int_{a}^{b}\left(\int_{g_{0}\left(x\right)}^{g_{1}\left(x\right)}\left(\int_{h_{0}\left(x,y\right)}^{h_{1}\left(x,y\right)}f\left(x,y,z\right)d z\right)d y\right)d x$$

\end{corollary}
\section{משפט חילוף משתנה}

\begin{theorem}[חלוף משתנה חד מימדי]
$$\int_{g(a)}^{g(b)} f=\int_{a}^{b}\left( f\circ g \right)g'$$

\end{theorem}
\begin{theorem}[חילוף משתנה רב מימדי]
תהא \(U\subseteq \mathbb{R}^k\) פתוחה, \(g\in C^1\left( U,\mathbb{R} \right)\) חח"ע, \(0\neq J(g(x))=\det(Dg)_{x}\) לכל \(x \in U\).
תהא \(\mathcal{U}\) פונקציית בעלת נפח וקומפקטית. אזי \(f:A\to\mathbb{R}\) אינטגרבילית כך ש-\(f\circ g\) אינטגרבילית. מתקיים:
$$\int_{A} f(x)dx=\int_{g^{-1}(A)}\left( f\circ g \right)(y)|Jg(y)|dy$$

\end{theorem}
\textbf{הערות}

\begin{enumerate}
  \item אם \(f\) רציפה, אז \(f\) אינטגרבילית, \(f\circ g\) אינטגרבלית 


  \item בפרט נובע אם \(A\) מבעלת נפח אז \(g^{-1}(A)\) בעלת נפח 


  \item ממשפט ההעתקה הפתוחה, \(g(U)=V\) היא פתוח. כלומר בעצם \(g:U\to V\) חח"ע ועל ו- 


\end{enumerate}
\begin{corollary}[חילוף משתנה בקבוצה פתוחה]
תהא \(U\subseteq \mathbb{R}^k\) פתוחה, \(g\in C^1\left( U,\mathbb{R}^k \right)\) חד-חד ערכית כך ש-\(Jg(x)\neq_{0}\) ב-\(U\). אזי לכל \(f:U\to \mathbb{R}\) אינטגרבילית כך ש-\(f\circ g\) אינטגרבילית, מתקיים:
$$\int_{U}f(x)d x=\int_{g^{-1}(U)}f\circ g|J g(y)|d y$$

\end{corollary}
\begin{example}
עבור הקבוצה \(E\) ונפונקציה \(f\) המוגדרות:
$$E=\left\{ (x,y)\quad|\quad0<x^{2}+x y+y^{2}<1 \right\}\qquad f(x,y)=\frac{1}{\sqrt{x^{2}+x y+y^{2}}}$$
נחשב את האינטגרל של \(f\) על התחום \(E\). ראשית נראה שאינטגרבילית.
נזכור כי \(f\) אינטגרבילית על \(A\) אם"ם קיים מיצוי קומפקטי \(D_{n}\) של \(A\) כך ש:
$$\operatorname*{lim}_{n\to\infty}\int_{D_{n}}|f|=L\in\mathbb{R}$$
הפונקציה \(f\) חיובית לכן אפשר להתעלם מהערך המוחלט. ניקח את המיצוי:
$$D_{n}=\left\{(x,y):\ {\textstyle{\frac{1}{n}}}\leq x^{2}+x y+y^{2}\leq1-{\textstyle{\frac{1}{n}}}\right\}$$
זהו למעשה מתאר את התחום בין אליספה קטנה לאליפסה גדולה. נרצה להמיר אותם בעיגולים ע"י חילופי משתנה לינארים. נשים לב כי ניתן לכתוב:
$$x^{2}+x y+y^{2}=\frac{3}{4}(x+y)^{2}+\frac{1}{4}(x-y)^{2}$$
ולכן ניתן לסמן \(u=x+y,v=x-y\) ונקבל \(x=\frac{1}{2}(u+v),y=\frac{1}{2}(u-v)\). למעשה נפטרנו כאן מהסיבוב. כעת ניתן להגדיר:
$$g(u,v)=\left( \frac{1}{2}(u+v),\frac{1}{2}(u-v) \right)\implies Dg=\begin{pmatrix}\frac{1}{2} & \frac{1}{2}\\ \frac{1}{2}  & -\frac{1}{2}
\end{pmatrix}$$
היעקוביאן שווה ל-\(\left\lvert  -\frac{1}{2}  \right\rvert=\frac{1}{2}\), ולכן האינטגרל יהיה:
$$\int_{D_{n}}f=\int_{\frac{1}{n}\leq\frac{3}{4}u^{2}+\frac{1}{4}v^{2}\leq1-\frac{1}{n}}\frac{1}{\sqrt{\frac{3}{4}u^{2}+\frac{1}{4}v^{2}}}\cdot\frac{1}{2}$$
נגדיר חילופי משתנה נוסף על מנת להפוך לטבעות עגולות:
$${{\displaystyle\frac{\sqrt{3}}{2}u=t,\;\frac{1}{2}v=s}}\quad  {{\displaystyle g(t,s)=\left( \frac{2}{\sqrt{3}}t,2s \right)}}\quad {{\displaystyle J g=\frac{2}{\sqrt{3}}\cdot2=\frac{4}{\sqrt{3}}}}$$

\end{example}
וכעת נקבל:
$$\int_{D_{n}}f={\frac{1}{2}}{\frac{4}{\sqrt{3}}}\int_{{\frac{1}{n}}\leq t^{2}+s^{2}\leq1-{\frac{1}{n}}}{\frac{1}{\sqrt{t^{2}+s^{2}}}}={\frac{2}{\sqrt{3}}}\int_{{\frac{1}{n}}\leq t^{2}+s^{2}\leq1-{\frac{1}{n}}}{\frac{1}{\sqrt{t^{2}+s^{2}}}}$$
נעשה חילוף משתנה אחרון לקורדינטות פולאריות. הקורדינטות הפולאריות מפספסות קבוצה ממידה אפס. ולכן האינטגרלים שווים. הפוקציה \(f\) רציפה בתחום ולכן ניתן להשתמש במשפט פוביני ולקבל:
$$=\frac{2}{\sqrt{3}}\int\limits_{\frac{1}{n}}^{1-\frac{1}{n}}\int\limits_{0}^{2\pi}\frac{1}{\sqrt{r^{2}}}r d\theta d r=\frac{4\pi}{\sqrt{3}}\left( 1-\frac{2}{n} \right)\implies\int_{E}f=\operatorname*{lim}_{n\to\infty}{\frac{4\pi}{\sqrt{3}}}\left( 1-{\frac{2}{n}} \right)={\frac{4\pi}{\sqrt{3}}}$$

\Chapter{אינטגרציה על מסילות}

\section{עקומות}

\begin{definition}[עקומה חלקה]
העתקה \(\gamma:[a,b]\to \mathbb{R}^{n}\) אשר גזירה מכל סדר.

\end{definition}
\begin{definition}[שקולים עד כדי פרמטריזציה]
שתי עקומות \(\mu:[c,d]\to \mathbb{R}^{n}\) ו-\(\gamma:[a,b]\to \mathbb{R}^{n}\) שקולים עד כדי פרמטריזציה אם קיים פונקצייה חלקה \(\phi:[c,d]\to[a,b]\) כך ש-\(\phi(c)=a\) ו-\(\phi(d)=b\) כאשר \(\phi'>0\) ומתקיים:
$$\mu=\gamma \circ  \phi$$

\end{definition}
\begin{remark}
התנאים על \(\phi\) הם שקולים לכך ש-\(\phi\) היא עולה, חח"ע ועל על מפה חלקה מ-\([c,d]\) ל-\([a,b]\) כך שההופכי הוא גם כן חלק.

\end{remark}
\begin{proposition}
אם \(\mu=\gamma \circ\phi\) כמו מקודם אז לכל \(t_{0} \in [c,d]\) מתקיים:
$$\mu^{\prime}(t_{0})=\gamma^{\prime}(\phi(t_{0}))\phi^{\prime}(t_{0})$$

\end{proposition}
\begin{proof}
זה מקרה מיוחד של כלל השרשרת.

\end{proof}
\end{document}