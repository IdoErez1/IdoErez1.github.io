\documentclass{tstextbook}

\usepackage{amsmath}
\usepackage{amssymb}
\usepackage{graphicx}
\usepackage{hyperref}
\usepackage{xcolor}

\begin{document}

\title{Example Document}
\author{HTML2LaTeX Converter}
\maketitle

\Chapter{יסודות}

\section{עקרונות הפיזיקה הקוונטית}

\begin{definition}[מערכת סגורה]
מערכת אשר מבודדת מהסביבה - מידע או אנרגיה לא יכול לצאת ממנה.

\end{definition}
\begin{definition}[האקסיומה הראשונה]
מצב קוונטי הוא קרן(ray) במרחב הילברט, וניתן לייצג אותו על ידי \(\ket{\psi} \in \mathcal{H}\).

\end{definition}
\begin{definition}[קרן]
לכל \(0\neq \lambda \in \mathbb{C}\) אוסף כל המצבים \(\lambda \ket{ \psi}\) נקרא קרן.

\end{definition}
\begin{reminder}[מרחב הילברט]
מרחב ווקטורי שלם עם מכפלה פנימית.
$$\braket{ \psi_{1} | \psi_{2} } =\overline{\braket{ \psi_{2} | \psi_{1} } } $$

\end{reminder}
\begin{definition}[נורמה של מצב קוונטי]
$$\left\lVert  \psi  \right\rVert =\sqrt{ \braket{ \psi | \psi }  }\geq 0 $$

\end{definition}
\begin{remark}
נבחר לנרמל את המצב:
$$\braket{ \psi | \psi } =1$$
אך גם נירמול אינו קובע את המצב ביחידות!
$$e^{ i\alpha }\ket{\psi} $$
הם שקולים פיזיקלית עבור כל \(\alpha \in \left[ 0,2\pi \right)\).

\end{remark}
\begin{definition}[האקסיומה השנייה]
גודל פיזקלי/גודל מדיד הם אופרטורים הרמיטים(צמודים לעצם) על \(\mathcal{H}\).

\end{definition}
\begin{corollary}
ניתן ללכסן אוניטרי כל גודל מדיד.

\end{corollary}
\begin{definition}[אופרטור הטלה]
אופרטור \(\Pi\) נקרא אופרטור הטלה אם:
$$\Pi ^{\dagger} = \Pi \qquad \Pi^{2}=\Pi$$
כאשר הספקטרום יהיה \(0,1\).

\end{definition}
\begin{definition}[האקסיומה השלישית]
תוצאות המדידה נקבעות בצורה הסתברותית על ידי כלל בורן(תלוי אינטרפרטציה - אנחנו משתמשים באינטרפטציית קופנהגן)

\end{definition}
\begin{proposition}[כלל בורן]
$$P(n)= \bra{\psi} \Pi_{n}\ket{\psi} =\left\lvert  \braket{ n | \psi }   \right\rvert ^{2}$$

\end{proposition}
לפי אינטרפטציית קופנהאגן אחרי המדידה המצב הקוונטי קורס:
$$\ket{\psi} \to \frac{\Pi_{n}\ket{\psi}}{\left\lVert  \Pi_{n}\ket{\psi}   \right\rVert } $$

\begin{corollary}
אם נרחב את מצב \(\ket{\psi}\) בסיס אורתונורמלי נקבל \(\ket{\psi}=\sum_{n} a_{n}\ket{n}\) נקבל:
$$p_{n}= \lvert a_{n} \rvert ^{2}\implies \sum_{n}p_{n}=1$$

\end{corollary}
באופן כללי ניתן לכתוב:
$$a_{n}=\sqrt{ p_{n} }e^{ i\phi_{n} }$$
אחרי מדידה של \(\ket{n}\) נקבל כי המצב הקוונטי יהיה:
$$\ket{n} \to e^{ i\phi_{n} }\ket{n} $$
כאשר הפאזה נהיית גלובאלית.

כאשר בנוסף:
$$\forall n\quad a_{n}\to e^{ i\theta }a_{n}$$
ונקבל:
$$p_{n}\to p_{n}$$
כלומר ההסתברות לא משתנה ו- \(\ket{\psi}\to e^{ i\theta }\ket{\psi}\) הם שקולים פיזיקלית

\begin{definition}[ערך תצפית]
$$\left\langle  Q  \right\rangle=\sum_{n} p_{n}q_{n} = \bra{\psi} \sum_{n}q_{n} \Pi_{n} \ket{\psi} = \bra{\psi} Q \ket{\psi}  $$

\end{definition}
\begin{corollary}
פאזה גלובאלית לא משנה את \(\langle Q \rangle\), וכן גם לא משנה את הפיזיקה של המערכת

\end{corollary}
\begin{definition}[האקסיומה הרביעית]
התקדמות בזמן נקבעת על ידי משוואת שרדינגר:
$$i\frac{\partial }{\partial t} \ket{\psi(t)} =H(t)\ket{\psi(t)} $$
כאשר ההתקדמות בזמן נקבעת על ידי אופרטור ההמילטוניאן

\end{definition}
עבור התקדמות אינפיניטסימלית:
$$\ket{\psi(t+dt)} =(1-iH(t)dt)\ket{\psi(t)} = e^{ -iH(t)dt }\ket{\psi(t)}=U(t+dt,t)\ket{\psi(t)}  $$
וההתקדות בזמן היא אינטרית

עבור זמן סופי \(T=Ndt\) כאשר \(N\to 0\) ו-\(dt \to 0\) נקבל:
$$U\left(t+T,t\right)=e^{-i H\left(t+\left(N-1\right)d t\right)d t}\cdot\cdot\cdot e^{-i H\left(t+d t\right)d t}e^{-i H\left(t\right)d t}$$
כאשר עבור המילטוניאן שתלוי בזמן נקבל:
$$U(t+T,t)=e^{ -iHT }$$

\begin{definition}[האקסיומה החמישית]
המרחב הילברט של מערכת מעורבת מכילה שתי מצבים קוונטים \(A,B\) עם \(\mathcal{H}_{A}\) ו-\(\mathcal{H}_{B}\) מהתאמה הוא המרחב מכפלה:
$$\mathcal{H}_{AB}=\mathcal{H}_{A}\times \mathcal{H}_{B}$$

\end{definition}
כאשר מתקיים:
$$\dim \mathcal{H}_{AB}=\dim \mathcal{H}_{A} \cdot \dim \mathcal{H}_{B}$$
והבסיס האורתונורמלי:
$$\left\{  \ket{ab} =\ket{a} _{A}\otimes \ket{b} _{B}  \right\}$$
כאשר \(\left\{  \ket{a}  \right\}_{A}\) ו-\(\ket{b}_{B}\) הם הבסיסים האורתונורמלים המתאימים.

\begin{summary}
במרחב סגור האסיומת של הפיזיקה הקוונטית יהיו:

  \begin{enumerate}
    \item כל מצב קוונטי הוא קרן במרחב הילברט \(\mathcal{H}\). 


    \item כל גודל מדיד הוא אופרטור צמוד לעצמו הפועל ב-\(\mathcal{H}\). 


    \item מדידה היא הטלה אורתוגונאלית. 


    \item קידום בזמן היא אופרטור אוניטרי. 


    \item מערכת אשר מורכבת משתי מערכות תהיה מיוצגת בעזרת מכפלה טנזורית של המערכות. 


  \end{enumerate}
\end{summary}
כאשר במערכת פתוחה אקסיומות \(1-4\) כבר לא יהיו נכונות.

\section{ספינים וסיבובים}

\begin{definition}[הקיו-ביט]
במערכת החישובית(computational basis)
$$\left\{  \ket{0}   ,\ket{1} \right\}$$
כאשר ניתן לתאר קיו-ביט כללית על ידי:
$$a\ket{0} +b\ket{1} =\ket{\psi} \qquad \lvert a \rvert ^{2}+\lvert b \rvert ^{2}=1$$

\end{definition}
\begin{corollary}
את דרישת הנורמול ניתן לבטא בעזרת שימוש בפונקציות טריגונומטריות וכן ניתן להשתמש בפאזה גלובאלית כדי לכתוב קיו-ביט בצורה הבאה:
$$\left|\psi\left(\theta,\varphi\right)\right\rangle=e^{-i\varphi/2}\cos\theta\left|0\right\rangle+e^{i\varphi/2}\sin\theta\left|1\right\rangle,\qquad\varphi \in\left[ 0,2\pi \right)\quad \theta\in\left[ 0,\pi \right]$$

\end{corollary}
\begin{corollary}
ניתן לבטא את הקיו ביט בתור נקודה על ספרת בלוך.

\includegraphics[width=0.8\textwidth]{diagrams/svg_1.svg}
\end{corollary}
\begin{proposition}
ניתן להציג מערכת שתי רמות על ידי מרחב ווקטורי דו מימדי מעל \(\mathbb{C}\). כאשר נסמן:
$$\ket{0} \leftrightarrow  \begin{pmatrix}1 \\ 0\end{pmatrix}\qquad \ket{1}  \leftrightarrow  \begin{pmatrix}0 \\ 1
\end{pmatrix}$$

\end{proposition}
\begin{definition}[מטריצות פאולי]
$$\begin{array}{c}{{\sigma_{1}=\sigma_{x}=\left(\begin{array}{c c}{{0}}&{{1}}\\ {{1}}&{{0}}\end{array}\right)}}\\ {{\sigma_{2}=\sigma_{y}=\left(\begin{array}{c c}{{0}}&{{-i}}\\ {{i}}&{{0}}\end{array}\right)}}\\ {{\sigma_{3}=\sigma_{z}=\left(\begin{array}{c c}{{1}}&{{0}}\\ {{0}}&{{-1}}\end{array}\right)}}\end{array}$$
כאשר כל אחת מהמטריצות הם אוניטריות והרמיטיות.

\end{definition}
\begin{proposition}
אופרטור כללי הפועל על מרחב הילברט ניתן להצגה על ידי מטריצה מרוכבות מגודל \(2\times 2\) מהצורה:
$$M=\left(\begin{array}{c c}{{m_{0}+m_{3}}}&{{m_{1}-i m_{2}}}\\ {{m_{1}+i m_{2}}}&{{m_{0}-m_{3}}}\end{array}\right)=m_{0}\mathbb{1}+\mathbf{m}\cdot\boldsymbol\sigma$$

\end{proposition}
\begin{proof}
המטריצה ההרמיטית הכללית ביותר:
$$M=\begin{pmatrix}a  & m_{1}-im_{2} \\m_{1}+im_{2} & b
\end{pmatrix}\qquad a,b,m_{1},m_{2} \in \mathbb{R}$$
אם נגדיר מחדש:
$$a=m_{0}+m_{3},\quad b=m_{0}-m_{3}$$
ניתן לקבל מערכת משוואות ולהגדיר מחדש את המטריצה כך שתראה מצורה הבאה:
$$M=\left(\begin{array}{c c}{{m_{0}+m_{3}}}&{{m_{1}-i m_{2}}}\\ {{m_{1}+i m_{2}}}&{{m_{0}-m_{3}}}\end{array}\right)=m_{0}\mathbb{1}+\mathbf{m}\cdot\boldsymbol{\sigma}$$

\end{proof}
\begin{proposition}[תכונות של מטריצות פאולי]
$$\sigma_{i}^{\dagger}=\sigma_{i}\qquad \sigma_{i}^{\dagger}\sigma_{i}=1\implies \sigma_{i}^{2}=1$$
ניתן להגיד שהאינטי קומוטטור יוצר אלגברת קליפורד(clifford algebra) כי מקיים:
$$\left\{\sigma_{i},\sigma_{j}\right\}=\sigma_{i}\sigma_{j}+\sigma_{j}\sigma_{i}=2\delta_{i j}\mathbb{1}$$
והקוממוטטור יוצר אלגברת לי כי מקיים:
$$[\sigma_{i},\sigma_{j}]=\sigma_{i}\sigma_{j}-\sigma_{j}\sigma_{i}=2i{\sum_{k}}\epsilon_{i j k}\sigma_{k}$$

\end{proposition}
\begin{definition}[מטריצת פאולי בכיוון כללי]
וכן לכל ווקטור יחידה \(\mathbf{n}\left( \theta,\varphi \right)\) על ספרת בלוך מתקיים:
$$\sigma_{\hat{n}}=\hat{n}\cdot \boldsymbol{\sigma}$$

\end{definition}
\begin{definition}[אופרטור הספין]
הספין של חלקיקים כמו אלקטרון נמדד על ידי אופרטור הספין, המוגדר על ידי:
$$S_{i}=\frac{1}{2}\sigma_{i}$$
כאשר ניתן לשייך את המרחב ההילברט של הקיו-ביט עם זה של ספין.
\begin{gather*}\ket{ 0} \leftrightarrow  \begin{pmatrix}1 \\ 0\end{pmatrix} \leftrightarrow  \ket{\uparrow _{z}} \equiv \ket{\uparrow }   \\\ket{ 0} \leftrightarrow  \begin{pmatrix}0 \\ 1\end{pmatrix} \leftrightarrow  \ket{\downarrow _{z}} \equiv \ket{\downarrow } 
\end{gather*}
זהו למעשה בחירה של בסיס.

\end{definition}
\begin{proposition}[תכונות של מטריצה פאולי כללית]
$$\forall \hat{n} \in S^{2}\quad \sigma_{\hat{n}}^{2}=1$$
ולכן ל-\(\sigma_{\hat{n}}\) יש ספקטרום של \(\pm 1\).
\begin{gather*}\sigma_{\hat{n}}\ket{\uparrow _{\hat{n}}} =\ket{\uparrow _{\hat{n}}} \qquad \sigma_{\hat{n}} \ket{\downarrow _{\hat{n}}} =-\ket{\downarrow _{\hat{n}}} 
\end{gather*}
כאשר עבור \(\hat{n}=\left( \theta,\varphi \right)\) נקבל 
$$\ket{\uparrow_{\hat{n}}}=\ket{\psi\left( \theta,\varphi \right)}=e^{ -i\varphi/2 }\cos\left( \theta \right)\ket{\uparrow}+e^{  i \varphi/2 }\sin\left( \theta \right)\ket{\downarrow } $$

\end{proposition}
\begin{proposition}[ספין x ו-y]
כאשר ניתן גם לכתוב את הערכים של ספין \(x\) ו-\(y\):
$$\begin{array}{c}{{\left|\uparrow_{x}\right\rangle=\frac{1}{\sqrt{2}}\left(\left|\uparrow\right\rangle+\left|\downarrow\right\rangle\right)}}\\ {{\left|\downarrow_{x}\right\rangle=\frac{1}{\sqrt{2}}\left(\left|\uparrow\right\rangle-\left|\downarrow\right\rangle\right)}}\\{{\left|\uparrow_{y}\right\rangle=\frac{1}{\sqrt{2}}\left(\left|\uparrow\right\rangle+i\left|\downarrow\right\rangle\right)}}\\ {{\left|\downarrow_{y}\right\rangle=\frac{1}{\sqrt{2}}\left(\left|\uparrow\right\rangle-i\left|\downarrow\right\rangle\right)}}\end{array}$$

\end{proposition}
\begin{example}[ספקטרומטר ראמזי]
נניח שיש לנו ספין יחיד, אשר מתואר על ידי ההמילטוניאן \(H=\omega \sigma_{z}\). ניתן לזה מצב התחלה:
$$\ket{\psi(0)} =\ket{\uparrow _{x}} = \frac{1}{\sqrt{ 2 }}\left( \ket{\uparrow } +\ket{\downarrow }  \right)$$
כאשר לאחר קידום בזמן נקבל:
$$\left|\psi\left(t\right)\right\rangle=e^{-i H t}\left|\psi\left(0\right)\right\rangle=\frac{1}{\sqrt{2}}\left(e^{-i\omega t}\left|\uparrow\right\rangle+e^{i\omega t}\left|\downarrow\right\rangle\right)$$
האם ניתן למדוד את \(\omega\)? הרעיון הוא לעבור לבסיס \(x\) ולקבל:
$$\left|\psi\left(t\right)\right\rangle=\cos\left(\omega t\right)\left|\uparrow_{x}\right\rangle+\sin\left(\omega t\right)\left|\downarrow_{x}\right\rangle$$
וכעת:
$$p_{\uparrow x}\left(t\right)=\cos^{2}\left(\omega t\right),\quad p_{\downarrow x}\left(t\right)=\sin^{2}\left(\omega t\right)$$
ולאחר חזרה על הניסוי מספר פעמים ניתן למצוא את \(\omega\) ולקבל את ההסתברות.

\end{example}
\begin{remark}
זה לא סותר את זה שאי אפשר לשכפל כיוון שהמשפט החוסר שיכפול אומר שאי אפשר לשכפל מצב קוונטי כללי, אבל זה לא אומר שלא קיים אופרטור אוניטרי אשר מעביר את המערכת למצב קוונטי ספציפי.

\end{remark}
\begin{proposition}[תכונות נוספות של מטריצות פאולי]
  \begin{enumerate}
    \item חסר עקבה - \(\mathrm{Tr}\left( \sigma_{i} \right)=0\). 


    \item מכפלה של מטריצות פאולי תהיה: 
$$\sigma_{i}\sigma_{j}=\delta_{i j}\mathbb{1}+i\sum_{k}\epsilon_{i j k}\sigma_{k}$$


    \item העקבה של המכפלה תהיה: 
$$\mathrm{Tr}\left[\sigma_{i}\sigma_{j}\right]=2\delta_{i j}$$


    \item הריבוע שלהם יחידה: 
$$\sigma_{\mathbf{\hat{n}}}^{2}=\left({\hat{\mathbf{n}}}\cdot \boldsymbol\sigma\right)^{2}=\mathbb{1}$$


    \item זהות אויילר עבור אקספוננט: 
$$e^{-i{\frac{\theta}{2}}{\hat{\mathbf{n}}}\cdot\boldsymbol{\sigma}}=\cos\left({\frac{\theta}{2}}\right)\mathbb{1}-i\sin\left({\frac{\theta}{2}}\right){\hat{\mathbf{n}}}\cdot\boldsymbol\sigma$$


  \end{enumerate}
\end{proposition}
\begin{proof}
  \begin{enumerate}
    \item נובע מיידית מהגדרות 


    \item בעזרת היחס קומוטטור והאנטי קומוטטור נקבל: 
$$\sigma_{i}\sigma_{j}={\frac{1}{2}}\left(\{\sigma_{i},\sigma_{j}\}+[\sigma_{i},\sigma_{j}]\right)={\frac{1}{2}}\left(2\delta_{i j}\mathbb{1}+2i{\underset{k}{\sum}}\epsilon_{i j k}\sigma_{k}\right)=\delta_{i j}\mathbb{1}+i{\underset{k}{\sum}}\epsilon_{i j k}\sigma_{k}$$


    \item מיידית מהשתי זהויות הקודמות 


    \item מהזהות של המכפלה נקבל: 
$$\sigma_{\mathbf{\hat{n}}}^{2}=\left({\hat{\mathbf{n}}}\cdot\boldsymbol\sigma\right)^{2}=\sum_{i j}n_{i}n_{j}\sigma_{i}\sigma_{j}=\sum_{i j}n_{i}n_{j}\delta_{i j}\mathbb{1}=\sum_{i}n_{i}n_{i}\mathbb{1}={\hat{\mathbf{n}}}\cdot{\hat{\mathbf{n}}}\mathbb{1}=\mathbb{1}$$


    \item מפתיחה כטור טיילור ושימוש בזהויות הקודמות נקבל: 
$$e^{-i{\frac{\theta}{2}}{\hat{\mathbf{n}}}\cdot\boldsymbol\sigma}=\sum_{n}\left({\frac{-i\theta}{2}}\right)^{2n}({\hat{\mathbf{n}}}\cdot\boldsymbol\sigma)^{2n}+\sum_{n}\left({\frac{-i\theta}{2}}\right)^{2n+1}({\hat{\mathbf{n}}}\cdot\boldsymbol\sigma)^{2n+1}=\cos\left({\frac{\theta}{2}}\right)-i\sin\left({\frac{\theta}{2}}\right){\hat{\mathbf{n}}}\cdot\boldsymbol\sigma$$


  \end{enumerate}
\end{proof}
\section{מטריצת צפיפות וצברים}

\begin{definition}[ערך תצפית של אופרטור]
נניח \(\mathcal{H}\) מרחב הילברט ממימד \(d\). נסמן בסיס אורתונורמלי \(\left\{  \ket{i}  \right\}_{i=1}^{d}\). יהי \(M\) ערך מדיד על \(\mathcal{H}\). נרצה למצוא את \(\langle M \rangle\) ביחס ל-\(\ket{\psi}\).
ראינו כי:
$$\langle M \rangle =\bra{\psi} M \ket{\psi} \impliedby \braket{ \psi | \psi } =1$$
כאשר מיחס השלמות:
$$\mathbb{1} =\sum_{i=1}^{d} \ket{i} \bra{i} $$

\end{definition}
\begin{definition}[עקבה של אופרטור]
$$\mathrm{Tr}(A)=\sum_{j}\left\langle\phi_{j}|A|\phi_{j}\right\rangle$$

\end{definition}
\begin{proposition}[תכונות של עקבה]
  \begin{enumerate}
    \item אם \(A=A^{\dagger}\) אז \(\mathrm{Tr}(A)\) הוא ממשי. 


    \item הומוגניות - \(\mathrm{Tr}(aA)=a\mathrm{Tr}(A)\). 


    \item אדטיביות - \(\mathrm{Tr}(A+B)=\mathrm{Tr}(A)+\mathrm{Tr}(B)\). 


    \item ציקליות - \(\mathrm{Tr}(AB)=\mathrm{Tr}(BA)\) כאשר במקרה של שלושה מכפלות \(\mathrm{Tr}(ABC)=\mathrm{Tr}(CAB)=\mathrm{Tr}(BAC)\). 


  \end{enumerate}
\end{proposition}
\begin{proposition}[ערך תצפית בעזרת עקבה]
\begin{gather*}\langle M \rangle =\bra{\psi} M\ket{\psi} =\sum_{i}\braket{ \psi | i } \bra{i} M\ket{\psi} =\sum_{i}\bra{i} M\ket{\psi} \braket{ \psi | i } = \\=\sum_{i}\bra{i} \left[ M\ket{\psi} \bra{\psi}  \right]\ket{i}  \equiv \mathrm{Tr}\left[ M\ket{\psi} \bra{\psi}  \right] 
\end{gather} $$

\end{proposition}
\begin{proposition}[קשר בין דטרמיננטה לעקבה]
$$\operatorname*{det}[\exp(A)]=\exp[\operatorname{Tr}(A)]$$

\end{proposition}
\begin{definition}[אופרטור הצפיפות]
לכל \(\ket{\psi}\in \mathcal{H}\) נגדיר:
$$\rho=\ket{\psi}\bra{\psi}   $$

\end{definition}
\begin{definition}[מצב טהור]
מצב במרחב הילברט אשר ניתן לבטא אותו בעזרת קאט.

\end{definition}
\begin{definition}[מצב מעורב]
מצב שאינו מצב טהור. כלומר מצב שלא ניתן לבטא אותו בעזרת קאט.

\end{definition}
נניח כי אנחנו יכולים לקבל כל אחד מהמצבים \(\left\{  \ket{\psi_{\mu}}  \right\}_{\mu=1}^{N}\) כאשר ניתן לקבל כל אחד עם הסתברות \(p_{\mu}\) כך ש-\(\sum_{\mu}p_{\mu} =1\) כאשר הסתברות זו היא הסתברות קלאסית - נובעת ממחסור במידע על המערכת. זהו צבר קלאסי.

\begin{proposition}
$$\langle M \rangle =\sum_{\mu=1}^{N}p_{\mu}\left\langle  \psi_{\mu}|M|\psi_{\mu}  \right\rangle =\sum_{\mu=1}^{N} p_{\mu}\mathrm{Tr}\left[ M\rho_{\mu} \right]=\mathrm{Tr}\left[ M\sum_{\mu}p_{\mu}\rho_{\mu} \right]$$

\end{proposition}
\begin{definition}[מטריצת צפיפות של צבר]
$$\rho = \sum_{\mu=1}p_{\mu}\rho_{\mu}=\sum_{\mu=1}^{N}p_{\mu}\ket{\psi_{\mu}}\bra{\psi_{\mu}}  $$
כאשר נשים לב כי כל עוד אין \(\mu\) כך ש-\(p_{\mu}=1\) נקבל כי זהו מצב מעורב.

\end{definition}
\begin{remark}
המטריצת הצפיפות של צבר זה הצירוף הקמור של מטריצות צפיפות של מצבים טהורים. זאת אומרת כי זהו צירוף לינארי של מצבים טהורים כך שסכום המקדמים יהיה 1.

\end{remark}
\begin{corollary}
$$\langle M \rangle = \mathrm{Tr}\left[ M\rho \right]$$

\end{corollary}
\begin{definition}[קוהרנטיות]
המידע שיש למערכת כתוצאה מהתאבכות. למעשה מתאר את ה-"קוונטיות" של המערכת. מיוצג על ידי האיברים במטריצת צפיפות אשר אינם על האלכסון. כאשר המטריצה צפיפות אלכסונית תהיה לא יהיה למערכת קוהרנטיות והמערכת תתנהג בפועל כמו צבר קלאסי(איבוד ה-"קווניטיות").

\end{definition}
\begin{definition}[דקוהרנטיות]
התהליך שבו נאבד קוהרנטיות למערכת. קורא באופן טבעי במערכות פתוחות בעקבות שזירה עם הסביבה.

\end{definition}
\section{מטריצת צפיפות של קיוביט}

\begin{proposition}
המטריצת צפיפות קיוביט צריכה לקיים:
- הרמיויות.
- עקבה 1.
- ערכים עצמיים ממשיים אי שליליים.

\end{proposition}
\begin{proposition}
מטריצת צפיפות של קיו-ביט צריכה להיות מהצורה:
$$\rho\left(\mathbf{P}\right)={\frac{1}{2}}\left(\mathbb{1}+\mathbf{P}\cdot\boldsymbol\sigma\right)$$
כאשר \(|\mathbf{P}|\leq 1\).

\end{proposition}
\begin{proof}
מטריצה הרמיטית כללית תהיה מהצורה:
$$M=m_{0}\mathbb{1} +\mathbf{m}\cdot \boldsymbol{\sigma}$$
אם נציב את הדרישה של העקבה נקבל:
$$\mathrm{Tr}\left( \rho\left( m_{0},\mathbf{m} \right) \right)=m_{0}\mathrm{Tr}\left[ \mathbb{1}  \right]=2m_{0}=1\implies m_{0}=\frac{1}{2}$$
כאשר השתמשנו בכך ש-\(\mathrm{Tr}\left[ \sigma_{i} \right]=0\). כעת נדרש רק לבדוק את דרישת הערכים העצמיים. נסתכל על הדטרמיננטה נזכור כי הדטרמיננטה זה מכפלת הערכים העצמיים, כאשר כיוון שהעקבה היא 1, מספיק לדרוש כי הדטרמיננטה חיובית. כלומר \(\det\left( \rho\left( \mathbf{P} \right) \right)\geq 0\). הדטרמיננטה תהיה:
$$\operatorname*{det}\rho\left(\mathbf{P}\right)={\frac{1}{4}}\operatorname*{det}\left(\begin{array}{c c}{{1+P_{3}}}&{{P_{1}-i P_{2}}}\\ {{P_{1}+i P_{2}}}&{{1-P_{3}}}\end{array}\right)={\frac{1}{4}}\left(1-\left|\mathbf{P}\right|^{2}\right)$$
ולכן המטריצת צפיפות האפשרית הקטנה ביותר תהיה מהצורה:
$$\rho(\mathbf{P})=\frac{1}{2}\Big(I+\mathbf{P}\cdot\boldsymbol{\sigma}\Big)=\frac{1}{2}\Bigg(\begin{array}{c c}{{1+P_{3}}}&{{P_{1}-i P_{2}}}\\ {{P_{1}+i P_{2}}}&{{1-P_{3}}}\end{array}\Bigg)$$
כאשר \(\left\lvert  \mathbf{P}  \right\rvert\leq 1\).

\end{proof}
\begin{corollary}
קבוצת כל המטריצות צפיפות של קיוביט יהיה כדור היחידה(כולל הפנים) ב-\(\mathbb{R}^{3}\). 

\end{corollary}
\begin{proposition}
מצבים טהורים נמצאים על ספרת בלוך, כלומר \(\left\lvert  \mathbf{P}  \right\rvert=1\) ויהיו עם מטריצת צפיפות מהצורה:
$$\rho\left(\mathbf{P}=\mathbf{\hat{n}}\left(\theta,\varphi\right)\right)={\frac{1}{2}}\left(\mathbb{1}+\mathbf{\hat{n}}\cdot\boldsymbol\sigma\right)$$
כאשר מצבים מעורבים יהיו בתוך ספרת בלוך ולא על השפה, כלומר \(\left\lvert  \mathbf{P}  \right\rvert< 1\).

\end{proposition}
\begin{proposition}
הערך המצופה של \(\hat{n}\cdot \sigma\) יהיה:
$$\left\langle  \boldsymbol\sigma_{\hat{n}}  \right\rangle= \left\langle{\hat{\mathbf{n}}}\cdot\boldsymbol\sigma\right\rangle={\hat{\mathbf{n}}}\cdot\mathbf{P}$$

\end{proposition}
\begin{proof}
הערך המצופה הכללי יהיה:
$$\langle\hat{\mathbf{n}}\cdot\boldsymbol\sigma\rangle={\frac{1}{2}}\sum_{i}\mathrm{Tr}\left[n_{i}\sigma_{i}\left(\mathbb{I}+\sum_{j}P_{j}\sigma_{j}\right)\right]={\frac{1}{2}}\left(\sum_{i}n_{i}\mathrm{Tr}\left[\sigma_{i}\right]+\sum_{i j}n_{i}P_{j}\mathrm{Tr}\left[\sigma_{i}\sigma_{j}\right]\right).$$
בעזרת הזהות \(\mathrm{Tr}\left[ \sigma_{i} \right]=0\) ו-\(\mathrm{Tr}\left[ \sigma_{i}\sigma_{j} \right]=2\delta_{ij}\) נקבל את המבוקש.

\end{proof}
\begin{remark}
הערך תצפית המקסימלי מתקבל כאשר \(\hat{n}=\hat{P}\) ובמקרה זה:
$$\left\langle{\hat{\mathbf{P}}}\cdot\boldsymbol\sigma\right\rangle={\hat{\mathbf{P}}}\cdot\mathbf{P}=|\mathbf{P}|$$

\end{remark}
נרצה להכליל. 

\begin{proposition}[אקסטריומליות המצבים הטהורים]
נניח \(\dim\mathcal{H}=d\) ו-\(\rho_{1},\rho_{2}\) הם שתי מטריצות צפיפות על \(\mathcal{H}\). נבחר \(0\leq c\leq 1\) ממשי כך ש:
$$\rho(c)=c\rho_{1}+(1-c)\rho_{2}$$
כאשר זוהי מטריצת צפיפות תקינה. המצבים הטהורים הם אקסטרימלים - כלומר הם בסוף בו בהתחלה של מצב \(\rho(c)\) כלשהו(\(c=0\) או \(c=1\)).

\end{proposition}
\begin{proof}
יהי \(\rho=\ket{\psi}\bra{\psi}\) מטריצת צפיפות של מצב טהור \(\ket{\psi}\). נניח בשלילה שקיימים \(0>c>1\) ומטריצות צפיפות \(\rho_{1},\rho_{2}\) כך ש-\(\rho=c\rho_{1}+(1-c)\rho_{2}\). יהי \(\ket{\phi}\) מצב שרירותי אורתוגונלי ל-\(\ket{\psi}\). מתקיים:
$$\bra{\phi} \rho \ket{\phi} =\braket{ \phi | \psi } \braket{ \psi | \phi } =0  $$
כאשר מתקיים:
$$0=\left\langle\phi\right|\rho\left|\phi\right\rangle=c\left\langle\phi\right|\rho_{1}\left|\phi\right\rangle+\left(1-c\right)\left\langle\phi\right|\rho_{2}\left|\phi\right\rangle$$
כאשר מתקיים:
$$\bra{\phi} \rho_{1} \ket{\phi} =0\qquad \bra{\phi} \rho_{2}\ket{\phi} =0$$
כיוון ש-\(\ket{\phi}\) הוא שרירותי נקבל כי לכל \(\ket{\phi}\perp \ket{\psi}\) ולכן:
$$\rho_{1}=\rho_{2}=\ket{\psi} \bra{\psi} =\rho$$
וקיבלנו סתירה.

\end{proof}
\begin{corollary}
  \begin{itemize}
    \item ניתן לייצר מצב מעורב מספר דרכים.
    \item ניתן לייצר מצב טהור רק בדרך אחת.
  \end{itemize}
\end{corollary}
\section{טיהור של מצב}

\begin{reminder}
כל מצב מעורב ניתן לכתיבה כצירוף קמור של מצבים טהורים.

\end{reminder}
\begin{proposition}
כל מטריצת צפיפות ניתנת לכתיבה בעזרת צירוף קמור של מצבים טהורים.

\end{proposition}
\begin{proposition}
עבור מטריצה מעורבת קיימים לפתוחות שתי דרכים כך שניתן לכתוב את המטריצה בעזרת במצבים טהורים. כלומר קיימות קבוצות של מצבים קוונטים \(\left\{\left|\phi_{i}^{(1)}\right\rangle\right\},\left\{\left|\phi_{i}^{(2)}\right\rangle\right\}\) כך ש:
$$\rho_{A}=\sum_{i}p_{i}^{(1)}\left|\phi_{i}^{(1)}\right\rangle\left\langle\phi_{i}^{(1)}\right|=\sum_{j}p_{j}^{(2)}\left|\phi_{j}^{(2)}\right\rangle\left\langle\phi_{j}^{(2)}\right|,$$
כאשר נשים לב כי אין שום סיבה שקבוצות אלו יהיו אורתונורמליות.

\end{proposition}
\begin{definition}[פיריפיקציה/טיהור]
עבור מצב מעורב המיוצג על ידי מטריצת צפיפות \(\rho_{A}\) במערכת \(\mathcal{H}_{A}\) מצב טהור המיוצג על ידי מטריצה \(\rho \in \mathcal{H}_{A}\times \mathcal{H}_{B}\) המתקבל על ידי הרחבת המערכת כאשר שארית המערכת היא עוברת טרייס אוט(כלומר \(\rho_{A}=\mathrm{Tr}_{B}\left( \rho \right)\)) תהיה הפיוריפיקציה של \(\rho_{A}\).

\end{definition}
\begin{theorem}[HJW/טיהור]
לכל מטריצת צפיפות קיים פיוריפיקציה.

\end{theorem}
\begin{example}
נסתכל על מטריצת צפיפות מהצורה:
$$\rho_{A}=\frac{1}{2}|0\rangle\langle0|+\frac{1}{2}|1\rangle\langle 1|$$
כאשר זהו מטריצת צפיפות מעורבת כיוון ש:
$$\mathrm{Tr}\left( \rho_{A}^{2} \right) = \frac{1}{4}+\frac{1}{4}< 1$$
כעת נגדיר מערכת \(B\) כך ש:
$$|\Psi\rangle=\frac{1}{\sqrt{2}}(|0\rangle_{A}\otimes|0\rangle_{B}+|1\rangle_{A}\otimes|1\rangle_{B})$$
כאשר המצב \(\rho=\ket{\Psi}\bra{\Psi}\) יהיה מצב טהור. מתקיים:
$$\rho_{A}=\mathrm{Tr}_{B}(|\Psi\rangle\langle\Psi|)=\frac{1}{2}|0\rangle\langle0|+\frac{1}{2}|1\rangle\langle1|$$

\end{example}
\begin{proposition}[מציאת פיוריפיקציה]
נכתוב את המטריצת צפיפות של המערכת על ידי הצירוף הקמור של מצבים טהורים:
$$\rho_{A}=\sum_{i}\lambda_{i}|\psi_{i}\rangle\langle\psi_{i}|,$$
כעת נכניס מרחב ביניים \(\mathcal{H}_{B}\) עם מימד של לכל היותר \(\dim \rho\) ובסיס אורתונורמלי \(\left\{  \ket{e_{i}}  \right\}\).
$$|\Psi\rangle=\sum_{i}\sqrt{\lambda_{i}}|\psi_{i}\rangle_{A}\otimes|e_{i}\rangle_{B}.$$
כאשר זהו מצב טהור במערכת החדשה כיוון שזה מבוטא על ידי קאט יחיד. כעת נשים לב כי אכן:
$$\mathrm{Tr}_{B}(|\Psi\rangle\langle\Psi|)=\sum_{i}\lambda_{i}|\psi_{i}\rangle\langle\psi_{i}|=\rho_{A}$$

\end{proposition}
\begin{proposition}
ערך תצפית של מטריצה פאולי כללית:
$$\left\langle  \sigma_{\hat{n}}  \right\rangle =\left\langle  \hat{n}\boldsymbol{\sigma}  \right\rangle =\mathrm{Tr}\left( \left( \hat{n}\boldsymbol{\sigma} \right)\rho \right)$$

\end{proposition}
\begin{proof}
עבור קיוביט אנחנו יודעים כי מטריצת הצפיפות תהיה:
$$\rho\left(\mathbf{P}\right)={\frac{1}{2}}\left(\mathbb{1}+\mathbf{P}\cdot\boldsymbol\sigma\right)$$
ולכן נקבל:
\begin{gather*}\left\langle  \hat{n}\boldsymbol{\sigma}  \right\rangle =\mathrm{Tr}\left( \left( \hat{n}\boldsymbol{\sigma} \right){\frac{1}{2}}\left(\mathbb{1}+\mathbf{P}\cdot\boldsymbol\sigma\right) \right)=\frac{1}{2}\mathrm{Tr}\left( \left( \sum_{i}n_{i}\sigma_{i} \right) \left( \mathbb{1} +\mathbf{P}\cdot \boldsymbol{\sigma} \right) \right)=\\=\frac{1}{2}\mathrm{Tr}\left( \sum_{i}n_{i}\sigma_{i} + \left( \sum_{i}n_{i}\sigma_{i} \right)\left( \sum_{j}P_{j}\sigma_{j} \right)\right)=\\=\frac{1}{2}\mathrm{Tr}\left( \sum_{i}n_{i}\sigma_{i} \right)+\frac{1}{2}\mathrm{Tr}\left( \sum_{i}n_{i}\sigma_{i}\sum_{j}P_{j}\sigma_{j} \right)=\\=\frac{1}{2}\sum_{i}n_{i}\mathrm{Tr}\left( \sigma_{i} \right)+\frac{1}{2}\sum_{i}n_{i}\mathrm{Tr}\left( \sigma_{i}\sum_{j}P_{j}\sigma_{j} \right)=\\=0+\frac{1}{2}\sum_{i}n_{i}\sum_{j}P_{j}\mathrm{Tr}\left( \sigma_{i}\sigma_{j} \right)=\frac{1}{2}\sum_{i}n_{i}P_{j}\cdot 2=\hat{n} \mathbf{P} 
\end{gather*}
כאשר השתמשנו בזה ש-\(\mathrm{Tr}\left( \sigma_{i}\sigma_{j} \right)=2\delta_{ij}\).

\end{proof}
\begin{example}[טיהור של אוסצילטור הרמוני]
נתונה מערכת של מתנד הרומוני פשוט \(A\) עם תדירות \(\omega\) המתוארת על ידי ההמילטוניאן \(H_{A}=\omega N_{A}\) כאשר \(N_{A}\) הוא אופרטור המספר כשמצבים העצמיים שלו הם כידוע \(N_{A}\ket{n}_{A}=n\ket{n}_{A}\). נמצא מתדנד הרמוני זהה \(B\) עם תדירות זהה כך ש-\(H_{B}=\omega N_{B}\). נניח כי \(B\) הוא אמבט תרמי(או סביבה) עבור מתנד \(A\).כיוון שנתון כי ניתן לתאר את המערכת כאמבט חום ניתן לכתוב:
$$\rho_{A}=Z^{-1} e^{ -\beta H_{A} } =Z^{-1}$$
כאשר:
$$Z=\mathrm{Tr}\left( e^{ -\beta H_{A} } \right)=\sum_{n=0}^{\infty} e^{ -\beta n\omega }= \frac{1}{1-e^{ -\beta \omega }}$$
ולכן כיוון שעבור אסוצילטור ניתן לכתוב \(H_{A}=\omega N_{A}\) נקבל:
$$\rho_{A}=Z^{-1}e^{-\beta H_{A}}=\left( 1-e^{ -\beta \omega } \right)e^{ -\beta \omega N_{A} }=\left( 1-e^{ -\beta \omega } \right)\sum_{n=0}^{\infty} e^{ -\beta \omega n }\ket{n} _{A}\bra{n} _{A}$$
כאשר ההסתברות להיות במצב \(\ket{n}_{A}\) יהיה (הריבוע של) המקדמי שמידט:
$$p_{n}=\left(1-e^{-\beta\omega}\right)e^{-\beta\omega n}$$
כדי שנשחזר את המצב \(A\) נדרש כי ההסתברות תהיה שווה. לכן אם נצמד למערכת זהה \(B\) נקבל:
$$|\psi\rangle_{A B}=\sum_{n=0}^{\infty}\sum_{n=0}^{\infty}{\sqrt{p_{n}}}\,|n\rangle_{A}\otimes|n\rangle_{B}={\sqrt{1-e^{-\beta\omega}}}\sum_{n=0}^{\infty}e^{-\beta\omega n/2}|n\rangle_{A}\otimes|n\rangle_{B}$$
כאשר אכן אם נחשב את האופרטור צפיפות הכולל נקבל:
\begin{gather*}\rho_{AB}=|\psi\rangle_{AB}\langle\psi|_{AB}=(\sum_{n=0}^{\infty}\sqrt{p_{n}}|n\rangle_{A}\otimes|n\rangle_{B})(\sum_{m=0}^{\infty}\sqrt{p_{m}}\langle m|_{A}\otimes\langle m|_{B}\rangle)=\\=\sum_{n=0}^{\infty}\sum_{m=0}^{\infty}\sqrt{(p_{n}p_{m})}|n\rangle_{A}\langle m|_{A}\otimes|n\rangle_{B}\langle m|_{B} 
\end{gather*}
וכעת אם נחשב את העקבה החלקית נקבל:
$$\rho_{A}=\mathrm{Tr}_{B}(\rho_{A B})=\sum_{n=0}^{\infty}\sum_{m=0}^{\infty}\sqrt{(p_{n}p_{m})}|n\rangle_{A}\langle m|_{A}\mathrm{Tr}(|n\rangle_{B}\langle m|_{B})$$
כאשר אנו יודעים כי \(\mathrm{Tr}\left( \ket{n}_{B}\bra{n}_{A} \right)=\delta_{nm}\) ולכן:
$$\rho_{A}=\sum_{n=0}^{\infty}\sum_{m=0}^{\infty}\sqrt{(p_{n}p_{m})}|n\rangle_{A}\langle m|_{A}\delta_{n m}=\sum_{n=0}^{\infty}p_{n}|n\rangle_{A}\langle n|_{A}=\rho_{A}$$

\end{example}
\section{שזירה}

\begin{definition}[מערכת שתי מערכות - Bipartite Hilbert Space]
מרחב מכפלה של שתי מצבים:
$$\mathcal{H}=\mathcal{H}_{A}\times \mathcal{H}_{B}$$
עם בסיס אורתונורמלי \(\{ \ket{ij} \}\). המצב הטהור הכללי ב-\(\mathcal{H}\) יהיה:
$$\ket{\psi} = \sum_{ij} \psi_{ij} \ket{i} _{A}\ket{j} _{B}$$
כאשר נסמן ב-\(\psi_{ij}\) את המטריצת אמפליטודות \(d_{A}\times d_{B}\).

\end{definition}
\begin{definition}[פירוק ערכים סינגולאריים \(SVD\)]
עבור מטריצה \(\psi_{ij}\) מגודל \(d_{A}\times d_{B}\) קיימת מטריצה אינוטרית \(U_{A}\) מגודל \(d_{A}\times d_{A}\) כאשר \(U_{B}\) היא מגודל \(d_{B}\times d_{B}\) ומטריצה מלבנית \(\Lambda\left( d_{A}\times d_{B} \right)\) כך ש:
$$\psi=U_{A}^{T}\Lambda U_{B}$$
כאשר נניח בלי הגבלת הכלליות כי \(d_{A}\geq d_{B}\equiv d\). כאשר מטריצת הערכים הסינגולרית \(\Lambda\) תהיה מהצורה:
$$\Lambda=\left(\begin{array}{c c c c}{{\lambda_{1}}}&{{0}}&{{\cdots}}&{{0}}\\ {{0}}&{{\ddots}}&&{{\vdots}}\\ {{\vdots}}&{{}}&{{\ddots}}&{{0}}\\ {{0}}&{{\cdots}}&{{0}}&{{\lambda_{d}}}\\ {{0}}&{{\cdots}}&{{\cdots}}&{{0}}\\ {{\vdots}}&{{}}&{{}}&{{\vdots}}\\ {{0}}&{{\cdots}}&{{\cdots}}&{{0}}\end{array}\right)$$

\end{definition}
\begin{proposition}
לכל מטריצה קיים פירוק ערכים סינוגלארים.

\end{proposition}
\begin{proposition}
בעזרת פירוק ערכים סינגולאריים ניתן לכתוב
$$\left|\psi\right\rangle=\sum_{i j}\sum_{\mu}\left(U_{A}^{T}\right)_{i\mu}\lambda_{\mu}\left(U_{B}\right)_{\mu j}\left|i\right\rangle_{A}\otimes\left|j\right\rangle_{B}\equiv\sum_{\mu}\lambda_{\mu}\left|\mu\right\rangle_{A}\otimes\left|\mu\right\rangle_{B}$$
כאשר הגדרנו בסיס אורתונורמלי חדש(מטריצת מעבר בסיס אוניטרית מעבירה בסיס אוניטרי לבסיס אוניטרי) בצורה הבאה:
$$|\mu\rangle_{A}=\sum_{i}\left(U_{A}\right)_{\mu i}|i\rangle_{A}\quad,|\mu\rangle_{B}=\sum_{j}\left(U_{B}\right)_{\mu j}|j\rangle_{B}$$

\end{proposition}
\begin{definition}[צורת שמידט]
$$|\psi\rangle=\sum_{\mu=1}^{d}\lambda_{\mu}\,|\mu\rangle_{A}\otimes|\mu\rangle_{B}$$
כאשר \(\lambda_{\mu}\) נקרא מקדמי שמידט.

\end{definition}
\begin{proposition}
כיוון שכל המקדמי שמידט הם חיוביים וממשיים ניתן לסדר אותם:
$$\lambda_{1}\geq \lambda_{2} \geq \dots \geq \lambda_{d}$$
כאשר דרישת הנירמול נותנת
$$\sum_{\mu=1}^{d} \lambda_{\mu}^{2}=1$$

\end{proposition}
\begin{definition}[דרגת שמידט]
מספר המקדמי שמידט שאינו אפס.

\end{definition}
\begin{remark}
הדרגת שמידט המינימלית תהיה 1. זאת נובע מדרישת הנירמול.

\end{remark}
\begin{proposition}[דרגת שמידט 1]
עבור דרגת שמידט 1 נקבל
$$\ket{\psi} = \ket{\mu=1}_{A} \otimes \ket{\mu=1}_{B}$$
ונקבל כי אין קשר בין התתי מערכות.

\end{proposition}
\begin{proposition}[דרגת שמידט גדולה מ-1]
עבור דרגת שמידט גדולה מ-1 נקבל כי \(\ket{\psi}\) היא שזורה.
$$\ket{\psi} =\sum_{\mu=1} \lambda_{\mu} \ket{\mu}_{A}\otimes \ket{\mu}_{B}   $$
כאשר מדידה של \(\ket{\mu}_{A}\) ב-\(A\) גורמת ל-\(\ket{\mu}_{B}\) מיידית.

\end{proposition}
\section{אנטרופיה שזירה}

\begin{proposition}
$$\rho_{A}=\mathrm{Tr_{B}}\left[\rho\right]=\sum_{\mu=1}^{d}\lambda_{\mu}^{2}\left|\mu\right\rangle_{A}\left\langle\mu\right|_{A}=\sum_{\mu=1}^{d}p_{\mu}\left|\mu\right\rangle_{A}\left\langle\mu\right|_{A}$$
- אם יש שזירה, אנחנו עושים טרייס אוט לאחד המרחבים אז המצב שנשאר יהיה מצב מעורב.
- ניתן להתייחס לזה כעת כמו צבר קלאסי, עם צפיפות הסתברות קלאסי, ולכן ניתן להגדיר את האטרופיה כדי לכמת את האי וודאות של המצב \(A\).

\end{proposition}
\begin{definition}[אנטרופיה פון נוימן]
מוגדר על ידי
$$S=-\sum_{\mu}p_{\mu}\ln p_{\mu}$$

\end{definition}
\begin{remark}
כיוון ש\(S\) הוא צירוף קמור נקבל \(S\geq 0\).

\end{remark}
\begin{proposition}
כאשר אין שזירה נקבל כי האנטרופיה אפס כאשר באופן כללי נקבל.
$$S_{A}=-\mathrm{Tr}\left[\rho_{A}\ln\rho_{A}\right]$$

\end{proposition}
\begin{proposition}[אנטרפיה מקסימלית]
האנטרופיה מקסימלית כאשר \(\rho_{A}\) הוא אקראי לחלוטין. כלומר
$$p_{\mu}=\frac{1}{d}\quad\forall\mu$$
וכעת:
$$S_{\mathrm{max}}=-\sum_{\mu=1}^{d}{\frac{1}{d}}\ln{\frac{1}{d}}=\ln d$$
כאשר המטריצת הצפיפות במקרה זה תהיה:
$$\rho=\frac{1}{d}\sum_{\mu}\left|\mu\right\rangle\left\langle\mu\right|=\frac{1}{d}\mathbb{1}$$

\end{proposition}
\begin{example}
נתון \(A\) מערכת 2 רמות ו-\(B\) מערכת 3 רמות. המערכת \(A\) ו-\(B\) שזורות באיזושהי דרך. המערכת הAB היא במצב טהור \(\ket{\psi}\). כלומר ניתן לכתוב איבר במערכת בצורה כללית:
$$\ket{\psi} =\sum_{ij}\psi_{ij}\ket{i} _{A}\otimes \ket{j} _{B}$$
כאשר בצורת שמידט נקבל:
$$\ket{\psi} = \sum_{\mu}\lambda_{\mu}\ket{\mu} _{A}\otimes \ket{\mu} _{B}$$
המספר המקסימלי של מקדמי שמידט יהיה:
$$\min \left( \dim A,\dim B \right)=2$$
כאשר במצב האנטרופיית שזירה המקסימלית תתקבל כאשר שתינן שווים, כלומר נדרש \(\lambda_{1}=\lambda_{2}=\frac{1}{2}\). לכן:
$$S_{\max }=-\sum_{i}\lambda_{i}\ln \lambda_{i}= -2\cdot\frac{1}{2}\ln \frac{1}{2}=\ln 2$$

\end{example}
\begin{proposition}
תהי \(\rho\) מטריצת צפיפות הפועלת על שתי מערכות \(A,B\). אופרטור אוניטרי הפועל רק על \(B\) לא תשנה את המטריצה המצומצמת \(\rho _A\).

\end{proposition}
\begin{proof}
נתון לנו המצב \(\rho\). נשים לב כי עבור מצב זה:
$$\rho_{A}=\mathrm{Tr}\left( \rho \right)$$
אם נבצע טרנספורמציה אוניטרית \(U_{B}\) הפועלת על תת מערכת \(B\) המצב \(\rho\) ישתנה למצב:
$$ ρ' = (I_A ⊗ U_B) ρ (I_A ⊗ U_B)^† $$
כאשר כעת:
$$ρ'_A = \text{Tr}_B[(I_A ⊗ U_B) ρ (I_A ⊗ U_B)^†] \\= \text{Tr}_B[(I_A ⊗ U_B) ρ (I_A ⊗ U_B^†)] \\= \text{Tr}_B[U_B ρ U_B^†]  = \mathrm{Tr}_{B}\left( \rho \right)
$$

\end{proof}
\begin{corollary}
אנטרופיית השזירה גם כן לא תשתנה

\end{corollary}
\begin{proof}
נזכור כי אנטרופיית השזירה נתונה על ידי:
$$S=-\mathrm{Tr}\left[\rho_{A}\ln\rho_{A}\right]$$
ולכן כיוון ש-\(\rho_{A}\) אינווריאנטית גם \(S\) אינווריאנטית.

\end{proof}
\begin{corollary}
אם ניתן לכתוב אופרטור אוניטרי בצורה \(U=U_{A}\otimes U_{B}\) נקבל כי אנטרופיית השזירה לא תשתנה

\end{corollary}
\begin{proof}
אפשר לכתוב:
$$U_{A}\otimes U_{B}=U_{A}\otimes \mathbb{1} _{B}+\mathbb{1} _{A}\otimes U_{B}$$
וכעת מלינאריות העקבה האנטרופיית שזירה תהיה שווה לסכום של האנטרופיות שזירה של כל אחד מהמצבים האלה - אשר אינם משתנים תחת הטרנספורמציה. פיזיקלית אם פועלים עם כל מערכת בנפרד נצפה כי השזירה של המערכת לא תשתנה, אבל אם פועל על שתיהם ביחד השזירה יכולה להשתנות ואף להפוך את המערכת למצב מכפלה. 

\end{proof}
\begin{example}
נראה זאת בעזרת פירוק שמידט על מצב טהור:
ניתן לכתוב כל מצב טהור של שתי חלקיקים בעזרת פירוק שמידט בצורה הבאה:
$$ |ψ⟩ = \sum_μ \sqrt{λ_μ} |μ⟩_A ⊗ |μ⟩_B $$
כאשר \(|μ⟩_A\) ו-\(|μ⟩_B\) הם בסיסים אורתונורמליים עבור תת-מערכות A ו-B, בהתאמה, ו-\(λ_μ\) הם מקדמי שמידט (מספרים ממשיים לא שליליים). בצורת שמידט ניתן לחשב את האנטרופיה של המערכת בצורה הבאה:
$$ S = -\sum_μ λ_μ \ln λ_μ $$
כאשר נפעיל את האופרטור \(U_{A}\otimes U_{B}\) נקבל את המצב:
$$ U|ψ⟩ = \sum_μ \sqrt{λ_μ} (U_A|μ⟩_A) ⊗ (U_B|μ⟩_B) $$
ניתן כעת להגדיר בסיס אורתונורמלי חדש:
$$ |\tilde{μ}⟩_A = U_A|μ⟩_A \qquad  |\tilde{μ}⟩_B = U_B|μ⟩_B $$
כאשר שזה אכן יהיה בסיס אורתונורמלי בגלל התכונה של מטריצות אוניטריות - בעבירה בסיס אורתונורמלי לבסיס אורתונורמלי. לכן במצב חדש נקבל:
$$ U|ψ⟩ = \sum_μ \sqrt{λ_μ} |\tilde{μ}⟩_A ⊗ |\tilde{μ}⟩_B $$
ומקדמי השמידט לא משתנים! ולכן האנטרופיית שזירה לא משתנה.

\end{example}
\Chapter{מערכות פתוחות}

\section{מדידה אורתוגונאלית}

\begin{definition}[מדידה אורתוגונאלית]
עבור ערך מדיד \(X\) אוסף של אופרטורי הטלה אשר מוגדרים לפי המצבים העצמים של \(X\) המוגדרים על ידי:
$$\Pi_{n}=\ket{n} \bra{n} $$

\end{definition}
\begin{proposition}[תכונות של מדידה אורתונורמלי]
מדידה אורתוגונאלית מקיימת:
$$\Pi_{n}\Pi_{m}=\delta_{mn}\Pi_{n}\quad \Pi_{n}^{2}=\mathbb{1} \quad \Pi_{n}=\Pi_{n}^{\dagger}$$
וכן מקיים יחס שלמות:
$$\Pi_{1}+\Pi_{2}+\dots+\Pi_{n}=\sum_{i}\Pi_{i}=\mathbb{1} $$

\end{proposition}
\begin{proposition}
ההסתברות לקרוס למצב \(n\) נתון על ידי:
$$p_{n}=\left\langle  \Pi_{n}  \right\rangle $$

\end{proposition}
\begin{proposition}
המערכת תקרוס למטיצת צפיפות:
$$\rho'= \frac{\Pi_{n}\rho \Pi_{n}}{\mathrm{Tr}\left( \Pi_{n}\rho \Pi_{n} \right)}$$

\end{proposition}
\begin{proof}
המערכת תעבור לבסיס שבה מופעל האופרטור \(\Pi_{n}\), כלומר \(\Pi_{n}^{\dagger}\rho \Pi_{n}\) כאשר נדרש לנרמל וכן נשים לב כי \(\Pi ^{\dagger}_{n}=\Pi_{n}\) ולכן:
$$\rho'= \frac{\Pi_{n}\rho \Pi_{n}}{\mathrm{Tr}\left( \Pi_{n}\rho \Pi_{n} \right)}$$

\end{proof}
\begin{remark}
נשים לב כי אם נבצע מדידה חוזרת מייד אחרי, נקבל את אותה מטריצת צפיפות.

\end{remark}
\begin{proposition}[ביצוע מדידה]
כעת נסתכל על איך ניתן לבצע מדידה בפועל. נדרש לשזור את המערכת המיקרוסקופית למערכת מאקרוסקופית, אותה ניתן למדוד. נבצע את השלבים הבאים:

  \begin{enumerate}
    \item נגדיר גודל מדיד \(Q\), אותו אנחנו יכולים לקרוא ומייצג מכשיר מדידה, ויהיה עם פירוק ספקטרלי \(Q\ket{q}=q\ket{q}\). 


    \item נאתחל את המערכת ב-\(\ket{q=0}\). נרצה שהמדידה תעביר את המערכת ל-\(\ket{q=x_{n}}\) עבור ערך נמדד \(x_{n}\) של \(X\). 


    \item כדי למדוד נפעיל את האופרטור האוניטרי: 
$$U=\sum_{n,q}\Pi_{n}\otimes|q+x_{n}\rangle\,\langle q|$$


  \end{enumerate}
\end{proposition}
\begin{proof}
ראשית נראה שאוניטרי:
\begin{gather*}U U^{\dagger}=\!\!\!\sum_{n,q\ n^{\prime},q^{\prime}}\left(\Pi_{n}\otimes\left|q+x_{n}\right\rangle\left\langle q\right|\right)\left(\Pi_{n^{\prime}}\otimes\left|q^{\prime}\right\rangle\left\langle q^{\prime}+x_{n^{\prime}}\right|\right)=\\=\sum_{n,q}\sum_{n^{\prime},q^{\prime}}\Pi_{n}\Pi_{n^{\prime}}\otimes\left|q+x_{n}\right\rangle\left\langle q|q^{\prime}\right\rangle\left\langle q^{\prime}+x_{n^{\prime}}\right|=\sum_{n,q}\Pi_{n}\otimes\left|q+x_{n}\right\rangle\left\langle q+x_{n}\right| 
\end{gather*}
כיוון ש-\(q+x_{n}\) הוא גם ערך עצמי של \(Q\) ניתן לשנות את אינדקס הסכימה ל-\(q''=q+x_{n}\) וכעת:
$$U U^{\dagger}=\sum_{n}\Pi_{n}\otimes\sum_{q^{\prime\prime}}|q^{\prime\prime}\rangle\,\langle q^{\prime\prime}|=\mathbb{I}\otimes\mathbb{I}$$
כאשר כאשר נפעיל את האונירי על מצב \(\ket{\psi}\otimes \ket{q=0}\) נקבל כי אם \(\ket{\psi}=\sum_{n}a_{n}\ket{n}\) נקבל:
$$|\Psi\rangle\equiv U\,|\psi\rangle\!\otimes\!|q=0\rangle=U\left(\sum_{n}a_{n}\,|n\rangle\right)\!\otimes\!|q=0\rangle=\sum_{n}a_{n}\,|n\rangle\!\otimes\!|q=x_{n}\rangle$$

\end{proof}
\begin{corollary}
נמצב שהתקבל יהיה:
$$|n\rangle\otimes|q=x_{n}\rangle$$
כאשר ההסתברות נתונה על ידי:
$$p_{n}=\left\langle\Psi\right|\left(\mathbb{1}\otimes\left|q=x_{n}\right\rangle\left\langle q=x_{n}\right|\right)\left|\Psi\right\rangle$$

\end{corollary}
\begin{example}[מדידה על קיוביט]
נניח כי נתון קיוביט מהצורה:
$$|\psi\rangle_{A}=a|0\rangle_{A}+b|1\rangle_{A}$$
נרצה למדוד את \(A\) בבסיס \(Z\)(הבסיס החישובי). על ידי שימוש בקיוביט נוסף \(E\) כמכשיר מדידה.
\textbf{דרך 1} - נרצה שאם \(A\) ב-\(\ket{0}\) אז \(E\) לא ישתנה(הרי מכשיר מדידה מאותחל ל-0) כאשר אם \(A\) ב-\(\ket{1}\) אז \(E\) יהפוך את המצב שלו, וספציפית יהפוך את \(\ket{0}\) ל-\(\ket{1}\). האופורטור שעושה את זה הוא \(X_{E}\). לכן נקבל:
$$U=|0\rangle\langle0|_{A}\otimes \mathbb{1}_{E}+|1\rangle\langle1|_{A}\otimes X_{E}$$\textbf{דרך 2} - נשתמש בצורה הכללית של מדידה אורתוגונאלית:
$$U=\sum_{n=0,1}\Pi_{n}\otimes|q+n\rangle\langle q|$$
אם נציב את ה-n-ים ונסכום נקבל:
$$U=|0\rangle\langle0|_{A}\otimes|q\rangle\langle q|+|1\rangle\langle1|_{A}\otimes|q\oplus1\rangle\langle q|$$
כיוון שמכישר המדידה הוא קיוביט וניתן לקבל רק את הערכים 0 ו-1 ניתן לכתוב:
$$U=|0\rangle\langle0|_{A}\otimes \mathbb{1}_{E}+|1\rangle\langle1|_{A}\otimes X_{E}$$

\end{example}
\begin{example}
נסתכל על מערכת של קיוביט יחיד:
$$|\psi\rangle=a\,\ket{\uparrow } +b\,\ket{\downarrow } $$
כאשר אם נמדוד ישירות בבסיס \(z\) יש לנו הסתברות של \(\lvert a \rvert^{2}\) ו-\(\lvert b \rvert^{2}\) עבור \(\ket{\uparrow}\) ו-\(\ket{\downarrow}\) בהתאמה. אם נמדוד בבסיס \(x\) נקבל הסתברויות של \(\frac{1}{2}\lvert a+b \rvert^{2}\) ו-\(\frac{1}{2}\lvert a-b \rvert^{2}\) עבור \(\ket{\uparrow_{x}}\) ו-\(\ket{\downarrow_{x}}\). כעת אם נרצה לבצע את המדידה בעזרת מכשיר מדידה, נדרש שזה יהיה אופרטור ספין. נתחל אותו במצב \(\ket{\uparrow}\) ונצמד אותו למערכת בעזרת האופרטור האוניטרי:
$$U=|\!\uparrow\rangle\left\langle\uparrow\!|\otimes\mathbb{1}+|\!\downarrow\rangle\left\langle\downarrow\!|\otimes\sigma_{x}\right.\right.$$
כך ש:
$$U\left|\psi\right\rangle\otimes\left|\uparrow\right\rangle=a\left|\uparrow\right\rangle\otimes\left|\uparrow\right\rangle+b\left|\downarrow\right\rangle\otimes\left|\downarrow\right\rangle$$
כעת נמדוד בבסיס ה-\(z\). נקבל \(\ket{\uparrow}\otimes \ket{\uparrow}\) או \(\ket{\downarrow}\otimes \ket{\downarrow}\) עם הסתברויות של \(\lvert a \rvert^{2}\) ו-\(\lvert b \rvert^{2}\) כמו שהיינו רוצים. אך אם נרצה למדוד בבסיס \(x\) נקבל:
$$U\left|\psi\right\rangle\otimes\left|\uparrow\right\rangle=\frac{1}{\sqrt{2}}\left(a\left|\uparrow\right\rangle+b\left|\downarrow\right\rangle\right)\otimes\left|\uparrow_{x}\right\rangle+\frac{1}{\sqrt{2}}\left(a\left|\uparrow\right\rangle-b\left|\downarrow\right\rangle\right)\otimes\left|\downarrow_{x}\right\rangle$$
עם הסתברות קבוע של \(\frac{1}{2}\) עבור שתי המקרים, שזה תוצאה שונה מאשר אם היינו מודדים את המערכת ישירות.

\end{example}
\begin{remark}
מדידה בצורה הזאת מעלה בעיה - אנחנו נדרשים למדוד בבסיס של \(Q\).

\end{remark}
\begin{remark}
נדרש להגדיר צורה חדשה למדוד כדי לפתור את הבעיה זו. צורה זו תהיה הכללה של המדידה האורתוגונאלית. זה נקרא מדידה מוכללת.

\end{remark}
\begin{example}[מדידת פון-נוימן]
נתון המילטוינאן \(H_{0}\) של המערכת, מכשיר מדידה עם מציין(pointer) \(Q\) ותנע \(P\). המצב ההתחלתי יהיה \(\ket{\psi}=\sum_{n}a_{n}\ket{n}\) ו-\(\ket{q=0}\). המילטוניאן האינטרקציה בין מכשיר המדידה למערכת נתון על ידי:
$$H_{i n t}(t)=\lambda\delta(t)X P$$
נראה איך המדידה תאפשר לקרוא במכשיר מדידה את המדידה של המערכת.
אנו יודעים כי יחס החילוף נתון על ידי \([Q,P]=i\mathbb{1}\). הקידום בזמן בחלק קטן זמני קטן סביב \(t=0\) יהיה:
$$U=\operatorname*{lim}_{\epsilon\rightarrow0}e^{-i\int_{-\epsilon}^{\epsilon}d t H_{\mathrm{int}}(t)}=e^{-i\lambda X P}$$
כאשר ניתן לכתוב את האופרטור בעזרת אופרטורי הטלה:
$$U=e^{ -i\lambda X\otimes P }=\sum_{n}\Pi_{n}\otimes e^{-i\lambda x_{n}P}$$
ולכן המצב השזור הסופי יהיה:
$$U|\psi\rangle\otimes|q=0\rangle=\sum_{n}a_{n}\left|n\right\rangle\otimes e^{-i\lambda x_{n}P}\left|q=0\right\rangle=\sum_{n}a_{n}|n\rangle\otimes|q=\lambda x_{n}\rangle$$
כך שהמציין נותן לנו את הערך עצמי הנמדד.

\end{example}
\begin{summary}
  \begin{itemize}
    \item כדי למדוד מצב קוונטי נדרש לשזור אותו למכשיר מדידה \(Q\).
    \item אופרטור אוניטרי אשר מאפשר הוא:
$$U=\sum_{n=0,1}\Pi_{n}\otimes|q+n\rangle\langle q|$$
    \item תלויה בבסיס של מכשיר המדידה \(Q\).
  \end{itemize}
\end{summary}
\section{מדידה מוכללת}

\begin{definition}[מדידה מוכללת]
נסמן במערכת \(A\) את המערכת שאותה אנחנו רוצים למדוד וב-\(B\) את מכשיר המדידה. כמו כן נניח כי נתון אופרטור \(U_{AB}\) אוניטרי אשר מבצע את המדידה, וכן נסמן ב-\(\left\{  \ket{\mu}  \right\}\) בסיס אורתונורמלי של מערכת \(B\). נגדיר:
$$\left|\Psi\right\rangle_{A B}=U_{A B}\left|\psi\right\rangle_{A}\otimes\left|0\right\rangle_{B}=\sum_{\mu}\left(M_{\mu}\left|\psi\right\rangle_{A}\right)\otimes\left|\mu\right\rangle_{B}$$
כאשר \(M_{\mu}\) נקראים \underline{אופרטורי קראוס} ומוגדרים על ידי:
$$M_{\mu}=\left\langle\mu\right|_{B}U_{A B}\left|0\right\rangle_{B}$$

\end{definition}
\begin{example}
עבור מדידה של קיוביט \(A\) בעזרת מכשיר מדידה \(E\) ראינו כי האופרטור האוניטרי של המדידה האורתוגונאלי תהיה:
$$U=|0\rangle\langle0|_{A}\otimes \mathbb{1}_{E}+|1\rangle\langle1|_{A}\otimes X_{E}$$
ולכן נקבל כי כיוון שיש 2 ערכים אפשריים עבור \(\mu\) נקבל 2 אופרטורי קראוס:
\begin{gather*}M_{0}=\bra{0} _{E}U\ket{0} _{B}=\bra{0}_{E} \left( |0\rangle\left\langle 0|_{A}\otimes \mathbb{1}_{E}+|1 \right\rangle\langle1|_{A}\otimes X_{E} \right)\ket{0} _{E} \\=\ket{0} _{A}\bra{0} _{A}\otimes \braket{ 0 | _{E}|0 }   _{E}+\ket{1} _{A}\bra{1} _{A}\otimes \left( \bra{0} _{E}\ket{1} _{E}  \right)= \\=\ket{0} _{A}\bra{0} _{A} \\M_{1}=\bra{1} _{E}U\ket{0} _{E}=\bra{1}_{E} \left( |0\rangle\left\langle 0|_{A}\otimes \mathbb{1}_{E}+|1 \right\rangle\langle1|_{A}\otimes X_{E} \right)\ket{0} _{E} \\=\ket{0} _{A}\bra{0} _{A}\otimes \braket{ 1 | _{E}|0 }   _{E}+\ket{1} _{A}\bra{1} _{A}\otimes \left( \bra{1} _{E}\ket{1} _{E}  \right)= \\=\ket{1} _{A}\bra{1} _{A}
\end{gather*}
כאשר מתקיים:
\begin{gather*}M_{0}^{\dagger}M_{0}=(|0\rangle\langle0|_{A})^{\dagger}(|0\rangle\langle0|_{A})=|0\rangle\langle0|_{A}\\ M_{1}^{\dagger}M_{1}=(|1\rangle\langle1|_{A})^{\dagger}(|1\rangle\langle1|_{A})=|1\rangle\langle1|_{A} 
\end{gather*}
ואכן נקבל:
$$M_{0}^{\dagger}M_{0}+M_{1}^{\dagger}M_{1}=\mathbb{1} _{A}$$

\end{example}
\begin{proposition}[יחס שלמות של אופרטורי קראוס]
אופרטורי קראוס מקיימים:
$$\sum_{\mu}M_{\mu}^{\dagger}M_{\mu}=\mathbb{1}_{A}$$

\end{proposition}
\begin{proof}
מנרמול של המצב \(\left|\Psi\right\rangle_{A B}=\sum_{\mu}\left(M_{\mu}\left|\psi\right\rangle_{A}\right)\otimes\left|\mu\right\rangle_{B}\) נקבל
$$1=\sum_{\mu,\nu}\left(\left\langle\psi\right|_{A}M_{\nu}^{\dagger}\otimes\left\langle\nu\right|_{B}\right)\left(M_{\mu}\left|\psi\right\rangle_{A}\otimes\left|\mu\right\rangle_{B}\right)=\left\langle\psi\right|_{A}\sum_{\mu}M_{\mu}^{\dagger}M_{\mu}\left|\psi\right\rangle_{A}$$
מזה נקבל כי:
$$\sum_{\mu}M_{\mu}^{\dagger}M_{\mu}=\mathbb{1}_{A}$$

\end{proof}
\begin{proposition}
ההסתברות למדוד את \(\mu\) יהיה:
$$p_{\mu}=\left\langle\Psi\right|_{A B}\left(\mathbb{1}_{A}\otimes\left|\mu\right\rangle\left\langle\mu\right|_{B}\right)\left|\Psi\right\rangle_{A B}=\left\langle\psi\right|_{A}M_{\mu}^{\dagger}M_{\mu}\left|\psi\right\rangle_{A B}$$

\end{proposition}
\begin{proposition}
המערכת תקרוס למצב:
$$\frac{M_{\mu}\left|\psi\right\rangle_{A}}{\left\|M_{\mu}\left|\psi\right\rangle_{A}\right\|}\otimes\left|\mu\right\rangle_{B}$$

\end{proposition}
\begin{remark}
נשים לב כי בניגוד למדידה אורתוגונאלית המערכת לאו דווקא תקרוס לאותו מצב אחרי מדידה חוזרת, היא תקרוס למצב:
$$\frac{M_{\nu}M_{\mu}\left|\psi\right\rangle_{A}}{\left\|M_{\nu}M_{\mu}\left|\psi\right\rangle_{A}\right\|}$$
כאשר התוצאה תהיה זהה אם"ם מתקיים:
$$M_{\nu}M_{\mu}=e^{i\theta}\delta_{\mu\nu}M_{\mu}$$
שזה מתקיים אם"ם האופרטורים אורתוגונאלים.

\end{remark}
\begin{definition}[אופרטור מדידה חיובי - POVM]
מוגדר בעזרת אופרטורי קראוס על ידי:
$$E_{\mu}=M_{\mu}^{\dagger} M_{\mu}$$
כאשר נזכור כי אופרטור מהצורה הזו הוא אופרטור חיובי למחצה.

\end{definition}
\begin{proposition}
ההסתברות להיות במצב \(\mu\) יהיה:
$$\mathrm{Prob}\left( \mu \right)=\mathrm{Tr}\left( \rho E_{\mu} \right)$$

\end{proposition}
\begin{proposition}
האופרטורים החיוביים \(\left\{  E_{\mu}  \right\}\) הוא אוסף שלם של אופרטורים הרמיטיים אי שליליים, כלומר:

  \begin{enumerate}
    \item הרמיטיים -  \(E_{\mu}=E_{\mu}^{\dagger}\). 


    \item חיוביים - \(\left\langle  \psi|E_{\mu}|\psi  \right\rangle \geq 0\) לכל \(\ket{\psi}\in \mathcal{H}\). 


    \item מקיימים יחס שלמות: 
$$\sum_{\mu}E_{\mu}=\mathbb{1} $$


  \end{enumerate}
\end{proposition}
\begin{corollary}
מהפירוק הפולארי של אופרטורים חיוביים ניתן לכתוב:
$$M_{\mu}=U_{\mu}\sqrt{ E_{\mu} }$$
כאשר \(U_{\mu}\) אופרטור אוניטרי כך שמתקיים \(M_{\mu}^{\dagger}M_{\mu}=E_{\mu}\).

\end{corollary}
\begin{proposition}
אם נציב את הפירוק הפולארי במשוואה:
$$U:|\psi\rangle_{A}\otimes|0\rangle_{B}\mapsto\sum_{a}M_{a}|\psi\rangle_{A}\otimes|a\rangle_{B}.$$

\end{proposition}
\begin{example}
נניח כי יש לנו צבר עם מטריצת צפיפות:
$$\rho=\frac{1}{2}\left(|\!\uparrow\rangle\left\langle\uparrow\right|+|\!\uparrow_{x}\rangle\left\langle\uparrow_{x}\right|\right)$$
כיוון ש-\(\braket{ \uparrow | \uparrow_{x} }\neq 0\)  לא ניתן להכריע אם אנחנו ב-\(\ket{\uparrow}\) או \(\ket{\uparrow_{x}}\) בעזרת מדידה אורתונורמלית. אם נמדוד את \(\sigma_{x}\) אם נקבל 1, אנחנו לא יכולים להגיד בוודאות באיזה מצב אנחנו. יש הסתברות של \(\frac{3}{4}\) שאנחנו במצב \(\ket{\uparrow_{x}}\) והסתברות של \(\frac{1}{4}\) שאנחנו במצב \(\ket{\uparrow}\). ניתן לקבל אחוזים טובים יותר בעזרת POVM:
נגדיר:
$$E_{1}=\lambda \ket{\downarrow } \bra{\downarrow } \qquad  E_{2}=\lambda \ket{\downarrow _{x}} \bra{\downarrow _{x}} \qquad E_{3}= \mathbb{1} -E_{1}-E_{2}$$
כאשר הגדרנו את \(E_{3}\) כדי שייקם את יחס השלמות \(\sum_{\mu}E_{\mu}=\mathbb{1}\), ונדרוש חיוביות של \(E_{3}\). ולכן:
$$E_{3}=\frac{1}{2}\left(\begin{array}{c c}{{2-\lambda}}&{{\lambda}}\\ {{\lambda}}&{{2-3\lambda}}\end{array}\right)$$
כאשר הערכים העצמיים יהיו:
$$1-\lambda\left(1\pm\frac{1}{\sqrt{2}}\right)\geq 0\implies \lambda \leq 2 - \sqrt{ 2 }$$
כעת ההסתברות שהיה במצב \(\mu=1\), כלומר במצב שבו היה ב-\(\ket{\uparrow_{x}}\) יהיה:
$$p_{\mu=1}=\mathrm{Tr}\left[\rho E_{1}\right]=\frac{\lambda}{2}\mathrm{Tr}[\left|\uparrow\right\rangle\left\langle\uparrow|\downarrow_{x}\rangle\left\langle\downarrow_{x}\right|\right]=\frac{\lambda}{4}$$
כאשר ההסתברות שהיה במצה \(\mu=2\) כלומר במצב שבו היה \(\ket{\uparrow}\) נקבל:
$$p_{\mu=2}=\mathrm{Tr}\left[\rho E_{2}\right]=\frac{\lambda}{2}\mathrm{Tr}\left[\left|\uparrow_{x}\rangle\left\langle\uparrow_{x}\right.\right|\downarrow\right\rangle\left\langle\downarrow\right|]=\frac{\lambda}{4}$$
כאשר \(\mu=3\) לא עוזר לנו. ולכן ההסתברות שנצליח לגלות באיזה מצב היה יהיה:
$$P=p_{\mu=1}+p_{\mu=2}=\frac{\lambda}{2}$$
כאשר הערך המקסימלי מתקבל כאשר נבחר \(\lambda = 2-\sqrt{ 2 }\) ובמקרה זה:
$$P=1-\frac{1}{\sqrt{ 2 }}\approx 0.29$$
שזה הסתברות קצת יותר טובה מ-\(\frac{1}{4}\).

\end{example}
\begin{summary}
  \begin{table}[htbp]
    \centering
    \begin{tabular}{|ccc|}
      \hline
      קריטריון & מדידה אורתוגונאלית & מדידה מוכללת \\ \hline
      קריסה & אופרטור הטלה - \(\prod_{n}=\ket{n}\bra{n}\) & אופרטור קראוס - \(\braket{ \mu \| _{B} U_{AB}\|0 }_{B}\) \\ \hline
      אופרטור חיובי & \(\Pi_{n}=\Pi_{n}^{\dagger}\Pi_{n}\) & \(E_{\mu}=M_{\mu}^{\dagger}M_{\mu}\) \\ \hline
      יחס שלמות & \(\sum_{n}\Pi_{n}=\mathbb{1}\) & \(\sum_{\mu} M_{\mu}^{\dagger}M_{\mu}=\mathbb{1}=\sum_{\mu}E_{\mu}\) \\ \hline
      התפלגות הסתברות & \(p_{n}=\mathrm{Tr}\left( \rho \Pi_{n} \right)\) & \(p_{\mu}=\mathrm{Tr}\left( \rho E_{m} \right)\) \\ \hline
      תוצאות המדידה & $$\rho\to \rho' = \frac{\Pi_{n}\rho \Pi_{n}}{\mathrm{Tr}\left( \Pi_{n}\rho \Pi_{n} \right)}$$ & \(\rho\to \rho' = \frac{M_{\mu}\rho M_{\mu}^{\dagger}}{\mathrm{Tr}\left( M_{\mu}\rho M_{\mu}^{\dagger} \right)}\) \\ \hline
    \end{tabular}
  \end{table}
\end{summary}
\section{ערוצים קוונטים}

\begin{definition}[ערוץ קוונטי]
עבור מרחב הילברט \(\mathcal{H}\) אופרטור \(\mathcal{E}\in \mathrm{End}\left( \mathcal{H} \right)\) נקרא ערוץ קוונטי אם לכל מטריצת צפיפות \(\rho\) מקיים:
$$\mathcal{E} \left( \rho \right)=\sum_{\mu}M_{\mu}\rho M_{\mu}^{\dagger}$$
כאשר \(M_{\mu}\) אופרטורי קראוס.

\end{definition}
\begin{remark}
ערוץ קוונטי למעשה מתאר את המקרה שבו אנחנו מבצעים מדידה אך לא מסתכלים על התוצאות של המדידה, כלומר המצב קרס אבל אנחנו לא יודעים לאיזה מצב. לכן ערוצים קוונטים עוזרים לתאר דיקוהרנטיות.

\end{remark}
\begin{proposition}
תכונות של ערוצים קוונטים

  \begin{enumerate}
    \item לינאריות: 
$${\mathcal{E}}(\alpha\rho_{1}+\beta\rho_{2})=\alpha{\mathcal{E}}(\rho_{1})+\beta{\mathcal{E}}(\rho_{2}).$$


    \item משמר הרמיטיות: 
$$\rho=\rho ^{\dagger}\implies \mathcal{E} \left( \rho \right)=\mathcal{E} \left( \rho \right)^{\dagger}$$


    \item משמר חיוביות: 
$$\rho \geq 0\implies \mathcal{E} \left( \rho \right)\geq 0$$


    \item משמר עקבה: 
$$\mathrm{Tr}\left( \mathcal{E} \left( \rho \right) \right)=\mathrm{Tr}\left( \rho \right)$$


  \end{enumerate}
\end{proposition}
\begin{proposition}
הרכבה של ערוצים קוונטים יהיה ערוץ קוונטי. בפרט אם \(\mathcal{E}_{1}\left( \rho \right)=\sum_{\mu}N_{\nu}\rho N_{\nu}^{\dagger}\) ו-\({} \mathcal{E}_{2}\left( \rho \right)=\sum_{\mu}M_{\mu}\rho M_{\mu}^{\dagger} {}\) ערוצים קוונטים אזי גם \(\mathcal{E}_{1}\circ\mathcal{E}_{2}\) יהיה ערוץ קוונטי ומתקיים:
$$\mathcal{E} _{2}\circ  \mathcal{E} _{1}\left( \rho \right)=\sum_{\mu,\nu} K_{\nu \mu}\rho K_{\nu \mu}^{\dagger}$$
כאשר \(K_{\nu \mu}=N_{\nu}M_{\mu}\) ומקיימות את יחס השלמות \(\sum_{\mu,\nu}K_{\nu \mu}^{\dagger}K_{\nu \mu}=\mathbb{1}\).

\end{proposition}
\begin{proposition}
ערוצים קוונטים הם לא הפיכים.

\end{proposition}
\section{שלושה ערוצי קיוביט}

\section{ערוץ דה-פולריזציה}

\begin{definition}[סוגי שגיאות של קיוביט]
אם יש לנו קיוביט יחיד שקשור לסביבה כלשהי. יש לנו שלושה סוגי שגיאות:

  \begin{enumerate}
    \item היפוך ביט: 
$$\ket{0} \leftrightarrow  \ket{1}  \text{ or } \ket{\uparrow } \leftrightarrow  \ket{ \downarrow }  $$
שזה נעשה על ידי \(X=\sigma_{x}\)


    \item טעות סימן/פאזה: 
$$\ket{+} \leftrightarrow  \ket{-} \text{ or } \ket{\uparrow _{x}} \leftrightarrow  \ket{\downarrow _{x}} $$
והאופרטור שמבצע טעות זו יהיה \(Z=\sigma_{z}\).


    \item שתי השגיאות מתקיימות ביחד: 
$$Y=\sigma _{y}=-iZX$$


  \end{enumerate}
\end{definition}
\begin{definition}[ערוץ דה-פולריזציה - Depolarizing]
מערכת שבה כל השגיאות מתקיימות בהסתברות שווה.

\end{definition}
\begin{proposition}
אם נניח כי יש הסתברות של \(1-p\) שלא תהיה שום שגיאה, כאשר יש הסתברות של \(\frac{p}{3}\) שיקרה כל אחד מבין השגיאות \(X,Y,Z\). ולכן נדרש 4 אופרטורי קראוס(3 עובר השגיאות ו-1 עבור המקרה שאין שגיאות).
$$M_{0}=\sqrt{ 1-p }\mathbb{1}  \qquad M_{i}=\sqrt{ \frac{p}{3} } \sigma_{i}
$$

\end{proposition}
\begin{proposition}
האופרטורים \(M_{0},M_{i}\) אכן אופרטורי קראוס.

\end{proposition}
\begin{proof}
נבדוק כי אכן אופרטורי קראוס:
$$M_{0}^{\dagger}M_{0}+\sum_{i}M_{i}^{\dagger} M_{i} = \mathbb{1} \left( 1-p+\frac{3}{p}\cdot 3 \right)=\mathbb{1} $$

\end{proof}
\begin{corollary}
ניתן להגדיר ערוץ קוונטי על ידי:
$$\mathcal{E} \left(\rho\right)=\left(1-p\right)\rho+\frac{p}{3}\left(\sigma_{x}\rho\sigma_{x}+\sigma_{y}\rho\sigma_{y}+\sigma_{z}\rho\sigma_{z}\right)$$
כאשר נשים לב כי לא נדרש לבדוק כי אכן אופרטור צפיפות כי ראינו כי אופקטור קלאוס אשר פועל על מטריצת צפיפות תתן מטריצת צפיפות.

\end{corollary}
\begin{proposition}
אם נפעיל את הערוץ הקוונטי על קיוביט כי השינוי יהיה מהצורה:
$$\mathbf{P}^{\prime}=\left(1-{\frac{4}{3}}p\right)\mathbf{P}$$

\end{proposition}
\begin{proof}
ניתן לכתוב מטריצת צפיפות של קיוביט בעזרת אופרטור הקיטוב:
$$\rho\left(\mathbf{P}\right)={\frac{1}{2}}\left(\mathbb{I}+\mathbf{P}\cdot\sigma\right)$$
נציב בערוץ ונקבל:
$$\epsilon\left(\rho\left(\mathbf{P}\right)\right)=\left(1-p\right)\rho\left(\mathbf{P}\right)+{\frac{p}{3}}\left(\sigma_{x}\rho\left(\mathbf{P}\right)\sigma_{x}+\sigma_{y}\rho\left(\mathbf{P}\right)\sigma_{y}+\sigma_{z}\rho\left(\mathbf{P}\right)\sigma_{z}\right)$$
כאשר ניתן מזה לקבל:
$$\epsilon\left(\rho\left(\mathbf{P}\right)\right)={\frac{1}{2}}\left(\mathbb{I}+\mathbf{P^{\prime}}\cdot\boldsymbol\sigma\right)=\rho\left(\mathbf{P^{\prime}}\right)$$
כאשר:
$$\mathbf{P}^{\prime}=\left(1-{\frac{4}{3}}p\right)\mathbf{P}$$

\end{proof}
\begin{remark}
למעשה קיבלנו כי הקיטוב \(\mathbf{P}\) קטן בגודלו בלבד, וזה למה זה נקרא depolarization.

\end{remark}
\begin{remark}
נשים לב כי אם \(p=\frac{3}{4}\) אנחנו מאבדים קוהרנטיות לחלוטין.

\end{remark}
\section{ערוץ דה-פאזה}

\begin{definition}[ערוץ דה-פאזה]
נניח כי יש חלקיק קוונטי \(A\) במערכת שתי רמות \(\left\{  \ket{0}_{A},\ket{1}_{A}  \right\}\). כאשר \(B\) יהיה חלקיק נוסף, ראשית ב-\(\ket{0}_{B}\) אשר יכול לבצע אינטגרציה על חלקיק \(A\). 
ערוץ דה פזה הוא ערוץ שבו בהסתברות \(1-p\) לא קורה כלום, כאשר מהסתברות \(p\) נקבל כי \(B\) יתפזר מעל \(A\) כך ש \(B\) ישתנה בצורה הבאה:

  \begin{enumerate}
    \item אם \(A=\ket{0}_{A}\)  נקבל כי \(B\) יעבור ל-\(\ket{1}_{B}\). 


    \item אם \(A=\ket{1}_{_{A}}\) נקבל כי \(B\) יעבור ל-\(\ket{2}_{B}\). 


  \end{enumerate}
\end{definition}
\begin{proposition}
ניתן לתאר את הערוץ בעזרת הטרנספורמציה האוניטרית הבאה:
$$\begin{array}{l}{{|0\rangle_{A}\otimes \ket{0}_{B} \mapsto\sqrt{1-p}\ |0\rangle_{A}\otimes|0\rangle_{B}+\sqrt{p}\ |0\rangle_{A}\otimes|1\rangle_{B}}}\\ {{|1\rangle_{A}\otimes \ket{0}_{B}\mapsto\sqrt{1-p}\ |1\rangle_{A}\otimes|0\rangle_{B}+\sqrt{p}\ |1\rangle_{A}\otimes|2\rangle_{B}}}\end{array}$$

\end{proposition}
\begin{proposition}
עבור מערכת זו יש לנו שלושה מצבים ונדרש להגדיר שלושה אופרטורי קראוס:

  \begin{enumerate}
    \item במצב שבו אין פיזור נגדיר: 
$$M_{0}=\left\langle0\right|_{B}U\left|0\right\rangle_{B}=\sqrt{1-p}\left(\left|0\right\rangle_{A}\left\langle0\right|_{A}+\left|1\right\rangle_{A}\left\langle1\right|_{A}\right)=\sqrt{1-p}\cdot\mathbb{1}_{A}$$


    \item במצב הפיזור הראשון נקבל: 
$$M_{1}=\left\langle1\right|_{B}U\left|0\right\rangle_{B}=\sqrt{p}\left|0\right\rangle_{A}\left\langle0\right|_{A}=\frac{\sqrt{p}}{2}\left(\mathbb{1}+\sigma_{z}\right)_{A}$$


    \item במצב הפיזור השני נקבל: 
$$M_{2}=\left\langle2\right|_{B}U\left|0\right\rangle_{B}=\sqrt{p}\left|1\right\rangle_{A}\left\langle1\right|_{A}=\frac{\sqrt{p}}{2}\left(\mathbb{1}-\sigma_{z}\right)_{A}$$
כאשר כדי להראות שאלו אופרטורי קראוס מספיק להראות כי מקיימים את יחס השלמות:
$$M_{0}^{\dagger}M_{0} + M_{1}^{\dagger}M_{1}+M_{2}^{\dagger}M_{2}=\mathbb{1} $$


  \end{enumerate}
\end{proposition}
\begin{corollary}
ערוץ דה-פאזה יהיה מהצורה:
$$\mathcal{E} \left(\rho\right)=\left(1-p\right)\rho+\frac{p}{4}\left(\left(\mathbb{1}+\sigma_{z}\right)\rho\left(\mathbb{I}+\sigma_{z}\right)+\left(\mathbb{1}-\sigma_{z}\right)\rho\left(\mathbb{1}-\sigma_{z}\right)\right)=\left(1-\frac{p}{2}\right)\rho+\frac{p}{2}\sigma_{z}\rho\sigma_{z}$$

\end{corollary}
\begin{proposition}
אם יש לנו מטריצת צפיפות מהצורה:
$$\rho=\left(\begin{array}{c c}{{\rho_{00}}}&{{\rho_{01}}}\\ {{\rho_{10}}}&{{\rho_{11}}}\end{array}\right)$$
מערוץ הדיפאזה נקבל:
$$\mathcal{E} \left(\begin{array}{c c}{{\rho_{00}}}&{{\rho_{01}}}\\ {{\rho_{10}}}&{{\rho_{11}}}\end{array}\right)=\left(\begin{array}{c c}{{\rho_{00}}}&{{\left(1-p\right)\rho_{01}}}\\ {{\left(1-p\right)\rho_{10}}}&{{\rho_{11}}}\end{array}\right)$$
כאשר נשים לב כי אלמנטי המטריצה על האלכסון נשמרים, כאשר אלמנטי המטריצה אשר מחוץ לאלכסון באיטיות דועכים, זה הרכיבים שמאכסנים את המידע על הפאזה.

\end{proposition}
\begin{remark}
כדי לראות מה קורה לקיטוב של חלקיק נסתכל על חלקיק עם מטרצית צפיפות \(\rho\left( \vec{P} \right)\). אם נפעיל על זה את הערוץ נקבל כי:
$$P_{3}\longrightarrow P_{3},\quad P_{1,2}\longrightarrow\left(1-\frac{p}{2}\right)P_{1,2}$$
כאשר למעשה נקבל כיווץ של הוקטור קיטוב עבור הרכיב \(x\) ו-\(y\) כאשר רכיב ה-\(z\) לא משתנה. 

\end{remark}
\begin{proposition}
אם יש שטף של חלקיקים עם קצב פיזור \(\Gamma\). אזי ההסתברות שיתפזר בזמן אינפיניטסימלי \(\mathrm{d}t\) יהיה \(p=\Gamma \mathrm{d}t\). עבור זמן סופי \(t=N\mathrm{d}t\) נקבל:
$$\mathcal{E} ^{N}\left(\begin{array}{c c}{{\rho_{00}}}&{{\rho_{01}}}\\ {{\rho_{10}}}&{{\rho_{11}}}\end{array}\right)=\left(\begin{array}{c c}{{\rho_{00}}}&{{\left(1-p\right)^{N}\rho_{01}}}\\ {{\left(1-p\right)^{N}\rho_{10}}}&{{\rho_{11}}}\end{array}\right)$$
ואם ניקח את הגבול כאשר \(t\to \infty\) (כלומר \(t\gg \Gamma^{-1}\)).
$$\left(1-p\right)^{N}=\left(1-\Gamma\frac{t}{N}\right)^{N}\underset{N\rightarrow\infty}{\longrightarrow}e^{-\Gamma t}$$

\end{proposition}
\section{ערוץ דעיכת אמפליטדה}

\begin{definition}[ערוץ דעיכת אמפליטודה]
מודל של דעיכה של מצב מעורער של של מערכת שתי רמות על ידי פליטה של פוטון.

\end{definition}
\begin{proposition}[ייצוג בעזרת אופרטור אוניטרי]
\begin{gather*}\ket{0} _{A}\ket{0} _{E}\mapsto \ket{0} _{A}\ket{0} _{E}  \\\ket{1} _{A}\ket{0} _{E}\mapsto \sqrt{ 1-p }\ket{1} \ket{0} _{E}+\sqrt{ p }\ket{0} _{A}\ket{1} _{E}
\end{gather*}
כאשר כמובן אם האטום מתחיל במצב יסוד, והמערכת במצב ווקום, לא קורה כלום.

\end{proposition}
\begin{proposition}
אופרטורי הקראוס המתאימים לערוץ דעיכת אמלפיטדה יהיו:
$$M_{0}=\begin{pmatrix}1 & 0 \\0 & \sqrt{ 1-p }\end{pmatrix}\qquad M_{1}=\begin{pmatrix}0 & \sqrt{ p } \\0 & 0
\end{pmatrix}$$

\end{proposition}
\begin{proof}
נרשום במפורש איך \(U\) פועל:
$$U=\ket{0} _{A}\ket{0} _{E}\bra{0} _{A}\bra{0} _{E}+\left( \sqrt{ 1-p }\ket{1}_{A} \ket{0} _{E}+\sqrt{ p }\ket{0} _{A}\ket{1} _{E} \right)\bra{1} _{A}\bra{0} _{E}$$
וכעת:
\begin{gather*}M_{0}=\left\langle0\right|_{E}U\left|0\right\rangle_{E}=\left|0\right\rangle_{A}\left\langle0\right|_{A}+\sqrt{1-p}\left|1\right\rangle_{A}\left\langle1\right|_{A}=\left(\begin{array}{c c}{{1}}&{{0}}\\ {{0}}&{{\sqrt{1-p}}}\end{array}\right)\\ M_{1}=\left\langle1\right|_{E}U\left|0\right\rangle_{E}=\sqrt{p}\left|0\right\rangle_{A}\left\langle1\right|_{A}=\left(\begin{array}{l l}{{0}}&{{\sqrt{p}}}\\ {{0}}&{{0}}\end{array}\right) 
\end{gather*}
כאשר אכן מתקיים:
$${ M}_{0}^{\dagger}{ M}_{0}+{ M}_{1}^{\dagger}{ M}_{1}=\left(\begin{array}{c c}{{1}}&{{0}}\\ {{0}}&{{1-p}}\end{array}\right)+\left(\begin{array}{c c}{{0}}&{{0}}\\ {{0}}&{{p}}\end{array}\right)=\mathbb{1} $$

\end{proof}
\begin{proposition}
ערוץ קוונטי כללי של קיוביט עובר בצורה הבאה:
$$\mathcal{E} \left( \rho_{i} \right)=\left(\begin{array}{c c}{{\rho_{00}+p\rho_{11}}}&{{\sqrt{1-p}\rho_{01}}}\\ {{\sqrt{1-p}\rho_{10}}}&{{\left(1-p\right)\rho_{11}}}\end{array}\right) $$

\end{proposition}
\begin{proof}
\begin{gather*}{\mathcal{E} \left(\rho_{i}\right)}{=\sum_{\mu}M_{\mu}\rho_{i}M_{\mu}^{\dagger}=\left(\begin{array}{c c}{{1}}&{{0}}\\ {{0}}&{{\sqrt{1-p}}}\end{array}\right)\left(\begin{array}{c c}{{\rho_{00}}}&{{\rho_{01}}}\\ {{\rho_{10}}}&{{\rho_{11}}}\end{array}\right)\left(\begin{array}{c c}{{1}}&{{0}}\\ {{0}}&{{\sqrt{1-p}}}\end{array}\right)}\\ \begin{array}{c c}{{+\left(\begin{array}{c c}{{0}}&{{\sqrt{p}}}\\ {{0}}&{{0}}\end{array}\right)\left(\begin{array}{c c}{{\rho_{00}}}&{{\rho_{01}}}\\ {{\rho_{10}}}&{{\rho_{11}}}\end{array}\right)\left(\begin{array}{c c}{{0}}&{{0}}\\ {{\sqrt{p}}}&{{0}}\end{array}\right)=\left(\begin{array}{c c}{{\rho_{00}+p\rho_{11}}}&{{\sqrt{1-p}\rho_{01}}}\\ {{\sqrt{1-p}\rho_{10}}}&{{\left(1-p\right)\rho_{11}}}\end{array}\right)}}\end{array} 
\end{gather*}

\end{proof}
\begin{proposition}
לאחר הפעלת הערוץ הקוונטי \(N\) פעמים נקבל:
$$\mathcal{E} ^{N}\left(\rho_{i}\right)=\left(\begin{array}{c c}{{\rho_{00}+\left(1-\left(1-p\right)^{N}\right)\rho_{11}}}&{{\left(1-p\right)^{N/2}\rho_{01}}}\\ {{\left(1-p\right)^{N/2}}}&{{\left(1-p\right)^{N}\rho_{11}}}\end{array}\right)$$

\end{proposition}
\begin{proof}
כל האיברים פרט לאיבר השמאלי העליון הם היו פשוט מכפלה ממושכת של המצבים שלפניהם. כלומר נעלה אותם בחזקת \(N\). האיבר הבעייתי היחיד הוא האיבר השמאלי העליון כי יש שם חיבור. נפעיל מספר פעמים ונזהה דפוס:
$$\rho_{00}\longrightarrow\rho_{00}+p\rho_{11}\longrightarrow\rho_{00}+p\rho_{11}+p\left(1-p\right)\rho_{11}=\rho_{00}+p\left(1+\left(1-p\right)\right)\rho_{11}$$
אם נחזור על התהליך הזה \(N\) פעמים נקבל טור הנדסי:
$$\rho_{00}\,\mathop{\longrightarrow}_{N\,\mathrm{times}}\rho_{00}+p\left(1+\left(1-p\right)+\left(1-p\right)^{2}+\cdots+\left(1-p\right)^{N-1}\right)\rho_{11}=\rho_{00}+p\left[\sum_{n=0}^{N-1}\left(1-p\right)^{n}\right]\rho_{11}\,.$$
בעזרת הנוסחה של טור הנדסי נקבל:
$$\rho_{00}\,\mathop{\longrightarrow}_{N\mathrm{~times}}\rho_{00}+\left(1-\left(1-p\right)^{N}\right)\rho_{11}$$

\end{proof}
\begin{corollary}
אם נניח כי הקצב דעיכה של מצב מעורער מטא-יציב הוא \(\Gamma\) כלומר \(p=\Gamma \mathrm{d}t\) נקבל:
$$\rho\left(t\right)=\left(\begin{array}{c c}{{\rho_{00}+\left(1-e^{-\Gamma t}\right)\rho_{11}}}&{{e^{-\Gamma t/2}\rho_{01}}}\\ {{e^{-\Gamma t/2}}}&{{e^{-\Gamma t}\rho_{11}}}\end{array}\right)$$

\end{corollary}
\begin{proof}
ניקח את הגבול כאשר \(N\to \infty\) ו-\(\mathrm{d}t\to 0\). נשים לב כי:
$$\left(1-p\right)^{N}=\left(1-{\frac{\Gamma t}{N}}\right)^{N}\underset{N\longrightarrow\infty}{\longrightarrow}e^{-\Gamma t}$$
כאשר אם נציב בטענה נקבל את המסקנה.

\end{proof}
\begin{corollary}
לאחר זמן ארוך נקבל את המצב:
$$\left(\begin{array}{c c}{{\rho_{00}+\rho_{11}}}&{{0}}\\ {{0}}&{{0}}\end{array}\right)=\begin{pmatrix}1 & 0 \\0 & 0
\end{pmatrix}=\ket{0} _{A}\bra{0}_{A}$$
כאשר השתמשנו בזמן ש-\(\rho_{00}+\rho_{11}=1\) כיוון שזו העקבה של המטריצה צפיפות המקורית.

\end{corollary}
\section{משוואת המאסטר}

\begin{definition}[ההנחה המרקובית]
כאשר יש לנו מערכת שמתקדמת בזמן בצורה הסתברויתית, כאשר ההתדמות בכל שלב תלויה אך ורק בשלב שלפנייה, נאמר שמקיימת את ההנחה המרקובית.

\end{definition}
\begin{remark}
דרך אחת לפרש את ההנחה המרקובית היא שהעתיד היא לא תלויה בעבר בהינתן ההווה.

\end{remark}
\begin{remark}
כאשר המערכת מקיימת את ההנחה המרקובית, ניתן לכתוב משוואה שתלויה ב-\(\rho(t)\) ו-\(\rho\left( t+\mathrm{d}t \right)\) בלבד, ולקבל משוואה דיפרנציאלית עבור המערכת.

\end{remark}
משוואת שרודינגר נכונה רק למערכות סגורות, ומניחה שימור הסתברות וקידום בזמן אוניטרי. נרצה למצוא משוואה אשר מכלילה את זה למערכת כללית, כאשר נדרש שתקיים את ההנחה המרקובית.

\begin{definition}[אופרטורי קראוס מוכללים]
$$M_{0}=\mathbb{1} +(-i H+K)\mathrm{d} t,\quad M_{i}=\sqrt{\mathrm{d} t}L_{i}$$
כאשר \(L_{i}\) נקראים אופרטורי קפיצה ומייצגים אופרטורי קראוס אשר אינם קרובים ליחידה(כלומר קפיצה), ו-\(K\) הוא אופקטור הרמיטי כך שמתקיים היחס:
$$K=-\frac{1}{2}\sum_{i}L_{i}^{\dagger}L_{i}$$

\end{definition}
הגדרות אלו נובעות באופן טבעי מהמשפט הבא.

\begin{theorem}[משוואת המאסטר - תמונת שרדינגר]
$$\frac{d}{d t}\rho=-i\left[H,\rho\right]+\sum_{i}\left({ L}_{i}\rho{ L}_{i}^{\dagger}-\frac{1}{2}{ L}_{i}^{\dagger}{ L}_{i}\rho-\frac{1}{2}\rho{ L}_{i}^{\dagger}{ L}_{i}\right)$$
כאשר \(L_{i}\) הם אופרטורים קפיצה המתארים את השינוי של המערכת בגלל הצימוד עם הסביבה.

\end{theorem}
\begin{proof}
נניח כי מערכת מתוארת על ידי ערוץ קוונטי:
$$\rho\left(t+d t\right)=\mathcal{E}_ {d t}\left(\rho\left(t\right)\right)=\sum_{\mu}M_{\mu}\rho\left(t\right)M_{\mu}^{\dagger}=\rho\left(t\right)+O\left(d t\right)$$
יש לנו אופרטור קראוס שמתאר את זה שלא קורה כלום:
$$M_{0}=\mathbb{1}+\left(-i H+K\right)d t$$
כאשר אם נדרוש שלמות של אופרטורי קראוס נקבל:
$$\mathbb{1}\ =\ \sum_{\mu}M_{\mu}^{\dagger}M_{\mu}\ =\ \mathbb{1}\ +d t\left(2K+\sum_{i}L_{i}^{\dagger}L_{i}\right)+O\left(d t^{2}\right)$$
נקבלאת  הגדרה של \(K=-\frac{1}{2}\sum_{i}L_{i}^{\dagger}L_{i}\).  נמצא כעת את הנגזרת \(\frac{\mathrm{d} \rho}{\mathrm{d} t}\) על ידי חישוב הגבול:
$${\frac{d}{d t}}\rho=\operatorname*{lim}_{d t\longrightarrow0}{\frac{\rho\left(t+d t\right)-\rho\left(t\right)}{d t}}=-i\left[H,\rho\right]+\sum_{i}\left(L_{i}\rho L_{i}^{\dagger}-{\frac{1}{2}}L_{i}^{\dagger}L_{i}\rho-{\frac{1}{2}}\rho L_{i}^{\dagger}L_{i}\right)$$

\end{proof}
\begin{remark}
נשים לב כי אם המערכת היא סגורה אין צימוד עם הסביבה ולכן \(L_{i}=0\) ונקבל:
$$\frac{d}{d t}\rho=-i\left[H,\rho\right]$$

\end{remark}
\begin{proposition}[משוואת המאסטר בתמונת היזנברג]
$$\frac{\mathrm{d} A}{\mathrm{d} t} =i[H,A]+\sum_{i}\left( L_{i}A^{\dagger}L_{i}-\frac{1}{2}L_{i}^{\dagger} L_{i}A-\frac{1}{2}AL_{i}^{\dagger}L_{i} \right)$$

\end{proposition}
\begin{proof}
ערך תצפית של גודל מדיד \(A\) בזמן \(t+\mathrm{d}t\) נתון על ידי:
$$\left\langle A\right\rangle_{t+d t}=\mathrm{Tr}\left[\rho\left(t+d t\right)A\right]=\mathrm{Tr}\left[\mathcal{E}_{d t}\left(\rho\left(t\right)\right)A\right]=\sum\mathrm{Tr}\left[M_{\mu}\rho\left(t\right)M_{\mu}^{\dagger}A\right]$$
נרצה להגדיר ערוץ בתמונת האיזברג על ידי:
$$A\left(t+d t\right)=\mathcal{E}_{d t}^{*}\left(A\left(t\right)\right)$$
כאשר זה יתן לנו:
$$\left\langle A\right\rangle_{t+d t}=\mathrm{Tr}\left[\rho\left(t\right)A\left(t+d t\right)\right]=\mathrm{Tr}\left[\rho\left(t\right)\mathcal{E}_{d t}^{\ast}\left(A\left(t\right)\right)\right].$$
נדרוש כי ערך התצפית יהיה זהה בשתי התמונות:
$$\mathrm{Tr}\left[\rho\left(t\right)\mathcal{E}_{d t}^{\ast}\left(A\left(t\right)\right)\right]=\sum_{\mu}\mathrm{Tr}\left[M_{\mu}\rho\left(t\right)M_{\mu}^{\dagger}A\left(t\right)\right]=\sum_{\mu}\mathrm{Tr}\left[\rho\left(t\right)M_{\mu}^{\dagger}A\left(t\right)M_{\mu}\right]$$
כאשר השתמשנו בציקליות של העקבה. מזה נסיק כי הערוץ \(\mathcal{E}_{\mathrm{d}t}\) יהיה הערוץ הדואלי של \(\mathcal{E}_{\mathrm{d}t}\) ויקיים:
$$A\left(t+d t\right)=\mathcal{E}_{d t}^{*}\left(A\left(t\right)\right)=\sum_{\mu}M_{\mu}^{\dagger}A\left(t\right)M_{\mu}$$
ולכן אם נחשב את הנגזרת \(\frac{\mathrm{d} }{\mathrm{d} t}A(t)\) נקבל:
\begin{gather*}{{\frac{d}{d t}A\left(t\right)=\operatorname*{lim}_{d t\longrightarrow0}\frac{A\left(t+d t\right)-A\left(t\right)}{d t}}}\\ {{=i\left[H,A\left(t\right)\right]+\sum_{i}\left(L_{i}^{\dagger}A\left(t\right)L_{i}-\frac{1}{2}L_{i}^{\dagger}L_{i}A\left(t\right)-\frac{1}{2}A\left(t\right)L_{i}^{\dagger}L_{i}\right)}}\end{gather*}

\end{proof}
\begin{example}[משוואת המאסטר על ערוץ דפולריזציה]
עבור ערוץ דיפולריזציה יש 3 אופרטורי קראוס אשר מתאימים לאופרטור קפיצה:
$$M_{i}=\sqrt{ \frac{p}{3} }\sigma_{i}$$
כאשר על ידי \(M_{i}=\sqrt{ \mathrm{d}t }L_{i}\) ביחד עם \(p=\Gamma \mathrm{d}t\) נקבל את האופרטורי קפיצה:
$$L_{i}=\sqrt{\frac{\Gamma}{3}}\sigma_{i}$$
נציב את אופרטורי הקפיצה במשוואת המאסטר תוך שימוש בזה שמטריצות פאולי הם הרמיטיות ואוניטריות(ולכן הריבוע שלהם עצמם) ונקבל:
$$\frac{d}{d t}\rho=\frac{\Gamma}{3}\sum_{i}\left(\sigma_{i}\rho\sigma_{i}-\rho\right)$$
כעת אם נציב על ערוץ קיוביט מהצורה \(\rho\left(\mathbf{P}\right)={\frac{1}{2}}\left(\mathbb{1}+\mathbf{P}\cdot\boldsymbol{\sigma}\right)\). נקבל עבור אגף שמאל:
$$\frac{d}{d t}\rho=\frac{1}{2}\frac{d{\bf P}}{d t}\cdot\boldsymbol{\sigma}$$
כאשר עבור אגף ימין נקבל:
$$\sum_{i}\sigma_{i}\rho\sigma_{i}=\frac{1}{2}\sum_{i}\left(\sigma_{i}^{2}+\sum_{j}P_{j}\sigma_{i}\sigma_{j}\sigma_{i}\right)$$
האיבר הראשון יהיה היחידה, כאשר אם נזכור כי \(\sigma_{j}\sigma_{i}=-\sigma_{i}\sigma_{j}+2\delta_{ij}\) נקבל:
$$\sum_{i}\sigma_{i}\rho\sigma_{i}=\frac{3}{2}\left(\mathbb{I}-\mathbf{P}\cdot\boldsymbol{\sigma}\right)$$
ולכן נקבל:
$$\frac{\Gamma}{3}\sum_{i}\left(\sigma_{i}\rho\sigma_{i}-\rho\right)=-\frac{2\Gamma}{3}{\bf P}\cdot\boldsymbol{\sigma}$$
כאשר המשוואה תהיה:
$${\frac{d\mathbf{P}}{d t}}\cdot\boldsymbol\sigma=-{\frac{4\Gamma}{3}}\mathbf{P}\cdot\boldsymbol\sigma\implies\frac{d{\bf P}}{d t}=-\frac{4{\Gamma}}{3}{\bf P}.$$
אם נניח כעת את התנאי התחלה \(P(t=0)=P_{0}\) נקבל:
$${\bf P}\left(t\right)={\bf P}_{0}e^{-\frac{4\Gamma}{3}t}$$
כך שאם \(t\ll 1\) נקבל:
$$\mathbf{P}\approx\mathbf{P}_{0}\left(1-{\frac{4\Gamma}{3}}d t\right)=\left(1-{\frac{4p}{3}}\right)\mathbf{P}_{0}$$

\end{example}
\begin{example}[אוסצילטור הרמוני קוונטי מרוסן]
ניתן לכתוב עם המילטוניאן מהצורה:
$$H_{\mathrm{{int}}}=g\sum_{k}\left(a^{\dagger}a_{k}+a_{k}^{\dagger}a\right)$$
אנו יודעים כי כאשר המערכת נמצאת במצב וואקום, היא לא יכולה לפלוט פוטון, ורק יכולה לקלוט פוטון. יהיה רק אופרטור קפיצה אחד. ולכן:
$$M=\sqrt{p}a$$
ולכן \(p=\Gamma \mathrm{d}t\). כלומר \(L=\sqrt{ \Gamma }a\). ניתן כעת לכתוב את משוואת המאסטר בתמונת שרדינגר:
$$\frac{d\rho}{d t}=-i\omega\left[a^{\dagger}a,\rho\right]+\Gamma\left(a\rho a^{\dagger}-\frac{1}{2}a^{\dagger}a\rho-\frac{1}{2}\rho a^{\dagger}a\right)$$
ניתן לעבור לתמונת היזנברג:
$${\frac{d a}{d t}}=i\omega[a^{\dagger}a,a]+\Gamma\sum_{i}\left(a^{\dagger}a a-{\frac{1}{2}}a^{\dagger}a a-{\frac{1}{2}}a a^{\dagger}a\right)$$
כאשר על ידי שימוש ב-\(\left[ a^{\dagger}a,a \right]=-a\) נקבל:
$${\frac{1}{2}}a^{\dagger}a a-{\frac{1}{2}}a a^{\dagger}a=-{\frac{1}{2}}\left[a,a^{\dagger}\right]a=-{\frac{1}{2}}a.$$
ולכן:
$${\frac{d a}{d t}}=-\left(i\omega+{\frac{\Gamma}{2}}\right)a\implies a\left(t\right)=e^{-\left(i\omega+\frac{F}{2}\right)t}a\left(0\right).$$
נשים לב כעת כי המצב עבור מצב התחלתי בבסיס המספר \(\ket{n}\) יהיה:
$$\langle a^{\dagger}\,a\rangle_{t=0}=n$$
כאשר אם נקדם בזמן נקבל:
$$\langle a^{\dagger}a\rangle_{t}=\langle n|a^{\dagger}(t)a(t)|n\rangle=e^{-\Gamma t}\langle a^{\dagger}a\rangle_{t=0}=n e^{-\Gamma t}$$
כיוון שהפוטציאל ההרמוני הוא פונקציאל ריבועי, נקבל כי התוכלת שלו מקיים את משוואות המילטון(משוואת אהרנפסט):
$$\frac{d}{d t}\left\langle X\right\rangle^{\mathrm{free}}=\frac{1}{m}\left\langle P\right\rangle^{\mathrm{free}},\qquad\frac{d}{d t}\left\langle P\right\rangle^{\mathrm{free}}=-m\omega^{2}\left\langle X\right\rangle^{\mathrm{free}}.$$
ונקבל:
$$\langle X\rangle_{t}=e^{-\frac{\Gamma_{t}}{2}}\langle X\rangle_{\mathrm{free}},\quad\langle P\rangle_{t}=e^{-\frac{\Gamma_{t}}{2}}\langle P\rangle_{\mathrm{free}}.$$
כאשר ממשפט האמסטר בתמונת שרדינגר עבור תנאי התחלה \(\rho(0)=\ket{\alpha}\bra{\alpha}\) נקבל:
$$\rho(t)=|\alpha e^{-(i\omega+\frac{\Gamma}{2})t}\rangle\langle\alpha e^{-(i\omega+\frac{\Gamma}{2})t}|$$

\end{example}
\begin{definition}[מצבי חתול  - Cat]
נסתכל על סופרבוזציה של שתי מצבים קוהרנטים:
$$|\psi\rangle=\frac{1}{\sqrt{C}}(|\alpha\rangle+|\beta\rangle)$$
כאשר \(C\) קבוע נקמול המתאים. המטריצת צפיפות המתאימה תהיה:
$$\rho(0)=\frac{1}{C}(|\alpha\rangle\langle\alpha|+|\alpha\rangle\langle\beta|+|\beta\rangle\langle\alpha|+|\beta\rangle\langle\beta|)$$

\end{definition}
\begin{proposition}
הקידום בזמן של \(\ket{\alpha}\bra{\alpha}\) ו-\(\ket{\beta}\bra{\beta}\) נתון על ידי:
\begin{gather*}|\alpha\rangle\langle\alpha|\mapsto|\alpha e^{-(i\omega+\frac{\Gamma}{2})t}\rangle\langle\alpha e^{-(i\omega+\frac{\Gamma}{2})t}|\\|\beta\rangle\langle\beta|\mapsto|\beta e^{-(i\omega+\frac{\Gamma}{2})t}\rangle\langle\beta e^{-(i\omega+\frac{\Gamma}{2})t}| 
\end{gather*}

\end{proposition}
\begin{proposition}
הגורם המעורב נתון על ידי:
$$|\alpha\rangle\langle\beta|\mapsto \langle\beta|\alpha\rangle^{1-e^{-\Gamma t}}|\alpha e^{-(i\omega+\frac{\Gamma}{2})t}\rangle\langle\beta e^{-(i\omega+\frac{\Gamma}{2})t}|$$

\end{proposition}
\begin{proof}
ננחש פתרון מהצורה:
$$|\alpha\rangle\langle\beta|\to f(t)|\alpha e^{-(i\omega+\frac{\Gamma}{2})t}\rangle\langle\beta e^{-(i\omega+\frac{\Gamma}{2})t}|$$
כאשר אם נמצב למשוואת המאסטר נקבל \(f(t)=\langle\beta|\alpha\rangle^{1-e^{-\Gamma t}}\).

\end{proof}
\begin{corollary}
עבור \(t\ll T_{\mathrm{decay}}\) ניתן לפתח:
$$1-e^{ -\Gamma t }\approx \Gamma t\implies f(t)\approx \left\lvert  \braket{ \beta | \alpha }   \right\rvert ^{\Gamma t}e^{ i\varphi\left( \beta,\alpha \right) }$$
כאשר הגורמים שלא באלכסון דועכים בצורה:
$$|\langle\beta|\alpha\rangle|^{\mathsf{T}t}=e^{-\frac{\mathsf{r}}{2}|\beta-\alpha|^{2}t}$$
והזמן דה-קוהרנטיות תהיה:
$$T_{\mathrm{decoherence}}={\frac{T_{\mathrm{decay}}}{|\beta-\alpha|^{2}}}$$

\end{corollary}
\begin{corollary}
עבור מערכות עם \(\left\lvert  \beta-\alpha  \right\rvert^{2}\gg 1\) הדה-קוהרנטיות הרבה יותר מהיר מהדעיכה. זבה זה יכול להסביר את האתגר של מערכות קוונטיות מקארוסקופיות.

\end{corollary}
\Chapter{שזירה}

\section{מצבי בל}

\begin{definition}[מצבי בל]
$$\begin{array}{c}{{|\phi^{+}\rangle=\frac{1}{\sqrt{2}}(|00\rangle+|11\rangle)}}\\ {{|\phi^{-}\rangle=\frac{1}{\sqrt{2}}(|00\rangle-|11\rangle)}}\\ {{|\psi^{+}\rangle=\frac{1}{\sqrt{2}}(|01\rangle+|10\rangle)}}\\ {{|\psi^{-}\rangle=\frac{1}{\sqrt{2}}(|01\rangle-|10\rangle)}}\end{array}$$

\end{definition}
\begin{proposition}
מצבי בל הם המצבים השזורים ביותר עבור מערכת של שתי רמות.

\end{proposition}
\begin{proof}
אנו יודעים כי המצב השזור ביותר הוא המצב שבו ההסתברות להיות בכל מצב יהיה שווה. נראה זאת בכל זאת עבור המצב \(\ket{\psi^{+}}\). נחשב את המטריצת צפיפות:
$$ ρ = |\psi^{+}⟩⟨\psi^{+}| = \frac{1}{2}(|00⟩⟨00| + |00⟩⟨11| + |11⟩⟨00| + |11⟩⟨11|) $$
כעת נעשה טרייס אוט למערכת \(B\) ונקבל:
$$\rho_{A}=\mathrm{Tr}_{B}\left( \rho \right)= \frac{1}{2}\ket{0} \bra{0}  +\frac{1}{2}\ket{1}\bra{1}   = \frac{1}{2}\mathbb{1} $$
למעשה זה מספיק כיוון שהראנו כי אנטרופיה מקסימלית מתקבלת כאשר \(\rho_{A}=\frac{1}{d}\mathbb{1}\) כאשר עבורנו \(d=2\). ניתן אך גם להראות מהאנטרופיה:
$$ S(ρ_A) = -\text{Tr}[ρ_A \log_2 ρ_A] $$
מכיוון ש-\(ρ_A = \frac{1}{2}\mathbb{1}\), כאשר הערכים העצמיים של \(\rho_{A}\) הם \(\frac{1}{2}\) ו-\(\frac{1}{2}\). לכן:
$$ S(ρ_A) = -\left( \frac{1}{2}\log_2\frac{1}{2} + \frac{1}{2}\log_2\frac{1}{2} \right) = -(-1) = 1 $$
כאשר זה \(\log_{2}2=1\) כמו שמתקבלת במקסימום שזירות. לחלופין ניתן ישר לשים לב כי המצב נמצא בפירוק שמידט שלו עם \(\lambda_{1}=\frac{1}{\sqrt{ 2 }}\) ו-\(\lambda_{2}=\frac{1}{\sqrt{ 2 }}\) ולכן ההסתרויות הם \(p_{0}=p_{2}=\frac{1}{2}\) וזה המצב האקראי ביותר - מקסימום שזירות.

\end{proof}
\begin{corollary}
המטריצה המצומצמת תהיה:
$$\rho_{A}=\mathrm{Tr}_{B}\left[\left|\mathrm{max}\right\rangle_{A B}\left\langle\mathrm{max}\right|_{A B}\right]=\frac{1}{2}\mathbb{1}_{A}$$

\end{corollary}
\begin{proposition}
האופרטורים האוניטרים אשר מעבירים בין מצבי הבל הם מטריצות פאולי.

\end{proposition}
\begin{proposition}
הפעלה של CNOT ואז אדמר יהפוך מצב בל למצב טהור. כיוון שאדמר פועל על כל תת מרחב בנפרד(ולמעשה רק משנה בסיס) לא משנה את אנטרופיית השזירה. לכן CNOT למעשה יכול לבטל את השזירה.

\end{proposition}
\section{שימושים של שזירה}

\begin{symbolize}
$$X\equiv \sigma_{x}\quad Y\equiv \sigma_{y}\quad Z\equiv \sigma_{z}$$

\end{symbolize}
\begin{proposition}[משפט הולבו - Holevo]
ניתן להעביר לכל היותר \(n\) ביטים קלאסים בעזרת \(n\) קיובטים כאשר המצבים אינם שזורים(כלומר מצבי מכפלה).

\end{proposition}
המצב משתנה כאשר המצבים שזורים.

\begin{proposition}[קידוד סופר דחוס - Superdense coding]
מאפשר להעביר 2 ביטים קלאסים בעזרת קיוביט אחד. פעל בצורה הבאה:

  \begin{enumerate}
    \item נניח כי לארנונה וברטלד יש שתי קיוביטים במצב שזור מקסימלית(מצב בל). לדוגמא: 
$$|\Phi^{+}\rangle=\frac{1}{\sqrt{2}}(|0\rangle_{A}\otimes|0\rangle_{B}+|1\rangle_{A}\otimes|1\rangle_{B})$$


    \item ארנונה יכול להפעיל אחד מ-4 אופרטורים שונים על הקיוביט שלה - \(\left\{\mathbb{1}\otimes\mathbb{1},X\otimes\mathbb{1},Y\otimes\mathbb{1},Z\otimes\mathbb{1}\right\}\). בהתאם לאופרטור המופעל נקבל את אחד המצבים הבאים: 
$$\left\{\left|\phi^{+}\right\rangle_{A B},\left|\psi^{+}\right\rangle_{A B},i\left|\psi^{-}\right\rangle_{A B},\left|\phi^{-}\right\rangle_{A B}\right\}$$


    \item ארנונה כעת יכולה להעביר לברטולד את הקיוביט שלה. לברטולד יש כעת גישה למצב הבל המלא. ולכן יכול למדוד אותו בבסיס של מצבי בל ולדעת באיזה מצב בל נמצא - וכיוון שיש 4 אפשריויות למעשה הועברו 2 ביטים קלאסים. 


  \end{enumerate}
\end{proposition}
\begin{remark}
נדרש איזושהיא מוסכמה על איזה מצב בל מייצג איזה ביט. לדוגמא אפשר להסכים
$$\left|\phi^{+}\right\rangle_{A B}\mapsto 01\quad \left|\psi^{+}\right\rangle_{A B}\mapsto 11\quad i\left|\psi^{-}\right\rangle_{A B} \mapsto 10\quad \left|\phi^{-}\right\rangle_{A B}\mapsto 00$$

\end{remark}
\begin{proposition}[טלפורטציה קוונטית]
דרך להעביר מצב קוונטי ממערכת \(A\) למערכת \(B\) בעזרת שתי ביטים קלאסים. מבצע בצורה הבאה:

  \begin{enumerate}
    \item נניח שלארנונה יש קיוביט מהצורה \(\left|Q\right\rangle_{a}=\alpha\left|0\right\rangle_{a}+\beta\left|1\right\rangle_{a}\). בנוסף לארנונה וברטולד וש מצב בל שזור משותף. עבור למשל \(\ket{\phi^{+}}\) נקבל כי המצב הכולל יהיה: 
$$\left|Q\right\rangle_{a}\otimes\left|\phi^{+}\right\rangle_{A B}=\frac{1}{\sqrt{2}}\left(\alpha\left|000\right\rangle+\alpha\left|011\right\rangle+\beta\left|100\right\rangle+\beta\left|111\right\rangle\right)$$


    \item לאחר קצת מניפולציה אלגברית ניתן לכתוב את המצב בצורה: 
$$\frac{1}{2}\left(\left|\phi^{+}\right\rangle_{a A}\otimes\mathbb{1}_{B}+\left|\psi^{+}\right\rangle_{a A}\otimes X_{B}-i\left|\psi^{-}\right\rangle_{a A}\otimes Y_{B}+\left|\phi^{-}\right\rangle_{a A}\otimes Z_{B}\right)\left|Q\right\rangle_{B}$$


    \item ארנונה מודדת את הקיוביט של המצב בל שלה עם המצב הנתון \(Q\) כאשר מדידה זו הורסת את המצב \(Q\). 


    \item בעזרת הקיוביט שמתקבל לארנונה היא יכולה להעביר שתי ביטים קלאסים לברטולד המייצגים איזה אופרטור הוא צריך להפעיל לפני שמודד את הקיוביט שלו בבסיס של מצבי בל. 


  \end{enumerate}
\end{proposition}
\begin{proof}
נראה את הפיתוח של שלב 3.
\begin{gather*} \frac{\alpha}{2}\left(\left|\phi^{+}\right\rangle_{aA}+\left|\phi^{-}\right\rangle_{aA}\right)\left|0\right\rangle_{B}+\frac{\alpha}{2}\left(\left|\psi^{+}\right\rangle_{aA}+\left|\psi^{-}\right\rangle_{aA}\right)\left|1\right\rangle_{B}+\\ \frac{\beta}{2}\left(\left|\psi^+\right\rangle_{aA}-\left|\psi^-\right\rangle_{aA}\right)\left|0\right\rangle_B+\frac{\beta}{2}\left(\left|\phi^+\right\rangle_{aA}-\left|\phi^-\right\rangle_{aA}\right)\left|0\right\rangle_B=\\ \frac{1}{2}\left|\phi^{+}\right\rangle_{aA}\left( \alpha\left|0\right\rangle_{B}+\beta\left|1\right\rangle_{B} \right)+\frac{1}{2}\left|\psi^{+}\right\rangle_{aA}\left( \alpha\left|1\right\rangle_{B}+\beta\left|0\right\rangle_{B} \right)+\\ \frac{1}{2}\left|\psi^{-}\right\rangle_{aA}\left( \alpha\left|1\right\rangle_{B}-\beta\left|0\right\rangle_{B} \right)+\frac{1}{2}\left|\phi^{-}\right\rangle_{aA}\left( \alpha\left|0\right\rangle_{B}-\beta\left|1\right\rangle_{B} \right)=\\ \frac{1}{2}\left(\left|\phi^{+}\right\rangle_{aA}\otimes\mathbb{1}_{B}+\left|\psi^{+}\right\rangle_{aA}\otimes X_{B}-i\left|\psi^{-}\right\rangle_{aA}\otimes Y_{B}+\left|\phi^{-}\right\rangle_{aA}\otimes Z_{B}\right)\left|Q\right\rangle_{B}  
\end{gather*}

\end{proof}
\begin{proposition}[Quantum Key Distribution]
  \begin{enumerate}
    \item נניח כי לארנונה וברטלד יש כמות גדולה של מצבי בל. כל אחד עם קיוביט יחיד. 


    \item נניח ש-\(A\) ו-\(B\) מודדים כל קיוביט בעזרת או \(X\) או \(Z\), בהסתברות \(\frac{1}{2}\).  


    \item אחרי מדידה הם מודיעים אחד לשני(בצורה קלאסית) מה הם קיבלנו. 


    \item נזרקים כל המדידות אשר בבסיסים שונים, כאשר המדידות באותו בסיס נשמרות. 


    \item ב 


  \end{enumerate}
\end{proposition}
\begin{proposition}
אם יש מאזין אלקטרה. ולכן המצב השזור יהיה נתון על ידי:
$$\ket{\Psi}_{ABE} = \alpha\ket{00}_{A}\ket{00}_{B}\ket{e_{00}}_{E}+\beta\ket{11}_{A}\ket{11}_{B}\ket{e_{11}}_{E}$$
כעת ארנונה וברטולד יכולים לוודא שלא היה האזנה על ידי הפעלה של \(X\otimes X\) ו-\(Z \otimes Z\). המצב \(\ket{\phi^{+}}_{AB}\) הוא מצב עצמי של האופרטורים האלו עם רקך עצמי 1.  המצב בל \(\ket{\phi^{+}}_{AB}\) הוא מצב עצמי של האופרטורים עם ערך עצמי 1. זה נכון אם"ם לא שזור עם הקיוביט של אלקטרה.

\end{proposition}
\begin{proof}
מתקיים:
\begin{gather*}\left( X \otimes X \right) \ket{\Psi}_{ABE} = \alpha\ket{11}_{A}\ket{11}_{B}\ket{e_{00}}_{E}+\beta\ket{00}_{A}\ket{00}_{B}\ket{e_{11}}_{E}  \\(Z \otimes Z) \ket{\Psi}_{ABE} = \alpha\ket{00}_{A}\ket{00}_{B}\ket{e_{00}}_{E}+\beta\ket{11}_{A}\ket{11}_{B}\ket{e_{11}}_{E}
\end{gather*}
כאשר נקבל כי זה מצב עם מי עם ערך עצמי 1 רק אם \(\ket{e_{00}}_E = \ket{e_{11}}_E\) אשר לא קורה אם יש מאזין \(E\). ולכן זה שובר את הסימטריה וניתן לדעת אם יש האזנה.

\end{proof}
\Chapter{מעגלים קלאסיים}

\section{חישוב}

\begin{definition}[מחרוזת]
איברים בקבוצה \(\{ 0,1 \}^{n}\). ניתן לכתוב את ה-\(n\) של מחרוזת \(x\) בצורה הבאה:
$$x=x_{n-1}x_{n-2}\cdot\cdot\cdot x_{2}x_{1}x_{0}$$
כאשר כל איבר \(x_{i}\) נקרא ביט.

\end{definition}
\begin{definition}[חישוב]
פונקציה מהצורה
$$f:\overbrace{ \{ 0,1 \}^{n} }^{ \text{input} }\to \overbrace{ \{ 0,1 \}^{m} }^{ \text{output} }$$
שזה שקול לקבוצה של \(m\) פונקציות בוליאניות מהצורה:
$$f:\{ 0,1 \}^{n}\to \{ 0,1 \}$$

\end{definition}
\begin{remark}
פונקציות מהצורה הזו נקראות פונקציות בוליאניות. קבוצות כל הפונקציות הבוליאניות מהצורה \(f:\mathbb{F} _{2}^{n}\to \mathbb{F} _{2}^{m}\) תהיה מגודל\(2^{2^{n}}\).

\end{remark}
\begin{example}
עבור \(n=4\) נקבל \(2^{2^{4}}=2^{16}=65536\) פונקציות בוליאניות אפשריות.
עבור \(n=5\) נקבל כבר \(2^{32}=4294967296\). 

\end{example}
\begin{definition}[קבוצת הביטים המתקבלים - Accepted bits]
$$\Sigma_{f}=\left\{x\in\left\{0,1\right\}^{n}|f\left(x\right)=1\right\}$$
זהו הקבוצה של כל הקלטים אשר הפונקציה תחזיר עבורם 1.

\end{definition}
\begin{symbolize}
נסמן את ה-\(n\) מחרוזות שמתקבלים על ידי \(f\) על ידי:
$$\Sigma=\left\{x^{(1)},x^{(2)},x^{(3)},\ldots\right\}$$

\end{symbolize}
\begin{definition}[קבוצת הביטים הנדחים - Rejected bits]
$$\overline{{{\Sigma}}}_{f}=\left\{x\in\left\{0,1\right\}^{n}|f\left(x\right)=0\right\}$$
קבוצת כל הקלטים אשר הפונקציה תחזיר עבורם \(0\).

\end{definition}
\begin{symbolize}
כמו שאמרנו ניתן לפרק פונקציה \(f:{\{ 0,1 \}^{n}\to\{ 0,1 \}^{m}}\) ל-\(m\) פונקציות \(f^{(a)}:\{ 0,1 \}^{n}\to \{ 0,1 \}^{m}\) כאשר \(1\leq a \leq m\).

\end{symbolize}
\begin{proposition}[הפעולות הבוליאניות גם, או ולא]
ניתן להגדיר בעזרת פונקציות בוליאניות את הפעולות הבאות:

  \begin{enumerate}
    \item ניתן להביע את הפעולה הלוגית "וגם"(AND) על ידי: 
$$a\land b = a\cdot b$$


    \item ניתן להביע את הפעולה הלוגית "לא"(NOT) על ידי: 
$$\lnot a = 1- a$$


    \item ניתן להביע את הפעולה הלוגית "או"(OR) על ידי: 
$$a\lor b = a+b - a\cdot b$$


  \end{enumerate}
\end{proposition}
\begin{definition}[קבוצת שערים]
קבוצה של פעולות בוליאניות. כל איבר בקבוצה נקרא שער.

\end{definition}
\begin{definition}[מעגל/חישוב]
סדרה סופית של פעולות בינאריות המופעלות על מחרוזת של ביטים. למעשה שקול לרכיב \(f^{(a)}(x)\) של פונקציה בינארית.

\end{definition}
\begin{definition}[קבוצת שערים אוניברסליים]
קבוצה של שערים שניתן בעזרתה להקבל כל פונקציה בוליאנית.

\end{definition}
\begin{example}
הקבוצת שערים:
$$\left\{  \text{And, Not, Or, Input}  \right\}$$
כאשר Input זה הפעולה של להכין ביט קבוע מהוות קבוצת שערים אוניברסליים.

\end{example}
\begin{proposition}
ניתן לתאר מעגל על ידי גרף מכוון אי-ציקלי(כלומר ללא לולאות) כאשר כל קודקוד של הגרף הוא שער.

\end{proposition}
\begin{definition}[משפחות של פונקציות בוליאניות]
קבוצה של פונקציות בוליאניות אשר פועלת על קלט עם גודל משתנה.  מוסמן:
$$f:\{0,1\}^{*}\to\{0,1\}$$

\end{definition}
\begin{definition}[שפה]
הקבוצה של המחרוזות אשר מתקבלות על ידי המשפחת פונקציות:
$$L=\{x\in\{0,1\}^{*}:f(x)=1\}$$

\end{definition}
\begin{definition}[בעיות הכרעה]
בעיה אשר יש לה תשובה של כן או לא.

\end{definition}
\section{מחלקות סיבוכיות}

ניתן לכמת כמה בעיה היא קשה לפתור על ידי כמה משאבים היא דורשת כדי לפתור בעיה עם גודל קלט \(n\).

\begin{proposition}
דרך אחת למדוד את היעילות של משפחת מעגלים היא על ידי איך שהכמות שערים גדלה כאשר מגדלים את גודל הקלט. 

\end{proposition}
\begin{definition}[מחלקת סיבוכיות]
כל המשפחות מעגלים אשר קצב גידול השערים שלהם גדל לפי תנאי מסויים נמצאים באותה מחלקת סיבוכיות.

\end{definition}
\begin{remark}
לחובבי תורת הקבוצות, זה אכן לא קבוצה, ולכן מתאים לקרא לזה "מחלקה".

\end{remark}
\begin{definition}[משפחת מעגלים פולינומיאליים]
משפחת מעגלים אשר גדל לכל היותר באיזושהי חזקה של גודל הקלט \(n\).

\end{definition}
\begin{definition}[מחלקת סיבוכיות P]
מחלקת הסיבוכיות של כל המשפחות מעגלים הפולינומיאליות.

\end{definition}
\begin{definition}[בעיה קלה וקשה]
בעיה אשר נפתרת על ידי משפחת מעגלים פולונימיאלים נקראת בעיה קלה. כל יתר הבעיות נקראות קשות.

\end{definition}
\begin{definition}[מחלקת הסיבוכיות NP]
מחלקת הסיבוכיות של כל המשפחות מעגלים אשר ניתן לוודא את הפיתרון שלהם בזמן פולינומיאלי.

\end{definition}
\section{חישוב אקראי}

\begin{definition}[חישוב שימושי]
אנחנו אומרים שחישוב אקראי נחשב שימושי אם קיים \(\delta>0\) בלתי תלוי בגודל הקלט אשר עבורו ההסתברות להיות במצב הנכון מקיים:
$$\mathrm{Prob}\left( \text{right} \right)\geq \frac{1}{2}+\delta$$

\end{definition}
\begin{corollary}
עבור חישוב שימושי ביצועים חוזרים של המדידה ולקיחת הערך שהופיע הכי הרבה פעמים תגדיל את הסיכיים.

\end{corollary}
\begin{proposition}[חסם צ'רנוף Chernoff]
עבור סדרה של \(N\) מדידות ההסתברות לטעות תהיה לכל היותר:
$$\mathrm{Prob}\,(\mathrm{wrong})\leq e^{-2N\delta^{2}}$$

\end{proposition}
\begin{proof}
כאשר יש שתי מדידות עבור \(N\) אופציות נקבל \(2^{N}\) סדרות שונות של אפשרויות שהיינו יכולים לקבל. ההסתברות לקבל סדרה עם \(N_{\omega}> \frac{N}{2}\) נקבל:
$${\left({\frac{1}{2}}-\delta\right)^{N_{W}}\left({\frac{1}{2}}+\delta\right)^{N-N_{W}}}{\!\!\!\!\leq\left({\frac{1}{2}}-\delta\right)^{N/2}\!\left({\frac{1}{2}}+\delta\right)^{N/2}}\!\!\!\!=\frac{(1-4\delta^{2})^{N/2}}{2^{N}}\!\leq\!\frac{(e^{-4\delta^{2}})^{N/2}}{2^{N}}$$
אנו יודעים כי הבחירה של הרוב תביא את הפתרון. לכן נכפיל ב-\(2^{N}\), נפתח את המכנה ונקבל את החסם הנקרא Chernoff bound:
כאשר אם מניחים הסתברות שגיאה מקסימלית \(\varepsilon\):
$$N\geq{\frac{1}{2\delta^{2}}}\ln\left({\frac{1}{\epsilon}}\right)$$

\end{proof}
\begin{remark}
אוהבים לבחור \(\delta=\frac{1}{6}\). 

\end{remark}
\begin{definition}[מחלקת הסיבוכיות BPP]
המחלקת סיבוכיות של כל המשפחות מעגלים עם קצב גידול היסתברותי חסום על ידי פולינום.

\end{definition}
\begin{corollary}
המחלקת סיבוכיות \(P\) מוכלת במחלקת סיבוכיות \(BPP\) כיוון שדטרמניזם זה מקרה פרטי של סיבוכיות.

\end{corollary}
\begin{remark}
אין לזה הוכחה אבל מאמינים כי יש שיוויון, ומתקיים \(BPP=P\). כלומר אקראיות לא משנה את היכולת החישובית.

\end{remark}
\section{מעגלים הפיכים}

\begin{proposition}[עקרון לנדאור]
מחיקה של ביט יוצר עבודה כיוון שמקטין את האי וודאות ולכן משנה את האנטרופיה

\end{proposition}
\begin{corollary}
שערים אי הפיכים דורשים עבודה.

\end{corollary}
\begin{proposition}
פונקציה בינארית מהצורה \(f:\{ 0,1 \}^{n}\to\{ 0,1 \}^{n}\) היא הפיכה אם"ם היא תמורה.

\end{proposition}
\begin{corollary}
עבור \(n\) ביטים יש \(2^{n}\) מחרוזות אפשריים.
באופן כללי יש \((2^{n})^{2^{n}}\) פונקציות אפשריות(כל האפשריות של לשלוח כל אחד מה-\(2^{n}\) מהמחרוזות קלט ל-\(2^{n}\) המחרוזות פלט). כאשר מתוכם יש רק \((2^{n})!\approx \left( \frac{2^{n}}{e} \right)^{2^{n}}\) פונקציות תמורה(=פונקציות הפיכות אפשריות). לכן היחס של הפונקציות ההפיכות יהיה:
$$\left({\frac{2^{n}}{e}}\right)^{2^{n}}/\left(2^{2^{n}}\right)^{n}=e^{-2^{n}}$$
שזה חלק די קטן.

\end{corollary}
\begin{remark}
נשים לב כי שערים כמו AND, OR הם לא הפיכים כיוון שפונקציה מהצורה \(f:\{ 0,1 \}^{2}\to\{ 0,1 \}\). כלומר לא ניתן לשחזר את המידע. למעשה כל פונקציה מהצורה \(f:\{ 0,1 \}^{n}\to \{ 0,1 \}\) לא תהיה הפיכה עבור \(n>1\).

\end{remark}
המטרה שלנו כעת היא למצוא מערכת אונברסלית הפיכה של שערים.

\begin{definition}[XOR]
זוהי פונקציה המוגדרת על ידי:
$$a\oplus b=a+b-2a\cdot b$$
אומנם זו לא פונקציה הפיכה אבל היא חשובה כי ניתן לבנות בעזרתה פונקציות הפיכות.

\end{definition}
\begin{definition}[CNOT]
פונקציה \(f:\{ 0,1 \}^{2}\to\{ 0,1 \}^{2}\) המוגדרת על ידי:
$$\operatorname{CNOT}\left(x,y\right)=\left(x,y\oplus x\right)$$
זה הופך את הביט השני אם הביט ראשון הוא 1. אבל לא עושה כלום אם הביט ראשון הוא אפס. ניתן להציג בתרשים:

 Created with Inkscape (http://www.inkscape.org/) \includegraphics[width=0.8\textwidth]{diagrams/svg_2.svg}
\end{definition}
\begin{symbolize}
לעיתים נסמן ב-\(\Lambda\left( \mathbf{X} \right)\) את השער CNOT. באופן כללי נסמן ב-\(\Lambda(G)\) את הפעולה של להפעיל את \(G\) על המטרה בהתאם לערך של ביט נשלט. כלומר \(G\) פועל אם ה-control bit הוא 1 ולא פועל אם ה-control bit הוא אפס. במקרה של CNOT אז זה יהיה מופעל על אופרטור פאולי \(\mathbf{X}\).

\end{symbolize}
\begin{proposition}[החלפת ביט]
ניתן להחליף ביט עם המעגל הבא:

 Created with Inkscape (http://www.inkscape.org/) \includegraphics[width=0.8\textwidth]{diagrams/svg_3.svg}
\end{proposition}
\begin{proposition}
כל שער של שתי ביטים הפיכים יהיה העתקה אפיינית, כלומר פונקציה מהצורה:
$${\binom{x}{y}}\mapsto{\binom{x^{\prime}}{y^{\prime}}}=M{\binom{x}{y}}+{\binom{a}{b}}$$
כאשר הזוג \(\begin{pmatrix}a\\b\end{pmatrix}\) יכול לקבל כל אחד מה-4 ערכים האפשריים ו-\(M\) יהיה אחד מששת המטריצות ה-\(2\times 2\) הפיכות עם ערכים בינארים:
$$M=\left(\begin{array}{l l}{{1}}&{{0}}\\ {{0}}&{{1}}\end{array}\right),\left(\begin{array}{l l}{{0}}&{{1}}\\ {{1}}&{{0}}\end{array}\right),\left(\begin{array}{l l}{{1}}&{{1}}\\ {{0}}&{{1}}\end{array}\right),\left(\begin{array}{c c}{{1}}&{{0}}\\ {{1}}&{{1}}\end{array}\right),\left(\begin{array}{c c}{{0}}&{{1}}\\ {{1}}&{{1}}\end{array}\right),\left(\begin{array}{c c}{{1}}&{{1}}\\ {{1}}&{{0}}\end{array}\right)$$
וכל חיבור הוא מודולו 2(כלומר אנחנו בשדה \(\mathbb{F} _2\)).

\end{proposition}
\begin{corollary}
כיוון שהעתקות לינאריות סגורות תחת הרכבה, כל מעגל תהיה פונקציה לינארית. כיוון שקיימים פונקציות בינאריות לא לינאריות, לא ניתן לקבל מערכת הפיכה שלמה עם שערים של שתי ביטים.

\end{corollary}
\begin{definition}[השער \(Toffoli\)]
שער הפועל על שלושה ביטים. מוגדר על ידי:
$$\Lambda^{2}(X):(x,y,z)\longrightarrow(x,y,z\oplus x y)$$
לעיתים נקרא controlled-controlled-NOT. שער זה הופך את הביט האחרון רק אם שתי הביטים הראשונים הם 1. ניתן לתאר בעזרת תרשים:

 Created with Inkscape (http://www.inkscape.org/) \includegraphics[width=0.8\textwidth]{diagrams/svg_4.svg}
\end{definition}
\begin{proposition}
ניתן לשחזר את השערים AND,OR,NOT ולכן Toffoli לבדו יוצר קבוצת שערים אוניברסילים.

\end{proposition}
\begin{corollary}
כיוון שיש לנו קבוצת שערים הפיכים אוניברסלים, ניתן לייצר כל מעגל בעזרת Toffoli ולכן ניתן לייצר כל מעגל ללא איבוד אנרגיה שנובע מעקרון לאנדאור.

\end{corollary}
\begin{remark}
כיוון שאנחנו לא נפטרים משום ביט, זה מעלה את השאלה של מה לעשות עם הביטים שאנחנו לא צריכים. הפתרון הפשוט זה אחרי שמסיימים את החישוב, מעתיקים את התוצאה, ואז מריצים את המעגל בכיוון ההפוך, כך שיחזור לתנאי ההתחלה. נשים לב כי זה לא יעבור במקרה הקוונטי בכלל עקרון חוסר שכפול.

\end{remark}
\begin{summary}
  \begin{itemize}
    \item עקרון לנדאור אומר כי מחיקה של ביט דורש עבודה, לכן נרצה לחשב בעזרת פונקציות הפיכות - אשר אינם דורשות עבודה.
    \item לא קיים קבוצה אוניברסלית של שערים הפיכים של שתי ביטים.
    \item קיים קבוצת שערים אונבירסלים הפיכים על שלוש ביטים, וזה יכיל השער אחד - Toffoli.
  \end{itemize}
\end{summary}
\Chapter{מעגלים קוונטים}

\section{שערים קוונטים}

נרצה לבנות מודל מתמטי של מעגלים קוונטים באופן דומה לאיך שבנינו את המודל המתמטי של מעגלים קלאסים.

\begin{proposition}[שער קוונטי]
זה המקביל של שער קלאסי. במקום לבצע אבל פעולה בינארית יפעיל למשעשה אופרטור אוניטרי שפועל על מספר קבוע של קיוביטים. 

\end{proposition}
\begin{definition}[דרישות של מחשב קוונטי]
  \begin{enumerate}
    \item המרחב ההילברט יהיה מרחב מכפלה של מרחבי קיוביט. 


    \item ניתן לאפס אותו. כלומר ניתן להביאו למצב \(\ket{000\dots 0}\). דרך לעשות את זה למשל זה לקרר לטמפרטורות קרובות ל-0 המוחלט. 


    \item כמות השערים הקוונטים הם סופיים. 


    \item המעגל נבנה ומנוהל על ידי מחשב קלאסי. 


    \item קל לקריאה בבסיס החישובי(computational basis), אשר ניתן למדד בסוף התהליך. 


  \end{enumerate}
\end{definition}
\begin{remark}
ניתן לקיים את המודל הזה עם מחשב קלאסי, אבל יהיה לא יעיל. וכן המקרה הקוונטי יהיה הכללה של המקרה הקלאסי(כי הרי ניתן להגביל את האופרטורים האוניטרים לאופרטורי תמורה).

\end{remark}
\begin{definition}[מחלקת P]
משפחת מעגלים אשר גדל לכל היותר באיזשהי חזקה של גודל הקלט \(n\).

\end{definition}
\begin{definition}[מחלקת NP]
כל המשפחות מעגלים אשר ניתן לוודא את הפיתרון שלהם בזמן פולינומיאלי.

\end{definition}
\begin{definition}[מחלקת BPP]
כל המשפחות מעגלים עם קצב גידול היסתברותי חסום על ידי פולינום.

\end{definition}
\begin{definition}[מחלקת MA]
כל המשפחות מעגלים אשר ניתן לוודא את הפיתרון שלהם בצורה הסתברותית בזמן פולינומיאלי.

\end{definition}
\begin{definition}[מחלקת BQP]
כל המשפחות מעגלים קוונטים עם קצב גידול היסתברותי פולינומיאלי(מקביל ל-BPP).

\end{definition}
\begin{definition}[מחלקת QMA]
כל המשפחות המעגלים הקוונטים אשר ניתן לוודא את הפתרון בזמן פולינומיאלי(מקביל ל-MA).

\end{definition}
\begin{proposition}
$$\mathsf{P}\subseteq{\mathsf{BPP}}\subseteq{\mathsf{BQP}}\subseteq{\mathsf{QMA}}.$$

\end{proposition}
\begin{proposition}
$${\mathsf{P}}\subseteq{\mathsf{N P}}\subseteq{\mathsf{M A}}\subseteq{\mathsf{Q M A}}$$

\end{proposition}
\begin{definition}[שער הדמאר]
$$H:|x\rangle\to\frac{1}{\sqrt{2}}\sum_{y}(-1)^{x y}|y\rangle$$
כלומר:
$$\ket{0} \mapsto \frac{1}{\sqrt{ 2 }}\left( \ket{0} +\ket{1}  \right)\qquad \ket{1} \mapsto \frac{1}{\sqrt{ 2 }}\left( \ket{0} -\ket{1}  \right)$$
כאשר המטריצה האונטירית המתאמיה ל-\(H\) תהיה:
$$H:\left(\begin{array}{c  c}{{\frac{1}{\sqrt{2}}}}&{{\frac{1}{\sqrt{2}}}}\\ {{\frac{1}{\sqrt{2}}}}&{{-\frac{1}{\sqrt{2}}}}\end{array}\right)$$

\end{definition}
\begin{definition}[control ו-target]
שער בו רכיב אחד נקרא ה-control והרכיב השני נקרא ה-target. כאשר ה-control מקבל \(\ket{0}\) אז מבצע את האופרטור \(U\), אחרת לא עושה כלום. מפורשות ניתן לכתוב:
$$\Lambda (U)=\ket{0} \bra{0} \otimes \mathbb{1} +\ket{1} \bra{1} \otimes U $$

\end{definition}
\begin{proposition}
ניתן לכתוב אופרטור נשלט על ידי:
$$\Lambda\left(U\right)=P_{A}\otimes\left|0_{u}\right\rangle\left\langle0_{u}\right|_{B}+Q_{A}\otimes\left|1_{u}\right\rangle\left\langle1_{u}\right|_{B}$$
כאשר
$$P=\left|0\right\rangle\left\langle0\right|+e^{i\phi_{0}}\left|1\right\rangle\left\langle1\right|\qquad Q=\left|0\right\rangle\left\langle0\right|+e^{i\phi_{1}}\left|1\right\rangle\left\langle1\right|$$

\end{proposition}
\begin{proof}
אנו יודעים כי:
$$\Lambda\left(U\right)=\left|0\right\rangle\left\langle0\right|_{A}\otimes\mathbb{1}_{B}+\left|1\right\rangle\left\langle1\right|_{A}\otimes U_{B}$$
כיוון ש-\(U\) אוניטרי קיים בסיס אשר מלכסן אותו, ולכן ניתן לכתוב:
$$U=e^{i\phi_{0}}\left|0_{u}\right\rangle\left\langle0_{u}\right|+e^{i\phi_{1}}\left|1_{u}\right\rangle\left\langle1_{u}\right|$$
כאשר אם נציב את \(U\) בתוך ה controlled \(U\) כאשר נכתוב את היחידה במפורש נקבל:
$$\Lambda\left(U\right)=\left|0\right\rangle\left\langle0\right|_{A}\otimes\left[\left|0_{u}\right\rangle\left\langle0_{u}\right|+\left|1_{u}\right\rangle\left\langle1_{u}\right|\right]_{B}+\left|1\right\rangle\left\langle1\right|_{A}\otimes\left[e^{i\phi_{0}}\left|0_{u}\right\rangle\left\langle0_{u}\right|+e^{i\phi_{1}}\left|1_{u}\right\rangle\left\langle1_{u}\right|\right]_{B}$$
אם נאחד את ההטלות על \(B\) נקבל:
$$\Lambda\left(U\right)=\underbrace{ \left[\left|0\right\rangle\left\langle0\right|+e^{i\phi_{0}}\left|1\right\rangle\left\langle1\right|\right]_{A} }_{ P }\otimes\left|0_{u}\right\rangle\left\langle0_{u}\right|_{B}+\underbrace{ \left[\left|0\right\rangle\left\langle0\right|+e^{i\phi_{1}}\left|1\right\rangle\left\langle1\right|\right]_{A} }_{ Q }\otimes\left|1_{u}\right\rangle\left\langle1_{u}\right|_{B}$$

\end{proof}
\begin{corollary}
עבור \(P=\mathbb{1}\) ניתן לחשוב על זה בתור אופרטור על \(A\) אשר נשלט על ידי \(B\).

\end{corollary}
\begin{corollary}
ניתן כעת לעבור לבסיס החישובי על ידי מעבר בסיס כלומר על ידי הפעלת אופרטור \(W\) אשר מקיים:
$$W|0_{u}\rangle=\left|0\right\rangle \qquad W|1_{u}\rangle=|1\rangle$$

\end{corollary}
\begin{proposition}
עבור \(P=\mathbb{1}\) ו \(Q=U\) נקבל כי בפועל מחליף את ה-control ואת ה-target(למעשה \(Q\) יהיה אופרטור זהה במערכת \(A\)).

\end{proposition}
\begin{corollary}
אם האופרטור \(U\) אלכסוני בבסיס החישובי ניתן להחליף את ה-control ואת ה-target.

\end{corollary}
\section{מעגלים קוונטים}

\section{דיוק של מעגל קוונטי}

\begin{symbolize}
עבור נתונים \(N\) שערים קוונטים, נסמן ב-\(\ket{\varphi_{i}}\) את הרכיב שהתקבל לאחר הפעלה של \(i\) שערים. כאשר השער יהיה שקול להפעלה \(U_{i}\)
כלומר עבור הפעלה של \(T\) שערים נקבל:
$$|\varphi_{T}\rangle=U_{T}U_{T-1}\ldots U_{2}U_{1}|\varphi_{0}\rangle$$

\end{symbolize}
\begin{proposition}
אם נפעיל שער \(\tilde{U_{1}}\) אשר לא מפעיל את \(U_{1}\) במדוייק, ניתן לכתוב: 
$$ \widetilde{U}\ket{\varphi_{0}}=U_{1}\ket{\varphi_{0}} +\ket{E_{1}} =\ket{\varphi_{1}}+\ket{E_{1}}\equiv \ket{\tilde{\varphi_{1}}} $$
כלומר אופרטור אשר מחזיר את הרכיב של הפעלה של \(U_{1}\)  ביחד עם רכיב שגיאה:
$$|E_{1}\rangle=( \widetilde{U}_{1}-U_{1})|\varphi_{0}\rangle$$

\end{proposition}
\begin{corollary}
עבור \(N\) שערים נקבל:
$$\left\lVert  |\hat{\varphi}_{T}\rangle-|\varphi_{T}\rangle  \right\rVert \leq~\left\lVert  |E_{T}\rangle  \right\rVert +\left\lVert  |E_{T-1}\rangle  \right\rVert +\dots+\lVert \ket{E_{1}} \rVert $$

\end{corollary}
\begin{proof}
עבור \(\ket{\varphi_{2}}\) נקבל:
$$|\tilde{\varphi}_{2}\rangle= \widetilde{U}_{2}|\tilde{\varphi}_{1}\rangle=|\varphi_{2}\rangle+|E_{2}\rangle+ \widetilde{U}_{2}|E_{1}\rangle,$$
וכן עבור \(\ket{\varphi_{3}}\) נקבל:
$$|\tilde{\varphi}_{3}\rangle= \widetilde{U}_{3}|\tilde{\varphi}_{2}\rangle=|\varphi_{3}\rangle+|E_{3}\rangle+ \widetilde{U}_{3}|E_{2}\rangle+ \widetilde{U}_{3} \widetilde{U}_{2}|E_{1}\rangle$$
כך שאחרי \(T\) שלבים נקבל:
$$|\tilde{\varphi}_{T}\rangle=|\varphi_{T}\rangle+|E_{T}\rangle+ \widetilde{U}_{T}|E_{T-1}\rangle+ \widetilde{U}_{T} \widetilde{U}_{T-1}|E_{T-2}\rangle+\cdot\cdot\cdot+\widetilde{U}_{T}\widetilde{U}_{T-1}\cdot\cdot\cdot\widetilde{U}_{2}|E_{1}\rangle$$
ואם נעביר את \(\ket{\varphi_{T}}\) אגף וניקח נורמה נקבל את המסקנה.

\end{proof}
\begin{proposition}
$$\left\lVert  |E_{t}\rangle  \right\rVert = \left\lVert  \left(\widetilde{U}_{t}-U_{t}\right)|\varphi_{t-1}\rangle  \right\rVert  \leq \lVert \widetilde{U}_{t}-U_{t} \rVert _{\mathrm{sup}}$$

\end{proposition}
\begin{corollary}
אם \(\parallel\widetilde{U}_{t}-U_{t}\parallel_{\mathrm{sup}}\ \ \leq\ \varepsilon\) אז לאחר שאנחנו מפעילים \(T\) שערים קוונטים נקבל:
$$\lVert |\tilde{\varphi}_{T}\rangle-|\varphi_{T}\rangle \rVert  \leq\ T\varepsilon$$

\end{corollary}
השאלה שעולה היא איזה דיוק של מדידות היא "מספיק טובה".

\begin{proposition}
אם נבצע מדידה בסוף החישוב, נקבל:
$$p(a)=|\left\langle  a|\varphi_{T} \right\rangle|^{2}\qquad \tilde{p}(a)=\left\lvert  \braket{ a | \tilde{\varphi} }   \right\rvert ^{2}$$

\end{proposition}
\begin{proposition}
הנורמה ב-\(L^{1}\) מקיימת:
$$\left\lVert  \ket{\psi} -\ket{\tilde{\psi}}   \right\rVert _{1}\leq2\left\|\left|\psi\right\rangle-\left|\widetilde{\psi}\right\rangle\right\|$$

\end{proposition}
\begin{corollary}
$$\frac{1}{2}\|\tilde{p}-p\|_{1}=\frac{1}{2}\sum_{a}|\tilde{p}(a)-p(a)|~\leq~\|~|\tilde{\varphi}_{T}\rangle-|\varphi_{T}\rangle~\|~\leq~T\varepsilon$$
ולכן אם אנחנו משאירים את \(T\varepsilon\) קבוע כש-\(T\) גדל נקבל כי החסם על השגיאה גם ישאר קבועה.

\end{corollary}
\begin{corollary}
אנו יודעים כי אם נרצה לפתור בעיית החלטה בעזרת אלגוריתם הסתברותי, נדרש כי אחוז ההצלחה של \(\widetilde{U}\) יהיה \(\frac{1}{2}+\tilde{\delta}\) כאשר \(\tilde{\delta}>0\). אם המערכת האידיאלית מכילה \(T\) שערים ועם הסתברות הצלחה של \(\frac{1}{2}+\delta\) כאשר \(\delta> 0\) נקבל מהמסקנה הקודמת כי נדרש:
$$\varepsilon<\frac{\delta}{T}$$

\end{corollary}
\section{אורך של מעגל קוונטי}

נרצה למצוא חסם על אורך המעגל \(T\) כדי ששגיאה (\(\left\lVert  \tilde{U}  - U\right\rVert_{\sup}\)) תהיה קטנה מ-\(\varepsilon\).
ראשית נרצה להגיע לכל האופרטורים האונטרים שמקבלים \(n\) קיוביטים.

\begin{proposition}
כל האופרטורים האוניטרים שמתקבלים מ-\(n\) קיוביטים יהיו בחבורה לי הרציפה \(U(2^{n})\)(כאשר הרציפה אומר כי הפעלה של אופרטורים "קרובים" לאופרטור אוניטרי \(U\) על איבר יהיה "קרוב" לתוצאה של הפעלה של אופרטור המתקבל מהפעלה על ידי \(U\)) חבורה זו נפרשת על ידי \(2^{2n}\) פרמטרים ממשיים.

\end{proposition}
אם נרצה שלמעגל שלנו יהיה \(T\) שערים, מספר המעגלים שניתן ליישם יהיה סופי.

\begin{proposition}
נניח כי מספר המעגלים שניתן ליישם עם \(T\) שערים שפועלים על \(n\) קיוביטים חסום על ידי:
$$N_{T}\leq\left[\mathrm{poly}\left(n\right)\right]^{T}$$
כלומר חסום על ידי איזושהו פולינום של \(n\) בחזקת מספר השערים.

\end{proposition}
\begin{corollary}
כמות האוניטרים שניתן לממש עם \(T\) שערים היא סופית.

\end{corollary}
כעת נרצה להראות כי ניתן בעזרת כמות סופית של שערים לקבל את כל האוניטריות עד כדי שגיאה.
ניתן לייצג את החבורה \(U(2^{n})\) בעזרת איזושהי יריעה רציפה, כאשר קבוצת הנקודות שניתן להגיע עליה עם \(\left\{  \widetilde{U}  \right\}\) עם אורך \(T\) סופי יהיה כדור \(\varepsilon\) מסביב ל-\(\widetilde{U}\).

 Created with Inkscape (http://www.inkscape.org/) \includegraphics[width=0.8\textwidth]{diagrams/svg_5.svg}
\begin{corollary}
ניתן לקבל את כל האוניטריות אם היריעה של אוניטריות תהיה מכוסה על ידי כדורי \(\varepsilon\).

\end{corollary}
\begin{proposition}
חסם תחתון עבור כמות הכדורים יהיה:
$$N_{\mathrm{balls}}\geq{\frac{\mathrm{Vol}\left(U\left(N\right)\right)}{\mathrm{Vol}\left(\epsilon{\mathrm{-ball}}\right)}}$$

\end{proposition}
\begin{lemma}
מטיעונים גאומטרים, היחס בין השטחים שווה ליחס בין הנפחים.

\end{lemma}
\begin{corollary}
$$N_{\mathrm{balls}}\,\geq\,\left({\frac{R}{\epsilon}}\right)^{2^{2n}}$$

\end{corollary}
\begin{proposition}
כדי שכל המעגלים מגודל \(T\) יהיו ניתנים למימוש בקירוב עד כדי \(\varepsilon\) נדרש:
$$N_{T}\geq N_{\mathrm{balls}}$$
כיוון שכל כדור יכול לממש לפחות שער אחד.

\end{proposition}
\begin{corollary}
מהטיעונים הקודמים נקבל כי חסם תחתון על גודל המעגל יהיה:
$$T\geq2^{2n}\frac{\log\left(R/\epsilon\right)}{\log\left[\mathrm{poly}\left(n\right)\right]}$$
כלומר גודל המעגל צריך לגדול באופן אקספוננציאלי.

\end{corollary}
\begin{summary}
  \begin{itemize}
    \item שער קוונטי מיוצג על ידי אופרטור אוניטרי. ולכן ניתן לכתוב הפעלה של \(T\) שערים קוונטים על ידי:
$$|\varphi_{T}\rangle=U_{T}U_{T-1}\ldots U_{2}U_{1}|\varphi_{0}\rangle$$
    \item אופרטור \(\tilde{U}\) נקרא אופרטור מקורב ו-\(|E\rangle=( \widetilde{U}-U)|\varphi_{0}\rangle\) יהיה השגיאה של \(\tilde{U}\).
    \item השגיאה של הפעלת \(T\) אופרטורים תקיים:
$$\left\lVert  |\hat{\varphi}_{T}\rangle-|\varphi_{T}\rangle  \right\rVert \leq\left\lVert  |E_{T}\rangle  \right\rVert +\left\lVert  |E_{T-1}\rangle  \right\rVert +\dots+\lVert \ket{E_{1}} \rVert $$
    \item אם קיים \(\varepsilon> 0\) כך ש-\(\parallel\widetilde{U}_{t}-U_{t}\parallel_{\mathrm{sup}} \leq\ \varepsilon\) נקבל לאחר הפעלת \(T\) שערים:
$$\lVert |\tilde{\varphi}_{T}\rangle-|\varphi_{T}\rangle \rVert  \leq\ T\varepsilon$$
    \item ההתסברות \(\tilde{p}\) להיות במצב \(\ket{\tilde{\varphi}}\) וההסתברות \(p\) להיות במצב \(\ket{\varphi}\) יקיימו:
$$\frac{1}{2}\|\tilde{p}-p\|_{1}\leq~T\varepsilon$$
    \item חסם למספר השערים במעגל \(T\) כדי שהשגיאה תהיה קטנה מ-\(\varepsilon\) יהיה:
$$T\geq2^{2n}\frac{\log\left(R/\epsilon\right)}{\log\left[\mathrm{poly}\left(n\right)\right]}$$
  \end{itemize}
\end{summary}
\section{אוניברסליות של מעגלים קוונטים}

\begin{definition}[אוניברסליות מדוייקת]
כל שתי שערי קיוביט הם אוניברסלים. כלומר כל \(\mathcal{U}\in U(N)\) (כאשר \(N=2^{n}\)) יכול להיות מושג על ידי מכפלה של שתי קיוביט אוניטרים.

\end{definition}
נרצה להראות שקיים אוניברסליות מדוייקת. נעשה בשתי שלבים:

\begin{enumerate}
  \item כל \(\mathcal{U}\in U(N)\) היא מכפלה של \(2\times 2\) אוניטריות. 


  \item כל אוניטרלי \(2\times 2\) הוא מכפלה של שתי קיוביט אוניטריות. 


\end{enumerate}
\begin{proposition}
כל \(\mathcal{U}\in U(N)\) היא מכפלה (טנזורית) של \(2\times 2\) אוניטריות.

\end{proposition}
\begin{proposition}
לכל \(V\) אוניטרי יש שורש אוניטרי \(U\) כך ש-\(V=U^{2}\).

\end{proposition}
\begin{proof}
כיוון ש-\(V\) אוניטרי אז לכסיון וקיים \(\Lambda\) לכסינה כך ש-\(V=W\Lambda W^{\dagger}\). כעת:
$$D=\begin{pmatrix}e^{ i\phi_{1}} &  &  \\ & \ddots &  \\ &  & e^{ i\phi_{N} }\end{pmatrix}\implies\sqrt{ D }=\begin{pmatrix}e^{ i\phi_{1}/2} &  &  \\ & \ddots &  \\ &  & e^{ i\phi_{N}/2 }
\end{pmatrix}$$
וכעת אם נגדיר \(U=W\sqrt{ \Lambda }W^{\dagger}\) נקבל:
$$U^{2}=W\sqrt{ D }\underbrace{ W^{\dagger}W }_{ \mathbb{1}  } \sqrt{ D }W^{\dagger}=W\Lambda W^{\dagger}=V$$

\end{proof}
\begin{reminder}
האופרטור \(\Lambda(U)\) מבצעת CNOT כך שאם הקיוביט השולט(control) יהיה 1 זה יבצע את \(U\) ואחרת לא יעשה כלום. באופן דומה \(\Lambda^{2}(U)\) פועל על שלושה הקיוביטים ומבצע את \(U\) רק אם שתי הביטים הראשונים יהיו \(\ket{1}\).

\end{reminder}
\begin{proposition}
עבור כל קיוביט יחיד ניתן ליישם את \(\Lambda^{2}(V)\) על ידי שתי שערי קיוביט בלבד.

\end{proposition}
\begin{proof}
באינדוקציה. לכל קיוביט אוניטרי \(U\) ניתן ליישם את \(\Lambda^{m}(U^{2})\) על ידי \(\Lambda^{m-1}(U),\Lambda^{m-1}(X),\Lambda(U),\Lambda\left( U^{\dagger} \right)\).

\end{proof}
\begin{definition}[אונברסליות גנארית - Generic]
קבוצה סופית של שערים היא אוניברסלית אם המעגלים שניתן ליישם בעזרתם היא תת מרחב צפוף במרחב האוניטריים \(U(N)\) כאשר \(N=2^{n}\).

\end{definition}
\begin{example}
  \begin{itemize}
    \item הקבוצה \(U(1)\) היא צפופה. האיברים שלו מייוצגים על ידי \(e^{ i\theta }\) עבור \(\theta \in \left[ 0,2\pi \right)\) שזו קבוצה צפופה.
    \item הקבוצה של המספרים האי רציונאלים בתחום \([0,1)\) הם צפופים.
  \end{itemize}
\end{example}
\begin{proposition}
הקבוצה של במספרים \(n\alpha \mod 1\) עבור \(\alpha \in [0,1) \setminus \mathbb{Q}\).

\end{proposition}
\begin{proof}
ראשית נוכיח כי הנקודות האלה הם זרים. לכל \(n,m,\alpha\) קיים מספר \(k \in \mathbb{Z}\) כך ש:
$$n\alpha = m\alpha+k$$
נניח בשלילה שקיימים \(m,n,k\) עבור \(\alpha\) כלשהו. אזי עבור \(\alpha = \frac{k}{n-m}\in \mathbb{Q}\) בסתירה.
ניתן לבחור \(N\) מספרים אי רציונילים שונים \(\left\{  n\alpha  \right\}_{n=1}^{N}\), ונזהה אותם עם נקודות ייחודיות על \([0,1)\) וסביבה פתוחה מסביבים עם אורך \(\varepsilon> \frac{1}{N}\). 
לפחות שתי תחומים כאלה חייבים לחתוך(כיוון שהאורך הכולל הוא \(\varepsilon N > 1\)). ולכן קיימים שתי מספרים \(m<n< \frac{1}{\varepsilon}\) קיימים:
$$r\alpha=(n-m)\alpha<\varepsilon$$
ולכן \(\left\{  kr\alpha \mid k \in \mathbb{Z}  \right\}\) היא קבוצה צפיפה. כיוון ש-\(\varepsilon> 0\) קטן כרצונינו נקבל כי \(\left\{  n\alpha \mod 1  \right\}\) היא קבוצה צפיפה לכל \(n \in \mathbb{N}\).

\end{proof}
\begin{corollary}
עבור מספר אי רציונלי \(\alpha \in \left[ 0,4\pi \right]\) החזקות השלמות של \(e^{ i\alpha/2 }\) יהיו צפיפות ב-\(U(1)\).

\end{corollary}
\begin{corollary}
עבור החבורה \(U(N)\) כל איבר ניתן לכתיבה בצורה 
$$U=\mathrm{diag}\left(e^{i{\frac{\theta_{1}}{2}}},...,e^{i{\frac{\theta_{N}}{2}}}\right)$$
כאשר כיוון שמתורת המספרים אנו יודעים כי \(\frac{\theta_{i}}{\pi},\frac{\theta_{i}}{\theta_{j}}\) יהיו אי רציונאלים כמעט תמיד נקבל כי הערכים העצמיים:
$$\left\{e^{i\frac{\theta_{i}}{2}k}:i\in\left\{1,...,N\right\},k\in\left\{1,2,3,...\right\}\right\}$$
יהיו צופיפים בקבוצה \(U(1)^{N}\).

\end{corollary}
נניח כי יש לנו שתי שערים גנארים \(U_{1}=e^{ iJ_{1} }, U_{2}=e^{ iJ_{2} }\) כאשר \(J_{1},J_{2}\) הם מטריצות הרמיטיות \(N\times N\) אשר לא מתחלפים. נקרא להם גנרים אם חזקות חיוביות של \(U_{i}\). זה נותן לנו לקרב את \(\exp\left( i\alpha_{i}J_{i} \right)\) טוב ככל שנרצה לכל \(\alpha \in \mathbb{R}^{+}\).

\begin{definition}[בר-השגה reachable]
אוניטרי \(U\) אשר לכל \(\varepsilon> 0\) קיים מעגלים אשר מיישים את \(\tilde{U}\) אשר קרוב ל-\(U\) כרצונינו. כלומר:
$$\left\lVert  U-\widetilde{U}  \right\rVert _{\sup }\leq \varepsilon$$

\end{definition}
\begin{corollary}
ניתן להגיע ל-\(e^{ i\alpha_{i}J_{i} }\) בעזרת \(U_{i}\)(כלומר \(U_{1}\) וגם \(U_{2}\)).

\end{corollary}
\begin{corollary}
ניתן להשיג את \(e^{ i\left( \alpha_{1}J_{1}+\alpha_{2}J_{2} \right) }\) אם \(e^{ i\alpha J_{1}/n }\) ו-\(e^{ i\alpha_{2}J_{2}/n }\) ניתנים להשגה

\end{corollary}
\begin{proof}
$${\operatorname*{lim}_{n\longrightarrow\infty}\left(e^{i\alpha_{1}J_{1}/n}e^{i\alpha_{2}J_{2}/n}\right)^{n}}{{}=\operatorname*{lim}_{n\longrightarrow\infty}\left(\mathbb{1} +{\frac{i}{n}}\left(\alpha_{1}J_{1}+\alpha_{2}J_{2}\right)+O\left(n^{-2}\right)\right)^{n}}=e^{ i\left( \alpha_{1} J_{1}+\alpha_{2} J_{2} \right) }$$

\end{proof}
\begin{corollary}
אקספוננט של קומוטטור הוא ניתן להשגה.

\end{corollary}
\begin{proof}
$$\operatorname*{lim}_{n\longrightarrow\infty}\left(e^{i J_{1}/\sqrt{n}}e^{i J_{2}/\sqrt{n}}e^{-i J_{1}/\sqrt{n}}e^{-i J_{2}/\sqrt{n}}\right)^{n}=\operatorname*{lim}_{n\longrightarrow\infty}\left(1-{\frac{1}{n}}\left[J_{1},J_{2}\right]+O\left(n^{-3/2}\right)\right)=e^{-\left[J_{1},J_{2}\right]}$$

\end{proof}
\begin{example}
בעזרת השערים \(U_{n}^{Z}\left( \phi \right)=e^{ -i\phi Z_{n} }\) ו-\(e^{ -i\phi X }\) עבור פאזה \(\phi \in \left[ 0,2\pi \right)\) ניתן ליצור כל אופרטור אוניטרי ב-\(SU(2)\) כיוון שיש לנו שתי יוצרים(\(X,Z\)) ולכן יש לנו אוניברסליות גנארית.

\end{example}
\begin{reminder}
מתקיים:
$$U(N)=SU(N)\times U(1)$$
כאשר \(U(1)\) היא פאזה גלובאלית ולכן מספיק להתייחס ל-\(SU(N)\)

\end{reminder}
\begin{reminder}
החבורה \(U(N)\) היא חבורת לי. ולכן ניתן להשיג כל איבר על ידי:

$$U=\exp\left(i{\sum_{i}}\alpha_{i}J_{i}\right)$$
כאשר:
$$[J_{i},J_{j}]=i\sum_{k}f_{i j k}J_{k}$$

\end{reminder}
\begin{corollary}
מספיק שיהיה שער גנרי לחלק מהיוצרים והאלגברת לי נותת לנו את השאר.

\end{corollary}
\begin{corollary}
עבור \(SU(2)\) יש לנו שלושה יוצרים, ולכן מספיק שתי שערים גנרים כדי להשיג את כל הסיבובים האוניטרים.

\end{corollary}
\begin{corollary}
ל-\(SU(4)\) יש 15 יוצרים. אם מסתכל על האלגברה ניתן לראות כי שתי שערים מספיק כדי להשיג כל שער אחר. וכיוון שהראנו כי כל שתי קיוביטים הם אוניברסלים, שתי שערים גנארים הם מספיקים בשביל קבוצה גנארית אוניברסלית.

\end{corollary}
\section{קבצות שערים אוניברסליים}

\begin{reminder}
ניתן לבנות כל אינטרי בעזרת אוסף כל השערים שפועלים על שתי קיוביטים. כלומר אוסף כל השערי שתי קיוביט(אוניטריות \(4\times 4\) היא קבוצת שערים אוניברסליים).

\end{reminder}
\begin{reminder}
ניתן לכתוב כל אוניטרי \(2\times 2\) בעזרת זוויות אויילר בצורה הבאה:
$$U\left(\alpha,\beta,\gamma\right)=e^{-i\alpha Z}e^{-i\beta Y}e^{-i\gamma Z}$$

\end{reminder}
\begin{lemma}
עבור:
$$A\left(\alpha,\beta\right)=e^{-i\alpha Z}e^{-i\beta Y/2}\quad B\left(\alpha,\beta,\gamma\right)=e^{i\beta Y/2}e^{i\left(\gamma+\alpha\right)Z/2}\quad C\left(\alpha,\gamma\right)=e^{-i\left(\gamma-\alpha\right)Z/2}$$
מתקיים:
$$A B C=\mathbb{1}\quad{\mathrm{and}}\quad A X B X C=U$$

\end{lemma}
\begin{proposition}
ניתן בעזרת \(A,B,C\) המוגדרים לעיל ו-CNOT ליצור את \(\Lambda(U)\) עבור כל אוניטרי \(U\).

\end{proposition}
\begin{proof}
נרצה שאם מקבלים 0 נפעיל עם \(ABC=\mathbb{1}\) ואם נקבל 1 נפעל עם \(U=AXBXC\). לכן נפעל ראשית עם \(C\) ואז עם \(\Lambda(X)\)(לא יעשה כלום אם 0 ואחרת יפעיל \(X\)) לאחר מכן נפעל עם \(B\) ואז שוב עם \(\Lambda(X)\) ולבסוף שוב עם \(A\).

\end{proof}
\begin{proposition}
קבוצת השערים שפועלים על קיוביט אחד, עם CNOT היא קבוצת שערים אוניברסליית

\end{proposition}
\begin{proof}
ראינו כי בעזרת אופרטורי קיוביט יחיד ו-CNOT ניתן לבנות את כל השערי שתי קיוביט מהצורה \(\Lambda(U)\) כאשר הראנו שכל קבוצת השערים הזו היא קבוצת שערים אונברסליים.

\end{proof}
\begin{proposition}
נגדיר:
$$S=\left(\begin{array}{l l}{{1}}&{{0}}\\ {{0}}&{{i}}\end{array}\right)$$
כעת האדמר(H) ו-\(\Lambda(S)\) יוצרים קבוצת שערים אוניברסליים.

\end{proposition}
\begin{lemma}
האופרטור \(T=\exp\left(-i\pi Z/8\right)\) מקיים \(T^{2}=e^{ -i\pi/4 }S\)

\end{lemma}
$$T^{2}\;=\;\exp\left(-i\pi Z/4\right)\;=\;\left(\begin{array}{c c}{{e^{-i\pi/4}}}&{{0}}\\ {{0}}&{{e^{i\pi/4}}}\end{array}\right)\;=\;e^{-i\pi/4}\left(\begin{array}{c c}{{1}}&{{0}}\\ {{0}}&{{i}}\end{array}\right)=e^{ i\pi/4 }$$

\begin{proposition}
קבוצת השערים \(\left\{  T,H,\mathrm{CNOT}  \right\}\) היא קבוצת שערים אוניברסליים.

\end{proposition}
\begin{proof}
ראשית נשים לב כי כיוון ש-\(S^{4}=\mathbb{1}\) אזי \(T^{8}=\left( e^{ -i\pi/4 }S \right)^{4}=-\mathbb{I}\), ולכן \(T^{\dagger}=-T^{7}\) ולכן עד כדי פאזה גלובלית נקבל כי \(T^{7}\) שקול ללהפעיל את ההופכי שלו (אם מאיזושהי סיבה אכפת לנו מפאזה גלובלית נדרש \(T^{15}=T^{\dagger}\)). כעת כיוון ש-\(T^{\dagger}T^{\dagger}T^{2}=\mathbb{1}\) ו-\(T^{\dagger}XT^{\dagger}XT^{2}=T^{2}\) ניתן לבחור \(A=B=-T^{\dagger}=T^{7}\), ו-\(C=T^{2}\) כך שנקבל \(\Lambda(T^{2})=\Lambda \left( e^{ -i\pi/4 }S \right)\) ולכן ניתן בעזרת \(T\) ו-CNOT לקבל את \(\Lambda(S)\) עד כדי פאזה גלובלית ובפרט כיוון ש-\(\left\{  H,\Lambda(S)  \right\}\) קבוצת שערים אוניברסלים אזי גם \(\{ H,CNOT,T \}\) תהי קבוצת שערים אוניברסליים.

\end{proof}
\begin{example}
נתון לנו מערכת של \(n\) קיוביטים שעליהם ניתן לבצע את האוניטריות \(U_{n}^{Z}\left( \phi \right)=e^{ -i\phi Z_{n} }\) ו-\(U_{n}^{X}\) לכל פאזה \(\phi \in \left[ 0,2\pi \right)\) וכן ניתן לבצע את \(V_{mn}^{Z Z}=e^{ i\pi Z_{m}Z_{n}/4 }\) עבור כל שתי קיוביטים \(m,n\). ראשית נשים לב כי ניתן לממש כל שער של קיוביט יחיד כיוון שראינו כי אם ניתן להשיג את \(e^{ i\alpha J_{1}/n  }\) ואת \(e^{ i\alpha J_{2}/n }\) אז ניתן להשיג גם את \(e^{ i\alpha(J_{1}+J_{2})/n }\), ואף גם את \(e^{ i\alpha [J_{1},J_{2}] }=e^{ i\alpha J_{3} }\) ולכן כיוון ש-\(Z,X\) יוצרים של \(SU(2)\) ניתן ליצור את כל השערים של קיוביט יחיד. כעת נסתכל על השער controlled-Z:
\begin{gather*}C_{m n}^{Z}=\left|0\right\rangle\left\langle0\right|_{m}\otimes\mathbb{I}_{n}+\left|1\right\rangle\left\langle1\right|_{m}\otimes Z_{n}=\frac{1}{2}\left( \mathbb{1}_{m} +Z_{m} \right)\otimes \mathbb{1}_{n} +\frac{1}{2}\left( \mathbb{1}_{m} -Z_{m} \right)\otimes Z_{n} = \\=\frac{1}{2}\left( \mathbb{1}_{m} \otimes \mathbb{1} _{n} +Z_{m}\otimes \mathbb{1} _{n}+\mathbb{1} _{m}\otimes Z_{n}-Z_{m}\otimes Z_{n} \right)
\end{gather*}
נשים לב כי ניתן לכתוב אותו על ידי \(C_{mn}^{Z}=e^{ i\alpha(1-Z_{m})(1-Z_{n}) }\) עבור \(\alpha=\frac{\pi}{4}\) כיוון שמתקיים:
$$e^{ i\frac{\pi}{4}(1-Z_{m})(1-Z_{n}) }=e^{ i\frac{\pi}{4} \left( \mathbb{1} -Z_{m} -Z_{n}-Z_{m}Z_{n}\right)}=e^{ i\frac{\pi}{4} }e^{ -i\frac{\pi}{4} Z_{n} }e^{ -i\frac{\pi}{4}Z_{m} }e^{ i\frac{\pi}{4} Z_{m}Z_{n} }$$
כאשר נזכור כי \(e^{ i\pi/4 }=\frac{1}{\sqrt{ 2 }}(1+i)\) וכן אנו יודעים כי עבור מטריצות פאולי מתקיים:
$$e^{ i\pi Z/4 }=\cos\left( \frac{\pi}{4} \right)\mathbb{1} +i\sin\left( \frac{\pi}{4} \right)Z=\frac{1}{\sqrt{ 2 }}\mathbb{1} +\frac{i}{\sqrt{ 2 }}Z=\frac{1}{\sqrt{ 2 }}\left( \mathbb{1} +iZ \right)$$
ולכן נקבל סה"כ בעזרת כתיבה קצת יותר קומפקטית:
\begin{gather*}e^{ i\frac{\pi}{4} }e^{ -i\frac{\pi}{4} Z_{n} }e^{ -i\frac{\pi}{4}Z_{m} }e^{ i\frac{\pi}{4} Z_{m}Z_{n} }=\frac{1}{4}(1+i)(1-iZ_{m})(1-iZ_{n})(1+iZ_{m}Z_{n})=  \\=\frac{1}{4}\cancelto{ 2 }{ (i+1)(i-1) }(1+Z_{m}+Z_{m}+Z_{m}Z_{n})=\frac{1}{2}\left( \mathbb{1}_{m} \otimes \mathbb{1} _{n} +Z_{m}\otimes \mathbb{1} _{n}+\mathbb{1} _{m}\otimes Z_{n}-Z_{m}\otimes Z_{n} \right)
\end{gather*}
כלומר עד כדי פאזה גלובלית ניתן לממש את \(C_{mn}^{Z}\) עם האוניטריות הנתונות, ולכן ניתן לממש כל שער של קיוביט יחיד ו-\(C_{mn}^{Z Z}\) וראינו כי זה קבוצת שערים אוניברסלית.

\end{example}
\begin{summary}
הקבוצות הבאות הם קבוצות שערים אוניברסליים:
- קבוצת כל השערים שפועלים של 2 קיוביטים.
- הקבוצה של השערים הנשלטים \(\Lambda(U)\).
- קבוצת השערים שפועלים על קיוביט אחד ו-CNOT.
- האדמר ו-\(S\) כאשר:
$$S=\left(\begin{array}{c c}{{1}}&{{0}}\\ {{0}}&{{i}}\end{array}\right)$$
- קבוצת השערים \(\left\{  T,H,\mathrm{CNOT}  \right\}\) כאשר \(T=\exp\left(-i\pi Z/8\right)\).

\end{summary}
\Chapter{מימוש פיזי של מחשב קוונטי}

\section{מלכודת פאול - קלאסי}

\begin{reminder}[תנאים למחשב קוונטי]
  \begin{enumerate}
    \item יכולת לייצג אינפורמצייה קוונטית. 


    \item לבצע קבוצה אינברסילת של טרנספורמציה אוניברסלית. 


    \item לאתחל את המערכת במצב התחלתי נתון. 


    \item למדוד את התוצאות. 


  \end{enumerate}
\end{reminder}
\begin{reminder}[משפט Earnshaw]
אין מצב אקטרוסטטי יציב של חלקיק טעון בשדה חשמלי.

\end{reminder}
\begin{proposition}
יש שתי דרכים נפוצות להתגבר על בעיה זו כדי ללכוד יון:

  \begin{enumerate}
    \item מלכודת פאול - יוצרים שדה חשמלי המשתנה בזמן. 


    \item מלכודת פנינג - משלבים שדה חשמלי ושדה מגנטי. 


  \end{enumerate}
\end{proposition}
אנחנו נתמקד במלכדת פאול. כלומר נקבל שדה חשמלי ופוטנציאל שתלוי בזמן

\begin{proposition}
עבור יון יחיד עם מסה \(M\) נקבל אפקטיבית פוטנציאל הרמוני עם תדירות \(\omega\). ולכן עבור פוטנצאיל של חלקיק יחדי נקבל:
$$V_{0}=\frac{1}{2}M\omega^{2}z^{2}$$
כאשר \(z\) הוא קורדינטה לאורך הישר עליו הוא כלוא.

\end{proposition}
\begin{proposition}
אם יש לנו \(N\) יונים זהים אשר כולם עם מסה \(m\) ומטען \(q\) הכלואים לאורך ישר הפוטנציאל בהתחשב האינטרקציה החשמלית יהיה:
$$V=\frac{1}{2}M\omega^{2}{\sum_{n}}z_{n}^{2}+\frac{q^{2}\alpha}{2}{\sum_{n}}{\sum_{m\neq n}}\frac{1}{|z_{n}-z_{m}|}$$

\end{proposition}
\begin{proposition}
הנקודות שיווי משקל של החלקיקים נתונים על ידי המשוואה:
$$Z_{n}-\sum_{m<n}\frac{1}{(Z_{n}-Z_{m})^{2}}+\sum_{m>n}\frac{1}{(Z_{n}-Z_{m})^{2}}=0$$
כאשר:
$$Z_{n}=\left(\frac{4M\omega^{2}}{q^{2}\alpha}\right)^{1/3}z_{n}^{(0)}$$

\end{proposition}
\begin{proof}
נדרוש:
$$\left.\frac{\partial V}{\partial z_{n}}\right|_{z_{n}=z_{n}^{(0)}}=0$$
כלומר:
$$M\omega^{2}z_{n}^{(0)}-\frac{q^{2}\alpha}{4}\sum_{m<n}\frac{1}{\left(z_{n}^{(0)}-z_{m}^{(0)}\right)^{2}}+\frac{q^{2}\alpha}{4}\sum_{m>n}\frac{1}{\left(z_{n}^{(0)}-z_{m}^{(0)}\right)^{2}}=0$$
כאשר נגדיר:
$$Z_{n}=\left(\frac{4M\omega^{2}}{q^{2}\alpha}\right)^{1/3}z_{n}^{(0)}$$
ומזה ניתן לקבל את המשוואה של הטענה.

\end{proof}
\begin{remark}
עבור \(N=2,3\) זה לא בעיה לפתור את המשוואה אנליטית. עבור ערכים גדולים יותר כבר פותרים את זה נומרית.

\end{remark}
כיוון שזה לא בעיה סטטית נצפה לתנודות קטנות סביב הנקודות שיווי משקל. נרצה לבצע לינאריזציה לבעיה.

\begin{proposition}
ניתן לכתוב את המיקום של כל חלקיק בתור המיקום של שיווי המשקל ועוד תנועות שתלויות בזמן מסביב לשיווי המשקל. כלומר:
$$z_{n}=z_{n}^{(0)}+\zeta_{n}$$
כאשר:
$$\dot{z}_{n}=\dot{\zeta}_{n}$$

\end{proposition}
\begin{proposition}
האנרגיה הקינטית תהיה מהצורה:
$$T=\frac{1}{2}M\sum_{n}\dot{\zeta}_{n}^{2}$$
כאשר ניתן לקרב את האנרגיה הפוטנציאלית על ידי:
$$V\approx\frac{1}{2}M\omega^{2}\sum_{m,n}\hat{V}_{m n}\zeta_{m}\zeta_{n}$$
כאשר:
$${\hat{V}}_{m n}={\frac{1}{M\omega^{2}}}\left.{\frac{\partial^{2}V}{\partial z_{m}\partial z_{n}}}\right|_{{\bf z}={\bf z}^{(0)}}=\left\{\begin{array}{l l}{{1+\sum_{l\neq m}{\frac{2}{|Z_{m}-Z_{l}|^{3}}}}}&{{m=n}}\\ {{-{\frac{2}{|Z_{m}-Z_{n}|^{3}}}}}&{{m\neq n}}\end{array}\right.$$

\end{proposition}
\begin{proposition}
עבור התנודות הקטנות נקבל כי משוואת התנועה תהיה:
$$\ddot{\zeta}_{m}+\omega^{2}\sum_{n}\hat{V}_{m n}\zeta_{n}=0$$

\end{proposition}
\begin{proof}
כעת נכתוב את הלגרנג'יאן \(L=T-V\) ואת המשוואות אויילר לגרנג':
$$\frac{d}{d t}\left(\frac{\partial L}{\partial\dot{\zeta}_{n}}\right)-\frac{\partial L}{\partial\zeta_{n}}=0$$
ונקבל עבור התנודות הקטנות:
$$\ddot{\zeta}_{m}+\omega^{2}\sum_{n}\hat{V}_{m n}\zeta_{n}=0$$

\end{proof}
כדי לפתור את משוואת התנועה נרצה ללכסן את המשוואות.

\begin{proposition}
ניתן ללכסן את הבעיה כך שמשוואות התנועה יהיו מהצורה:
$$z_{n}=z_{n}^{(0)}+\sum_{k}C_{k}D_{m k}\cos\left(\omega_{k}t\right)$$
והלגרנג'יאן יהיה:
$$ L = \frac{M}{2} \sum_k (\dot{Q}_k^2 - \omega_k^2 Q_k^2) $$

\end{proposition}
\begin{proof}
נציב טור פורייה מרוכב מהצורה \(\zeta_{m}=\sum_{k}K_{m k}e^{-i\omega_{k}t}\) כך ש-\(\ddot{\zeta}_{m}=-\sum_{k}\omega_{k}^{2}K_{m k}e^{-i\omega_{k}t}\) ונקבל:
$$-\sum_{k}\omega_{k}^{2}K_{m k}e^{-i\omega_{k}t}+\omega^{2}\!\sum_{n,k}\!\hat{V}_{m n}K_{n k}e^{-i\omega_{k}t}=0$$
כלומר:
$$\sum_{k}\left(-\omega_{k}^{2}K_{m k}+\omega^{2}\sum_{n}\hat{V}_{m n}K_{n k}\right)e^{-i\omega_{k}t}=0$$
כאשר מיחידות טור פורייה נדרש כי כל המקדמים מתאפסים, ולכן לכל \(k\) נדרש:
$$\sum_{n}\left(-\omega_{k}^{2}\delta_{m n}+\omega^{2}\hat{V}_{m n}\right)K_{n k}=0\implies  \sum_n \hat{V}_{mn} K_{nk} = \frac{\omega_k^2}{\omega^2} K_{mk} $$
ולכן העמודות של \(K\) הם ווקטורים עצמיים של \(V_{mn}\) עם ערכים עצמיים \(\mu_{k}=\frac{\omega_{k}^{2}}{\omega^{2}}\). בנוסף נדרש:
$$\operatorname*{det}\left(-\omega_{k}^{2}\mathbb{1}+\omega^{2}\hat{V}\right)=0\implies \omega_{k}=\omega \sqrt{ \mu_{k} }$$
ניתן לנרמל את \(K_{nk}\) ולקבל מטריצה אורתונורמלית \(K_{nk}=C_{k}D_{nk}\). כעת נגדיר אופני תנודה \(Q_{k}(t)=C_{k}e^{ -i\omega_{k}t }\) כך שניתן לכתוב:
$$\zeta_{m}\left(t\right)=\sum_{k}D_{m k}Q_{k}\left(t\right)$$
כאשר התנועה של היונים תהיה נתונה על ידי:
$$z_{n}=z_{n}^{(0)}+\sum_{k}C_{k}D_{m k}\cos\left(\omega_{k}t\right)$$

\end{proof}
\begin{example}
עבור \(\mu=1\) נקבל אופן תנודה שבה כל היונים נעים ביחד עם אותה אמפליטודה ופאזה(center of mass mode):
$$ D_{m1} = \frac{1}{\sqrt{N}}(1, 1, ..., 1)^T $$
עבור \(\mu=3\) נקבל אופן תנודה כך שהמרחק ביניהם משתנה בצורה מחזורית(Breathing mode):
$$ D_{m2} = \frac{1}{\sqrt{\sum_n Z_n^2}} (Z_1, Z_2, ..., Z_N)^T $$

\end{example}
\begin{proposition}
המצב מרכז מסה:
$$ D_{m1} = \frac{1}{\sqrt{N}}(1, 1, ..., 1)^T $$
תמיד אופן תנודה עם תדירות \(\omega\) ללא תלות ב-\(n\).

\end{proposition}
\begin{proof}
משוואת הערכים העצמיים היא מהצורה:
$$ \sum_{n}(-\omega^{2}\delta_{mn}+\omega^{2}\hat{V}_{mn})D_{n1}=0 $$
כיוון ש-\(D_{n 1}\) הוא 1 עד כדי כפל בקבוע, וכן ניתן לחלק ב-\(\omega\) ולקבל את הדרישה:
$$ \sum_{n}(-\delta_{mn}+\hat{V}_{mn})=0 $$
ואכן אם אנחנו סוכמים על \(m\) קבוע נקבל:
$$\sum_{n} V_{mn}=1 + {\frac{2}{|Z_{m}-Z_{n}|^{3}}}- \frac{2}{|Z_{m}-Z_{n}|^{3}} = 1$$
ולכן \(\sum_{n}(-\delta_{mn}+\hat{V}_{mn}) = 0\) ואכן אופן תנודה עם תדירות \(\omega\).

\end{proof}
\section{מלכודת פאול - קוונטי}

\begin{proposition}
נניח כי ההמילטוניאן של כל יון יהיה מהצורה:
$$H_{\mathrm{ion}}=\sum_{\mu}\Omega_{\mu}\left|\mu\right\rangle\left\langle\mu\right|$$
כך שעבור \(n\) יונים נקבל \(\sum_{n}H_{\text{ion},n}\).

\end{proposition}
\begin{proposition}
האינטרקציה בין היונים היא אינטרקציה קולומבית. לכן מספיק לעשות קוונטיזיציה של התנודות הקטנות הקלאסיות ולקבל:
$$H_{\mathrm{ph}}=\sum_{k}\left(\frac{P_{k}^{2}}{2M}+\frac{1}{2}M\omega_{k}^{2}Q_{k}^{2}\right)$$
כאשר \(P_{k},Q_{k}\) הם אופרטורים המקיימים את היחסים הקנונים:
$$[P_{k},P_{k^{\prime}}]=[Q_{k},Q_{k^{\prime}}]=0$$
כך שמתקיים:
$$\left[Q_{k},P_{k^{\prime}}\right]=i\delta_{k k^{\prime}}$$

\end{proposition}
\begin{definition}[אופרטורי העלה והורדה]
נגדיר את \(a_{k}\) ו-\(a_{k}^{\dagger}\) על ידי:
$$Q_{k}=\frac{1}{\sqrt{2M\omega_{k}}}\left(a_{k}+a_{k}^{\dagger}\right)\qquad P_{k}=-i\sqrt{\frac{M\omega_{k}}{2}}\left(a_{k}-a_{k}^{\dagger}\right)$$
כך שנקבל:
$$\left[a_{k},a_{k^{\prime}}\right]=\left[a_{k}^{\dagger},a_{k^{\prime}}^{\dagger}\right]=0\qquad \left[a_{k},a_{k^{\prime}}^{\dagger}\right]=\delta_{k k^{\prime}}$$

\end{definition}
\begin{corollary}
ניתן לכתוב:
$$H_{\mathrm{ph}}=\sum_{k}\left(\omega_{k}a_{k}^{\dagger}a_{k}+\frac{1}{2}\right)$$
כאשר נסמן את אופרטור המספר על אופן \(k\) על ידי \(n_{k}=a_{k}^{\dagger}a_{k}\). 

\end{corollary}
\begin{proposition}
$$a_{k}\left|n_{k}\right\rangle=\sqrt{n_{k}}\left|n_{k}-1\right\rangle \qquad a_{k}^{\dagger}\left|n_{k}\right\rangle=\sqrt{n_{k}+1}\left|n_{k}+1\right\rangle$$

\end{proposition}
\begin{proposition}
המצב הווקום של כל אופן מקיים:
$$a_{k}\left|0_{k}\right\rangle=0$$
כאשר קיים ווקום גלובאלי $$|0\rangle=\otimes_{k}|0_{k}\rangle$$
אשר מקיים:
$$a_{k}\left|0\right\rangle=0,\qquad\forall k$$

\end{proposition}
\begin{remark}
האקסיטציות הבוזוניות האלה נקראות פונונים(phonons), ומכאן הסימון \(H_{\text{ph}}\). אלו קווזי חלקיקים(לא באמת חלקיקים אבל אפשר להתחייס עליהם כחלקיקים) ונוצרים ומושמדים על ידי האופרטורים \(a_{k},a_{k}^{\dagger}\).

\end{remark}
\section{קירוב דיפולי}

\begin{symbolize}
$$\ket{g} \equiv \ket{0} \qquad \ket{e} \equiv \ket{1} $$
כאשר המצב \(\ket{g}\) מסמן את מצב ביסוד ולכן לא יכול לדעוך(מצב יציב) כאשר המצב \(\ket{e}\) זה מצב מעורער אשר אנחנו מניחים שסמי יציב - יכול לדעוך ספונטנית אבל עם זמן מחצית חיים ארוך חסית. בנוסף נסמן:
$$\left|g\right\rangle\left\langle e\right|_{n}=\sigma_{-}^{n},\quad\left|e\right\rangle\left\langle g\right|_{n}=\sigma_{+}^{n}$$

\end{symbolize}
\begin{proposition}
אם המצבים \(\ket{g}\) ו-\(\ket{e}\) מופרדים על ידי תדירות \(\Omega\) עבור כל סיבה פרקטית ניתן לכתוב:
$$H_{\mathrm{ion,}n}=\frac{\mathcal{Q}}{2}\left(|e\rangle\left\langle e|_{n}-|g\rangle\langle g|_{n}\right\rangle\equiv\frac{\Omega}{2}\sigma_{n}^{z}\right)$$

\end{proposition}
\begin{definition}[לייזר]
שדה אלקטרומגנטי עם תדירות \(\omega_{L}\) ווקטור גל \(\kappa=\frac{\omega_{L}}{c}\). נסמן את ווקטור הפולריזציה שלו ב-\(\hat{\epsilon}\).

\end{definition}
\begin{corollary}
השדה האלקטרומגנטי המתאים לליזר ניתן לכתיבה על ידי:
$$\mathbf{E}\left(\mathbf{x},t\right)=E_{0}\cos\left(\omega_{L}t-\kappa\cdot\mathbf{x}+\phi_{0}\right){\hat{\mathbf{\epsilon}}}$$
כאשר \(E_{0}\) היא האמפליטודה ו-\(\phi_{0}\) היא הפזאה. 

\end{corollary}
\begin{proposition}[קירוב דיפול של האינטרקציה]
ניתן לקרב את האינטקרציה על ידי קירוב דיפולי. כלומר המילטוניאן האינטרקציה יהיה מהצורה:
$$H_{\mathrm{int},n}=-\mathbf{d}_{n}\cdot\mathbf{E}\left(\mathbf{x}_{n},t\right)$$
כאשר \(\mathbf{d}_{n}\) הוא אופרטור הדיפול של היון ויהיה מהצורה \(\mathbf{d}_{n}=-e\mathbf{r}_{n}\) כאשר \(e\) הוא מטען האלקטרון ו-\(\mathbf{r}_{n}\) הוא המיקום של האלקטורון החיצוני.

\end{proposition}
\begin{proposition}
אכפת לנו רק מהרמות הזהות \(\ket{g}_{n}\) ו-\(\ket{e}_{n}\) כאשר הרמות האחרות לא מאוכלסות אף פעם. לכן ניתן לכתוב את היחידה על ידי:
$$\mathbb{1}_{n}=|g\rangle\,\langle g|_{n}+|e\rangle\,\langle e|_{n}$$

\end{proposition}
\begin{corollary}
ניתן לכתוב את אופרטור הדיפול על ידי:
$${\bf d}_{n}=-e\left(\left|g\right\rangle\left\langle g\right|_{n}+\left|e\right\rangle\left\langle e\right|_{n}\right){\bf r}_{n}\left(\left|g\right\rangle\left\langle g\right|_{n}+\left|e\right\rangle\left\langle e\right|_{n}\right)$$

\end{corollary}
\begin{proposition}
אם נסמן \(u_{g}\left( \mathbf{r} \right)=\braket{ \mathbf{r} | g }\) ו-\(u_{e}\left( \mathbf{r} \right)=\braket{ \mathbf{r} | e }\) ונניח כי \(u_{g},u_{f}\) סימטריות מרחבית אז:
$$\mathbf{d}_{n}=-e\left[\left\langle g\right|\mathbf{r}\left|e\right\rangle\left|g\right\rangle\left\langle e\right|_{n}+\left\langle e\right|\mathbf{r}\left|g\right\rangle\left|e\right\rangle\left\langle g\right|_{n}\right]$$

\end{proposition}
\begin{proof}
כיוון שיש סימטריה מרחבית נקבל \(u_{g}\left( \mathbf{r} \right)=u_{g}\left( \mathbf{-r} \right)\) ו- \(u_{f}\left( \mathbf{r} \right)=u_{f}\left( \mathbf{-r} \right)\) ולכן:
$$\left\langle  g|\,\mathbf{r}\,|g \right\rangle=\int d^{3}r\mathbf{r}\,|u_{g}\left(\mathbf{r}\right)|^{2}=0\qquad \left\langle  e|\,\mathbf{r}\,|e \right\rangle=\int d^{3}r\mathbf{r}\,|u_{e}\left(\mathbf{r}\right)|^{2}=0.$$
כאשר נציב בביטוי שקיבלנו עבור אופרטור הדיפול ונקבל:
$$\mathbf{d}_{n}=-e\left[\left\langle g\right|\mathbf{r}\left|e\right\rangle\left|g\right\rangle\left\langle e\right|_{n}+\left\langle e\right|\mathbf{r}\left|g\right\rangle\left|e\right\rangle\left\langle g\right|_{n}\right]$$

\end{proof}
\begin{proposition}
ניתן לסמן את המילטוניאן האינטרקציה על ידי:
$$H_{\mathrm{int},n}=E_{0}\cos\left(\omega_{L}t-\kappa\cdot{\bf x}_{n}+\phi_{0}\right)\left(d\sigma_{-}^{n}+\overline{{{d}}}\sigma_{+}^{n}\right)$$
כאשר \(d=e\left\langle g|\,\mathbf{r}\,|e\right\rangle\cdot{\hat{\boldsymbol{\epsilon}}}\).

\end{proposition}
\begin{corollary}
כיוון ש-\(\mathbf{x}_{n}\) הוא המיקום של היון ה-\(n\) של המלכודת נקבל:
$$\mathbf{x}_{n}=(0,0,z_{n})=\left[z_{n}^{(0)}+\sum_{k}{\frac{1}{\sqrt{2M\omega_{k}}}}D_{m k}\left(a_{k}+a_{k}^{\dagger}\right)\right]\hat{\mathbf{z}}$$

\end{corollary}
\begin{corollary}
$$e^{i\left(\omega_{L}t-\kappa\cdot{\bf x}_{n}+\phi_{0}\right)}\equiv e^{i\omega_{L}t}e^{i\sum_{k}\eta_{k}\left(a_{k}+a_{k}^{\dagger}\right)}e^{i\phi}$$

\end{corollary}
\begin{corollary}
ניתן לכתוב את ההמילטוניאן אינטרקציה על ידי:
$$H_{\mathrm{int},n}=\frac{E_{0}}{2}\left(e^{i\omega_{L}t}e^{i\sum_{k}\eta_{k}\left(a_{k}+a_{k}^{\dagger}\right)}e^{i\phi}+h.c.\right)\left(d\sigma_{-}^{n}+h.c.\right)$$

\end{corollary}
\begin{proposition}
ניתן למצוא את הקידום בזמן בעזרת המשוואה:
$$i\frac{\partial}{\partial t}\left|\widetilde{\psi}\right\rangle=e^{i H_{0}t}H_{\mathrm{int}}\left|\psi\right\rangle\equiv\widetilde{H}\left(t\right)\left|\widetilde{\psi}\right\rangle$$
כאשר:
$$\widetilde{H}=\frac{E_{0}}{2}\left(e^{i\omega_{L}t}\exp\left(i\sum_{k}\eta_{k}\left(a_{k}e^{-i\omega_{k}t}+a_{k}^{\dagger}e^{i\omega_{k}t}\right)\right)e^{i\phi}+h.c.\right)\left(d\sigma_{-}^{n}e^{-i\Omega t}+h.c.\right)$$

\end{proposition}
\begin{proof}
נעבור למערכת:
$$\left|\psi\right\rangle\longrightarrow\left|\widetilde{\psi}\right\rangle=e^{i H_{0}t}\left|\psi\right\rangle$$
ממשוואת שרדינגר:
$$i\frac{\partial}{\partial t}\left|\widetilde{\psi}\right\rangle=i\frac{\partial}{\partial t}\left(e^{i H_{0}t}\left|\psi\right\rangle\right)=-H_{0}\left|\widetilde{\psi}\right\rangle+e^{i H_{0}t}\left(i\frac{\partial}{\partial t}\left|\psi\right\rangle\right)$$
כאשר כיוון ש:
$$i\frac{\partial}{\partial t}\left|\psi\right\rangle=H\left|\psi\right\rangle=\left(H_{0}+H_{\mathrm{int}}\right)\left|\psi\right\rangle$$
נקבל:
$$i\frac{\partial}{\partial t}\left|\widetilde{\psi}\right\rangle=e^{i H_{0}t}H_{\mathrm{{int}}}\left|\psi\right\rangle\equiv\widetilde{H}\left(t\right)\left|\widetilde{\psi}\right\rangle$$
כאשר \(\tilde{H}(t)=e^{ iH_{0}t }H_{\text{int}}e^{ -iH_{0} }t\) כנתון בטענה.

\end{proof}
\begin{proposition}
אם נפעל בתחום Lamb-Dicke שבו \(\eta_{k}\ll 1\) ניתן לפתוח את הסוגריים ולקבל המילטוניאן ללא כוללים את דרגת החופש של הפונונים:
$$\widetilde{H}_{\mathrm{carrier}}\left(t\right)\equiv\frac{E_{0}}{2}e^{i\phi}\left(d e^{i\left(\omega_{L}-\Omega\right)t}\sigma_{-}^{n}+\overline{{{d}}}e^{i\left(\omega_{L}+\Omega\right)t}\sigma_{+}^{n}\right)+h.c.$$
ועוד גורם אשר יוצרים או משמידים פוטון יחיד:
$$\widetilde{H}_{\mathrm{sidelband}}\left(t\right)\equiv i\frac{E_{0}}{2}e^{i\phi}\left(d e^{i\left(\omega_{L}-\Omega\right)t}\sigma_{-}^{n}+\overline{{{d}}}e^{i\left(\omega_{L}+\Omega\right)t}\sigma_{+}^{n}\right)\sum_{k}\eta_{k}\left(a_{k}e^{-i\omega_{k}t}+a_{k}^{\dagger}e^{i\omega_{k}t}\right)+h.c.$$

\end{proposition}
\begin{corollary}
על ידי בחירה של תדירות הלייזר \(\omega_{L}\) ניתן לבחור איזה גורמים מבצעים אוסצילציות גבוהות ואיזה ברסוננס ויהיו עם תדירות אפסית(או כמעט אפסית).

\end{corollary}
\begin{proposition}
כדי לממש שער קיוביט יחיד ניתן לבחר \(\omega_{L}=\Omega\).

\end{proposition}
\begin{proof}
עבור \(\omega_{L}=\Omega\) נקבל:
$$\tilde{H}\left(t\right)\approx\tilde{H}_{\mathrm{carrier}}\approx\frac{E_{0}}{2}e^{i\phi}d\sigma_{-}^{n}+h.c.\equiv\mu e^{i\theta}\sigma_{-}^{n}+h.c.$$
אינטרקציה זו לא משנה את מספר הפונונים ופועלת לוקאלית על היונים. נקבל:
$$\widetilde{H}\approx\mu\left(e^{i\theta}\sigma_{-}^{n}+e^{-i\theta}\sigma_{+}^{n}\right)=\mu\left(\cos\theta\sigma_{x}^{n}+\sin\theta\sigma_{y}^{n}\right)\equiv\mu\sigma_{\theta}^{n}$$
כאשר הקידום בזמן יתן:
$$e^{-i\tilde{H}t}=e^{-i\mu\sigma_{\theta}^{n}t}=\cos\left(\mu t\right)\mathbb{1}-i\sin\left(\mu t\right)\left(\cos\theta\sigma_{x}^{n}+\sin\theta\sigma_{y}^{n}\right)$$

\end{proof}
\begin{example}
אם נפעיל את הלייזר ל-\(t=\frac{\pi}{4\mu}\) ו-\(\theta=\frac{\pi}{2}\) נקבל:
$$e^{-i\tilde{H}t}=\frac{1}{\sqrt{2}}\left(\mathbb{1}-i\sigma_{y}\right)=\frac{1}{\sqrt{2}}\left(\begin{array}{c c}{{1}}&{{1}}\\ {{-1}}&{{1}}\end{array}\right)$$
כאשר אם נפעל על המצב בסיס נקבל:
$$\frac{1}{\sqrt{2}}\left(\mathbb{1}-i\sigma_{y}\right)|g\rangle=\frac{1}{\sqrt{2}}\left(|g\rangle+|e\rangle\right)\qquad \frac{1}{\sqrt{2}}\left(\mathbb{I}-i\sigma_{y}\right)|e\rangle=-\frac{1}{\sqrt{2}}\left(|g\rangle-|e\rangle\right)$$
אם נכפיל בשמאל ב-\(\sigma_{z}\) נקבל:
$$\frac{1}{\sqrt{2}}\sigma_{z}\left(\mathbb{I}-i\sigma_{y}\right)=\frac{\sigma_{z}+\sigma_{x}}{\sqrt{2}}=U_{H}$$

\end{example}
\begin{proposition}
כדי לממש שער Control ניתן לבחור \(\omega_{L}=\Omega-\omega_{1}\). כאשר במקרה זה נקבל כי הגורמים של carrier הם זניחים, ויש רזונס עם אופן מרכז המסה ובקירוב הגל המסתובב נקבל:
$$\tilde{H}\approx\tilde{H}_{\mathrm{sideband}}\approx i\mu e^{i\theta}\eta a^{\dagger}\sigma_{-}+h.c.$$

\end{proposition}
\begin{example}[מודל ג'יימס קומינגס]
נתון המילטוניאן מהצורה:
$$H=\omega\left(a^{\dagger}a+\sigma_{+}\sigma_{-}\right)+\left(\gamma a^{\dagger}\sigma_{-}+\overline{{{\gamma}}}\sigma_{+}a\right)\equiv H_{0}+H_{\mathrm{int}}$$
וכן נגדיר:
$$Q=a^{\dagger}a+\sigma_{+}\sigma_{-}$$
נשים לב ראשית כי \(Q\) הוא גודל שמור כיוון ש:
$$H_{0}=\omega Q\implies [H_{0},Q]=\left[ \omega Q,Q \right]=0$$
ולכן מספיק להראות כי \(H_{\text{int}}\) מתחלף עם \(Q\). זה מתקיים מטיעונים פיסיקלים. נשים לב כי \(a^{\dagger}\sigma_{-}\) מעלה את \(a^{\dagger}a\) ב-1 ומוריד את \(\sigma_{+}\sigma_{-}\) ב-1. ולכן משמר את \(Q\) ומתחלף איתו. באופן דומה \(\sigma_{+}a\) מוריד את \(a^{\dagger}a\) ב-1 ומעלה את \(\sigma_{+}\sigma_{-}\) ב-1 ולכן מתחלף עם \(Q\). לכן בפרט אם מצב מתחיל באיזהו מצב עם ערך \(Q\) הוא ישאר עם אותו עם \(Q\).
הערכים העצמיים של \(Q\) יהיו הסכום של הערכים העצמיים של \(\sigma_{+}\sigma_{-}=\frac{1}{2}\left( \mathbb{1}+\sigma_{z} \right)\) ושל \(a^{\dagger}a\).
ניתן ללכסן את \(H_{\text{int}}\) ולקבל את היחס:
\begin{gather*}e^{-i H_{\mathrm{int}}t}\left|g\right\rangle\left|1\right\rangle=\cos\left(\mu\eta t\right)\left|g\right\rangle\left|1\right\rangle-e^{-i\theta}\sin\left(\mu\eta t\right)\left|e\right\rangle\left|0\right\rangle\\ e^{-i H_{\mathrm{int}}t}\left|e\right\rangle\left|0\right\rangle=\cos\left(\mu\eta t\right)\left|e\right\rangle\left|0\right\rangle+e^{i\theta}\sin\left(\mu\eta t\right)\left|g\right\rangle\left|1\right\rangle 
\end{gather*}

\end{example}
\Chapter{אלגוריתמים קוונטים}

\section{בעיות קבועה-מאוזנת}

\begin{definition}[קופסא שחורה]
פונקציה בוליאנית \(f:\{ 0,1 \}^{m}\to \{ 0,1 \}^{n}\) כאשר \(m,n \in \mathbb{N}_{0}\). אנחנו יודעים מה הפונקציה הזאת עושה אך איננו יודעים איך מבצעת אותה.

\end{definition}
\begin{definition}[קריאה ל-\(f\) - Query]
הרצה של הפונקציה של הקופסא השחורה. תחת המודל שלנו אלגוריתם יותר יעיל ככל שמבצע כמות יותר קטנה של קריאות ל-\(f\).

\end{definition}
\begin{proposition}[ייצוג של הקופסא השחורה]
נרצה לייצג את הקוספא השחורה בעזרת אופרטור אוניטרי. אופרטור זה יהיה מהצורה:
$$U_{f}\left|x,y\right\rangle=\left|x,y\oplus f\left(x\right)\right\rangle$$
כאשר \(\oplus\) מייצג \(XOR\) כלומר חיבור במודולו 2.

\end{proposition}
\begin{definition}[בעיית דויטש-ג'וזה]
נניח כי יש קוספא שחורה מהצורה \(f:\{ 0,1 \}^{n}\to \{ 0,1 \}\). נניח בנוסף כי \(f\) יכול להיות בדיוק באחד משתי המצבים הבאים:

  \begin{enumerate}
    \item קבועה - לכל קלט \(x\) מתקיים \(f(x)=0\) או שלכל קלט \(x\) מתקיים \(f(x)=1\). 


    \item מאוזנת - בדיוק על חצי מהקלטים מתקבל 0 ועל החצי השני מתקבל 1. כלומר אם נסמן \(N=2^{n}\) נקבל: 
$$\left|\left\{x\in\left\{0,1\right\}^n\,|\,f\left(x\right)=0\right\}\right|=\left|\left\{x\in\left\{0,1\right\}^n\,|\,f\left(x\right)=1\right\}\right|=\frac{N}{2}$$


  \end{enumerate}
\end{definition}
ראשית נפתור בעיה פשוטה יותר.

\begin{definition}[בעיית דויטש]
להכריע אם פונקציה \(f:\{ 0,1 \}\to\{ 0,1 \}\) היא קבועה או מאוזנת.

\end{definition}
נשים לב כי במקרה הקלאסי נדרשים בדיוק שתי מדידות.

\begin{remark}
כפי שנראה בהמשך, זה למעשה התמרת פורייה בוליאנית.

\end{remark}
\begin{lemma}
$$U_{f}\ket{x} \otimes \ket{-} =\ket{x} \otimes (-1)^{f(x)}\ket{-} $$

\end{lemma}
\begin{proof}
\begin{gather*}U_{f}\ket{x} \otimes \ket{-} =\frac{1}{\sqrt{ 2 }}\left( U_{f}\ket{x} \otimes \ket{0} -U_{f }\ket{x} \otimes \ket{1}  \right)= \\=\frac{1}{\sqrt{ 2 }}\left( \ket{x} \otimes \ket{f(x)}  \right)-\ket{x } \otimes \ket{f(x)\oplus 1} =\ket{x} \otimes \begin{cases}\ket{-}  & f(x)=0 \\-\ket{-} & f(x)=1 \end{cases}  \\\ket{x} \otimes (-1)^{f(x)}\ket{-} 
\end{gather*}

\end{proof}
\begin{proposition}[פתרון בעיית דויטש]
המעגל הבא פותר את בעיית דויטש:

 Created with Inkscape (http://www.inkscape.org/) \includegraphics[width=0.8\textwidth]{diagrams/svg_6.svg}
\end{proposition}
\begin{proof}
האולגוריתם מקבל קלט מהצורה \(\ket{0}\ket{1}\) ופועל בצורה הבאה:

  \begin{enumerate}
    \item נתחיל מלהפעיל שער אדמר \(H\otimes H\) ולקבל: 
$$H\left|0\right\rangle=\frac{1}{\sqrt{2}}\left( \left|0\right\rangle+\left|1\right\rangle \right)=\ket{+} ,\quad H\left|1\right\rangle=\frac{1}{\sqrt{2}}\left( \left|0\right\rangle-\left|1\right\rangle \right)=\ket{-} $$
כך שהמצב הכולל נהיה:
$${\frac{1}{2}}\left( |0\rangle+|1\rangle \right)\left( |0\rangle-|1\rangle \right)=\ket{+} \ket{-} $$


    \item נפעיל את הקופסא השחורה \(U_{f}\ket{x}\ket{y}=\ket{x}\ket{y\oplus f(x)}\) ונקבל: 
$$\frac{1}{2}\left((-1)^{f(0)}\left|0\right\rangle+(-1)^{f(1)}\left|1\right\rangle\right)\left(\left|0\right\rangle-\left|1\right\rangle\right)$$
כאשר \((-1)^{f(0)}\) ו-\((-1)^{f(1)}\) מתקבלים מהלמה מהפעלה של \(U_{f}\) על המצבים \(\ket{0}\otimes \ket{-}\) ו-\(\ket{1}\otimes \ket{ -}\) בהתאמה.


    \item נפעיל אדמר על הקיביט הראשון ונקבל עבור הקיוביט הראשון: 
$${\frac{1}{\sqrt{2}}}\left[(-1)^{f(0)}{\frac{1}{\sqrt{2}}}(|0\rangle+|1\rangle)+(-1)^{f(1)}{\frac{1}{\sqrt{2}}}(|0\rangle-|1\rangle)\right]$$
כאשר אם נאחד את הגורמים של \(\ket{0}\) ושל \(\ket{1}\) נקבל כי המצב הכולל(שתי הקיובטים ביטים):
$${\frac{1}{2}}\Bigg[\left((-1)^{f(0)}+(-1)^{f(1)}\right)|0\rangle+\left((-1)^{f(0)}-(-1)^{f(1)}\right)|1\rangle\Bigg]\otimes{\frac{1}{\sqrt{2}}}(|0\rangle-|1\rangle).$$
כאשר נשים לב כי אם הפונקציה קבועה נקבל \(f(0)=f(1)\) ולכן המקדם של \(\ket{1}\) נהיה 0, ואם \(f(0)\neq f(1)\) נקבל כי המקדם של \(\ket{0}\) יהיה 0.
כלומר מדידה של \(\ket{0}\) בקיוביט הראשון יתן לנו שהפונקציה קבועה, כאשר מדידה של \(\ket{1}\) בקיוביט הראשון תתן לנו שהפונקציה מאוזנת.


  \end{enumerate}
\end{proof}
\begin{remark}
אם אנחנו יכולים לאתחל ולמדוד בבסיס של \(X\) העבודה קצת יותר קלה. מתקיים:
$$U_{f}\left|+\right\rangle\otimes\left|-\right\rangle=\frac{1}{\sqrt{2}}U_{f}\left(\left|0\right\rangle\otimes\left|-\right\rangle+\left|1\right\rangle\otimes\left|-\right\rangle\right)=\frac{1}{\sqrt{2}}\left((-1)^{f(0)}\left|0\right\rangle+(-1)^{f(1)}\left|1\right\rangle\right)\otimes\left|-\right\rangle.$$
ואם נבטא בבסיס \(\ket{\pm}\) נקבל:
$$U_{f}\left|+\right\rangle\otimes\left|-\right\rangle=\left(\left(\frac{(-1)^{f(0)}+(-1)^{f(1)}}{2}\right)\left|+\right\rangle+\left(\frac{(-1)^{f(0)}-(-1)^{f(1)}}{2}\right)\left|-\right\rangle\right)\otimes\left|-\right\rangle$$
וקיבלנו את אותו ביטוי ללא הפעלה של כל האדמר.

\end{remark}
כעת נחזור לבעיית דויטש ג'וזה.

\begin{definition}[שער אדמר מוכלל]
$${ H}^{(n)}={ H}\otimes{ H}\otimes\ldots\otimes{ H}$$
ניתן גם לכתוב \(H^{\otimes n}\).

\end{definition}
\begin{corollary}
מצב של \(n\) קיוביטים תעבור בצורה הבאה:
$${H}^{(n)}:|x\rangle\mapsto\prod_{i=1}^{n}\left({\frac{1}{\sqrt{2}}}\sum_{y_{i}=\{0,1\}}(-1)^{x_{i}y_{i}}|y_{i}\rangle\right)\equiv{\frac{1}{2^{n/2}}}\sum_{y=0}^{2^{n}-1}(-1)^{x\cdot y}|y\rangle$$
כאשר \(x\cdot y\) מסמן AND או כפל מודולו 2:
$$x\cdot y=(x_{1}\wedge y_{1})\oplus(x_{2}\wedge y_{2})\oplus\ldots\oplus(x_{n}\wedge y_{n})$$

\end{corollary}
\begin{remark}
פה ממש כבר אפשר לראות שזוהי התמרת פורייה בינארית - כאשר ערכים במעריך יהיו \(\left\{  e^{ i 0 }=1,e^{ i\pi }=-1  \right\}\).

\end{remark}
\begin{lemma}
$$\frac{1}{2^{n}}\sum_{x=0}^{2^{n}-1}(-1)^{x\cdot a}(-1)^{x\cdot y}=\delta_{y,a}$$

\end{lemma}
\begin{proof}
ניתן לפשט את האספוננט כך שנקבל:
$$(-1)^{x\cdot a}(-1)^{x\cdot y}=(-1)^{x\cdot(a+y)}$$
כאשר \(a+y\) בבינארי זה כמו XOR. 
- עבור \(y=a\) נקבל כי כי יש 0 באקספוננט ולכן כל הגורמים יהיו 1, ולכן נקבל \(\frac{1}{2^{n}}\cdot 2^{n}=1\).
- עבור \(y\neq a\) נקבל כי יש \(2^{n-1}\) איברים בהם \(x\cdot(a+y)=0\) ולכן נסכום על 1 ו-\(2^{n-1}\) מרים בהם \(x\cdot(a+y)=1\) ולכן נסכום על \(-1\). 
סה"כ נקבל 0.

\end{proof}
\begin{proposition}[פתרון בעיית דויטש ג'וזה]
המעגל הבא פותר את בעיית דויטש ג'וזה:

 Created with Inkscape (http://www.inkscape.org/) \includegraphics[width=0.8\textwidth]{diagrams/svg_7.svg}
\end{proposition}
\begin{proof}
נתחיל מהקלט \(\ket{0}^{\otimes n}\ket{1}\). 

  \begin{enumerate}
    \item נפעיל ראשית \(H^{(n)}\otimes H\) ונקבל: 
$${\frac{1}{2^{n/2}}}\sum_{x=0}^{2^{n}-1}|x\rangle\otimes{\frac{1}{\sqrt{2}}}{\big(}|0\rangle-|1\rangle{\big)}$$


    \item נפעיל כעת את \(U_{f}\) ונקבל: 
$${\frac{1}{2^{n/2}}}\sum_{x=0}^{2^{n}-1}(-1)^{f(x)}|x\rangle\otimes{\frac{1}{\sqrt{2}}}(|0\rangle-|1\rangle)$$


    \item כעת נפעיל \(H^{(n)}\otimes \mathbb{1}\) ונקבל: 
$${\frac{1}{2^{n}}}\sum_{x=0}^{2^{n}-1}\sum_{y=0}^{2^{n}-1}(-1)^{f(x)+x\cdot y}|y\rangle\otimes{\frac{1}{\sqrt{2}}}(|0\rangle-|1\rangle)$$


    \item נמדוד את ה-\(n\) קיוביטים הראשונים ונקבל: 
$${\frac{1}{2^{n}}}\sum_{y=0}^{2^{n}-1}{ \left(\sum_{x=0}^{2^{n}-1}(-1)^{f(x)+x\cdot y}\right) }|y\rangle={\frac{1}{2^{n}}}\sum_{y=0}^{2^{n}-1}\underbrace{ \left(\sum_{x=0}^{2^{n}-1}(-1)^{f(x)}(-1)^{x\cdot y}\right) }_{ (*) }|y\rangle$$


    \item אם \(f\) קבועה אז הפאזה \((-1)^{f(x)}\) קבוע לכל \(x\) כאשר \((*)\) יתן: 
$$\sum_{x=0}^{2^{n}-1}(-1)^{x\cdot y}={\begin{cases}2^{n}&{\mathrm{if~}}y=0,\\ 0&{\mathrm{otherwise.}}\end{cases}}$$
כלומר ההסתברות לקרוס למצב בו \(\ket{y}=0\) תהיה 1 ולכן ה-\(n\) קיוביטים קורסים למצב \(\ket{0}^{\otimes n}\).


    \item אם \(f\) מאוזנת אז \((-1)^{f(x)}=1\) עבור חצי מהקלטים ו-\(-1\) עבור החצי השני. עבור \(y=0\) נקבל: 
$$\sum_{x=0}^{2^{n}-1}(-1)^{f(x)}=\sum_{\underbrace{x:f(x)=0}_{2^{n-1}}}1+\underbrace{\sum_{x:f(x)=1}(-1)}_{-2^{n-1}}=0.$$
ולא ייתכן כי נקבל \(\ket{0}^{\otimes n}\). 


  \end{enumerate}
\end{proof}
\begin{example}
נתונות לנו שתי פונקציות \(f_{1,2}:\{0,1\}^{n}\rightarrow\{0,1\}\) אשר יכולות לקיים או ששווה לכל ערך (\(f_{1}(x)=f_{2}(x)\) לכל \(x\)) או ששווה לחצי מהערכים ועבור החצי השני מקיימות \(f_{1}(x)=1-f_{2}(x)\). נגדיר את הפונקציה:
$$g:\left\{0,1\right\}^{n}\longrightarrow\left\{0,1\right\}\qquad g\left(x\right)=f_{1}\left(x\right)\oplus f_{2}\left(x\right)$$
ונשים לב כי זוהי פונקציה קבועה אם שוות לכל ערך ופונקציה מאוזנת אם שוות לחצי מהערכים. נשים לב כי אם:
$$U_{f_1}\ket{x}\otimes y =\ket{x} \otimes \ket{y\oplus f_{1}(x)}\qquad U_{f_2}\ket{x}\otimes y =\ket{x} \otimes \ket{y\oplus f_{2}(x)} $$
אזי:
$$U_{f_1}U_{f_2}\ket{x} \otimes \ket{y} =U_{f_1}\left( \ket{x} \otimes \ket{y\oplus f_{2}(x)}  \right)=\ket{x} \otimes \ket{y \oplus\left( f_{1}(x)\oplus f_{2}(x) \right)}=U_{f_2}U_{f_1}\ket{x} \otimes \ket{y}  $$
ולכן ניתן לפתור בעיה זו עם דויטש-ג'וזה כאשר הקופסא השחורה שלנו תהיה \(U_{g}=U_{f_{1}}U_{f_{2}}\).

\end{example}
\begin{definition}[בעיית ברנשטיין-ואזיריאני]
נתונה פונקציה \(f:\{ 0,1 \}^{n}\to\{ 0,1 \}\) כך ש-\(f\left( \mathbf{x} \right)=\mathbf{a}\cdot \mathbf{x}\). אנחנו רוצים למצוא את \(a\).

\end{definition}
\begin{proposition}
ניתן לפתור את בעיית ברנשטיין-ואזיריאני עם אותו מעגל כמו בבעיית דויטש ג'וזה. נקבל בסוף המעגל עבור ה-\(n\) קיוביטים הראשונים:
$${\frac{1}{2^{n}}}\sum_{y=0}^{2^{n}-1}{ \left(\sum_{x=0}^{2^{n}-1}(-1)^{f(x)}(-1)^{x\cdot y}\right) }|y\rangle$$
כאשר כיוון ש-\(f(x)=ax\) ניתן לכתוב:
$$\sum_{y=0}^{2^{n}-1}\left[{\frac{1}{2^{n}}}\sum_{x=0}^{2^{n}-1}\left(-1\right)^{x\cdot a}\left(-1\right)^{x\cdot y}\right]|y\rangle$$
כאשר מהלמה שהראנו נקבל כי בסוגרים יש \(\delta_{a,y}\) ולכן המצב שלנו יהיה:
$$\sum_{y=0}^{2^{n}-1}\delta_{a,y}|y\rangle=\ket{a} $$

\end{proposition}
\begin{remark}
אם נקבל פונקציה מהצורה \(f\left( \mathbf{x} \right)=\mathbf{a}\cdot \mathbf{x}+\mathbf{b}\) אז ה-\(\mathbf{b}\) יתן רק פאזה גלובאלית, אך עם נרצה למצוא בכל זאת ניתן להריץ \(f(0)=\mathbf{b}\) אפילו קלאסית.

\end{remark}
\begin{summary}
  \begin{itemize}
    \item בעיית דויטש זה הבעיה של לבדוק פונקציה \(f:\{ 0,1 \}\to \{ 0,1 \}\) אם היא קבועה או מאוזנת.
    \item נפתר על ידי מעגל אשר מקבל קלט מהצורה \(\ket{0}\ket{1}\) ומבצע את הפעולות הבאות:
\begin{gather*}|0\rangle|1\rangle\xrightarrow{H\otimes H}\frac{1}{2}(|0\rangle+|1\rangle)(|0\rangle-|1\rangle)\\ \xrightarrow{U_{f}}\frac{1}{2}\left((-1)^{f(0)}|0\rangle+(-1)^{f(1)}|1\rangle\right)\left(|0\rangle-|1\rangle\right)\\ \xrightarrow{H\otimes \mathbb{1} }\frac{1}{2}\Bigg[\left((-1)^{f(0)}+(-1)^{f(1)}\right)|0\rangle+\left((-1)^{f(0)}-(-1)^{f(1)}\right)|1\rangle\Bigg]\frac{1}{\sqrt{2}}(|0\rangle-|1\rangle) 
\end{gather*}
ואם נמדוד את הקיוביט הראשון נקבל את \(\ket{0}\) בהסתברות 1 אם \(f(0)=f(1)\) (פונקציה קבועה) ואת \(\ket{1}\) בהסתברות 1 אם \(f(0)\neq f(1)\)(פונקציה מאוזנת). 
    \item בעיית דויטש-ג'וזה זה הבעיה של לבדוק האם פונקציה \(f:\{ 0,1 \}^{n}\to \{ 0,1 \}\) תהיה קבועה או מאוזנת(כאשר בהכרח אחד מהם).
    \item נפתר על ידי מעגל אשר מקבל קלט מהצורה \(\ket{0}^{n}\ket{1}\) ומבצע את הפעולות הבאות:
\begin{gather*}{{\left( |0\rangle \right)^{n}|1\rangle\xrightarrow{H^{(n)}\otimes H}\left(\frac{1}{2^{n/2}}\sum_{x=0}^{2^{n}-1}|x\rangle\right)\frac{1}{\sqrt{2}}\left( |0\rangle-|1\rangle \right)}}\\ {{\xrightarrow{U_{f}}\left(\frac{1}{2^{n/2}}\sum_{x=0}^{2^{n}-1}(-1)^{f(x)}|x\rangle\right)\frac{1}{\sqrt{2}}\left( |0\rangle-|1\rangle \right)}}\\ {{\xrightarrow{H^{(n) }\otimes \mathbb{ 1}}\left(\frac{1}{2^{n}}\sum_{x=0}^{2^{n}-1}\sum_{y=0}^{2^{n}-1}(-1)^{f(x)}(-1)^{x\cdot y}|y\rangle\right)\frac{1}{\sqrt{2}}\left( |0\rangle-|1\rangle \right)}} 
\end{gather*}
כאשר נמדוד את ה-\(n\) קיוביטים הראשונים. אם נקבל \(\ket{0}^{\otimes n}\) אז קבוע אחרת מאוזן.
    \item ניתן לפתור את בעיית ברנשטיין-וזיראני של למצוא את ווקטור \(\mathbf{a}\) של פונקציה מהצורה \(f:\{ 0,1 \}^{n}\to\{ 0,1 \}\) המוגדרת על ידי \(f\left( \mathbf{x} \right)=\mathbf{a}\cdot \mathbf{x}\) בעזרת אותו המעגל, כאשר הפלט שלנו יהיה פשוט \(\ket{a}\).
  \end{itemize}
\end{summary}
\section{מחזוריות}

\section{בעיית סימון}

\begin{definition}[בעיית סימון]
נתון פונקציה \(f:\{ 0,1 \}^{n}\to \{ 0,1 \}^{n}\) שהיא דו-חד ערכית(\(\text{2 to 1}\)) ומחזורית עם מחזור שאנחנו רוצים למצוא. המחזור מוגדר על ידי \(f(x)=f(y)\) אם"ם \(x=y\) או \(x\oplus a =y\).

\end{definition}
\begin{proposition}[פתרון קוונטי של בעיית סימון]
ניתן לפתור את בעיית סימון בעזרת המעגל הבא:

 Created with Inkscape (http://www.inkscape.org/) \includegraphics[width=0.8\textwidth]{diagrams/svg_8.svg}
\end{proposition}
כאשר ל-\(\ket{0}^{\otimes n}\) הראשונים נקרא הרג'יסטר הראשון ול-\(\ket{0}^{\otimes n}\) השני נקרא הרג'יסטר השני. אם נמדוד כעת בבסיס החישובי נקבל כי עבור \(y\) התוצאה \(a\cdot y = 1\) לא יכולה להתקבל, כאשר התוצאה \(a\cdot y = 0\) תתקבל בסתברות שווה \(2^{-(n-1)}\). נריץ שוב ושוב ונקבל \(n\) תוצאות \(\left\{  y_{1},\dots,y_{n}  \right\}\)  אשר בלתי תלויות לינארית מעל \(\mathbb{Z}_{2}^{n}\) וניתן לפתור את המערכת \(\left\{  a\cdot y_{i}=0  \right\}\)

\begin{proof}
הקלט הוא 2 רג'יסטרים של \(n\) קיוביטים, כלומר מהצורה \(\ket{0}^{\otimes n}\otimes \ket{0}^{\otimes n}\). נפעיל עליו עם אדמר ונקבל:
$$\left|0\right\rangle^{\otimes n}\otimes\left|0\right\rangle^{\otimes n}\longrightarrow\left({\frac{1}{2^{n/2}}}\sum_{x=0}^{2^{n}-1}\left|x\right\rangle\right)\otimes\left|0\right\rangle^{\otimes n}$$
כעת נבצע את הקריאה שלנו - נפעיל את ההעתקה האוניטרית של הקופסא השחורה: \(U_{f}\left|x,y\right\rangle=\left|x,y\oplus f\left(x\right)\right\rangle\) ונקבל:
$${\frac{1}{2^{n/2}}}\sum_{x=0}^{2^{n}-1}\left|x\right\rangle\otimes\left|f\left(x\right)\right\rangle$$
כעת אם נמדוד את הרגיסטר השני נקבל את \(f(x_{0})\). לפי האינפורמציה שיש לנו יש איזשהו קבוע לא ידוע \(a\) כך ש \(f(x_{0})=f\left( x_{0}\oplus a \right)\). המצב לאחר המדידה יהיה:
$${\frac{1}{\sqrt{2}}}\left(\left|x_{0}\right\rangle+\left|x_{0}\oplus a\right\rangle\right)\otimes\left|f\left(x_{0}\right)\right\rangle$$
למדוד את הרג'יסטר הראשון בבסיס החישובי לא יתן מידע, לכן ניתן לפעיל שוב את האדמר(הרג'יסטר השני כבר לא מעניין אותנו כי נמדד) ונקבל:
$${\frac{1}{\sqrt{2}}}\left(|x_{0}\rangle+|x_{0}\oplus a\rangle\right)\longrightarrow{\frac{1}{2^{(n+1)/2}}}\sum_{y=0}^{2^{n}-1}\left(\left(-1\right)^{x_{0}\cdot y}+\left(-1\right)^{(x_{0}\oplus a)\cdot y}\right)|y\rangle$$
כאשר נשים לב כי:
$$(-1)^{(x_{0}\oplus a)\cdot y}=(-1)^{x_{0}\cdot y}\left(-1\right)^{a\cdot y}$$
ולכן הרגיסטר הראשון יהיה:
$${\frac{1}{2^{(n-1)/2}}}\sum_{y:a\cdot y=0}\left(-1\right)^{x_{0}\cdot y}|y\rangle$$
נשים לב כי אם נמדוד כעת בבסיס החישובי נקבל כי עבור \(y\) התוצאה \(a\cdot y = 1\) לא יכולה להתקבל, כאשר התוצאה \(a\cdot y = 0\) תתקבל בסתברות שווה \(2^{-(n-1)}\). נריץ שוב ושוב ונקבל \(n\) תוצאות \(\left\{  y_{1},\dots,y_{n}  \right\}\)  אשר בלתי תלויות לינארית מעל \(\mathbb{Z}_{2}^{n}\) וניתן לפתור את המערכת \(\left\{  a\cdot y_{i}=0  \right\}\) ולמצוא את \(a\).

\end{proof}
\begin{example}
נתונות שתי פונקציות בוליאניות הפועלות על \(n\) קיוביטים:
$$f_{1,2}:\left\{0,1\right\}^{n}\longrightarrow\left\{0,1\right\}$$
כאשר ידוע כי קיים \(a=\left( a_{1},\dots, a_{n} \right)\) כך ש-\(f_{2}(x)=f_{1}\left( x\oplus a \right)\). נשים לב כי זה גורר כי:
$$f_{1}(x)=f_{1}\left( x\oplus 0 \right)=f_{1}\left( x\oplus a\oplus a \right)=f_{2}\left( x\oplus a \right)$$
ולכן אם נגדיר:
$$g:\left\{0,1\right\}^{n}\longrightarrow\left\{0,1\right\}\qquad g\left(x\right)=f_{1}\left(x\right)\oplus f_{2}\left(x\right)$$
נקבל כי זהוי בעיה של מציאת מחזור אשר נפתרת עם האלגוריתם של בעיית סימון על ידי הקופסא השחורה \(U_{g}=U_{f_{1}}U_{f_{2}}=U_{f_{2}}U_{f_{1}}\).

\end{example}
\section{מציאת מחזור כללי}

\begin{definition}[מחזור של פונקציה]
נניח כי ל-\(f:\{ 0,1 \}^{n}\to\{ 0,1 \}^{m}\) יש מחזור לא יודע \(r\) כאשר \(1< r < 2^{n}\) ומקיים:
$$f(x)=f(x+mr)$$

\end{definition}
\begin{remark}
קלאסית, בעיה זו נחשבת קשה.

\end{remark}
כדי לפתור זאת ראשית נגדיר התמרת פורייה קוונטי.

\begin{definition}[התמרת פורייה קוונטית]
טרנספורמציה אוניטרית אשר פועלת על הבסיס החשובי באופן הבא:
$$Q F T:|x\rangle\to{\frac{1}{\sqrt{N}}}\sum_{y=0}^{N-1}e^{2\pi i x y/N}|y\rangle,$$
כאשר \(N=2^{n}\). למעשה מפעילה התמרת פורייה דיסקרטית.

\end{definition}
כעת נפתור את בעיית המחזור הכללי.

 Created with Inkscape (http://www.inkscape.org/) \includegraphics[width=0.8\textwidth]{diagrams/svg_9.svg}
כאשר נקבל בקירוב טוב \(\frac{y}{2^{n}}\approx \frac{k}{R}\), וניתן לחלץ את \(R\) בעזרת אלגוריתם קלאסי.

\begin{proof}
  \begin{enumerate}
    \item נתחיל עם שתי רג'יסטרים \(\ket{0}^{\otimes n}\otimes \ket{0}^{\otimes m}\) כאשר הרג'יסטר הראשון יאכסן את הקלט \(x\) והרג'יסטר השני יאכסן את הפלט \(f(x)\). 


    \item נפעיל את שער ההדבר על הרג'יסטר הראשון: 
$$|0\rangle^{\otimes n}\xrightarrow{H^{\otimes n}}\frac{1}{2^{n/2}}\sum_{x=0}^{2^{n}-1}|x\rangle\,.$$
כך שנקבל שהמצב הכולל שלנו יהיה:
$$\left({\frac{1}{2^{n/2}}}\sum_{x=0}^{2^{n}-1}|x\rangle\right)\otimes|0\rangle^{\otimes m}$$


    \item נפעיל את הקופסא השחורה \(U_{f}\) על הרג'יסטר השני: 
$$U_{f}\left|x\right\rangle\left|0\right\rangle^{\otimes m}=\left|x\right\rangle\left|f(x)\right\rangle$$
כך שהמצב הכולל יהיה:
$${\frac{1}{2^{n/2}}}\sum_{x=0}^{2^{n}-1}|x\rangle\otimes|f(x)\rangle$$


    \item כעת אפשר למדוד את הרג'יסטר השני. נקבל \(f(x_{0})\) עבור איזשהו \(x_{0}\). יש הרבה מצבים(\(x_{0},x_{0}+R,x_{0}+2R, \dots,x_{0}+(N_{0}-1)R\)) שמתאימים למצב הזה, לכן זה יקריס את הרג'יסטר הראשון למצב הבא: 
$$\frac{1}{\sqrt{N_{0}}}\sum_{\ell=0}^{N_{0}-1}|x_{0}+\ell R\rangle$$


    \item נפעיל את ההתמרת הפורייה הקטוונטית \(QFT_{2^{n}}\) עבור הרג'יסטר הראשון כי להפוך את הסופרפוזציה של המחזור לנקודות מקסימום עבור התדירות: 
$$\frac{1}{\sqrt{2^{n}N_{0}}}\sum_{y=0}^{2^{n}-1}\sum_{\ell=0}^{N_{0}-1}e^{i\frac{2\pi}{2^{n}}(x_{0}+\ell R)y}\mid\!\!y\!\rangle$$
נפשט את האמפלטודה עבור \(\ket{y}\) ונקבל:
$$p(y)=\frac{1}{2^{n}N_{0}}\left|\sum_{\ell=0}^{N_{0}-1}e^{i\frac{2\pi}{2^{n}}\ell R y}\right|^{2}$$


    \item כעת אם \(R\) מחלק את \(2^{n}\) נקבל כי נקודות קיצון מתרחשות ב-\(y=k\cdot \frac{2^{n}}{R}\) עם הסתברות של \(\frac{1}{R}\). אחרת \(R\) לא מחלק את \(2^{n}\) ונצברים נקודות מקסימום סביב כפולות של \(\frac{2^{n}}{R}\). ההתפלגות ההסתברות מקיימת: 
$$\left|{\frac{y}{2^{n}}}-{\frac{k}{R}}\right|\leq{\frac{1}{2^{n+1}}}$$
עבור איזשהו \(k \in \mathbb{Z}\). זה מאפשר לשחזר את \(R\). כלומר נקבל בקירוב טוב \(\frac{y}{2^{n}}\approx \frac{k}{R}\).


    \item אנחנו אומנם יודעים רק את \(y\) ו-\(2^{n}\) ניתן להשתמש באלגוריתם קלאסי כדי לחלץ את \(R\) בעזרת זה שמתמשים בכך ש-\(k \in \mathbb{Z}\). 


  \end{enumerate}
\end{proof}
\begin{summary}
  \begin{itemize}
    \item בעיית סימון היא בעיה שבה נתונה לנו פונקציה שהיא דו-חד ערכית ואנחנו מוצאים את המחזור שלה. מבצוע בצורה הבאה:
\begin{gather*}\ket{0}^{\otimes n} \otimes \ket{0}^{\otimes n} \xrightarrow{H^{\otimes n} \otimes I} \left( \frac{1}{2^{n/2}} \sum_{x=0}^{2^n - 1} \ket{x} \right) \otimes \ket{0}^{\otimes n} \xrightarrow{U_f}  \\\frac{1}{2^{n/2}} \sum_{x=0}^{2^n - 1} \ket{x} \otimes \ket{f(x)} \xrightarrow{\text{Measure second register}} \frac{1}{\sqrt{2}} \left( \ket{x_0} + \ket{x_0 \oplus a} \right) \xrightarrow{H^{\otimes n}}  \\\frac{1}{2^{(n+1)/2}} \sum_{y=0}^{2^n - 1} \left( (-1)^{x_0 \cdot y} + (-1)^{(x_0 \oplus a) \cdot y} \right) \ket{y} \xrightarrow{\text{Simplify phase}} \frac{1}{2^{(n-1)/2}} \sum_{y: a \cdot y = 0} (-1)^{x_0 \cdot y} \ket{y}.
\end{gather*}
    \item עבור פונקציה כללית ניתן להכליל את בעיית סימון בעזרת התמרת פורייה קוונטית, ולקבל את האלגוריתם הבא:
\begin{gather*}\ket{0}^{\otimes n} \otimes \ket{0}^{\otimes m}  \xrightarrow{H^{\otimes n}}   \left( \frac{1}{2^{n/2}} \sum_{x=0}^{2^n - 1} \ket{x} \right) \otimes \ket{0}^{\otimes m}  \xrightarrow{U_f}  \\ \frac{1}{2^{n/2}} \sum_{x=0}^{2^n - 1} \ket{x} \otimes \ket{f(x)}  \xrightarrow{\text{Measure second register}}  \\ \frac{1}{\sqrt{N_0}} \sum_{\ell=0}^{N_0 - 1} \ket{x_0 + \ell R}  \xrightarrow{QFT_{2^n}}  \\ \frac{1}{\sqrt{2^n N_0}} \sum_{y=0}^{2^n - 1} \sum_{\ell=0}^{N_0 - 1} e^{i \frac{2\pi}{2^n} (x_0 + \ell R) y} \ket{y}. 
\end{gather*}
אם \(R\) מחלק את \(2^{n}\) נקבל קיצון ב-\(y=k\cdot \frac{2^{n}}{R}\) אחרת נקבל בקירוב טוב \(\frac{y}{2^{n}}\approx \frac{k}{R}\) אשר ניתן לפתור בצורה קלאסית.
  \end{itemize}
\end{summary}
\section{התמרת פורייה קוונטית}

\begin{definition}[הצגה בינארית של מספר]
ניתן לכתוב כל מספר על ידי ההצגה הבינארית שלו:
$$j=j_{1}2^{n-1}+j_{2}2^{n-2}+\cdots+j_{n}$$
כאשר \(j_{i} \in \{ 0,1 \}\). ניתן גם לתאר את השבר הבינארי על ידי:
$$0.j_{1}j_{2}\cdot\cdot\cdot j_{n}=j_{1}/2+j_{2}/4+\ \cdot\cdot\cdot+j_{n}/2^{n}=j/2^{n}$$

\end{definition}
\begin{definition}[התמרת פורייה דיסקרטית]
מקבל ווקטור \(x_{0},\dots,x_{N-1}\) מאורך \(N\) של ווקטורים עם פרמטים קבועים, ומתמיר את הווקטור באופן הבא:
$$y_{k}\equiv\frac{1}{\sqrt{N}}\sum_{j=0}^{N-1}x_{j}e^{2\pi i j k/N}$$

\end{definition}
\begin{definition}[התמרת פורייה קוונטית]
ביצוע של התמרת פורייה דיסקרטי על האמפליטודות של מצב קוונטי:
$$\sum_{j=0}^{N-1}x_{j}|j\rangle\longrightarrow\sum_{k=0}^{N-1}y_{k}|k\rangle$$
כאשר למשל בבסיס החישובי נקבל:
$$\left|j\right\rangle\longrightarrow\frac{1}{\sqrt{N}}\sum_{k=0}^{N-1}e^{2\pi i j k/N}\left|k\right\rangle$$

\end{definition}
\begin{proposition}
התמרת פורייה קוונטית הוא אופרטור אוניטרי.
$$\hat{F}=\sum_{j,k=0}^{N-1}\frac{e^{2\pi i j k/N}}{\sqrt{N}}|k\rangle\langle j|$$

\end{proposition}
\begin{proof}
\begin{gather*}{{\hat{F}^{\dagger}\hat{F}}}={{\frac{1}{N}\sum_{j,k,j^{\prime},k^{\prime}}e^{2\pi i(j^{\prime}k^{\prime}-j k)/N}|j\rangle\langle j^{\prime}|\delta_{k k^{\prime}}}}\\ ={{\frac{1}{N}\sum_{j,k,j^{\prime}}e^{2\pi i(j^{\prime}-j)k/N}|j\rangle\langle j^{\prime}|}}\\ ={{\sum_{j,j^{\prime}}|j\rangle\langle j^{\prime}|\delta_{j j^{\prime}}=\sum_{j}|j\rangle\langle j|=\hat{I}.}} 
\end{gather*}

\end{proof}
\begin{proposition}
ניתן לכתוב כל ווקטור מותמר יהיה סופרפוזציה שווה אפליטודה של כל רכיב שלו בבסיס החישובי.

\end{proposition}
\begin{proof}
\begin{gather*}{|\!\left\langle \hat{x}|y \right\rangle|^{2}}={{\left\langle  y|\hat{x} \right\rangle\langle \hat{x}|y \rangle=\langle  y|\hat{F}|x \rangle\langle  x|\hat{F}^{\dagger}|y \rangle}} = \\\frac{e^{ 2\pi ixy/ N }}{\sqrt{ N }}\frac{e^{ -2\pi ixy/ N }}{\sqrt{ N }}=\frac{1}{N}
\end{gather*}

\end{proof}
\begin{proposition}
ניתן לכתוב אותו מטריציונית בצורה הבאה:
$$QFT_{M}={\frac{1}{\sqrt{M}}}\begin{pmatrix}1&1&1&1&\cdots&1\\ 1&\omega&\omega^{2}&\omega^{3}&\cdots&\omega^{M-1}\\ 1&\omega^{2}&\omega^{4}&\omega^{6}&\cdots&\omega^{2M-2}\\ 1&\omega^{3}&\omega^{6}&\omega^{9}&\cdots&\omega^{3M-3}\\ \vdots&\vdots&\vdots&\vdots&\ddots&\vdots\\ 1&\omega^{M-1}&\omega^{2M-2}&\omega^{3M-3}&\cdots&\omega^{(M-1)(M-1)}\end{pmatrix}$$
כאשר \(\omega=e^{ 2\pi i/N }\).

\end{proposition}
\begin{example}
עבור קיוביט יחיד נקבל:
$$\mathrm{QFT}=H={\frac{1}{\sqrt{2}}}\left({\begin{array}{l l}{1}&{1}\\ {1}&{-1}\end{array}}\right)$$
כלומר אדמר זה פשוט מקרה פרטי של \(QFT\). עבור שתי קיוביטים:
$$\mathrm{QFT}={\frac{1}{2}}\left[{\begin{array}{l l l l}{1}&{1}&{1}&{1}\\ {1}&{i}&{-1}&{-i}\\ {1}&{-1}&{1}&{-1}\\ {1}&{-i}&{-1}&{i}\end{array}}\right]$$

\end{example}
\begin{lemma}
ניתן לכתוב את ההתמרת פורייה גם בצורה הבאה:
$$\left|j_{1},\ldots,j_{n}\right\rangle\rightarrow\frac{\left(\left|0\right\rangle+e^{2\pi i0.j_{n}}\left|1\right\rangle\right)\left(\left|0\right\rangle+e^{2\pi i0.j_{n-1}j_{n}}\left|1\right\rangle\right)\cdots\left(\left|0\right\rangle+e^{2\pi i0.j_{1}j_{2}\cdots j_{n}}\left|1\right\rangle\right)}{2^{n/2}}\cdot$$

\end{lemma}
\begin{proof}
\begin{gather*}|j\rangle \rightarrow \frac{1}{2^{n/2}} \sum_{k=0}^{2^n-1} e^{2\pi ijk/2^n} |k\rangle = \frac{1}{2^{n/2}} \sum_{k_1=0}^1 \cdots \sum_{k_n=0}^1 e^{2\pi ij (\sum_{l=1}^n k_l 2^{l-1})} |k_1 \dots k_n\rangle \\= \frac{1}{2^{n/2}} \sum_{k_1=0}^1 \cdots \sum_{k_n=0}^1 \bigotimes_{l=1}^n e^{2\pi ijk_l 2^{l-1}} |k_l\rangle = \frac{1}{2^{n/2}} \bigotimes_{l=1}^n \left[ \sum_{k_l=0}^1 e^{2\pi ijk_l 2^{l-1}} |k_l\rangle \right] \\= \frac{1}{2^{n/2}} \bigotimes_{l=1}^n \left[ |0\rangle + e^{2\pi ij2^{l-1}} |1\rangle \right] \\= \frac{(|0\rangle + e^{2\pi i0,j_n} |1\rangle)(|0\rangle + e^{2\pi i0.j_{n-1}j_n} |1\rangle) \cdots (|0\rangle + e^{2\pi i0.j_1j_2\cdots j_n} |1\rangle)}{2^{n/2}}
\end{gather*}
כאשר השתמשנו בזהות:
$$\sum_{k_{1}=0}^{1}\cdots\sum_{k_{n}=0}^{1}\bigotimes_{l=1}^{n}f_{k_{l}}=\sum_{k_{1}=0}^{1}\cdots\sum_{k_{n}=0}^{1}f_{k_{1}}f_{k_{2}}\cdot\cdot\cdot f_{k_{n}}=\bigotimes_{l=1}^{n}\sum_{k_{l}=0}^{1}f_{k_{l}}=\sum_{k_{1}=0}^{1}f_{k_{1}}\sum_{k_{2}=0}^{1}f_{k_{2}}\cdots\sum_{k_{n}=0}^{1}f_{k_{n}}$$

\end{proof}
 Created with Inkscape (http://www.inkscape.org/) \includegraphics[width=0.8\textwidth]{diagrams/svg_10.svg}
כאשר \(R_{n}\) הוא אופרטור הסיבוב הנשלט המתואר על ידי:
$$U_{d}\equiv\left[\begin{array}{c c}{{1}}&{{0}}\\ {{0}}&{{e^{i\pi/2^{d}}}}\end{array}\right]$$

\begin{proof}
  \begin{enumerate}
    \item נבצע אדמר על הקיוביט הראשון ונקבל: 
$$\frac{1}{2^{1/2}}\left(\left|0\right\rangle+e^{2\pi i0.j_{1}}\right|1\rangle)|j_{2}\cdot\cdot\cdot j_{n}\rangle.$$


    \item נבצע את אופרטור הסיבוב הנשלט \(U_{1}\) ונקבל: 
$$\frac{1}{2^{1/2}}\left(\left|0\right\rangle+e^{2\pi i0.j_{1}j_{2}}\left|1\right\rangle\right)\left|j_{2}\cdot\cdot\cdot j_{n}\right\rangle$$
אם נמשיך להפעיל את אופרטורי הסיבוב הנשלטים בסופו של דבר נקבל:
$$\frac{1}{2^{1/2}}\left(\left|0\right\rangle+e^{2\pi i0.j_{1}j_{2}\dots j_{n}}\left|1\right\rangle\right)\left|j_{2}\cdot\cdot\cdot j_{n}\right\rangle$$


    \item אם נבצע תהליך זהה עבור הקיוביט השני נקבל לאחר הפעלת אדמר: 
$${\frac{1}{2^{2/2}}}\left(\left|0\right\rangle+e^{2\pi i0.j_{1}j_{2}\ldots j_{n}}\left|1\right\rangle\right)\left(\left|0\right\rangle+e^{2\pi i0.j_{2}}\left|1\right\rangle\right)\left|j_{3}\cdot\cdot\cdot j_{n}\right\rangle$$
כאשר הפעלת אופרטורי הסיבוב הנשלט מ-\(U_{1}\) עד \(U_{n-2}\) יתן:
$${\frac{1}{2^{2/2}}}\left(\left|0\right\rangle+e^{2\pi i0.j_{1}j_{2}\ldots j_{n}}\left|1\right\rangle\right)\left(\left|0\right\rangle+e^{2\pi i0.j_{2}\ldots j_{n}}\left|1\right\rangle\right)\left|j_{3}\ldots j_{n}\right\rangle$$


    \item נמשיך באופן הזה עבור שאר הקיוביטים ונקבל: 
$${\frac{1}{2^{n/2}}}\left(\left|0\right\rangle+e^{2\pi i0.j_{1}j_{2}\ldots j_{n}}\left|1\right\rangle\right)\left(\left|0\right\rangle+e^{2\pi i0.j_{2}\ldots j_{n}}\left|1\right\rangle\right)\ldots\left(\left|0\right\rangle+e^{2\pi i0.j_{n}}\left|1\right\rangle\right)$$


  \end{enumerate}
\end{proof}
\begin{corollary}
כמות השערים שאנחנו משתמשים יהיה:
$$n+(n-1)+(n-2)+\dots+1= \frac{n(n+1)}{2}$$
ולכן סיבוכיות המעגל הזה הוא \(O(n^{2})\).

\end{corollary}
\begin{proposition}
כיוון שהאוניטרי:
$$U_{d}=\left[\begin{array}{c c}{{1}}&{{0}}\\ {{0}}&{{e^{i\pi/2^{d}}}}\end{array}\right]$$
היא אלכסונית בבסיס החישובי, היא מהצורה שניתן להחליף control ו-target.

\end{proposition}
\begin{proposition}
אם אנחנו מדודים ישר אחרי שאנחנו מפעילים את האדמר נקבל כי האוניטרי פועל רק על קיוביט אחד, ולכן מספיק להפעיל רק שער קיוביט אחד. כעת ניתן לאחד את כל השערי קיוביט בכל שורה ולקבל כי יהיה רק אוניטרי אחד בכל שורה, ולכן זה למעשה מצמצם את הסיבוכיות ל-\(O(n)\).

\end{proposition}
\begin{definition}[Quantum Phase Estimation]
מעגל אשר מקבל ווקטור עצמי של אופרטור אוניטרי ומחזיר את הערך עצמי המתאים. כלומר נקבל \(\ket{\phi}\) שהוא ערך עצמי של \(U\) כך שמתקיים:
$$U\left|\phi\right\rangle=e^{2\pi i\phi}\left|\phi\right\rangle$$
כאשר אנחנו מחפשים את \(\phi\).

\end{definition}
\begin{proposition}
ניתן לממש את אלגוריתם ה-Quantum phase estimation בעזרת המעגל הבא:

\includegraphics[width=0.8\textwidth]{diagrams/svg_11.svg}
\end{proposition}
\begin{proof}
  \begin{enumerate}
    \item מתחילים במצב \(|0\rangle^{\otimes n}|\phi\rangle\). מפעילים אדמר ומקבלים: 
$$\left( H^{\otimes n}\otimes I_{m} \right)|0\rangle^{\otimes n}|\phi\rangle=\frac{1}{2^{\frac{n}{2}}}\left( |0\rangle+|1\rangle \right)^{\otimes n}|\phi\rangle=\frac{1}{2^{n/2}}\sum_{j=0}^{2^{n}-1}|j\rangle|\phi\rangle.$$


    \item אחרי הפעלת האונטריות הנשלטים נקבל עבור הקיוביט ה-\(k\): 
$$\Lambda(U^{2^{k-1}})\mapsto {\frac{1}{\sqrt{2}}}\left(\left|0\right\rangle+e^{2\pi i\cdot2^{k-1}\phi}\left|1\right\rangle\right)$$
כך שנקבל שהקיוביט עזר תהיה:
$${\frac{1}{2^{n/2}}}\left(\left|0\right\rangle+e^{2\pi i\phi}\left|1\right\rangle\right)\otimes\cdots\otimes\left(\left|0\right\rangle+e^{2\pi i\cdot2^{n-1}\phi}\left|1\right\rangle\right)={\frac{1}{2^{n/2}}}\sum_{x=0}^{2^{n}-1}e^{2\pi i x\phi}\left|x\right\rangle$$


    \item נזרוק כעת את \(\ket{\Psi}\). התמרת פורייה הפוכה תתן: 
$${\frac{1}{2^{n}}}\sum_{x=0}^{2^{n}-1}e^{2\pi i x\phi}\sum_{y=0}^{2^{n}-1}e^{-{\frac{2\pi}{2^{n}}}i x y}\left|y\right\rangle=\sum_{y=0}^{2^{n}-1}{\frac{1}{2^{n}}}\sum_{x=0}^{2^{n}-1}e^{{\frac{2\pi}{2^{n}}}i x(2^{n}\phi-y)}\left|y\right\rangle$$


    \item אם \(2^{n}\phi\) הוא מספר שלם, נקבל: 
$$\frac{1}{2^{n}}\sum_{x=0}^{2^{n}-1}e^{\frac{2\pi}{2^{n}}i x(2^{n}\phi-y)}\,=\,\delta_{2^{n}\phi,y}$$
ולכן נקבל \(2^{n}\phi\) בנסתברות 1. אחרת נקבל כי:
$${\frac{1}{2^{n}}}\sum_{x=0}^{2^{n}-1}e^{{\frac{2\pi}{2^{n}}}i x(2^{n}\phi-y)}$$
עדיין נותן קיצון עבור מספר שלם \(y\) אשר קרוב ל-\(2^{n}\phi\). ולכן זה התוצאה הסבירה ביותר(עם הסתברות של לפחות \(\frac{4}{\pi^{2}}\)) כאשר אם נחלק ב-\(2^{n}\) נקבל את \(\phi\) עד כדי דיוק של \(2^{-n}\). לכן ככל ש-\(n\) גדול יותר נקבל בקירוב טוב יותר את התוצאה האמיתית.


  \end{enumerate}
\end{proof}
\begin{summary}
  \begin{itemize}
    \item התמרת פורייה קוונטית מבוצעת באופן הבא:
\begin{gather*}\ket{j_1 j_2 \dots j_n} \xrightarrow{H \otimes I^{\otimes (n-1)}} \frac{1}{\sqrt{2}} \left( \ket{0} + e^{2\pi i 0.j_1} \ket{1} \right) \ket{j_2 \dots j_n}  \\\xrightarrow{U_1} \frac{1}{\sqrt{2}} \left( \ket{0} + e^{2\pi i 0.j_1 j_2} \ket{1} \right) \ket{j_2 \dots j_n}  \\\xrightarrow{U_2 \dots U_{n-1}} \frac{1}{\sqrt{2}} \left( \ket{0} + e^{2\pi i 0.j_1 j_2 \dots j_n} \ket{1} \right) \ket{j_2 \dots j_n}\xrightarrow{H \otimes I^{\otimes (n-2)}}  \\\frac{1}{2} \left( \ket{0} + e^{2\pi i 0.j_1 j_2 \dots j_n} \ket{1} \right) \left( \ket{0} + e^{2\pi i 0.j_2} \ket{1} \right) \ket{j_3 \dots j_n}  \\\xrightarrow{U_1 \dots U_{n-2}} \frac{1}{2} \left( \ket{0} + e^{2\pi i 0.j_1 j_2 \dots j_n} \ket{1} \right) \left( \ket{0} + e^{2\pi i 0.j_2 \dots j_n} \ket{1} \right) \ket{j_3 \dots j_n} \\\xrightarrow{\dots}  \frac{1}{\sqrt{2^n}} \left( \ket{0} + e^{2\pi i 0.j_1 j_2 \dots j_n} \ket{1} \right) \left( \ket{0} + e^{2\pi i 0.j_2 \dots j_n} \ket{1} \right) \dots \left( \ket{0} + e^{2\pi i 0.j_n} \ket{1} \right)
\end{gather*}
    \item שיערוך פאזה מאפשר בהנתן ווקטור עצמי למצוא את הערך עצמי המתאים בצורה הבאה:
\begin{gather*}\ket{0}^{\otimes n} \ket{\phi} \xrightarrow{H^{\otimes n} \otimes I_m} \frac{1}{2^{n/2}} \sum_{j=0}^{2^n-1} \ket{j} \ket{\phi} \xrightarrow{\Lambda(U^{2^{k-1}})} \frac{1}{2^{n/2}} \sum_{x=0}^{2^n-1} e^{2\pi i x\phi} \ket{x} \xrightarrow{\text{QFT}^{-1}} \sum_{y=0}^{2^n-1} \frac{1}{2^n} \sum_{x=0}^{2^n-1} e^{\frac{2\pi}{2^n} i x (2^n \phi - y)} \ket{y}
\end{gather*}
כאשר אם \(2^{n}\phi \in \mathbb{Z}\) נקבל \(\frac{1}{2^{n}}\sum_{x=0}^{2^{n}-1}e^{\frac{2\pi}{2^{n}}i x(2^{n}\phi-y)}\,=\,\delta_{2^{n}\phi,y}\) ולכן \(2^{n}\phi\) בנסתברות 1. אחרת הפייקים יהיו במספר השלם הכי קרוב ל-\(2^{n}\phi\) בהסתברות טובה(של לפחות \(\frac{4}{\pi^{2}}\)) וניתן כעת לחלק ב-\(2^{n}\) כדי לקבל את \(\phi\) בדיוק של \(2^{-n}\).
  \end{itemize}
\end{summary}
\section{חיפוש}

\begin{definition}[בעיית החיפוש]
מציאת ערך מתוך רשימה ללא כל אינפורמציה.

\end{definition}
\begin{symbolize}
נסמן את הפונקציה \(f_{w}:\{ 0,1 \}^{n}\to\{ 0,1 \}\) אשר מחזירה 1 עבור הערך הנכון ו-0 עבור הערך הלא נכון:
$$f_{w}\left(x\right)=\begin{cases}1&x=w\\ 0&x\neq w\end{cases}$$
כאשר האופרטור האוניטרי אשר מבצע את \(U\) מוגדר על ידי:
$$U_{w}\left|x\right\rangle\otimes\left|y\right\rangle=\left|x\right\rangle\otimes\left|y\oplus f_{w}\left(x\right)\right\rangle$$

\end{symbolize}
\begin{proposition}
$$(-1)^{f_{w}(x)}=\begin{cases}-1&x=w\\ 1&x\neq w\end{cases}$$

\end{proposition}
\begin{lemma}[ביצוע אופרטור "גרובר"]
עבור מצב:
$$|s\rangle={\frac{1}{\sqrt{N}}}\sum_{x=0}^{N-1}|x\rangle$$
ניתן לממש את האופרטור האוניטרי:
$$U_{s}=2\,|s\rangle\!\langle s|-\mathbb{1} $$
בסיבוכיות \(O(N)\).

\end{lemma}
\begin{proof}
\end{proof}
נתון \(U_{s}=2\left|s\right\rangle\left\langle s\right|-\mathbb{1}\). אנו יודעים כי \(H^{\otimes n}\ket{0}^{\otimes n}=\ket{s}\). לכן אם נעבור לבסיס \(x\) נקבל:
$$U'_s= H^{\otimes n}U_{s}H^{\otimes n}=2\ket{000\dots 0}\bra{000\dots 0} -\mathbb{1}  $$
נגדיר אוניטרי נוסף:
$$U_{s}^{\prime\prime}=X^{\otimes n}U_{s}^{\prime}X^{\otimes n}$$
ונקבל \(U_{s}^{\prime\prime}\left|111...1\right\rangle=\left|111...1\right\rangle\) כאשר עבור כל מחרוזת אחרת \(x\) בגודל \(n\) נקבל:
$$U_{s}^{\prime\prime}\left|x\right\rangle=-\left|x\right\rangle$$
נשים לב כי האופרטור הנשלט \(\Lambda^{n-1}(Z)\) פועל על הקיוביט האחרון רק אם כל ה-\(n-1\) קיוביטים הראשונים הם במצב \(\ket{1}\). ולכן אם כל ה-\(n\) קיוביטים הם במצב \(\ket{111 \dots 1}\) נקבל \(-\ket{111\dots 1}\). אחרת נקבל את אותו המצב. לכן נסיק:
$$U''_{s}=-\Lambda^{n-1}(Z)$$
ולכן למעשה שווים עד כדי פאזה גלובאלית(אשר לא משפיעה על הפיזיקה). לכן ניתן לכתוב:
$$U_{s}=-H^{\otimes n}\left(\prod_{i=1}^{n}X_{i}\right)X_{n}H_{n}\Lambda^{n-1}\left(X\right)H_{n}X_{n}\left(\prod_{i=1}^{n}X_{i}\right)H^{\otimes n}$$
כאשר ניתן להשתמש בזה שמתקיים \(H^{\otimes n}=\prod_{i=1}^{n}H\) וכן:
$$\prod_{i=1}^{n-1}\left(H_{i}X_{i}\right)H_{n}X_{n}X_{n}H_{n}\,=\,\prod_{i=1}^{n-1}\left(H_{i}X_{i}\right),$$
ולקבל:
$$U_{s}=-\prod_{i=1}^{n-1}\left(H_{i}X_{i}\right)\Lambda^{n-1}\left(X\right)\prod_{i=1}^{n-1}\left(H_{i}X_{i}\right)$$
כאשר כל השערים single qubit הם יהיו סה"כ מ-\(O(1)\) ולא תלויים בוגל הפלט כאשר באמצע יש לנו את \(\Lambda^{n-1}(X)\) שהוא מ-\(O(n)\) ולכן \(U_{s}\) מגודל \(O(n)\). 

\begin{remark}
בפועל מה שאופרטור הזה עושה זה שזה משקף את הערך של האופליטודה ביחס לאמפליטדה הממצוע.

\end{remark}
\begin{proposition}
ניתן לממש את המעגל בצורה הבאה:

\includegraphics[width=0.8\textwidth]{diagrams/svg_12.svg}
\end{proposition}
\begin{proof}
  \begin{enumerate}
    \item מתחילים עם \(\ket{0^{n}}\). מפעלים האדמר כך שנקבל: 
$$|s\rangle={\frac{1}{\sqrt{N}}}\sum_{x=0}^{N-1}|x\rangle$$


    \item נפעיל את הקופסא השחורה \(U_{f}\). זה יקטין את האמפליטודה של \(\ket{w}\) ולא ישנה את כל השאר. זה יגרום לך שהאמליטודה הממוצעת תקטן. 


    \item נפעיל את אופרטור גרובר \(U_{s}=2\,|s\rangle\!\langle s|-\mathbb{1}\). זה יבצע שיקוף סביב האמפליטדה הממצועת.  


    \item אחרי בערך \(\sqrt{ N }\) חזרות של 2 ו-3 ניתן את האפליטודות ולקבל בסבירות טובה את \(\ket{z}\). 


  \end{enumerate}
\end{proof}
\begin{summary}
$$\begin{gather*}\ket{0}^{\otimes n} \xrightarrow{H^{\otimes n}} \quad \ket{s} = \frac{1}{\sqrt{N}}\sum_{x=0}^{N-1}\ket{x} \xrightarrow{U_f = I-2\ket{z}\bra{z}} \quad \frac{1}{\sqrt{N}}\Bigl(\sum_{x\neq z}\ket{x} - \ket{z}\Bigr) \\\xrightarrow{U_s = 2\ket{s}\bra{s}-I} \quad \frac{1}{\sqrt{N}}\Bigl(\frac{N-4}{N}\sum_{x\neq z}\ket{x} + \frac{3N-4}{N}\ket{z}\Bigr) \quad \text{(amplitude amplification)} \\\xrightarrow{\bigl(U_sU_f\bigr)^{\lceil \frac{\pi}{4}\sqrt{N}\rceil}} \quad \ket{z}.
\end{gather*}$$

\end{summary}
\Chapter{סימולציה קוונטית והצפנה}

\section{טרוטריזציה}

כאשר \(H\) מטריצה גדולה יכול להיות מאוד מורכב לעשות עליה אקספוננט. 

\begin{definition}[המילטוניאן \(k\) לוקאלי]
כאשר ההמילטוניאן \(H\) עצמו פועל על \(m\) קיוביטים אך ניתן לפרק אותו ל-\(L\) אופרטורים:
$$H=\sum_{l=1}^{L}H_{l}$$
כך שכל \(H_{l}\) פועל לכל היותר על \(k\) קיוביטים.

\end{definition}
\begin{example}
נתונה שרשרת פרמיונית מאורך \(N\) - ישר עם \(N\) אתרים(מסומנים 1 עד \(N\)) אשר יכולים להכיל פרמיון יחיד. פרמיון נוצר על ידי \(c_{n}^{\dagger}\) ומושמד על ידי \(c_{n}\) כאשר מתקיים:
$$\left\{  c_{m},c_{n}^{\dagger}  \right\}=\delta_{mn}\qquad \left\{c_{m},c_{n}\right\}=\left\{c_{m}^{\dagger},c_{n}^{\dagger}\right\}=0$$
כן כדי למפות את הפרמיונים לקיוביטים לצורכי חישוב נשתמש במפוי ג'ורדן ויגנר:
$$c_{n}=\left(\prod_{m=1}^{n-1}\sigma_{m}^{z}\right)\sigma_{m}^{-}\qquad c_{n}^{\dagger}=\left(\prod_{m=1}^{n-1}\sigma_{m}^{z}\right)\sigma_{m}^{+}$$
כאשר העתקה זו משמרת יחסי אנטי קומוטטור. אם נרצה למשל לבצע סימולציה של ההמילטוניאן:
$$H=\left(c_{1}^{\dagger}c_{2}+c_{2}^{\dagger}c_{3}+\ldots+c_{N-1}^{\dagger}c_{N}\right)+h.c.$$
אשר בפרוש 2 לוקאלי במונחי הפרמיונים, אבל הוא גם 2 לוקאלי במונחי הספין כי מתקיים:
\begin{gather*}c_{n}^{\dagger}c_{n+1}=\sigma^{(+)}_{n}\left( \prod_{m<n}\sigma_{z} \right)\left( \prod_{m<n+1}\!\!\sigma_{z} \right)\sigma_{n+1}^{(-)}=\sigma_{n}^{(+)}\sigma_{z}^{(n)}\sigma_{n+1}^{(-)} = \\=\left( \ket{\uparrow } \bra{\downarrow } \sigma_{z} \right)_{n}\otimes \sigma_{n+1}^{(-)}=-\sigma_{n}^{(+)}\sigma_{n+1}^{(0)}\implies \\\implies H=-\left( \sigma_{1}^{(+)}\sigma_{2}^{(-)}+\sigma_{2}^{(+)}\sigma_{3}^{(-)} +\dots \right)+ \mathrm{h.\!c.}
\end{gather*}
לעומת זאת ההמילטוניאן:
$$H=\left(c_{1}^{\dagger}c_{7}+c_{2}^{\dagger}c_{8}+\ldots+c_{N-6}^{\dagger}c_{N}\right)+h.c.$$
הוא 2 לוקאלי במונחי הפרמיונים אך 7 לוקאלי במונחי הספין! הסיבוכיות עדיין נשארת אבל פולינומיאלית(למורת שייתכן יהיה קשה יותר לממש את זה עם חומרה קוונטית).

\end{example}
כעת נניח כי ההמילטוניאן שלו \(k\) לוקאלי, וכן בנוסף נניח כי קיים \(h\) כך שכל ההמילטוניאנים שלנו מקיימים \(\lVert H_{i} \rVert\leq h\).

\begin{proposition}[נוסחאת סוזוקי-טרוטר]
$$e^{\hat{A}+\hat{B}}=\operatorname*{lim}_{n\rightarrow\infty}\left(e^{\frac{\hat{A}}{n}}e^{\frac{\hat{B}}{n}}\right)^{n}$$

\end{proposition}
\begin{corollary}
$$e^{-i\sum_{l=1}^{L}\hat{H}_{l}t}=\operatorname*{lim}_{n\rightarrow\infty}\left(\prod_{l=1}^{L}e^{-i\frac{\hat{H}_{l}t}{n}}\right)^{n}$$

\end{corollary}
\begin{remark}
למעשה זהו קירוב ראשון של נוסחאת בייקר-קאמפבל-האסדורף. אנחנו מתייחסים לאופרטורים כאילו הם מתחלפים.

\end{remark}
\begin{symbolize}
נסמן את הפלט האמיתי ב-\(\widetilde{U}(t)\) כך שהמצב שמתקבל יהיה:
$$\left|\widetilde{\psi}\left(t\right)\right\rangle=\widetilde{U}\left(t\right)\left|\psi_{0}\right\rangle$$

\end{symbolize}
\begin{proposition}
אם נדרוש כי השגיאה תהיה:
$$\left\|\left|\psi\left(t\right)\right\rangle-\left|\widetilde{\psi}\left(t\right)\right\rangle\right\|\leq\delta$$
אז מספיק לדרוש כי:
$$\left\|U\left(t\right)-\widetilde{U}\left(t\right)\right\|\leq\delta$$
כאשר אם נבצע \(N=\frac{t}{\epsilon}\) צעדים השגיאה תקיים:
$$\left\|e^{-i\sum_{i=1}^{M}H_{i}\epsilon}-\prod_{i=1}^{M}e^{-i H_{i}\epsilon}\right\|\leq\epsilon^{2}\!\sum_{i<j}\!h^{2}+O\left(\epsilon^{3}\right)=O\left(M\epsilon^{2}h^{2}\right)$$

\end{proposition}
\begin{proof}
נשים לב כי:
$$e^{-i\sum\limits_{i=1}^{M}H_{i}\epsilon}=1-i\epsilon\sum\limits_{i=1}^{M}H_{i}-\frac{1}{2}\epsilon^{2}\sum\limits_{i=1}^{M}\sum\limits_{j=1}^{M}H_{i}H_{j}+O\left(\epsilon^{3}\right)$$
וכן:
$$\prod_{i=1}^{M}e^{-iH_{i}\epsilon}=\prod_{i=1}^{M}\left(1-i\epsilon H_{i}+\frac{1}{2}\epsilon^{2}H_{i}^{2}+O\left(\epsilon^{3}\right)\right)=1-i\epsilon\sum_{i=1}^{M}H_{i}-\epsilon^{2}\sum_{i<j}H_{i}H_{j}-\frac{1}{2}\epsilon^{2}\sum_{i=1}^{M}H_{i}^{2}+O\left(\epsilon^{3}\right)$$
ולכן ההפרש יהיה:
$$e^{-i\sum\limits_{i=1}^{M}H_{i}\epsilon}-\prod\limits_{i=1}^{M}e^{-iH_{i}\epsilon}=\frac{1}{2}\epsilon^{2}\left(\sum\limits_{i<j}H_{i}H_{j}-\sum\limits_{i>j}H_{i}H_{j}\right)+O\left(\epsilon^{3}\right)=\frac{1}{2}\epsilon^{2}\sum\limits_{i<j}\left[H_{i},H_{j}\right]+O\left(\epsilon^{3}\right)$$
באופן כללי נדרש עבור \(\sum\limits_{i<j}\left[H_{i},H_{j}\right]\) לסכום על \(\frac{1}{2}M(M-1)=O(M^{2})\) איברים, כאשר כיוון שאנחנו דורשים לקוליות, הקומוטטור לא יהיה אפס פרט למספר קבוע של גורמי המילטוניאן. ולכן \(O(M)=O(n)\). עבור כל קומוטטור שאינו אפסי נקבל:
$$\|[H_{i},H_{j}]\|\leq2\,\|H_{i}H_{j}\|\leq2h^{2}$$
ולכן השגיאה תקיים:
$$\left\|e^{-i\sum_{i=1}^{M}H_{i}\epsilon}-\prod_{i=1}^{M}e^{-i H_{i}\epsilon}\right\|\leq\epsilon^{2}\sum_{i<j}\!h^{2}+O\left(\epsilon^{3}\right)=O\left(M\epsilon^{2}h^{2}\right)$$

\end{proof}
\begin{corollary}
אם נבצע \(N=\frac{t}{\epsilon}\) צעדים כאשר כל אחת עם שגיאה מקסימלית מסדר גודל \(O\left( M\epsilon^{2}h \right)\) נקבל:
$$\delta\sim\frac{t}{\epsilon}M\epsilon^{2}h^{2}=M h^{2}t\epsilon\implies\epsilon=O\left(\frac{\delta}{M h^{2}t}\right)$$

\end{corollary}
\begin{corollary}
גודל המעגל(מספר השערים) הנדרשים יהיו:
$$L=\frac{M t}{\epsilon}=O\left(\frac{\left(M h t\right)^{2}}{\delta}\right)$$

\end{corollary}
\begin{corollary}
ממשפט Solovay-Kitaev ניתן לבצע סימולציה על כל שער עד כדי דיוק \(O\left( \epsilon^{2}h^{2} \right)\) עם קבוצת שערכים אונברלסים על ידי שימוש ב:
$$\mathrm{polylog}\left(\frac{1}{\epsilon^{2}h^{2}}\right)=\mathrm{polylog}\left(\left(\frac{M h t}{\delta}\right)^{2}\right)$$
שערים ולכן ניתן לבצע סימולציה של הקידום בזמן עם מעגל מגודל:
$$L=O\left(\frac{\left(M h t\right)^{2}}{\delta}\mathrm{polylog}\left(\left(\frac{M h t}{\delta}\right)^{2}\right)\right)$$
כיוון ש-\(M=O(n)\) והנפח גם כן \(O(n)\) נקבל כי \(Mt\) מסדר גודל של המרחב זמן של הסימולציה \(\Omega=Vt\) ולכן:
$$L=O\left(\Omega^{2}\mathrm{polylog}{\left( \Omega \right)}\right)$$

\end{corollary}
\begin{proposition}[נוסחאת טרוטר מסדר שני]
$$U\left(t\right)\approx\widetilde{U}^{\left(2\right)}\left(t\right)=\left(\prod_{i=1}^{M}e^{-i H_{i}\epsilon/2}\prod_{i=M}^{1}e^{-i H_{i}\epsilon/2}\right)^{N}$$

\end{proposition}
\begin{lemma}
עבור שתי איברים\((M=2)\) נקבל:
$$\left\|e^{-i H\epsilon}-e^{-i H_{1}\epsilon/2}e^{-i H_{2}\epsilon}e^{-i H_{1}\epsilon/2}\right\|=O\left(\epsilon^{3}\right)$$

\end{lemma}
\begin{proof}
ראשית נפתח את שתי הגורמים לסדר שלישי ב-\(\epsilon\):
\begin{gather*}e^{-i\epsilon(H_{1}+H_{2})}=1-i\epsilon\left(H_{1}+H_{2}\right)-\frac{\epsilon^{2}}{2}\left(H_{1}^{2}+H_{2}^{2}+H_{1}H_{2}+H_{2}H_{1}\right)+\\ \frac{ie^{3}}{6}\left(H_{1}^{3}+H_{2}^{3}+H_{1}^{2}H_{2}+H_{1}H_{2}^{2}+H_{2}^{2}H_{1}+H_{2}H_{1}^{2}+H_{1}H_{2}H_{1}+H_{2}H_{1}H_{2}\right)+O\left(\epsilon^{4}\right) 
\end{gather*}
וכן:
\begin{gather*}e^{-i\epsilon H_{1}/2}e^{-i\epsilon H_{2}}e^{-i\epsilon H_{1}/2}=\left(1-i\frac{\epsilon}{2}H_{1}-\frac{\epsilon^{2}}{8}H_{1}^{2}+i\frac{\epsilon^{3}}{48}H_{1}^{3}+O\left(\epsilon^{4}\right)\right)\times\\ \times\left(1-i\epsilon H_{2}-\frac{\epsilon^{2}}{2}H_{2}^{2}+i\frac{\epsilon^{3}}{6}H_{2}^{3}+O\left(\epsilon^{4}\right)\right)\times\left(1-i\frac{\epsilon}{2}H_{1}-\frac{\epsilon^{2}}{8}H_{1}^{2}+i\frac{\epsilon^{3}}{48}H_{1}^{3}+O\left(\epsilon^{4}\right)\right)=\\=1-i\epsilon\left(H_{1}+H_{2}\right)-\frac{\epsilon^{2}}{2}\left(H_{1}^{2}+H_{2}^{2}+H_{1}H_{2}+H_{2}H_{1}\right)\\+i\epsilon^{3}\left(\frac{1}{24}H_{1}^{3}+\frac{1}{6}H_{2}^{3}+\frac{1}{8}H_{1}^{2}H_{2}+\frac{1}{4}H_{1}^{*}H_{2}^{2}+\frac{1}{4}H_{2}^{2}H_{1}+\frac{1}{8}H_{2}H_{1}^{2}+\frac{1}{4}H_{1}H_{2}H_{1}\right)+O\left(\epsilon^{4}\right) 
\end{gather*}
כאשר נשים לב כי הגורמים מסדר ראשון ושני הם זהים, ולכן יבטלו אחד את השני כשנחשב שגיאה, ונשאר רק עם גורמים מסדר שלישי.

\end{proof}
\begin{proposition}
עבור מספר כללי \(M\geq 2\) של גורמי המילטוניאן נקבל כי השגיאה תקיים:
$$\left\lVert  U-\widetilde{U}^{(2)}  \right\rVert =\left\|e^{-iH\epsilon}-e^{-iH_{1}\epsilon/2}\cdots e^{-iH_{M-1}\epsilon/2}e^{-iH_{M}\epsilon}e^{-iH_{M-1}\epsilon/2}\cdots e^{-iH_{1}\epsilon/2}\right\|=\left(M-1\right)O\left(\epsilon^{3}\right)=O\left(M\epsilon^{3}\right)$$

\end{proposition}
\begin{proof}
ניתן להוסיף ולהחסיר מהשגיאה:
$$e^{-i H_{1}\epsilon/2}e^{-i\epsilon\sum_{i=2}^{M}H_{i}}e^{-i H_{1}\epsilon/2}$$
ולהשתמש באי שוויון המשולש ולקבל:
\begin{gather*}\left\|e^{-iH_{\epsilon}}-e^{-iH_{1\epsilon}/2}\ldots e^{-iH_{M-1\epsilon}/2}e^{-iH_{M\epsilon}\epsilon}e^{-iH_{M-1\epsilon}/2}\ldots e^{-iH_{1\epsilon}/2}\right\|\leq\\\left\|e^{-iH_{\epsilon}}-e^{-iH_{1\epsilon}/2}e^{-i\epsilon\sum\limits_{i=2}^{M}H_{i}}e^{-iH_{1\epsilon}/2}\right\|+\\+\left\|e^{-iH_{1\epsilon}/2}e^{-i\epsilon\sum\limits_{i=2}^{M}H_{i}}e^{-iH_{1\epsilon}/2}-e^{-iH_{1\epsilon}/2}\ldots e^{-iH_{M-1\epsilon}/2}e^{-iH_{M\epsilon}\epsilon}e^{-iH_{M-1\epsilon}/2}\ldots e^{-iH_{1\epsilon}/2}\right\| 
\end{gather*}
כאשר האיבר הראשון יהיה מסדר גודל \(O\left( \epsilon^{3} \right)\) מהלמה וניתן להפעיל את הלמה באופן אידנוקטיבי על האיבר השני, ולכן נקבל סה"כ \(M-1\) איברים ולכן השגיאה תגדל בסדר גודל של \((M-1)O\left( \epsilon^{3} \right)=O\left( M\epsilon^{3} \right)\).

\end{proof}
\begin{example}[מודל אייזינג]
נתון ההמילטוניאן של מודל אייזינג אשר נרצה:
$$H=a{\sum_{i}}Z_{i}+c{\sum_{i}}X_{i}X_{i+1}\equiv H_{Z}+H_{X}$$
נרצה לבנות מעגל קוונטי אשר מקרב את \(e^{ -iHt }\) עבור ההמילטוניאן הזה בעזרת שערים של אחד או שתיים קיוביטים. ראשית נשים לב כי כל הגורמים של \(H_{Z}\) מתחלפים אחד עם השני כי \(Z_{i}\) מתחלף עם עצמו וכל הגורמים של \(H_{X}\) מתחלפים אחד עם השני כי \(X_{i},X_{j}\) מתחלפים זה עם זה. ולכן מתקיים:
$$e^{ -i\epsilon H_{Z} }=\prod_{i}e^{ -i\epsilon aZ_{i}}\qquad e^{ -i\epsilon H_{X} }=\prod_{i}e^{ -i\epsilon c\sum_{i}X_{i}X_{i+1} }$$
כעת נרצה למצוא את השגיאה מסדר מוביל עבור קירוב טרוטר מסדר ראשון, כלומר $$\widetilde{U}\left(t\right)=\left(\prod_{i=1}^{M}e^{-i H_{i}\epsilon}\right)^{N}$$ כאשר \(\epsilon=\frac{t}{N}\) וכן נדרוש כי \(\left\lVert  U(t)-\tilde{U}(t)  \right\rVert\leq \delta\) עבור \(\delta> 0\) כלשהו. לפי נוסחאת השגיאה נקבל כי הגורם המוביל של השגיאה הוא:
$$\frac{\epsilon^{2}}{2}\left\|\left[H_{Z},H_{X}\right]\right\|$$
כדי לחשב את הקומוטטור ניתן לכתוב:
$$[H_{Z},H_{X}]=a c{\sum_{i j}{\left[Z_{i},X_{j}X_{j+1}\right]}}=a c{\sum_{i j}{\left(\delta_{i j}\left[Z_{i},X_{j}X_{j+1}\right]+\delta_{i,j+1}\left[Z_{i},X_{j}X_{j+1}\right]\right)}}$$
ולכן:
$$\left[H_{Z},H_{X}\right]=a c{\sum_{i}\left[Z_{i},X_{i}\right]\left(X_{i+1}+X_{i-1}\right)}=2i a c{\sum_{i}}Y_{i}\left(X_{i+1}+X_{i-1}\right).$$
כאשר החסם יהיה:
\begin{gather*}\frac{\epsilon^{2}}{2}\left\|\left[H_{Z},H_{X}\right]\right\|=2\epsilon^{2}\left|a c\right|\left\|\sum_{i}Y_{i}\left(X_{i+1}+X_{i-1}\right)\right\|\leq\\ 2\epsilon^{2}\left|a c\right|\sum_{i}\left\|Y_{i}\left(X_{i+1}+X_{i-1}\right)\right\|{=}2\epsilon^{2}\left|a c\right|n\left\|Y_{1}\left(X_{2}{+}X_{n}\right)\right\| 
\end{gather*}
נשים לב כי:
$$\left\|Y_{1}\left(X_{2}+X_{n}\right)\right\|\leq\left\|Y_{1}X_{2}\right\|+\left\|Y_{1}X_{n}\right\|=2$$
ולכן:
$$\left\|e^{-i H\epsilon}-\prod_{i=1}^{M}e^{-i H_{i}\epsilon}\right\|\leq4\epsilon^{2}\left|a c\right|n\implies \left\|U\left(t\right)-\widetilde{U}\left(t\right)\right\|\leq\delta\approx4N\epsilon^{2}\left|a c\right|n$$
ואם נרצה לחשב עד דיוק \(\delta\) נדרש:
$$\delta\approx4\epsilon^{2}\left|a c\right|n N=4\epsilon\left|a c\right|n t\implies\epsilon\approx\frac{\delta}{4\left|a c\right|n t}$$

\end{example}
\begin{example}[מערכת מעגלית]
נתון מערכת מחזורית של ספינים עם \(n\) אתרים וההמילטוניאן:
$$H=\sum_{i=1}^{n}X_{i}X_{i+1}+\sum_{i=1}^{n}Y_{i}Y_{i+1}+\lambda\sum_{i=1}^{n}Z_{i}\equiv H_{X}+H_{Y}+H_{Z}$$
נרצה לבצע קידום בזמן \(U=e^{ -iHt }\) בעזרת השערים ה-single qubit מהצורה \(V=e^{ -i\phi Z_{i} }\) ובעזרת שערי שתי קיוביטים של ספינים שכנים מהצורה \(W_{i}\left( \phi \right)=e^{ -i\phi X_{i}X_{i+1} }\). ניתן להפעיל שערים במקביל כל עוד אינם פועלים על אותו קיוביט. ראשית נשים לב שנדרש לעשות פה טרוטטריזציה למערכת כיוון שההמילטוניאן דורש לפעול על יותר מקיוביט אחד באותו זמן והשערים שלנו לא מאפשרים לעשות זאת. נקרב:
$$U(t)=e^{ -iHt }\approx \left( e^{ iH_{X}\varepsilon }\cdot e^{ -iH_{Y} \varepsilon}\cdot e^{ iH_{Z}\varepsilon } \right)^{M}\qquad M=\frac{t}{\varepsilon}$$
כאשר אם נניח (בה"כ) כי \(X<Y<Z\) השגיאה נתונה על ידי:
$$\frac{\varepsilon^{2}}{2}\left\lVert  \sum_{i<j}[H_{i},H_{j}]  \right\rVert +O\left( \varepsilon^{3} \right)=\frac{\varepsilon^{2}}{2}\lVert [H_{X},H_{Y}]+[H_{Y},H_{Z}]+[H_{X},H_{Z}] \rVert +O\left( \varepsilon^{3} \right)$$
כאשר לאחר פתרון אלגברי מפורש של הקומוטטורים נקבל כי השגיאה תהיה \(4n\varepsilon^{2}+O\left( \varepsilon^{3} \right)\).
כדי לממש את \(H_{Z}\) נשים לב כי כל הגורמים הלוקאלים של \(H_{Z}\) מתחלפים ולכן:
$$e^{ -i\varepsilon H_{Z} }=e^{ -i\varepsilon \lambda \sum Z_{i} }=\prod_{i}e^{ -i\varepsilon \lambda Z_{i} }=\prod_{i}V_{i}$$
כאשר ניתן להפעיל אותם במקביל ולקבל כי גדל ב-\(O(1)\). באופן דומה כדי לממש את \(H_{X}\) נשים לב כי כל הגורמים הלוקאלים מתחלפים ולכן:
$$e^{ -i\varepsilon H_{X} }=e^{ -i\varepsilon \sum_{i}X_{i}X_{i+1}}=\prod_{i}e^{ -i\varepsilon X_{i}X_{i+1} }=\prod_{i} W_{i}\left( \varepsilon \right)$$
אומנם נראה שיש בעיה כיוון שאנחנו פועלים בזוגות ולכן נראה כאילו לא ניתן להריץ את כולם ביחד, ניתן לפרק את המכפלה לזוגיים ואי זוגיים וכעת ניתן להריץ על הזוגיים בהרצה אחת, ולהריץ על האי זוגיים בשתי הרצות, ולכן גם זה ממומש ב-\(O(1)\). עבור \(H_{Y}\) נשתמש בזה שמתקיים:
$$e^{ -i\frac{\pi}{4} Z_{i} }X_{i}e^{ i\frac{\pi}{4} Z_{i} }=Y_{i}\implies V_{i}\left( \frac{\pi}{4} \right)X_{i}V_{i}^{\dagger}\left( \frac{\pi}{4} \right)$$
ולכן:
$$H_{Y}=\left[ \prod_{i}V_{i}\left( \frac{\pi}{4} \right) \right]H_{X}\left[ \prod_{i}V_{i}^{\dagger}\left( \frac{\pi}{4} \right) \right]\implies e^{ iH_{Y}\varepsilon }=Ve^{ -i\varepsilon H_{X} }V^{\dagger}$$
וגם ממומש ב-\(O(1)\).

\end{example}
\section{תיקון שגיאות - קוד שור}

\begin{proposition}[סוגי שגיאות]
  \begin{enumerate}
    \item היפוך ביט - \(X\)


    \item היפוך פאזה - \(Z\). 


    \item שניהם - \(Y\). 


  \end{enumerate}
\end{proposition}
\begin{lemma}
ניתן לקודד את \(\ket{0}\) ו-\(\ket{1}\) על ידי:
$$\begin{array}{r}{|0\rangle\longrightarrow|\overline{{{0}}}\rangle\equiv|000\rangle}\\ {|1\rangle\longrightarrow|\overline{{{1}}}\rangle\equiv|111\rangle}\end{array}$$

\end{lemma}
\begin{corollary}
$$\alpha\left|0\right\rangle+\beta\left|1\right\rangle\longrightarrow\alpha\left|0\right\rangle+\beta\left|1\right\rangle=\alpha\left|000\right\rangle+\beta\left|111\right\rangle$$

\end{corollary}
\begin{remark}
זה לא מפר את משפט חוסר השכפול כי אנחנו מאתחלים למצב ידוע שלוש פעמים, \textbf{ולא} מבצעים:
$$a\,|0\rangle\,+\,b\,|1\rangle\,\stackrel{.}{\mapsto}\,\left(a\,|0\rangle+b\,|1\rangle\right)^{\otimes3}$$

\end{remark}
\begin{proposition}[תיקון היפוך ביט]
ניתן לבצע את האלגוריתם הבא לתיקון של היפוך קיוביט יחיד.

  \begin{enumerate}
    \item מתחלים עם שלושה קיוביטים. 


    \item מבצעים מדידות של האופרטורים \(Z_{2}Z_{3}\) ו-\(Z_{1}Z_{2}\) כאשר ניתן כי פועלים על בסיס \(z\) ולא משנים את המצב. 


    \item לפי התוצאות ניתן לדעת האם התבצע היפוך ביט ואיזה אופרטור נדרש להפעיל כדי לקבל את התוצאה המקורית, לפי הטבלה הבאה: 


  \end{enumerate}
  \begin{table}[htbp]
    \centering
    \begin{tabular}{|cccc|}
      \hline
      מקרה & \(Z_1 Z_2\) & \(Z_2 Z_3\) & תיקון נדרש \\ \hline
      אין היפוך ביט & \(1\) & \(1\) & \(\mathbb{1}\) \\ \hline
      היפוך של 1 & \(-1\) & \(1\) & \(X\) \\ \hline
      היפוך של 2 & \(-1\) & \(-1\) & \(X_2\) \\ \hline
      היפוך של 3 & \(-1\) & \(1\) & \(X_3\) \\ \hline
    \end{tabular}
  \end{table}
\end{proposition}
\begin{definition}[מדידת syndrome]
מדידה אשר לא רק אמרת לנו איפה הטעות, אבל גם מאפשרת לתקן אותה.

\end{definition}
\begin{proposition}[היפוך ביט וגם היפוך פאזה]
  \begin{enumerate}
    \item נדרש כעת לשמור 9 קיוביטים, אשר מחלוקים 3 שלישיות של קיוביטים. 


    \item עבור כל שלושה, נבצע תיקון היפוך ביט כמו שראינו בטענה הקודמת. 


    \item כל שלושה תהיה אינווריאנטית להפעלה של \(X\) על כל שלושת הקיוביטים בשלושה. כלומר ניתן להפעיל \(X_{1}X_{2}X_{3},\;X_{4}X_{5}X_{6},\;X_{7}X_{8}X_{9}\). 


    \item נבדוק עבור כל שלושה אם יש הפרש פאזה ביחס יחסי על ידי הפעלת אופרטור ה-\(X\) על שניהם. כלומר נפעיל את \(X_{1}X_{2}X_{3}X_{4}X_{5}X_{6}\) ו-\(X_{4}X_{5}X_{6}X_{7}X_{8}X_{9}\).  


    \item אם יש פאזה יחסית בין השלשה הראשונה לשנייה נדרש להפעיל \(Z_{1},Z_{2}\) או \(Z_{3}\).  ניתן לבצע באופן דומה עבור המקרים האחרים. נסכם אותם בטבלה: 


  \end{enumerate}
  \begin{table}[htbp]
    \centering
    \begin{tabular}{|cccc|}
      \hline
      מקרה & \(X_{4}X_{5}X_{6}X_{7}X_{8}X_{9}\) & \(X_{1}X_{2}X_{3}X_{4}X_{5}X_{6}\) & תיקון נדרש \\ \hline
      אין היפוך פאזה & \(1\) & \(1\) & \(\mathbb{1}\) \\ \hline
      היפוך פאזה 1 & \(-1\) & \(1\) & \(Z_{1},Z_{2}\text{ or }Z_{3}\) \\ \hline
      היפוך פאזה 2 & \(-1\) & \(-1\) & \(Z_{4},Z_{5}\text{ or }Z_{6}\) \\ \hline
      היפוך פאזה 3 & \(-1\) & \(1\) & \(Z_{7},Z_{8}\text{ or }Z_{9}\) \\ \hline
    \end{tabular}
  \end{table}
\end{proposition}
\begin{proposition}
ניתן להציג שגיאה על ידי אופרטור אוניטרי מהצורה:
$$U\,=\,e^{i{\bf m}\cdot\sigma}\qquad {\bf m}=\left(\epsilon_{x},\epsilon_{y},\epsilon_{z}\right)$$
כאשר \(\left\lvert  \epsilon_{i}  \right\rvert\ll 1\). ובפרט כל שגיאת קיוביט יחיד ניתן לייצג בצורה הזו.

\end{proposition}
\begin{proof}
ניתן לכתוב:
$$U=e^{i(\epsilon_{x},\epsilon_{y},\epsilon_{z})\cdot\sigma}\approx\mathbb{1}+i\epsilon_{x}X+i\epsilon_{y}Y+i\epsilon_{z}Z$$
כאשר היחידה \(\mathbb{1}\) מסמן שאין שגיאה, \(X\) מייצג היפוך ביט, \(Z\) מיצג היפוך פאזה ו-\(Y\) מייצג שילוב של היפוך פאזה וביט.

\end{proof}
\begin{remark}
נשים לב כי האופציה הכי סבירה זה שלא קורה כלום. אחר כך יש שגיאה מסדר גודל של \(O\left( |\epsilon_{i}|^{2}  \right)\) כאשר אם מתרחש ניתן לפעול לפי קוד שור. ואחר כך יש שגיאות מסדר גודל של \(O\left( |\epsilon_{i}|^{4} \right)\) שיש שתי שגיאות במקביל. עבור מקרה זה אנחנו כבר לא יודעים להתמודד, אך הסיכוי שיקרה מאוד נמוך גם ככה.

\end{remark}
\end{document}