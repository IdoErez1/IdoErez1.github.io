\documentclass{tstextbook}

\usepackage{amsmath}
\usepackage{amssymb}
\usepackage{graphicx}
\usepackage{hyperref}
\usepackage{xcolor}

\begin{document}

\title{Example Document}
\author{HTML2LaTeX Converter}
\maketitle

\Chapter{שדות ומרחבים ווקטורים והעתקות לינאריות}

\section{שדות}

\begin{definition}[שדה]
קבוצה \(\mathbb{F}\) ביחד עם פעולות \(+:\mathbb{F} \to \mathbb{F}\) ו-\(\times:\mathbb{F} \setminus \{ 0 \}\to \mathbb{F} \setminus \{ 0 \}\) תקרא שדה אם מקיים את התנאים הבאים:

  \begin{enumerate}
    \item קומוטטיביות - לכל \(a,b \in \mathbb{F}\) מתקיים \(a+b=b+a\). 


    \item אסוצייטיביות - לכל \(a,b,c \in \mathbb{F}\) מתקיים \((a+b)+c=a+(b+c)\). 


    \item קיום איבר זהות - קיים \(0 \in \mathbb{F}\) כך ש-\(0+a=0\). 


    \item קיום נגדי - לכל \(a \in \mathbb{F}\) קיים \((-a) \in \mathbb{F}\) אשר מקיים \(a+(-a)=0\). 


    \item קומטטיביות - לכל \(a,b \in \mathbb{F}\) מתקיים \(a\cdot b=b\cdot a\). 


    \item אסוצייטיביות - לכל \(a,b,c \in \mathbb{F}\) מתקיים \(\left( a\cdot b \right)\cdot c=a\cdot\left( b\cdot c \right)\). 


    \item קיום איבר זהות - קיים \(0\neq 1\) כך ש-\(a=a\cdot 1\) לכל \(a \in \mathbb{F}\)


    \item קיום איבר \(a^{-1}\) כך ש-\(a\cdot a ^{-1}= 1\) לכל \(a \in \mathbb{F}\). 


    \item חוק הפילוג - לכל \(a,b,c \in \mathbb{F}\) מתקיים \(a\cdot(b+c)=a\cdot b+a\cdot c\). 


  \end{enumerate}
\end{definition}
\begin{remark}
נשים לב שתכונות 1-4 עם הקבוצות \(\mathbb{F}\) ותכונות \(5-8\) עם הקבוצה \(\mathbb{F} \setminus \{ 0 \}\) זה בדיוק אותו הדבר עם פעולה שונה. ולכן הדבר היחיד המבדיל בין כפל וחיבור בשדה זה הפילוג.

\end{remark}
\begin{remark}
קבוצה עם פעולה המקיימת אסוצייטיביות, קיום איבר זהות, קיום נגדי וקומוטטיביות נקראת חבורה אבלית, לכן מספיק לדרוש שהחיבור יהיה חבורה אבלית עם הקבוצה \(\mathbb{F}\), שהכפל יהיה חבורה אבלית עם הקבוצה \(\mathbb{F} \setminus \{ 0 \}\) ושיתקיים חוק פילוג.

\end{remark}
\begin{symbolize}
נסמן את הקבוצה \(\mathbb{F} \setminus \{ 0 \}\) ע"י \(\mathbb{F} ^{\times}\).

\end{symbolize}
השדות הכי משומשים זה כמובן הממשיים, הרציונאליים והמרוכבים, אך קיימים שדות נוספות:
\textbf{הגדרה} שדה הפולינומים
אוסף כל הפולינומים עם מקדמים משדה \(\mathbb{F}\). נסמן את זה ב-\(\mathbb{F} [x]\). נגדיר את הפעולת החיבור להיות חיבור של הפולינומים עם מקדמים המתאימים, ואת הכפל להיות כמו שאנחנו מכירים עם חוק הפילוג. 

\begin{example}
עבור למשל \(p_{1},p_{2} \in \mathbb{Q}[x]\) המוגדרים:
$$p_{1}(x)=2x^2+3x+1\quad p_{2}=x+1$$
נקבל עבור החיבור:
$$p_{1}(x)+p_{2}(x)=2x^{2}+(3+1)x+(1+1)=2x^{2}+4x+2$$
ועבור הכפל:
$$p_{1}(x)\cdot p_{2}(x)=(2x^{2}+3x+1)(x+1)=(2x^{2}+3x+1)x+(2x^{2}+3x+1)=2x^{3}+5x^{2}+4x+1$$

\end{example}
\begin{definition}[דרגה של פולינום]
נגדיר את הפונקציה \(\deg:\mathbb{F} [x]\to \mathbb{N}\) על ידי המעלה הגבוהה ביותר של הפולינום. אם הפולינום הוא קבוע המעלה תהיה 0, כאשר אם הפולינום הוא פולינום האפס, נגדיר את המעלה כמינוס אינסוף.

\end{definition}
\begin{proposition}
עבור פולינומים \(p_{1},p_{2} \in \mathbb{F} [x]\) מתקיים:
$$\deg(p_{1}+p_{2})=\max \left( \deg(p_{1}) ,\deg(p_{2})  \right) $$
וכן:
$$\deg\left( p_{1}(x)\cdot p_{2}(x) \right) =\deg(p_{1}) +\deg(p_{2}) $$

\end{proposition}
\begin{remark}
אחת הסיבות שהגדרנו את הדרגה של פולינום האפס בתור מינוס אינסוף הוא כדי שיהיה קונסיסטנטי עם הטענה הזו.

\end{remark}
\begin{definition}[שדה מודולרי]
עבור \(p\) ראשוני ו-\(n \in \mathbb{N}\) נסתכל על הקבוצה \(\mathbb{F} _{p^{n}}=\left\{  0,1,\dots,p^{n}-1,p^{n}  \right\}\). נגדיר את הפעולת החיבור על ידי זה שניקח את התוצאה של החיבור תחת \(\mathbb{F}\), נסמן אותו ב-\(x\). כעת ניקח את האיבר היחיד \(b\in \mathbb{F} _{q^{n}}\) המקיים עבור \(a \in \mathbb{F}_{q^{n}}\) ש:
$$x=aq^{n}+b$$
כאשר היחידות נובעת מלמת החלוקה של אוקלידס. עבור הכפל נגדיר בצורה דומה, אם התוצאה של הכפל ב-\(\mathbb{F}\) תהיה \(y\) אז התוצאה של הכפל תהיה האיבר היחיד \(b \in \mathbb{F}_{q^{n}}\) המקיים עבור \(a \in \mathbb{F} _{q^{n}}\).
$$y=aq^{n}+b$$

\end{definition}
\begin{proposition}
עבור \(n \neq p^{r}\) עבור \(r\) כלשהו נקבל כי הקבוצה \(\mathbb{F} _n\) לא תיצור שדה.

\end{proposition}
\begin{example}
השדה \(\mathbb{F} _2\) יהיו עם טבלת הכפל:

  \begin{table}[htbp]
    \centering
    \begin{tabular}{|ccccccc|}
      \hline
      + & 0 & 1 &  & * & 0 & 1 \\ \hline
      0 & 0 & 1 &  & 0 & 0 & 0 \\ \hline
      1 & 1 & 0 &  & 1 & 0 & 1 \\ \hline
    \end{tabular}
  \end{table}
\end{example}
אומנם יש הרבה שאפשר להתעמק על שדות, לא נעשה את זה כעת, זה במסגרת הנושא של תורת שדות או תורת גלואה.

\section{מרחבים ווקטורים ותתי מרחבים}

\begin{definition}[מרחב ווקטורי]
יהי \(\mathbb{F}\) שדה. קבוצה \(V\) נקראת מרחב ווקטורי מעל \(\mathbb{F}\) ביחד עם הפעולות חיבור \(+:V\times V\to V\) וכפל בסקלאר \(\cdot:F\times V\to V\) אשר מקיימות:

  \begin{enumerate}
    \item קווטטיביות  של החיבור - לכל \(u,v \in V\) מתקיים \(u+v=v+u\). 


    \item אסוצייטיביות של החיבור - לכל \(u,v,w \in V\) מתקיים \((u+v)+w=u+(v+w)\). 


    \item קיום איבר ניטרלי לחיבור -  קיים \(0_{V}\in V\) כך שלכל \(v \in V\) מתקיים \(v+0_{V}=v\). 


    \item קיום נגדי חיבורי - לכל \(v \in V\) קיים \((-v) \in V\) כך ש-\(v+(-v)=0_{V}\). 


    \item אסוצייטביות של כפל בסקלר - לכל \(v \in V\) ו-\(c_{1},c_{2} \in \mathbb{F}\) מתקיים \((c_{1}c_{2})v=c_{1}(c_{2}v)\). 


    \item האיבר הניטרלי של הכפל של השדה 1 יקיים \(1\cdot v=v\) לכל \(v \in V\). 


    \item חוק פילוג ראשון - לכל \(u,v \in V\) ולכל \(c \in \mathbb{F}\) מתקיים \(c(u+v)=cu+cv\). 


    \item חוק פילוג שני - לכל \(v \in V\) ולכל \(c_{1},c_{2} \in \mathbb{F}\) מתקיים \((c_{1}+c_{2})v=c_{1}v+c_{2}v\). 


  \end{enumerate}
\end{definition}
\begin{symbolize}
איבר של השדה במרחב ווקטורי נקרא סקלר, כאשר איבר בקבוצה \(V\) נקרא ווקטור.

\end{symbolize}
\begin{remark}
הכפל בסקלר מוגדר \(cv\) עבור \(c \in \mathbb{F}\) ו-\(v \in V\) כאשר הסדר ההפוך של המכפלה לא מוגדר.

\end{remark}
\begin{example}
אם נסתכל על אוסף הסדרות של איברים בשדה \(\mathbb{R}\) - אשר מסומן \(\mathrm{Seq}\left( \mathbb{R} \right)\) ביחד עם פעולת החיבור איבר איבר וכפל בסקלאר:
$$(a+b)_{n}=a_{n}+b_{n}\qquad \left( c\cdot a \right)_{n}=c\cdot a_{n}$$
כאשר \(a,b \in \mathrm{Seq}\left( \mathbb{R} \right)\) ו-\(c \in \mathbb{R}\) ניתן להראות כי מרחב ווקטורי. כאשר במקרה זה נקבל כי אביר \(c \in \mathbb{R}\) נקרא סקלאר ואיבר \((a_n)_{n=1}^\infty \in \mathrm{Seq}\left( \mathbb{R}\right)\) נקרא ווקטור. האיבר הניטרלי לחיבור תהיה סדרת האפס \(a_{n}=0\) לכל \(n \in \mathbb{N}\).

\end{example}
\begin{example}
אוסף הפונקציות הרציפות הממשיות בקטע הסגור בין 0 ל-1 המסומן \(C([0,1])\) יוצר מרחב ווקטור עם פעולות החיבור וכפל בסקלר המוגדרות:
$$(f+g)(x)=f(x)+g(x)\qquad (cf)(x)=c\cdot f(x)$$
כאשר \(f,g \in C([0,1])\) ו-\(c \in \mathbb{R}\) יוצר מרחב ווקטורי כאשר האיבר הניטרלי לחיבור יהיה פונקציית האפס \(f(x)=0\).

\end{example}
\begin{definition}[תת מרחב ווקטורי]
יהי \(V\) מרחב ווקטורי מעל שדה \(\mathbb{F}\) ותהי \(\varnothing\neq U\subseteq V\) קבוצה. אזי אם \(U\) היא מרחב ווקטורי בפני עצמה ביחד עם הפעולות של \(V\) כאשר הם מצומצמות ל-\(U\), נקרא ל-\(U\) תת מרחב ווקטורי של \(V\). מסומן לעיתים \(U\leq V\).

\end{definition}
\begin{proposition}
לכל מרחב ווקטורי \(V\) יש לפחות שתי תתי מרחבים - \(\{ 0 \}\) והמרחב כולו \(V\).

\end{proposition}
\begin{proposition}
יהי \(V\) מרחב ווקטורי מעל שדה \(\mathbb{F}\) ו-\(U\subseteq V\) קבוצה. אזי \(U\) הוא תת מרחב ווקטורי של \(V\) אם"ם מקיים את שלושת התנאים הבאים:

  \begin{enumerate}
    \item מכיל את האיבר זהות חיבורי \(0 \in U\)


    \item סגור תחת חיבור - לכל \(u,w \in U\) מתקיים \(u+w \in U\)


    \item סגור תחת כפל בסקלר - לכל \(c \in \mathbb{F}\) ו- \(u \in U\) מתקיים \(cu \in U\)


  \end{enumerate}
\end{proposition}
\begin{example}
עבור \(V=\mathbb{F} ^{n}\) הקבוצה:
$$U=\left\{  \left.   \begin{pmatrix}x_{1} \\\vdots\\x_{n}
\end{pmatrix}   \;\right\rvert \;x_{n}=0  \right\}$$
תהיה תת מרחב ווקטורי. נראה שמקיים את הדרישות:

  \begin{enumerate}
    \item נשים לב כי \(0 \in U\) כיוון שמקיים את הדרישות. 


    \item אם \(\begin{pmatrix}x_{1}\\\vdots\\x_{n}\end{pmatrix},\begin{pmatrix}y_{1}\\\vdots\\y_{n}\end{pmatrix} \in U\)  אז \(x_{n}=y_{n}=0\) ולכן מתקיים \(x_{n}+y_{n}=0\). כלומר: 
$$\begin{pmatrix}x_{1}\\\vdots\\x_{n}\end{pmatrix}+\begin{pmatrix}y_{1}\\\vdots\\y_{n}\end{pmatrix}=\begin{pmatrix}x_{1}+y_{1}\\\vdots \\ x_{n}+y_{n}\end{pmatrix}= \begin{pmatrix}x_{1}+y_{1}\\\vdots \\ x_{n-1} + y_{n-1}\\ 0
\end{pmatrix}\in U$$


    \item אם \(\begin{pmatrix}x_{1}\\\vdots\\x_{n}\end{pmatrix}\in U\) ו-\(c \in \mathbb{F}\) אזי \(x_{n}=0\) ולכן \(cx_{n}=0\) ולכן הווקטור מקיים את הדרישה ונמצא ב-\(U\). 


  \end{enumerate}
\end{example}
\begin{proposition}
מספיק להראות כי לכל \(u,v \in U\) ולכל \(c \in \mathbb{F}\) מתקיים \(v+cu \in U\). 

\end{proposition}
\begin{proof}
נראה שמקיים את שלושת התנאים מהתענה הקודמת.

  \begin{enumerate}
    \item יהי \(u \in U\) עבור \(c = -1\) נקבל: 
$$u+(-1)u=u-u=0 \in U$$


    \item עבור \(u,w \in U\) ו-\(c=1\) נקבל כי \(v+w \in U\)


    \item כיוון ש-\(0 \in U\) נקבל כי עבור \(u \in U\) ו-\(c \in \mathbb{F}\) כי \(0+cu =cu \in U\). 


  \end{enumerate}
\end{proof}
\begin{proposition}
חיתוך של תתי מרחבים ווקטורים יהיה תת מרחב ווקטורי.

\end{proposition}
\begin{proposition}
איחוד של שתי תתי מרחבים ווקטורים יהיה תת מרחב ווקטורי רק אם תת מרחב אחד מוכל באחרים.

\end{proposition}
כיוון שאיחוד של תתי מרחבים ווקטורים הוא תת מרחב ווקטורי במקרים מועטים, נגדיר מושג חדש.
\textbf{הגדרה} סכום של תתי מרחבים
נניח \(V_{1},\dots,V_{m}\) הם תתי מרחבים של \(V\). נגדיר את הסכום \(V_{1}+\dots+V_{m}\) להיות הקבוצה של הצירוף של הסכומים של כל אחד מהמרחבים. כלומר:
$$V_{1}+\cdots+V_{m}=\{v_{1}+\cdots+v_{m}:v_{1}\in V_{1},...,v_{m}\in V_{m}\}$$

\begin{proposition}
סכום של תתי מרחבים ווקטורים הוא תת מרחב ווקטורי, והוא יהיה גם התת מרחב ווקטורי הקטן ביותר המכיל את כל המרחבים(במובן שלא היתן להוציא ממנו איברים ולהשאר עם תת מרחב ווקטורי המכיל את כל האיברים).

\end{proposition}
\begin{remark}
למעשה הסכום של תתי מרחבים זה המקביל של תורת הקבוצות של איחוד. התורת הקבוצות איחוד של קבוצות זה הקבוצה הקטנה ביותר המכילה את האיברים של כל אחת מהקבוצות, כאשר כאן הסכום הוא התת מרחב ווקטורי הקטן ביותר המכיל את כל אחד מהקבוצות.

\end{remark}
\begin{definition}[סכום ישר]
נניח \(V_{1},\dots, V_{m}\) הם תתי מרחבים של \(V\). אזי \(U=V_{1}+\dots+V_{m}\) נקרא סכום ישר אם כל איבר ב-\(U\) ניתן לכתיבה בצורה יחידה בצורה:
$$v_{1}+\dots+v_{m}$$
כאשר \(v_{k}\in V_{k}\). במקרה זה נסמן:
$$U=V_{1}\oplus\dots \oplus V_{m}$$

\end{definition}
\begin{remark}
סכום ישר זה המקביל של מרחבים ווקטורים של איחוד זר בקבוצות.

\end{remark}
\begin{proposition}
נניח \(V_{1},\dots, V_{m}\) הם תתי מרחבים של \(V\). אזי \(U=V_{1}+\dots+V_{m}\) יהיה סכום ישר אם"ם הדרך היחידה לכתוב 0 כסכום \(v_{1}+\dots+v_{m}\) כאשר \(v_{k}\in V_{k}\) זה כאשר \(v_{k}=0\) לכל \(k\).

\end{proposition}
\begin{proof}
ראשית נניח \(V_{1}+\dots+V_{m}\) הוא סכום ישר. לפי ההגדרה של סכום ישר קיימת דרך יחידה לכתוב את 0 ע"י סכום \(v_{1}+\dots+v_{m}\) כאשר \(v_{k}\in V_{k}\). אנו יודעים כי לכל \(V_{k}\) מתקיים \(0 \in V_{k}\) ואכן \(0+\dots+0=0\) ולכן זו תהיה הדרך היחידה.
כעת נניח כי הדרך היחידה לכתוב את 0 כסכום \(v_{1}+\dots+v_{m}\) כאשר \(v_{k}\in V_{k}\)  זה כאשר \(v_{k}=0\) ונראה כי סכום ישר. יהיה \(v \in U=V_{1}+\dots+V_{m}\). מהגדרת הסכום קיימים \(v_{1},\dots,v_{m}\) כך ש:
$$v=v_{1}+\dots+v_{m}$$
נניח בשלילה כי הצגה זו אינה יחידה. לכן קיימים \(u_{1}\in V_{1},\dots,u_{m}\in V_{m}\) כך ש:
$$v=u_{1}+\dots+u_{m}$$
כאשר \(u_{i}\neq v_{i}\) עבור \(i\) כלשהו(אחרת היו זההים). נסתכל כעת על ההפרש:
$$v-v=(v_{1}-u_{1})+\dots+(v_{m}-u_{m})=0$$
כאשר כל אחד מהאיברים \(v_{i}-u_{i} \in U\) כיוון שתת מרחב ווקטורי ולכן מההנחה נקבל כי \(v_{i}-u_{i}=0\) ולכן \(v_{i}=u_{i}\) לכל \(i\) בסתירה.

\end{proof}
\begin{proposition}
נניח \(U,W \subseteq V\) הם תתי מרחבים ווקטורים. אזי \(U+W\) הוא סכום ישר אם"ם \(U\cup W=\{ 0 \}\).

\end{proposition}
\begin{corollary}
עבור תתי מרחבים \(U_{1},\dots,U_{n}\) נקבל כי \(U_{1}+\dots+U_{n}\) בסכום ישר אם"ם:
$$U_{i}\cap\sum_{i\neq j}U_{j}=\{ 0 \}$$
כאשר מסימטרייה מספיק לבדוק:
$$ U_{i}\cap\sum_{i<j}U_{j}=\{ 0 \}$$

\end{corollary}
כאשר ניתן להראות טענה זו בעזרת אינדוקציה ביחד עם הטענה הקודמת.

\begin{remark}
יש מספר טענות נוספות שקשורות לסכום ישר שנתייחס עליהם בחלק של מטריצות בלוקים.

\end{remark}
\section{מרחבים ווקטורים נוצרים סופית}

\begin{definition}[צירוף לינארי]
יהי \(V\) מרחב ווקטורי. עבור אוסף ווקטורים \(v_{1},\dots,v_{n}\in V\) ואיברים בשדה \(a_{1},\dots,a_{n}\in \mathbb{F}\) הווקטור:
$$v=a_{1}v_{1}+\dots+a_{n}v_{n}$$
נקרא הצירוף הלינארי של הווקטורים \(v_{1},\dots,v_{n}\).

\end{definition}
\begin{definition}[פרוש]
יהי \(V\) מרחב ווקטורי ו-\(v_{1},\dots,v_{n}\in V\) אוסף של ווקטורים. אזי הפרוש של \(v_{1},\dots,v_{n}\) יהיה כל הצירופים הלינאריים של הווקטורים \(v_{1},\dots,v_{n}\). כלומר:
$$\mathrm{Span}\left( v_{1},\dots,v_{n} \right)=\left\{  a_{1}v_{1}+\dots+a_{n}v_{n}\mid a_{1},\dots,a_{n}\in \mathbb{F}   \right\}$$

\end{definition}
\begin{proposition}[תכונות של פרוש]
יהי \(V\) מרחב ווקטורי ו-\(S\subseteq V\) קבוצה של ווקטורים מ-\(V\). אזי מתקיים:

\end{proposition}
\begin{enumerate}
  \item מתקיים \(S\subseteq \mathrm{Span}(S)\)


  \item מונוטוניות - אם \(S_{1},S_{2}\subseteq V\) קבוצה של ווקטורים כך ש-\(S_{2}\subseteq S_{1}\) אזי \(\mathrm{Span}(S_{2})\subseteq \mathrm{Span(S_{1})}\). 


  \item אם \(U\) תת מרחב של \(V\) אז \(U=\mathrm{Span}(U)\). 


\end{enumerate}
\begin{definition}[מרחב ווקטורי נוצר סופית]
אם קיימים אוסף \(v_{1},\dots,v_{n}\) של ווקטורים כך שמתקיים:
$$V=\mathrm{Span}\left( v_{1},\dots,v_{n} \right)$$
אזי נקרא ל-\(V\) מרחב ווקטורי נוצר סופית.

\end{definition}
\begin{definition}[קבוצה פורסת של מרחב ווקטורי]
יהי \(V\) מרחב ווקטורים. אזי קבוצה של ווקטורים \(A=\left\{  v_{1},\dots,v_{n}  \right\}\) נקראת קבוצה פורשת של \(V\) אם:
$$V=\mathrm{Span}(A)=\mathrm{Span}\left\{  v_{1},\dots,v_{n}  \right\}$$

\end{definition}
\begin{definition}[קבוצה בלתי תלוייה לינארית של ווקטורים]
תהי \(A=\left\{  v_{1},\dots,v_{n}  \right\}\) קבוצה של ווקטורים במרחב ווקטורי \(V\). אזי \(A\) קבוצה בלתי תלוייה לינארית(בת"ל) אם הקבוצה הריקה, או שהבחירה היחידה של \(a_{1},\dots,a_{n}\in \mathbb{F}\) כך שמתקיים \(a_{1}v_{1}+\dots+a_{n}v_{n}=0\) יהיה \(a_{1}=a_{2}=\dots=a_{n}\).

\end{definition}
\begin{example}
יהי \(V\) מרחב ווקטורי של סדרות בגודל 4. נסתכל על הקבוצה:
$$A=\{ (1,0,0,0),(0,1,0,0),(0,0,1,0) \}$$
נסתכל על מקדימים \(a_{1},a_{2},a_{3} \in\mathbb{F}\) ונבדוק מתי מתקיים:
$$a_{1}(1,0,0,0)+a_{2}(0,1,0,0)+a_{3}(0,0,1,0)=(0,0,0,0)$$
ונקבל:
$$(a_{1},a_{2},a_{3},0)\,=\,(0,0,0,0)\implies a_{1}=a_{2}=a_{3}=0$$
ולכן בלתי תלוי לינארית.

\end{example}
\begin{proposition}
  \begin{enumerate}
    \item רשימה באורך 1 בלתי תלוייה לינארית אם"ם האיבר היחיד אינו 0. 


    \item רשימה באורך 2 בלתי תלוייה לינארית אם"ם איבר אחד אינו מכפלה סקלארית של האיבר השני. 


  \end{enumerate}
\end{proposition}
\begin{definition}[קבוצה תלוייה לינארית של ווקטורים]
קבוצה של ווקטורים \(A=\left\{  v_{1},\dots,v_{n}  \right\}\) נקראת תלוייה לינארית אם אינה בלתי תלוייה לינארית. כלומר קיימים מקדימים \(a_{1},\dots,a_{n}\in \mathbb{F}\) לא כולם אפס כך שמתקיים:
$$a_{1}v_{1}+\dots+a_{n}v_{n}=0$$

\end{definition}
\begin{example}
יהי \(V\) אוסף הסדרות באורך 4. נסתכל על הקבוצה:
$$V=\{ (1,0,0,0), (0,1,0,0),(1,1,0,0) \}$$
ונשים לב כי עבור \(a_{1}=1,a_{2}=1,a_{3}=-1\) נקבל כי:
$$a_{1}(1,0,0,0)+a_{2}(0,1,0,0)+a_{3}(1,1,0,0)=(1,1,0,0)-(1,1,0,0)=(0,0,0,0)=0$$
ולכן לא בלתי תלוי לינארית ולכן תלוי לינארית.

\end{example}
\begin{proposition}
אם \(A=\left\{  v_{1},\dots,v_{n}  \right\}\) אוסף תלוי לינארית. אז קיים \(1\leq k\leq n\) כך ש- \(v_{k}\in \mathrm{Span}\left( v_{1},\dots ,v_{k-1} \right)\) ומתקיים:
$$\mathrm{Span}  (A)  =\mathrm{Span}\left( A\setminus \{ v_{k} \} \right)$$

\end{proposition}
\begin{proof}
כיוון ש-\(A\) תלוייה לינארית, קיימים \(a_{1},\dots,a_{n}\in \mathbb{F}\) לא כולם אפס כך שמתקיים:
$$a_{1}v_{1}+\cdots+a_{m}v_{m}=0.$$
יהי \(1\leq k\leq n\) הגדול ביותר כך ש-\(a_{k}\neq 0\). כעת ניתן להעביר אגפים, לחלק ולקבל:
$$v_{k}=-{\frac{a_{1}}{a_{k}}}v_{1}-\cdots-{\frac{a_{k-1}}{a_{k}}}v_{k-1}$$
וקיבלנו כי \(v_{k}\in \mathrm{Span}\left( v_{1},\dots ,v_{k-1} \right)\). כעת ניתן לבטא את הפרוש:
\begin{gather*}\mathrm{Span}(A)=\left\{  b_{1}v_{1}+\dots+b_{k-1}v_{k-1} +b_{k}v_{k}+\dots+b_{n}\mid b_{1},\dots,b_{n}\in \mathbb{F}   \right\} = \\=\left\{  b_{1}v_{1}+\dots+b_{k-1}v_{k-1} +b_{k}\left( -{\frac{a_{1}}{a_{k}}}v_{1}-\cdots-{\frac{a_{k-1}}{a_{k}}}v_{k-1} \right)+\dots+b_{n}\mid b_{1},\dots,b_{n}\in \mathbb{F}   \right\}= \\\left\{  \left( b_{1}-\frac{b_{k}a_{1}}{a_{k}} \right)v_{1}+\dots+\left( b_{k-1}-\frac{b_{k}a_{k-1}}{a_{k}} \right)v_{k-1}+0v_{k}+\dots+\left( b_{n}-\frac{b_{k}a_{n}}{a_{k}} \right)v_{n}\mid b_{1},\dots,b_{n}\in \mathbb{F}     \right\}= \\=\mathrm{Span}\left( A \setminus  \{ v_{k} \} \right)
\end{gather*}

\end{proof}
\section{בסיס}

\begin{proposition}
יהי \(V\) מרחב ווקטורי. אם \(A\) קבוצה בלתי תלוייה לינארית ב-\(V\), ו-\(B\) קבוצה פורשת של \(V\). אזי גודל של \(B\) יהיה גדול או שווה לגודל של \(A\). כלומר \(|B|\geq |A|\).

\end{proposition}
\begin{definition}[בסיס]
יהי \(V\) מרחב ווקטורי נוצר סופית. אזי קבוצה של ווקטורים \(\mathcal{A}=\left\{  w_{1},\dots,w_{n}  \right\}\) נקראת בסיס של \(V\) אם קבוצה פורשת וגם קבוצה בלתי תלוייה לינארית.

\end{definition}
\begin{proposition}
יהי \(V\) מרחב ווקטורי נוצר סופית עם בסיס \(\mathcal{A}=\left\{  w_{1},\dots,w_{n}  \right\}\). לכל \(v \in V\) קיימים מקדימים \(a_{1},\dots,a_{n}\in \mathbb{F}\) יחידים כך ש:
$$v=a_{1}w_{1}+\dots+a_{n}w_{n}$$

\end{proposition}
\begin{symbolize}
כאשר אנחנו כותבים את הווקטור \(v\) בבסיס \(\mathcal{A}=\left\{  w_{1},\dots,w_{n}  \right\}\) נסמן \([v]_{\mathcal{A} }\).

\end{symbolize}
\begin{proposition}
יהי \(V\) מרחב ווקטורי. אזי:

  \begin{enumerate}
    \item אם \(A\) קבוצה בלתי תלוייה לינארית המוכלת ב-\(V\). אז ניתן להרחיב אותה לבסיס. כלומר ניתן להוסיף עליה ווקטורים ולקבל בסיס. 


    \item אם \(A\) קבוצה פורשת של \(V\), אז מכילה בסיס. כלומר ניתן להוריד מהקבוצה ווקטורים ולקבל בסיס. 


  \end{enumerate}
\end{proposition}
\begin{proposition}
כל שתי בסיסים יהיו באותו הגודל. 

\end{proposition}
\begin{proof}
נניח כי \(V\) נוצרת סופית. יהיו \(B_{1},B_{2}\) שתי בסיסים של \(V\). אזי \(B_{1}\) בלתי תלוייה לינארית ב-\(V\), ו-\(B_{2}\) פורשת את \(V\) ולכן מטענה שראינו \(|B_{1}|\leq|B_{2}|\). באותו אופן, כיוון ש-\(B_{2}\) בלתי תלוייה לינארית ו-\(B_{1}\) קבוצה פורשת נקבל כי \(|B_{2}|\leq|B_{1}|\). ולכן נקבל סה"כ כי \(|B_{1}|=|B_{2}|\).

\end{proof}
\begin{definition}[מימד של מרחב ווקטורי]
יהי \(V\) מרחב ווקטורי. נגדיר את המימד של המרחב הווקטורי להיות גודל הבסיס. 

\end{definition}
כאשר המימד מוגדר היטב כיוון שלא תלוי בבחירת הבסיס.

\begin{proposition}[מימד של סכום של מרחבים]
$$\dim(V_{1}+V_{2})=\dim V_{1}+\dim V_{2}-\dim(V_{1}\cap V_{2})$$

\end{proposition}
\section{העתקות לינאריות}

\begin{definition}[העתקה לינארית]
תהי \(V,W\) מרחבים ווקטוריים מעל השדה \(\mathbb{F}\). אזי פונקציה \(T:V\to W\) נקראת העתקה לינארית אם מקיימת:

  \begin{enumerate}
    \item לכל \(v, u \in V\) מתקיים \(T(v+u)=T(v)+T(u)\). 


    \item לכל \(c \in \mathbb{F}\) ו-\(v \in V\) מתקיים \(T(cv)=cT(v)\). 


  \end{enumerate}
\end{definition}
\begin{proposition}
אם פונקציה \(T:V\to W\) מקיימת לכל \(v,u \in V\) ו-\(c \in \mathbb{F}\) כי:
$$T(cv+u)=cT(v)+T(u)$$
אז \(T\) היא העתקה לינארית.

\end{proposition}
החשיבות של העתקות לינאריות מגיעה מכך שמשמרת את המבנה של המרחב הווקטורי:

\begin{proposition}
העתקה לינארית מעבירה מרחב ווקטורי למרחב ווקטורי. כלומר אם \(V\) מרחב ווקטורי, ו-\(U\leq V\) תת מרחב ווקטורי, אז אם \(T:V\to W\) העתקה לינארית \(T(U)\) יהיה תת מרחב ווקטורי של \(W\).

\end{proposition}
\begin{proof}
מספיק להראות כי אם \(T(u),T(v) \in T(U)\) ו-\(c \in \mathbb{F}\) אזי:
$$T(u)+cT(v) \in T(U)$$
אנו יודעים כי \(U\) תת מרחב ווקטורי לכן
$$u+cv \in U$$
ולכן קיים \(\tilde{u} \in U\) כך ש:
$$u+cv = \tilde{u}$$
נפעיל את \(T\) ונקבל:
$$T(u)+cT(v)=T\left( \tilde{u} \right)\in T(U)$$
ואכן תת מרחב ווקטורי.

\end{proof}
\begin{proposition}
יהי \(T:V\to W\) העתקה לינארית. אזי האפס ב-\(V\) לאפס ב-\(W\). כלומר \(T(0_{V})=0_{W}\).

\end{proposition}
\begin{proof}
$$T(0_{V})=T\left( 0\cdot 0_{V} \right)=0\cdot T(0_{V})=0_{W}$$

\end{proof}
\begin{proposition}
תהא \(\left( a_{1},\dots,a_{n} \right)\) בסיס של \(V\). אזי ההעתקה הלינארית \(T:V\to W\) נקבעת ביחידות על ידי לאן שולחת את איברי הבסיס.

\end{proposition}
\begin{proof}
כיוון שנתון איברי הבסיס קיימים מקדימים \(v_{1},\dots,v_{n}\in \mathbb{F}\) כך שמתקיים:
$$v=v_{1}a_{1}+v_{2}a_{2}+\dots+v_{n}a_{n}$$
נפעיל את ההעתקה \(T:V\to W\) ונקבל:
$$T(v)=T\left( v_{1}a_{1}+v_{2}a_{2}+\dots+v_{n}a_{n} \right)=v_{1}T(a_{1})+v_{2}T(a_{2})+\dots+v_{n}T(a_{n})$$
כאשר נשים לב כי קיבלנו העתקה מפורשת - \(T(a_{i})\) זה ווקטור מפורש לכל \(i\) כאשר \(v_{i}\) הם קבועים.

\end{proof}
\begin{definition}[גרעין של העתקה]
יהי \(V,W\) מרחבים וקטורים ו-\(T:V\to W\) העתקה לינארית. הגרעין(kernel) של \(T\) תהיה קבוצת כל האיברים ב-\(V\) אשר הולכים ל-0. כלומר:
$$\ker (T)=\left\{  v\mid T(v)=0_{W}  \right\}\subseteq V$$

\end{definition}
\begin{proposition}
הגרעין הוא תת מרחב ווקטורי.

\end{proposition}
\begin{proof}
יהיו \(v_{1},v_{2} \in \ker T\) ו-\(c \in \mathbb{F}\). מספיק להראות כי \(v_{1}+cv_{2}\in \ker T\). מתקיים:
$$T(v_{1}+cv_{2})=T(v_{1})+cT(v_{2})=0_{W}+c\cdot 0_{W}=0_{W}\implies v_{1}+cv_{2}\in \ker  T$$

\end{proof}
\begin{proposition}
העתקה לינארית \(T:V\to W\) חד חד ערכית אם"ם \(\ker(T)=\{ 0_{V} \}\).

\end{proposition}
\begin{proof}
  \begin{enumerate}
    \item נניח כי \(\ker (T)=0\). נניח בשלילה כי \(T\) לא חח"ע. כלומר קיימים \(v_{1},v_{2} \in V\) כך ש-\(v_{1}\neq v_{2}\) ומתקיים \(T(v_{1})=T(v_{2})\). 


    \item כעת \(T(v_{1})-T(v_{2})=0\) ולכן \(T(v_{1}-v_{2})=0\) מלינאריות. 


    \item קיבלנו כי \(v_{1}-v_{2} \in \ker(T)\) ולכן \(v_{1}-v_{2}=0_{V}\) כלומר \(v_{1}=v_{2}\) בסתירה. 


  \end{enumerate}
\end{proof}
\begin{definition}[תמונה של העתקה]
יהי \(V,W\) מרחבים וקטורים ו-\(T:V\to W\) העתקה לינארית. התמונה של \(T\) תהיה קבוצת כל האיברים ב-\(W\)$$\mathrm{Im}(T)=T(V)=\left\{  T(v)\mid v \in V  \right\}\subseteq W$$
כאשר נשים לב כי זה תהיה תת קבוצה של \(V\).

\end{definition}
\begin{proposition}
התמונה של העתקה \(T:V\to W\) יהיה תת מרחב ווקטורי של \(W\).

\end{proposition}
\begin{symbolize}
נסמן את המימד של הגרעין ב-\(\text{null } T\), ואת המימד של התמונה ב-\(rk(T)\).

\end{symbolize}
\begin{theorem}[המימדים]
יהיו \(V,W\) מרחבים ווקטורים מעל \(\mathbb{F}\) כך ש-\(V\) נוצר סופית. תהי \(T:V\to W\) העתקה לינארית. אזי \(\ker (T),\mathrm{Im}(T)\) נוצרים סופית ומתקיים:
$$\text{null }(T)+\mathrm{rk}(T)=\dim V$$

\end{theorem}
\begin{proof}
  \begin{enumerate}
    \item מאחר ש-\(V\) נוצר סופית, גם \(\ker(T)\) נוצר סופית, ולכן קיים בסיס \(\left( v_{1},\dots,v_{k} \right)\) של \(\mathrm{\ker}(T)\). 


    \item מאחר ש-\(\left( v_{1},\dots,v_{k} \right)\) בלתי תלוייה לינארית, קיימים \(\left( v_{k+1},\dots,v_{n} \right)\) כך ש-\(\left( v_{1},\dots,v_{n} \right)\) מהווים בסיס של \(V\). 


    \item נוכיח כי \(\left( T(v_{k+1}),\dots ,T(v_{n}) \right)\) בסיס של \(\mathrm{\mathrm{Im}}(T)\). מאחר ש-\(T(v_{1})=T(v_{2})=\dots =T(v_{k})=0_{W}\) (כי נמצאים בגרעין) נקבל כי: 
$$\mathrm{Span}\left( \left\{  T(v_{k+1}),\dots T(v_{n})  \right\} \right)=\mathrm{Span}\left( \left\{  T(v_{1}),\dots,T(v_{n})  \right\} \right)=\mathrm{\mathrm{Im}}(T)$$


    \item נוכיח כי הסדרה \(\left( T(v_{k+1}),\dots,T(v_{n}) \right)\) בלתי תלוייה לינארית. יהיו \(c_{k+1},\dots,c_{n}\in \mathbb{F}\). כך שמתקיים: 
$$c_{k+1}T(v_{k+1})+\dots+c_{n}T(v_{n})=0_{W}\implies T\left( c_{k+1}v_{k+1}+\dots+c_{n}v_{n} \right)=0_{W}$$


    \item קיבלנו כי \(c_{k+1}v_{k+1}+\dots+c_{n}v_{n}\in \ker (T)\). ולכן ניתן לבטא בעזרת הבסיס של הגרעין. כלומר קיימים \(d_{1},\dots,d_{k}\) כך ש: 
$$c_{k+1}v_{k+1}+\dots+c_{n}v_{n}=d_{1}v_{1}+\dots+d_{k}v_{k}\implies c_{k+1}v_{k+1}+\dots+c_{n}v_{n}-d_{1}v_{1}-\dots-d_{k}v_{k}=0$$
וכיוון שהווקטורים \(v_{1},\dots,v_{n}\) בלתי תלויים לינארית נקבל כי:
$$-d_{1}=-d_{2}=\dots=-d_{k}=c_{k+1}=\dots=c_{n}=0$$


    \item קיבלנו כי הסדרה \(\left( T(v_{k+1}),\dots,T(v_{n}) \right)\) בלתי תלוייה לינארית ולכן בסיס של \(\mathrm{Im}(T)\). ולכן מתקיים: 
$$\text{null }(T)+\mathrm{rk}(T)=k+(n-k)=n=\dim(V)$$


  \end{enumerate}
\end{proof}
\begin{corollary}
יהיו \(V,W\) מרחבים ווקטורים כך ש-\(V\) נוצר סופית. אם העתקה לינארית \(T:V\to W\) היא חח"ע ועל אז \(\dim(V)=\dim(W)\).

\end{corollary}
\begin{proof}
  \begin{enumerate}
    \item כיוון שחח"ע מתקיים \(\ker(T)=\{ 0_{V} \}\) ולכן \(\text{null }T=0\). 


    \item כיוון שעל נקבל כי \(\mathrm{Im}(T)=W\) ולכן \(\mathrm{rk}(T)=\dim (W)\). 


    \item כעת ממשפט במימדים: 
$$\dim(V)=\text{null }T+\mathrm{rk\;}T=0+\dim W=\dim W$$


  \end{enumerate}
\end{proof}
\section{הרכבה של העתקות והעתקה הופכית}

\begin{proposition}
הרכבה של העתקות לינאריות היא העתקה לינארית

\end{proposition}
\begin{proof}
תהי \(T_{1}:V\to U\) ו-\(T_{2}:U\to W\) העתקות לינאריות. לכן מתקיים לכל \(c \in \mathbb{F}\):
$$\forall v_{1},v_{2} \in V\quad T_{1}(v_{1}+cv_{2})=T_{1}(v_{1})+cT_{1}(v_{2})\qquad \forall u_{1},u_{2} \in U\quad  T_{2}(u_{1}+cu_{2})=T_{2}(u_{1})+cT_{2}(u_{2})$$
וכעת עבור \(v_{1},v_{2} \in V\) נקבל:
$$T_{2}\circ T_{1}(v_{1}+cv_{2})=T_{2}(T_{1}(v_{1})+cT_{1}(v_{2}))=T_{2}\circ T_{1}(v_{1})+cT_{2}\circ T_{1}(v_{2})$$
ואכן העתקה לינארית.

\end{proof}
\begin{definition}[הופכי שמאלי]
תהי \(V, W\)  מרחבים ווקטורים ו-\(T:V\to W\) העתקה לינארית. העתקה \(S:W\to V\) נקראת ההעתקה הופכית שמאלית אם מקיימת:
$$S\circ  T=Id_{V}$$

\end{definition}
\begin{proposition}[תכונות של ההופכי השמאלי]
  \begin{enumerate}
    \item כדי שיהיה קיים הופכי שמאלי, נדרש כי \(T\) תהיה חח"ע. 


    \item ההופכי השמאלי \(S\) יהיה על. 


    \item חוסר יחידות - אם קיים הופכי שמאלי, הוא לא בהכרח יהיה היחיד. 


  \end{enumerate}
\end{proposition}
\begin{proof}
  \begin{enumerate}
    \item נניח שקיים הופכי שמאלי. כעת: 
$$T(v_{1})=T(v_{2})\implies S(T(v_{1}))=S(T(v_{2}))\implies Id(v_{1})=Id(v_{2})\implies v_{1}=v_{2}$$
ולכן חח"ע.


    \item אנו יודעים כי מתקיים \(S(T(v))=v\). לכן עבור כל \(v \in V\) קיים איבר \(w \in W\) כך ש-\(S(w)=v\), כאשר במקרה שלנו \(w=T(v)\), ולכן על. 


    \item מספיק להראות דוגמא. נגדיר \(T(x,y)=(x,y,z)\). וכן נגדיר שתי הופכיים שמאליים: 
$$S_{1}(a,b,c)=(a,b)\qquad S_{2}(a,b)=(a,b)$$
וניתן להראות כי אכן הופכיים שמאליים:
$$S_{1}\circ  T(a,b,c)=S_{1}(a,b)=(a,b,0)\neq Id_{V}$$


  \end{enumerate}
\end{proof}
\begin{definition}[העתקה הופכית]
תהי \(V, W\)  מרחבים ווקטורים ו-\(T:V\to W\) העתקה לינארית. העתקה \(T^{-1}:W\to V\) נקראת ההעתקה ההופכית אם מקיימת:
$$T^{-1}\circ  T=Id_{W}\qquad T\circ T^{-1}=Id_{V}$$

\end{definition}
\begin{proposition}
הביטויים הבאים שקולים

  \begin{enumerate}
    \item להעתקה לינארית \(T\) קיים הופכי. 


    \item ההעתקה \(T\) היא חח"ע ועל. 


  \end{enumerate}
\end{proposition}
\section{מרחב ההעתקות הלינאריות}

\begin{proposition}
אוסף כל ההעתקות הלינאריות ממרחב \(V\) למרחב \(W\) יהיה בעל מבנה של מרחב ווקטורי. מסומן לעיתים \(\mathrm{Hom}(V,W)\) או \(\mathcal{L}(V,W)\).

\end{proposition}
\begin{definition}[איזומורפיזם]
העתקה לינארית הפיכה

\end{definition}
\begin{definition}[מרחבים איזומורפיים]
מרחבים ווקטורים שקיים ביניהם איזומורפיזם.

\end{definition}
\begin{remark}
הרעיון באיזומורפיזם שזה משמר מבנה - כלומר \(T(v+w)=Tv+Tw\) וגם \(Tcv=cTv\) - כלומר העתקה לינארית, וגם ממפה כל ווקטור לווקטור מתאים אחר במרחב השני, ולכן אפשר לחשוב על זה למעשה כשינוי שם של האיברים.

\end{remark}
\begin{proposition}
שתי מרחבים ווקטורים \(V,W\) מעל \(\mathbb{F}\) הם איזומורפיים אם"ם הם מאותו מימד.

\end{proposition}
\begin{proof}
אם איזומורפיים, אז קיימת העתקה לינארית \(T:V\to W\) חח"ע ועל. כיוון שחח"ע מתקיים \(\text{null}(T)=0\). כיוון שעל מתקיים \(rk(T)=\dim W\). וכעת ממשפט המימדים \(\dim V=\dim W\).
כעת נניח כי מאותו מימד. לכן ניקח בסיס ל-\(V\) אשר נסמן \(\left( v_{1},\dots,v_{n} \right)\) ובסיס ל-\(W\) אשר נסמן \(\left( w_{1},\dots,w_{n} \right)\). נגדיר העתקה המקיימת:
$$\forall i\quad T(v_{i})=w_{i}$$
כאשר נשים לב כי העתקה זו היא הפיכה ולכן איזומורפיזם.

\end{proof}
\begin{proposition}
מימד המרחב \(\mathrm{Hom}(V,W)\) יקיים:
$$\dim \left( \mathrm{Hom}(V,W) \right)=\left( \dim V \right)\left( \dim  W \right)$$

\end{proposition}
\Chapter{אלגברת מטריצות}

\section{מטריצות - הגדרות וסימונים}

\begin{definition}[מטריצה מגודל \(m\times n\)]
בהנתן שדה \(\mathbb{F}\) נגדיר את הטבלה הדו מימדית 
$$A=\begin{bmatrix}a_{11}&a_{12}&\cdot\cdot\cdot&a_{1n}\\ a_{21}&a_{22}&\cdot\cdot\cdot&a_{2n}\\ \vdots&\vdots&\cdot\cdot&\vdots\\ a_{m1}&a_{m2}&\cdot\cdot\cdot&a_{m n}\end{bmatrix}$$
בתור מטריצה, כאשר \(a_{ij}\in \mathbb{F}\). הגודל \(m,n\) נקרא המימד של המטריצה.

\end{definition}
\begin{symbolize}
נסמן מטריצות באותיות אנגליות גדולות(למשל \(A,B,C\)) כאשר נסמן את הרכיב בשורה ה-\(i\) ועמודה ה-\(j\) ע"י \(A_{ij}\). 

\end{symbolize}
\begin{symbolize}
אוסף המטריצות מגודל \(m\times n\) תחת שדה \(\mathbb{F}\) יסמון \(M_{m,n}\left( \mathbb{F} \right)\) או \(M_{m\times n}\left( \mathbb{F}  \right)\).

\end{symbolize}
\begin{definition}[n-יה]
מטריצה עם עמודה אחת. למעשה אוסף של \(n\) איברים בשדה.

\end{definition}
\begin{symbolize}
את המטריצה שיש בכל הערכים \(0\) פרט לרכיב ה-\(ij\), אשר שם יהיה 1, נסמן ב-\(e_{ij}\). כך אפשר למשל כל מטריצה לכתוב בצורה:
$$A=\sum_{i}\sum_{j}a_{ij}e_{ij}=\sum_{i,j}a_{ij}e_{ij}$$
כאשר אם כתוב רק אינדקס אחד(כלומר כתוב \(e_{i}\)), זה מסמן את ה-\(n\)-יה שמכילה אפסים בכל מקום פרט למקום ה-\(i\).

\end{symbolize}
\begin{definition}[הצגה של ווקטור בעזרת בסיס]
יהי \(V\) מרחב ווקטורי נוצר סופית, ו-\(\left( a_{1},\dots,a_{n} \right)\) בסיס למרחב ווקטורי.

\end{definition}
\begin{definition}[חיבור מטריצות]
בהינתן \(A,B \in M_{n,m}\left( \mathbb{F} \right)\), נדיר את הסכום \(A,B\) להיות הסכום איבר איבר. כלומר:
$$A+B=\sum_{i,j}(a_{i j}+b_{i j})e_{ij}$$
כאשר נשים לב כי אם \(A,B\) אינם מאותם מימדים, אז החיבור לא מוגדר.

\end{definition}
\begin{definition}[נגדי של מטריצה]
זו תהיה מטריצה \(-A\) שכל רכיב שלה יהיה הנגדי של הרכיב המתאים במטריצה \(A\). כלומר:
$$-A=\sum_{i,j}-a_{i j} e_{ij}$$

\end{definition}
\begin{definition}[חיסור של מטריצות]
זה יהיה החיבור של מטריצה עם הנגדי:
$$A-B=A+(-B)=\sum_{i,j}(a_{i j}-b_{i j})e_{ij}$$

\end{definition}
\begin{definition}[כפל בסקלאר של מטריצה]
יהי \(k \in \mathbb{F}\) סקלאר. נגדיר את המכפלה של מטריצה בסקלר להיות:
$$k A=\sum _{i,j}k a_{i j}e_{ij}$$

\end{definition}
\begin{proposition}
יהיו \(A,B,C\in M_{n,m}\), ו-\(k,p \in \mathbb{F}\) סקלארים. אזי מתקיים:

  \begin{enumerate}
    \item \(A+B=B+A\)


    \item \(A+(B+C)=(A+B)+C\)


    \item \(0+A=A\)


    \item \(A+(-A)=0\)


    \item \(k(A+B)=kA+kB\)


    \item \((k+p)A=kA+pA\)


    \item \((kp)A=k(pA)\)


    \item \(1A=A\)


  \end{enumerate}
\end{proposition}
\begin{corollary}
עבור \(n,m\) הקבוצה \(M_{m,n}\left( \mathbb{F}  \right)\) ביחד עם השדה \(\mathbb{F}\) והפעולות של חיבור מטריצות וכפל בסקלאר נקבל מרחב ווקטורי. לכן גם הביטוי "כפל בסקלר" הוא מתאים.

\end{corollary}
\begin{example}
מקרה חשוב זה \(\mathbb{F} ^{n}\). זה אוסף כל ה-\(n\)-יות.

\end{example}
\begin{definition}[דרגה של מטריצה]
כמות העמודות הבלתי תלוייות שיש למטריצה. כלומר אם \(A \in M_{n,m}\left( \mathbb{F}  \right)\) מטריצה, ניתן לסמן את השורות ב-\(a_{i}\in \mathbb{F} ^{n}\). אזי:
$$\mathrm{rank}(A)=\mathrm{dim}\left( \mathrm{Span}\left\{  a_{1},a_{2},\dots,a_{m}  \right\} \right)$$

\end{definition}
\section{העתקות לינאריות ומכפלת מטריצות}

המוטיבציה של מכפלת מטריצות מגיעה מהעולם של העתקות לינאריות. כפי שנראה, מכפלת מטריצה בווקטור זה כמו הפעלה של העתקה לינארית על ווקטור ומכפלה של שתי מטריצות זה כמו הרכבה של שתי העתקות לינאריות. למעשה נגדיר אותם כך שזה מתקיים.

\begin{definition}[מכפלת מטריצה בווקטור]
נניח ו-\(A\in M_{m,n}\left( \mathbb{F} \right)\) היא מטריצה עם עמודות \(a_{1},\dots,a_{n}\in \mathbb{F} ^{m}\). כלומר:
$$A=\begin{bmatrix}\mathbf{a}_{1}&\mathbf{a}_{2}&\cdots&\mathbf{a}_{n}\end{bmatrix}$$
עבור \(x \in \mathbb{F} ^{n}\) נגדיר את מכפלת המטריצה בווקטור:
$$A\mathbf{x}:=x_{1}\mathbf{a}_{1}+x_{2}\mathbf{a}_{2}+\cdots+x_{n}\mathbf{a}_{n}\in\mathbb{R}^{m}$$

\end{definition}
\begin{example}
עבור:
$$A={\left[\begin{array}{c c c}{2}&{-1}&{0}\\ {3}&{1/2}&{\pi}\\ {-2}&{1}&{1}\\ {0}&{0}&{0}\end{array}\right]}\quad\quad{\mathbf{x}}={\left[\begin{array}{c}{-1}\\ {1}\\ {2}\end{array}\right]}$$
נקבל:
$$A\mathbf{x}=-1\begin{bmatrix}2\\ 3\\ -2\\ 0\end{bmatrix}+1\begin{bmatrix}-1\\ 1/2\\ 1\\ 0\end{bmatrix}+2\begin{bmatrix}0\\ \pi\\ 1\\ 0\end{bmatrix}=\begin{bmatrix}-3\\ -5/2+2\pi\\ 5\\ 0\end{bmatrix}$$

\end{example}
\begin{definition}[מטריצה מייצגת של העתקה לינארית]
בהנתן בסיס \(\mathcal{A}=\left( a_{1},\dots,a_{n} \right)\) של \(V\) ובסיס \(\mathcal{ B}=\left( b_{1},\dots,b_{m} \right)\) של \(W\) ניתן להציג את ההעתקה הלינארית בצורה הבאה:
\begin{gather*}[T(v)]_{\mathcal{B} }=\left[ T\left( v_{1}a_{1}+v_{2}a_{2}+\dots+v_{n}a_{n} \right) \right]_{\mathcal{_{B}} }= \\=v_{1}[T(a_{1})]_{\mathcal{B} }+v_{2}[T(a_{2})]_{\mathcal{B} }+\dots+v_{n}[T(a_{n})]_{\mathcal{B} }= \\=\begin{pmatrix}[T(a_{1})]_{\mathcal{B} }\bigg| [T(a_{2})]_{\mathcal{B} }\bigg| \dots \bigg| [T(a_{n})]_{\mathcal{B} }\end{pmatrix}\begin{pmatrix}v_{1}\\v_{2}\\\vdots\\v_{n}\end{pmatrix}_{\mathcal{A} } 
\end{gather*}
כאשר נסמן את המטריצה 
$$[T]_{\mathcal{B}}^{\mathcal{A}}=\begin{pmatrix}[T(a_{1})]_{\mathcal{B}}\bigg| T(a_{2})_\left[ \mathcal{B} \right]\bigg| \dots \bigg| [T(a_{n})]_{\mathcal{B}}\end{pmatrix}$$
ונקרא לה המטריצה המייצגת של ההעתקה \(T\) תחת הבסיס \(\mathcal{ A}\). זוהי מטריצה שמקבלת ווקטור ממרחב \(V\) בהצגה בבסיס \(\mathcal{A}\), מפעילה את ההעתקה ומחזירה את בתוצאה תחת בסיס \(\mathcal{B}\).

\end{definition}
\begin{remark}
למעשה אנחנו מגדירים את המכפלת מטריצה בווקטור כך כדי שיתקיים:
$$[T(v)]_{\mathcal{B} }=[T]_{\mathcal{B} }^{\mathcal{A} }[v]_{\mathcal{A} }$$

\end{remark}
כעת נרצה להגדיר את המכפלת מטריצות כמקביל להרכבה של העתקות לינאריות.

\begin{definition}[מכפלת מטריצות]
עבור \(A \in M_{m,n}\left( \mathbb{F}  \right)\) ו-\(B \in M_{n,k}\left( \mathbb{F}  \right)\) נסמן ב-\(b_{i}\in \mathbb{F} ^{n}\) את העמודה ה-\(i\) של מטריצה \(B\). אזי נגדיר את מכפלת המטריצות להיות:
$$A B:=\left[A{\bf b}_{1}\quad A{\bf b}_{2}\quad\cdot\cdot\cdot\quad A{\bf b}_{k}\right]$$
כאשר נקבל מטריצה \(AB \in M_{m,k}\left( \mathbb{F}  \right)\)

\end{definition}
\begin{corollary}
הרכיב ה-\(lk\) של המטריצה \(AB\) יהיה:
$$AB_{lk}=\sum_{i=1}^{n}\sum_{j=1}^{m} a_{il}b_{kj}$$

\end{corollary}
למעשה הגדרנו את המכפלת מטריצת כך שהמטריצה המייצגת של הרכבה של העתקות לינאריות זה יהיה המכפלת מטריצות של המטריצות המייצגות, כלומר כך שיתקיים הטענה הבאה:

\begin{proposition}
תהי \(T_{1}:V\to U\) ו-\(T_{2}:U\to W\) העתקות לינאריות כך של-\(U\) יש בסיס \(\mathcal{C}\), ל-\(W\) יש בסיס \(\mathcal{A}\) ול-\(V\) יש בסיס \(\mathcal{B}\). אזי המטריצה המייצגת של הרכבה של העתקות לינאריות תהיה המכפלה של המטריצות המייצגות של כל העתקה בנפרד. כלומר:
$$\left[ T_{1}\circ  T_{2} \right]_{\mathcal{A} }^{\mathcal{B} }=[T_{1}]_{\mathcal{A} }^{\mathcal{C} }[T_{2}]_{\mathcal{C} }^{\mathcal{B} }$$
כאשר נשים לב כי התוצאה לא תלויה בבסיס \(\mathcal{C}\).

\end{proposition}
\begin{proof}
נסמן \(\mathcal{B=\left( b_{1},\dots,b_{n} \right)}\).
אנו יודעים כי העמודות של \(\left[ T_{1}\circ T_{2} \right]_{\mathcal{A}}^{\mathcal{B}}\) יהיה:
$$\left[ T_{1}\circ T_{2} \right]_{\mathcal{A} }^{\mathcal{B} }=\begin{bmatrix}\left[ T_{1}\circ  T_{2}(b_{1}) \right]_{\mathcal{A} }\bigg| \left[ T_{1}\circ  T_{2}\left( b_{2} \right) \right]_{\mathcal{A} }\bigg|\dots \bigg| \left[ T_{1}\circ T_{2} (b_{n}) \right]\mathcal{_{A}}  
\end{bmatrix} $$
נסתכל על העמודה ה-\(i\). נקבל כי היא תהיה שווה:
$$\left[ T_{1}\circ T_{2}(b_{i}) \right]_{\mathcal{A} }=[(T_{1}(T_{2}(b_{i}))]_{A}$$
כאשר ניתן לכתוב את \(T_{2}(b_{i})\) בעזרת הבסיס \(\mathcal{C}=\left( c_{1},\dots,c_{m} \right)\):
$$T_{2}(b_{i})=\sum_{k=1}^{m}\left([T_{2}(b_{i})]c\right)_{k}c_{k}$$
וכעת:
$$T_{1}(T_{2}(b_{i}))=T_{1}\left(\sum_{k=1}^{m}\left([T_{2}(b_{i})]c\right)_{k}c_{k}\right)=\sum_{k=1}^{m}\left([T_{2}(b_{i})]c\right)_{k}T_{1}(c_{k})$$
כאשר ניתן לכתוב את \(T_{1}(c_{k})\) בעזרת הבסיס \(\mathcal{A}=\left( a_{1},\dots,a_{p} \right)\):
$$T_{1}(c_{k})=\sum_{j=1}^{p}\left([T_{1}(c_{k})]_{\mathcal{A}}\right)_{j}a_{j}$$
וכעת:
$$T_{1}(T_{2}(b_{i}))=\sum_{k=1}^{m}\sum_{j=1}^{p}\left([T_{2}(b_{i})]c\right)_{k}\left([T_{1}(c_{k})]_{\mathcal{A}}\right)_{j}a_{j}$$
וקיבלנו כי זה בדיוק המכפלת מטריצה תחת הבסיס \(\mathcal{ A}\).

\end{proof}
\begin{theorem}
פונקציה היא העתקה לינארית אם"ם היא מייצגת מכפלת מטריצות תחת בסיס כלשהו. 

\end{theorem}
\section{מכפלת מטריצות מ-4 נקודות מבט}

\subsection{לפי ערכים}

\begin{definition}[מכפלת מטריצות לפי ערכים]
יהי \(A \in M_{n,m}\left( \mathbb{F}  \right)\) ו-\(B\in M_{m,p}\left( \mathbb{F}  \right)\) מטריצות. אזי הרכיב ה-\(k\ell\) של \(AB\) יהיה:
$$(AB)_{k\ell}=\mathrm{row}_{k}(A)\cdot \mathrm{col}_{\ell}(B)=\sum_{i}^{n} A_{ik}\cdot B_{\ell j}$$

\end{definition}
מגישה זו ניתן להגיע לתובנות הבאות:

\begin{enumerate}
  \item כמות העמודות של \(A\) צריך להיות שווה לכמות השורות של \(B\). 


  \item המכפלה \(AB\) תהיה מטריצה עם אותה כמות שורות כמו \(A\) וכמות עמודות כמו \(B\). 


  \item ערך של 0 אומר כי המכפלה הסקלארית הסטנרטית תהיה 0, ולכן הווקטור המייצג את השורה ב-\(A\) ועמודה ב-\(B\) יהיו אורתוגונאלים. 


\end{enumerate}
\subsection{לפי עמודה}

\begin{definition}[מכפלת מטריצות לפי עמודות]
עבור \(A \in M_{m,n}\left( \mathbb{F}  \right)\) ו-\(B \in M_{n,k}\left( \mathbb{F}  \right)\) נסמן ב-\(b_{i}\in \mathbb{F} ^{n}\) את העמודה ה-\(i\) של מטריצה \(B\). אזי נגדיר את מכפלת המטריצות להיות:
$$A B:=\left[A{\bf b}_{1}\quad A{\bf b}_{2}\quad\cdot\cdot\cdot\quad A{\bf b}_{k}\right]$$
כאשר נקבל מטריצה \(AB \in M_{m,k}\left( \mathbb{F}  \right)\)

\end{definition}
מהגישה הזו ניתן להגיע לתובנות הבאות:

\begin{enumerate}
  \item מכפלת מטריצה בווקטור \(Ax\) זה פשוט הצירוף הלינארי של העמודות של \(A\) בערך המתאימים של \(x\). ולכן העמודות של \(A\) הם בלתי תלויות לינארית אם"ם למשוואה \(Ax=0\) יש פתרון יחיד. 


  \item העמדות של \(AB\) מתקבלות על ידי הפעלת אופרטור לינארי על העמודות של \(B\). המוגדר על ידי כפל ב-\(A\). 


  \item אם העמדות של \(A\) הם בלתי תלויות לינארית, אז אם ל-\(AB\) הייתה עמודת אפסים, גם ל-\(B\) הייתה עמודת אפסים. 


  \item אם העמודות של \(AB\) הם תלויות לינארית ועמודות של \(B\) הם בלתי תלויות לינארית אז העמודות של \(A\) תלויות לינארית. זה נובע מזה שאם \(x\) פתרון לא טריוויאלי של \(ABx=0\) אז \(Bx\) פתרון לא טריוויאלי של \(Ax=0\). 


  \item אם למשוואה \(Ax=0\) אין פתרון אז גם ל-\(ABx=0\) אין פתרון, הרי \(AB\) זה פשוט צירוף לינארי של העמודות של \(A\). 


  \item המרחב עמודות של \(AB\)(כלומר הפרוש של עמודות) מוכל במרחב עמודות של \(A\), ולכן \(\mathrm{rk}(AB)\leq \mathrm{rk}(A)\). 


  \item אם \(B\) הפיך אז המרחב עמודות של של \(A\) ושל \(AB\) יהיה זהה. נובע מהעובדה הקודמת ומזה שמתקיים: 
$$\mathrm{rk}(A)=\mathrm{rk}(AB)=\mathrm{rk}(ABB^{-1})\leq\mathrm{rk}(AB)$$


\end{enumerate}
\subsection{לפי שורה}

\begin{definition}[מכפלת מטריצות לפי שורות]
עבור \(A \in M_{m,n}\left( \mathbb{F}  \right)\) ו-\(B \in M_{n,k}\left( \mathbb{F}  \right)\) נסמן ב-\(a_{i}\in M_{1,n}\) את השורה ה-\(i\) של מטריצה \(A\). אזי נגדיר את מכפלת המטריצות להיות:
$$A B:=\begin{bmatrix}a_{1}B\\ a_{2} B\\ \vdots \\a_{m}B
\end{bmatrix}$$
כאשר נקבל מטריצה \(AB \in M_{m,k}\left( \mathbb{F}  \right)\)

\end{definition}
מהגישה הזו ניתן לקבל את התובנות הבאות:

\begin{enumerate}
  \item אם השורות של \(AB\) הם בלתי תלויות לינארית אז גם השורות של \(B\). 


  \item אם ל-\(A\) יש שורת אפסים אז גם ל-\(AB\) יש שורת אפסים. 


  \item מרחב השורות של \(B\) כולל את מרחב השורות של \(AB\). 


  \item אם \(E\) מטריצת \(n\times n\) הפיכה ו-\(B\) היא מטריצת \(n\times m\) אזי ל-\(EB\) יש את אותו מרחב שורות כמו ל-\(B\). בפרט פעולות שורה אלמנטרית משמרות את מרחב השורות. 


\end{enumerate}
\subsection{לפי שורות ועמודות}

\begin{definition}[מכפלת מטריצות לפי שורות עמודות]
עבור \(A \in M_{m,n}\left( \mathbb{F}  \right)\) ו-\(B \in M_{n,k}\left( \mathbb{F}  \right)\) נסמן ב-\(b_{i}\in \mathbb{F} ^{n}\) את השורה ה-\(i\) של מטריצה \(B\), וב-\(a_{i}\in M_{1,n}\) את העמודה ה-\(i\) של מטריצה \(A\). אזי נגדיר את מכפלת המטריצות להיות:
$$AB=\sum_{i=1}^{n} a_{i}\cdot b_{i}$$
כאשר זה מוגדר היטב כי המכפלה מוגדרת רק כאשר כמות העמודות של המטריצה \(A\) שווה לכמות השורות של מטריצה \(B\).

\end{definition}
תובנות:
- ניתן להחליף שתי שורות של \(A\) ולקבל את אותה מכפלה כל עוד מחליפים את השתי שורות המתאימות ב-\(B\).

\section{מטריצות בלוקים ומטריצות אלכסוניות}

\begin{definition}[מטריצת בלוקים]
ניתן לסמן חלקים של מטריצות בתור "בלוקים" ולקבל מטריצה שמכילה "מטריצות" בתור רכיבים.

\end{definition}
\begin{example}
נסתכל על המטריצה:
$$A=\begin{bmatrix}1&2&-1&3&5\\ 0&3&0&4&7\\ 5&-4&2&0&1\end{bmatrix}$$
ניתן לסמן את המטריצה בעזרת בלוקים בצורה הבאה:
$$\left[ \begin{array}{c} B \\ D \end{array} \quad  \begin{array}{c} C \end{array} \right],\quad\text{where}\quad B=\begin{bmatrix}1&2\end{bmatrix},\quad C=\begin{bmatrix}-1&3&5\\ 0&4&7\\ 2&0&1\end{bmatrix},\quad D=\begin{bmatrix}0&3\\ 5&-4\end{bmatrix}$$

\end{example}
\begin{proposition}
חיבור וחיסור של מטריצות בלוקים נעשה בלוק בלוק.

\end{proposition}
\begin{proposition}[מכפלת מטריצת בלוקים]
כל עוד המימדים מתאימים למכפלות, ניתן להכפיל בלוקים כמו מכפלה רגילה.

\end{proposition}
\begin{example}
$${\begin{bmatrix}A&B\\ C&D\end{bmatrix}}\begin{bmatrix}X\\ Y\end{bmatrix}={\begin{bmatrix}A X+B Y\\ C X+D Y\end{bmatrix}}$$

\end{example}
\begin{proposition}[שחלוף של מטריצת בלוקים]
ניתן לקבל את המטריצה המשוחלפת של מטריצת בלוקים על ידי שלחוף המטריצה ואז שיחלוף כל אחד מהבלוקים.

\end{proposition}
\begin{example}
$$\begin{array}{l l l}{{\left[A\quad B\quad C\right]^{T}=\left[\begin{array}{l}{{A^{T}}}\\ {{B^{T}}}\\ {{C^{T}}}\end{array}\right]}}\end{array}\qquad \begin{bmatrix}A & B \\C & D \end{bmatrix}^{T}=\begin{bmatrix}A^{T} & C^{T} \\B^{T} & D^{T}
\end{bmatrix}$$

\end{example}
\begin{definition}[מטריצת בלוקים אלכסונית]
מטריצה שמכילה בלוקים על האלכסון, וכל שאר הערכים אפס.

\end{definition}
\begin{example}
$$\left[\begin{matrix}1 & 0 & 0 & 0 & 0 & 0 & 0\\0 & 2 & 0 & 0 & 0 & 0 & 0\\0 & 0 & 1 & 2 & 3 & 0 & 0\\0 & 0 & 4 & 5 & 6 & 0 & 0\\0 & 0 & 0 & 0 & 0 & 1 & 2\\0 & 0 & 0 & 0 & 0 & 3 & 4\end{matrix}\right],\quad \quad  \left[\begin{matrix}1 & 1 & 0 & 0 & 0 & 0\\0 & -2 & 0 & 0 & 0 & 0\\0 & 0 & 1 & 2 & 0 & 0\\0 & 0 & 2 & 3 & 0 & 0\\0 & 0 & 0 & 0 & 1 & 1\\0 & 0 & 0 & 0 & 0 & 1\end{matrix}\right]$$

\end{example}
\begin{symbolize}
נסמן מטריצת בלוקים אלכסונית בסימון:
$$\mathrm{diag}\left( M_{1},M_{2},M_{3},\dots ,M_{n}\right)$$
כאשר \(M_{1},\dots,M_{n}\) הם מטריצות המייצגות בלוקים.

\end{symbolize}
\begin{proposition}[מכפלה של מטריצות בלוקים אלכסוניות]
המכפלה של שתי מטריצות בלוקים אלכסוניים עם בלוקים בגדלים מתאימים יהיה המכפלה של הבלוקים המתאימים. כלומר:
$$\begin{cases}M_{1}=\mathrm{diag}\left( A_{1},\dots,A_{n} \right) \\M_{2} =\mathrm{diag}\left( B_{1},\dots, B_{n}\right)
\end{cases}\implies M_{1}M_{2}=\mathrm{diag}\left( A_{1}B_{1},\dots A_{n}B_{n} \right)$$

\end{proposition}
\begin{corollary}[חזקה של מטריצת בלוקים אלסונית]
אם \(M\) מטריצת בלוקים אלכסונית אז \(M^{k}\) יהיה המטריצה האלכסונית שנוצרת מהעלאת כל בלוק בנפרד בחזקת \(k\). כלומר:
$$M=\mathrm{diag}\left( M_{1},\dots ,M_{n} \right)\implies M^{k}=\mathrm{diag}\left( M_{1}^{k},\dots, M_{n}^{k} \right)$$

\end{corollary}
\begin{definition}[מטריצה אלכסונית]
מטריצה שכל האיברים שלה הם על האלכסון.

\end{definition}
\begin{corollary}
מכפלה של מטריצות אלכסויות יהיה מכפלת האיברים על האלכסון בתור בלוקים מגודל \(1\times 1\).

\end{corollary}
\section{מרחבים שמקבלים ממטריצה}

\begin{definition}[מרחב השורות]
יהי \(A \in M_{n,m}\left( \mathbb{F}  \right)\). הפרוש של כל הווקטורי שורה יהיה מרחב השורות. זה יהיה תת מרחב של \(\mathbb{F} ^{m}\)

\end{definition}
\begin{definition}[מרחב העמודות]
יהי \(A \in M_{n,m}\left( \mathbb{F}  \right)\). הפרוש של כל ווקטורי העמודה יהיה מרחב העמודות. זה יהיה תת מרחב של \(\mathbb{F} ^{n}\).

\end{definition}
\begin{remark}
נשים לב כי אלו תתי מרחבים כיוון שהם נוצרים מפרוש של ווקטורים. 

\end{remark}
\begin{definition}[שחלוף של מטריצה]
יהי \(A \in M_{n,m}\left( \mathbb{F}  \right)\). השיחלוף של \(A\) תהיה \(A^{T}\in M_{m,n}\left( \mathbb{F}  \right)\) כך שמתקיים \((A^{T})_{ij}=A_{ji}\). כלומר הופכים את השורות לעמודות.

\end{definition}
\begin{proposition}
המרחב עמודות של \(A\) יהיה המרחב שורות של \(A^{T}\)

\end{proposition}
\begin{definition}[מרחב הפתרונות]
הי \(A \in M_{n,m}\left( \mathbb{F}  \right)\). אוסף כל הווקטורים \(v\) כך שמתקיים \(Av=0\). כיוון ש-\(v\) הוא ווקטור עמודה נקבל כי \(v\in \mathbb{F} ^{n}\).

\end{definition}
\begin{proposition}
נסמן ב-\(R\) את מרחב השורות וב-\(N\) את מרחב הפתרונות. אזי לכל \(v \in R\) ולכל \(u \in N\) נקבל:
$$v^{T}u=0$$

\end{proposition}
\begin{definition}[מרחב הפתרונות של השחלוף]
מרחב הפתרונות של \(A^{T}\). אוסף כל הווקטורים \(v \in V\) אשר מקיימים \(A^{T}v=0\).

\end{definition}
\Chapter{מערכות משוואות}

\section{צורה מדורגת מצומצת}

\begin{definition}[מרחב העמודות]
יהי \(A \in M_{n,m}\left( \mathbb{F}  \right)\) מטריצה עם עמודות \(a_{i}\in \mathbb{F}^{n}\). אזי מרחב העמודות יהיה:
$$U=\mathrm{Span}\left\{  a_{1},a_{2},\dots,a_{m}  \right\}$$
כאשר זהו תת מרחב של \(\mathbb{F} ^{n}\) כיוון שזהו פרוש של איברים במרחב ווקטורי.

\end{definition}
\begin{definition}[מרחב השורות]
יהי \(A \in M_{n,m}\left( \mathbb{F}  \right)\) מטריצה עם שורות \(a_{i}\in \mathbb{F}^{m}\). אזי מרחב השורות יהיה:
$$U=\mathrm{Span}\left\{  a_{1},a_{2},\dots,a_{m}  \right\}$$
כאשר זהו תת מרחב של \(\mathbb{F} ^{m}\) כיוון שזהו פרוש של איברים במרחב ווקטורי.

\end{definition}
\begin{definition}[פעולת שורה אלמנטרית]
פעולה שניתן לבצע על המטריצה אשר אינה משנה את מרחב העמודות. ניתן לחלק אותם לשלושה סוגים:

  \begin{enumerate}
    \item להחליף שתי שורות של המטריצה - זה פשוט לשנות את הסדר שלהם בפרוש, ולכן לא ישנה את המרחב המתקבל. נסמן את החלפת השורה ה-\(i\) עם השורה ה-\(j\) בסימון \(R_{i}\leftrightarrow R_{j}\). 


    \item להכפיל שורה של המטריצה בקבוע - לא ישפיע על הפרוש כי ניתן להכפיל כל ווקטור בפרוש בסקלר. נסמן את כפל השורה ה-\(i\) בקבוע \(c\) בסימון \(R_{i}\to cR_{i}\). 


    \item להוסיף כפולה של שורה אחת לשורה אחרת - לא ישפיע על הפרוש כי הפרוש כולל גם את החיסור של הכפולה. נסמן את הוספה של השורה ה-\(j\) כפול \(c\) לשורה ה-\(i\) בסימון \(R_{i}\to R_{i}+cR_{j}\). 


  \end{enumerate}
\end{definition}
\begin{proposition}
לכל פעולת שורה אלמטרית \(\varepsilon\) קיים פעולת שורה אלמטרית הופכית:

  \begin{enumerate}
    \item אם \(\varepsilon=R_{i}\leftrightarrow R_{j}\) אז הפעולה ההפוכה תהיה אותה הפעולה - \(\varepsilon ^{-1}=R_{i}\leftrightarrow R_{j}\). 


    \item אם \(\varepsilon=R_{i}\to cR_{j}\) אז הפעולה ההפוכה תהיה \(\varepsilon ^{-1}=R_{i}\to \frac{1}{c}R_{j}\). 


    \item אם \(\varepsilon=R_{i}\to R_{i}+cR_{j}\) אז הפעולה ההפוכה תהיה \(\varepsilon ^{-1}=R_{i}\to R_{i}+(-cR_{j})\). 


  \end{enumerate}
\end{proposition}
\begin{definition}[מטריצת שורה אלמטרית]
אלו המטריצות שהמכפלה בהם מימין שקול לביצוע פעולת שורה אלמנטרית. למעשה מקבלים אותם מביצוע הפעולות על מטריצת היחידה:

  \begin{enumerate}
    \item לכל \(1\leq i,j\leq n\) כך ש-\(i\neq j\) המטריצה \(P_{i,j}\) זה המטריצה היחידה שמחליפים את השורה ה-\(i\) וה-\(j\). 


    \item לכל \(1\leq i\leq n\) ו-\(a \in \mathbb{F} ^{\times}\) נגדיר \(M_{i}(a)\) המטריצת שמתקבלת ממטריצת היחידה לאחר הכפלת השורה ה-\(i\) ב-\(a\). 


    \item לכל \(1\leq i,j\leq n\) כך ש-\(i\neq j\) ו-\(a \in \mathbb{F}\) נגדיר את \(E_{i,j}(a)\) להיות המטריצה האלמנטרית המתקבלת ממטריצת היחידה ע"י הוספה של השורה ה-\(i\) כפול \(a\) לשורה ה-\(j\). 


  \end{enumerate}
\end{definition}
\begin{example}
$$P_{1,3}=\begin{bmatrix}0&0&1&0\\ 0&1&0&0\\ 1&0&0&0\\ 0&0&0&1\end{bmatrix}\qquad M_{4}(-2)=\begin{bmatrix}1&0&0&0\\ 0&1&0&0\\ 0&0&1&0\\ 0&0&0&-2\end{bmatrix}\qquad E_{2,4}(3)=\begin{bmatrix}1&0&0&0\\ 0&1&0&0\\ 0&0&1&0\\ 0&3&0&1\end{bmatrix}$$

\end{example}
\begin{corollary}
כמו הפעולות שורה האלמנטריות, כל המטריצות האלמטריות הפיכות כך שמתקיים:
$$P_{i,j}P_{i,j}=I,\quad M_{i}(a)M_{i}(a^{-1})=I,\quad E_{i,j}(a)E_{i,j}(-a)=I$$

\end{corollary}
\begin{remark}
ההכפלה מימין זה חשוב. אם נכפיל בשמאל נקבל כי המטריצה פועלת על העמודות, ולכן למעשה יהיה שקול לביצוע פעולת עמודה אלמנטרית. 

\end{remark}
\begin{definition}[מטריצות שקולות שורה]
מטריצות אשר ניתן להגיע מאחת לשנייה על ידי פעולות שורה אלמטריות נקראות שקולות שורה. כלומר מטריצות \(A,B\) יהיו שקולות שורה אם קיים סדרה של פעולות שורה אלמנטריות \(\varepsilon_{1},\dots,\varepsilon_{p}\) כך ש:
$$B=\varepsilon_{p}\left( \dots\varepsilon_{2}\left( \varepsilon_{1}(A) \right) \right)$$

\end{definition}
\begin{definition}[מטריצה מדורגת מצומצמת]
מטריצה אשר מקיימת את התנאים הבאים:

  \begin{enumerate}
    \item איבר מוביל(האיבר הראשון שלא אפס) של כל שורה(אם הוא קיים - זה בסדר אם יש שורה של אפסים) שווה ל-1. 


    \item כל המקדמים בעמודה של איבר מוביל כלשהו הם 0 פרט לאיבר מוביל עצמו - כלומר בכל עמודה תהיה רק איבר אחד ששווה ל-1(כמובן, שוב ייתכן כי כל העמודה אפסים). 


    \item צריכה להיות "בצורה של מדרגות יורדות" - אם בשורה כלשהי יש איבר מוביל, אז גם בשורה שמעליה יש איבר מוכיל והוא נמצא בצד שמאל מהאיבר המוביל של השורה המקורית. 


  \end{enumerate}
\end{definition}
\begin{remark}
מתנאי 3 נובע כי אם בשורה כלשהי אין איבר מוביל, אז גם בכל השורות שמתחתיה אין איבר מוביל.

\end{remark}
\begin{definition}[איבר חופשי וקשור]
נניח כי המטריצה המורחבת של מערכת משוואות כלשהי היא מדורגת מצומצמת.

  \begin{enumerate}
    \item נעלם של המערכת בעמודה המתאימה לעמודה שיש בה איבר מוביל יקרא קשור - כלומר נעלם המתאים לעמודה שיש בה 1. 


    \item נעלם של המערכת בעמודה שבה אין איבר מוביל יקרא חופשי. 


  \end{enumerate}
\end{definition}
\begin{proposition}
ניתן להעביר כל מטריצה לצורה מדורגת מצומצת

\end{proposition}
\begin{example}
נסתכל על המטריצה:
$$A=\begin{pmatrix} 0 & 2 & 4 & -2 \\3 & 2 & -2 & -5 \\1 & -1 & -4 & 1\end{pmatrix}$$
וננסה צורה שקולת שורה אשר מדורגת מצומצמת:
\begin{gather*}\begin{pmatrix}0 & 2 & 4 & -2 \\3 & 2 & -2 & -5 \\1 & -1 & -4 & 1\end{pmatrix}\xrightarrow{R_{1}\to \frac{1}{2}R_{2}}\begin{pmatrix}0 & 1 & 2 & -1 \\3 & 2 & -2 & -5 \\1 & -1 & -4 & 1\end{pmatrix}\xrightarrow{R_{3}\to  R_{3}+R_{1}}\begin{pmatrix}0 & 1 & 2 & -1 \\3 & 2 & -2 & -5 \\1 & 0 & -2 & 0\end{pmatrix}\xrightarrow{R_{2}\to R_{2}-2R_{1}} \\\begin{pmatrix}0 & 1 & 2 & -1 \\3 & 0 & -6 & -3 \\1 & 0 & -2 & 0\end{pmatrix}\xrightarrow{R_{2}\to \frac{1}{3}R_{2}}\begin{pmatrix}0 & 1 & 2 & -1 \\1 & 0 & -2 & -1 \\1 & 0 & -2 & 0\end{pmatrix}\xrightarrow{R_{3}\to R_{3}-R_{2}}\begin{pmatrix}0 & 1 & 2 & -1 \\1 & 0 & -2 & -1 \\0 & 0 & 0 & 1\end{pmatrix}\xrightarrow{R_{1}\to R_{1}+R_{3}} \\\begin{pmatrix}0 & 1 & 2 & 0 \\1 & 0 & -2 & -1 \\0 & 0 & 0 & 1\end{pmatrix}\xrightarrow{R_{2}\to R_{2}+R_{3}}\begin{pmatrix}0 & 1 & 2 & 0 \\1 & 0 & -2 & 0 \\0 & 0 & 0 & 1\end{pmatrix}\xrightarrow{R_{1}\leftrightarrow  R_{2}}\begin{pmatrix}1 & 0 & -2 & 0 \\0 & 1 & 2 & 0 \\0 & 0 & 0 & 1\end{pmatrix}
\end{gather*}
והגענו לצורה מדורגת מצומצת. 

\end{example}
\begin{proposition}
מטריצה ריבועית היא הפיכה אם"ם היא מכפלה של מטריצות אלמטריות

\end{proposition}
\begin{proof}
  \begin{enumerate}
    \item עבור הכיוון הראשון אנו יודעים כי מכפלה של מטריצות הפיכות היא הפיכה, ולכן כיוון שכל מטריצה אלמנטרית היא הפיכה נקבל כי אם מטריצה היא מכפלה של מטריצות אלמנטריות אז היא הפיכה. 


    \item בכיוון השני נניח כי \(A\) הפיכה. לכן היא שקולת שורה למטריצת היחידה, ולכן קיים מכפלה של מטריצות אלמטריות אשר יהיה שווה למטריצה. 


  \end{enumerate}
\end{proof}
\section{מציאת הצגה פרמטרית למערכת משוואות}

\begin{definition}[פתרון של משוואה לינארית]
נקרא ל-\(n\)-יה \(\begin{pmatrix} d_{1}\\\vdots\\d_{n}\end{pmatrix}\) פתרון של המשוואה לינארית 
$$a_{1}x_{1}+a_{2}x_{2}+\dots+a_{n}x_{n}=b$$
אם מתקיים:
$$a_{1}d_{1}+a_{2}d_{2}+\dots+a_{n}d_{n}=b$$

\end{definition}
\begin{definition}[הצגה פרמטרית של פתרון של משוואה לינארית]
ניתן לכתוב את אוסף כל הפתרונות של משוואה לינארית בצורה פרמטרית. כלומר לכתוב את הפתרון הכללי בתור \(n\)-יה עם פרמטר חופשי. 

\end{definition}
\begin{example}
למשוואה:
$$x_{1}-x_{2}=1$$
נקבל כי ה-\(n\)-יות \(\begin{pmatrix}1 \\0\end{pmatrix}\) ו-\(\begin{pmatrix}0 \\ 1\end{pmatrix}\) הם פתרונות. אם נרצה לרשום את הצורה הכללית של הפתרון:
$$\left\{  \begin{pmatrix}x_{1}\\x_{2}\end{pmatrix}  \in \mathbb{R}\mid x_{1}-x_{2}=1\right\}=\left\{  \begin{pmatrix}1+x_{2}\\x_{2}\end{pmatrix}\mid x_{2} \in \mathbb{R}  \right\}=\left\{  \begin{pmatrix}1\\0\end{pmatrix}+t\begin{pmatrix}1\\1
\end{pmatrix}\mid t \in \mathbb{R}  \right\}$$
זוהי ההצגה הפרמטרית של הפתרון. לכל \(t\) שנבחר קיים פתרון למשוואה, ולכן פתרון למשוואה קיים \(t\) כלשהו. 

\end{example}
\begin{remark}
לקבוצה הריקה אין הצגה פרמטרית, לכן למשל למשוואה \(0x_{1}+0x_{2}+\dots+0x_{n}=1\) אין פתרונות ואין הצגה פרמטרית.

\end{remark}
\begin{definition}[מערכת משוואות לינאריות]
מערכת של משוואות עם מספר משתנים, כאשר כל גורם יהיה עם דרגה של לכל היותר 1. זה אומר כי לא ייתכן גורם מהצורה \(x_{1}x_{2}\) כיוון שזה יהיה גורם מסדר 2. מערכת של \(m\) משוואות לינאריות עם \(n\) משתנים תהיה מהצורה:
$$\begin{cases}a_{11}x_{1}+a_{12}x_{2}+\dots+a_{1n}x_{n}=b_{1} \\a_{21}x_{1}+a_{22}x_{2}+\dots+a_{2n}x_{n}=b_{2} \\\qquad \qquad \qquad \quad \;\;\vdots \\a_{m 1}x_{1}+a_{m2}x_{2}+\dots+a_{mn}x_{n}=b_{m}
\end{cases}$$
כאשר הסוגרים המסולסלות מסמנות "וגם"

\end{definition}
\begin{example}
נסתכל על המערכת משוואות:
$$\begin{cases}a_{1}x+a_{2}y=a_{3}  \\b_{1}x+b_{2}y=b_{3}
\end{cases}\qquad a_{1},a_{2},b_{1},b_{2}\in \mathbb{F}  ^{\times}=\mathbb{F}  \setminus \{ 0 \}$$
ניתן כעת לנסות למצוא את הפתרון בעזרת השיטה הבאה. ניקח את המשוואה השנייה, נכפיל ב-\(\frac{a_{1}}{b_{1}}\) ונחסר מהראשונה:
$$\begin{cases}a_{1}x+a_{2}y=a_{3} \\a_{1}x+\frac{b_{2}}{b_{1}}a_{1}y=\frac{b_{3}}{b_{1}}a_{1} \end{cases}\implies \begin{cases}a_{1}x+a_{2}y=a_{3} \\0x +\left( \frac{b_{2}}{b_{1}}a_{1}-a_{2} \right)y=\frac{b_{3}}{b_{1}}a_{1}-a_{3}
\end{cases}$$
וכעת אם נניח כי \(c\equiv\frac{b_{2}}{b_{1}}a_{1}-a_{2}\neq 0\) אז ניתן להכפיל את המשוואה השנייה ב-\(\frac{a_{2}}{c}\) ואז להחסיר את המשוואה השנייה מהראשונה:
$$\begin{cases}a_{1}x+a_{2}y=a_{3} \\0x+a_{2}y=\left( \frac{b_{3}}{b_{1}}a_{1}-a_{3} \right) \frac{a_{2}}{c}\end{cases}\implies \begin{cases}a_{1}x+0y=a_{3}-\left( \frac{b_{3}}{b_{1}}a_{1}-a_{3} \right) \frac{a_{2}}{c} \\0x+a_{2}y=\left( \frac{b_{3}}{b_{1}}a_{1}-a_{3} \right) \frac{a_{2}}{c}
\end{cases}$$
ומכאן הפתרון הוא די מיידי. זוהי דוגמא פשוטה לאלגוריתם הנקרא דירוג ג'ורדן.

\end{example}
\begin{definition}[פתרון של מערכת משוואות לינארית]
נקרא ל-\(n\)-יה פתרון של מערכת משוואות לינארית אם היא פתרון של כל אחת מהמשוואות. כלומר זה יהיה החיתוך של קבוצת הפתרון של כל אחת מהמשוואות של המערכת בנפרד.

\end{definition}
עבור מערכת משוואות מהצורה:
$$\begin{cases}a_{11}x_{1}+a_{12}x_{2}+\dots+a_{1n}x_{n}=b_{1} \\a_{21}x_{1}+a_{22}x_{2}+\dots+a_{2n}x_{n}=b_{2} \\\qquad \qquad \qquad \quad \;\;\vdots \\a_{m 1}x_{1}+a_{m2}x_{2}+\dots+a_{mn}x_{n}=b_{m}
\end{cases}$$
ניתן לכתוב את המשוואה בצורה הבאה:
$$\underbrace{ \begin{pmatrix}a_{11} & a_{12} & \dots & a_{1n} \\a_{21} & a_{22} & \dots  & a_{2n} \\\vdots & \vdots & \ddots & \vdots \\a_{m 1} & a_{m 2} & \dots & a_{mn}\end{pmatrix} }_{ A }\underbrace{ \begin{pmatrix}x_{1} \\ x_{2}\\\vdots \\ x_{n}\end{pmatrix} }_{ \mathbf{x} }=\underbrace{ \begin{pmatrix}b_{1} \\ b_{2} \\ \vdots \\ b_{n}
\end{pmatrix} }_{ \mathbf{b} }$$

\begin{definition}[מטריצה המייצגת של מערכת משוואות]
מתייחסים מטריצה של המקדמים \(A\). 

\end{definition}
\begin{proposition}[מטריצה מורחבת של מערכת משוואות]
עבור מערכת משוואות מהצורה שראינו, נגדיר את המטריצה המורחבת להיות:

$$\left(\begin{array}{cccc|c}a_{11} & a_{12} & \dots & a_{1n} &b_{1} \\a_{21} & a_{22} & \dots  & a_{2n}  &b_{2}\\\vdots & \vdots & \ddots & \vdots & \vdots\\a_{m 1} & a_{m 2} & \dots & a_{mn} & b_{m}\end{array}\right)
$$
אם \(A\) הצורה המצומצמת נסמן לעיתים צורה זו ב-\(A|b\).

\end{proposition}
באופן כללי כדי למצוא הצגה פרמטרית למערכת משוואות יהיה נוח לעבור לצורה המדורגת מצומצמת. נראה זאת בעזרת דוגמא:

\begin{example}
נסתכל על המערכת משוואות המטריצה ההבאה המתאימה לה:
$$\begin{cases}x_{2}-2x_{3}-x_{5}=3 \\x_{4}+2x_{5}=-1 \\0=0\end{cases}\implies\left(\begin{array}{ccccc|c}0&\boxed{1}&-2&0&-1 &3\\0&0&0&\boxed{1}&2 &-1\\0&0&0&0&0&0
\end{array}\right)$$
נשים לב כי העבודה השנייה והרביעית הם עם איברים מובלים(מסומנים במלבן) לכן \(x_{2},x_{4}\) הם משתנים קשורים. לעומת זאת המשתנים \(x_{1},x_3,x_{5}\) הם משתנים חופשיים.

\end{example}
\begin{example}
נסתכל על המערכת משוואות המטריצה ההבאה המתאימה לה:
$$\begin{cases}x_{2}-2x_{3}-x_{5}=3 \\x_{4}+2x_{5}=-1 \\0=0\end{cases}\implies\left(\begin{array}{ccccc|c}0&\boxed{1}&-2&0&-1 &3\\0&0&0&\boxed{1}&2 &-1\\0&0&0&0&0&0
\end{array}\right)$$
נשים לב כי העבודה השנייה והרביעית הם עם איברים מובלים(מסומנים במלבן) לכן \(x_{2},x_{4}\) הם משתנים קשורים. לעומת זאת המשתנים \(x_{1},x_3,x_{5}\) הם משתנים חופשיים. נשים לב כי מתקיים:
$$\begin{cases}x_{2}=3+2x_{3}+x_{5} \\x_{4}=-1-2x_{5}
\end{cases}$$
כעת קבוצת הפתרונות יהיה:
\begin{gather*}\left.\left\{  \begin{pmatrix}x_{1} \\3+2x_{3}+x_{5} \\x_{3} \\-1-2x_{5} \\x_{5}\end{pmatrix} \in \mathbb{R}^{5} \;\right\rvert\;   x_{1},x_{3},x_{5} \in \mathbb{R}   \right\}= \\=\left.\left\{  \begin{pmatrix}0 \\3 \\0 \\-1 \\0\end{pmatrix} +\begin{pmatrix}1\\0\\0\\0\\0\end{pmatrix}x_{1}+\begin{pmatrix}0\\2\\1\\0\\0\end{pmatrix}x_{3}+\begin{pmatrix}0\\1\\0\\-2\\1 \end{pmatrix} x_{5}\;\right\rvert\;x_{1},x_{3},x_{5} \in \mathbb{R}\right\} 
\end{gather*}

\end{example}
עכשיו שראינו כמה נוח הצורה הזאת כדי למצוא את כל הפתרונות של מערכות משוואות, נראה איך מגיעים עליה.

\begin{proposition}
כמות המשתנים החופשיים זה יהיה כמות הפרמטריים הנדרשים כדי לייצג את קבוצת הפתרונות. זאת כיוון שברגע שבוחרים את המשתנים החופשיים, המשתנים הקשורים נקבעיים לפיהם.

\end{proposition}
\section{שימוש במטריצות הופכיות}

\begin{definition}[מערכת עם עודף משוואות]
באנגלית נקרא Over-determined system. זוהי מערכת עם יותר משוואות מאשר משתנים.

\end{definition}
\begin{definition}[מערכת עם חוסר משוואות]
באנגלית נקרא under-determined system. זוהי מערכת משוואות עם יותר משתנים מאשר משוואות.

\end{definition}
\begin{definition}[מטריצה רחבה]
מטריצה עם יותר עמודות משורות. כלומר \(A \in M_{n \times m}\left( \mathbb{F}  \right)\) כאשר \(m>n\).

\end{definition}
\begin{definition}[מטריצה גבוהה]
מטריצה עם יותר שורות מעמודות. כלומר \(A \in M_{n \times m}\left( \mathbb{F}  \right)\) כאשר \(m<n\).

\end{definition}
כאשר נשים לב כי:

\begin{enumerate}
  \item מערכת עם עודף משוואות היא מיוצגת על ידי מטריצה גבוהה. 


  \item מערכת עם חוסר משוואות מיוצגת על ידי מטריצה רחבה. 


\end{enumerate}
\subsection{הופכי שמאלי}

\begin{definition}[הופכי שמאלי]
יהי \(A \in M_{n,m}\left( \mathbb{F}  \right)\) מטריצה. מטריצה \(X \in M_{l,n}\left( \mathbb{F}  \right)\) נקראת מטריצה הופכית שמאלית אם מקיימת:
$$XA=Id$$

\end{definition}
\begin{example}
עבור
$$\mathbf{a}={\left[\begin{array}{c}{2}\\ {0}\\ {-1}\\ {1}\end{array}\right]}$$
נקבל כי:
$$\begin{array}{l l l l l}{{\left[1/2\quad0\quad0\quad0\right]}}&{{\left[0\quad0\quad-1\quad0\right]}}&{{\left[0\quad0\quad0\quad1\right]}}\end{array}$$
ולמשל עבור:
$$A={\left[\begin{array}{l l}{4}&{3}\\ {-6}&{-4}\\ {-1}&{-1}\end{array}\right]}$$
נקבל כי למשל המטריצות הבאות יהיו הופכיות שמאליות:
$$B={\frac{1}{9}}\begin{bmatrix}-7&-8&11\\ 11&10&-16\end{bmatrix}\quad\quad C={\frac{1}{2}}\begin{bmatrix}0&-1&4\\ 0&1&-6\end{bmatrix}$$

\end{example}
\begin{proposition}
אם ל-\(A\) יש הופכי שמאלי, אז העמודות של \(A\) הם בלתי תלויות לינארית

\end{proposition}
\begin{proof}
נניח כי ל-\(A \in M_{m,n}\left( \mathbb{F}  \right)\) יש הופכי שמאלי \(B\). ויהי \(a_{1},\dots,a_{n}\in F^{m}\) העמודות של \(A\). אזי:
$$x_{1}\mathbf{a}_{1}+\cdots+x_{n}\mathbf{a}_{n}=0$$
עבור \(x_{1},\dots,x_{n}\in \mathbb{F}\) כלשהם. כעת נסמן \(x=\left( x_{1},\dots,x_{n} \right)\) ונקבל:
$$x=Ix=BAx=B 0 = 0\implies x_{1}=x_{2}=\dots=x_{n}=0$$
ונקבל כי בלתי תלוי לינארית.

\end{proof}
\begin{corollary}
אם ל-\(A\) יש הופכי שמאלי, אזי \(A\) היא או מטריצה ריבועית, או מטריצה "גבוהה".

\end{corollary}
\begin{proof}
נניח בשלילה כי \(A\) מטריצה רחבה. כלומר \(A \in M_{m,n}\left( \mathbb{F}  \right)\) כך ש-\(m<n\). כעת יש לו \(n\) עמודות, שכל אחד מהם הוא ווקטור ב-\(\mathbb{F} ^{m}\). כיוון שמימד המרחב קטן מאוסף הווקטורים נקבל סתירה לזה שבלתי תלוי לינארית.

\end{proof}
\begin{corollary}
כלומר נקבל כי ההופכי השמאלי אם קיים זה מתי שיש לנו עודף משוואות. 

\end{corollary}
\begin{proposition}
עבור מערכת משוואות \(Ax=b\) כאשר ל-\(A\) יש הופכי שמאלי \(C\), נקבל כי:

  \begin{enumerate}
    \item אם \(ACb=b\) אזי \(x=Cb\) הוא הפתרון היחיד. 


    \item אם \(ACb\neq b\) אזי לא קיים פתרון כלל. 


  \end{enumerate}
\end{proposition}
\begin{proof}
אם קיים \(x\) אשר מקיים את המשוואה אז נקבל:
$$Ax=b\implies CAx=Cb\implies x=Cb$$
כאשר אם לא קיים זה כמובן לא יהיה פתרון.

\end{proof}
\begin{proposition}
להעתקה לינארית \(T_{A}=Ax\) יש הופכי שיאמלי אם"ם \(T_{A}\) היא חח"ע.

\end{proposition}
\subsection{הופכי ימיני}

\begin{definition}[הופכי ימיני]
יהי \(A \in M_{n,m}\left( \mathbb{F}  \right)\) מטריצה. מטריצה \(X \in M_{m,l}\left( \mathbb{F}  \right)\) נקראת מטריצה הופכית ימינית אם מקיימת:
$$AX=Id$$

\end{definition}
\begin{definition}[מטריצה משוחלפת]
יהי \(A\in M_{n,m}\left( \mathbb{F}  \right)\) מטריצה. אזי המטריצה \(A^{t}\in M_{m,n}\left( \mathbb{F}  \right)\) תהיה המטריצה שעבורה השורות והעמודות מתהפכות. כלומר:
$$(A^{t})_{ij}=A_{ji}$$

\end{definition}
\begin{example}
$$\begin{bmatrix}\pi&i&-1\\ 5&7&3/2\end{bmatrix}^{T}=\begin{bmatrix}\pi&5\\ i&7\\ -1&3/2\end{bmatrix}$$

\end{example}
\begin{proposition}
$$(AB)^{t}=B^{t}A^{t}$$

\end{proposition}
\begin{proposition}
אם \(B\) הוא הופכי ימיני של \(A\) אז \(B^{t}\) הוא הופכי שמאלי של \(A^{t}\). ובאופן מקבל אם \(B\) הוא הופכי שמאלי של \(A\) אז \(B^{t}\) הוא הופכי ימיני של \(A^{t}\).

\end{proposition}
\begin{proof}
נניח כי \(B\) הופכי ימיני של \(A\). לכן:
$$B^{T}A^{T}=(A B)^{T}=Id$$
ואם \(B\) הוא הופכי שמאלי של \(A\) מתקיים באופן זהה:
$$A^{T}B^{T}=(BA)^{T}=Id$$

\end{proof}
כעת ניתן להמיר את התוצאות הקודמות שלנו על הופכי שמאלי לגבי הופכי ימיני:

\begin{proposition}
  \begin{enumerate}
    \item אם ל-\(A\) יש מטריצה הופכית ימינית היא תהיה ריבועית או רחבה. 


    \item אם למטריצה יש הופכית ימינית אז השורות שלה יהיו בלתי תלויות לינארית. 


    \item אם \(T_{A}=Ax\) אופרטור לינארי אז קיים הופכי ימיני אם"ם \(T_{A}\) הוא על. 


  \end{enumerate}
\end{proposition}
\subsection{הופכי דו צדדי}

\begin{definition}[מטריצה הפוכית]
יהי \(A\) מטריצה. \(A^{-1}\) יקרא המטריצה ההופכית אם מתקיים:
$$AA^{-1}=Id=A^{-1}A$$
כאשר אם קיים הופכי אז המטריצה \(A\) נקראת הפיכה.

\end{definition}
\begin{corollary}
מטריצה היא הפיכה אם"ם קיים הופכי ימיני וגם קיים הופכי שמאלי.

\end{corollary}
\begin{proposition}
קיימת מטריצה הופכית יחידה.

\end{proposition}
\begin{proposition}
אם מטריצה היא הפיכה, אז היא ריבועית.

\end{proposition}
\begin{proposition}
יהי \(A \in M_{n\times n}\left( \mathbb{F}  \right)\) מטריצה ריבועית. אזי הביטויים הבאים שקולים:

  \begin{enumerate}
    \item המטריצה \(A\) הפיכה. 


    \item למטריצה \(A\) קיים הופכי ימיני. 


    \item למטריצה \(A\) קיים הופכי שמאלי. 


    \item השורות של \(A\) הם בלתי תלויות לינאריות. 


    \item העמודות של \(A\) הם בלתי תליות לינארית. 


  \end{enumerate}
\end{proposition}
\section{פסודו הופכי}

\begin{definition}[מטריצת גראם]
עבור \(A\in M_{n,m}\left( \mathbb{R} \right)\) נגדיר את המטריצת גראם להיות המטריצה הבאה:
$$A^{T}A \in M_{n,n}\left( \mathbb{R} \right)$$

\end{definition}
\begin{proposition}
למטריצה \(A\) יש עומדות בלתי תלויות לינארית אם"ם המטריצת גראם \(A^{T}A\) הפיכה, ובמקרה זה קיים הופכי שמאלי.

\end{proposition}
\begin{proposition}
למטריצה \(A\) יש שורות בלתי תלויות לינארית אם"ם המטריצה \(AA^{T}\) הפיכה. ובמקרה זה קיים הופכי שמאלי.

\end{proposition}
\section{פירוק למכפלה של מטריצות משולשיות}

\section{פירוק LU}

אנו יודעים כי כל משוואה אפשר להביא לצורה מדורגת מצומצמת. באיזה מצבים 

\Chapter{אופרטורים לינאריים ודטרמיננטה}

\section{אופרטורים לינארים ומעברי בסיס}

\begin{definition}[אופרטור]
פונקציה שהתחום שלה שווה לטווח. כלומר מחזירה את אותו אובייקט אשר לקחה.

\end{definition}
\begin{example}
  \begin{itemize}
    \item אופרטור הנגזרת \(\frac{d}{dx}\) לוקח פונקציה ומחזיר פונקציה.
    \item אופרטור האינטגרל באותו דרך לוקח פונקציה ומחזיר פונקציה
    \item פונקציה \(\Omega(f)=f+1\) כלומר מקבלת פונקציה ומחזירה את הפונקציה אשר מבצעת את הפונקציה המקורית ומוספיה 1.
  \end{itemize}
\end{example}
\begin{remark}
אופרטורים זה נפוץ במיוחד כיוון שהרבה פעמים אנחנו לוקחים אופבייקט במרחב, ועושים עליו פעולות אך לא מעבירים אותו למרחב אחר, ולכן הפונקציה אשר מבצעת את הפעולות תהיה אופרטור.

\end{remark}
\begin{definition}[אופרטור לינארי]
אופרטור אשר לינארי. כלומר \(T:V\to V\) אשר מקיים \(T(v+cu)=T(v)+cT(u)\).

\end{definition}
\begin{proposition}
במקרה הנוצר סופית, אופרטורים לינארים מיוצגים על ידי מטריצה ריבועית, ומטריצות ריבועיות יוצרות אופרטורים לינאריים.

\end{proposition}
\begin{definition}[מטריצות דומות]
שתי מטריצות ריבועיות אשר מייצגות את אותה העתקה לינארית בבסיסים שונים נקראות מטריצות דומות.

\end{definition}
\begin{symbolize}
עכשיו כשנתייחס לאופרטורים, נסמן את המטריצה המייצגת של אופרטור אשר מקבל ווקטור בבסיס \(\mathcal{A}\) מחזיר ווקטור בבסיס \(\mathcal{A}\) ע"י \([T]_{\mathcal{A}}\) במקום \([T]_{\mathcal{A}}^{\mathcal{A}}\).

\end{symbolize}
\begin{example}
נסתכל על העתקה
$$T\begin{pmatrix}x\\ y\end{pmatrix}=\begin{pmatrix}x\\ x+y
\end{pmatrix}$$
נשים לב כי מצד אחת בבסיס הסטנדרטי \(\mathcal{A}=(e_{1},e_{2})\) ניתן לייצג את המטריצה המייצגת בצורה הבאה:
$$[T]_{\mathcal{A} }=\begin{bmatrix}[T(e_{1})]_{\mathcal{A}}  \bigg|[T(e_{2})]_{\mathcal{A} } \end{bmatrix}=\begin{bmatrix}1  &  0 \\1 & 1
\end{bmatrix}$$
אך מצד שני עבור הבסיס \(\mathcal{B}=(e_{1}+e_{2},e_{1})\) נקבל:
$$[T]_{\mathcal{B} }=\begin{bmatrix}[T(e_{1}+e_{2})]_{\mathcal{B}}  \bigg|[T(e_{2})]_{\mathcal{B} } \end{bmatrix}=\begin{bmatrix}1  &  1 \\2 & 1
\end{bmatrix}$$
ולכן המטריצות \([T]_{\mathcal{A}}=[T]_{\mathcal{B}}\) יהיו דומות

\end{example}
\begin{definition}[מטריצת מעבר בסיס]
מטריצה שמעבירה ווקטור מבסיס \(\mathcal{A}=\left( a_{1},\dots,a_{n} \right)\) לבסיס \(\mathcal{B}=\left( b_{1},\dots,b_{n} \right)\). זה למעשה יהיה המטריצה המייצגת של אופרטור היחידה אשר מקבל ווקטור בבסיס \(\mathcal{A}\) ומחזיר ווקטור מבסיס \(\mathcal{B}\). כלומר המטריצה:
$$[Id]_{\mathcal{B} }^{\mathcal{A} }=\begin{bmatrix}[Id(a_{1})]_{\mathcal{B} } \bigg| \dots \bigg| [Id(a_{n})]_{\mathcal{B} }\end{bmatrix}=\begin{bmatrix}[a_{1}]_{\mathcal{B} } \bigg| \dots \bigg| [a_{n}]_{\mathcal{B} }
\end{bmatrix}$$

\end{definition}
\begin{proposition}[ההופכי של מטריצת המעבר בסיס]
המטריצה ההופכית למטריצה שמעבירה ווקטור מבסיס \(\mathcal{A}=\left( a_{1},\dots,a_{n} \right)\) לבסיס \(\mathcal{B}=\left( b_{1},\dots,b_{n} \right)\) תקיים:
$$\left( [Id]_{\mathcal{B} }^{\mathcal{A} } \right)^{-1}=[Id]_{\mathcal{A} }^{\mathcal{B} }=\begin{bmatrix}[b_{1}]_{\mathcal{A} } \bigg| \dots \bigg| [b_{n}]_{\mathcal{A} }
\end{bmatrix}$$

\end{proposition}
\begin{proof}
מתקיים:
$$[Id]_{\mathcal{B} }^{\mathcal{A} }[Id]_{\mathcal{A} }^{\mathcal{B} }=\left[ Id\circ  Id \right]_{\mathcal{B
} }^{\mathcal{B} }=[Id]_{\mathcal{B} }$$
וזה מספיק כיוון שזוהי מטריצה ריבועית(אחרת היה צריך להראות ששווה ליחידה גם בכפל בכיוון השני) ולכן הופכי.

\end{proof}
\begin{proposition}
מטריצות \(A,B\) יהיו דומות אם"ם קיימת מטריצה הפיכה \(P\) כך ש:
$$A=P ^{-1}BP$$
כאשר המטריצה \(P\) תהיה המטריצת מעבר בסיס

\end{proposition}
\begin{proof}
  \begin{enumerate}
    \item נניח ראשית כי \(A,B\) מטריצות דומות. לכן קיימות העתקה לינארית \(T\) ובסיסים \(\mathcal{A},\mathcal{B}\) כך ש: 
$$A=[T]_{\mathcal{A} }\qquad B=[T]_{\mathcal{B} }$$


    \item אנו יודעים כי מתקיים: 
$$[T]_{\mathcal{A} }=[T]_{\mathcal{A} }^{\mathcal{A} }=[Id]_{\mathcal{A} }^{\mathcal{B} }[T]_{\mathcal{B} }^{\mathcal{B} } [Id]_{\mathcal{B} }^{\mathcal{A} }$$


    \item נסמן \(P=[Id]_{\mathcal{B}}^{\mathcal{A}}\) ולכן \(P ^{-1}=[Id]_{\mathcal{A}}^{\mathcal{B}}\) ולכן: 
$$[T]_{\mathcal{A} }=P ^{-1} [T]_{\mathcal{B} }P\implies A=P ^{-1} BP$$


    \item בכיוון השני נניח כי קיימת \(P\) הפיכה כך ש-\(A=P ^{-1} B P\). נרצה להראות כי מייצגים את אותה העתקה \(T\) תחת בסיסים שונים. נגדיר העתקה לינארית \(T(v)=Av\). מתקיים: 
$$[Tv]_{\mathcal{A} }=[T]_{\mathcal{A} }^{A}[v]_{\mathcal{A} }=P ^{-1}BP [v]_{A}$$


    \item נזכור כי המטריצת \(P\) היא המטריצת מעבר בסיס. ולכן \(P[v]_{A}=[v]_{B}\). ולכן נקבל: 
$$A[v]_{\mathcal{A}}=P^{-1}B[v]_{\mathcal{B}}=P^{-1}[T(v)]_{\mathcal{B}}$$


  \end{enumerate}
\end{proof}
\begin{proposition}
מטריצות דומות משמרות דרגה

\end{proposition}
\begin{proof}
$$\operatorname{rank}B=\operatorname{rank}(P^{-1}A P)=\operatorname{rank}(A P)=\operatorname{rank}A$$

\end{proof}
\begin{proposition}[חזקה של מטריצות דומות]
אם \(A\) דומה ל-\(B\) עם מטריצת מעבר \(P\) מתקיים \(A^{n}=P ^{-1} B^{n}P\)

\end{proposition}
\begin{proof}
$$A^{k}=(P ^{-1} B P)^{k}=\overbrace{ (P ^{-1}B P) (P ^{-1} B P)\dots (P ^{-1} B P) }^{ k\;\mathrm{ times} }= P ^{-1} B (P ^{-1} P) B (P ^{-1} P)\dots P=P ^{-1} B^{ k} P$$

\end{proof}
\begin{remark}
אם \(A=PBP ^{-1}\) אז עדיין דומות כי \(A=(P ^{-1})^{-1} B P ^{-1}\). 

\end{remark}
\begin{definition}[עקבה]
סכום האיברים על האלכסון של המטריצה. כלומר:
$$\mathrm{Tr}(A)=\sum_{i=1}^{n} A_{i,i}$$

\end{definition}
\begin{proposition}[תכונות של עקבה]
  \begin{enumerate}
    \item \(\mathrm{tr}(A+B)=\mathrm{tr}A+\mathrm{tr}B\)


    \item \(\mathrm{tr}(k A)=k\cdot \mathrm{tr}(A)\)


    \item \(\mathrm{tr}(A^{T})=\mathrm{tr}(A)\)


    \item \(\mathrm{tr}(A B)=\mathrm{tr}(B A)\)


  \end{enumerate}
\end{proposition}
\begin{proposition}
מטריצות דומות משמרות עקבה. לכן עקבה של אופרטור מוגדר היטב.

\end{proposition}
\begin{proof}
אם \(A\) דומה ל-\(B\) אז קיים \(P\) הפיכה כך שמתקיים \(B=P ^{-1}A P\) וכעת:
$$\mathrm{tr}B=\mathrm{tr}(P^{-1}A P)=\mathrm{tr}((A P)P^{-1})=\mathrm{tr}A$$

\end{proof}
\section{אופרטורים ומטריצות מיוחדות}

\begin{definition}[אופרטור הטלה]
אם \(V=W\oplus U\) קיים אופרטור \(P_{W}:V\to V\) כך שלכל \(v\in V\) מתקיים \(P_{W}v \in W\). העתקה זו למעשה לוקח את הרכיב ה-\(W\) של הווקטור.

\end{definition}
\begin{proposition}
אופרטור ההטלה מקיים \(P^{2}=P\).

\end{proposition}
\section{פונקציות נפח}

ננסה לשוב כעת על מטריצה בתור אוסף של \(n\) ווקטורים. נרצה למצוא פונקציה אשר מחזירה את הנפח ה-\(n\) מימדי של המטריצה.

\begin{definition}[פונקציונאל לינארי]
העתקה לינארית \(T:V\to\mathbb{F}\).

\end{definition}
נסתכל כרגע על פונקציונלים לינארים מהמרחב \(V=M_{n\times n}\).

\begin{symbolize}
הסימון \(L_{k}^{M}\) יסמן את השורה ה-\(k\) של מטריצה \(M\).

\end{symbolize}
\begin{definition}[לינאריות לפי שורה]
פונקציה \(D:M_{n\times n}\left( \mathbb{F}  \right)\to \mathbb{F}\) נקראת לינארית לפי השורה ה-\(i\) כאשר:

  \begin{enumerate}
    \item אם \(A,B,C \in M_{n\times n}\left( \mathbb{F}  \right)\) כך שלכל \(1\leq j\leq n\) כך ש-\(i\neq j\) המקיימים: 
$$L_{j}^{A}=L_{j}^{B}=L_{j}^{C}\implies L_{i}^{A}+L_{i}^{B}=L_{i}^{C}$$
אז תקיים:
$$D(C)=D(A)+D(B)$$
  \end{enumerate}
\end{definition}
\begin{example}
עבור \(n=2\) הפונקציונאל המוגדר על ידי:
$$D\begin{pmatrix}a_{11} & a_{12} \\a_{21} & a_{22}
\end{pmatrix}=a_{11}$$
תהיה לינארית לפי השורה הראשונה אך לא לינארית לפי השורה השנייה.

\end{example}
\begin{definition}[מולטילינאריות]
פונקציה \(D:M_{n\times n}\left( \mathbb{F}  \right)\) אשר לינארית לפי השרוה ה-\(i\) לכל \(1\leq i\leq n\).

\end{definition}
\begin{proposition}
אם \(D,D':M_{n\times n}\left( \mathbb{F}  \right)\to \mathbb{F}\) פונקציות מוליטלינאריות ו-\(c \in \mathbb{F}\) אזי גם:
$$c D, D+D'$$
יהיו פונקציות מולטילינאריות.

\end{proposition}
\begin{definition}[פונקציה מתחלפת]
פונקציה \(D:M_{n\times n}\left( \mathbb{F}  \right)\to \mathbb{F}\) אשר לכל \(A \in M_{n\times n}\left( \mathbb{F}  \right)\) שיש לה שתי שורות סמוכות שוות מתקיים \(D(A)=0\).

\end{definition}
\begin{example}
עבור \(n=2\) הפונקציה המוגדרת על ידי:
$$D\begin{pmatrix}a_{11} & a_{12} \\a_{21} & a_{22}
\end{pmatrix}=a_{11}\cdot a_{22}-a_{12}\cdot a_{21}$$
היא מתחלפת כי:
$$D\begin{pmatrix}a_{11} & a_{12} \\a_{11} & a_{12}
\end{pmatrix}=a_{11}\cdot a_{12}-a_{11}\cdot a_{12}=0$$

\end{example}
\begin{definition}[פונקציית נפח]
פונקציה \(D:M_{n\times n}\left( \mathbb{F}  \right)\to \mathbb{F}\) אשר מוליטלינארית ומחלפת.

\end{definition}
\begin{corollary}
אם \(D',D:M_{n\times n}\left( \mathbb{F}  \right)\to \mathbb{F}\) פונקציות נפח ו-\(c \in \mathbb{F}\) אזי גם \(D'+D\) ו-\(cD\) פונקציות נפח.

\end{corollary}
\begin{proposition}
תהי \(D:M_{n\times n}\left( \mathbb{F}  \right)\to \mathbb{F}\) פונקציית נפח. אם \(A,B \in M_{n\times n}\left( \mathbb{F}  \right)\) כך ש-\(B\) מתקבל מ-\(A\) ע"י פעולת שורת אלמטרית \(R_{k}\leftrightarrow R_{k+1}\) כאשר \(1\leq k\leq n+1\) אזי \(D(B)=-D(A)\).

\end{proposition}
\begin{proof}
נסמן \(L_{i}^{A}=\ell_{i}\). כיוון ש-\(A,B\) מתקבלות על ידי חילוף של שורות סמוכות ניתן לכתוב:
$$A=\begin{pmatrix}\ell_{1} \\\vdots \\ \ell_{k}\\ \ell_{k+1}\\ \vdots \\ \ell_{n}\end{pmatrix}\quad B=\begin{pmatrix}\ell_{1} \\\vdots \\ \ell_{k+1}\\ \ell_{k}\\ \vdots \\ \ell_{n}
\end{pmatrix}$$
אם נחבר את המטריצות נקבל את המטריצה החדשה:
$$C=A+B=\begin{pmatrix}\ell_{1} \\ \vdots \\ \ell_{k}+\ell_{k+1} \\\ell_{k}+\ell_{k+1}\\ \vdots \\ \ell_{n}
\end{pmatrix}$$
כאשר נשים לב כי כיוון שיש שתי שורות סמוכות שוות נקבל \(D(C)=0\) ולכן:
$$D(C)=D(A+B)=D(A)+D(B)\implies D(A)=-D(B)$$

\end{proof}
\begin{corollary}
עבור שתי מטריצות \(A,B\) הנבדלות בחילוף שורה \(R_{i}\leftrightarrow R_{j}\) מתקיים:
$$D(A)=-D(B)$$

\end{corollary}
\begin{corollary}
פונקציית נפח \(D\) עם שתי שורות זההות מקיימת \(D(A)=0\). נובע מכך שניתן להחליף את השורות כך שיהיו סמוכות ואז נקבל כי \(-D(A)=0\) ולכן \(D(A)=0\).

\end{corollary}
\begin{proposition}
פונקציית נפח הפועלת של מטריצה \(A\) אשר שורה אחת שלה היא כפל בסקלר של שורה אחרת תקיים \(D(A)=0\).

\end{proposition}
\begin{proof}
נסמן את השורות של \(A\) ב-\(\ell_{1},\dots,\ell_{n}\). נניח כי \(\ell_{j}=c\ell_{i}\) כאשר \(c \in \mathbb{F}\). נגדיר \(B=\varepsilon(A)\) כאשר \(\varepsilon=R_{i}\to cR_{i}\) אז יש ל-\(B\) 2 שורות זההות ולכן מתקיים \(D(B)=0\). מאחר ש-\(D\) מולטילנארית מתקיים:
$$0=D(B)=cD(A)$$
ולכן עבור \(c\neq 0\) מתקיים \(\ell_{j}=0\) ולכן מלינאריות של \(D\) לפי השורה ה-\(j\) נובע כי \(D(A)=0\).

\end{proof}
\begin{proposition}
תהי \(D:M_{n\times m}\left( \mathbb{F}  \right)\to \mathbb{F}\) פונקציית נפח, ו-\(A,B \in M_{n\times n}\left( \mathbb{F}  \right)\) כך ש-\(B\) מתקבל מ-\(A\) על ידי ביצוע פעולת שורה אלמטרית \(R_{i}\to R_{i}+c\cdot R_{j}\) אז \(D(B)=D(A)\).

\end{proposition}
הרעיון של ההוכחה הוא לפרק את זה לפעולות של החלפה וכפל בסקלר.

\begin{definition}[מאפיין של פעולת שורה אלמנטרית]
פונקציה המקבל פעולת שורה אלמטרית ומחזירה מספר בצורה הבאה:
\begin{gather*}\mu\left( R_{i}\to cR_{i} \right)=c  \\\mu\left( R_{i}\leftrightarrow  R_{j}  \right) = -1 \\\mu\left( R_{i}\to R_{i}+cR_{j} \right)=1
\end{gather*}

\end{definition}
\begin{corollary}
אם \(D:M_{n\times n}\left( \mathbb{F}  \right)\to \mathbb{F}\) פונקציית נפח, ו-\(\varepsilon\) פעולת שורה אלמנטרית, מתקיים:
$$D\left( \varepsilon(A) \right)=\mu\left( \varepsilon \right)D(A)$$
בפרט אם מטריצות שקולות שורה אז ניתן לחשב את הערך של הפונקציית נפח ע"י הפעלה חוזרת של פעולות שורה אלמטריות.

\end{corollary}
\begin{proposition}
אם \(D\) פונקציית נפח ו-\(A\) מטריצה הפיכה, אזי \(D(A)=0\).

\end{proposition}
\begin{proposition}
פונקציית נפח \(D\) של מכפלה \(A,B\) מקיימת:
$$D(AB)=D(A) D (B)$$

\end{proposition}
\section{חישוב דטרמיננטה}

\begin{definition}[דטרמיננטה]
פונקציית נפח מנורמלת.

\end{definition}
\begin{definition}[דטרמיינטה של מטריצה]
זה יהיה שווה לדטרמיננטה של האופרטור המייצג את המטריצה.

\end{definition}
\begin{proposition}
קיימת פונקציית דטרמיינטה יחידה.

\end{proposition}
\begin{proposition}
דטרמיננטה של מטריצה \(1\times 1\) תהיה שווה:
$$\det([a])=a$$

\end{proposition}
\subsection{חישוב בעזרת מינורים}

\begin{definition}[המינור ה-ij של מטריצה]
נגדיר את המינור של ה-ij של מטריצה \(A\in M_{n\times n}\left( \mathbb{F}  \right)\) בתור מטריצה \((n-1)\times(n-1)\) אשר נוצרה על ידי מחיקה של השורה ה-\(i\) והעמודה ה-\(j\) ממטריצה \(A\). נסמן מטריצה זו ב- \(\text{minor}(A)_{ij}\).

\end{definition}
\begin{definition}[קו-פאקטור ה-ij של מטריצה]
יהי \(A \in M_{n\times n}\left( \mathbb{F}  \right)\). נגדיר:
$$\operatorname{cof}(A)_{i j}=(-1)^{i+j}m i n o r(A)_{i j}$$

\end{definition}
\begin{proposition}
יהי \(A \in M_{n\times n}\left( \mathbb{F}  \right)\) שורה של מטריצה. אזי:
$$\det A=\sum_{j=1}^{n} a_{ij}\text{cof}(A)_{ij}$$

\end{proposition}
\begin{proposition}
דטרמיננטה של מטריצה \(2\times 2\) תהיה:
$$\operatorname*{det}{\left[\begin{array}{l l}{a}&{b}\\ {c}&{d}\end{array}\right]}={\left|\begin{array}{l l}{a}&{b}\\ {c}&{d}\end{array}\right|}=a d-b c$$

\end{proposition}
\begin{proposition}
דטרמיננטה של מטריצה משולשית תהיה המכפלה של האיברים על האלכסון.

\end{proposition}
\begin{proposition}
למטריצות דומות יש את אותו דטרמיננטה. 

\end{proposition}
\begin{proof}
$$\operatorname*{det}B=\operatorname*{det}(P^{-1}A P)=\operatorname*{det}P^{-1}\operatorname*{det}A\operatorname*{det}P=\operatorname*{det}A P$$

\end{proof}
\begin{remark}
למעשה הטענה הזו חייבת להתקיים כי הגדרנו את הדטרמיננטה על אופרטור, ולכן צריך להיות נכון לא משנה באיזה בסיס היינו מציגים את האופרטור.

\end{remark}
\begin{corollary}
דטרמיננטה של \(A^{T}\) יהיה שווה לדטרמיננטה של \(A\). העובדה שהם דומות מוצגת בהמשך בעזרת צורת ג'ורדן.

\end{corollary}
\begin{proposition}[דטרמיננטה של הופכי]
כל עוד \(A\) הפיכה מתקיים:
$$\operatorname*{det}(A^{-1})={\frac{1}{\operatorname*{det}(A)}}$$

\end{proposition}
\subsection{חישוב בעזרת תמורות}

\begin{definition}[תמורה]
העתקה \(\sigma\) חח"ע ועל אשר מחליפה את האיברים

\end{definition}
\begin{definition}[מטריצת תמורה]
מטריצה \(P_\sigma\) אשר כפל בה שקול לביצוע התמורה. העמודה ה\(i\) של המטריצה זה \(e_\sigma(i)\).

\end{definition}
\begin{definition}[סימן של מטריצת תמורה]
הסימן הוא -1 או 1. שווה לדטרמיננטה של המטריצה תמורה.

\end{definition}
\begin{proposition}
מתקיים:

  \begin{enumerate}
    \item \(P_\sigma \cdot P_\pi = P\sigma \circ \pi\)


    \item \(\text{sgn}(\sigma) = \text{sgn}(\sigma^{-1})\)


    \item \(P_\sigma ^T = P_{\sigma^{-1}} = P_\sigma ^{-1}\)


  \end{enumerate}
\end{proposition}
\begin{proposition}[נוסחאת לייבניץ לדטרמיננטה]
$$\det A = \sum_{\sigma\in S_n} \text{sgn}(\sigma) \cdot a_{\sigma(1),1}\cdot \dots a_{\sigma(n),n} $$

\end{proposition}
\section{שימושים של דטרמיננטה}

\begin{definition}[מטריצת הקו פאקטור]
מטריצה המוגדרת על ידי:
$$cof(A)=[cof(A)_{ij}]=[(-1)^{i+j}m i n o r(A)_{i j}]$$

\end{definition}
\begin{definition}[מטריצה מצורפת]
באנגלית נקרא adjoint או adjugate. זוהי המטריצה המשוחלפת של המטריצת קו פאקטור. מסומן \(adj(A)\).

\end{definition}
\begin{example}
עבור:
$$A={\left[\begin{array}{l l}{a}&{b}\\ {c}&{d}\end{array}\right]}$$
נקבל:
$$a d j\left(A\right)=\left[\begin{array}{c c}{{d}}&{{-b}}\\ {{-c}}&{{a}}\end{array}\right]$$

\end{example}
\begin{proposition}
עבור \(A \in M_{n\times n}\left( \mathbb{F}  \right)\) מתקיים:
$$A\,a d j\,(A)=a d j\,(A)\,A=\operatorname*{det}{(A)I}$$
ובפרט אם \(\det A\neq 0\) נקבל כי \(A\) הפיכה ומתקיים:
$$A^{-1}=\frac{1}{\operatorname*{det}\left(A\right)}a d j\left(A\right)$$

\end{proposition}
\begin{proposition}[כלל קרמר]
נניח כי \(A \in M_{n\times n}\left( \mathbb{F}  \right)\) ונתון המשוואה \(Ax=b\). אזי:
$$x_{i}=\frac{\det(A_{i})}{\det (A)}$$
כאשר \(A_{i}\) יהיה המטריצה \(A\) כאשר מחליפים את העמודה ה-\(i\) של \(A\) עם המטריצה \(b\).

\end{proposition}
\begin{example}
נסתכל על המערכת משוואות הבאה:
 $${\left[\begin{array}{l l l}{1}&{\quad2\quad1}\\ {3}&{\quad2\quad1}\\ {2}&{-3\quad2}\end{array}\right]}\ {\left[\begin{array}{l}{x}\\ {y}\\ {z}\end{array}\right]}={\left[\begin{array}{l}{1}\\ {2}\\ {3}\end{array}\right]}$$
 מתקיים:
 $$x={\frac{\operatorname*{det}\left(A_{1}\right)}{\operatorname*{det}\left(A\right)}}={\frac{\left|\begin{array}{ccc}{1}&{2}&{1}\\ {2}&{2}&{1} \\ 3 & -3 & 2\end{array}\right|}{\left|\begin{array}{ccc} {1}&{2}&{1}\\{3}&{2}&{1}\\ 2 & -3 & 2\end{array}\right|}}={\frac{1}{2}}$$$$y={\frac{\operatorname*{det}\left(A_{2}\right)}{\operatorname*{det}\left(A\right)}}={\frac{\left|\begin{array}{l l l}{1}&{1}&{1}\\ {3}&{2}&{1}\\ {2}&{3}&{2}\end{array}\right|}{\left|\begin{array}{l l l}{1}&{2}&{1}\\ {3}&{2}&{1}\\ {2}&{-3}&{2}\end{array}\right|}}=-{\frac{1}{7}}$$$$z={\frac{\operatorname*{det}\left(A_{3}\right)}{\operatorname*{det}\left(A\right)}}={\frac{\left|\begin{array}{c c c}{1}&{2}&{1}\\ {3}&{2}&{2} \\ 2  & -3 & 3\end{array}\right|}{\left|\begin{array}{c c c}{1}&{2}&{1}\\ {3}&{2}&{1}\\2 & -3 & 2\end{array}\right|}}={\frac{11}{14}}$$

\end{example}
\Chapter{פולינומים}

\section{הגדרות בסיסיות}

\begin{definition}[פולינום]
ביטוי מתמטי שהוא צירוף של חזקות של משתנה(למשל \(x\)) כאשר כל חזקה של המשתנה יהיה עם מקדם מהשדה \(\mathbb{F}\). אוסף הפולינומים עם מקדמים מהשדה \(\mathbb{F}\) מסומן \(\mathbb{F} [x]\).

\end{definition}
\begin{definition}[מקדם מוביל]
המקדם של החזקה הגבוהה ביותר בפולינום.

\end{definition}
\begin{definition}[פולינום מתוקן]
פולינום אשר המקדם המוביל שלו הוא 1. 

\end{definition}
\begin{definition}[דרגה של פולינום]
המעלה של החזקה הגבוהה ביותר בפולינום. הדרגה של פולינום ה-0 מוגדר \(-\infty\). אם \(P\) פולינום הדרגה מסומנת ב-\(\mathrm{\deg(P)}\).

\end{definition}
\begin{proposition}[דרגה של סכום ומכפלה של פולינומים]
אם מכפילים 2 פולינומים אז המעלה של התוצאה תהיה סכום המעלות של כל אחד מהם. כאשר אם מחברים שתי פולינומים המעלה של הפולינומים תהיה לכל היותר המעלה מקסימלית מבין שתי הפולינומים. כלומר:
$$\deg\left( P_{1}\cdot P_{2} \right)=\deg(P_{1}) +\deg(P_{2}) \qquad \mathrm{\deg(P_{1}) }+\deg(P_{2}) \leq \max \left( \deg(P_{1}) ,\deg(P_{2})  \right)  $$

\end{proposition}
\begin{example}
עבור:
$$P_{1}(x)=x^{2}+x+1\qquad P_{2}(x)=-x^2-x+1$$
נקבל:
$$P_{1}(x)\cdot P_{2}(x)=- x^{4} - 2 x^{3} - x^{2} + 1 \qquad P_{1}+P_{2}=2$$
כאשר מתקיים:
$${\deg\left( P_{1}\cdot P_{2} \right) }=4\qquad \deg(P_{1}+P_{2})=0 $$

\end{example}
\begin{definition}[חלוקה]
פולנום \(P\) מחלק את \(Q\) אם קיים פולינום \(S\) כך ש-\(P=Q\cdot S\). 

\end{definition}
\begin{theorem}[השארית]
עבור פולינומים \(P,G\) קיימים פולינומים יחידים \(Q,R\) כך ש-\(P=QG+R\) כאשר \(\deg(R)<\deg(G)\).

\end{theorem}
\begin{remark}
זהו המקביל למשפט השארית של אוקלידוס עבור מספרים שלמים. הם למעשה אותו דבר כי המשפט הוא נכון לגבי מבנה אלגברי הנקרא חוג אוקלידי כאשר פולינומים ומספרים שלמים הם מקרה פרטי שלו.

\end{remark}
\begin{definition}[שורש של פולינום]
איבר \(a \in \mathbb{F}\) יהיה שורש של הפולינום \(P(x) \in \mathbb{F} [x]\) אם מתקיים:
$$P(a)=0$$

\end{definition}
\begin{definition}[סגור אלגברי]
שדה שעבורו לכל פולינום עם מקדמים מהשדה יש שורש.

\end{definition}
\begin{proposition}[קיום ויחידות הסגור האלגברי]
לכל שדה קיים סגור אלגברי יחיד.

\end{proposition}
\begin{remark}
טענה זו משתמשת בלמה של צורן, אשר שקול לאקסיומת הבחירה. זהו אקסיומה שלעיתים שנוייה במחלקות ואחת הסיבות לכך היא שאומנם היא אומרת שקיים כזה, אך לא נותנת שום דרך למצוא אותה. כך למשל אנחנו יכולים להגיד שלשדה \(\mathbb{Q}\) יש סגור אלגברי, אך אין לנו דרך טובה למצוא אותה במפורש(זה לא יהיה \(\mathbb{C}\) כי למשל \(\pi\) לא נמצא בסגור האלגברי של \(\mathbb{Q}\)). זה הופך את המשפט הבא ללא טריוויאלי.

\end{remark}
\begin{theorem}[המשפט היסודי של האלגברה]
השדה \(\mathbb{C}\) סגור אלגברית, והוא הסגור האלגברי של \(\mathbb{R}\). כלומר לכל פולינום עם מקדמים ממשיים יש שורש ב-\(\mathbb{C}\).

\end{theorem}
\begin{definition}[הצבת אופרטור בפולינום]
יהי \(P(x)\in \mathbb{F} [x]\) פולינום. אזי אם \(T:V\to V\) אופרטור לינארי ניתן להציב את האופרטור \(T\) בפולינום ולקבל אופרטור חדש.

\end{definition}
\begin{remark}
זה מוגדר היטב כיוון שכפל בסקלר מהשדה, חיבור של שתי אופרטורים, והרכבה של אופרטורים בתור הכפל הם מוגדרים. נחליף את הכפל ב-1 אופרטור היחידה כדי שהמימדים יהיו מתאימים.

\end{remark}
\begin{example}
עבור הפולינום \(P(x)=3x^{2}+2x+1\) ניתן להציב את האופרטור \(T\begin{pmatrix}x\\y\end{pmatrix}=\begin{pmatrix}y\\x\end{pmatrix}\) ולקבל:
$$P(T)=3T^2+2T+1=3+2T+1=2T+4\implies P(T)\begin{pmatrix}x \\ y\end{pmatrix}=2\begin{pmatrix}y \\ x\end{pmatrix}+4=\begin{pmatrix}2y + 4 \\ 2x + 4
\end{pmatrix}$$
כאשר השתמשנו בכך ש-\(T^{2}=Id\) והכפלנו את הקבועים באופרטור היחידה כדי שהפעולות יהיו נכונות מבחינת מימדים.

\end{example}
\section{פולינום מינימלי}

\begin{definition}[פולינום מינימלי של ווקטור]
יהי \(V\) מרחב ווקטורי ו-\(T:V\to V\) אופרטור. הפולינום המינמלי של ווקטור \(v \in V\) יהיה הפולינום \(m_{T,v}\) מהמעלה הנמוכה ביותר כך שמתקיים:
$$m_{T,v}(T)(v)=0$$

\end{definition}
\begin{proposition}[יחידות הפולינום המינימלי]
עבור ווקטור \(v\in V\) קיים פולינום מימינלי מתוקן יחיד \(m_{T,v}\).

\end{proposition}
\begin{remark}
ללא הדרישה שיהיה מתוקן היה אפשר להכפיל בסקלר ולקבל פולינום מינימלי של הווקטור חדש.

\end{remark}
\begin{definition}[הפולינום המינימלי של אופרטור]
זהו הפולינום \(m_{T} \in \mathbb{F} [x]\) מהמעלה הנמוכה ביותר המקיים לכל \(v \in V\):
$$m_{T}(T)(v)=0$$

\end{definition}
\begin{proposition}
המעלה של הפולינום המינימלי יהיה לכל היותר המימד של המרחב \(V\).

\end{proposition}
\begin{proposition}
הפולינום המינימלי של ווקטור יחלק את הפולינום המינימלי של האופרטור. כלומר עבור אופרטור \(T:V\to V\)  קיים \(S \in \mathbb{F} [x]\) כך שמתקיים:
$$m_{T}(x)=S(x)\cdot m_{T,v}(x)$$

\end{proposition}
\begin{corollary}
הדרגה של הפולינום המינימלי של ווקטור יהיה קטן או שווה לדרגה של הפולינום המינימלי של אופרטור. ולכן בפרט גם תמיד יהיה קטן מהדרגה של האופרטור.

\end{corollary}
\begin{definition}[פולינום מינימלי של מטריצה]
זהו הפולינום המינימלי של האופרטור אותו המטריצה מייצגת תחת בסיס כלשהו. זה מוגדר היטב בזכות הטענה הבאה.

\end{definition}
\begin{proposition}
למטריצות דומות יש את אותו הפולינום המינימלי.

\end{proposition}
\begin{remark}
למצוא פולינום מינימלי של מטריצה בדרך כלל אינו פשוט ודורש עבודה ידנית. ניתן לעשות זאת בעזרת הצעדים הבאים:

  \begin{enumerate}
    \item אם המטריצה היא כפל בסקלר של מטריצת היחידה, הפולינום המינימלי יהיה \(m_{A}=1\). 


    \item אם המטריצה היא צ"ל של כפל בסקלר של המטריצה עם כפל בסקלר של מטריצה היחידה, כלומר קיימים קבועים \(a,b \in \mathbb{F}\) כך ש: 
$$a\cdot A+b\cdot Id = 0$$
אז הפולינום המינימלי יהיה:
$$m_{A}=ax+b$$


    \item ניתן כעת לבדוק אם קיים פולינום מסדר שני, כלומר אם קיימים קבועים \(a,b,c \in \mathbb{F}\) כך שמתקיים: 
$$a\cdot A^{2}+b\cdot A+ c\cdot Id = 0$$
אז הפולינום המינימלי יהיה \(m_{A}=ax^{2}+bx+c\)


    \item ניתן להמשיך הלאה למעלות גבוהות יותר, כאשר ניתן להגיע לכל היותר למימד של המטריצה. 


  \end{enumerate}
\end{remark}
\section{פולינום אופייני}

\begin{definition}[פולינום אופייני]
יהי \(T:V\to V\) אופרטור לינארי. תהי \(A \in M_{n,n}\left( \mathbb{F}  \right)\) מטריצה המייצגת את הפולינום האופייני בבסיס כלשהו. אזי נגדיר את הפולינום האופייני להיות:
$$P_{A}(x)=\det(xI_{n}-A)$$
כאשר מוגדר היטב בגלל הטענה הבאה.

\end{definition}
\begin{proposition}
למטריצות דומות יש את אותו הפולינום האופייני

\end{proposition}
\begin{proof}
\begin{gather*}{{\operatorname*{det}(x I-B)=\operatorname*{det}(P^{-1}x I P-P^{-1}A P)}}=\\ {{}}{{=}}{{\operatorname*{det}[P^{-1}(x I-A)P]=\operatorname*{det}(x I-A)=c_{A}(x)}}\end{gather*}

\end{proof}
\begin{proposition}[קיום ויחידות]
לכל אופרטור לינארי \(T:V\to V\) קיים פולינום אופייני יחיד.

\end{proposition}
\begin{theorem}[קיילי המילטון]
הפולינום האופייני מאפס את האופרטור. כלומר מתקיים:
$$P_{A}(A)=0$$

\end{theorem}
\begin{proposition}
הפולינום המינימלי מחלק את הפולינום האופייני. כלומר קיים פולינום \(S \in \mathbb{F} [x]\) כך שמתקיים:
$$P_{A}(x)=S(x)\cdot m_{A}(x)$$

\end{proposition}
\begin{proposition}
האיבר החופשי של הפולינום האופייני יהיה שווה לדטרמיננטה כאשר \(n\) זוגי ושווה למינוס הדטרמיננטה כאשר \(n\) אי זוגי.

\end{proposition}
\begin{proof}
זאת כיוון שמתקיים:
$$P(0)=\det(-A)=\begin{cases}\det A & \text{\(n\) is even} \-\det A & \text{\(n\) is odd}
\end{cases}$$

\end{proof}
\begin{proposition}
המקדם של האיבר השני(האיבר מהמעלה הגבוהה ביותר שאינו המוביל) יהיה שווה למינוס העקבה.

\end{proposition}
\begin{proposition}
אם \(C=A\oplus B\) אזי \(P_{C}=P_{A}\cdot P_{B}\).

\end{proposition}
\Chapter{מרחבים אינוורינטים ונילפוטנטיות}

\section{מרחבים ציקלים ומרחבים אינוורינטיים}

\begin{definition}[תת מרחב ציקלי]
תהי \(V\) מרחב ווקטורי ו-\(T:V\to V\) אופרטור לינארי. התת מרחב ה-\(T\) ציקלי של ווקטור \(v \in V\) הוא התת מרחב \(U\leq V\) שנוצר מהפרוש של כל החזקות של \(T^{k}v\). כלומר:
$$U=\mathrm{Span}\left( v,Tv,T^{2}v, \dots, T^{k}v, \dots \right)$$

\end{definition}
\begin{proposition}
כאשר המרחב ווקטורי \(V\) נוצר סופית עם מימד \(n\), נקבל כי קיים \(k\leq n\) כך ש:
$$U=\mathrm{Span}\left( v,Tv, \dots, T^{k}v \right)$$
כאשר \(v,Tv, \dots, T^{k}v\) הם בלתי תלויים לינארית, ולכל \(m>k\) מתקיים \(T^{m}(v)\in U\).

\end{proposition}
\begin{proof}
נניח בשלילה שלא קיים \(k\leq n\) כזה. אזי יש אוסף של \(k>n\) ווקטורים בת"ל במרחב הווקטורי, בסתירה לכך שהמימד הוא \(n\).

\end{proof}
\begin{corollary}
אם \(U\) תת מרחב ציקלי עם מימד \(n\), אזי \(T^{n}v\) הוא החזקה הראשונה \(n\) כך ש-\(T^{n}v\) הוא צירף הלינארי של כל האיברים שלפניו. כלומר קיימים \(a_{0},a_{1},\dots,a_{n-1}\in \mathbb{F}\) כך ש:
$$T^{n}(v)=a_{0}v+a_{1}T^{1}(v)+ \dots + a_{n-1}T^{n-1}(v)$$

\end{corollary}
\begin{proposition}
ניתן להציג את האופרטור \(T\) במרחב ציקלי בבסיס \(\mathcal{A}=\left( v,Tv, \dots,T^{n-1}v \right)\), אז אם \(a_{0},\dots ,a_{n-1} \in \mathbb{F}\) הם כך שמתקיים:
$$a_{0}v+a_{1}Tv+ \dots + a_{n-1}T^{n-1}v + T^{n}v = 0$$
נקבל כי המטריצה המייצגת תהיה:
$$[T]_{\mathcal{A} }=\begin{pmatrix}0&0&0&\cdot\cdot\cdot&-a_{0}\\ 1&0&0&\cdot\cdot\cdot&-a_{1}\\ 0&1&0&\cdot\cdot\cdot&-a_{2}\\ \cdot&\cdot&\cdot\cdot&\cdot\cdot&\cdot\\ \vdots&\vdots&\cdot\cdot&\cdot\cdot&\cdot\\ 0&0&0&\cdot\cdot\cdot&-a_{n-1}\end{pmatrix}$$
כך שהפולינום המינימלי יהיה:
$$m_{T,v}=x^{n}-\sum_{i=0}^{n-1} a_{i}x^{i}$$

\end{proposition}
\begin{proof}
ראשית נשים לב כי:
$$T^{n}(v)=-a_{0}v-a_{1}T^{1}(v)- \dots - a_{n-1}T^{n-1}(v)$$
אנו יודעים כי:
\begin{gather*}[T]_{\mathcal{A} }=\begin{bmatrix}[T(v)]_{\mathcal{A} } \bigg| [T(Tv)]_{\mathcal{A} } \bigg| [T^{2}(Tv)]_{\mathcal{A} }\bigg|\dots \bigg| [T(T^{n-1}v)]_{\mathcal{A} }\end{bmatrix}=  \\=\begin{bmatrix}[T(v)]_{\mathcal{A} } \bigg| [T^2 v]_{\mathcal{A} } \bigg| [T^{2}(Tv)]_{\mathcal{A} }\bigg|\dots \bigg| T^{n}(v)\end{bmatrix} =  \\= \begin{bmatrix}e_{2}\big| e_{3} \big| e_{4} \big| \dots \big| -a_{1}e_{1}-a_{2}e_{2}-\dots a_{n-1}e_{n-1}\end{bmatrix}
\end{gather*}
וזוהי מטריצה הרצויה.

\end{proof}
\begin{definition}[תת מרחב אינווריאנטי]
תת מרחב \(U\leq V\) נקרא אינווריאנטי תחת העתקה \(T:V\to V\) אם לכל \(v \in U\) מתקיים \(Tv \in U\). כלומר זהו מרחב שההעתקה משאירה אותה במרחב.

\end{definition}
\begin{remark}
המרחב כולו ומרחב האפס הם המרחבים שתמיד \(T\)-אינווריאנטיים.

\end{remark}
\begin{proposition}
התת מרחב ה-\(T\) אינווריאנטי הקטן ביותר(כלומר עם המימד הקטן ביותר) שמכיל את הווקטור \(v \in V\) יהיה המרחב ה-\(T\) ציקלי שנוצר על ידי \(v\).

\end{proposition}
\begin{proposition}
יהי \(V\) מרחב וקטורי נוצר סופית. אם \(U\leq V\) תת מרחב \(T:V\to V\) איווריאנטי, אז קיים תת מרחב \(W\leq V\) כך שמתקיים \(V=U \oplus W\), כאשר קיים בסיס של \(T\) כך שהמטריצה תהיה מטריצת בלוקים משולשית.

\end{proposition}
\begin{proof}
  \begin{enumerate}
    \item אנו יודעים כי לכל \(u \in U\) מתקיים \(Tu \in U\). לכן ניתן להסתכל על האופרטור \(T|_{U}:U\to U\). 


    \item כיוון ש-\(U\) תת מרחב של מרחב ווקטורי נוצר סופית קיים עבורו בסיס \(\mathcal{U}=\left( u_{1},\dots, u_{m} \right)\).  


    \item מטענה שראינו, ניתן להרחיב את הבסיס לבסיס של \(V\). נסמן \(\mathcal{C}=\left( u_{m+1},\dots,u_{n} \right)\), וכן \(W=\mathrm{Span}(C)\). 


    \item ניתן לראות כי \(V=W\oplus U\) מטעמי מימד(מתקיים \(m+(n-m)=n\)). 


    \item נותר להראות רק כי מטריצת בלוקים משולשית.  מספיק להראות כי \(T(u)\in U\) לכל \(u \in \mathcal{ U}\), אך זה מתקיים כיוון ש-\(T\) אינווריאנטית. 


  \end{enumerate}
\end{proof}
\begin{corollary}
אם קיימים סכום ישר של שתי מרחבים \(T\)-אינווריאנטיים, נקבל כי קיים בסיס שבו המטריצה היא בלוקים על האלכסון.

\end{corollary}
\section{אופרטור נילפוטנטי}

\begin{definition}[אופרטור נילפוטנטי]
אופרטור \(T:V\to V\) אשר הפעלה ממושכת שלו שקולה לאופרטור האפס. כלומר קיים \(k \in \mathbb{N}\) כך ש-\(T^{k}=0\), כאשר ה-\(k\) המינימלי אשר מקיים דרישה זו נקראת דרגה הנילפוטנטיות.

\end{definition}
\begin{definition}[גובה של ווקטור]
אם \(T:V\to V\) אופרטור נילפוטנטי אז הגובה \(h\) של ווקטור \(v \in V\) זה הכמות המינימלית של פעמים שצריך להפעיל את \(T\) עד שהווקטור מתאפס. כלומר המספר \(k\in \mathbb{N}\) כך ש-\(T^{k}v=0\).

\end{definition}
\begin{proposition}
הפולינום האופייני של אופרטור נילפוטנטי יהיה \(x^{k}\)

\end{proposition}
\begin{proposition}
העקבה היא אפס והדטרמיננטה הם אפס.

\end{proposition}
\begin{proposition}
מתקיים:
$$\begin{array}{c}{\{0\}=\ker T^{0}<\ker T^{1}<\ldots<\ker T^{k}=V}\\ {V=\mathrm{Im}\;T^{0}>\mathrm{Im}\;T^{1}>\ldots>\mathrm{Im}\;T^{k}=\{0\}}\end{array}$$

\end{proposition}
\begin{proposition}
קיים בסיס עבורה המטריצה משולשית עם אפס על האלכסון.

\end{proposition}
\begin{proposition}
מטריצה משולשית עם 0 על האלכסון היא נילפוטנטית.

\end{proposition}
\begin{definition}[שרשרת]
יהי \(T:V\to V\) אופרטור. שרשרת תהיה סדרה \(v_{1},\dots,v_{k}\in V\) של ווקטורים אשר נקבעת לחלוטין על ידי האיבר הראשון \(v_{1} \in V\) באופן הבא:

  \begin{itemize}
    \item לכל איבר נגדיר \(v_{i}=Tv_{i-1}\) עד אשר מקבלים \(Tv_{i}=0\) בפעם הראשונה, וזה יהיה סוף השרשרת.
  \end{itemize}
\end{definition}
\begin{definition}[גובה של שרשרת]
גובה השרשרת תהיה כמות האיברים בשרשרת. כלומר עבור ווקטור \(v\) נקבל כי השרשרת שנוצרת מ-\(v\) תהיה:
$$S=\left( v, Tv, T^{2}v ,\dots ,T^{k}v \right)$$
כאשר קבוצה זו היא שרשרת מגובה \(k\).

\end{definition}
\begin{example}
נסתכל על \(\mathbb{R}^{3}\) ועל אופרטור הנקבע על ידי כפל במטריצה:
$$J_{3}(0)=\left[\begin{matrix}0 & 0 & 0\\1 & 0 & 0\\0 & 1 & 0\end{matrix}\right]$$

  \begin{enumerate}
    \item השרשרת \(S\) הנוצרת על ידי \(e_{1}\) תהיה: 
$$e_{1}\xrightarrow{\;T\;} e_{2}\xrightarrow{\;T\;} e_{3}\xrightarrow{\;T\;} 0\implies S=(e_{1},e_{2},e_{3})$$


    \item השרשרת \(S\) הנוצרת על ידי \(e_{2}\) תהיה: 
$$e_{2}\xrightarrow{\;T\;} e_{3}\xrightarrow{\;T\;} 0\implies S=(e_{2},e_{3})$$


    \item השרשרת הנוצרת על ידי \(e_{1}+e_{2}\) תהיה: 
$$e_{1}+e_{2}\xrightarrow{\;T\;} e_{2}+e_{3}\xrightarrow{\;T\;} e_{3}+0\xrightarrow{\;T\;} 0+0=0\implies S=(e_{1}+e_{2},e_{2}+e_{3},e_{3})$$


  \end{enumerate}
\end{example}
\begin{remark}
נשים לב כי השרשרות מוגדרות בצורה טובה רק עבור אופרטור נילפוטנטי. במובן שעבור אופרטור שלא נילפוטנטי ייתכן שהשרשרת חוזרת על עצמה - לא הייתה לה סוף.

\end{remark}
\begin{definition}[בסיס שרשראות]
אוסף של שרשראות שהסכום הישר שלהם פורש את המרחב - מה שהופך אותם לבסיס.

\end{definition}
\begin{proposition}
עבור אופרטור נילפוטנטי קיים בסיס שרשראות.

\end{proposition}
כאשר הטענה הזו נובעת בהאלגוריתם למציאת בסיס שרשראות:

\begin{proposition}[מציאת בסיס שרשראות]
  \begin{enumerate}
    \item מתחילים מקבוצה פורשת(כמו הבסיס הסטנדרטי). יוצרים מכל איבר שרשרת עד שמגיעים ל-0. 


    \item יש לנו אוסף שרשראות פורש. אם בת"ל, סיימנו. 


    \item אם לא, נדרש לצמצם ווקטורים עד שיהיה בסיס בעזרת הדרכים הבאות: 


    \item אם ראש של שרשרת שייך לפרוש של אחד מהשרשראות האחרות - מיותר. 


    \item אם יש לנו תלות לינארית, אז יש תלות לינארית בין הווקטורים בסוף השרשרת. נמצא את הצירוף הלינארי המאפס וניקח את הווקטור שכאשר מעלים עליו את האופרטור מגיעים לווקטור האפס. 


  \end{enumerate}
\end{proposition}
\begin{remark}
בסיס שרשראות רלוונטי רק עבור אופרטורים נילפוטנטיים כיוון שרק עבור אופרטורים נילפוטנטיים יפרוש את המרחב.

\end{remark}
\begin{proposition}[פירוק פיטינג]
לכל אופרטור \(T\) קיים פירוק יחיד לתתי מרחבים אינווריאנטים \(V_{I},V_{N}\) כאשר \(V=V_{I}\oplus V_{N}\) וכן \(T|_{V_{I}}\) אינווריאנטי והפיך ו-\(T|_{V_{N}}\) אינווריאנטי ונילפוטנטי. 

\end{proposition}
\begin{proposition}[מציאת פירוק פיטינג]
נחפש את הפעם הראשונה שבה מתקיים \(T^{k}=T^{k+1}\). במקרה זה הגרעין לא יגדל יותר ומתקיים:
$$V_{I}=\mathrm{Im}(T^{k})\qquad  V_{N}=\ker (T^{k})$$
כאשר נמצאים בסכום ישר מטעמי מימד(ממשפט המימדים הראשון). כלומר:
$$\ker \left( T \right)^{k}\oplus \mathrm{Im}\left( T\right)^{n-k}$$

\end{proposition}
\section{מטריצת ג'ורדן נילפוטנטית}

\begin{definition}[בלוק ג'ורדן אלמנטרי נילפוטנטי]
בלוק אשר מכיל את הערך \(\lambda\) על האלכסון, 1 מתחת לאלכסון, ו-0 בכל מקום אחר. נסמן אותו ב-\(J_{n}\left( \lambda \right)\) כאשר \(n\) זה גודל הבלוק. 

\end{definition}
\begin{example}
$$J_{3}(0)= \left[\begin{matrix}0 & 0 & 0\\1 & 0 & 0\\0 & 1 & 0\end{matrix}\right]\qquad J_{4}(0)=\left[\begin{matrix}0 & 0 & 0 & 0\\1 & 0 & 0 & 0\\0 & 1 & 0 & 0\\0 & 0 & 1 & 0\end{matrix}\right]\qquad  J_{6}(0)=\left[\begin{matrix}0 & 0 & 0 & 0 & 0 & 0\\1 & 0 & 0 & 0 & 0 & 0\\0 & 1 & 0 & 0 & 0 & 0\\0 & 0 & 1 & 0 & 0 & 0\\0 & 0 & 0 & 1 & 0 & 0\\0 & 0 & 0 & 0 & 1 & 0\end{matrix}\right]$$

\end{example}
\begin{remark}
לעיתים שמים את ה-1 מעל האלכסון. זה עניין של מוסכמה ולא באמת משנה.

\end{remark}
\begin{proposition}
המטריצה המייצגת של אופרטור נילפוטנטי תחת בסיס שרשראות יהיה בצורה של בלוקי ג'ורדן על האלכסון, כאשר גודל כל בלוק יהיה גודל השרשרת. זה למעשה יהפוך את זה למטריצה משולשית תחתונה ממש.

\end{proposition}
\begin{proof}
  \begin{enumerate}
    \item אנו יודעים כי לאופרטור נילפוטנטי קיים בסיס שרשראות. כלומר קיימים אוסף של סדרות בלתי תלויות: 
$$(a_{1,n})_{n=1}^{b_{1}},(a_{2,n})_{n=1}^{b_{2}},\dots,(a_{k,n})_{n=1}^{b_{k}}$$
כאשר \(b_{i}\) זה אורך השרשרת ה-\(i\). כיוון שיוצרים בסיס מתקיים:
$$\mathrm{Span}\left( (a_{1,n})_{n=1}^{b_{1}}\cup (a_{2,n})_{n=1}^{b_{2}}\cup\dots (a_{k,n})_{n=1}^{b_{k}} \right)=V$$


    \item כיוון שבלתי תלויות אחת בשנייה ופורשות אז נמצאות בסכום ישר. כלומר: 
$$\mathrm{Span}(a_{1,n})_{n=1}^{b_{1}}\oplus \mathrm{Span}(a_{2,n})_{n=1}^{b_{2}}\oplus \dots \oplus \mathrm{Span}(a_{k,n})_{n=1}^{b_{k}}$$
כאשר כל אחת מהם היא \(T\) אינווריאנטית ולכן מטענה קודמת נקבל כי מייצגות מטריצת בלוקים אלכסונית. 


    \item נסתכל על הבלוק הנוצר מ-\((a_{i,n})_{n=1}^{b_{i}}\). אנו יודעים כי העמודה הראשונה תהיה \(T(e_{1})\) כאשר זה מוגדר בתור הווקטור השני \(e_{2}\). באופן כללי העמודה ה-\(i\) תהיה \(T(e_{i})=e_{i+1}\) עבור \(i\leq b_{i}-1\) כאשר עבור האיבר האחרון \(T(e_{b_{i-1}})=0\). זוהי בדיוק הצורת ג'ורדן הנילפוטנטית. 


  \end{enumerate}
\end{proof}
\begin{corollary}
קיימת צורת ג'ורדן נילפוטנטית יחידה עד כדי שינוי סדר הבלוקים.

\end{corollary}
\begin{proposition}
הפולינום המינימלי יהיה גודל הבלוק הגדול ביותר, כיוון שזה יהיה הפעם הראשונה שכל הווקטורים יתאפסו.

\end{proposition}
\begin{proposition}
הדרגה של המטריצה תהיה הגודל שלה(למשל \(n\) עבור \(M \in M_{n\times n}\)) פחות כמות הבלוקים. 

\end{proposition}
\begin{proof}
כל בלוק מייצג שרשאת שמתאפסת - כלומר איבר בגרעין, ולכן כמות הבלוקים יהיה כמות האיבר בבסיס אשר בגרעין, כלומר הפרוש שלהם יהיה הבסיס לגרעין ולכן זה כמות שלהם יהיה המימד - שזה יהיה כמות הבלוקים.

\end{proof}
\begin{proposition}
כמות השרשראות שמתאפסות בכל שלב יהיה שווה לכמות שמאפסות פחות הכמות שהתאפסו פעם קודמת: 
$$\dim \ker T^k -\dim \ker T^{k+1}$$

\end{proposition}
\begin{proposition}
כמות השרשראות באורך \(k\) יהיה שווה בדיוק ל:
$$2\dim \ker T^k - \dim \ker T^{k-1} - \dim \ker T^{k+1}$$

\end{proposition}
\begin{proposition}[חזקה של בלוק ג'ורדן נילפוטנטי]
כל הכפלה נוספת של בלוק ג'ורדן נילפוטנטי "מורידה" את האיברים רמה אחת באלכסון.

\end{proposition}
\begin{example}
M^2 = \left[\begin{matrix}0 & 0 & 0 & 0 & 0\0 & 0 & 0 & 0 & 0\1 & 0 & 0 & 0 & 0\0 & 1 & 0 & 0 & 0\0 & 0 & 1 & 0 & 0\end{matrix}\right]$$$$
$$M^1 = \left[\begin{matrix}0 & 0 & 0 & 0 & 0\\1 & 0 & 0 & 0 & 0\\0 & 1 & 0 & 0 & 0\\0 & 0 & 1 & 0 & 0\\0 & 0 & 0 & 1 & 0\end{matrix}\right]M^3 = \left[\begin{matrix}0 & 0 & 0 & 0 & 0\\0 & 0 & 0 & 0 & 0\\0 & 0 & 0 & 0 & 0\\1 & 0 & 0 & 0 & 0\\0 & 1 & 0 & 0 & 0\end{matrix}\right]
M^4 = \left[\begin{matrix}0 & 0 & 0 & 0 & 0\\0 & 0 & 0 & 0 & 0\\0 & 0 & 0 & 0 & 0\\0 & 0 & 0 & 0 & 0\\1 & 0 & 0 & 0 & 0\end{matrix}\right]$$

\end{example}
\begin{remark}
לעיתים נוח לפצל מטריצה למטריצה אלכסונית + מטריצת ג'ורדן נילפוטנטית. זאת כיוון שקל לחשב חזקות של בלוק ג'ורדן נילפוטנטי וחזקות של מטריצות אלכסוניות, וכן אם הערכים על האלכסון הם זההים(כלומר כפולה של היחידה) אז המטריצה מתחלפת. לדוגמא:
$$A=\begin{bmatrix}2 & 0 \\1 & -1\end{bmatrix}=\begin{bmatrix}2 & 0 \\0 & -1\end{bmatrix}+\begin{bmatrix}0 & 0 \\1 & 0
\end{bmatrix}$$
וכעת נקבל:
$$A^{2}=\begin{bmatrix}2 & 0 \\0 & -1 \end{bmatrix}^{2}+\begin{bmatrix}2 & 0 \\0 & -1\end{bmatrix}\begin{bmatrix}0 & 0 \\1 & 0\end{bmatrix}+\begin{bmatrix}0 & 0 \\1 & 0\end{bmatrix}\begin{bmatrix}2 & 0 \\0 & -1\end{bmatrix}+\begin{bmatrix}0 & 0 \\1 & 0\end{bmatrix}=\begin{bmatrix}4+2 & 0 \\-1 & 1
\end{bmatrix}$$

\end{remark}
\Chapter{לכסון וצורת ג'ורדן}

\section{ערכים, ווקטורים ומרחבים עצמיים}

\begin{definition}[ערך עצמי ווקטור עצמי]
יהי \(V\) מרחב ווקטורי ו-\(T:V\to V\) אופרטור לינארי. מספר \(\lambda \in \mathbb{F}\) נקרא ערך עצמי המתאים לווקטור עצמי \(v \in V\) אם מקיים \(T(v)=\lambda v\).

\end{definition}
\begin{proposition}[תנאים שקולים ללהיות ערך עצמי]
  \begin{enumerate}
    \item הערך \(\lambda\) הוא ערך עצמי של \(T\). כלומר קיים \(v \in V\) כך ש-\(Tv=\lambda v\). 


    \item קיים \(0\neq v\in V\) כך ש \(\left( T-\lambda I \right)v=0\). 


    \item קיים \(0\neq v \in V\) כך ש- \(v \in \ker\left( T-\lambda I \right)\).  


    \item העתקה \(T-\lambda I\) היא לא חח"ע. 


    \item ההעתקה \(T-\lambda I\) היא לא על. 


    \item ההעתקה \(T-\lambda I\) לא הפיכה. 


    \item מתקיים \(\det\left( T-\lambda I \right)=0\). כלומר \(\lambda\) היא שורש של הפולינום האופייני. 


  \end{enumerate}
\end{proposition}
זה הופך את הפולינום האופייני לשימושי במיוחד כיוון שיכול להראות לנו אם ערך הוא ערך עצמי, ומאפשר למצוא אותם ביותר קלות.

\begin{proposition}[תכונות של ערכים עצמיים]
  \begin{enumerate}
    \item שורשים של הפולינום האופייני. 


    \item המימד של הגרעין זה הריבוי האלגברי של הערך עצמי 0. 


    \item סכום הע"ע זה העקבה. 


  \end{enumerate}
\end{proposition}
\begin{proposition}
אם \(v\) וקטור עצמי אז גם \(cv\) וקטור עצמי.

\end{proposition}
\begin{proof}
אם מתקיים \(Tv=\lambda v\) אז:
$$T(cv)=c(Tv)=c\lambda v=\lambda(cv)$$

\end{proof}
\begin{proposition}
וקטורים עצמיים עם ע"ע שונים הם בלתי תלויים לינארית

\end{proposition}
\begin{proof}
נניח בשלילה כי אוסף הווקטורים \(\{X_{1},X_{2},\cdots,X_{k}\}\) תלויים לינארית. מטענה שראינו קיים \(j\) כך ש האוסף \(\left\{X_{1},X_{2},\cdots,X_{j-1}\right\}\)
בלתי תלוי לינארית כאשר \(\{X_{1},X_{2},\cdots,X_{j}\}\) בלתי תלוי לינארית. ניקח צירוף לינארי \(a_{1},\dots,a_{j}\in \mathbb{F}\) מאפס:
$$a_{1}X_{1}+a_{2}X_{2}+\cdot\cdot\cdot+a_{j}X_{j}=0$$
כאשר כיוון שתלוי לינארית נניח כי לא כולם 0. נכפיל את המשוואה כולה ב-\(\lambda_{j}\) ונקבל:
$$a_{1}\lambda_{j}X_{1}+a_{2}\lambda_{j}X_{2}+\cdots+a_{j}\lambda_{j}X_{j}=0$$
כעת נחסיר את שתי המשוואות ונקבל:
$$a_{1}(\lambda_{1}-\lambda_{j})X_{1}+a_{2}(\lambda_{2}-\lambda_{j})X_{2}+\cdot\cdot\cdot+a_{j-1}(\lambda_{j-1}-\lambda_{j})X_{j-1}=0$$
וכיוון שקיבלנו צירוף לינארי של האיברים הבלתי תלויים לינארית \(\left\{  X_{1},\dots, X_{j-1}  \right\}\) נדרש כי כל המקדמים הם 0:
$$a_{1}(\lambda_{1}-\lambda_{j})=a_{2}(\lambda_{2}-\lambda_{j})=\cdots=a_{j-1}(\lambda_{j-1}-\lambda_{j})=0$$
וקיבלנו כי \(a_{j}X_{i}=0\) ולכן סתירה. כלומר האוסף \(\{X_{1},X_{2},\cdots,X_{k}\}\) הוא בלתי תלוי לינארית.

\end{proof}
\begin{definition}[מרחב עצמי]
תהי \(A\) מטריצה. עבור ערך עצמי \(\lambda\) נגדיר את המרחב העצמי של \(A\) להיות:
$$V_{\lambda}=\ker(\lambda I-A)$$
כלומר זהו אוסף כל הווקטורים שהם וקטורים העצמים עם ע"ע \(\lambda\).

\end{definition}
\begin{proposition}
הריבוי גאומטרי של הערך עצמי \(\lambda\) יהיה מימד המרחב העצמ \(\dim V_{\lambda}\).

\end{proposition}
\begin{definition}[אופרטור משולשי]
אופרטור שקיים בסיס שעבורו יהיה משולשי

\end{definition}
\begin{proposition}[תנאים שקולים לבסיס משולשי]
יהי \(\mathcal{A}=\left( v_{1},\dots,v_{n} \right)\) בסיס של \(V\). התנאים הבאים שקולים:

\end{proposition}
\begin{enumerate}
  \item האופרטור \(T\) תחת הבסיס \(\mathcal{ A}\) הוא משולשי. 


  \item המרחב \(\mathrm{Span\left( v_{1},\dots,v_{i} \right)}\) אינווריאנטי תחת \(T\) לכל \(1\leq i\leq n\).  


  \item מתקיים \(Tv_{i}\in \mathrm{Span\left( v_{1},\dots,v_{i} \right)}\) לכל \(1\leq i\leq n\). 


\end{enumerate}
כהאינטואציה מאחורי טענה זו ברורה. הווקטור הראשון יכול ללכת רק לכפולה של עצמו כדי שרק הרכיב על האלכסון יהיה שונה מ0. הווקטור השני יכול ללכת לצירוף לינארי של עצמו ושל ווקטור הראשון, וכן הלאה.

\begin{proposition}
הפולינום האופייני(או המינימלי) של אופרטור מתפרק לגורמים לינארים אם"ם האופרטור משולשי.

\end{proposition}
\begin{proposition}
הע"ע של אופרטור משולשי יהיו הערכים על האלכסון.

\end{proposition}
\begin{definition}[אופרטור לכסין]
אופרטור אשר עבורו קיים בסיס שבה המטריצה המייצגת שלו תהיה אלכסונית.

\end{definition}
\begin{definition}[מטריצה לכסינה]
מטריצה אשר דומה למטריצה אלכסונית.

\end{definition}
\begin{proposition}
אם \(\dim V=n\) ויש \(n\) ערכים עצמיים שונים אז \(T\) לכסין.

\end{proposition}
\begin{proof}
יש אוסף של \(n\) ווקטורים עצמיים עם ערכים עצמיים שונים ולכן מטענה שראינו הוקטורים העצמיים יהיו בת"ל. יש אוסף של \(n\) ווקטורים כאלה ולכן פורשים את המרחב ובסיס.

\end{proof}
\begin{proposition}
אופרטור הוא לכסין אם"ם קיים בסיס של ווקטורים עצמיים.

\end{proposition}
\begin{proposition}
אופרטור הוא לכסין אם"ם הפולינום המינימלי מתפרק לגורמים עם ריבוי אלגברי של לכל היותר 1.

\end{proposition}
\begin{proposition}
עבור מטריצה \(2\times 2\) הערכים העצמיים נתונים על ידי:
$$\lambda_{1},\lambda_{2}=m\pm \sqrt{ m^{2}-p }$$
כאשר \(m\) זה הממוצע של האלכסון הראשי ו-\(p\) זה הדטרימיננטה.

\end{proposition}
\begin{proof}
נכתוב מטריצה כללית:
$$A=\begin{pmatrix}a & b \\c & d
\end{pmatrix}$$
הפולינום האופייני יהיה:
$$\lambda^{2}-\mathrm{Tr}(A)\lambda+\det(A)=\lambda^{2}-(a+d)\lambda+\det A\equiv \lambda^{2}-2m\lambda+p$$
כאשר סימנו כמו בטענה \(m=\frac{a+d}{2}\) ו-\(p=\det A\). כעת מנוסחאת השורשים נקבל:
$$\lambda_{1,2}=\frac{2m\pm\sqrt{ 4 m^{2}-4p}}{2}=m\pm \sqrt{ m^{2}-p }$$

\end{proof}
\begin{corollary}
כיוון שהערכים העצמיים מקיימים שהממוצע שלהם הוא \(m\) והמכפלה שלהם היא הדטטרמיננטה(\(p\)) נוסחה זו מאפשרת למצוא שתי מספרים שהממוצע שלהם הוא \(m\) והמכפלה שלהם היא \(p\).

\end{corollary}
\section{צורת ג'ורדן}

\begin{definition}[מרחב עצמי מוכלל]
$$V_\lambda^+ = \bigcup_{k=1}^\infty \ker\left[(\lambda I - A)^k \right] $$
כלומר זהו המרחב שמתקבל מהפעלה חוזרת של \(\lambda I-A\) על האופרטור. במרחב נוצר סופית בהכרך ייתיצב בשלב מסויים ולכן קיים \(k\) כך ש:
$$V_{\lambda}^{+}=\ker \left( \lambda I-A \right)^{k}$$

\end{definition}
\begin{proposition}
הריבוי האלגברי יהיה המימד של המרחב העצמי המוכלל.

\end{proposition}
\begin{definition}[בלוק ג'ורדן]
בלוק אשר מכיל אפסים בכל האיברים פרט לאיבר מתחת לאלכסון. נסמן אותו ב-\(J_{n}(0)\) כאשר \(n\) זה גודל הבלוק. נשים לב מיידית כי זוהי מטריצה משולשית ממש ולכן נילפונטנטית.

\end{definition}
\begin{example}
$$J_{3}\left( \lambda \right)= \left[\begin{matrix}\lambda & 0 & 0\\1 & \lambda & 0\\0 & 1 & \lambda\end{matrix}\right]\qquad J_{4}\left( \lambda \right)=\left[\begin{matrix}\lambda & 0 & 0 & 0\\1 & \lambda & 0 & 0\\0 & 1 & \lambda & 0\\0 & 0 & 1 & \lambda\end{matrix}\right]\qquad  J_{6}\left( \lambda \right)=\left[\begin{matrix}\lambda & 0 & 0 & 0 & 0 & 0\\1 & \lambda & 0 & 0 & 0 & 0\\0 & 1 & \lambda & 0 & 0 & 0\\0 & 0 & 1 & \lambda & 0 & 0\\0 & 0 & 0 & 1 & \lambda & 0\\0 & 0 & 0 & 0 & 1 & \lambda\end{matrix}\right]$$

\end{example}
\begin{theorem}[ג'ורדן]
יהי \(T:V\to V\) העתקה. אם הפולינום האופייני של \(T\) מתפרק לגורמים לינארים אזי קיים בסיס כך שהמטריצה המייצגת של \(T\) תהיה בלוקי ג'ורדן על האלכסון:
$$A=\begin{pmatrix}J_{a_{1}}(\lambda_{1})&0&\cdots&0\\ 0&J_{a_{2}}(\lambda_{2})&\ddots&0\\ \vdots&\ddots&J_{a_{i}}(\lambda_{i})&0\\ 0&0&0&J_{a_{n}}(\lambda_{n})\end{pmatrix}$$
כאשר נקרא לצורה זו צורת ג'ורדן.

\end{theorem}
\begin{proof}
  \begin{enumerate}
    \item נוכיח באינדוקציה שלמה על מספר המימדים \(\dim V\). אם \(\dim V=1\) אז כל מטריצה היא תהיה בצורה של בלוקי ג'ורדן, והטענה מתקיימת. כעת נניח כי מתקיים עבור \(\dim V-1\).  


    \item כיוון ש-\(V\) מרחב ווקטורי מעל \(\mathbb{C}\), לפולינום האופייני יש לפחות שורש אחד \(\lambda\), אשר הוא ערך עצמי של \(T\). לכן האופרטור \(S=T-\lambda I\) הוא לא הפיך(0 יהיה ערך עצמי ולכן הגרעין לא טריוויאלי ולא הפיך) 


    \item נפעיל את משפט פירוק פיטנג על \(S\), כלומר קיימים \(V_{I},V_{N}\) כך שמתקיים \(V=V_{I}\oplus V_{N}\) ומתקיים \(S|_{V_{I}}\) הוא הפיך ו-\(S|_{V_{N}}\) נילפוטנטי. 


    \item כיוון ש-\(S\) לא הפיך, נקבל כי \(\dim V_{I}<\dim V\). אנו יודעים כי כיוון ש-\(S|_{V_{N}}\) אופרטור נילפוטנטי קיים צורת ג'ורדן נילפוטנטית. וכן מהנחת האינדוקציה קיים ל-\(S|_{V_{I}}\) צורת ג'ורדן.  


    \item קיבלנו כי יש בסיס \(\mathcal{ B}\) עבורו \([S]_{\mathcal{B}}\) בצורת ג'ורדן, כאשר יש \(J(0)\) בבלוק העליון. כעת עבור \(T=S+\lambda I\) נקבל כי \([T]_{\mathcal{ B}}\) תהיה מטריצה עם בלוק ג'ורדן \(J\left( \lambda\right)\) בבלוק העליון. ולכן בצורת ג'ורדן. 


  \end{enumerate}
\end{proof}
\begin{corollary}[אלגוריתם מטריצת מעבר לצורת ג'ורדן]
  \begin{enumerate}
    \item מוצאים ע"ע. עבור כל ע"ע \(\lambda\) עוברים על השלבים הבאים 


    \item מסמנים \(S=T-\lambda Id\). זה יהיה העתקה נילפוטנטית. 


    \item מוצאים את הסיס השרשראות של \(S\). 


    \item אם נאחד את כל הבסיסי שרשראות נקבל את הבסיס שעבורה המטריצה תהיה בצורת ג'ורדן. 


  \end{enumerate}
\end{corollary}
\begin{proposition}[תכונות של צורת ג'ורדן]
  \begin{enumerate}
    \item קיימת צורת ג'ורדן יחידה עד כדי שינוי סדר הבלוקים. 


    \item הפילונום המינימלי הוא יהיה המכפלה של הע"ע כאשר הריבוי האלגברי זה גודל הבלוק הגדול ביותר של הע"ע. 


    \item כל בלוק מייצג בסיס שרשאות. כמות השרשאות יהיו מספר הבלוקים וגודל כל שרשראת יהיה גודל של בלוק. 


    \item כל בלוק מייצג מרחב עצמי מוכלל. 


  \end{enumerate}
\end{proposition}
\begin{remark}
אם לא מעניין אותנו מטריצת מעבר והמטריצה קטנה מספיק ניתן למצוא את המטריצת ישירות בעזרת השלבים והשיקולים הבאים:

\end{remark}
\begin{enumerate}
  \item מחשבים פולינום אופייני ומוצאים ע"ע.  


  \item מחשבים את הגרעין של האופרטור \(S=T-\lambda Id\). זה יהיה כמות הבלוקים של הע"ע.  ניתן לבדוק אם ניתן לשבץ בלוקים כאשר אנו יודעים שהגודל המקסימלי של בלוק יהיה הריבוי האלגברי שלו. 


  \item ניתן להעלות בריבוע ולחשב את הגרעין. ההפרש בין המימד הגרעין של \(A^2\) למימד הגרעין של \(A\) יהיה כמות הבלוקים בגודל 2. 


  \item ניתן לנסות שוב לשבץ בלוקים תוך זה שאנו יודעים כי סך הגדלים של הבלוקים של כל ע"ע יהיה הריבוי האלגברי. 


  \item במידת הצורך להמשיך הלאה עד שאנו משבצים את כל הבלוקים 
כאשר הרעיון כאן הוא למעשה להשתמש בתכונות של מטריצת ג'ורדן נילפוטנטית כדי לשבץ את הבלוקים בכל שלב.


\end{enumerate}
\begin{definition}[צורת ג'ורדן ממשית]
מטריצה המורכת מבלוקים על האלכסון, כאשר מתחתיהם יש מטריצות יחידה.

\end{definition}
\begin{proposition}
לכל מטריצה ממשית קיימת צורת ג'ורדן ממשית. 

\end{proposition}
\begin{proposition}[שיטת מציאת מטריצת ג'ורדן ממשית]
  \begin{enumerate}
    \item נסמן \(C(a+bi)=\begin{pmatrix} a & b \\ -b & a\end{pmatrix}\)


    \item נמצא את הצורת ג'ורדן תחת המרוכבים. 


    \item אם קיימת מטריצת ג'ורדן ממשית, אז מופיע בלוק של איבר והצמוד שלו. כלומר אם יש בלוק של \(\lambda=a+bi\) ע"ע אז צריך להיות בלוק של \(\overline{\lambda}=a-bi\) מאותו גודל. 


    \item נחליף את הבלוק המרוכב בבלוק \(C(\lambda)\). 


    \item מתחת לאלכסון נמקם לפי הצורה של \(\lambda\) בצורת ג'ורדן את מטריצת היחידה. 


    \item נצפה לקבל מטריצה של בלוקים שכל בלוק של ע"ע מרוכב יהיה גדול פי 2 מהבלוק בצורת ג'ורדן וכמות המטריצות יחידה מתחתיו שווה לכמות ה-1 מתחת לאלכסון בצורת ג'ורדן. 


  \end{enumerate}
\end{proposition}
\begin{proposition}
אם מטריצה מורכבת מבלוקים של \(A\) כאשר מופיע ה-\(\overline{A}\) אז קיימת צורת ג'ורדן ממשית.

\end{proposition}
\begin{proposition}
אם קיים ע"ע מדומה שהצמוד שלו הוא לא ע"ע אז לא קיים צורת ג'ורדן ממשית.

\end{proposition}
\begin{proposition}
כל מטריצה דומה לשחלוף שלה.

\end{proposition}
\begin{proof}
נניח \(A \in M_{n\times n}\) מטריצה. נסתכל על המרחב ההווקטורי תחת השדה הסגור אלגברי. אזי קיים עבורה צורת ג'ורדן \(J_{A}\). נקבל כי \(J_{A}\) דומה למטריצה \(J_{A}^{T}\) בעזרת המטריצת מעבר בסיס:
$$\begin{pmatrix}0 & 0 & \dots & 0 & 1\\0 & 0 & \ddots & 1 & 0\\\vdots & \ddots & \ddots & \ddots & \vdots\\0 & 1 & \ddots & 0 & 0\\1 & 0 & \dots & 0 & 0
\end{pmatrix}$$
כלומר בעזרת המטריצת המכילה 1 על האלכסון הנגדי. בנוסף נשים לב כי מתקיים:
$$A=P ^{-1} J_{A}P\implies A^{T}=\left( P ^{-1} J_{A} P \right)^{T}=(P)^{T}J_{A}^{T}(P ^{-1})^{T}$$
ולכן \(A^{T}\) דומה לצורת ג'ורדן \(J_{A}^{T}\) אשר דומה לצורת ג'ורדן \(J_{A}\) אשר דומה ל-\(A\) וכיוון שדימיון הוא יחס שקילות מתקיים \(A\) דומה ל-\(A^{T}\).

\end{proof}
\section{אקספוננט של מטריצות}

\begin{reminder}
עבור מספר \(x \in \mathbb{C}\) מתקיים:
$$e^{x}=1+x+{\frac{x}{2}}+{\frac{x^{3}}{3!}}+\cdots=\sum_{m=0}^{\infty}{\frac{x^{m}}{m!}}$$
ניתן לחשוב על זה בתור הגדרה של אקספוננט של מספר מרוכב.

\end{reminder}
\begin{definition}[אקספוננט של מטריצה]
עבור \(A \in M_{n,m}\left( \mathbb{C} \right)\) נגדיר:
$$e^{A}=\sum_{m=0}^{\infty}\frac{1}{m!}A^{m}$$

\end{definition}
\begin{proposition}
אם מטריצות \(A,B\) מתחפות, מתקיים:
$$e^{A+B}=e^{A}e^{B}$$

\end{proposition}
\begin{proposition}
אם \(A\) מטריצה אלכסונית אזי:
$$e^{A}=\sum_{m=0}^{\infty}{\frac{1}{m!}}A^{m}=\sum_{m=0}^{\infty}{\frac{1}{m!}}\operatorname{diag}(a_{1}^{m},\ldots,a_{n}^{m})=\operatorname{diag}(e^{a_{1}},\ldots,e^{a_{m}})$$

\end{proposition}
\begin{proposition}
אם \(A\) דומה למטריצה אלכסונית בעזרת מטריצת מעבר \(P\) מתקיים:
$${\frac{1}{m!}}A^{m}={\frac{1}{m!}}\left(P B P^{-1}\right)^{m}=P\left({\frac{1}{m!}}B^{m}\right)P^{-1}$$
ולכן בפרט:
$$e^{A}=P e^{B}P^{-1}$$

\end{proposition}
כאשר המטריצה לא לכסינה, ניתן להשתמש בצורת ג'ורדן

\begin{proposition}
כאשר מטריצה \(A\) בצורת ג'ורדן, נקבל:
$$A=\mathrm{diag}(J_{1},\ldots,J_{k})$$
ולכן מתקיים:
$$A^{m}=\mathrm{diag}\bigl(J_{1}^{m},\ldots,J_{k}^{m}\bigr)\implies e^{A}=\mathrm{diag}(e^{J_{1}},\ldots,e^{J_{k}})$$

\end{proposition}
\begin{proposition}[נגזרת של אקספוננט]
$${\frac{d}{d t}}e^{A t}=A\,e^{A t}.$$

\end{proposition}
\begin{proof}
$$F^{\prime}(t)=\operatorname*{lim}_{h\to0}{\frac{e^{A(t+h)}-e^{A t}}{h}}=\operatorname*{lim}_{h\to0}{\frac{e^{A t}\,e^{A h}-e^{A t}}{h}}=e^{A t}\left(\operatorname*{lim}_{h\to0}{\frac{e^{A h}-I}{h}}\right)$$
כאשר אם נשתמש כעת בהגדרה של אקספוננט בתור טור חזקות נקבל:
$$F^{\prime}(t)=e^{A t}\left[\operatorname*{lim}_{h\to0}{\frac{1}{h}}\Big(A h+{\frac{A^{2}h^{2}}{2!}}+\cdots\Big)\right]=e^{A t}\,A$$

\end{proof}
\begin{proposition}[אקספוננט של הופכי]
$$\left(e^{A}\right)^{-1}=e^{-A}.$$

\end{proposition}
\begin{proposition}
עבור מטריצה \(A\) מתקיים:
$$\operatorname*{det}e^{A}=e^{\mathrm{tr}\,A}.$$

\end{proposition}
\section{לכסון סימולטני}

\begin{reminder}
מטריצה \(A\) נקראת לכסינה אם קיימת מטריצה הפיכה \(P\) כך ש- \(P ^{-1} A P\) מטריצה אלכסונית.

\end{reminder}
\begin{definition}[מטריצות לכסינות סימוטנית]
מטריצות \(A\) ו-\(B\) נקראות לכסינות סימולטנית אם קיים מטריצה \(P\) כך ש:
$$P ^{-1} A P\quad P ^{-1} B P$$
שניהם אלכסוניות.

\end{definition}
\begin{proposition}
מטריצות \(A,B\) לכסינות סימולטנית אם"ם \(AB=BA\). כלומר המטריצות מתחלפות.

\end{proposition}
\begin{proof}
נניח \(A=P D P ^{-1}\) וגם \(B=P D' P ^{-1}\). אזי:
$$AB=PDP ^{-1} P D' P ^{-1} = P D D' P ^{-1}$$
כאשר נזכור כי מטריצות אלכסוניות מתחלפות, ולכן \(D, D'\) בפרט מתחלפות ולכן:
$$AB=PD' D P ^{-1}= PD' P ^{-1} P D P ^{-1}=BA$$
ואכן מתחלפות.
כעת נניח כי מתחלפות, כלומר נניח כי \(AB=BA\). מספיק למצוא בסיס \(\mathcal{B}\) של ווקטורים עצמים שהם ווקטורים עצמיים גם של \(A\) וגם של \(B\), כיוון שמקרה זה נקבל כי תחת הבסיס \(\mathcal{B}\) המטריצות יהיו אלכסוניות. נניח כי ל-\(A\) יש ע"ע \(\lambda_{1},\dots,\lambda_{k}\). כאשר אם \(A\) לכסין אנו יודעים כי:
$$F^{N}= E_{\lambda_{1}}\oplus \dots \oplus E_{\lambda_{k}}$$
כאשר \(E_{\lambda_{i}}\) זה המרחב העצמי המוכלל של הערך עצמי \(\lambda_{i}\). 
נרצה להראות כי \(E_{\lambda}\) הוא תת מרחב \(B\) אינווריאנטי, כלומר אם \(v \in E_{\lambda_{i}}\) אז \(Bv \in E_{\lambda_{i}}\).
נניח \(v \in E_{\lambda}\). אנו יודעים כי כיוון ש-\(A,B\) מתחלפות מתקיים:
$$ABv = BAv=B(Av)=B\left( \lambda v \right)=\lambda Bv$$
כיוון ש-\(E_{\lambda}\) היא \(B\) אינווריאנטית ניתן להגדיר:
$$B|_{E_{\lambda}}:E_{\lambda}\to E_{\lambda}\quad B|_{E_{\lambda}}(v)=Bv$$
נשתמש בזה ש-\(B|_{E_{\lambda_{i}}}\) לכסין. ניתן למצוא כעת לכל \(i\) בסיס \(\mathcal{ B}\) כך ש-\(E_{\lambda_{i}}\) מתאים לווקטורים עצמיים של \(B\). ולכן נקבל תחת בסיס זה גם \(A\) וגם \(B\) אלכסוני.

\end{proof}
\begin{proposition}
אם \(A,M\) מטריצות ממשיות מאותו גודל כך ש-\(M\) מטריצה חיובית אזי קיים מטריצה הפיכה \(C\) כך ש:
$$C^{T}MC=I\quad C^{T}AC = D$$
כאשר \(D\) אלכסונית.

\end{proposition}
\begin{proof}
כיוון ש-\(M\) חיובית קיימת מטריצה הפיכה \(R\) כך שמתקיים:
Loading latex...

\end{proof}
\Chapter{תבניות בילינאריות}

\section{תבנית בילינארית}

\begin{definition}[תבנית בילינארית]
פונקציה \(b:V\times V\rightarrow \mathbb F\) המקיימת:

  \begin{enumerate}
    \item לינאריות לחיבור לפי הרכיב הראשון: 
$$b(u+v,w)=b(u,w)+b(v,w)$$


    \item לינאריות לחיבור לפי הרכיב השני: 
$$b(u,v+w)=b(u,v)+b(u,w)$$


    \item לינאריות לכפל בסקלאר לפי הרכיב הראשון: 
$$b(\alpha u,v)=\alpha b(u,v)$$


    \item לינארית לכפל בסקלאר לפי הרכיב השני: 
$$b\left( u,\alpha v \right)=\alpha b(u,v)$$


  \end{enumerate}
\end{definition}
\begin{example}
  \begin{enumerate}
    \item המכפלה הסקלארית \(b(v,u)=v^{T}u\) היא תבנית בילינארית. 


    \item אם \(V=\mathbb{F} [x]\) מרחב הפולינומים ו-\(a,b \in \mathbb{F}\) אזי \(b(p,q)=p(a)q(b)\) היא תבנית בילינארית. 


    \item אם \(V=C[a,b]\) מרחב הפונקציות הרציפות על \([a,b]\), אז \(b(f,g)=\int_{a}^{b} f(x)g(x) \, dx\) היא תבנית בלינארית. 


    \item אם \(A \in M_{n\times n}\left( \mathbb{F}  \right)\) אז \(b(v,u)=v^{T}Au\) היא תבנית בילינארית. 


  \end{enumerate}
\end{example}
\begin{definition}[תבנית ססקילינארית]
תבנית שבמקום תנאי 3 מקיימת:
$$ \alpha b(v, w) = \overline \alpha b( v, w )$$

\end{definition}
\begin{definition}[מטריצת גראם]
יהי \(V\) מרחב ווקטורי נוצר סופית, \(\mathcal{C}=\left( v_{1},\dots,v_{n} \right)\) בסיס של \(V\), ו-\(b:V\to V\) תבנית בילינארית. אזי:
$$[b]_{\mathcal{C} }=\begin{pmatrix}b(v_{1},v_{1})&b(v_{1},v_{2})&\dots&b(v_{1},v_{n})\\ b(v_{2},v_{1})&b(v_{2},v_{2})&\dots&b(v_{2},v_{n})\\ &&\dots\\ b(v_{n},v_{1})&b(v_{n},v_{2})&\dots&b(v_{n},v_{n})\end{pmatrix}$$

\end{definition}
\begin{corollary}
לכל תבנית בליניארית קיימת מטריצה גראם יחידה.

\end{corollary}
\begin{proposition}
יהי \(\mathcal{C}=\left( v_{1},\dots,v_{n} \right)\)  בסיס ו-\(b\) תבנית בילינארית. אזי לכל \(x,y \in V\) מתקיים:
$$b(x,y)=[x]_{\mathcal{C}}[b]_{\mathcal{C}}[y]_{\mathcal{C}}^{T}$$

\end{proposition}
\begin{proof}
נסמן את הרכיבים של \([x]_{\mathcal{C}}\) ו-\([u]_{\mathcal{C}}\):
$$[x]_{\mathcal{C}}=\left( \alpha_{1},\ldots,\alpha_{n} \right)\qquad [y]_{\mathcal{C}}=\left( \beta_{1},\ldots,\beta_{n} \right)$$
כעת ניתן לכתוב:
\begin{gather*}b(x,y)=b\left( \alpha_{1}v_{1}+\ldots+\alpha_{n}v_{n},\beta_{1}v_{1}+\ldots+\beta_{n}v_{n} \right)=\sum_{i=1}^{n}\sum_{j=1}^{n}\alpha_{i}\beta_{j}b(v_{i},v_{j})  \\=\sum_{i=1}^{n}\sum_{j=1}^{n}\alpha_{i}\beta_{j}\big{(}[b]_{\mathcal{C}}\big{)}_{i,j}=[x]_{\mathcal{C}}[b]_{\mathcal{C}}[y]_{\mathcal{C}}^{T}
\end{gather*}

\end{proof}
\begin{corollary}
יהי \(V\) מרחב ווקטורי עם בסיס \(\mathcal{C}=\left( v_{1},\dots,v_{n} \right)\). לכל מטריצה \(M \in M_{n\times n}\left( \mathbb{F}  \right)\) קיימת תבנית בילינארית יחידה כך ש:
$$\forall i,j\quad  1\leq i,j\leq n\qquad b(v_{i},v_{j})=M_{i,j}$$

\end{corollary}
\begin{proof}
נגדיר \(b(x,y)=[x]_{\mathcal{C}}M[y]_{\mathcal{C}}^{T}\). נשים לב כי זהו תבנית בלינארית. וכן לכל תבנית בילינארית קיימת מטריצת גראם \(M\) יחידה ולכן קיימת תבנית בילינארית יחידה כזו.

\end{proof}
\begin{remark}
ניתן לראות כי למעשה מטריצה ריבועית מאפיינת תבנית בילינארית. זה נותן לנו שמטריצה ריבועית למעשה מייצגת שתי "יצורים" שונים - מצד אחד מייצגת אופרטור לינארי, ומצד שני תבנית בילינארית, תלוי איך מסתכלים על המטריצה.

\end{remark}
\begin{proposition}[מעבר בסיס]
אם \(\mathcal{B}=\left( v_{1},\dots,v_{n} \right)\) ו-\(\mathcal{C}=\left( u_{1},\dots,u_{n} \right)\) אזי אם נסמן \(S=[Id]_{\mathcal{B}}^{\mathcal{C}}\) מתקיים:
$$[b]_{B}=S^{T}[b]_{\mathcal{C}}S.$$

\end{proposition}
\begin{proof}
\begin{gather*}{{\left(S^{T}[b]_{\mathcal{C}}S\right)_{i,j}=e_{i}^{T}S^{T}[b]_{\mathcal{C}}S e_{j}}} {{=[v_{i}]_{B}S^{T}[b]_{\mathcal{C}}S[v_{j}]_{B}^{T}}}\\ {{=[v_{i}]_{\mathcal{C}}[b]_{\mathcal{C}}[v_{j}]_{\mathcal{C}}^{T}}} {{=b(v_{i},v_{j})=([b]_{B})_{i,j}}}\end{gather*}

\end{proof}
\begin{definition}[מטריצות חופפות]
שתי מטריצות \(A, B \in M_{n\times n}\left( \mathbb{F}  \right)\) נקראות חופפות אם מייצגות את אותו תבנית בילינארית בבסיסים שונים. כלומר קיימת מטריצה הפיכה \(P\) כך שמתקיים:
$$B=P^{T}AP$$

\end{definition}
\begin{definition}[תבנית מנוונת]
תבנית בילינארית \(b\) נקראת מנוונת עם קיים \(0\neq x \in V\) כך שמתקיים:
$$b(x,v)=0$$
אחרת נאמר כי לא מנוונת.

\end{definition}
\section{תבנית בלינאריות סימטריות וליכסון}

\begin{definition}[תבנית בילינארית סימטרית]
תבנית בילינארית המקיימת:
$$b(u,v)=b(v,u)$$

\end{definition}
\begin{proposition}
אם התבנית הבילינארית היא סימטרית אז המטריצת גראם בכל בסיס תהיה סימטרית.

\end{proposition}
\begin{proof}
הערך ה-\(ij\) של המטריצת גראם נתונה על ידי \(G_{ij}=b(e_{i},e_{j})\) כאשר מתקיים:
$$G_{ij}=b(e_{i},e_{j})=b(e_{j},e_{i})=G_{ji}$$
ולכן סימטרית.

\end{proof}
\begin{definition}[תבנית הפוכה]
ההתבנית \(b^{T}:V\times V\to \mathbb{F}\) נקראת התבנית ההפוכה, ומקיימת:
$$b(v,w)=b^{T}(w,v)$$
כאשר ניתן לראות כי התבנית תהיה סימטרית אם"ם \(b=b^{T}\).

\end{definition}
\begin{definition}[תבנית בילינארית לכסינה]
תבנית אשר קיים עבורה בסיס שהמטריצת גראם בו תהיה אלכסונית.

\end{definition}
\begin{remark}
כמו שאמרנו מקודם, תבניות בלינאריות מהעתקות לינאריות הם יוצורים שונים! מטריצה יכולה לייצג תבנית בילינארית לכסינה אך העתקה לינארית לא לכסינה. למעשה, עבור תבניות בלינאריות הרבה יותר קל לאפיין לכסינות מאופרטורים בזכות הטענה הבאה.

\end{remark}
\begin{proposition}
תבנית בילינארית היא לכסינה אם"ם סימטרית.

\end{proposition}
\begin{proof}
  \begin{enumerate}
    \item הכיוון שמטריצה אלכסונית מייצגת תבנית ביל 
  \end{enumerate}
\end{proof}
\begin{corollary}
אם המציין שונה מ-2 כל מטריצה סימטרית חופפת למטריצה אלכסונית.

\end{corollary}
\begin{proposition}[ליכסון סימולטני]
קיים אלגוריתם ללכסון מטריצה הדומה למציאת הופכי בעזרת פעולות שונה אלמטריות:

\end{proposition}
\begin{enumerate}
  \item 
\end{enumerate}
\section{תבנית ריבועית}

\begin{definition}[תבנית ריבועית]
פונקציה \(Q:V\to \mathbb{F}\) עבורה קיימת תבנית בילינארית \(b:V\times V\to \mathbb{F}\)  כך ש:
$$Q(v)=b(v,v)$$
התבנית \(Q\) נקראת התנית הריבועית המאימה לתבנית הבילינאית \(b\).

\end{definition}
\begin{remark}
ניתן לראות מיידית כי תבנית ריבועית היא תבנית בילינארית סימטרית. הרי \(b(v,v)=b(v,v)\) גם אם מחליפים את \(v\) עם \(v\). כמו כן ניתן לראות כי עבור בסיס \(\mathcal{ C}\) מתקיים \(Q(v)=[v]_{\mathcal{ C} }^{T}[b]_{\mathcal{C} }[v]_{\mathcal{ C} }\).

\end{remark}
\begin{proposition}[פולריזציה]
יהי \(Q\) תבנית ריבועית. אם \(\mathrm{char}\left( \mathbb{F}  \right)\neq 2\) אז התבנית הבילינארית \(b\) המתאימה ל-\(Q\) נקבעת ביחידות ע"י:
$$b\left( \mathbf{v},\mathbf{w} \right)=\frac{1}{2}\left[Q\left( {\bf v}+{\bf w} \right)-
Q\left( {\bf v} \right)-Q\left( {\bf w} \right)\right]$$

\end{proposition}
\begin{proof}
נובע מזה שמתקיים:
$$Q\left( \mathbf{v}+\mathbf{w} \right)\,=\,b\left( \mathbf{v}+\mathbf{w},\mathbf{v}+\mathbf{w} \right)\,=b\left( \mathbf{v},\mathbf{v} \right)+b\left( \mathbf{v},\mathbf{w} \right)+b\left( \mathbf{w},\mathbf{w} \right)=Q\left( \mathbf{v} \right)+2b\left( \mathbf{v},\mathbf{w} \right)\,+\,Q\left( \mathbf{w} \right)$$

\end{proof}
\begin{proposition}
פונקציה היא תבנית ריבועית אם מתקיים:

\end{proposition}
\begin{enumerate}
  \item לכל \(\alpha \in \mathbb{F}\) נקבל \(Q\left( \alpha \mathbf{v} \right)=\alpha^{2}Q\left( \mathbf{v} \right)\). 


  \item התבנית \(Q\left( \mathbf{v}+\mathbf{w} \right)-Q\left( \mathbf{v} \right)-Q\left( \mathbf{w} \right)\) היא תבנית בילינארית. 


\end{enumerate}
קריטריון זה מאפשר לנו להראות כי תבנית היא ריבועית ללא תבנית בילינארית מפורשת.

\begin{example}
  \begin{enumerate}
    \item נראה ש-\(Q[(x,y)]=x^{2}+6x y+4y^{2}\) היא תבנית ריבועית. מתקיים: 
$$Q[\alpha(x,y)]=(\alpha x)^{2}+6(\alpha x)(\alpha y)+4(\alpha y)^{2}=\alpha^{2}(x^{2}+6x y+4y^{2})=\alpha^{2}Q(x,y)$$
כמו כן ניתן לוודא כי התבנית הבאה היא בילינארית:
$$Q[(x_{1},y_{1})+(x_{2},y_{2})]-Q[(x_{1},y_{1})]-Q[(x_{2},y_{2})]=2x_{1}x_{2}+6x_{1}y_{2}+6x_{2}y_{1}+8y_{1}y_{2}$$


    \item נראה כי עבור \(V=C([a,b])\) התבנית \(Q(f)=\int_{a}^{b}f(x)^{2}\,d x\) היא תבנית ריבועית. מתקיים: 
$$Q(\alpha f)=\int_{a}^{b}[\alpha f(x)]^{2}\,d x=\alpha^{2}\int_{a}^{b}f(x)^{2}\,d x=\alpha^{2}Q(f)$$
וכן ניתן לוודא ידנית כי הביטוי הבא הוא תבנית בילינארית:
$$Q(f+g)-Q(f)-Q(g)=\int_{a}^{b}[f(x)+g(x)]^{2}\,dx-\int_{a}^{b}f(x)^{2}\,dx-\int_{a}^{b}g(x)^{2}\,dx=\int_{a}^{b}2f(x)g(x)\,dx$$


  \end{enumerate}
\end{example}
\begin{proposition}
אם \(A\) מטריצה סימטרית אזי:
$$b(v)=vAv^{T}$$
היא תבנית ריבועית, כאשר \(A\) תהיה המטריצת גראם של התבנית הריבועית.

\end{proposition}
\begin{proposition}
תבנית ריבועית מקיימת:

  \begin{enumerate}
    \item עבור \(\alpha=0\) נקבל: 
$$Q\left( \alpha v \right)=0Q(v)=0$$


    \item עבור \(\alpha=-1\) נקבל: 
$$Q(-v)=-Q(v)=0$$


  \end{enumerate}
\end{proposition}
\begin{proposition}
כל פונקציה ריבועית על \(\mathbb{F} ^{n}\)(כלומר פולינום שהחזקה הגבוהה ביותר שלה היא 2) מייצגת תבנית ריבועית, כאשר אם נכתוב את התבנית הריבועית בצורה:
$$Q\left( \mathbf{x} \right)=b\left( \mathbf{x},\mathbf{x} \right)=\sum_{1\leq i\leq j\leq n}a_{i,j}x_{i}x_{j}$$
המטריצת גראם המתאימה לתבנית תהיה:
$$G_{j,i}=\begin{cases}a_{ij} & i=j \\\frac{1}{2}a_{ij} & i\neq j
\end{cases}$$

\end{proposition}
\Chapter{מרחבי מכפלה פנימית}

\section{מכפלה פנימית ונורמה}

\section{מרחבי מכפלה פנימית}

\begin{definition}[מכפלה פנימית ממשית]
יהי \(V\) מרחב ווקטורי מעל \(\mathbb{R}\). מכפלה פנימית ממשית היא תבנית בילינארית המקיימת לכל \(v,w \in V\):

  \begin{enumerate}
    \item סימטרית: \(\langle v, w \rangle = \langle w, v \rangle\). 


    \item חיוביות \(\langle v, v \rangle > 0\) לכל \(v\neq 0\). 


  \end{enumerate}
\end{definition}
\begin{definition}[מכפלה פנימית מרוכבת]
יהי \(V\) מרחב ווקטורי מעל \(\mathbb{C}\). מכפלה פנמימת מרוכבת היא תבנית ססקילינארית המקיימת לכל \(v,w \in V\)

  \begin{enumerate}
    \item הרמיטית: \(\langle v, w \rangle = \overline{\langle w, v \rangle}\). 


    \item חיובית: \(\langle v, v \rangle > 0\) לכל \(v\neq 0\). 


  \end{enumerate}
\end{definition}
\begin{definition}[מרחב מכפלה פנימית]
מרחב ווקטורי שמצוייד במכפלה פנימית.

\end{definition}
\begin{proposition}[תכנות של מרחב מכפלה פנימית]
נכילל את התכונות מכפלה פנמית ממשית ומרוכבת, כאשר קיום כל התנאים ביחד שקול ללהיות מרחב מכפלה פנימית:

  \begin{enumerate}
    \item סימטריצה - \(\langle x,y \rangle=\overline{\langle y,x \rangle}\) כאשר עבור ממשי נקבל כי הצמוד שווה לעצמו. 


    \item לינאריות לרכיב הראשון - \(\left\langle  ax+\beta y  ,z\right\rangle=\alpha \langle x,z \rangle+\beta \langle y,z \rangle\) לכל ווקטורים \(x,y,z\) וסקלארים \(\alpha,\beta\). 


    \item אי שליליות - לכל ווקטור \(x\) מתקיים \(\langle x,x \rangle\geq 0\). 


    \item לא מנוונת - מתקיים \(\langle x,x \rangle=0\) אם"ם \(x=0\)


  \end{enumerate}
\end{proposition}
\begin{corollary}
עבור הרכיב השני במרחב פנמי מרוכב מתקיים:
$$\left\langle  x,\alpha y+\beta z  \right\rangle =\overline{\left\langle  \alpha y+\beta z,x  \right\rangle } =\overline{\left\langle  \alpha y,x  \right\rangle +\left\langle  \beta z,x  \right\rangle } =\overline{\alpha\left\langle   y,x  \right\rangle }+\overline{\beta \langle z,x \rangle }  =\overline{\alpha} \langle x,y \rangle +\overline{\beta} \langle x,z \rangle $$

\end{corollary}
\begin{example}
ניתן להראות כי המכפלות הבאות הם מכפלות פנימיות:

  \begin{enumerate}
    \item המכפלה הפנימית הסטנדרטית - יהי \(V=\mathbb{R}^n\) או \(V=\mathbb{C}^{n}\). נגדיר: 
$$\langle x,y \rangle = y^{*}x=\sum_{k=1}^{n}x_{k}\overline{y} _{k}$$


    \item אם \(V\) הוא מרחב  הפולינומים ממעלה של לכל היותר \(n\), אזי ניתן להגדיר את המכפלה הפנימית: 
$$\langle f,g \rangle =\int_{-1}^{1}f(t){\overline{{g(t)}}}d t.$$


    \item המכפלה הפנימית של פרוביניוס - יהי \(V\) מרחב במטריצות ה-\(M_{m\times n}\). אזי נגדיר: 
$$\langle A,B \rangle =\mathrm{Tr}(B^{*}A)=\sum_{j,k} A_{j,k}\overline{B} _{j,k}$$


  \end{enumerate}
\end{example}
\begin{proposition}
יהי \(x \in V\). אזי \(x=0\) אם"ם:
$$\forall y \in V\quad \langle x,y \rangle=0 $$

\end{proposition}
\begin{corollary}
יהיו \(x,y \in V\). אזי \(x=y\) אם"ם לכל \(z \in V\) מתקיים \(\langle x,z \rangle=\langle y,z \rangle\)

\end{corollary}
\begin{corollary}
יהיו \(A,B:X\to Y\) שתי אופרטורים המקיימים לכל \(x \in X\) ולכל \(y \in Y\):
$$\langle Ax,y \rangle =\langle Bx,y \rangle $$
אזי \(A=B\).

\end{corollary}
\begin{definition}[נורמה]
הנורמה של \(v\) תהיה \(||v|| = \sqrt{\langle v, v \rangle}\). כאשר נשים לב כי תמיד תהיה ממשית וחיובית.

\end{definition}
\begin{remark}
זוהי תבנית ריבועית תחת \(\mathbb{R}\).

\end{remark}
\begin{theorem}[אי שיוויון קושי שוורץ]
לכל שתי ווקטורים \(x,y \in V\) מתקיים:
$$|\langle\mathbf{x},\mathbf{y}\rangle|\leq\|\mathbf{x}\|\cdot\|\mathbf{y}\|$$
כאשר מתקבל שיוויון רק אם \(x,y\) תלוים לינארית.

\end{theorem}
\begin{proof}
נסתכל על הביטוי הבא:
\begin{gather*}0\leq\|\mathbf{x}-t\mathbf{y}\|^{2}=\left\langle \mathbf{x}-t\mathbf{y},\mathbf{x}-t\mathbf{y} \right\rangle=\left\langle \mathbf{x},\mathbf{x}-t\mathbf{y} \right\rangle-t\left\langle \mathbf{y},\mathbf{x}-t\mathbf{y} \right\rangle\\=\|\mathbf{x}\|^{2}-t\left\langle \mathbf{y},\mathbf{x} \right\rangle-\overline{t}\left\langle \mathbf{x},\mathbf{y} \right\rangle+|t|^{2}\|\mathbf{y}\|^{2} 
\end{gather*}
כאשר עבור:
$$t={\frac{\langle\mathbf{x},\mathbf{y}\rangle}{\|\mathbf{y}\|^{2}}}={\frac{\overline{{\langle\mathbf{y},\mathbf{x}\rangle}}}{\|\mathbf{y}\|^{2}}}$$
נקבל:
$$0\leq\|\mathbf{x}\|^{2}-{\frac{|\langle\mathbf{x},\mathbf{y}\rangle|^{2}}{\|\mathbf{y}\|^{2}}}$$

\end{proof}
\begin{corollary}[אי שיוויון המשולש]
לכל שתי ווקטורים \(x,y \in V\) מתקיים:
$$\|\mathbf{x}+\mathbf{y}\|\leq\|\mathbf{x}\|+\|\mathbf{y}\|$$

\end{corollary}
\begin{proof}
$$\langle v,w \rangle = \frac{1}{4}\left\langle||v+w||^2 + ||v-w||^2\right\rangle$$$$\langle\mathbf{x},\mathbf{y}\rangle={\frac{1}{4}}\sum_{\alpha=\pm1,\pm i}\alpha\|\mathbf{x}+\alpha\mathbf{y}\|^{2}$$$$\left\|\mathbf{u}+\mathbf{v}\right\|^{2}+\left\|\mathbf{u}-\mathbf{v}\right\|^{2}=2(\left\|\mathbf{u}\right\|^{2}+\left\|\mathbf{v}\right\|^{2})$$$$\|\mathbf{x}\|_{p}=(|x_{1}|^{p}+|x_{2}|^{p}+\ldots+|x_{n}|^{p})^{1/p}=\left(\sum_{k=1}^{n}|x_{k}|^{p}\right)^{1/p}$$$$\|\mathbf{x}\|_{\infty}=\operatorname*{max}\{|x_{k}|:k=1,2,\ldots,n\}$$$$\|\mathbf{u}+\mathbf{v}\|^{2}+\|\mathbf{u}-\mathbf{v}\|^{2}=2(\|\mathbf{u}\|^{2}+\|\mathbf{v}\|^{2})$$

\end{proof}
\section{אורתוגונאליות}

\begin{definition}[ווקטורים ניצבים]
וקטורים הם ניצבים אם \(\langle v, w \rangle = 0\). לעיתים נקראים ווקטורים אורתוגונאלים, ומסומנים \(v\perp w\).

\end{definition}
\begin{proposition}
אפס אורתוגונאלי להכל. כלומר מתקיים לכל \(x \in V\):
$$\langle 0,x \rangle =\langle x,0 \rangle =0$$

\end{proposition}
\begin{theorem}[פתגורס]
אם \(v\perp w\) אז \(||v+w|| = ||v||^2 + ||w||^2\). המשפט ההפוך נכון עבור הממשיים כאשר לא נכון עבור המרוכבים.

\end{theorem}
\begin{proof}
$$\|\mathbf{u}+\mathbf{v}\|^{2}=\langle\mathbf{u}+\mathbf{v},\mathbf{u}+\mathbf{v}\rangle=\langle\mathbf{u},\mathbf{u}\rangle+\langle\mathbf{v},\mathbf{v}\rangle+\langle\mathbf{u},\mathbf{v}\rangle+\langle\mathbf{v},\mathbf{u}\rangle=\|\mathbf{u}\|^{2}+\|\mathbf{v}\|^{2}$$

\end{proof}
\begin{definition}[ווקטור אורתוגונלי לתת מרחב]
נאמר כי ווקטור \(v\) אורתוגונאלי לתת מרחב \(E\) אם \(v\) אורתוגונלי לכל \(w \in E\). כלומר:
$$\forall w \in E\quad \langle w,v \rangle =0$$

\end{definition}
\begin{proposition}
ווקטור אורתוגונאלים למרחב אם"ם אורתוגונאלי לכל ווקטור בקבוצה פורשת של התת מרחב. בפרט מספיק שאורתוגונאלי לכל אחד מאיברי הבסיס.

\end{proposition}
\begin{definition}[מערכת אורתוגונאלית]
קבוצה של ווקטורים בממ"פ אם כל 2 ווקטורים בקבוצה מאונכים אחת לשנייה. כאשר כל אחד מהווקטורים במערכת עם נורמה 1 אז נקרא למערכת אורתונורמלית.

\end{definition}
\begin{theorem}[פתגורס המוכלל]
אם \(v_{1},\dots,v_{n}\) מערכת אורתוגונאלית, אזי מתקיים:
$$\left\|\sum_{k=1}^{n}\alpha_{k}\mathbf{v}_{k}\right\|^{2}=\sum_{k=1}^{n}|\alpha_{k}|^{2}\|\mathbf{v}_{k}\|^{2}$$

\end{theorem}
\begin{proof}
$$\left\|\sum_{k=1}^{n}\alpha_{k}\mathbf{v}_{k}\right\|^{2}=\left(\sum_{k=1}^{n}\alpha_{k}\mathbf{v}_{k},\sum_{j=1}^{n}\alpha_{j}\mathbf{v}_{j}\right)=\sum_{k=1}^{n}\sum_{j=1}^{n}\alpha_{k}{\overline{{\alpha}}}_{j}\langle\mathbf{v}_{k},\,\mathbf{v}_{j}\rangle$$
כאשר בכלל אורתוגונאליות כאשר \(k\neq j\) נקבל \(\langle v_{k},v_{j} \rangle=0\) ולכן:
$$\left\|\sum_{k=1}^{n}\alpha_{k}\mathbf{v}_{k}\right\|^{2} = \sum_{k=1}^{n}|\alpha_{k}|^{2}\langle{\bf v}_{k},{\bf v}_{k}\rangle=\sum_{k=1}^{n}|\alpha_{k}|^{2}||{\bf v}_{k}||^{2}$$

\end{proof}
\begin{corollary}
כל מערכת אורתוגנאלית של ווקטורים שאינם אפס היא בלתי תלוייה לינארית

\end{corollary}
\begin{proof}
נניח כי יש לנו צרוף מאפס. כלומר קיימים \(\alpha_{1},\dots,\alpha_{n}\) כך ש:
$$\sum_{k=1}^{n}\alpha_{k}v_{k}=0$$
ולכן ממשפט פתגורס המוכלל נקבל:
$$0=\|\mathbf{0}\|^{2}=\sum_{k=1}^{n}|\alpha_{k}|^{2}\|\mathbf{v}_{k}\|^{2}$$
כאשר כיוון ש-\(\lVert v_{k} \rVert\neq 0\) לכל \(k\) לפי ההנחה, נקבל כי \(\alpha_{k}=0\) ולכן הצירוף היחיד המאפס הוא הטריוויאלי ובת"ל.

\end{proof}
\begin{definition}[בסיס אורתוגונאלי]
מערכת אורתוגונאלית אשר היא גם בסיס של המרחב.

\end{definition}
\begin{proposition}[ייצוג ווקטור בסיס אורתונורמלי]
יהי \(v \in V\) ו-\(\left( e_{1},\dots,e_{n} \right)\) בסיס אורתוגונאלי של \(V\). אזי:
$$v=\sum_{k=1}^{n}\frac{\langle v,v_{k} \rangle}{\lVert v_{k} \rVert ^{2}} $$
כאשר אם הבסיס אורתונורמלי נקבל פשוט:
$$v=\sum_{k=1}^{n} \langle v,v_{k} \rangle $$

\end{proposition}
\begin{proof}
מתקיים:
$$\mathbf{x}=\alpha_{1}\mathbf{v}_{1}+\alpha_{2}\mathbf{v}_{2}+\ldots+\alpha_{n}\mathbf{v}_{n}=\sum_{j=1}^{n}\alpha_{j}\mathbf{v}_{j}$$
אם ניקח את המכפלה הפנימית של שתי האגלפים עם \(v_{k}\) נקבל:
$$\langle\mathbf{x},\mathbf{v}_{k}\rangle=\sum_{j=1}^{n}\alpha_{j}\langle\mathbf{v}_{j},\mathbf{v}_{k}\rangle=\alpha_{k}\langle\mathbf{v}_{k},\mathbf{v}_{k}\rangle=\alpha_{k}||\mathbf{v}_{k}||^{2}$$
כאשר כל המכפלות הפנימיות נתאפסות בפט למכפלה \(\langle x,v_{k} \rangle\). לכן:
$$\alpha_{k}={\frac{\left\langle\mathbf{x},\mathbf{v}_{k}\right\rangle}{\|\mathbf{v}_{k}\|^{2}}}$$

\end{proof}
\begin{remark}
כדי למצוא בבסיס כללי נזכור שהיינו צריכים לפתור מערכת משוואות. טענה זו הופכת בסיסים אתוגונאליים לשימושיים במיוחד.

\end{remark}
\begin{definition}[המרחב הניצב]
עבור \(W\leq V\)  נגדיר את \(W^{\perp}\) להיות קבוצת כל הווקטורים שמאונכים לכל הווקטורים ב-\(W\). כלומר:
$$W^{\perp}=\left\{  v\mid v\perp W  \right\}=\left\{  v\mid \forall w \in W\quad \langle v,w \rangle =0  \right\}$$
כאשר נשים לב כי זה יהיה תת מרחב ווקטורי כיוון שכל צירוף לינארי של ווקטורים מאונכים גם כן יהיה מאונך.

\end{definition}
\begin{proposition}
יהי \(W \leq V\) תת מרחב ווקטורי. אזי לכל \(v \in V\) קיים הצגה יחידה כסכום של ווקטור ב-\(W\) ווקטור ב-\(W^{\perp}\). ולכן \(V=W\oplus W^{\perp}\). וכן בפרט מתקיים:
$$\dim W^{\perp}=\dim  V-\dim  W$$

\end{proposition}
\begin{proposition}
כאשר המרחב נוצר סופית(לא נכון כאשר לא) מתקיים:
$$\left( W^{\perp} \right)^{\perp}=W$$

\end{proposition}
\begin{reminder}[אופרטור הטלה]
אם \(V=W\oplus U\) קיים אופרטור \(P_{W}:V\to V\) כך שלכל \(v\in V\) מתקיים \(P_{W}v \in W\). העתקה זו למעשה לוקח את הרכיב ה-\(W\) של הווקטור(הרי ניתן לכתוב כל ווקטור כסכום של ווקטור מ-\(W\) ווקטור מ-\(U\)). אופרטור זה מקיים את התכונות הבאות:

\end{reminder}
\begin{enumerate}
  \item לכל \(W\leq V\) קיימת הטלה יחידה על \(W\). 


  \item התמונה של \(P\) תהיה \(W\). 


  \item מתקיים \(P_{W}|_{W}=Id\) - זאת כיוון שלקחת את הרכיב של \(W\) כאשר נמצאים ב-\(W\) יחזיר את עצמו. 


  \item מתקיים \(P^{2}=P\) - זאת כיוון שאם לקחנו פעם אחת את הרכיב של \(W\) אז אנחנו ב-\(W\), ואז התכונה הקודמת לא עשית כלום. 


\end{enumerate}
\begin{definition}[הטלה אורתוגונאלית]
העתקה \(P:V\rightarrow W\) שמקיימת
$$\forall v\in V\quad v-P(v) \in W^\perp$$
כאשר נשים לב כי בפרט הטלה ולכן מקיימת את כל התכונות.

\end{definition}
\begin{proposition}
ההטלה האורתוגונאלית של הבסיס האותונורמלי יהיה $$P_{w_1,...,w_n}=\sum_{j=1}^{i-1}\langle  w_j, v_i \rangle w_j$$

\end{proposition}
\begin{remark}
עבור אופרטור כללי לא פשוט למצוא את אופרטור ההטלה. כאשר עבור אופרטור אורתונורמלית הראנו כרגע שיש נוסחה מפורשת ישירה.

\end{remark}
\begin{proposition}[פירוק גראם שמידט]
אלגוריתם אשר הופך כל אוסף ווקטורים לאוסף אורתונורמלי תוך כדי זה שמשמר את הפרוש של האוסף. נדרש לבצע את השלבים הבאים:

  \begin{enumerate}
    \item מתחילים עם אוסף ווקטורים \(v_1,...,v_n\).  


    \item נבחר את אחד הווקטורים ונגדיר \(w_1=\frac{v_1}{||v_1||}\)


    \item עבור \(2\leq i \leq n\) נגדיר 
$$w_i' = v_i - P_{span(v_1,..., v_{i-1})} (v_i)$$
וננרמל:    $$w_i = \frac{w_i'}{||w_i'||}$$


    \item הסדרה \((w_1,...,w_n)\) תהיה קבוצה אותונורמלית עם אותו פרוש כמו הקבוצה המקורית. 


  \end{enumerate}
\end{proposition}
\begin{corollary}
לכל מרחב ווקטורי קיים בסיס אורתונורמלי.

\end{corollary}
\begin{definition}[המרחב הניצב]
לכל מרחב ווקטורי \(W\leq V\) קיים מרחב \(W^{\perp}\) אשר יהיה אוסף כל הווקטורים אשר ניצבים ל-\(W\).

\end{definition}
\section{צמוד ואונטרי}

\begin{definition}[מטריצה צמודה]
מוגדרת להיות \(A^* = (\overline{A})^t\).

\end{definition}
\begin{definition}[אופרטור צמוד]
האופרטור היחיד \(T^*:V\rightarrow V\) המקיים 
$$\forall v,w \quad \langle T(v), w\rangle = \langle v, T^*(w)\rangle$$

\end{definition}
\begin{proposition}[תכונות של הצמוד]
  \begin{enumerate}
    \item קיים ויחיד. 


    \item לינאריות לחיבור - \((T+S)^* = T^* + S^*\). 


    \item הרכבה - \((T\circ S)^* = S^* \circ T^*\). 


    \item צמוד של כפל בסקלאר - \((\alpha T)^* = \overline{\alpha} T^*\). 


    \item צמוד של צמוד - \((T^{*})^{*}=T\). 


    \item הפעלה על האיבר השני - \(\forall v,w: \langle v, T(w) \rangle = \langle T^*(v), w \rangle\)


  \end{enumerate}
\end{proposition}
\begin{definition}[צמוד לעצמו]
  \begin{enumerate}
    \item עבור מטריצה ממשית אם \(A=A^{t}\) - כלומר סימטרי. 


    \item עבור מטריצה מרוכבת אם \(A=A^{*}\) כלומר הרמיטי. 


    \item עבור אופרטור אם \(T=T^{*}\). 


  \end{enumerate}
\end{definition}
\begin{remark}
למרות שצמוד לעצמו והרמיטי הם באופן כללי אותו דבר - בדרך כלל כשמתייחסים לאופרטור קוראים לזה צמוד לעצמו וכשמתייחסים למטריצה קוראים לזה הרמיטי.

\end{remark}
\begin{proposition}
אם \(\lambda\) ערך עצמי של אופרטור \(T\) אז \(\overline{\lambda}\) הוא ערך עצמי של אופרטור \(T^{*}\). בפרט אם \(T\) צמוד לעצמו מתקיים כי \(\lambda\) ממשי.

\end{proposition}
\begin{proposition}
עבור אופרטור צמוד לעצמו ווקטורים עצמיים עם ערכים עצמיים שונים יהיו אורתוגונאליים.

\end{proposition}
\begin{proof}
עבור ערך עצמי \(\lambda\) מתקיים:
$$ \langle A\mathbf{u},\mathbf{v})=\langle\lambda\mathbf{u},\mathbf{v})=\lambda\langle\mathbf{u},\mathbf{v})$$
כאשר מצד שני:
$$\langle A\mathbf{u},\mathbf{v})=\langle\mathbf{u},A^{*}\mathbf{v} \rangle =\langle\mathbf{u},A\mathbf{v} \rangle =\langle\mathbf{u},\mu\mathbf{v} \rangle ={\overline{{\mu}}}\langle\mathbf{u},\mathbf{v} \rangle =\mu\langle\mathbf{u},\mathbf{v} \rangle $$
כאשר כל עוד \(\mu \neq \lambda\) נדרש כי ייתקיים \(\langle u,v \rangle=0\).

\end{proof}
\section{איזומטריה}

\begin{definition}[איזומטריה]
אופרטור \(U:X\to Y\) נקרא איזומטריה אם משמר את הנורמה. כלומר:
$$\forall\mathbf{x}\in X\quad \|U\mathbf{x}\|=\|\mathbf{x}\|$$

\end{definition}
\begin{example}
אם \(\lambda \in \mathbb{F}\) אזי \(\lambda I\) הוא איזומטריה אם"ם \(\left\lvert  \lambda  \right\rvert=1\).

\end{example}
\begin{example}
אם \(T\) אופרטור עם \(n\) ערכים עצמיים \(\lambda_{i}\)(ולכן קיים בסיס \(e_{1},\dots,e_{n}\) של ווקטורים עצמים) כך שהגודל של כל ערך עצמי יהיה \(\left\lvert  \lambda_{i}  \right\rvert=1\). יהי \(v \in V\). מתקיים:
$$\nu=\langle\nu,e_{1}\rangle e_{1}+\cdots+\langle\nu,e_{n}\rangle e_{n}$$
וכעת:
$$\left|\left|\nu\right|\right|^{2}=|\left\langle\nu,e_{1}\right\rangle|^{2}+\cdot\cdot\cdot+|\left\langle\nu,e_{n}\right\rangle|^{2}.$$
כאשר ניתן להפעיל את \(T\) ולקבל:
$$Tv=\langle v,e_{1}\rangle Te_{1}+\cdots+\langle v,e_{n}\rangle Te_{n}=\lambda_{1}\langle v,e_{1}\rangle e_{1}+\cdots+\lambda_{n}\langle v,e_{n}\rangle e_{n}$$
וכעת:
$$\left\|T v\right\|^{2}=|\left\langle v,e_{1}\right\rangle|^{2}+\cdot\cdot\cdot+|\left\langle v,e_{n}\right\rangle|^{2}$$
ולכן \(\lVert v \rVert=\lVert Tv \rVert\), ו-\(T\) איזומטרייה.

\end{example}
\begin{proposition}
אופרטור \(U:X\to Y\) הוא איזומטריה אם"ם משמר את המכפלה הפנימית. כלומר אם:
$$\forall\mathbf{x},\mathbf{y}\in X\quad \left\langle \mathbf{x},\mathbf{y} \right \rangle =\left\langle U\mathbf{x},U\mathbf{y} \right \rangle $$

\end{proposition}
\begin{proof}
משתמשת בהות פולריזציה. עבור מרחב ווקטורי מרוכב:
\begin{gather*}{{ \langle U\mathbf{x},U\mathbf{y} \rangle =\frac{1}{4}\sum_{\alpha=\pm1,\pm i}\alpha\|U\mathbf{x}+\alpha U\mathbf{y}\|^{2}}}\\ {{=\frac{1}{4}\sum_{\alpha=\pm1,\pm i}\alpha\|U( f{x}+\alpha\mathbf{y})\|^{2}}}\\ {{=\frac{1}{4}\sum_{\alpha=\pm1,\pm i}\alpha\|\mathbf{x}+\alpha\mathbf{y}\|^{2}=\langle \mathbf{x},\mathbf{y} \rangle }}\end{gather*}
כאשר באופן דומה עבור מרחב ווקטורי ממשי:
\begin{gather*}{{\langle U\mathbf{x},U\mathbf{y} \rangle =\frac{1}{4}\left( \|U\mathbf{x}+U\mathbf{y}\|^{2}-\|U\mathbf{x}-U\mathbf{y}\|^{2}\right)}}\\ {{=\frac{1}{4}\left(\|U(\mathbf{x}+\mathbf{y})\|^{2}-\|U ( \mathbf{x}-\mathbf{y})\|^{2}\right)}}\\ {{=\frac{1}{4}\left \langle \|\mathbf{x}+\mathbf{y}\|^{2}-\|\mathbf{x}-\mathbf{y}\|^{2}\right)= \langle \mathbf{x},\mathbf{y} \rangle }}\end{gather*}

\end{proof}
\begin{proposition}
אופרטור \(U:X\to Y\) הוא איזומטרייה אם"ם \(U^{*}U=I\).

\end{proposition}
\begin{proof}
אם \(U^{*}U=I\) אז לפי הגדרת האופרטור הצמוד:
$$\forall\mathbf{x}\in X\quad \left \langle  \mathbf{x},\mathbf{x} \right \rangle =\left \langle  U^{*}U\mathbf{x},\mathbf{x} \right \rangle =\left \langle  U\mathbf{x},U\mathbf{x} \right \rangle $$
ולכן \(\lVert x \rVert=\lVert Ux \rVert\), ולכן איזומטריה. לחלופין אם \(U\) הוא איזומטריה נקבל:
$$\forall{\bf y}\in X\quad \left \langle  U^{*}U{\bf x},{\bf y} \right \rangle =\left \langle  U{\bf x},U{\bf y} \right \rangle =\left \langle  {\bf x},{\bf y} \right \rangle $$

\end{proof}
\begin{corollary}
כל איזומטרייה היא הפיכה מצד שמאל. 

\end{corollary}
\begin{definition}[מטריצה איזומטרית]
מטריצה \(U\) תהיה איזומטריה אם מקיימת \(U^{*}U=Id\).

\end{definition}
\begin{proposition}
מטריצה היא איזומטרית אם"ם העמודות שלה יוצרות מערכת אורתונורמלית.

\end{proposition}
נסכם את אוסף השקילויות של איזומטרייה

\begin{proposition}
יהי \(S\) אופרטור לינארית. הביטיים הבאים שקולים:

  \begin{enumerate}
    \item האופרטור \(S\) הוא איזומטרייה. 


    \item לכל \(u, v \in V\) מתקיים \(\langle Su,Sv \rangle=\langle u,v \rangle\). 


    \item לכל אוסף אורתונורמלי \(e_{1},\dots , e_{n}\in V\) נקבל \(Se_{1},\dots,Se_{n}\) אוסף אורתונורמלי. 


    \item קיים בסיס אורתונורמלי \(e_{1},\dots , e_{n}\in V\) כך ש- \(Se_{1},\dots,Se_{n}\) אוסף אורתונורמלי.  


    \item מתקיים \(S^{*}S = I\) או \(S S^{*}=I\). 


    \item האופרטור \(S^{*}\) הוא איזומטרייה. 


    \item האופרטור \(S\) הפיך וצמוד לעצמו - כלומר \(S ^{-1} = S^{*}\). 


  \end{enumerate}
\end{proposition}
\begin{proposition}
עבור העתקה לינארית \(S\) תחת המרוכבים נקבל \(S\) הוא איזומטרייה אם"ם קיים בסיס אורתונורמלי של \(V\) כך שכל הווקטורים עצמיים יהיו עם ערך עצמי עם נורמה 1.

\end{proposition}
\section{אוניטריות}

\begin{definition}[אופרטור אוניטרי]
איזומטריה אשר הפיך.

\end{definition}
\begin{corollary}
איזומטריה \(U:X\to Y\) הוא אוניטרי אם"ם \(\dim X=\dim Y\)(רק מטריצות ריבועיות הם הפיכות). לכן אם יש אופרטור המקיים \(U U^{*}=Id\) אז אוניטרי.

\end{corollary}
\begin{definition}[אופרטור אורתוגונאלי]
אופרטור אוניטרי עם ערכים ממשים נקרא אורתוגונאלי.

\end{definition}
\begin{proposition}[תכונות של אופרטור אוניטרי]
  \begin{enumerate}
    \item אם \(U\) אוניטרי אזי \(U^{-1}=U^{*}\), כאשר אופרטורים אלו גם כן יהיו אוניטריים. 


    \item אם \(U\) אוניטרי(או אפילו איזומטריה) אז אם \(\left( v_{1},\dots,v_{n} \right)\) מערכת אורתונורמלית גם \(\left( Uv_{1},\dots,Uv_{n}  \right)\) תהיה מערכת אורתונורמלית. 


    \item מכפלה(כלומר הרכבה) של אופרטורים אוניטרלים יהיה אוניטרי. 


  \end{enumerate}
\end{proposition}
\begin{definition}[מטריצה אוניטרית]
מטריצה המקיימת \(U^{*}U=Id\). אם המטריצה היא עם ערכים ממשיים אז נקרא לה אורתוגונאלית.

\end{definition}
\begin{corollary}
מטריצה היא אוניטרית אם"ם היא הפיכה והעמודות שלה אורתוגונאליים.

\end{corollary}
\begin{example}
המטריצת סיבוב:
$$\left(\begin{array}{l l}{{\cos\alpha}}&{{-\sin\alpha}}\\ {{\sin\alpha}}&{{\cos\alpha}}\end{array}\right)$$
היא אוניטרית. ניתן לראות כי העמודות אורתוגונאליים.

\end{example}
\begin{proposition}[שקילות לאוניטריות]
הטענות הבאות שקולות:

  \begin{enumerate}
    \item אופרטור \(T\) אורתוגונאלי/איניטרי 


    \item קיים בסיס אותונורמלי \(\mathcal{B}\) כך ש-\([T]_\mathcal{B}\) אורתוגונאל/אניטרי 


    \item אופרטור \(T\)\(T\) משמרת נורמה: \(\forall v\in V\quad ||T(v)||=||v||\)


    \item מתקיים: 
$$T^* \circ T = Id \iff T^* =T^{-1} \iff T\circ T^* = Id$$


    \item משמר יחידה: 
$$\forall v\in V\quad ||v||=1 \Rightarrow ||Tv||=1$$


  \end{enumerate}
\end{proposition}
\begin{proposition}[תכונות של אופרטור אוניטרי]
  \begin{enumerate}
    \item מתקיים: 
$$v\perp w \iff T(v)\perp T(w)$$


    \item ההעתקה \(T\) חח"ע. 


    \item משמר נורמה - \(||T(v)|| = ||v||\). 


    \item אם \(\lambda\) ע"ע של \(T\) אז \(|\lambda|=1\). 


    \item ו"ע עם ע"ע שונים הם ניצבים. 


  \end{enumerate}
\end{proposition}
\section{המשפט הספקטרלי}

\begin{theorem}[שור]
כל מטריצה מרוכבת ניתנת להצגה כמטריצה משולשית אוניטרית.

\end{theorem}
\begin{proof}
ראינו כי כל מטריצה מרוכבת ניתנת להצגה כמטריצה משולשית. ולכן בעזרת גראם שמידט ניתן למצוא בסיס אשר משמר את הפרוש ויהיה אורתוגונאלי. כיוון שמשמר את הפרוש של כל רכיב נמשיך לקבל כי הרכיב הראשון שייך לפרוש של עצמו, הרכיב השני שייך לפרוש של הרכיב הראשון ושל עצמו, וכן ראינו כי תכונה זו מאפיינת את זה שבסיס הוא בסיס משלשי.

\end{proof}
\begin{corollary}
לכל \(A \in M_{n \times n}\left( \mathbb{C} \right)\)  קיימת \(U\) אוניטרית כך ש-\(U^{*}AU\) משולשית.

\end{corollary}
\begin{theorem}[המשפט הסקטראלי הממשי]
  \begin{enumerate}
    \item אופרטור \(T\) על ממ"פ ממשית הוא לכסין אותונורמלי אם"ם הוא צמוד עצמית. 


    \item עבור מטריצה \(A\in M_{n\times n}\left( \mathbb R \right)\) היא לכסינה אותונורמלית אם"ם \(A^t = A\). 


  \end{enumerate}
\end{theorem}
\begin{proof}
ממשפט שור קיים בסיס עבורו משולשית עם מטריצת מעבר אוטוגונאלית. כיוון שהאופרטור צמוד לעצמו, נקבל כי תחת הממשיים סימטרי תחת כל בסיס, ובפרט תחת הבסיס המשולשי. מטריצה סימטרית משולשית תהיה אלכסונית.

\end{proof}
\begin{definition}[מטריצה נורמלית]
מטריצה שמקיימת \(A^* A = A A^*\).

\end{definition}
\begin{definition}[אופרטור נורמלי]
העתקה המקיימת \(T^*T = TT^*\).

\end{definition}
\begin{proposition}
אופרטור הוא נורמלי אם"ם לכל \(v \in V\) מתקיים:
$$\lVert Tv \rVert =\lVert T^{*}v \rVert $$

\end{proposition}
\begin{proof}
מתקיים:
$$\left\vert\left\vert T x\right\vert\right\vert^{2}=\left\langle T^{*}T x,x\right\rangle=\left\langle T T^{*}x,x\right\rangle=\left\vert\left\vert T^{*}x\right\vert\right\vert^{2}$$
כעת בעזרת פולריזציה נקבל:
\begin{gather*}\left( N^{*}N\mathbf{x},\mathbf{y} \right)=\left( N\mathbf{x},N\mathbf{y} \right)=\frac{1}{4}\sum_{\alpha=\pm1,\pm i}\alpha\|N\mathbf{x}+\alpha N\mathbf{y}\|^{2}=\\=\frac{1}{4}\sum_{\alpha=\pm1,\pm i}\alpha\|N\left( \mathbf{x}+\alpha\mathbf{y} \right)\|^{2}=\frac{1}{4}\sum_{\alpha=\pm1,\pm i}\alpha\|N^{*}\left( \mathbf{x}+\alpha\mathbf{y} \right)\|^{2}\\=\frac{1}{4}\sum_{\alpha=\pm1,\pm i}\alpha\|N^{*}\mathbf{x}+\alpha N^{*}\mathbf{y}\|^{2}=\left( N^{*}\mathbf{x},N^{*}\mathbf{y} \right)=\left( NN^{*}\mathbf{x},\mathbf{y} \right) 
\end{gather*}
וכיוון שמתקיים לכל \(x,y \in V\) נקבל כי \(N^{*}N=N N^{*}\).

\end{proof}
\begin{proposition}
  \begin{enumerate}
    \item צמודה עצמית גוררת נורמליות. 


    \item אורתוגונאליות/אוניטריות גוררת נורמליות. 


  \end{enumerate}
\end{proposition}
\begin{theorem}[הספקטרלי המרוכב]
  \begin{enumerate}
    \item אופרטור \(T\) על ממ"פ מרוכב היא לכסינה אותונורמלית אם"ם הוא נורמלי. 


    \item עבור מטריצה \(A\in M(\mathbb C)\) היא לכסינה אותונורמלית מעל \(\mathbb R\) אם"ם היא נורמלית, כלומר \(A^* A = A A^*\). 


  \end{enumerate}
\end{theorem}
\begin{proof}
ממשפט שור קיימת מטריצה משולשית עם מטריצת מעבר אורתונורמלית. כמו במקרה הממשי, מספיק להראות כי מטריצה מושלישית כזו תהיה סימטרית אם"ם היא נורמלית. נוכיח בעזרת אינדוקציה על מימד המטריצה. המקרה הבסיס טריוויאלי(מטריצה \(1\times1\) היא לכסינה). כעת ניתן לכתוב:
$$N=\left(\begin{array}{c|ccc} a_{1,1} & a_{1,2}  & \dots & a_{1,n} \\\hline 0 &  &  &  \\\vdots &  & N_{1} &  \\0 &  &  & 
\end{array}\right)$$
ניתן כעת לחשב את המכפלות \(N^{*}N\) ו-\(N N^{*}\) ישירות בעזרת מכפלת בלוקים:
$$(N^{*}N)_{1,1}=\overline{{{a}}}_{1,1}a_{1,1}=|a_{1,1}|^{2}$$$$(N N^{*})_{1,1}=|a_{1,1}|^{2}+|a_{1,2}|^{2}+\ldots+|a_{1,n}|^{2}$$
ולכן מתקיים \((N^{*}N)_{1,1}=(N N^{*})_{1,1}\) אם"ם מתקיים:
$$a_{1,2}=\ldots=a_{1,n}=0$$
כלומר ניתן להציג את המטריצה בצורה:
$$N=\left(\begin{array}{c|ccc} a_{1,1} & 0   & \dots & 0 \\\hline 0 &  &  &  \\\vdots &  & N_{1} &  \\0 &  &  & 
\end{array}\right)$$
וכיוון שזוהי מטריצת בלוקים אלכסונית ניתן לחשב בקלות את המכפלות הבאות:
$$NN^{*}=\left(\begin{array}{c|ccc} |a_{1,1}|^{2} & 0  & \dots & 0 \\\hline 0 &  &  &  \\\vdots &  & N_{1}^{*}N_{1} &  \\0 &  &  & \end{array}\right)\qquad N^{*}N=\left(\begin{array}{c|ccc} |a_{1,1}|^{2} & 0  & \dots & 0 \\\hline 0 &  &  &  \\\vdots &  & N_{1}N_{1}^{*} &  \\0 &  &  & 
\end{array}\right)$$
כאשר ניתן לראות כי כיוון שאנו דורשים \(N^{*}N=N N^{*}\) אז מתקיים \(N_{1}^{*}N_{1}=N_{1} N_{1}^{*}\) ולכן \(N_{1}\) נורמלי, ומהנחת האינדוקציה נקבל כי לכסינה ולכן \(N\) לכסינה ובפרט הצורה המשולשית תהיה אלכסונית ולכסינה אורתונורמלית.

\end{proof}
\section{מטריצות חיוביות}

\begin{definition}[מטריצה חיובית]
כל הערכים עצמיים מקיימים \(\lambda >0\)

\end{definition}
\begin{definition}[מטריצה חיובית למחצה]
כל הערכים העצמיים מקיימים \(\lambda \geq 0\)

\end{definition}
\begin{proposition}[תנאים שקולים למטריצה חיובית]
עבור מטריצות \underline{סימטריות} S התנאים הבאים שקולים:

  \begin{enumerate}
    \item כל הערכים העצמיים חיוביים. 


    \item כל הערכים הקשורים בצורה המדורגת יהיו חיוביים. 


    \item כל הדטרמיננטות של בלוקים עליוניים יהיו חיוביים. 


    \item קיים \(A\) כך ש-\(S=A^{ T}A\) כאשר העמודות של \(A\) בת"ל. 


    \item לכל \(x \in V\) מתקיים \(x^{T}Sx > 0\) פרט עבור \(x=0\). 


  \end{enumerate}
\end{proposition}
\begin{proposition}[תנאים שקולים למטריצה חיובית בהחלט]
עבור מטריצות \underline{סימטריות} S התנאים הבאים שקולים:

  \begin{enumerate}
    \item כל הערכים העצמיים אי שליליים 


    \item כל הערכים הקשורים בצורה המדורגת יהיו אי שליליים. 


    \item כל הדטרמיננטות של בלוקים עליוניים יהיו אי שליליים. 


    \item קיים \(A\) כך ש-\(S=A^{ T}A\). 


    \item לכל \(x \in V\) מתקיים \(x^{T}Sx \geq 0\). 


  \end{enumerate}
\end{proposition}
\begin{definition}[אופרטור חיובי]
אופרטור \(T:V\to V\) נקרא חיובי אם \(T\) צמוד לעצמו ומקיים לכל \(v \in V\):
$$\langle Tv,v \rangle \geq 0$$

\end{definition}
\begin{remark}
כאשר אם \(V\) הוא מרחב ווקטורי תחת המרוכבים לא צריך לדרוש ש-\(T\) צמוד לעצמו כיוון שזה נובע מהאי שיוויון.

\end{remark}
\begin{example}
  \begin{itemize}
    \item אם \(U\) הוא תת מרחב של \(V\) אז ההטלה האורתוגונאלית \(P_{U}\) הוא אופרטור חיובי.
    \item אם \(T\) הוא צמוד לעצמו כך ו-\(b,c \in \mathbb{R}\) כך ש-\(b^{2}\leq 4c\) אזי:
$$T^{2}+bT+cI$$
הוא אופרטור חיובי.
  \end{itemize}
\end{example}
\begin{definition}[שורש של אופרטור]
אופרטור \(R\) יהיה השורש של אופרטור \(T\) אם \(R^{2}=T\).

\end{definition}
\begin{example}
אם \(T:\mathbb{F} ^{3}\to \mathbb{F} ^{3}\) מוגדר על ידי:
$$T(z_{1},z_{2},z_{3})=(z_{3},0,0)$$
אזי השורש של \(T\) יהיה \(R:\mathbb{F} ^{3}\to \mathbb{F} ^{3}\) המוגדר על ידי:
$$R(z_{1},z_{2},z_{3})=(z_{2},z_{3},0)$$

\end{example}
\begin{example}
$$T=\begin{pmatrix}36 & 0 & 0 \\0 & 25 & 0 \\0 & 0 & 16\end{pmatrix}\implies R=\begin{pmatrix}6 & 0 & 0 \\0 & 5 & 0 \\0 & 0 & 4
\end{pmatrix}=\sqrt{ T }$$

\end{example}
\begin{proposition}
הביטיים הבאים שקולים

  \begin{enumerate}
    \item האופרטור \(T\) הוא חיובי. 


    \item האופרטור \(T\) צמוד לעצמו וכל הערכים העצמים שלו הם אי שליליים 


    \item לאופרטור \(T\)  יש שורש חיובי(כלומר האופרטור המקיים \(R^{2}=T\) הוא אופרטור חיובי) 


    \item לאופרטור \(T\) יש שורש צמוד לעצמו 


    \item קיים אופרטור \(R\) כך ש-\(T=R^{*}R\). 


  \end{enumerate}
\end{proposition}
\begin{proof}
  \begin{enumerate}
    \item נניח כי \(T\) אופרטור חיובי. מההגדרה \(T\) יהיה צמוד לעצמו. נניח \(\lambda\) ערך עצמי של \(T\) עם ווקטור עצמי מתאים \(v\). מתקיים: 
$$0\leq\langle Tv,v \rangle =\left\langle  \lambda v,v  \right\rangle =\lambda \langle v,v \rangle \implies \lambda \geq 0$$


    \item כעת נניח כי \(T\) צמוד לעצמו וכל הערכים העצמיים אי שליליים. לפי המשפט הספקטרלי קיים בסיס אורתונורמלי \(e_{1},\dots,e_{n}\) של \(V\) כך ווקטורים עצמיים של \(T\). יהיו \(\lambda_{1},\dots,\lambda_{n}\) הערכים העצמים המתאימים. מההנחה \(\lambda_{j}\geq 0\). נגדיר את האופרטור \(R\) לפי איך שפועל על איברי הבסיס: 
$$R (e_{j})=\sqrt{ \lambda_{j} }e_{j}$$
ניתן להראות כי \(R\) אופרטור חיובי. בנוסף \(R^{2}e_{j}=\lambda_{j}e_{j}=Te_{j}\) ולכן \(R^{2}=T\).


    \item נניח כי ל-\(T\) יש שורש חיובי. אזי ל-\(T\) יש שורש שצמוד לעצמו כיוון שחיוביות דורש שיהיה צמוד לעצמו. 


    \item נניח כי ל-\(T\) יש שורש צמוד לעצמו. כלומר קיים אופרטור צמוד לעצמו כך ש: 
$$T=R^{2}=R^{*}R\implies R=R ^{*}R$$


    \item נניח כי קיים אופרטור \(T\) כך ש-\(T=R^{*}R\). האופרטור \(T\) צמוד לעצמו כי: 
$$T^{*}=(R^{*}R)^{*}=R^{*}(R^{*})^{*}=R^{*}R=T$$
כעת מתקיים לכל \(v \in V\):
$$\langle Tv,v \rangle =\langle R^{*}Rv,v \rangle =\langle Rv,Rv \rangle \geq 0$$
ולכן \(T\) אופרטור חיובי.


  \end{enumerate}
\end{proof}
\begin{proposition}
לכל אופרטור חיובי קיים שורש חיובי יחיד.

\end{proposition}
\begin{remark}
ייתכן עדיין אינסוף שורשים שאינם חיוביים. לדוגמא לאופרטור היחידה יש אינסוף שורשים אם מימד המרחב גדול מ-1.

\end{remark}
\section{פירוק פולארי}

\begin{reminder}
ניתן לכתוב כל מספר מכוב בצורה הבאה:
$$z=\left( \frac{z}{\lvert z \rvert } \right)\lvert z \rvert = \left( \frac{z}{\lvert z \rvert } \right)\sqrt{ \overline{z} z }$$

\end{reminder}
נראה את האנלוגיה בין המספרים המרוכבים לאופרטורים בעזרת טבלה.

\begin{table}[htbp]
  \centering
  \begin{tabular}{|cc|}
    \hline
    \(\mathbb{C}\) & \(\mathrm{End}(V)\) \\ \hline
    \(z\) & \(T\) \\ \hline
    \(z^{*}\) & \(T^{*}\) \\ \hline
    \(z \in \mathbb{R}\left( z=\overline{z} \right)\) & האופרטור \(T\) צמוד לעצמו(\(T=T^{*}\)) \\ \hline
    \(z=\overline{w}w\) & \(T=R^{*}R\) \\ \hline
    \(\lvert z \rvert=1\) & האופרטור \(T\) הוא איזומטרייה \\ \hline
    $$z = \left( \frac{z}{\lvert z \rvert } \right)\sqrt{ \overline{z} z }$$ & \(T=S\sqrt{ T^{*}T }\) עבור איזומטריה \(S\) \\ \hline
  \end{tabular}
\end{table}
\begin{theorem}[פירוק פולארי]
נניח \(T \in \mathrm{End}(V)\) אזי קיים איזומטרייה \(S \in \mathrm{End}(V)\) כך ש:
$$T=S\sqrt{ T^{*}T }$$

\end{theorem}
\begin{proof}
נניח \(v \in V\). מתקיים:
$$\left\|T v \right\|^{2}=\left\langle T v ,T v \right\rangle =\left\langle T^{\ast}T v , v \right\rangle=\left\langle\sqrt{T^{\ast}T}\sqrt{T^{\ast}T} v , v \right\rangle =\left\langle\sqrt{T^{\ast}T} v ,\sqrt{T^{\ast}T} v \right\rangle=\left\|\sqrt{T^{\ast}T} v \right\|^{2}$$
ולכן \(\lVert Tv \rVert=\left\lVert  \sqrt{ T^{*}T }  \right\rVert\). כעת נגדיר \(S:\mathrm{Im}\left( \sqrt{ T^{*}T } \right)\to \mathrm{Im}(T)\) על ידי:
$$S\left( \sqrt{ T^{*}T } v\right)=Tv$$
כאשר כיוון שמתקיים \(\left\lVert  \sqrt{ T^{*}T }v  \right\rVert=\lVert Tv \rVert\) נקבל לכל \(u \in \mathrm{Im}\left( \sqrt{ T^{*}T } \right)\) כי מתקיים:
$$\lVert Su \rVert =\lVert u \rVert $$
אם התמונה של \(\sqrt{ T^{*}T }\) אינה שווה ל-\(V\) ניתן להרחב את \(S\) לאיזומטריה.

\end{proof}
\section{פירוק SVD ופירוק שמידט}

\begin{proposition}
המטריצות \(A^{*}A\) ו-\(AA^{*}\) תמיד ריבועיות ומייצגות אופרטור חיוביות.

\end{proposition}
\begin{proposition}
הערכים העצמיים של \(AA^{T}\) ושל \(A^{T}A\) הם לא שליליים, ושווים לריבוע של הערכים עצמיים של \(A\).

\end{proposition}
\begin{proof}
אם \(A^{T}Ax=\lambda x\) אז:
$$x^{T}A^{T}Ax=\lambda x^{T}x\implies \lVert Ax \rVert ^{2}=\lambda \lVert x \rVert ^{2}\implies \lambda \geq 0$$

\end{proof}
\begin{definition}[ערך סינגולרי]
יהי \(T\) אופרטור לינארי. הערכים הסינגולרים יהיו הערכים העצמיים של \(\sqrt{  T^{*}T }\) כאשר כל ערך עצמי \(\lambda\) מופיע
$$\dim \ker \left( \sqrt{ T^{*}T} -\lambda I\right)$$
פעמים.

\end{definition}
\begin{corollary}
מהמשפט הסקטרלי לכל אופרטור על מרחב ממימד \(n\) יש \(n\) ערכים סינגולאריים.

\end{corollary}
\begin{example}
נגדיר אופרטור על ידי:
$$T(z_{1},z_{2},z_{3},z_{4})=(0,3z_{1},2z_{2},-3z_{4})$$
נקבל:
$$T^{*}T(z_{1},z_{2},z_{3},z_{4})=(9z_{1},4z_{2},0,9z_{4})$$
וכעת:
$$\sqrt{T^{*}T}(z_{1},z_{2},z_{3},z_{4})=(3z_{1},2z_{2},0,3z_{4})$$
ולכן הערכים עצמיים של \(\sqrt{ T^{*}T }\) הם \(3,2,0\) כאשר מתקיים:
$$\dim\ker\left( \sqrt{T^{*}T}-3I \right)=2\qquad \dim\ker\left( \sqrt{T^{*}T}-2I \right)=1\qquad \dim\ker\sqrt{T^{*}T}=1$$
ולכן הערכים הסינגולארים יהיו \(3,3,2, 0\)

\end{example}
\begin{theorem}[הפירוק הספקטרלי]
נניח \(T\) אופרטור לינארי עם ערכים סינגולאריים \(s_{1},\dots,s_{n}\) אזי קיים בסיס אורתונואמלי \(\mathcal{B}=\{ e_{1},\dots,e_{n} \}\) ו-\(\mathcal{C}=\{ f_{1},\dots,f_{n} \}\) כך שלכל \(v \in V\) מתקיים:
$$T v=s_{1}\langle v,e_{1}\rangle f_{1}+\cdot\cdot\cdot+s_{n}\langle v,e_{n}\rangle f_{n}$$
כלומר נקבל כי מטריצת המעבר בסיס תהיה:
$$[T]_{\mathcal{B} }^{\mathcal{C} }=\begin{pmatrix}s_{1} &  & 0 \\ & \ddots &  \\0 &  & s_{n}
\end{pmatrix}$$

\end{theorem}
\begin{proof}
נשתמש במשפט הספקטרלי על \(\sqrt{ T^{*}T }\) כיוון שזהו אופרטור צמוד לעצמו. נקבל בסיס \(e_{1},\dots,e_{n}\) כך ש-\(\sqrt{ T^{*}T }e_{j}=s_{j}e_{j}\) לכל \(1\leq j\leq n\). כעת לכל \(v \in V\) מתקיים:
$$v=\langle v,e_{1}\rangle e_{1}+\cdots+\langle v,e_{n}\rangle e_{n}$$
כאשר אם נפעיל את \(\sqrt{ T^{*}T }\) על שתי האגפים נקבל לכל \(v \in V\):
$${\sqrt{T^{*}T}}v=s_{1}\langle v,e_{1}\rangle e_{1}+\cdots+s_{n}\langle v,e_{n}\rangle e_{n}$$
כאשר לפי משפט הפירוק הפולארי קיימת איזומטריה \(S\) כך ש-\(T=S\sqrt{ T^{*}T }\) ולכן אם נפעיל את \(S\) זה על שתי האגפים נקבל לכל \(v \in V\):
$$T v=s_{1}\langle v,e_{1}\rangle S e_{1}+\cdots+s_{n}\langle v,e_{n}\rangle S e_{n}$$
כעת ניתן לסמן \(f_{j}=Se_{j}\). כיוון ש-\(S\) איזומטרייה נקבל כי \(f_{1},\dots,f_{n}\) בסיס אורתונורמלי. כעת ניתן לכתוב:
$$T\nu=s_{1}\langle\nu,e_{1}\rangle f_{1}+\cdot\cdot\cdot+s_{n}\langle\nu,e_{n}\rangle f_{n}$$\textbf{טענה}
הערכים הסינגולארים יהיו השורשים של הערכים העצמיים של \(T^{*}T\) כאשר כל ערך עצמי מופיע:
$$\dim\ker(T^{*}T-\lambda I)$$
פעמים.

\end{proof}
\begin{example}
נגדיר \(T:\mathbb{F} ^{4}\to \mathbb{F} ^{4}\) על ידי:
$$T(z_{1},z_{2},z_{3},z_{4})=(0,3z_{1},2z_{2},-3z_{4})$$
כעת:
$$T^{*}T(z_{1},z_{2},z_{3},z_{4})=(9z_{1},4z_{2},0,9z_{4})$$
כאשר מתקיים:
$$\ker(T^{*}T-9I)=2 \qquad \ker(T^{*}T-4I)=1 \qquad \ker\,T^{*}T=1$$
ניקח כעת את השורש ונקבל כי הערכים הסינגולארים יהיו \(3,3,2,0\) כמו שראינו מקודם.

\end{example}
\begin{proposition}
אם \(T\) אופרטור צמוד לעצמו אז הערכים העצמיים שווים לערכים הסינגולאריים, כאשר מספר החזרות שווה לריבוי המתאים.

\end{proposition}
\begin{proposition}
לאופרטור \(T\) ולאופרטור הצמוד \(T^{*}\) יש אותם ערכים סינגולאריים.

\end{proposition}
\begin{proposition}
ל-\(T\) יש את הערך הסינגולאי 0 אם"ם \(T\) לא הפיך.

\end{proposition}
\begin{proposition}
הדרגה של האופרטור שווה למספר הערכים הסינגולאריים שאינם אפס.

\end{proposition}
\begin{proposition}
אופרטור \(S\) הוא איזומטרייה אם"ם כל הערכים הסינגולריים הם 1.

\end{proposition}
\begin{proposition}
אם \(\hat{s}\) הערך הסינגולארי הקטן ביותר של \(T\) ו-\(s\) הוא הערך הסינגולארי הגדול ביותר אל \(T\) אזי מתקיים לכל \(v \in V\):
$$\hat{s}\lVert v \rVert \leq \lVert Tv \rVert \leq s\lVert v \rVert $$
וכן אם \(\lambda\) הוא ערך עצמי של \(T\) אזי:
$$\hat{s}\leq \left\lvert  \lambda  \right\rvert \leq s$$

\end{proposition}
\begin{proposition}
אם לאופרטור \(T\) יש את הפרוק הסינגולארי:
$$T v=s_{1}\langle v,e_{1}\rangle f_{1}+\cdots+s_{n}\langle v,e_{n}\rangle f_{n}$$
כאשר \(e_{1},\dots,e_{n}\) ו-\(f_{1},\dots,f_{n}\) בסיסים אורתונורמליים אזי הפירוק לערכים סינגולרים של \(T^{*}v,T^{*}Tv,\sqrt{ T^{*}T }v,T^{-1}v\): יהיה:
$$T^{*}v=s_{1}\langle v,f_{1}\rangle e_{1}+\cdots+s_{n}\langle v,f_{n}\rangle e_{n}$$$$T^{*}T v={s_{1}}^{2}\langle v,e_{1}\rangle e_{1}+\cdots+{s_{n}}^{2}\langle v,e_{n}\rangle e_{n}$$$${\sqrt{T^{*}T}}v=s_{1}\langle v,e_{1}\rangle e_{1}+\cdots+s_{n}\langle v,e_{n}\rangle e_{n}$$$$T^{-1}\nu=\frac{\langle\nu,f_{1}\rangle e_{1}}{s_{1}}+\cdot\cdot\cdot+\frac{\langle\nu,f_{n}\rangle e_{n}}{s_{n}}$$

\end{proposition}
\Chapter{מרחבים דואלי וטנזורים}

\section{הדואלי של מרחב מכפלה פנימי}

\begin{theorem}[ההצגה של ריס]
יהי \(\left( V,\langle  \rangle \right)\) מרחב מכפלה פנימי, ו-\(L\) פונקציונאל לינארי על \(V\). אזי קיים ווקטור יחיד \(u \in V\) אשר מקיים:
$$\forall v \in V\quad Lv = \langle v,u \rangle $$

\end{theorem}
\section{המרחב הדואלי}

\begin{definition}[פונקציונאל לינארי]
יהי \(V\) מרחב ווקטורי. פונקציונאל לינארי תהיה טרנספומציה לינארית \(L:V\to \mathbb{F}\).

\end{definition}
\begin{definition}[המרחב הדואלי של מרחב ווקטורי]
יהי \(V\) מרחב ווקטורי. קבוצת כל הפונקציואליים הלינארים \(L:V\to \mathbb{F}\) יוצר מרחב ווקטורי אשר נקרא המרחב הדואלי. מרחב זה מסומן ב-\(V^{*}\). לעיתים קוראים לווקטורים מהמרחב הדואלי קו-ווקטורים.

\end{definition}
\begin{example}
אם נסתכל על המרחב \(\mathbb{F} ^{n}\) בתור אוסף של ווקטורים מגודל \(n\), אז המרחב הדואלי ייוצג על ידי הווקטורי שורה המתאימים. ולכן הרבה פעמים נסמן גם את הווקטור או המטריצה תחת בסיס ע"י \([L]_{\mathcal{A}}^{T}\) כדי להדגיש שזה ווקטור שורה.

\end{example}
\begin{proposition}
למרחב הווקטורי \(V\) ולמרחב הדואלי המתאים \(V^{*}\) יש את אותו מימד.

\end{proposition}
\begin{proof}
מתקיים:
$$\dim  V^{*}=\dim \left( \mathrm{Hom}\left( V,\mathbb{F}   \right) \right)=\left( \dim V \right)\left( \dim \mathbb{F}   \right)=\left( \dim  V \right)\cdot 1 = \dim V$$

\end{proof}
\begin{corollary}
המרחב \(V\) והמרחב הדואלי המתאים \(V^{*}\) הם איזומורפיים.

\end{corollary}
\begin{remark}
לעיתים אנשים חושבים שאיזומורפיזם זה למעשה אומר "שהמרחבים אותו דבר". אך יש הבדל בין שיוויון לבין איזומורפיזם, כמו שנראה בטענות הבאות.

\end{remark}
\begin{proposition}[שינוי בסיס]
נגדיר את הבסיסים הבאים:
$${\mathcal{A}}=\{{\mathbf{a}}_{1},{\mathbf{a}}_{2},\ldots,{\mathbf{a}}_{n}\},\qquad{\mathcal{B}}=\{{\mathbf{b}}_{1},{\mathbf{b}}_{2},\ldots,{\mathbf{b}}_{n}\}$$
אזי ניתן להעביר את הווקטור \([L]^{T}_{\mathcal{B}}\in V^{*}\) לבסיס \([L]_{\mathcal{A}}^{T}\) על ידי הכלל:
$$[L]_{B}^{T}=(S^{-1})^{T}[L]_{A}^{T}$$

\end{proposition}
\begin{corollary}
אם \(S\) היא מטריצת המעבר בסיס של \(V\), אזי \((S^{-1})^{T}\) היא מטריצת המעבר המתאימה במרחב \(V^{*}\).

\end{corollary}
\begin{proposition}[האיזומורפיזם הטבעי של המרחב הדואלי הכפול]
קיים איזומורפיזם טבעי בין המרחב \(V\) למרחב הדואלי \((V^{*})^{*}\) המוגדר על ידי \(Tv=L_{v}\) כאשר \(L_{v}\) מוגדר על ידי:
$$\forall v \in V^{*}\quad L_{v}(f)=f(v)$$
זהו איזומורפיזם אשר אינו תלוי בבסיס, ולכן איזומורפיזם חזק יותר מאיזומורפזים רגיל.

\end{proposition}
\begin{definition}[בסיס דואלי]
יהי \(e_{1},\dots,e_{n}\) בסיס של \(V\). אזי \(e_{1}',\dots,e'_{n}\in V^{*}\) יהיה פסיס של \(V^{*}\) אם מקיים:
$$e'_{k}(e_{k})=\delta_{k,j}=\begin{cases}1 & j=k \\0 & j\neq k
\end{cases}$$

\end{definition}
ניתן להראות כי הגדרה זו קונסיסטנטי עם התכונות שאחנו מכירים על בסיס של מרחב ווקטורי. 

\begin{proposition}
כל פונקציונאל לינארי ניתן לביטוי על ידי \(L=\sum L_{k}e_{k}'\).

\end{proposition}
\begin{proof}
ניקח ווקטור שרירותי $$\mathbf{v}\,=\,\sum_{k}\alpha_{k}\mathbf{b}_{k}\,\in V$$
כאשר מתקיים:
$$\mathbf{b}_{k}^{\prime}(\mathbf{v})=\mathbf{e}_{k}^{\prime}\left(\sum_{j}\alpha_{j}\mathbf{e}_{j}\right)=\sum_{j}\alpha_{j}\mathbf{e}_{k}^{\prime}(\mathbf{e}_{j})=\alpha_{k}$$
ולכן:
$$L{\bf v}=[L]_{\mathcal{B}}[{\bf v}]_{\mathcal{B}}=\sum_{k}L_{k}\alpha_{k}=\sum_{k}L_{k}{\bf e}_{k}^{\prime}({\bf v})$$\textbf{טענה}
בסיס דואלי יהיה קבוצה בלתי תלויה לינארית.

\end{proof}
\begin{proof}
ניקח צירוף לינארי מאפס \(0=\sum L_{k}e'_{k}\). אזי מתקיים:
$$0={\bf0e}_{j}=\left(\sum_{k}L_{k}{\bf e}_{k}^{\prime}\right)({\bf e}_{j})=\sum_{k}L_{k}{\bf e}_{k}^{\prime}({\mathbf{}e}_{j})=L_{j}$$
ולכן \(L_{j}=0\) ולכן \(L_{k}=0\) ובתל. 

\end{proof}
\begin{definition}[העתקה דואלית]
אם \(T \in \mathrm{Hom}(V,W)\) אז ההעתקה הדואלית \(T^{*}\in \mathrm{Hom}(W^{*},V^{*})\) תהיה מוגדרת על ידי:
$$T^{*}\left( \varphi \right)=\varphi \circ  T$$
עבור \(\varphi \in W^{*}\).

\end{definition}
\begin{example}
נגדיר \(D:\mathbb{R}[x]\to \mathbb{R}[x]\) המוגדר על ידי אופרטור הגזירה \(Dp=\frac{\mathrm{d} p}{\mathrm{d} x}\).

  \begin{enumerate}
    \item עבור \(\varphi \in \left( \mathbb{R}[x] \right)^{*}\) המוגדר על ידי: 
$$\varphi(p)=p(3)$$
אז נקבל כי \(D^{*}\left( \varphi \right)\) נתון על ידי:
$$\left( D^{*}\left( \varphi \right) \right)(p)=\left( \varphi \circ  D \right)(p)=\varphi(Dp)=\varphi\left( \frac{\mathrm{d} p}{\mathrm{d} x}  \right)=\frac{dp}{dx}|_{x=3}$$
כלומר זהו הפונקציונאל שמקבל פולינום ומחזיר את הערך של הנגזרת שלו ב-\(x=3\).


    \item עבור \(\varphi \in \left( \mathbb{R}[x] \right)^{*}\) המוגדר על ידי: 
$$\varphi(p)=\int_{0}^{1}p\;\mathrm{d}x$$
נקבל כי \(D^{*}( \varphi )\) נתון על ידי:
$$\left( D^{*}\left( \varphi \right) \right)(p)=\left( \varphi \circ  D \right)(p)=\varphi(Dp)=\varphi\left( \frac{\mathrm{d} p}{\mathrm{d} x}  \right)=\int_{0}^{1}\frac{\mathrm{d} p}{\mathrm{d} x} \;\mathrm{d}x=p(1)-p(0)$$


  \end{enumerate}
\end{example}
\begin{proposition}
להעתקה הדואלית יש את התכונות הבאות:

  \begin{enumerate}
    \item לכל \(S,T \in \mathrm{Hom}(V,W)\) מתקיים \((S+T)^{*}=S^{*}+T^{*}\). 


    \item לכל \(\lambda \in \mathbb{F}\) ולכל \(T \in \mathrm{Hom}(V,W)\) נקבל כי \(\left( \lambda T \right)^{*}=\lambda T^{*}\). 


    \item לכל \(T \in \mathrm{Hom}(U,V)\) ו-\(S \in \mathrm{Hom(V,W)}\) מתקיים \((ST)^{*}=T^{*}S^{*}\). 


  \end{enumerate}
\end{proposition}
\section{אניהלטור}

\begin{definition}[אניהלטור]
עבור תת קבוצה \(U\subseteq V\) האניהלטור(annihilator) שלו יהיה קבוצת כל הפונקציונאלים הלינארים אשר מאפסים את כל האיברים בקבוצה. כלומר:
$$U^{0}=\{\varphi\in V^*:\varphi(u)=0{\mathrm{~for~all~}}u\in U\}$$

\end{definition}
\begin{example}
$$\{ 0 \}^{0}=V\quad \quad V^{0}=\{ 0 \}$$

\end{example}
\begin{proposition}
אם \(U\subseteq V\) אז \(U^{0}\) הוא תת מרחב של \(V^{*}\).

\end{proposition}
\begin{proposition}
אם \(V\) מרחב ווקטורי נוצר סופית ו-\(U\) תת מרחב של \(V\) אזי:
$$\dim U+\dim U^{0}=\dim V$$

\end{proposition}
\begin{proposition}
נניח \(V,W\) מרחבים ווקטורים ו-\(T \in \mathrm{Hom}(V,W)\) אזי מתקיים:
\begin{gather*}\ker (T^{*})=\left( \mathrm{Im}\;T \right)^{*} \\\mathrm{Im}(T^{*})=\left( \ker T \right)^{*}
\end{gather*}

\end{proposition}
\begin{proposition}
אם \(V,W\) מרחבים ווקטורים נוצרים סופית ו-\(T \in \mathrm{Hom}(V,W)\) אזי:

  \begin{enumerate}
    \item האופרטור \(T\) על אם"ם \(T^{*}\) חח"ע. 


    \item האופרטור \(T\) חח"ע אם"ם \(T^{*}\) על. 


  \end{enumerate}
\end{proposition}
\begin{proposition}
אם \(V,W\) מרחבים ווקטורים נוצרים סופית ו-\(T \in \mathrm{ Hom}(V,W)\) אזי מתקיים:
$$\mathrm{rk(T^{*})}=\mathrm{rk}(T)$$

\end{proposition}
\begin{proof}
\begin{gather*}\mathrm{rk}(T^{*})=\dim \left( \mathrm{Im}(T^{*}) \right)=\dim  W^{*}-\dim  \ker (T^{*}) = \\=\dim  W - \dim \left( \mathrm{Im}(T) \right)^{*}=\dim \left( \mathrm{Im}(T) \right) 
\end{gather*}

\end{proof}
\begin{proposition}[הצגה מטריציונית]
ההצגה המטריציונית של האופרטור הדואלי יהיה השיחלוף של ההצגה המטריציונית של האופרטור. כלומר אם \(V,W\) מרחבים ווקטורים נוצרים סופית ו-\(T \in \mathrm{Hom}(V,W)\) אזי:
$$[T^{*}]=[T]^{t}$$
כאשר הבסיס המתאים יהיה הבסיס הדואלי של הבסיס של \(V,W\) שאיתו מיוצג המטריצה \([T]\).

\end{proposition}
\section{מכפלת קרונקר}

\begin{definition}[מכפלת קרונקר]
יהי \(A \in M_{k\times \ell}\left( \mathbb{F}  \right)\) ו-\(B \in M_{m\times n}\left( \mathbb{F}  \right)\) אזי נגדיר את הכפלת קרונקר בצורה הבאה:
$$A\otimes B={\left[\begin{array}{l l l}{A_{11}B}&{\ldots}&{A_{1L}B}\\ {\vdots}&{\ddots}&{\vdots}\\ {A_{k1}B}&{\ldots}&{A_{k\ell}B}\end{array}\right]}$$
כלומר המכפלה \(A\otimes B\) תהיה מטריצת בלוקים שהרכיב ה-\(k,\ell\) יהיה הרכיב ה-\(k,\ell\) של המטריצה \(A\) כפול המטריצה \(B\).

\end{definition}
\begin{example}
עבור המטריצות:
$$A=\begin{bmatrix}2 & 0 \\1 & 3\end{bmatrix}\qquad B=\begin{bmatrix}5 & -1 \\-1 & 4
\end{bmatrix}$$
נקבל:
$$A\otimes B=\begin{bmatrix}2B & 0B \\1B & 3B
\end{bmatrix}=\left[\begin{array}{c c c c}{{10}}&{{-2}}&{{0}}&{{0}}\\ {{-2}}&{{8}}&{{0}}&{{0}}\\ {{5}}&{{-1}}&{{15}}&{{-3}}\\  {{-1}}&{{4}}&{{-3}}&{{12}}\end{array}\right]$$

\end{example}
\begin{remark}
באופן כללי המכפלת קרונקר היא לא קומוטטיבית.

\end{remark}
\begin{proposition}[פילוגיות]
  \begin{enumerate}
    \item \((A+B)\otimes C=A\otimes C+B\otimes C\)


    \item \(A\otimes(B+C)=A\otimes B+A\otimes C\)


  \end{enumerate}
\end{proposition}
\begin{proof}
יהי \(A \in M_{k\times \ell}\left( \mathbb{F}  \right)\) ו-\(B \in M_{m\times n}\left( \mathbb{F}  \right)\) עבור 1 נקבל:
$${{(A+B)_{\mathrm{{kl}}}C=(A_{\mathrm{{kl}}}+B_{\mathrm{{kl}}})C}}{{=A_{\mathrm{{kl}}}C+B_{\mathrm{{kl}}}C}}$$
כאשר עבור 2 ניתן לקבל באותו אופן:
$$A_{k l}(B+C)=A_{k l}B+A_{k l}C$$

\end{proof}
\begin{proposition}[לינארית לכפל בסקלר]
מתקיים:
$$(\alpha A)\otimes(\beta B)=(\alpha\beta)(A\otimes B)$$

\end{proposition}
\begin{proof}
$$\begin{array}{c}{{(\alpha A)\otimes(\beta B)}} {{=\left[\begin{array}{c c c}{{(\alpha A_{11})(\beta B)}}&{{\ldots}}&{{(\alpha A_{1L})(\beta B)}}\\ {{\vdots}}&{{\ddots}}&{{\vdots}}\\ {{(\alpha A_{K1})(\beta B)}}&{{\ldots}}&{{(\alpha A_{K L})(\beta B)}}\end{array}\right]}}\\ {{=\alpha\beta\left[\begin{array}{c c c}{{A_{11}B}}&{{\ldots}}&{{A_{1L}B}}\\ {{\vdots}}&{{\ddots}}&{{\vdots}}\\ {{A_{K1}B}}&{{\ldots}}&{{A_{K L}B}}\end{array}\right]}}{{=(\alpha\beta)(A\otimes B)}}\end{array}$$

\end{proof}
\begin{proposition}
עקבה של מכפלת קרונקר מקיימת:
$$\mathrm{Tr}\left( A\otimes B \right)=\mathrm{Tr}(A)\cdot \mathrm{Tr}(B)$$

\end{proposition}
\begin{proof}
\end{proof}
\Chapter{פתרון נומרי}

\section{פתרון מקורב והמשפט היסודי}

\begin{theorem}[המשפט היסודי של אלגברה לינארית]
אם \(A\in M_{m\times n}\left( \mathbb{F}  \right)\) וכאשר \(\mathbb{F}\) הוא \(\mathbb{C}\) או \(\mathbb{R}\) אזי:
$$\left( \mathrm{Im}\;A \right)^{\perp} = \ker  A^{*}\qquad \left( \ker A \right)^{\perp}=\mathrm{Im}\;A^{*}$$

\end{theorem}
\begin{proof}
אם \(v \in \ker A^{*}\) ו-\(w \in \mathrm{Im}(A)\) אז יש \(u\) כך ש-\(w=Au\) ואז:
$$\langle w,v \rangle =\langle Au,v \rangle =(Au)^{*}v=u^{*}A^{*}v=0$$
כאשר קיבלנו:
$$\ker  A^{*}\subseteq \left( \mathrm{Im}(A) \right)^{\perp}$$
בנוסף:
$$\dim\left( \mathrm{Im}\;A \right)^{\perp}=m-\dim\left( \mathrm{Im}\;A \right)=m-\mathrm{rk}(A)=m-\mathrm{rk}(A^{*})=\dim \ker A^{*}$$

\end{proof}
\begin{remark}
נפעמים גם עבור המשפט היסודי מציינים את השתי משוואות המקביליות הנוספות:
$$\mathrm{Im}(A^{*})=\left( \ker A \right)^{\perp}\qquad \ker (A^{*})=\left( \mathrm{Im}\;A \right)^{\perp}$$

\end{remark}
\begin{remark}
ניתן לכתוב את המשפט היסודי גם בצורה של מטריצות ממשיות. במקרה זה נקבל:

  \begin{enumerate}
    \item \(\ker A^{T}A=\ker A\)


    \item \(\ker AA^{T}=\ker A^{T}\)


    \item \(\mathrm{Im}(A^{T} A)=\mathrm{Im}(A^{T})\)


    \item \(\mathrm{Im}(AA^{T})=\mathrm{Im}(A)\)


  \end{enumerate}
\end{remark}
כאשר התמונה של מטריצה תהיה המרחב העמודות, התמונה של \(A^{T}\) יהיה מרחב העמודות, התמונה של \(A\) תהיה מרחב השורות, והגרעין של \(A\) יהיה מרחב הפתרונות.

אחד השימושיים של המשפט היסודי זה מציאת פתרון מקורב:

\begin{definition}[פתרון מקורב]
יהי \(A \in M_{m\times n}\left( \mathbb{R} \right)\) מטריצה. אזי הפתרון המקורב של המערכת \(Ax=b\) יהיה הפתרון עבורו מתקיים:
$$d(Ax,b)=\min_{y \in \mathbb{R}} d(Ay,b)$$

\end{definition}
\begin{proposition}
מתקיים ש-\(x \in \mathbb{R}^{n}\) הוא פתרון מקורב אופטימלי אם"ם:
$$A^{t}A x = A^{t} b$$
(או צמוד במקום שחלוף תחת המרוכבים) כאשר למשוואה זו תמיד תהיה פתרון.

\end{proposition}
\begin{lemma}
$$\ker  (A^{*}A)=\ker A$$

\end{lemma}
\begin{proof}
אם \(Av=0\) אז \(A^{*}A=0\). מצד שני אם \(A^{*}Av=0\) אז:
$$\lVert Av \rVert =(Av)^{*}(Av)=v^{*}A^{*}Av=0\implies Av=0$$

\end{proof}
$$\mathrm{Im}(A^{*}A)=\mathrm{Im}(A)$$

כעת נוכיח את טענת הקירוב:

\begin{proof}
ראשית, מתקיים מהלמה:
$$\mathrm{Im}(A^{*}A)=\mathrm{Im}(A)\implies A^{*}b \in \mathrm{Im}(A^{*}A)$$
כלומר יש פתרון ל-\(A^{*}AxA^{*}b\). נסתכל על \(Ay\) האפשריים. אנחנו מרחפשים את הקרוב ביותר ל-\(b\). אנחנו רוצים:
$$Ax=P_{\mathrm{Im}(A)}(b)$$
כאשר \(P_{w}\) הוא הווקטור היחיד \(w \in W\) שמתקיים \(v-w\in W^{\perp}\) אם"ם \(b-Ax \in \left( \mathrm{Im}\;A \right)^{\perp}\). כלומר הסה"כ \(x\) הוא פתרון מקורב אם מתקיים:
$$A^{*}(b-Ax)=0\iff b-Ax \in \left( \mathrm{Im}\;A \right)^{\perp}=\ker A^{*}\iff A^{*}Ax=A^{*}b$$

\end{proof}
\begin{example}[התאמת פרבולה]
נניח כי יש לנו אוסף נקודות שאנחנו רוצים להתאים לפרבולה - כלומר להתאמה מהצורה \(y=a+bx+cx^{2}\). המשתנים שלנו צריכים לקיים את המשוואה הבאה לכל \(k=1,\dots, n\):
$$a+bx_{k}+cx^{2}_{k}=y_{k}$$
או בצורה מטריציונית:
$$\left(\begin{array}{c c c}{{1}}&{{x_{1}}}&{{x_{1}^{2}}}\\ {{1}}&{{x_{2}}}&{{x_{2}^{2}}}\\ {{\vdots}}&{{\vdots}}&{{\vdots}}\\ {{1}}&{{x_{n}}}&{{x_{n}^{2}}}\end{array}\right)\;\left(\begin{array}{c}{{a}}\\ {{b}}\\ {{c}}\end{array}\right)\;=\;\left(\begin{array}{c}{{y_{1}}}\\ {{y_{2}}}\\ {{\vdots}}\\ {{y_{n}}}\end{array}\right)$$
נניח שיש לנו את הערכים הבאים:
$${\left(\begin{array}{l l l}{1}&{-2}&{4}\\ {1}&{-1}&{1}\\ {1}&{0}&{0}\\ {1}&{2}&{4}\\ {1}&{3}&{9}\end{array}\right)}\ {\left(\begin{array}{l}{a}\\ {b}\\ {c}\end{array}\right)}={\left(\begin{array}{l}{4}\\ {2}\\ {1}\\ {1}\\ {1}\end{array}\right)}$$
אזי מתקיים:
$$A^{*}A=\left(\begin{array}{c c c c}{{1}}&{{1}}&{{1}}&{{1}}&{{1}}\\ {{-2}}&{{-1}}&{{0}}&{{2}}&{{3}}\\ {{4}}&{{1}}&{{0}}&{{4}}&{{9}}\end{array}\right)\left(\begin{array}{c c c c}{{1}}&{{-2}}&{{4}}\\ {{1}}&{{-1}}&{{1}}\\ {{1}}&{{0}}&{{0}}\\ {{1}}&{{2}}&{{4}}\\ {{1}}&{{3}}&{{9}}\end{array}\right)=\left(\begin{array}{c c c c}{{5}}&{{2}}&{{18}}\\ {{2}}&{{18}}&{{26}}\\ {{18}}&{{26}}&{{114}}\end{array}\right)$$
וכן:
$$A^{*}{\bf b}=\left(\begin{array}{c c c c}{{1}}&{{1}}&{{1}}&{{1}}&{{1}}\\ {{-2}}&{{-1}}&{{0}}&{{2}}&{{3}}\\ {{4}}&{{1}}&{{0}}&{{4}}&{{9}}\end{array}\right)\left(\begin{array}{c}{{4}}\\ {{2}}\\ {{1}}\\ {{1}}\\ {{1}}\end{array}\right)=\left(\begin{array}{c}{{9}}\\ {{-5}}\\ {{31}}\end{array}\right)$$
כלומר המשוואה שצריך לפתור תהיה:
$${\left(\begin{array}{l l l}{5}&{2}&{18}\\ {2}&{18}&{26}\\ {18}&{26}&{114}\end{array}\right)}\ {\left(\begin{array}{l}{a}\\ {b}\\ {c}\end{array}\right)}=\ {\left(\begin{array}{l}{9}\\ {-5}\\ {31}\end{array}\right)}$$
כאשר נקבל את הפתרון:
$$a=86/77,\qquad b=-62/77,\qquad c=43/154.$$
ולכן הקירוב הטוב ביותר יהיה:
$$y=86/77-62x/77+43x^{2}/154$$

\end{example}
\section{נורמה אופרטורית ורגישות של מערכות משוואות}

\begin{definition}[נורמה אופרטורית]
יהי \(A \in M_{m,n}\left( \mathbb{F}  \right)\) מטריצה. אם \(||\cdot||_{p},||\cdot||_{q}\) נורמות על \(\mathbb{F} ^{n}\) ו-\(\mathbb{F} ^{m}\) בהתאמה, נגדיר את:
$$\|A\|_{p,q}=\operatorname*{max}\left\{{\frac{\|A\mathbf{x}\|_{q}}{\|\mathbf{x}\|_{p}}}:\mathbf{x}\in\mathbb{F}^{n},\ \mathbf{x}\neq0\right\}.$$
להיות הנורמה האופרטורית המתאימה לנורמות \(||\cdot||_{p}\) ו- \(||\cdot||_{q}\).

\end{definition}
\begin{proposition}
מתקיים:
$$\|A\|_{q,p}=\operatorname*{max}\left\{\|A\mathbf{x}\|_{q}:\mathbf{x}\in\mathbb{F}^{n},\ \|\mathbf{x}\|_{p}=1\right\}.$$

\end{proposition}
\begin{symbolize}
כאשר יש רק אינדקס אחד על הנורמה נניח כי הנורמה תהיה אותה נורמה עבור \(||\cdot||_{p}\) ו-\(||\cdot||_{q}\). למשל:
$$\|A\|_{1}:=\|A\|_{1,1},\quad\|A\|_{2}:=\|A\|_{2,2},\quad\|A\|_{\infty}:=\|A\|_{\infty,\infty}$$

\end{symbolize}
\begin{example}
נחשב את \(\lVert A \rVert_{1}\) עבור \(\mathbb{F} =\mathbb{R}\).  מספיק להסתכל על הקבוצה:
$$\{(x,y):\|(x,y)\|_{1}=1\}=\{(x,y):|x|+|y|=1\}$$
כאשר מתקיים:
$$\begin{array}{l}{=\operatorname*{max}\left\{\left|2x+3y\right|+\left|x-5y\right|:\left|x\right|+\left|y\right|=1\right\}}\\ {\leq\operatorname*{max}\left\{2|x|+3|y|+|x|+5|y|:\left|x\right|+\left|y\right|=1\right\}}\\ {=\operatorname*{max}\left\{3|x|+8|y|:\left|x\right|+\left|y\right|=1\right\}}\\ {=\operatorname*{max}\left\{3|x|+8(1-|x|):0\leq|x|\leq1\right\}}\\ {=\operatorname*{max}\left\{8-5x:0\leq x\leq1\right\}}\\ {=8.}\end{array}$$$$\|A\|_{1}=\operatorname*{max}\left\{\sum_{i=1}^{m}|a_{i j}|:1\leq j\leq n\right\}$$$$\|A\|_{\infty}=\operatorname*{max}\left\{\sum_{j=1}^{n}|a_{i j}|:1\leq i\leq m\right\}$$$$\|A\|_{2}\;\leq\;\sqrt{\sum_{i=1}^{m}\sum_{j=1}^{n}|a_{i j}|^{2}}\;=:\;\|A\|_{F}.$$$$A\mathbf{x}=\mathbf{b}\quad{\mathrm{and}}\quad A\mathbf{x}^{\prime}=\mathbf{b}^{\prime}$$$$A={\begin{bmatrix}1&1\\ 1&1.00001\end{bmatrix}},\quad\mathbf{b}={\begin{bmatrix}2\\ 2.00001\end{bmatrix}},\quad\mathbf{b}^{\prime}={\begin{bmatrix}2\\ 2.00002\end{bmatrix}}.$$$$\mathbf{x}={\begin{bmatrix}1\\ 1\end{bmatrix}}\qquad{\mathrm{and}}\qquad\mathbf{x}^{\prime}={\begin{bmatrix}0\\ 2\end{bmatrix}}$$$$\kappa(A)=\|A\|\,\|A^{-1}\|\geq1$$$$\kappa(cA)=\kappa(A)$$$$\kappa(cA)=\|cA\|\,\|c^{-1}A^{-1}\|=|c|\,|c^{-1}|\,\|A\|\,\|A^{-1}\|=|cc^{-1}|\kappa(A)=\kappa(A)$$$$(c A)^{-1}=c^{-1}A^{-1}$$$$\mathbf{b}^{\prime}=\mathbf{b}+\Delta\mathbf{b}$$$${\frac{\|\Delta\mathbf{x}\|}{\|\mathbf{x}\|}}\leq\kappa(A){\frac{\|\Delta\mathbf{b}\|}{\|\mathbf{b}\|}}$$$$A(\Delta\mathbf{x})=A(\mathbf{x}^{\prime}-\mathbf{x})=A\mathbf{x}^{\prime}-A\mathbf{x}=\mathbf{b}^{\prime}-\mathbf{b}=\Delta\mathbf{b},$$$$\|\Delta\mathbf{x}\|\leq\|A^{-1}\|\,\|\Delta\mathbf{b}\|\quad{\mathrm{and}}\quad\|\mathbf{b}\|\leq\|A\|\,\|\mathbf{x}\|.$$$${\frac{\|\Delta\mathbf{x}\|}{\|\mathbf{x}\|}}\leq{\frac{\|A^{-1}\|\ \|\Delta\mathbf{b}\|}{\|\mathbf{b}\|/\|A\|}}=\kappa(A){\frac{\|\Delta\mathbf{b}\|}{\|\mathbf{b}\|}}$$$$A=\begin{bmatrix}1&1\\ 1&1.00001\end{bmatrix},\quad\mathbf{b}=\begin{bmatrix}2\\ 2.00001\end{bmatrix},\quad\mathbf{b}^{\prime}=\begin{bmatrix}2\\ 2.00002\end{bmatrix},\quad\mathbf{x}=\begin{bmatrix}1\\ 1\end{bmatrix},\quad\mathbf{x}^{\prime}=\begin{bmatrix}0\\ 2\end{bmatrix}.$$$$\kappa(A){\frac{\|\Delta\mathbf{b}\|}{\|\mathbf{b}\|}}=(2.00001)^{2}\cdot10^{5}\cdot{\frac{10^{-5}}{4.00001}}\geq1={\frac{\|\Delta\mathbf{x}\|}{\|\mathbf{x}\|}}$$$${\frac{\|\Delta\mathbf{x}\|}{\|\mathbf{x}\|}}\leq c\cdot\kappa(A)\left({\frac{\|\Delta\mathbf{b}\|}{\|\mathbf{b}\|}}+{\frac{\|\Delta A\|}{\|A\|}}\right)$$

\end{example}
\section{שיטת החזקה למציאת ערכים עצמיים}

\begin{proposition}
אם ל-\(A \in M_{n\times n}\left( \mathbb{R} \right)\) יש \(n\) ערכים עצמיים שונים כאשר נקבל כי:
$$A^{k}v\to \lambda^{k}v$$
כאשר \(\lambda\) הערך עצמי עם הגודל הגדול ביותר. 

\end{proposition}
\begin{proof}
נניח כי יש \(n\) ערכים עצמיים שונים. נניח בלי הגבלת הכלליות כי הערך עצמי \(\lambda\) הוא הגדול ביותר.
$$v=\lambda_{1}v_{1}+\lambda_{2}v_{2}+\dots+\lambda_{n}v_{n}$$
כאשר \(v_{i}\) הוא הווקטור העצמי המתאים ל-\(\lambda_{i}\). כעת:
$$A^{k}v=\lambda_{1}^{k}v_{1}+\lambda_{2}^{k}v_{2}+\dots+\lambda_{n}^{k}v_{n}=\lambda_{1}^{k}\left( v_{1}+\left( \frac{\lambda_{2}}{\lambda_{1}} \right)^{k}v_{2}+\dots+\left( \frac{\lambda_{n}}{\lambda_{1}} \right)^{k}v_{n} \right)\to \lambda_{1}^{k}v_{1}$$

\end{proof}
\begin{remark}
בפועל נקבל כי הפעלה ממשוכת מחזירה ווקטור עצמי, כאשר ייתכן וגודלו יהיה גדול. ניתן לנרמל אותו ולמצוא את הערך עצמי המתאים על ידי \(Av=\lambda v\).

\end{remark}
\begin{proposition}[שיטת החזקה המוזזת]
ניתן למצוא ערך עצמי שלא הכי גדול על ידי שימוש בשיטת החזקה על \(\left( A-\alpha I \right) ^{-1}\). 

\end{proposition}
ניתן ליעל את התהליך על ידי זה שנתחיל עם ווקטור קרוב כמה שיותר לווקטור עצמי הגדול ביותר.

\begin{proposition}[עיגולי גירשגורין]
כל ערך עצמי של מטריצה \(A\) צריך להיות בתוך עיגול במישור המרוכב סביב \(A_{i i}\) כאשר רדיוס העיגול יהיה סכום הגודל של הערכים שאינם על האלכסון, כאשר נדרש כי העיגולים זרים.

\end{proposition}
\begin{proof}
$$Av=\lambda v$$
נניח כי \(i\) זה הערך עם הגודל הגדול ביותר:
$$\sum_{c=1}^{n} A_{i c}v_{c}=\lambda v_{i}$$$$A_{i i}v_{i}+\sum_{c\neq i} A_{ic} v_{c} = \lambda v_{i}$$$$\left\lvert  \sum_{c\neq i} A_{i c} \frac{v_{c}}{v_{i}}  \right\rvert =\left\lvert  \left( \lambda- A_{ii} \right)  \right\rvert \leq \sum_{i \neq c} \lvert A_{i c} \rvert \left\lvert  \frac{v_{c}}{v_{i}}  \right\rvert \leq \sum_{i\neq c} \lvert A_{i c} \rvert  $$

\end{proof}
\begin{example}
עבור המטריצה:
$$A=\begin{pmatrix}2 & 0 &  0 \\0 & -1 & 2 \\5 & 6 & -20
\end{pmatrix}$$
נקבל מיידית מהשורה הראשונה כי 2 הוא ערך עצמי כיוון שהערך עצמי צריך להיות בעיגול ברדיוס \(0+0\) מ-2. נקבל מהשורה השנייה כי הערך העצמי צריך להיות ברדיוס של \(|2|+|0|=2\) של \(-1\). כמו כן עבור השורה האחרונה נקבל כי הערך עצמי צריך להיות בתוך עיגול ברדיוס \(|5|+|6|=11\) סביב \(-20\).

\end{example}
\begin{corollary}
הערכים העצמיים צריכים גם להיות בתוך העיגולים של העמודות כיוון שיש ל-\(A^{T}\) אותם ערכים עצמיים וניתן להפעיל את המשפט עליו.

\end{corollary}
\begin{definition}[מטריצה בעל אלכסון שולט]
מטריצה שבה הערך המוחלט של האלכסון גדול ממש מסכום יתר האיברים בשורה.

\end{definition}
\begin{corollary}
מטריצה בעל אלכסון שולט היא הפיכה כיוון ש-0 לא יכול להיות ערך עצמי.

\end{corollary}
\end{document}