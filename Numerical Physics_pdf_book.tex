\documentclass{tstextbook}

\usepackage{amsmath}
\usepackage{amssymb}
\usepackage{graphicx}
\usepackage{hyperref}
\usepackage{xcolor}

\begin{document}

\title{Example Document}
\author{HTML2LaTeX Converter}
\maketitle

\section{אלגברה ואנליזה נומרית}

\subsection{אינטגרציה נומרית}

\begin{definition}[קוודריטור - Quadriture]
חישוב מקורב של אינטגרל מסויים מהצורה:
$$I=\int_{a}^{b}f(x)\,d x$$
באופן כללי הקירוב נעשב בעזרת סכום על ערך הפונקציה בנקודות כפול איזשהו משקל:
$$I\approx\sum_{i=0}^{n}w_{i}f(x_{i})$$

\end{definition}
\begin{definition}[שגיאות קטיעה לוקאליות]
שגיאות של כל תת תחום הנוצר על ידי קירוב האינטרל.

\end{definition}
\begin{definition}[שגיאות קטיעה גלובאליות]
סך השגיאות הלוקאליות.

\end{definition}
\begin{definition}[כלל אמצע המלבן]
נניח כי עבור תת תחום \([a,b]\) עם עובי \(h=b-a\) נקרב את הפונקצייה בממוצע לפונקציה קבועה אשר שווה לערך באמצע שלה כך שנקבל:
$$I\approx f\left({\frac{a+b}{2}}\right)h$$

\end{definition}
\begin{corollary}
מקירוב טיילור סביב המרכז ניתן להראות כי השגיאה תהיה:
$$E_{\mathrm{mid}}=-{\frac{(b-a)^{3}}{24}}f^{\prime\prime}(\xi)$$
עבור איזשהו \(\xi \in [a,b]\). השיטה הזאת היא עם שגיאה \(O(h^{3})\) לוקלית כאשר נסכם על הרבה תחומים השגיאה הגלובלית תהיה \(O(h^{2})\).

\end{corollary}
\begin{definition}[כלל הטרפז]
נקרב כל תת תחום על ידי:
$$I\approx\frac{h}{2}\left[f(a)+f(b)\right]$$

\end{definition}
\begin{corollary}
השגיאה הכוללת תהיה מסדר גודל של \(O(h^{2})\)

\end{corollary}
\begin{proof}
נפתח \(f(b)\) סביב a ונקבל:
$$f(b)=f(a)+h f^{\prime}(a)+{\frac{h^{2}}{2}}f^{\prime\prime}(\xi)$$
כאשר נקבל כי הקירוב הלוקאלי של השגיאה(ממשפט טיילור) יהיה מסדר גודל של:
$$E_{\mathrm{trap}}=-{\frac{h^{3}}{12}}f^{\prime\prime}(\xi).$$
ולכן השגיאה הכוללת תהיה מסדר גודל של \(O(h^{2})\).

\end{proof}
\begin{proposition}[נוסחת סיפסון]
משתמש בפולינום ריבועי כדי לקרב את \(f(x)\) על התחום \([a,b]\)(כאשר \(b-a=2h\)) כך שהנוסחא תהיה:
$$I\approx\frac{h}{3}\left[f(a)+4f(a+h)+f(b)\right]$$

\end{proposition}
\begin{corollary}
ניתן לקבל את השגיאה מנוסחאת טיילור ולהראות כי תהיה מהצורה:
$$E_{\mathrm{Simpson}}=-\frac{(b-a)^{5}}{2880}f^{(4)}(\xi)$$
כלומר השגיאה הכוללת תהיה מסדר גודל של \(O(h^{4})\).

\end{corollary}
\subsection{אינטרפולציה}

\begin{definition}[אינטרפולציה]
כאשר יש לנו ערכים בדידים אינטפוציה זה התהליך של לנסות לחזות את הערך של הפונקציה בערכים שאינם מדוייקים(כלומר בין הערכים הבדידים).

\end{definition}
\begin{definition}[אינטרפולציה לינארית]
בהנתן שתי נקודות סמוכות \((x_{i},f_{i})\) ו-\((x_{i+1},f_{i+1})\) האינטרפולציה הלינארית תהיה:
$$f(x)\approx f_{i}+{\frac{f_{i+1}-f_{i}}{x_{i+1}-x_{i}}}\,(x-x_{i}),\quad x_{i}\leq x\leq x_{i+1}.$$

\end{definition}
\begin{definition}[אינטרפולציה פולינומית]
מחפש פולינום יחיד אשר עובר דרך קבוצה של \(n+1\) נקודות.

\end{definition}
יש מספר שיטות אשר מבצעות אינטרפולציה פולינומית

\begin{definition}[אינטרפולציית לגרנג']
שיטת אינטרפלציה פולינומית. בהנתן נקודות \((x_{0},f_{0}),(x_{1},f_{1}),\dots,(x_{n},f_{n})\) הפולינום הייחודי \(P(x)\) של דרגה לכל היותר \(n\) ומקיים \(P(x_{i})=f_{i}\) נתון על ידי נוסחאת לגרנג':
$$P(x)=\sum_{i=0}^{n}f_{i}\,\ell_{i}(x)$$
כאשר הבסיס של לגרנג' \(\ell_{i}\) נתון על ידי:
$$\ell_{i}(x)=\prod_{\stackrel{j=0}{j\neq i}}^{n}{\frac{x-x_{j}}{x_{i}-x_{j}}}.$$
כאשר בפרט \(\ell_{i}(x)=1\) עבור \(x=x_{i}\) ו-\(\ell_{i}=0\) בכל נקודה \(x=x_{j}\) עבור \(j\neq i\).

\end{definition}
\begin{proposition}
השגיאה של לגרנג' נתונה על ידי:
$$f(x)-P(x)={\frac{f^{(n+1)}(\xi)}{(n+1)!}}\prod_{i=0}^{n}(x-x_{i}),$$
כאשר \(\xi\) זה איזשהו נקודה בתחום.

\end{proposition}
\begin{definition}[אינטרפולציה ניוטונית]
שיטת אינטרפלציה פולינומית. בהנתן נקודות \((x_{0},f_{0}),(x_{1},f_{1}),\dots,(x_{n},f_{n})\) הפולינום:
$$P(x)=a_{0}+a_{1}(x-x_{0})+a_{2}(x-x_{0})(x-x_{1})+\cdots+a_{n}(x-x_{0})(x-x_{1})\cdots(x-x_{n-1}),$$
כאשר המקדמים נתונים על ידי:
$$a_{0}=f(x_{0}),\quad a_{1}=\frac{f(x_{1})-f(x_{0})}{x_{1}-x_{0}},\quad a_{2}=\frac{f[x_{1},x_{2}]-f[x_{0},x_{1}]}{x_{2}-x_{0}}$$
כאשר נשים לב ליתרון משמעותי של שיטה זו - אם מסופים נקודה נוספת, נדרש להוסיף איבר לפלינום ללא חישוב מחדש

\end{definition}
\begin{proposition}
השגיאה של אינטרפולציה ניוטונית נתונה על ידי:
$$f(x)-P(x)={\frac{f^{(n+1)}(\xi)}{(n+1)!}}\prod_{i=0}^{n}(x-x_{i}),$$

\end{proposition}
\begin{definition}[אינטרפולציית הרמיט]
שיטת אינטרפלציה פולינומית אשר משתמשת לא רק בערכים בנקודות, אלה גם בנגזרות בנקודות. אינטרפלציית הרמית בונה פולינומים \(H(x)\) אשר מקיימים:
$$H(x_{i})=f(x_{i})\quad{\mathrm{and}}\quad H^{\prime}(x_{i})=f^{\prime}(x_{i})$$
עבור שתי נקודות \(x_{0},x_{1}\) למשל ניתן לכתוב:
$$H(x)=f(x_{0})h_{00}(x)+f(x_{1})h_{11}(x)+f^{\prime}(x_{0})h_{10}(x)+f^{\prime}(x_{1})h_{01}(x),$$
כאשר את איברי הבסיס \(h_{00}(x),h_{11}(x),h_{10}(x),h_{01}(x)\) מתקבלות על ידי האילוץ של הערכים והנגזרת.

\end{definition}
\begin{definition}[אינטרפולציית ספליין קוביות]
בגישה זו עבור כל קטע \([x_{i},x_{i+1}]\) ניתן להגדיר:
$$S_{i}(x)=a_{i}+b_{i}(x-x_{i})+c_{i}(x-x_{i})^{2}+d_{i}(x-x_{i})^{3}$$
כאשר הספליין מורכב כך ש:
$$S_{i}(x_{i})=f(x_{i}),S_{i}(x_{i+1})=f(x_{i+1}),S_{i}^{\prime}(x_{i+1})=S_{i+1}^{\prime}(x_{i+1}),S_{i}^{\prime\prime}(x_{i+1})=S_{i+1}^{\prime\prime}(x_{i+1})$$

\end{definition}
\subsection{גזירה נומרית}

\begin{reminder}
עבור פונקציה \(f\) שהיא חלקה, הטור טיילור מסביב ל-\(x_{0}\) נתון על ידי:
$$f(x_{0}+h)=f(x_{0})+h f^{\prime}(x_{0})+{\frac{h^{2}}{2}}f^{\prime\prime}(x_{0})+{\frac{h^{3}}{6}}f^{\prime\prime\prime}(x_{0})+{\cal O}(h^{4}).$$
כאשר באופן דומה עבור צעד בכיוון השלילי נקבל:
$$f(x_{0}-h)=f(x_{0})-h f^{\prime}(x_{0})+\frac{h^{2}}{2}f^{\prime\prime}(x_{0})-\frac{h^{3}}{6}f^{\prime\prime\prime}(x_{0})+O(h^{4}).$$

\end{reminder}
\begin{definition}[שגיאת קיטוע - Trucation Error]
נובע מכך שאנחנו חותכים את הטור טיילור באיזשהו איבר. זה נובע מהנזנחה של איברים מסדר קבוע

\end{definition}
\begin{definition}[שגיאת עיגול - Roundoff Error]
נובע מאי דיוקים של החישוב עצמו

\end{definition}
\begin{proposition}[קירוב לנגזרת מימין]
$$f^{\prime}(x_{0})\approx{\frac{f(x_{0}+h)-f(x_{0})}{h}}$$
עד כדי שגיאת קיטוע מסדר גודל של \(O(h)\).

\end{proposition}
\begin{proof}
זה למעשה הגבול של הגדרת הנגזרת. נמצא את הסדר גודל של השגיאה של \(h\) בעזרת טור טיילור:
$$f(x_{0}+h)=f(x_{0})+h f^{\prime}(x_{0})+{\frac{h^{2}}{2}}f^{\prime\prime}(x_{0})+O(h^{3}).$$
נחלק ב-\(h\) ונסדר מחדש כך שנקבל:
$$\frac{f(x_{0}+h)-f(x_{0})}{h}=f^{\prime}(x_{0})+\frac{h}{2}f^{\prime\prime}(x_{0})+O(h^{2}).$$
כך שניתן לראות כי זה שווה לנגזרת עד כדי \(O(h)\).

\end{proof}
\begin{proposition}[קירוב הפרש מרכזי]
$$f^{\prime}(x_{0})\approx\frac{f(x_{0}+h)-f(x_{0}-h)}{2h}.$$
עד כדי שגיאת קיטוע מסדר גודל של \(O(h^{2})\)

\end{proposition}
\begin{proof}
נחסר את הטור טיילור של \(f(x_{0}-h)\) מהטור טיילור של \(f(x_{0}+h)\) כך שנקבל:
$$f(x_{0}+h)-f(x_{0}-h)=\left[f(x_{0})+h f^{\prime}(x_{0})+\frac{h^{2}}{2}f^{\prime}(x_{0})+\frac{h^{3}}{6}f^{\prime\prime}(x_{0})+O(h^{4})\right]-\left[f(x_{0})-h f^{\prime}(x_{0})+\frac{h^{2}}{2}f^{\prime}(x_{0})-\frac{h^{3}}{6}f^{\prime\prime}(x_{0})+O(h^{4})\right].$$
הרבה גורמים משותפים מצטמצמים(בפרט הגורם הריבועי) כך שנקבל:
$$f(x_{0}+h)-f(x_{0}-h)=2h f^{\prime}(x_{0})+{\frac{2h^{3}}{6}}f^{\prime\prime\prime}(x_{0})+O(h^{5}).$$
נחלק כעת ב-\(2h\) ונקבל:
$$\frac{f(x_{0}+h)-f(x_{0}-h)}{2h}=f^{\prime}(x_{0})+\frac{h^{2}}{3!}f^{\prime\prime\prime}(x_{0})+O(h^{4})$$
כלומר קיבלנו ששווה לנגזרת ב-\(x_{0}\) עד כדי \(O(h^{2})\).

\end{proof}
\begin{remark}
ניתן לקבל נוסחאות עבור דיוקים אפילו טובים יותר על ידי שימוש בנגזרת בנקודות של \(x_{0}-2h,x_{0}-h,x_{0}+h,x_{0}+2h\) ולקבל שגיאה מסדר גודל של \(O(h^{4})\) או אפילו יותר טוב. ניתן לכתוב באופן כללי:
$$f^{\prime}(x_{0})\approx{\frac{A f(x_{0}-2h)+B f(x_{0}-h)+C f(x_{0})+D f(x_{0}+h)+E f(x_{0}+2h)}{h}},$$
ולהשוואות את המקדמים לערכים של הטור טיילור.

\end{remark}
\begin{corollary}[קירוב לשגיאת קיטוע]
ממשפט השארית של קושי נקבל:
$$f^{\prime}(x)=\frac{f(x+h)-f(x-h)}{2h}-\frac{h^{2}}{6}f^{\prime\prime\prime}(\xi)$$
עבור \(\xi \in (x-h,x+h)\) ולכן השגיאת קיטוע תהיה שווה ל:
$$E_{\mathrm{trunc}}\approx{\frac{h^{2}}{6}}f^{\prime\prime\prime}(\xi)$$

\end{corollary}
\begin{proposition}[קירוב לשגיאת עיגול]
כיוון ש-\(f(x+h)\) ו-\(f(x)\) יחסית קרובים נקבל כי בערך שווים וקירוב טוב עבורם יהיה:
$$E_{\mathrm{round}}\approx{\frac{\epsilon\left|f(x)\right|}{h}}$$
כאשר \(\epsilon\) זה השגיאה של המכשיר.

\end{proposition}
\begin{remark}
קיבלנו שהשגיאת קיטוע קטנה ככל ש-\(h\) קטן יותר כאשר השגיאת עיגול גדלה ככל ש-\(h\) קטן יותר, ולכן נצפה כי יהיה ערך אידיאלי ל-\(h\).

\end{remark}
\begin{proposition}
הביטוי עבור ה-\(h\) האידיאלי עבור הפרש מרכזי נתון על ידי:
$$h_{\mathrm{opt}}=\left({\frac{C_{2}\epsilon}{2C_{1}}}\right)^{\frac{1}{3}}.$$
כאשר \(C_{1}\) הוא קבוע התלוי בנגזרות מסדר גבוה(עבור הפרש מרכזי \(C_{1}\approx \max_{x-\delta \leq \xi \leq x+\delta}\lvert f'''(\xi) \rvert\)), \(C_{2}\) הוא קבוע שתלוי בגודל של \(f(x)\), ו-\(\epsilon\) זה השגיאה של המכשיר.

\end{proposition}
\begin{proposition}
ניתן לקרב את השגיאה הכוללת כתלות ב-\(h\) על ידי:
$$E(h)\approx C_{1}h^{p}+\frac{C_{2}\epsilon}{h}$$
כאשר \(p\) זה קבוע אשר שווה ל-2 במקרה של הפרש מרכזי. נקבל:
$${\frac{d E}{d h}}=p C_{1}h^{p-1}-{\frac{C_{2}\epsilon}{h^{2}}}=0.$$
ולכן אם נפתור עבור \(h\):
$$h_{\mathrm{opt}}=\left({\frac{C_{2}\epsilon}{p C_{1}}}\right)^{\frac{1}{p+1}}.$$
נציב \(p=2\) עבור המקרה שלנו:
$$h_{\mathrm{opt}}=\left({\frac{C_{2}\epsilon}{2C_{1}}}\right)^{\frac{1}{3}}$$

\end{proposition}
\begin{example}
נניח כי נתון פונקציה כך שמתקיים \(\max\lvert f'''(x) \rvert\approx M_{3}\) ולכן הקבוע \(C_{1}\) עבור ההפרש המרכזי יהיה \(\frac{M_{3}}{6}\). אם \(C_{2}\) זה בקירוב \(\lvert f(x) \rvert\) נקבל:
$$h_{\mathrm{opt}}=\left({\frac{6|f(x)|\epsilon}{2M_{3}}}\right)^{\frac{1}{3}}=\left({\frac{3|f(x)|\epsilon}{M_{3}}}\right)^{\frac{1}{3}}$$

\end{example}
\subsection{הצגות של מספרים ממשיים}

\begin{definition}[שגיאה מחולטת]
כמה שאנחנו שוגאים באופן מוחלט - כלומר ערך \(\epsilon\) כך ש- \(x\pm \varepsilon\) זה הטווח ערכים שלנו כתוצאה מהשגיאה.

\end{definition}
\begin{definition}[שגיאה יחסית]
שגיאה ביחס לתוצאה. כלומר אם יש לנו שגיאה מחולטת של \(\epsilon\) אז השגיאה היחסית שלנו היא \(\frac{x+\epsilon}{x}\). זה כמובן תלוי בערך של \(x\), וככל ש-\(x\) קטן יותר ביחס לשגיאה אז השגיאה היחסית יותר גדולה.

\end{definition}
\begin{proposition}
מחשב הדרך כלל מייצג מספרים על ידי בצורה הבאה:
$$x=\pm m\times2^{e},$$
כאשר \(m\) נקרא המאנטיסה(mantissa) ומאכסן בקירוב 24 ביטים של אינפורמציה(באיזור 7 ספרות עשרוניות) והאקספוננט מאכן בערך 8 ביטים.

\end{proposition}
\begin{corollary}
לא ניתן ייצג כל מספר ממשי בצורה מדוייקת, ולכן המספר יהיה בקירוב:
$$\tilde{x}=x(1+\epsilon_{x})$$
כאשר \(\varepsilon_{x}\) זה השגיאה היחסית(יהיה בסדר גודל של הדיוק של המכשיר \(\epsilon_{\text{mach}}\), שזה יהיה שווה למספר הקטן ביותר כך שבהצגה שלו \(1+\epsilon_{\text{mach}}\neq 1\)).

\end{corollary}
\begin{proposition}[שגיאת עיגול בכפל]
נניח כי שתי מספרים שמורים על ידי \(\tilde{x}=x(1+\delta_{x})\) ו-\(\tilde{y}=y(1+\delta_{y})\) כאשר \(\delta_{x},\delta_{y}\) הם שגיאות יחסיתיות קטנות , אזי השגיאה היחסית של המכפלה \(\tilde{x}\tilde{y}\) תקיים:
$$\frac{\tilde{x}\tilde{y}-xy}{xy}\approx \delta_{x}+\delta_{y}$$
כלומר כאשר אנחנו כופלים מספרים עם שגיאות אז השגיאות היחסית נסכמות.

\end{proposition}
\begin{proof}
נסמן:
$$\tilde{z}=\tilde{x}\cdot\tilde{y}=x\,y\,(1+\delta_{x})(1+\delta_{y})$$
כאשר אם נפתח את המכפלה נקבל:
$$\tilde{z}=x y\left(1+\delta_{x}+\delta_{y}+\delta_{x}\delta_{y}\right)$$
כאשר נניח ש-\(\delta_{x}\delta_{y}\) זניח כיוון שמכפלה של שתי גדלים קטנים ונבודד את השגיאה היחסית כך שנקבל:
$${\frac{\tilde{z}-x y}{x y}}\approx\delta_{x}+\delta_{y}$$

\end{proof}
\begin{proposition}[שגיאה של חילוק]
נניח כי שתי מספרים שמורים על ידי \(\tilde{x}=x(1+\delta_{x})\) ו-\(\tilde{y}=y(1+\delta_{y})\) כאשר \(\delta_{x},\delta_{y}\) הם שגיאות יחסיתיות קטנות , אזי השגיאה היחסית של המנה \(\frac{\tilde{x}}{\tilde{y}}\) תקיים:
$$\frac{\frac{\tilde{x}}{\tilde{y}}-\frac{x}{y}}{\frac{x}{y}}\approx \delta_{x}-\delta_{y}$$
כאשר אנחנו מחלקים מספרים עם שגיאות אז השגיאות היחסית יהיו ההפרש.

\end{proposition}
\begin{proof}
נסמן:
$$\tilde{q}=\frac{\tilde{x}}{\tilde{y}}=\frac{x(1+\delta_{x})}{y(1+\delta_{y})}$$
כאשר ניתן לכתוב את זה על ידי:
$$\tilde{q}=\frac{x}{y}(1+\delta_{x})\cdot\frac{1}{1+\delta_{y}}$$
כאשר על ידי קירוב טיילור מסדר ראשון עבור \(\frac{1}{1+\delta_{y}}\) כיוון ש-\(\delta_{y}\) קטן נקבל:
$$\tilde{q}\approx\frac{x}{y}(1+\delta_{x})(1-\delta_{y})=\frac{x}{y}(1+\delta_{x}-\delta_{y}-\delta_{x}\delta_{y})$$
אם נזניח את הגורם \(\delta_{x}\delta_{y}\) נקבל:
$$\tilde{q}\approx\frac{x}{y}(1+\delta_{x}-\delta_{y})$$
וכעת השגיאה היחסית תהיה:
$${\frac{\tilde{q}-{\frac{x}{y}}}{{\frac{x}{y}}}}\approx\delta_{x}-\delta_{y}$$

\end{proof}
\begin{proposition}[חיבור של שגיאות]
נניח כי שתי מספרים שמורים על ידי \(\tilde{x}=x(1+\delta_{x})\) ו-\(\tilde{y}=y(1+\delta_{y})\) כאשר \(\delta_{x},\delta_{y}\) הם שגיאות יחסיתיות קטנות , אזי השגיאה היחסית של \(\tilde{x}+\tilde{y}\) תהיה:
$$\frac{x\delta_{x}+y\delta_{y}}{x+y}$$

\end{proposition}
\begin{proof}
$${\tilde{x}}+{\tilde{y}}=x{\big(}1+\delta_{x}{\big)}+y{\big(}1+\delta_{y}{\big)}={\big(}x+y{\big)}+{\big(}x\delta_{x}+y\delta_{y}{\big)}$$
ולכן השגיאה המוחלטת תהיה \(x\delta_{x}+y\delta_{y}\), ולכן השגיאה היחסית תיהיה:
$$\frac{x\delta_{x}+y\delta_{y}}{x+y}$$

\end{proof}
\begin{corollary}
עבור הפרש נקבל:
$${\frac{x\delta_{x}+y\delta_{y}}{|x-y|}}$$
וזה יכול ליצור בעיות משמעותיות כאשר \(x \approx y\) כיוון שהשגיאה היחסית תגדל משמעותית.

\end{corollary}
\begin{corollary}
נוסחאות עם הפרש של גדלים מאותו סדר גודל יהיו פחות טובות, לדוגמא עבור נוסחאת השורשים:
$$x={\frac{-b\pm{\sqrt{b^{2}-4a c}}}{2a}}$$
יש בעיה עבור הביטוי עם ה-\(-\). ניתן לחשב עבור שורש אחד את ה-\(+\):
$$x_{1}={\frac{-b+{\sqrt{b^{2}-4a c}}}{2a}}$$
ועבור השורש השני(של ה-\(-\)) בעזרת נוסחא שקולה אשר עוקפת את הבעיה של החיסור:
$$x_{2}={\frac{-2c}{b+{\sqrt{b^{2}-4a c}}}}$$

\end{corollary}
\subsection{מציאת שורשים}

\begin{definition}[שורש של פונקציה]
נקודה \(p\) נקראת שורש של פונקציה \(f\) אם:
$$f(p)=0$$

\end{definition}
\begin{remark}
כל עוד לא מציינים

\end{remark}
\begin{lemma}
אם \(f\) רציפה ב-\([a,b]\) ו-\(f(a)f(b)< 0\) קיים לפחות \(p \in (a,b)\) אחד עם \(f(p)=0\).

\end{lemma}
\begin{proof}
כיוון ש-\(f(a)\cdot f(b)< 0\) יש ל-\(f(a)\) ו-\(f(b)\) סימן שונה ולכן נובע ממשפט ערך הביניים. 

\end{proof}
\begin{proposition}[שיטת החצייה - Bisection]
אם \(f\) רציפה ב-\([a,b]\) ו-\(f(a)f(b)< 0\) אזי ניתן למצוא את השורש על ידי בניית הסדרה הבאה:
נגדיר סדרות \((a_{n}),(b_{n})\) ו-\((p_{n})\) באופן רקורסיבי באופן הבא:

  \begin{enumerate}
    \item נגדיר \(a_{1}=a\) ו-\(b_{1}=b\). נגדיר בנוסף \(p_{n}=\frac{a_{n}+b_{n}}{2}\). 


    \item אם \(f(p_{n})\cdot f(b)< 0\) אז נגדיר \(a_{n+1}=p_{n},b_{n+1}=b_{n}\). אחרת בהכרח \(f(p_{n})\cdot f(a_{n})< 0\) ונגדיר \(b_{n+1}=p_{n},a_{n+1}=a_{n}\). 


    \item אם באיזשהו שלב \(f(p_{n})=0\) אז מצאנו שורש, אחרת \(p_{n}\) מתכנס לשורש \(p\) בצורה אקספוננציאלית: 
$$|p_{n}-p|\leq{\frac{b-a}{2^{n}}}$$
כיוון שבכל שלב אנחנו חוצים את התחום.


  \end{enumerate}
\end{proposition}
\begin{proposition}[שיטת ניוטון רפסון]
נניח כי \(f\) פונקציה גזירה פעמיים עם שורש. אם נתחיל מניחוש התחלתי \(p_{0}\). ניתן להתכנס לשורש \(p\) על ידי:
$$p_{n+1}=p_{n}-{\frac{f(p_{n})}{f^{\prime}(p_{n})}}$$

\end{proposition}
נסמן \(e_{n}=p_{n}=p\). משפט טיילור אומר לנו כי עבור איזשהו \(\xi_{n}\) בין \(p_{n}\) ל-\(p\) נקבל:
$$f(p)=f(p_{n})+(p-p_{n})f^{\prime}(p_{n})+{\frac{(p-p_{n})^{2}}{2}}f^{\prime\prime}(\xi_{n}).$$
כיוון ש-\(f(p)=0\) נקבל:
$$0=f(p_{n})-e_{n}f^{\prime}(p_{n})+\frac{e_{n}^{2}}{2}f^{\prime\prime}(\xi_{n}).$$
על ידי סידור מחדש עבור \(e_{n}\) נקבל:
$$e_{n}=\frac{f(p_{n})}{f^{\prime}(p_{n})}-\frac{e_{n}^{2}}{2}\frac{f^{\prime\prime}(\xi_{n})}{f^{\prime}(p_{n})}.$$
כאשר ניתן להניח כי \(e_{n}\) קטן ולכן \(e_{n}^{2}\) זניח ונקבל:

\begin{proposition}[שירת הקטע - Secant]
וריאציה של שיטת ניוטון רפסון אשר מקרבת את הנגזרת כך שנקבל:
$$p_{n+1}=p_{n}-f(p_{n})\cdot{\frac{p_{n}-p_{n-1}}{f(p_{n})-f(p_{n-1})}}$$

\end{proposition}
\begin{proposition}
שיטת הסקאנט מתכנס מסדר \(\alpha\), כאשר:
$$\alpha={\frac{1+{\sqrt{5}}}{2}}\approx1.618$$

\end{proposition}
\end{document}