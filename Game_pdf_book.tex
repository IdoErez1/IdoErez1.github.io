\documentclass{tstextbook}

\usepackage{amsmath}
\usepackage{amssymb}
\usepackage{graphicx}
\usepackage{hyperref}
\usepackage{xcolor}

\begin{document}

\title{Example Document}
\author{HTML2LaTeX Converter}
\maketitle

\chapter{משחקים קומבינטוריים}

\section{גרפים}

\begin{definition}[משחק קומבינטורי]
2 שחקנים, שחקן אחד מנצח והשחקן השני מפסיד. אינפורמציה מלאה. 

\end{definition}
\begin{proposition}
עבור משחקים קומבינטורים ניתן לתאר את המשחק בעזרת גרף מכווין א-ציקלי כאשר:
- אוסף הקודקודים זה המצבים האפשריים - מסומן ב-\(\mathcal{N}\) - Nodes.
- הענפים זה המהלכים האפשריים - מסומן ב- \(\mathcal{B}\) - Branches.
- הקודקוד הראשי יהיה המצב ההתחלתי של המשחק. מסומן - \(R\) - Root.
- הקודקודים ללא מוצא נקראים עלים. מוסומנים ב-\(\mathcal{L}\) - Leaves.

\end{proposition}
\begin{definition}[אסטרטגיה]
העתקה \(\sigma_{i}:\mathcal{N}\to \mathcal{B}_{N}\). כלומר בפרט אם אנחנו בקודקוד \(N\) זה אומר לנו לאיזה ענף \(B\) ללכת עליו. למעשה אומר מה שחקן \(i\) צריך לעשות במצב נתון.

\end{definition}
\begin{symbolize}
ניתן לתת לקודקוד שבהם לשחקן הראשון יש אסטרטגיה מנצחת ערך 1 ולקודקוד שבהם יש לשחקן השני אסטרטגיה מנצחת \(-1\). בעזרת סימון זה אפשר להגדיד שהערך של קודקוד בתור של שחקן אחד 

\end{symbolize}
\begin{proposition}[זרמלו Zermelo]
בכל משחק קומבינטורי או שלשחקן 1 יש אסטרטגיה מנצחת או שלשחקן 2 יש.

\end{proposition}
\begin{proof}
יהי \(t_{max}\) המספר המקסימלי של תורות שהוא לא טריוויאלי(כלומר לא עולה).
השחקן ב-\(i\) ב-\(t_{\max}\) מחליט להלן:

  \begin{enumerate}
    \item אם קיים \(B_{N}\) כך שהקודקוד העוקב(כלומר מ-\((N,B_{N})\)) הוא מנצח, אז לבחור את את \(B_{N}\). 


    \item אחרת, לא משנה מה השחקן \(i\) מפסיד. 
כעת נוכיח בצורה אינדוטיבית. נסמן עבור כל קודקוד האם מנצח או מפסיד בהתאים לשלבים לעיל. ב-\(t\) נשתמש באותה שיטה:


    \item אם \(\exists B_{N_{t}}\) כך ש-\((N_{t},B_{N_{t}})\) מנצח, נבחר ענף כזה. 


    \item אחרת לא משנה מה \(i\) מפסיד. 


  \end{enumerate}
\end{proof}
\begin{proposition}[זרמלו עם תיקו]
בכל משחק קומבינטורי לאחד השחקנים יש אסטרטגיה מנצחת או ששניהם יכולים לכפות תיקו.

\end{proposition}
\begin{proof}
  \begin{enumerate}
    \item נגדיר את הערך של כל קודקוד \(w(N)\) באופן רקורסיבי. 


    \item \textbf{מקרה בסיס} - נתחיל מהעלים, נגדיק לכל \(L\in \mathcal{L}\) את הערך \(w(L)=1\) אם ניצחון, \(w(L)=-1\) אם הפסד של השחקן הראשון ו-\(w(L)=0\) אם תיקו. 


    \item \textbf{צעד אינדוקטיבי} - אם \(N\) הוא הקודקוד בתור של שחקן 1 נגדיר: 
$$w(N)=\operatorname*{max}\{w(N^{\prime})\mid N^{\prime}{\mathrm{~is~a~legal~move~from~}}N\}$$
כלומר השחקן הראשון בוחר את האופציה הטובה ביותר בשבילו, ואם \(N\) הוא קודקוד בתור של שחקן 2 נגדיר:
$$w(N)=\operatorname*{min}\{w(N^{\prime})\mid N^{\prime}{\mathrm{~is~a~legal~move~from~}}N\}$$


    \item הגדרנו פונקציה \(w:\mathcal{N}\to\{ -1,0,1 \}\) על כל הקודקודים, אשר מכריע האם המצב מנצח לשחקן 1, מנצח לשחקן 2 או תיקו בהנחה שבכל תור כל שחקן משחק את המהלך הטוב ביותר עבורו. בפרט מוגדר עבור השורש \(R\) של העץ. אם \(w(R)=1\) השחקן הראשון מנצח, אם \(w(R)=-1\) השחקן השני מנצח כאשר אם \(w(R)=0\) שתי השחקנים יכולים לכפות תיקו. 


  \end{enumerate}
\end{proof}
\begin{summary}
  \begin{itemize}
    \item משחק קומבינטורי ניתן לתאר בעזרת גרף מכווין א-ציקלי. הקוקודים זה מצבים אפשריים של המשחק, הענפים ההמהלכים האפשריים, השורש זה המצב ההתחלתי והעלים הם המצבים הסופיים.
    \item ניתן לתת לעלה ערך של \(+1\) אם זה מצב של ניצחון של שחקן הראשון ו-\(-1\) אם זה מהלך של נצחון של השחקן השני. כעת ניתן להגדיר באופן אידוקטיבי את הערך של הקודקוד ברמה ה-\(i\) להיות הערך המקסימלי של הקודקודים שמחוברים עליו ברמה ה-\(i+1\) אם זה התור של השחקן הראשון והקודקוד המינימלי אם זה המהלך של השחקן השני.
    \item משפט זרמלו טוען שלשחקן 1 יש אסטרטגיה מנצחת או לשחקן 2 יש אסטרטגיה מנצחת.
    \item זרמלו עם תיקו טוען שלשחקן 1 יש אסטרטגיה מנצחת או ששני השחקנים יכולים לכפות תיקו.
  \end{itemize}
\end{summary}
\section{גנבת אסטרטגיות}

\begin{definition}[גניבת אסטרטגיות]
גניבת אסטרטגיות היא דרך להפרכת אסטרטגיות מנצחות של אחד השחקנים. הרעיון הוא שאם מניחים בשלילה שלשחקן השני יש אסטרטגיה מנצחת, אז מראים שהשחקן הראשון יכול לבנות אסטרטגיה מנצחת מתוך האסטרטגיה המנצחת של השחקן השני, וזה יוכיח שלשחקן השני אין אסטרטגיה מנצחת.

\end{definition}
\begin{example}[Chomp]
במשחק Chomp יש גריד עם משבצת "רעילה". בכל תור שחקן בוחר משבצת, כך שכל המשבצות מעליה ומימין אליה יוצאות מהמשחק (ובתורות הבאים כבר אי אפשר לבחור בהם). השחקן שצריך לקחת את המשבצת הרעילה מפסיד. במשחק זה תמיד יש מצב מנצח או מפסיד.

\end{example}
\begin{example}[Hex]
במשחק Hex יש גריד של משושים, כך שכל שני צדדים מקבילים בגריד באותו צבע (אדום וכחול). בכל תור שחקן בוחר משבצת אותה הוא ממלא לפי הצבע שלו. המטרה היא לחבר אותם. במשחק זה לא ייתכן תיקו.

\end{example}
\begin{proposition}
במשחק Chomp לשחקן הראשון תמיד יש אסטרטגיה מנצחת.

\end{proposition}
\begin{proof}
ההוכחה מבוססת על הנחה בשלילה שלשחקן 2 יש אסטרטגיה מנצחת. אם שחקן 2 מנצח לא משנה מה שחקן 1 יעשה, אז בפרט יש לו אסטרטגיה מנצחת גם אם שחקן 1 בוחר את המשבצת העליונה-ימנית \((m,n)\) בתור הראשון. קיימים \(k^*, l^*\) כך שבחירה של \((k^*, l^*)\) על ידי שחקן 2 מובילה לניצחון. שחקן 1 בונה אסטרטגיה מנצחת בכך שהוא בוחר \((k^*, l^*)\) בתור הראשון. קבוצת המשבצות שתישאר היא בדיוק אותה קבוצה שהייתה נשארת אם שחקן 1 היה בוחר \((m,n)\) ושחקן 2 היה בוחר \((k^*, l^*)\). מכיוון שבמצב זה לשחקן 2 הייתה אסטרטגיה מנצחת (בהנחה השלילה), כעת שחקן 1 יבצע את המהלכים ששחקן 2 היה מבצע, וזה יוביל לניצחון של שחקן 1. מכאן שלשחקן 1 יש אסטרטגיה מנצחת, בסתירה לכך שרק לאחד מהשחקנים יש אסטרטגיה מנצחת. לכן לשחקן 2 לא יכולה להיות אסטרטגיה מנצחת, ובהכרח לשחקן 1 יש אסטרטגיה מנצחת.

\end{proof}
\begin{proposition}
במשחק Hex לשחקן הראשון תמיד יש אסטרטגיה מנצחת.

\end{proposition}
\begin{proof}
ההוכחה מבוססת על הנחה בשלילה שלשחקן 2 יש אסטרטגיה מנצחת. אם יש אסטרטגיה מנצחת בלוח במצב מסוים, אז יש אסטרטגיה מנצחת גם באותו לוח עם תא נוסף הצבוע בצבע שלנו. שחקן 1 בוחר בתור הראשון תא אקראי כלשהו. לאחר מכן שחקן 1 מחקה את מה ששחקן 2 היה עושה, כחלק מהאסטרטגיה המנצחת שלו, אם מצב הלוח היה זהה, למעט זה שכל הצבעים היו הפוכים, והתא הרנדומלי שבחר השחקן הראשון בהתחלה לא היה צבוע. אם שחקן 2 בוחר לצבוע את התא הרנדומלי, שחקן 1 פשוט יצבע תא רנדומלי אחר. כיוון ששחקן 1 למעשה מחקה את אסטרטגיית הניצחון של שחקן 2 (בהיפוך), והוספת תא צבוע לא פוגעת באסטרטגיה מנצחת, שחקן 1 יכול לבנות אסטרטגיה מנצחת.

\end{proof}
\begin{definition}[פעולת ה-xor (או NIM-SUM)]
פעולת ה-xor מסומנת ב- \(\oplus\) ומוגדרת עבור סדרת מספרים טבעיים \(\{x_n\}_{k=1}^N\) כ-:
$$x_1 \oplus x_2 \oplus \dots \oplus x_N = \sum_{k=0}^{\infty} \left[ \left( \sum_{n=1}^{N} x_n^k \right) \mod 2 \right] 2^k$$
כאשר \(x_n^k\) הוא הביט ה-\(k\) של \(x_n\) בייצוג בינארי.

\end{definition}
\begin{remark}
פעולת xor
אינטואיטיבית, את \(\oplus\) ניתן להגדיר באמצעות \(1 \oplus 0 = 0 \oplus 1 = 1\) ו- \(0 \oplus 0 = 1 \oplus 1 = 0\). עבור מספרים בינאריים גדולים יותר, הפעולה פועלת ספרה-ספרה, לדוגמה: \(1001 \oplus 0101 = 1100\). פעולת ה-xor היא אסוציאטיבית, קומוטטיבית ואינבולוציה (הפוכה לעצמה).

\end{remark}
\begin{proposition}
במשחק NIM השחקן הנוכחי בעל אסטרטגיה מנצחת אם ורק אם יש מהלך כך שלאחר שהשחקן משחק, מתקיים כי \(s = x_1 \oplus x_2 \oplus \dots \oplus x_N = 0\).

\end{proposition}
\begin{proof}
שלב ראשון: נניח כי \(s=0\) בתחילת התור. אם שחקן כלשהו מבצע מהלך, ה-NIM-SUM לאחר המהלך לא יהיה אפס.
$$s' = x_1 \oplus x_2 \oplus \dots \oplus y_n \oplus \dots \oplus x_N$$$$= x_1 \oplus x_2 \oplus \dots \oplus y_n \oplus \dots \oplus x_N \oplus (x_n \oplus x_n)$$$$= x_1 \oplus x_2 \oplus \dots \oplus x_n \oplus \dots \oplus x_N \oplus (y_n \oplus x_n)$$$$= s \oplus x_n \oplus y_n$$$$= 0 \oplus x_n \oplus y_n \neq 0$$
שלב שני: עתה נוכיח שאם \(s \neq 0\) אז אפשר להגיע לסכום-נים אפס. יהי \(k^*\) כך ש-\(2^{k^*} \le s < 2^{k^*+1}\). קיים \(x_n\) כך ש-\(2^{k^*} \le x_n\) וגם הביט ה-\(k^*\) של \(x_n\) הוא 1. נגדיר \(y_n = x_n \oplus s\). מתקיים \(y_n < x_n\). כמו כן, \(s' = s \oplus x_n \oplus y_n = s \oplus x_n \oplus (x_n \oplus s) = 0\).
משני השלבים נובע שבכל תור של שחקן 1, הוא יכול לבחור מהלך כך שסכום-נים \(s\) יהיה 0, ולכן כל מהלך ששחקן 2 יבצע יוביל לכך ש-\(s \neq 0\). היות ובשלב מסוים הגפרורים ייגמרו, ובמצב זה סכום-נים הוא 0, אנו מסיקים כי מי שיוביל לכך יהיה שחקן 1, ולכן לא ייתכן ששחקן 2 ינצח. מכאן בהכרח שחקן 1 ינצח לפי אסטרטגיה זו.

\end{proof}
\begin{theorem}[Sprague-Grundy]
כל משחק אובייקטיבי שקול למשחק של NIM עם ערימה אחת.

\end{theorem}
\begin{summary}
  \begin{itemize}
    \item גניבת אסטרטגיות היא דרך להפרכת אסטרטגיות מנצחות של אחד השחקנים.
    \item דוגמאות למשחקים בהם ניתן לגנוב אסטרטגיות הן Chomp ו-Hex.
    \item במשחק NIM השחקן הנוכחי בעל אסטרטגיה מנצחת אם ורק אם יש מהלך כך שלאחר שהשחקן משחק, מתקיים כי \(s = x_1 \oplus x_2 \oplus \dots \oplus x_N = 0\).
  \end{itemize}
\end{summary}
\section{נים}

\begin{definition}[משחק חיסור]
נניח כי יש לנו אוסף של \(n\) מטבעות. כל תור שחקן יכול לקחת \(\left\{  a_{1},\dots,a_{n}  \right\}\) מטבעות כאשר \(a_{i}\in [n]\). השחקן שלוקח את המטבע האחרון מנצח.

\end{definition}
\begin{example}
נניח כי נאפשר לקחת רק 1-2 מטבעות. כעת:
$$\begin{array}{c|c|c}n & \text{possible moves} & \text{winner} \\ \hline 1 & 0 & \checkmark \\ \hline2 & 1,0 & \checkmark  \\\hline3 & 2,1 & \mathsf{X} \\ \hline4 & 3,2 & \checkmark \\ \hline5 & 4,3 & \checkmark
\end{array}$$
ונשים לב כי השחקן הראשון מנצח אם רק אם \(n\nmid 3\). הרעיון לאיך שמנצחים עבור \(n=4,5\) זה שמעבירים את המצב למצב שבו \(n=3\) והתור של השחקן השני, ולכן השחקן השני מפסיד.

\end{example}
\begin{definition}[מספר נימבר]
נגדיר עבור הערך הטרמינאלי \(G(0)=0\) בתור הערך שעבורו מספידים. כעת נגדיר באופן רקורסיבי, אם אנחנו ב-\(n\) וניתן להגיע ל-\(\left\{  a_{1},\dots,a_{n}  \right\}\) אזי נגדיר:
$$G(n)=\mathrm{mex}\left\{  G(a_{1}),\dots G(a_{n})  \right\}:= \min \left\{  \mathbb{N}\setminus \left\{  G(a_{1}),\dots,G(a_{n})  \right\}  \right\}$$
כאשר \(\mathrm{mex}\) מוגדר בתור המספר הטבעי(כולל אפס) הקטן ביותר שלא מופיע ברשימה.

\end{definition}
\begin{example}
אם ניתן לקחת \(1-2\) מטבעות נקבל:
$$\begin{array}{c|c|c}n & \text{possible moves} &\text{Nimber}& \text{winner} \\ \hline 1 & 0 & G(1)=\text{mex}(G(0))=\text{mex}\{ 0 \}=1&\checkmark \\ \hline2 & 1,0 &  G(2)=\mathrm{mex}\{ G(1),G(0) \}=\text{mex}\{ 0,1 \}=2& \checkmark  \\\hline3 & 2,1 &  G(3)=\text{mex}\{ 1,2 \}=0& \mathsf{X} \\ \hline4 & 3,2 & G(4)=\text{mex}\{ 0,2 \}= 1& \checkmark \\ \hline5 & 4,3 & G(5)=\text{mex}\{ 0,1 \}=2&  \checkmark
\end{array}$$

\end{example}
\begin{proposition}
שחקן 1 יכול לנצח אם"ם \(G(n)\neq 0\).

\end{proposition}
\begin{proof}
נניח כי \(G(n)=0\). לפי הגדרת מספר נימבר, כל מהלך אפשרי מוביל למצב שבו \(G(a_{i})\neq 0\). כלומר, כל תור של השחקן הראשון יוביל למצב שבו השחקן השני נמצא במצב מנצח (\(G(a_{i})\neq 0\)). השחקן השני יוכל תמיד להחזיר את המצב ל-\(G=0\) בתורו, ולכן השחקן הראשון יפסיד.
לעומת זאת, אם \(G(n)\neq 0\), השחקן הראשון יכול לבחור מהלך שמוביל למצב שבו \(G(a_{i})=0\) (קיים \(a_{i}\) כזה לפי הגדרת \(\mathrm{mex}\)). כך בכל תור, השחקן הראשון יכול להחזיר את המצב ל-\(G=0\) בתורו של השחקן השני, עד שינצח.
לכן, השחקן הראשון מנצח אם ורק אם \(G(n)\neq 0\).

\end{proof}
\begin{proposition}
מהעקרון של גנבת אסטרטגיה נקבל כי הנימר הוא לא אפס לכל קבוצה סדורה חלקית שיש לה סופרמום.

\end{proposition}
\begin{definition}[סכום של משחקים]
ניתן לאחד מספר משחקים בלתי תלויים ולשחק מאחד המשחקים כל תור. אם \(G_{1},G_{2},\dots,G_{n}\) משחקים אזי הסכום מסומן על ידי:
$$G=G_{1}+G_{2}+\cdot\cdot\cdot+G_{n}$$

\end{definition}
\begin{proposition}
אם משחק \(G=G_{1}+G_{2}+\cdot\cdot\cdot+G_{n}\)  הוא סכום של משחקים כך שלכל משחק \(G_{i}\) יש נימר \(g_{i}\) נקבל כי הנימר של הסכום יהיה:
$$G(G_{1}+\cdot\cdot\cdot+G_{n})=g_{1}\oplus g_{2}\oplus\cdot\cdot\cdot\oplus g_{n}$$
כאשר \(\oplus\) מייצג \(\text{XOR}\) בינארי. 

\end{proposition}
\begin{proof}
נניח כי \(G_{1},\dots,G_{n}\) הם משחקים קומבינטוריים בלתי תלויים עם נימרים \(g_{1},\dots,g_{n}\). בכל תור, השחקן יכול לבחור משחק אחד ולבצע בו מהלך חוקי. נרצה להראות כי הנימר של הסכום הוא \(g_{1}\oplus g_{2}\oplus\cdots\oplus g_{n}\).
נשתמש באינדוקציה על מספר המשחקים \(n\).
\textbf{בסיס האינדוקציה (\(n=1\)):} ברור כי \(G(G_{1})=g_{1}\).
\textbf{צעד האינדוקציה:} נניח כי הטענה נכונה עבור \(n-1\) משחקים. נוסיף משחק נוסף \(G_{n}\) עם נימר \(g_{n}\). מצב המשחק הוא \((g_{1},\dots,g_{n})\). כל מהלך אפשרי הוא שינוי אחד מהערכים \(g_{i}\) לערך אחר \(g_{i}'\) (בהתאם למהלך חוקי במשחק \(G_{i}\)).
הנימר של מצב \((g_{1},\dots,g_{n})\) מוגדר להיות \(\mathrm{mex}\) של כל הנימרים האפשריים שמתקבלים ממהלכים אפשריים:
$$G(g_{1},\dots,g_{n}) = \mathrm{mex}\left\{ G(g_{1},\dots,g_{i}',\dots,g_{n}) \mid g_{i}' \text{ מתקבל ממהלך חוקי ב-} G_{i} \right\}
$$
לפי הגדרת הנימר, עבור כל \(i\), כל מהלך במשחק \(G_{i}\) משנה את \(g_{i}\) לערך \(g_{i}'\) אפשרי. הנימר של המצב החדש הוא \(g_{1}\oplus\cdots\oplus g_{i}'\oplus\cdots\oplus g_{n}\).
המספר הקטן ביותר שלא מתקבל על ידי אף מהלך כזה הוא בדיוק \(g_{1}\oplus\cdots\oplus g_{n}\), כי כל ערך אחר מתקבל על ידי שינוי אחד מה-\(g_{i}\) לערך קטן יותר (לפי הגדרת הנימר של משחק חיסור).
לכן, הנימר של הסכום הוא \(g_{1}\oplus g_{2}\oplus\cdots\oplus g_{n}\).

\end{proof}
\begin{example}
נסתכל על משחק של שתי משחקי חיסור, משחק \(A\) עם \(n=3\) ומשחק \(B\) עם \(n=4\). כך שניתן להחסיר כל מספר של מטבעות מכל ערימה. אזי:
$$G(A)=3\qquad  G(B)=4$$
ולכן:
$$G(A+B)=3\oplus 4= 11_{\mathbb{F}  _{2}}\oplus 100_{\mathbb{F}_{2} }=111_{\mathbb{F_{2}} }=7$$
וכיוון שזהו שונה מ-0 נקבל כי יש אסטרטגיה מנצחת! כדי למצוא את האסטרטגיה המנצחת נדרש למצוא את המהלך שהופך את הנימר ל-0(כלומר שיריב בוודאות יפסיד בתור שלו). כיוון שמספר \(\text{XOR}\) עם עצמו הוא תמיד אפס ניתן להוריד מטבע אחד ממשחק \(B\) ולכן:
$$G(A+B)=3\oplus 3 = 0$$
ולכן זה המהלך המנצח.

\end{example}
\begin{summary}
  \begin{itemize}
    \item משחק חיסור הוא משחק קומבינטורי שבו שחקן יכול לקחת מספר מטבעות מהערימה בכל תור.
    \item נימר של משחק חיסור מוגדר באופן רקורסיבי, כאשר \(G(0)=0\) והנימר של מצב \(n\) הוא \(\mathrm{mex}\) של הנימרים של כל המהלכים האפשריים.
    \item סכום של משחקים הוא סכום של נימרים של המשחקים, כאשר הנימר של הסכום הוא \(\oplus\) של כל הנימרים של המשחקים.
    \item שחקן 1 יכול לנצח אם ורק אם הנימר של המשחק הראשוני שונה מ-0.
  \end{itemize}
\end{summary}
\chapter{משחקי Normal Form}

\section{משחקים אסטרטגיים}

\begin{definition}[משחק אסטרטגי]
שלושה \((N,(S_{i})_{i\in N},(u_{i})_{i\in N})\) כאשר \(N\) זה אוסף של שחקנים, \(S_{i}\) זה אוסף סופי של אסטרטגיות עבור השחקן ה-\(i\), ו-\(u_{i}:S_{1}\times\dots \times S_{N}\to \mathbb{R}\) זה פונקציית תגמול של השחקן ה-\(i\).

\end{definition}
\begin{definition}[משחק סכום אפס]
משחק אסטרטגי אשר מקיים:
$$\sum_{i=1}^{N} u_{i} = 0$$

\end{definition}
\begin{corollary}[משחק אסטרטגי דו משתתפי]
עבור שתי שחקנים משחק אסטרטגי עם שתי משתתפים הוא אוסף של אסטרטגיות \(S_{1},S_{2}\) ופונקציות \(u_{1},u_{2}:S_{1}\times S_{2}\to \mathbb{R}\). המשחק יהיה סכום אפס אם:
$$u_{1}(s_{1},s_{2})+u_{2}(s_{1},s_{2})=0\,$$

\end{corollary}
\begin{definition}[טבלת תגמול]
טבלה אשר השורות שלה מייצגות את האסטרטגיות \(s_{1} \in S_{1}\) והעמודות שלה מייצגות את האסטרטגיות \(s_{2} \in S_{2}\). הערכים במשבצות יהיו \((u_{1}(s_{1},s_{2}),u_{2}(s_{1},s_{2}))\) כאשר עבור משחק סכום אפס כיוון שתגמול של אחד זה הפסד של השני מספיק לציין את \(u_{1}(s_{1},s_{2})\). לעיתים נקרא גם מטריצת תגמול.

\end{definition}
\begin{definition}[אסטרטגיה מעורבת]
לעיתים יעיל להכניס הסתברות לאסטרטגיה. תהי \(S_{1}=\left\{  s_{1},\dots,s_{n}  \right\}\) קבוצת אסטרטגיות סופית של שחקן 1. אסטרטגיה מעורבת של שחקן 1 תהיה התפלגות הסתברות מעל \(S_{1}\). כלומר אוסף \(x=\left( x_{1},\dots,x_{n} \right)\) כך ש-\(\sum_{i=1}^{n}x_{i}=1\).

\end{definition}
\begin{remark}
אוסף כל האסטרטגיות המעורבות יהיה הסימפלקס הסתברות \(\Delta(S_{1})\subseteq\mathbb{R}^{n}\) מסומן גם \(\Delta _{S_{1}}\).

\end{remark}
\begin{example}[אבן נייר ומספריים]
מתואר על ידי הטבלה:
$$\begin{array}{c|c|c|c}  & R & P & S \\ \hlineR & 0 & -1 & 1 \\ \hlineP & 1 & 0 & -1 \\ \hlineS  & -1 &1 & 0 
\end{array}$$
לכל אסטרטגיה טהורה קיימת איזשהי אסטרטגיה שמנצחת אותה. לעומת זאת עבור האסטרטגיה המעורבת \(\left( \frac{1}{3},\frac{1}{3},\frac{1}{3} \right)\) לא קיים אסטרגיה ששולטת עליה.

\end{example}
\begin{definition}[תוחלת של אסטרטגיה מעורבת]
נניח כי לשחקן 1 יש אסטרטגיה מעורבת המתוארת על ידי ווקטור \(\vec{x}\) ולשחקן 2 יש אסטרטגיה מעורבת המתוארת על ידי \(\vec{y}\). אזי עבור מטריצת תגמול \(A\) התוחלת של האסטרטגיה המעורבת תהיה:
$$u_{1}(x,y)=x^{T}A y$$

\end{definition}
\begin{example}
נניח כי נתונה המטריצת תגמול והאסטרטגיות המעורבות הבאות:
$$A=\begin{pmatrix}2 & 0 \\1 & 3\end{pmatrix}\qquad \vec{x}=\begin{pmatrix}0.4 \\ 0.6\end{pmatrix}\qquad \vec{y}=\begin{pmatrix}0.5 & 0.5
\end{pmatrix}$$
אזי התוחלת של האסטרטגיות יהיה:
$$u_{1}(x,y)=\begin{pmatrix}0.4&0.6\end{pmatrix}\left( \begin{pmatrix}2 & 0 \\1 & 3\end{pmatrix} \begin{pmatrix}0.5\\0.5\end{pmatrix}\right)=\begin{pmatrix}0.4 & 0.6\end{pmatrix}\begin{pmatrix}1\\2
\end{pmatrix}=0.4+1.2=1.6$$

\end{example}
\begin{definition}[אסטרטגיה אופטימלית טהורה]
אסטרטגיה \(x \in S_{1}\) של שחקן 1 נקראת אופטימלית אם לכל \(x_{i}\in S_{1}\) מתקיים:
$$\min _{y \in S_{2}}u(x,y)\geq \min _{y \in S_{2}}u(x_{i},y)$$

\end{definition}
\begin{remark}
זה אומר שאם אנחנו מסתכלים בטבלה, אנחנו מסתכלים על השורות ולוקחים את הערך המינימלי של כל שורה. האסטרטגיה האופטימלית הטהורה תהיה הערך הכי גדול מבין כל הערכים האלה.

\end{remark}
\begin{corollary}[אסטרטגיית maximin]
התועלת המקסימלית תהיה:
$$\max _{x \in S_{1}}\min _{y\in S_{2}}u(x,y)$$
כאשר האסטרטגיה האופטימלית תהיה ה-\(x\) שעבורו יהיה מקסימלי.

\end{corollary}
\begin{remark}
אינטואיטיבית זה זה המהלך האופטימלי אומר שאנחנו לוקחים את הדבר שמגדלי את התועלת הכי הרבה בהנתן שהשחקן השני לוקח את הדבר שהכי פוגע בנו.

\end{remark}
\begin{proposition}
עבור השחקן השני אסטרטגיה \(y \in S_{2}\) נקראת אופטימלית אם לכל \(y_{i}\in S_{2}\) מתקיים:
$$\max _{x \in S_{1}} u(x,y)\leq \max_{x \in S_{1}}  u(x, y_{i})$$

\end{proposition}
\begin{proof}
בגלל שהמשחק הוא סכום אפס אז נקבל כי התועלת של השחקן השני היא מינוס התועלת של הפונקציית תועלת. לכן כדי להפטר מהמינוס ניתן להפוך את המינימום למקסימום ולהפוך את סימן האי שיוויון.

\end{proof}
\begin{corollary}[אסטרטגיית minimax]
עבור השחקן השני התועלת המקסימלית תהיה:
$$\min _{y \in S_{2}}\max _{x \in S_{1}}u(x,y)$$

\end{corollary}
\begin{definition}[ערך של משחק טהור]
אם:
$$\min_{y \in S_{2}} \max _{x \in S_{1}}u(x,y)=\max_{x \in S_{1}}\min _{y \in S_{2}} u(x,y)$$
נאמר כי ערכים אלו הם הערך של המשחק. זה יהיה המהלך האופטימלי עבור שתי השחקנים.

\end{definition}
\begin{remark}
עבור אסטרטגיות טהורות לא תמיד קיים ערך כזה. במקרה זה לכל אסטרטגיה שיש לשחקן אחד יש איזשהי אסטרטגיה של השחקן השני אשר יכול לסתור אותה. במקרה זה אין שיווי משקל והדבר הכי טוב יהיה אסטרטגיה מעורבת.

\end{remark}
\begin{definition}[אסטרטגיה אופטימלית מעורבת]
אסטרטגיה \(\vec{x} \in \Delta _{S_{1}}\) נקראת אומפטימלית עבור שחקן 1 אם לכל \(\vec{x}_{i}\in \Delta _{S_{1}}\) מתקיים:
$$\min _{y \in \Delta_{S_{2}}}u(x,y)\geq \min _{y \in \Delta_{S_{2}}}u(x_{i},y)$$
כאשר אסטרטגיה \(\vec{y} \in \Delta_{S_{2}}\) נקראת אופטימלית עבור שחקן 2 אם לכל \(\vec{y}_{i}\in \Delta_{S_{2}}\) מתקיים:
$$\max _{x \in \Delta_{S_{1}}}u(x,y)\leq \max _{x \in \Delta_{S_{1}}}u(x,y_{i})$$

\end{definition}
\begin{corollary}
עבור משחק מעורב האסטרטגיה האופטימלית של השחקן הראשון נתונה על ידי:
$$\max_{x \in \Delta_{S_{1}}}\min _{y \in \Delta_{S_{2}}} u(x,y)$$
כאשר עבור השחקן השני:
$$\min_{y \in \Delta_{S_{2}}} \max _{x \in \Delta_{S_{1}}}u(x,y)$$

\end{corollary}
\begin{definition}[ערך של משחק מעורב]
כאשר:
$$\min_{y \in \Delta_{S_{2}}} \max _{x \in \Delta_{S_{1}}}u(x,y)=\max_{x \in \Delta_{S_{1}}}\min _{y \in \Delta_{S_{2}}} u(x,y)$$
אז ערך זה נקרא ערך של המשחק.

\end{definition}
\begin{remark}
תמיד קיים ערך של משחק מעורב עבור משחק סכום אפס. זה נובע ממשפט המינימקס. 

\end{remark}
\begin{example}
נסתכל על המשחק:

$$\begin{array}{c|c|c}  & L & R\\ \hline  U & 2 & 4\\ \hline D & 3 & 1\\ \hline
\end{array}$$
נגדיר אסטרטגיה מעורבת כללית \(\vec{x}=(p,1-p)\) המינימאקס התועלת של אסטרטגיה זו תהיה:
$$u(x,L)=2\cdot p+3\cdot(1-p)=3-p\qquad u(x,R)=4\cdot p +1\cdot(1-p)=1+3p$$
ונדרש למצוא:
$$\max _{p}\min \{ 3-p,1+3p \}$$
כאשר המקסימום של המינימום מתקבל בחיתוך, כלומר כאשר \(3-p=1+3p\) או \(p=\frac{1}{2}\) ובמקרה זה התועלת הממוצעת תהיה:
$$\min_{p} (3-p,1+3p)=\frac{1}{2}+\frac{3}{2}=2.5$$
ולכן האסטרטגיה המעורבת הכי טובה של שחקן 1 יהיה לבחור כל אופציה בהסתברות \(\frac{1}{2}\), כאשר התועלת הממוצעת במקרה זה יהיה \(2.5\).
עבור השחקן השני באופן דומה נסתכל על המקסימום של העמודות. נגדיר \(\vec{y}=(q,1-q)\) ונקבל:
$$u(U,y)=2q+4(1-q)=4-2q\qquad u(L,y)=3q+1-q=2q+1$$
וכדי למצוא את המקסימום נקבל:
$$\operatorname*{min}_{q}\operatorname*{max}\{4-2q,2q+1\}$$
וכיוון שיש לנו שתי פונקציות לינאריות מונוטוניות המינימום של המקסימום מתקבל בחיתוך, כלומר כאשר \(4-2q=2q+1\) או \(q=\frac{3}{4}\) כאשר התועלת הממוצעת תהיה \(2.5\) ואכן הערכים הם אותו הדבר!

\end{example}
\begin{remark}
בדוגמא הזאת זה נראה כאילו לא התייחסנו לזה שהשחקן השני משחק גם אסטרטגיה מעורבת, אך כיוון שבפונקציות לינאריות מעל קבוצות קמורותהנקודות קיצון הם תמיד על השפה, ניתן להניח כאילו עושים אסטרטגיות טהורות ולקבל את אותה התוצאה.

\end{remark}
\begin{example}[מציאת אסטרטגיה אופטימלית בעזרת גזירה]
נניח כי נתון המשחק סכום אפס הבא:
$$\begin{array}{c|c|c}  & L & R\\ \hline  U & 2 & 0\\ \hline D & 0 & 1\\ \hline
\end{array}$$
כאשר השחקן הראשון בוחר \(U\) או \(D\) והשחקן השני בוחר \(L\) או \(R\). נסמן את ההסתברות של השחקן הראשון לבחור \(U\) ב-\(p\) ושל השחקן השני לבחור \(L\) ב-\(q\). כעת התוחלת של השחקן הראשון תהיה:
$$E(p,q)=2\cdot P(U,L)+0\cdot P(U,R)+0\cdot P(D,L)+1 \cdot P(D,R)=2pq+(1-p)(1-q)=1-p-q+3pq$$
כעת נדרוש כי עבור התוחלת האופטימלית:
$$\operatorname*{max}_{p\in[0,1]}\operatorname*{min}_{q\in[0,1]}E(p,q)=\operatorname*{min}_{q\in[0,1]}\operatorname*{max}_{p\in[0,1]}E(p,q)$$
נמצא את המינימום והמקסימום בעזרת גזירה חלקית:
$$\frac{\partial E}{\partial p} =-1+3q\overset{!}{=} 0\implies q=\frac{1}{3}\qquad \frac{\partial E}{\partial q} =-1+3p\overset{!}{=}  0\implies q=\frac{1}{3}$$
ולכן הנקודה הקריטית ב-\((p,q)=\left( \frac{1}{3},\frac{1}{3} \right)\). לכן זוהי נקודת האוכף של הפונקציה \(E(p,q)\) אשר מבטיחה שלשחקן הראשון ולשני יהיה תוחלת של \(\frac{1}{3}\).

\end{example}
\begin{summary}
  \begin{itemize}
    \item משחק אסטרטגי הוא משחק שבו לכל שחקן יש אסטרטגיות שונות המסומנות ב-\(S_{i}\) והוא בוחר אסטרטגיה אחת. יש פונקציית תועלת אשר מחזירה מספר ממשי מתאים לכל השילוב של האסטרטגיות של השחקנים.
    \item משחק סכום אפס הוא משחק אסטרטגי שבו סכום התועלות של השחקנים הוא אפס.
    \item אסטרטגיה אופטימלית טהורה של השחקן הראשון היא האסטרטגיה עם התועלת הכי גבוהה בהנתן שהשחקן השני בוחר את המהלך שיביא את התועלת הכי נמוכה. הערך של התועלת תהיה:
$$\max _{x \in S_{1}}\min _{y \in S_{2}}u(x,y)$$
    \item באותו אופן האסטרטגיה האופטימלית הטהורה של השחקן השני היא האסטרטגיה עם התועלת הכי נמוכה בהנחה שהשחקן הראשון בוחר את האסטרטגיה עם התועלת הכי גבוהה. הערך של התועלות האלו יהיה:
$$\min _{y \in S_{2}}\max _{x \in S_{1}}u(x,y)$$
    \item אסטרטגיה מעורבת זה אסטרטגיה הסתברותית, לוקחים בהסתברות כלשהי את אחת האסטרטגיות הטהורות. זה יהיה איבר בסימפלקס הסתברות \(\Delta_{S_{1}}\) עבור השחקן הראשון ו-\(\Delta_{S_{2}}\) עבור השחקן השני.
    \item ערך של משחק הוא התועלת הממוצעת של השחקן הראשון כאשר שני השחקנים משחקים את האסטרטגיה האופטימלית שלהם.
    \item משפט המינימקס טוען כי עבור משחק סכום אפס תמיד קיים ערך של המשחק, כלומר תמיד קיים אסטרטגיה אופטימלית טהורה או מעורבת עבור השחקן הראשון והשני.
  \end{itemize}
\end{summary}
\section{משפט המינימקס}

\begin{proposition}
עבור כל משחק מתקיים:
$$\operatorname*{max}_{x\in\Delta_{m}}\operatorname*{min}_{y\in\Delta_{n}}x^{T}A y\leq\operatorname*{min}_{y\in\Delta_{n}}\operatorname*{max}_{x\in\Delta_{m}}x^{T}A y$$

\end{proposition}
\begin{proof}
  \begin{enumerate}
    \item נבחר אסטרטגיה מעורבת כלשהי \(x \in \Delta _m\) ו-\(y \in \Delta_{n}\). 


    \item מההגדרה של מינימום ומקסימום: 
$$\min _{y'}x^{T}Ay'\leq x^{T}Ay\leq \max _{x'}(x')^{T}Ay$$


    \item כיוון שזה נכון לכל \(x,y\) זה זה בפרט נכון עבור ה-\(y\) הקטן ביותר עבור אגף ימין ועבור ה-\(x\) הגדול ביותר עבור אגף שמאל: 
$$\operatorname*{max}_{x}\operatorname*{min}_{y}x^{T}A y\leq\operatorname*{min}_{y}\operatorname*{max}_{x}x^{T}A y.$$


  \end{enumerate}
\end{proof}
\begin{proposition}[משפט המישור העל המפריד]
יהי \(C\) קבוצה סגורה וקמורה כך ש-\(0 \not \in C\). אזי קיים על מישור מפריד, כלומר קיימים ווקטורים \(\vec{a},\vec{b}\in \mathbb{R}^{n}\) כך שלכל \(x \in C\) מתקיים:
$$\langle a,x \rangle > b > 0$$

\end{proposition}
\begin{proof}
אם \(C\) ריק אז הטענה מתקיימת טריוויאלית. לכן נניח כי \(C\) לא ריקה. נבחר \(r> 0\) כך ש:
$$C\cap B_{r}(0)\neq \varnothing $$
כאשר זוהי קבוצה סגורה כחיתוך של שתי קבוצות סגורות. כיוון ש-\(\lVert x \rVert\leq r\) לכל \(x\) בקבוצה נקבל כי קומפקטי כי סגור וחסום(מהיינה בורל). לכן הפונקציה הרציפה \(\lVert \cdot \rVert^{2}\) משיגה את המינימום בקבוצה. נסמן את המינימום ב-\(\vec{a}\).
כעת אם \(x \in C\setminus B_{r}(0)\) נקבל:
$$\left\lVert  \vec{a}  \right\rVert \leq r^{2}\leq\left\lVert  \vec{x}  \right\rVert $$
נגדיר כעת את הפונקציה:
$$g_{x}(t)=\left\lVert  t\vec{x}+(1-t)\vec{a}  \right\rVert ^{2}$$
זוהי פונקציה רציפה כהרכבה של רציפות, אשר משיגה את המינימום שלה ב-\(t=0\). נזכור כי הנגזרת של נורמה נתונה על ידי:
$${\frac{d}{d t}}\|\mathbf{v}(t)\|^{2}={\frac{d}{d t}}\langle\mathbf{v}(t),\mathbf{v}(t)\rangle=\langle\mathbf{v}^{\prime}(t),\mathbf{v}(t)\rangle+\langle\mathbf{v}(t),\mathbf{v}^{\prime}(t)\rangle=2\langle\mathbf{v}(t),\mathbf{v}^{\prime}(t)\rangle$$
ולכן אם נגזור את הפונקציה לפי \(t\) נקבל:
$$g_{x}'(t)=2\left\langle  t\vec{x}+(1-t)\vec{a},\vec{x}-\vec{a}  \right\rangle$$
ועבור \(t=0\) נקבל מינימום(אשר על השפה) לכן:
$$g'_{x}(0)=2\left\langle  \vec{a},x-\vec{a}  \right\rangle \geq 0$$
ולכן:
$$\left\langle  \vec{a},\vec{x}  \right\rangle -\left\langle  \vec{a},\vec{a}  \right\rangle \geq 0\implies \left\langle  \vec{a},\vec{x}  \right\rangle \geq \left\langle  \vec{a},\vec{a}  \right\rangle \geq 0$$
ובפרט עבור אם נגדיר \(b:=\|a\|^{2}>0\) נקבל:
$$0\leq b\leq \left\langle  \vec{a},\vec{x}  \right\rangle $$
וקיבלנו כי העל מישור \(\left\{  y\mid \langle a,y \rangle=b  \right\}\) הוא על מישור מפריד.

\end{proof}
\begin{theorem}[מינימקס]
עבור משחק סכום אפס סופי של שתי שחקנים נקבל:
$$\operatorname*{max}_{x\in\Delta(S_{1})}\operatorname*{min}_{y\in\Delta(S_{2})}u(x,y)=\operatorname*{min}_{y\in\Delta(S_{2})}\operatorname*{max}_{x\in\Delta(S_{1})}u(x,y)$$
כלומר אם \(A\) זה מטריצת התגמול של שחקן 1 נקבל:
$$\operatorname*{max}_{x\in\Delta(A_{1})}\operatorname*{min}_{y\in\Delta(A_{2})}x^{T}A y=\operatorname*{min}_{y\in\Delta(A_{2})}\operatorname*{max}_{x\in\Delta(A_{1})}x^{T}A y$$

\end{theorem}
\begin{proof}
  \begin{enumerate}
    \item הכיוון \(\operatorname*{max}_{x\in\Delta_{m}}\operatorname*{min}_{y\in\Delta_{n}}x^{T}A y\leq\operatorname*{min}_{y\in\Delta_{n}}\operatorname*{max}_{x\in\Delta_{m}}x^{T}A y\) נכון גם עבור משחק שהוא לא סכום אפס מהטענה לעיל. נראה את הכיוון השני. 


    \item נניח בשלילה שלא מתקיים האי שיוון בכיוון השני. כלומר מתקיים: 
$$\operatorname*{max}_{x}\operatorname*{min}_{y}x^{T}A y<\operatorname*{min}_{y}\operatorname*{max}_{x}x^{T}A y$$
ובפרט קיים איזשהו \(\lambda \in \mathbb{R}\) כך ש:
$$\operatorname*{max}_{x}\operatorname*{min}_{y}x^{T}A y<\lambda<\operatorname*{min}_{y}\operatorname*{max}_{x}x^{T}A y$$
הסתירה שנרצה להראות היא שלכל \(x\) מתקיים \(\operatorname*{min}_{y}x^{T}A y\leq\lambda\).


    \item נגדיר מתריצת תגמול מוזזת \(\hat{A}=A-\lambda \cdot \mathbb{1}\) כאשר \(\mathbb{1}\) זה המטריצה שכל הערכים של 1, כלומר ערכי המטריצה יהיו \(\hat{a}_{ij}=a_{ij}-\lambda\). 


    \item נגדיר את הקבוצה: 
$$K=\{\hat{A}y+v\mid y\in\Delta_{n},v\geq0\},$$
כאשר \(v\geq 0\) אומר שגדול מ-0 בכל קורדינטה. כלומר זהו אוסף של כל הווקטורים אשר שולטים על איזשהו ווקטור \(\hat{A}y\).


    \item נשים לב כי \(K\) קמורה בתור העתקה לינארית על קבוצה קמורה וכן סגורה באופן דומה, לכן ניתן להפעיל את משפט ההפרדה של העל מישור ונקבל \(z \in \mathbb{R}^{m}\) וסקלר \(c> 0\) כך ש לכל \(w \in K\) מתקיים: 
$$z^{T}w>c> 0 \qquad  z^{T}0=0<c$$


    \item נגדיר את \(x\) להיות הנרמול של \(x\) כדי שייצג הסתברות(כלומר שסכום הרכיבים שלו יהיה 1): 
$$s \equiv \sum_{i=1}^{m} z_{i}\qquad x=\frac{z}{s}\in \Delta_{m}$$


    \item כעת: 
$$x^{T}\hat{A}y=\frac{z^{T}\hat{A}y}{s}>\frac{c}{s}>0\implies \operatorname*{min}_{y\in\Delta_{n}}x^{T}{\hat{A}}y>0$$
וכיוון ש-\(\hat{A}=A-\lambda \mathbb{1}\) נקבל:
$$\operatorname*{min}_{y}x^{T}A y-\lambda>0\implies\operatorname*{min}_{y}x^{T}A y>\lambda$$
וזה סותר את ההנחה שאין \(x\) שמקיים \(\min_{y}x^{T}Ay> 0\) ולכן מתקיים האי שיוויון בכיוון השני והטענה מתקיימת.


  \end{enumerate}
\end{proof}
\begin{summary}
  \begin{itemize}
    \item משפט המינימקס טוען כי עבור משחק סכום אפס תמיד קיים ערך של המשחק, כלומר תמיד קיים אסטרטגיה אופטימלית טהורה או מעורבת עבור השחקן הראשון והשני:
$$\operatorname*{max}_{x\in\Delta(S_{1})}\operatorname*{min}_{y\in\Delta(S_{2})}u(x,y)=\operatorname*{min}_{y\in\Delta(S_{2})}\operatorname*{max}_{x\in\Delta(S_{1})}u(x,y)$$
    \item ניתן למצוא את הערך של המשחק על ידי הגדרה של \((p,1-p)\) עבור ההסתברויות לבחור משחק, ומספיק להשוואות לאסטרטגיות הטהורות מקמירות.
  \end{itemize}
\end{summary}
\section{שיווי משקל נאש}

\begin{definition}[שיווי משקל נאש טהור]
בהינתן משחק אסטרטגי שילוב של אסטרטגיות שאופטימליות גם לשחקן הראשון וגם לשחקן השני נקרא שיווי משקל נאש. כלומר זהו שילוב של אסטרטגיות \((x^{*},y^{*})\in S_{1} \times S_{2}\) כך שלכל \(x \in S_{1}\) מתקיים:
$$u_{1}(x^{*},y^{*})\geq u_{1}(x,y^{*})$$
ולכל \(y \in S_{2}\) מתקיים:
$$u_{2}(x^{*},y^{*})\geq u_{2}(x^{*},y)$$

\end{definition}
\begin{remark}
אם מסתכלים על הטבלה זה הערך שבו הערך של שחקן 1 הכי גבוה באותה עמודה(כיוון שעבור \(y^{*}\) מקובע לשנות את השורה זה להסתכל על הערכים בעמודה) והתועלת של שחקן 2 זה הערך הכי גבוה באותה שורה.

\end{remark}
\begin{definition}[שיווי משקל נאש מעורב]
זהו שילוב של אסטרטגיות \((x^{*},y^{*})\in \Delta_{S_{1}} \times \Delta_{S_{2}}\) כך שלכל \(x \in \Delta _{S_{1}}\) מתקיים:
$$u_{1}(x^{*},y^{*})\geq u_{1}(x,y^{*})$$
ולכל \(y \in\Delta_{S_{2}}\) מתקיים:
$$u_{2}(x^{*},y^{*})\geq u_{2}(x^{*},y)$$

\end{definition}
\begin{definition}[תומך של אסטרטגיה מעורבת]
עבור אסטרטגיה מעורבת ה קבוצה של אסטרטגיות טהורות עבורה השחקן נותן הסתברות חיובית.

\end{definition}
\begin{example}
אם לשחקן 1 יש 4 אסטרטגיות טהורות \(S_{1}=\{ s_{1},s_{2},s_{3},s_{4} \}\) אז אסטרטגיה מעורבת אחת יכולה להיות:
$$x=(0.5,0.5,0,0)$$
כאשר התומך שלה יהיה \(\{ s_{1},s_{2} \}\).

\end{example}
\begin{proposition}[עקרון העדישות]
כל נקודות שיווי משקל נאש של אסטרטגיה מעורבת הם בעלות הסתברות שווה(עדישות) לכל אחד מהאסטרטגיות הטהורות בתמוך שלהם.

\end{proposition}
\begin{remark}
למעשה ניתן להסתכל על כל הזוגות תומכים האפשריים ולמצוא מתי עדישים. אבל נשים לב שאם מסתכים על תומכים יחידונים זו תהיה אסטרטגיה טהורה והשיווי משקל המתקבל יהיה טהור.

\end{remark}
\begin{example}
נסתכל על המשחק של האם זוג צריך ללכת לאופרה או לכדורגל:
$$\begin{array}{c|c|c} & O & F \\ \hlineO & (2,1) & (0,0) \\ \hlineF & (0,0) & (1,2)
\end{array}$$
השיווי משקל נאש הטהורים הם \((2,1)\) ו-\((1,2)\). כעת לפי עקרון העדישות נקבל עוד שייוי משקל אם שני השחקנים עדישים לגבי הבחירה שלהם. נניח כי לשחקן 1 יש אסטרטגיה מעורבת \(\vec{x}=(p,1-p)\) ונקבל כי התגמול הממוצע יהיה:
$$u_{1}\left( \vec{x},O \right)=2p+0\cdot(1-p)=2p\qquad u_{1}\left( \vec{x},F \right)=0\cdot p+1\cdot(1-p)=1-q$$
ונהיה עדישים במקרה של שיוויון:
$$u_{1}\left( \vec{x},O \right)\overset{!}{=} u_{1}\left( \vec{x},F \right)\implies 2p\overset{!}{=}  1-p\implies p=\frac{1}{3}$$
ואם נניח כי עבור השחקן השני יש אסטרטגיה מעורבת \(\vec{y}=(q,1-q)\) נקבל:
$$u_{2}\left( O,\vec{y} \right)=1\cdot q+0\cdot(1-q)=q\qquad u_{2}\left( F,\vec{y} \right)=0\cdot q+2(1-q)=2-2q$$
ואם נדרוש עדישות נקבל:
$$u_{2}\left( O,\vec{y} \right)\overset{!}{=} u_{2}\left( F,\vec{y} \right)\implies q=2-2q\implies q=\frac{2}{3}$$
ולכן נקבל כי \(\left( \left( \frac{2}{3},\frac{1}{3} \right),\left( \frac{1}{3},\frac{2}{3} \right)  \right)\) היא אסטרטגיה מעורבת.

\end{example}
\begin{example}
נרצה למצוא את השיווי משקל נאש של המשחק:
$$\begin{array}{c|c|c} & X & Y \\ \hlineA & (7,5) & (1,3) \\ \hlineB & (4,4) & (7,7) \\ \hlineC & (5,2) & (4,5) 
\end{array}$$
נשים לב מיידית כי \((7,7)\) ו-\((7,5)\) הם אסטרטגיות טהורות שולטות. כעת נסתכל על כל הזוגות תומכים:
$$\left( \{ A,B \},\{ X,Y \} \right),\quad \left( \{ B,C \},\{ X,Y \} \right),\quad \left( \{ A,C \},\{ X,Y \} \right),\quad \left( \{ A,B,C \},\{ X,Y \} \right)$$
עבור הזוג הראשון נגדיר \(\vec{x}=(p,1-p)\) ו-\(\vec{y}=(q,1-q)\). נדרוש עבור השחקן הראשון:
$$u_{1}\left( A,\vec{y} \right)=7\cdot q+(1-q)\cdot 1=6q+1\qquad u_{1}\left( B,\vec{y} \right)=4q+7(1-q)=-3q+7$$
נדרוש שיוויון:
$$6q+1=-3q+7\implies 9q=6\implies q=\frac{2}{3}$$
כעת עבור השחקן השני:
$$u_{2}\left( \vec{x},X \right)=5\cdot p+4\cdot(1-p)=p+4\qquad u_{2}\left( \vec{x},Y \right)=3\cdot p+7\cdot(1-p)=-4p+7$$
נדרוש שיוויון:
$$p+4=-4p+7\implies 5p=3\implies p=\frac{3}{5}$$
ולכן \(\left( \left( \frac{2}{3},\frac{1}{3} \right),\left( \frac{3}{5},\frac{2}{5} \right) \right)\) אסטרטגיה מעורבת. ניתן לעשות באופן דומה על הזוגות ונראה על הזוג עם התמוך באורך 3. נגדיר עבור שחקן 1:
$$p_{1},p_{2},p_{3}\ge0,\quad p_{1}+p_{2}+p_{3}=1$$
כך ש-\(\vec{x}=(p_{1},p_{2},p_{3})\) ועבור שחקן 2 נגדיר \(\vec{y}=(q,1-q)\). עבור השחקן הראשון נדרש:
$$u_{1}\left( A,\vec{y} \right)=u_{1}\left( B,\vec{y} \right)=u_{1}\left( C,\vec{y} \right)$$
ולמעשה מספיק לנו למצוא בעזרת שתי משוואות את הערך של \(q\), המשוואה השלישית מאפשרת לנו לפסול את הפתרון. נקבל:
\begin{gather*}u_{1}\left( A,\vec{y} \right)=7q+1(1-q)=7q+1-q=6q+1  \\u_{2}\left( B,\vec{y} \right)=4q+7(1-q)=4q+7-7q=-3q+7 \\u_{3}\left( C,\vec{y} \right)=5q+4(1-q)=5q+4-4q=q+4
\end{gather*}
כלומר נקבל את המשוואות:
$$6q+1=-3q+7\quad\mathrm{and}\quad6q+1=q+4$$
מהמשוואה הראשונה נקבל \(q=\frac{2}{3}\) כאשר מהמשוואה השנייה נקבל \(q=\frac{3}{5}\) בסתירה, ולכן לא קיים שיווי משקל נאש.

\end{example}
\begin{summary}
  \begin{itemize}
    \item אוסף של אסטרטגיות הוא שיווי משקל נאש אם לכל שחקן בהנתן המהלך של השחקנים האחרים האסטרטגיה שלו היא הטובה ביותר.
    \item עבור מקרה של שתי שחקנים, זה אומר עבור השחקן הראשון בהנתן שהשחקן השני בחר עמודה, אז מה השורה הטובה ביותר שיכול להיות.
    \item תמיד קיימת לפחות נקודת שיווי משקל נאש אחת מעורבת.
    \item ניתן למצוא את שיווי המשקל נאש על ידי חיפוש של זוגות תומכים, כאשר עבור כל זוג נדרוש שיוויון בין התועלות של השחקנים(עקרון העדישות).
  \end{itemize}
\end{summary}
\section{שליטה וסימטרייה}

\begin{definition}[משחק סימטרי]
משחק של שתי שחקנים נקרא סימטרי אם המהלכים זההים עבור שתי השחקנים (כלומר \(S_{1}=S_{2}\)) וגם הפונקציות תועלת מקיימות לכל \(i,j \in S_{1}\)$$u_{1}(i,j)=u_{2}(j,i)$$
ובפרט כל אסטרטגיה מנצחת של שחקן 1 תהיה אסטרטגיה מנצחת של שחקן 2.

\end{definition}
\begin{corollary}
עבור משחק סכום אפס כיוון ש-\(u_{1}=-u_{2}\) נקבל כי מטריצת התגמול היא אנטי סימטרית(כלומר \(A^{t}=A\)).

\end{corollary}
\begin{proposition}
הערך של משחק סכום אפס סימטרי הוא אפס.

\end{proposition}
\begin{proposition}[סימטרייה תחת תמורות]
אם קיים מטריצת תמורה \(P\) כך שהמטריצת תגמול מקיימת \(A=P^{T}AP\) אזי גם האסטרטגיות המעורבות מקיימות \(x \in \Delta _{S_{1}}\) ו-\(\vec{y}=\Delta_{S_{2}}\) מקיימות \(\vec{x}=P\vec{x}\) ו-\(\vec{y} = P\vec{y}\).

\end{proposition}
\begin{proof}
אם \(A=P^{T}AP\) אזי:
$$\vec{x}^{T}A\vec{y}=\vec{x}^{T}P^{T}AP\vec{y}=\left( P\vec{x} \right)^{T}A\left( P\vec{y} \right)$$
וכעת נקבל \(P\vec{x}=\vec{x}\) ו-\(P\vec{y} = \vec{y}\).

\end{proof}
\begin{example}
עבור מטריצת התועלת:
$$A=\begin{pmatrix}5 & 4 & -1 \\3 & 2 & 3 \\-1 & 4 & 5
\end{pmatrix}$$
נשים לב כי:
$$\begin{pmatrix}5 & 4 & -1 \\3 & 2 & 3 \\-1 & 4 & 5\end{pmatrix}\xrightarrow{R_{1}\leftrightarrow  R_{3}}\begin{pmatrix}-1 & 4 & 5 \\3 & 2 & 3 \\5 & 4 & -1\end{pmatrix}\xrightarrow{C_{1}\leftrightarrow  C_{3}}\begin{pmatrix}5 & 4 & -1 \\3 & 2 & 3 \\-1 & 4 & 5
\end{pmatrix}$$
כלומר קיבלנו את אותה מטריצה. כאשר זוהי למעשה אומר אינווריאנטיות תחת המטריצה:
$$P=\begin{pmatrix}0 & 0 & 1 \\0 & 1 & 0 \\1 & 0 & 0
\end{pmatrix}$$
כאשר נזכור כי המטריצה המתאימה להחלפה של שורות היא כפל בימין ב-\(P^{T}\)(אשר במקרה שלנו שווה ל-\(P\)) והמטריצה המתאימה להחלפה של עמודה מתאימה לכפל בשמאל ב-\(P\). לכן אם נגדיר \(\vec{x}=(p_{1},p_{2},p_{3})\) נקבל \(\vec{x}=P\vec{x}\) כלומר \(p_{1}=p_{3}\). באופן דומה עבור \(\vec{y}=(q_{1},q_{2},q_{3})\) נקבל \(q_{1}=q_{3}\).

\end{example}
\begin{example}
נסתכל על המשחק סכום אפס הבא:
$$\begin{array}{c|c|c|c} & D & E & F  \\ \hline A & 0 & -1 & -2 \\ \hline B & 1 & 0 & -1  \\ \hlineC & -2 & 1 & 0 
\end{array}$$
כאשר כיוון שזהו משחק סימטרי הערך של המשחק יהיה 0. נרצה למצוא אסטרטגיה מנצחת \(x^{*}=(x_{1},x_{2},x_{3})\) עבור שחקן \(1\). מתקיים:
$$0=\operatorname*{min}_{y\in\Delta_{3}}U(x^{*},y)=\operatorname*{min}_{1\leq i\leq3}U(x^{*},e_{i})=\operatorname*{min}_{1\leq i\leq3}(x^{*})^{T}A e_{i}=\operatorname*{min}(x_{2}-2x_{3},-x_{1}+x_{3},2x_{1}-x_{2})$$
תחת האילוץ \(x_{1}+x_{2}+x_{3}=1\). זה נותן את המערכת אי שוויונות:
$$2x_{3}\leq x_{2}\qquad x_{1}\leq x_{3}\qquad 2x_{1}\geq x_{2}$$
וניתן לקבל מזה את השרשרת אי שוויונות:
$$2x_{3}\leq x_{2}\leq {2}x_{1}\leq {2}x_{3}\implies 2x_{1}=2x_{3}=x_{2}$$
ומהאילוץ \(x_{1}+x_{2}+x_{3}=1\) נקבל \(x^{*}=\left( \frac{1}{4},\frac{1}{2},\frac{1}{4} \right)\). כיוון שהמשחק סימטרי נקבל \(y^{*}=\left( \frac{1}{4},\frac{1}{2},\frac{1}{4} \right)\).

\end{example}
\begin{definition}[אסטרטגיה מעורבת שולטת]
יהיו \(x,x' \in \Delta_{n}\) אסטרטגיות מעורבות של השחקן ה-\(i\).

  \begin{enumerate}
    \item האסטרטגיה \(x\) שולטת חזק על \(x'\) אם לכל \(y \in \Delta_{m}\) מתקיים: 
$$u_{i}(x,y)> u_{i}(x',y)$$


    \item אסטרטגיה \(x\) שולטת חלש על \(x'\) אם לכל \(y \in \Delta_{m}\) מתקיים: 
$$u_{i}(x,y)\geq u_{i}(x',y)$$


  \end{enumerate}
\end{definition}
\begin{proposition}
אם אסטרטגיה \(e_{i}\) נשלטת חזק, אז באסטרטגיה האופטימלית \(x^{*}_{i}=0\).

\end{proposition}
\begin{proof}
נניח \(\sigma_{i}\) אסטרטגיה שולטת חזק של שחקן 1. כלומר לכל 

\end{proof}
\begin{proposition}
אם אסטרטגיה \(e_{i}\) נשלטת חלש במשחק סכום אפס \(A\), אז הסרה של האסטרטגיה יוצרת משחק חדש \(A'\) כך ש:

  \begin{enumerate}
    \item הערך של \(A'\) זהה לערך של \(A\). 


    \item כל אסטרטגיה אופטימלית של \(A'\) תהיה אסטרטגיה אופטימלית של \(A\). 


  \end{enumerate}
\end{proposition}
\begin{example}
נתחיל מהמשחק הבא:
$$\begin{pmatrix}7,5 & 5,2 & 1,1 & 1,3 \\4,4 & 2,6 & 0,1 & 7,7 \\5,2 & 3,0 & 2,3 & 4,5 \\0,4 & 4,8 & 5,6 & 0,6
\end{pmatrix}$$
נשים לב כי העמודה הרביעית שולטת(חלש) על העמודה השלישית, לכן ניתן להסתכל על המשחק המצומצם:
$$\begin{pmatrix}7,5 & 5,2 &  1,3 \\4,4 & 2,6 &  7,7 \\5,2 & 3,0 &  4,5 \\0,4 & 4,8 &  0,6
\end{pmatrix}$$
וכעת השורה הראשונה שולטת על השורה הרביעית:
$$\begin{pmatrix}7,5 & 5,2 &  1,3 \\4,4 & 2,6 &  7,7 \\5,2 & 3,0 &  4,5 \\
\end{pmatrix}$$
וכעת העמודה השלישית שולטת על העמודה השנייה:
$$\begin{pmatrix}7,5 &  1,3 \\4,4 &   7,7 \\5,2 &   4,5 \\
\end{pmatrix}$$
ולא נראה שיש מכאן אף אסטרטגיה שולטת טהורה. ולכן מספיק לפתור את המשחק הזה, אשר עשינו בעבר.

\end{example}
\chapter{משחקים שיתופיים}

\section{משחקים קואליציונים}

\begin{definition}[קואליציה]
תהי \(N\) קבוצה שחקנים. תת קבוצה \(S\subseteq N\) נקראת קואליציה. כאשר \(S=N\) נאמר כי על קאליציה(grand coalition)

\end{definition}
\begin{definition}[משחק קואליציוני]
זוג \(G=(N,V)\) כאשר \(N\) היא קבוצת השחקנים, ו-\(V\) היא פונקציה \(V:2^{N}\to \mathbb{R}\) המקיימים \(V\left( \varnothing \right)=0\).

\end{definition}
\begin{remark}
ה-\(2^{N}\) זה כמות הדרכים לקבל שילוב של שחקנים אפשריים בקואליציה(עבור כל שחקן או שנמצא, או שלא נמצא) ולכן זה למעשה פונקציה שמקבלת קואליציה אפשרית ותחזיר את התועלת אם ישתפו פעולה.

\end{remark}
\begin{example}
נניח כי שלושה חברות \(A,B,C\) רוצות לשצף פעולה, כך שכל חברה לבד מפיקה את התועלת:
$$v\left( \{A\} \right)=2,\quad v\left( \{B\} \right)=3,\quad v\left( \{C\} \right)=1$$
ועם משתפים פעולה יקבלנו יותר תועלת:
$$v\left( \{A,B\} \right)=7,\quad v\left( \{B,C\} \right)=5,\quad v\left( \{A,C\} \right)=4$$
כך שאם שלושתם משתפים פעולה יקבלנו אפילו יותר תועלת:
$$v\left( \{ A,B,C \} \right)=10$$

\end{example}
\begin{example}[משחק הכפפות]
נניח שיש לנו 3 כפפות, מתוכם שתים הם שמליות, \(\left( \{ L_{1},L_{2} \} \right)\) וכפפה אחת היא ימנית (\(\{ R_{1} \}\)). ערך של קואליציה היא מספר הזוגות שניתן ללבוש(זוג זרים). אזי:
$$V\left(\varnothing\right)=V\left(L_{1}\right)=V\left(L_{1},L_{2}\right)=0,\qquad\qquad V\left(L_{1},R_{1}\right)=V\left(L_{2},R_{1}\right)=V\left(L_{1},L_{2},R_{1}\right)=1$$

\end{example}
\begin{definition}[ווקטור תשלומים - Imputation]
דרך לחלק את כל התועלת של הקואליציה \(v(N)\) בין השחקנים. זהו ווקטור \(\vec{x} \in \mathbb{R}^{n}\) אשר כל רכיב \(x_{i}\) בווקטור מייצג כמה השחקן ה-\(i\) מקיים. כלומר זהו ווקטור אשר מקיים:
$$\sum_{i=1}^{n}x_{i}=v{\big(}N{\big)}\quad{\mathrm{and}}\quad x_{i}\geq v(\{i\})\,{\mathrm{for~all}}\,i.$$

\end{definition}
\section{ליבה}

\begin{definition}[ליבה - core]
אוסף ווקטורי תשלומים כך שלאף קואליציה \(S\) אין תמריץ לעזוב. כלומר זוהי קבוצת הווקטורים \(x=\left( x_{1},\dots,x_{n} \right)\) אשר מקיימים:

  \begin{enumerate}
    \item יעילות - \(\sum_{i\in N}x_{i}=v(N)\)


    \item רציונאליות קואליציונית - לכל קואליציה \(S\subseteq N\) מתקיים: 
$$\sum_{i\in S}x_{i}\geq v(S)$$


  \end{enumerate}
\end{definition}
\begin{definition}[משחק קמור]
משחק שיתופי \((N,V)\) נקרא קמור אם לכל קואליציונת \(S,T\subseteq N\) מתקיים:
$$V(S)+V(T)\;\leq\;V(S\cup T)+V(S\cap T)$$

\end{definition}
עבור המשחק עם המשתתפים \(N=\{ 1,2,3 \}\) והפונקציית:
$$V\left( \{ 1 \} \right)=1\qquad V\left( \{ 2 \} \right)=2\qquad V\left( \{ 3 \} \right)=3$$$$V\left( \{ 1,2 \} \right)=4\qquad V\left( \{ 2,3 \} \right)=6\qquad V\left( \{ 1,3 \} \right)=5\qquad V\left( \{ 1,2,3 \} \right)=10$$
נקבל כי המשחק קמור. ניתן לבדוק ידנית:
$$V\left( \{ 1 \} \right)+V\left( \{ 2 \} \right)=3 \leq 4=V\left( \{ 1,2 \} \right)+V\left( \varnothing  \right)$$

\begin{proposition}[marginal vector theorem]
יהי \((N,V)\) משחק שיתופי קמור. כל תמורה \(\pi\) של השחקנים יוצר את הווקטור:
$$m^{\pi}=\left( m_{1}^{\pi},\dots,m_{n}^{\pi} \right)$$
כאשר לשכל שחקן \(i\) נגדיר:
$$m_{i}^{\pi}=V\left( P_{i}^{\pi}\cup \{ i \} \right)-V\left( P_{i}^{\pi} \right)$$
כאשר \(P_{i}^{\pi}\) זה אוסף הקודמים ל-\(i\) בסדר של \(\pi\). אזי:
$$\mathrm{Core}(N,V)=\mathrm{conv}\{\,m^{\pi}:\pi\mathrm{~a~permutation~of}\,N\}.$$
כלומר הסגור הקומור של הווקטורים \(m^{\pi}\).

\end{proposition}
\begin{proof}
  \begin{enumerate}
    \item נראה כי הווקטורים האלו נמצאים בליבה. יהי \(\pi\) תמורה. נקבל: 
$$\sum_{k=1}^{n}m_{\pi(k)}^{\pi}=\sum_{k=1}^{n}\left(v(\{\pi(1),\ldots,\pi(k)\})-v(\{\pi(1),\ldots,\pi(k-1)\})\right)=v(N)$$
עבור כל קואליציה \(S\subseteq N\) נסתכל על השחקנים של \(S\) מסודרים לפי \(\pi\):
$$S=\{\pi(i_{1}),\pi(i_{2}),\ldots,\pi(i_{k})\},\quad i_{1}<i_{2}<\cdots<i_{k}$$
נרצה כעת להראות כי \(\sum_{j=1}^{k}m_{\pi(i_{j})}^{\pi}\geq v(S)\). מהקמירות של המשחק נקבל:
$$v(A\cup\{j\})-v(A)\leq v(B\cup\{j\})-v(B)\quad\mathrm{for~}A\subseteq B\subseteq N\setminus\{j\}$$
לכן ניתן להסתכל על שרשרת האי שיוויונות:
$$\sum_{j=1}^{k}m_{\pi(i_{j})}^{\pi}=\sum_{j=1}^{k}\left(v\left(\bigcup_{l=1}^{i_{j}}\{\pi(l)\}\right)-v\left(\bigcup_{l=1}^{i_{j}-1}\{\pi(l)\}\right)\right)$$
כיוון שהקבוצה \(A_{j}:=\bigcup_{l=1}^{i_{j}-1}\{\pi(l)\}\) מוכלת ב-\(A_{j+1}\) ו-\(\{\pi(i_{j})\}\notin A_{j}\) ולכן:
$$\sum_{j=1}^{k}m_{\pi(i_{j})}^{\pi}\geq v(S)\implies m^{\pi}\in \mathrm{Core}(v)$$


    \item נזכור כי מההגדרה הליבה היא קבוצה קמורה בתור חיתוך של אי שיוויונות לינאריות. לכן  


  \end{enumerate}
\end{proof}
\begin{example}
עבור המשחק מהדוגמא הקודמת נקבל:

  \begin{table}[htbp]
    \centering
    \begin{tabular}{|ccccc|}
      \hline
      \(\pi\) & \(m_1^\pi\) & \(m_2^\pi\) & \(m_3^\pi\) & vector \\ \hline
      \((1 2 3)\) & \(V({1})-V(\varnothing)=1\) & \(V({1,2})-V({1})=3\) & \(V(N)-V({1,2})=6\) & \((1,3,6)\) \\ \hline
      \((1 3 2)\) & \(1\) & \(V(N)-V({1,3})=5\) & \(V({1,3})-V({1})=4\) & \((1,5,4)\) \\ \hline
      \((2 1 3)\) & \(V({2,1})-V({2})=2\) & \(2\) & \(V(N)-V({1,2})=6\) & \((2,2,6)\) \\ \hline
      \((2 3 1)\) & \(V(N)-V({2,3})=4\) & \(2\) & \(V({2,3})-V({2})=4\) & \((4,2,4)\) \\ \hline
      \((3 1 2)\) & \(V({3,1})-V({3})=2\) & \(V(N)-V({1,3})=5\) & \(3\) & \((2,5,3)\) \\ \hline
      \((3 2 1)\) & \(V(N)-V({2,3})=4\) & \(V({2,3})-V({3})=3\) & \(3\) & \((4,3,3)\) \\ \hline
      וכעת: &  &  &  &  \\ \hline
      $$\operatorname{Core}(N,V)=\operatorname{conv}\bigl\{(1,3,6),(1,5,4),(2,2,6),(4,2,4),(2,5,3),(4,3,3)\bigr\}.$$ &  &  &  &  \\ \hline
      כאשר זהו משושה אשר נמצא במישור: &  &  &  &  \\ \hline
      $$x_{1}+x_{2}+x_{3}=10$$ &  &  &  &  \\ \hline
    \end{tabular}
  \end{table}
\end{example}
\begin{corollary}
עבור משחק קמור הליבה לא ריקה.

\end{corollary}
\begin{example}[משחק לא קמור - כפפות]
נניח כי נתון אוסף של כפפות שמאליות \(L=\left\{  L_{1},\dots,L_{m}  \right\}\) וכפפות ימניות \(R=\left\{  R_{1},\dots,R_{m}  \right\}\) כך שהערך של כל קואליציה היא כמות הזוגות התקינות(כפפה ימינית + שמאלית) שניתן להרכיב. כלומר עבור \(S\subseteq L\cup R\) נקבל:
$$v(S)=\min \left( \left\lvert  S\cap L  \right\rvert ,\left\lvert  S\cap R  \right\rvert  
\right)$$
הווקטור התשלומים יהיה ווקטור מהצורה \(x=\left(x_{L_{1}},...\,,x_{L_{m}},x_{R_{1}},...\,,x_{R_{m}}\right)\) המקיים:
$$\sum_{i=1}^{m}x_{L_{i}}+\sum_{j=1}^{m}x_{R_{j}}=m\qquad \forall i,j\quad x_{L i}\geq0,\ x_{R_{j}}\geq0$$
כדי לתאר את הליבה, ניקח קואליציה כללית \(S\) עם:
$$\ell=|S\cap L|,\qquad r=|S\cap R|,\qquad k=\operatorname*{min}(\ell,r)=v(S)$$
כיוון ש-\(k\leq \ell\) ו-\(k\leq r\) ניתן לבחור \(k\) זוגות זרות של כפפו ימניות ושמאליות ב-\(S\). מתקיים:
$$\sum_{p=1}^{k}\left(x_{L_{i_{p}}}+x_{R_{j_{p}}}\right)\geq k=v(S)$$
וכיוון שלכל הגורמים האחרים בקואליציה יש תשלום אי שלילי נקבל:
$$\sum_{u\in S}x_{u}\geq v(S)$$
ולכן אף קואליציה לא יכולה לשפר את \(x\), והליבה תהיה:
$$\mathcal{C} =\left\{  x \in \mathbb{R}^{2m}\mid \sum_{i=1}^{m}x_{L_{i}}+\sum_{j=1}^{m} x_{R_{j}}=m\qquad \forall i,j\quad x_{L_{i}},x_{R_{j}}\geq 0\quad  x_{L_{i}}+x_{R_{j}}\geq 1 \right\}$$

\end{example}
\end{document}