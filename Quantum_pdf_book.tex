\documentclass{tstextbook}

\usepackage{amsmath}
\usepackage{amssymb}
\usepackage{graphicx}
\usepackage{hyperref}
\usepackage{xcolor}

\begin{document}

\title{Example Document}
\author{HTML2LaTeX Converter}
\maketitle

\chapter{מבוא}

\section{חזרה - אלגברה לינארית}

\section{הגדרות בסיסיות}

\begin{definition}[צמוד מרוכב]
הצמוד של מספר מרוכב \(z=a+bi\) יהיה המספר \(\overline{z}=a-bi\).

\end{definition}
\begin{definition}[צמוד הרמיטי של מטריצה]
הצמוד ההרימטי של מטריצה \(U\) יהיה השיחלוף של הצמוד המרוכב של כל האיברים. מסומן \(U^{\dagger}\)

\end{definition}
\begin{remark}
הגדרה זו רלוונטית גם עבור ווקטורים. ניתן להעביר את הווקטור לווקטור עמודה ולקחת את הצמוד המרוכב שלהם. מסמן \(\vec{v}^{\dagger}\).

\end{remark}
\begin{definition}[מטריצה אוניטרי]
מטריצה המקיימת \(U U^{\dagger}=U^{\dagger} U=Id\).

\end{definition}
\begin{definition}[מטריצה הרמיטית]
מטריצה שמקיימת \(U^{\dagger}=U\).

\end{definition}
\begin{definition}[מרחב הילברט]
מרחב מכפלה פנימית שלם. לעיתים מסמנים ב-\(\mathcal{H}\).

\end{definition}
\begin{definition}[מכפלה פנימית]
תבנית ססקילינארית \(\left\langle  \cdot \mid\cdot  \right\rangle:\mathbb{C}\times \mathbb{C}\to \mathbb{R}\) מוגדרת על ידי:
$$\left\langle  \vec{v},\vec{u}  \right\rangle ={\vec{v}}^{\dagger} \vec{u}$$

\end{definition}
\begin{definition}[קאט]
נקרא לאיבר(ווקטור) במרחב הילברט קאט. נסמן אותו \(\ket{\psi}\)

\end{definition}
\begin{definition}[ברא]
עבור קאט \(\ket{\psi}\) נגדיר את הברא המתאים בתור \(\bra{\psi}\) אשר מקיים \(\bra{\psi}=\ket{\psi}^{\dagger}\).

\end{definition}
\begin{proposition}
כפל של קאט וברא מחזיר את המכפלה הפנימית הסטנדרטית תחת המרוכבים:
$$\left\langle  \bra{\phi} \mid \ket{\psi}   \right\rangle =\bra{\phi}\ket{\psi} :=\left\langle  \phi \mid \psi  \right\rangle  $$

\end{proposition}
\section{קומוטטור}

\subsection{אופרטורים לינארים}

\begin{definition}[אופרטור לינארי]
עבור מרחב הילברט \(\mathcal{H}\) אופרטור \(\hat{X}\) נקרא אופרטור לינארי אם לכל \(v,u \in \mathcal{H}\) ו-\(c \in \mathbb{R}\) מתקיים:
$$\hat{X}(v+cu)=\hat{X}v+c\hat{X}u$$

\end{definition}
האופרטור בניגוד למטריצה לא מיוצג תחת בסיס מסויים. זוהי פונקציה כללית. כאשר הטענה המרכזית שמקשרת ביניהם היא שאופרטור הוא לינארי אם"ם ניתן להציג אותו בעזרת מטריצה בבסיס כלשהו.

\begin{definition}[אופרטור הצמוד]
יהי \(\mathcal{H}\) מרחב הילברט. עבור אופרטור \(\hat{X}\) נגדיר את \(\hat{X}^{\dagger}\) ע"י האופרטור היחיד שמקיים לכל \(\ket{v},\ket{w}\):
$$\hat{X}\bra{v} \ket{w} =\bra{v} \hat{X}^{\dagger}\ket{w} $$

\end{definition}
כלומר זה האופרטור שלהפעיל אותו  על הקאט שקול ללהפעיל את האופרטור המקורי על הברא.

\begin{proposition}[תכונות של אופרטור הצמוד]
  \begin{enumerate}
    \item קיים ויחיד 


    \item לינארי לסכום - \((S+T)^{\dagger}=S^{\dagger}+T^{\dagger}\)


    \item הרכבה של צמודים מקיים - \(\left( S\circ T \right)^{\dagger}=T^{\dagger}\circ S^{\dagger}\)


    \item כפל עם סקלר מקיים - \((aT)^{\dagger}=\overline{a}T^{\dagger}\)


    \item הצמוד של הצמוד  מקיים - \(\left( T^{\dagger} \right)^{\dagger}=T\)


    \item הצמוד של מכפלה פנימית - \(\left( \left\langle  \psi \mid \phi  \right\rangle \right)^{\dagger}=\ket{\psi}^{\dagger}\bra{\phi}^{\dagger}\)


    \item הצמוד של ברא הוא הקאט המתאים ולהפך - \(\bra{\psi}^{\dagger}=\ket{\psi},\ket{\psi}^{\dagger}=\bra{\psi}\)


  \end{enumerate}
\end{proposition}
\begin{definition}[אופרטור נורמלי]
אופרטור \(\hat{X}\) המקיים \(\hat{X}^{\dagger}\hat{X}=\hat{X}\hat{X}^{\dagger}\).

\end{definition}
\begin{corollary}
אופרטור \(X\) הוא נורמלי אם"ם \(\left[ X,X^{\dagger} \right]=0\).

\end{corollary}
\begin{proposition}[המשפט הספקטרלי]
אופרטור נורמלי הוא לכסין אורתונורמלי ופורש את כל המרחב. כלומר קיים בסיס למרחב של ווקטורים עצמיים אשר עבורו ווקטורים עצמיים עם ערכים עצמיים שונים יהיו אורתוגונאלים אחד לשני. עבור מטריצות זה אומר כי המטריצה המלכסנת היא אוניטרית.

\end{proposition}
\begin{definition}[אופרטור הרמיטי]
אופרטור \(\hat{X}\) המקיים \(\hat{X}^{\dagger}=\hat{X}\).

\end{definition}
\begin{remark}
זהו המקביל המרוכב לאופרטור צמוד לעצמו.

\end{remark}
\begin{proposition}[תכונות של אופרטור הרמיטי]
  \begin{enumerate}
    \item כל הערכים העצמיים ממשיים. 


    \item נורמלי. 


    \item הערכים על האלכסון הראשי הם כולם ממשיים. 


  \end{enumerate}
\end{proposition}
\begin{definition}[אופרטור אנטי-הרמיטי]
אופרטור \(\hat{X}\) המקיים \(\hat{X}=-\hat{X}^{\dagger}\)

\end{definition}
\begin{proposition}[תכונות של אופרטור אנטי-הרמיטי]
  \begin{enumerate}
    \item כל הערכים העצמיים מדומים לחלוטין. 


    \item כל הערכים על האלכסון הראשי צריכים להיות מדומים לחלוטין. 


    \item סכום של שתי מטריצות אנטי הרמיטיות הוא אנטי הרמיטי. 


    \item נורמלי. 


    \item אופרטור \(\hat{X}\) הוא אנטי הרמיטי אם"ם \(i\hat{X}\) הרמטי. 


    \item אופרטור \(\hat{X}\) הוא אנטי הארמיטי אם"ם החלק המדומה אנטי סימטרי והחלק הממשי סימטרי. 


    \item אם \(A\) מטריצה אנטי הרמיטית, אז עבור \(k\) זוגי נקבל \(A^k\) הרמיטית ועבור \(k\) אי זוגי \(A^k\) אנטי הרמיטית. 


    \item אם \(A\) אנטי ברמיטית, אז \(e^A\) אוניטרית. 


    \item הנגזרת של אופרטור הרמיטי יהיה אנטי הרמיטי. ונגזרת של אופרטור אנטי הרמיטי יהיה הרמיטי 


  \end{enumerate}
\end{proposition}
\begin{definition}[אופרטור אוניטרי]
יהי \(\mathcal{H}\) מרחב הילברט. האופרטור \(\hat{X}\) יקרא אניטרי אם לכל \(v,w \in \mathcal{H}\) מתקיים:
$$\bra{T(v)}\ket{T(w)} =\bra{v} \ket{w}  $$

\end{definition}
\begin{proposition}
הטענות הבאות שקולות:

  \begin{enumerate}
    \item האופרטור \(\hat{X}\) אוניטרי, כלומר: 
$$\hat{X}^{\dagger} \circ \hat{X} = Id \iff \hat{X}^{\dagger} =\hat{X}^{-1} \iff \hat{X}\circ \hat{X}^{\dagger} = Id$$


    \item קיים בסיס אורתונורמלי כך שתחת הבסיס המטריצה המייצגת אוניטרית 


    \item אופרטור \(\hat{X}\) משמרת נורמה. כלומר: 
$$\forall \ket{\psi}  \in \mathcal{H}\qquad \left\lVert  \hat{X}\ket{\psi}  \right\rVert =\left\lVert  \ket{\psi}   \right\rVert  $$


    \item כל הווקטורים בנורמה 1 נשלחים לווקטור בנורמה 1. כלומר: 
$$\forall \ket{\psi}  \in \mathcal{H}\qquad \left\lVert  \ket{\psi}   \right\rVert =1\implies \left\lVert  \hat{X}\ket{\psi}   \right\rVert =1  $$


  \end{enumerate}
\end{proposition}
\begin{proposition}[תכונות של של אופרטור אוניטרי]
  \begin{enumerate}
    \item ווקטורים מאונכים הולכים לווקטורים מאונכים. 


    \item משמר נורמה 


    \item הגודל של הערך עצמי הוא 1. כלומר אם הע"ע נמצאים על מעגל היחידה 


    \item נורמלי ולכן לכסין אורתונורמלית כאשר ווקטורים עצמיים עם ערכים עצמיים שונים יהיו מאונכים. 


    \item חח"ע. 


  \end{enumerate}
\end{proposition}
\section{תכונות נוספות של אופרטורים}

\begin{definition}[תת מרחב ווקטורי]
עבור מרחב הילברט \(\mathcal{H}\) כל מרחב ווקטורי שמוכל ב-\(\mathcal{H}\) יהיה תת מרחב ווקטורי.

\end{definition}
\begin{definition}[סכום ישר]
שתי מרחבים נמצאים בסכום ישר אם החיתוך שלהם הוא אפס והסכום שלהם הוא המרחב כולו.

\end{definition}
\begin{proposition}
אם שתי מרחבים ווקטורים נמצאים בסכום ישר, ניתן להציג כל וקטור בצורה יחידה סכום של ווקטור ממרחב הראשון עם הווקטור מהמרחב השני.

\end{proposition}
\begin{definition}[אופרטור הטלה]
אם שתי מרחבים נמצאים בסכום ישר, ניתן להגדיר אופרטור הטלה על מרחב, שלוקח את הרכיב של הווקטור ששייך למרחב.

\end{definition}
קיים העתקה יחידה כזו.
\textbf{טענה}
אופרטור הטלה \(P\) מקיים \(P=P^2\).

\begin{remark}
האינטואיציה זה שאם אנחנו מטלים אז אנחנו נמצאים כבר במרחב של ההטלה, ולכן הטלה נוספת לא תעשה כלום.

\end{remark}
\begin{definition}[אופרטור הטלה אורתוגונאלי]
יהי \(V\) מרחב ווקטורי ו- \(W\leq V\) תת מרחב וקטורי עם בסיס \(\ket{1},\dots \ket{n}{}\). 
נגדיר אופרטור \(\Pi_{n}= \ket{n}\bra{n}\) אופרטור מעל המרחב \(W\).

\end{definition}
כאשר כפל של קאט באופרטור יחזיר את הקאט שמוטל על התת מרחב. כמו כן מתקיים:
$$\Pi_{n}\ket{\psi} =\sum_{n}\bra{n} \left( \ket{n} \bra{\psi}  \right)$$
שזה למעשה כפל של קאט בסקלר.

\begin{definition}[מרחב הניצב]
עבור תת מרחב ווקטור \(V\) קיים מרחב מאונך \(V^\perp\) אשר כל ווקטור ב-\(V\) באונך לכל ווקטור ב-\(V^\perp\). מרחבים אלו נמצאים בסכום ישר.

\end{definition}
\begin{proposition}[הטלה אורתוגונאלית]
אם \(P\) הטלה על המרחב \(V\), אז \(Id-P\) יהיה ההטלה על המרחב הניצב \(V^\perp\)

\end{proposition}
\begin{definition}[אקספוננט של מטריצה]
$$e^{X}=\sum_{k=0}^{\infty}{\frac{1}{k!}}X^{k}$$

\end{definition}
\begin{proposition}[אקספוננט של מטריצה אלכסונית]
$$A={\left[\begin{array}{l l l l}{a_{1}}&{0}&{\cdot\cdot\cdot}&{0}\\ {0}&{a_{2}}&{\cdot\cdot\cdot}&{0}\\ {\vdots}&{\vdots}&{\cdot\cdot\cdot}&{\vdots}\\ {0}&{0}&{\cdot\cdot\cdot}&{a_{n}}\end{array}\right]} \implies e^{A}={\left[\begin{array}{l l l l}{e^{a_{1}}}&{0}&{\cdot\cdot\cdot}&{0}\\ {0}&{e^{a_{2}}}&{\cdot\cdot\cdot}&{0}\\ {\vdots}&{\vdots}&{\cdot\cdot\cdot}&{\vdots}\\ {0}&{0}&{\cdot\cdot\cdot}&{e^{a_{n}}}\end{array}\right]}$$

\end{proposition}
\begin{corollary}
עבור מטריצה לכסינה, מתקיים \(A= UDU^{-1}\) ולכן \(e^A=Ue^DU^{-1}\)

\end{corollary}
\begin{example}
$$A={\left[\begin{array}{l l}{1}&{4}\\ {1}&{1}\end{array}\right]}=\left[{\begin{array}{c c}{-2}&{2}\\ {1}&{1}\end{array}}\right]\left[{\begin{array}{c c}{-1}&{0}\\ {0}&{3}\end{array}}\right]\left[{\begin{array}{c c}{-2}&{2}\\ {1}&{1}\end{array}}\right]^{-1}$$
ולכן:
$$e^{A}={\left[\begin{array}{l l}{-2}&{2}\\ {1}&{1}\end{array}\right]}e^{{\left[\begin{array}{l l}{-1}&{0}\\ {0}&{3}\end{array}\right]}}{\left[\begin{array}{l l}{-2}&{2}\\ {1}&{1}\end{array}\right]}^{-1}={\left[\begin{array}{l l}{-2}&{2}\\ {1}&{1}\end{array}\right]}{\left[\begin{array}{l l}{{\frac{1}{e}}}&{0}\\ {0}&{e^{3}}\end{array}\right]}{\left[\begin{array}{l l}{-2}&{2}\\ {1}&{1}\end{array}\right]}^{-1}={\left[\begin{array}{l l}{{\frac{e^{4}+1}{2e}}}&{{\frac{e^{4}-1}{e}}}\\ {{\frac{e^{4}-1}{4e}}}&{{\frac{e^{4}+1}{2e}}}\end{array}\right]}$$

\end{example}
\begin{remark}
ניתן לבצע תהליך זהה עבור צורת ג'ורדן.

\end{remark}
\begin{proposition}[אקספוננט של הטלה]
עבור אופרטור הטלה מתקיים \(P^2 = P\). ולכן ניתן לקצר את הטור טיילור לשתי איברים ולקבל:
$$e^{P}=\sum_{k=0}^{\infty}{\frac{P^{k}}{k!}}=I+\left(\sum_{k=1}^{\infty}{\frac{1}{k!}}\right)P=I+(e-1)P$$

\end{proposition}
\begin{proposition}[פתרון משוואה דיפרנציאלית]
$$y' = Ay\implies y(t)=e^{At}y(0)$$

\end{proposition}
\begin{definition}[טור טיילור של אופרטור]
ניתן לכתוב פונקציה \(F\left( \hat{A} \right)\) על ידי טור טיילור:
$$F({\hat{A}})=\sum_{n=0}^{\infty}{\frac{F^{(n)}(0)}{n!}}{\hat{A}}^{n}$$

\end{definition}
\section{קומוטטור}

\begin{definition}[קומוטטור של חוגים]
$$[A,B]=AB-BA$$

\end{definition}
\begin{remark}
הקומוטטור של חוגים, בדומה לקומוטטור של חבורות בודק כמה שהחוג קומוטטיבי.

\end{remark}
\begin{proposition}
$$[A,B]=-[B,A]$$

\end{proposition}
\begin{proof}
$$[A,B]=A B-B A=-(B A-A B)=-[B,A]$$

\end{proof}
\begin{proposition}
$$[A,B+C]=[A,B]+[A,C]$$

\end{proposition}
\begin{proof}
$$[A,B+C]=A(B+C)-(B+C)A=A B-B A+A C-C A=[A,B]+[A,C]$$

\end{proof}
\begin{proposition}
$$[A,BC]=[A,B]C+B[A,C]$$

\end{proposition}
\begin{proof}
$$[A,B]C+B[A,C]=(A B-B A)C+B(A C-C A)=A B C-B C A=[A,B C]$$

\end{proof}
\begin{corollary}
$$[A,B^{2}]=[A,B]B+B[A,B]$$

\end{corollary}
\begin{proposition}
$$[A B,C]=A[B,C]+[A,C]B$$

\end{proposition}
\begin{proof}
$$A[B,C]+[A,C]B=A(B C-C B)+(A C-C A)B=A B C-C A B=[A B,C]$$

\end{proof}
\begin{corollary}
$$[B^{2},C]=B[B,C]+[B,C]B$$

\end{corollary}
\begin{proposition}
$$[k A,B]=[A,k B]=k[A,B]$$

\end{proposition}
\begin{proof}
$$[k A,B]=k A B-B k A={\left\{\begin{array}{l l}{A k B-k B A}&{=[A,k B]}\\ {k(A B-B A)}&{=k[A,B]}\end{array}\right.}$$

\end{proof}
\begin{proposition}[זהות יעקובי]
$$[A,[B,C]]+[B,[C,A]]+[C,[A,B]]=0$$

\end{proposition}
\begin{proof}
$$\begin{array}{l}{{[A,[B,C]]+[B,[C,A]]+[C,[A,B]]=[A,B C-C B]+[B,C A-A C]+[C,A B-B A]}}\\ {{=A(B C-C B)-(B C-C B)A+B(C A-A C)-(C A-A C)B+C(A B-B A)-(A B-B A)C=0}}\end{array}$$

\end{proof}
\begin{proposition}
$$[A,f(A)]=0$$

\end{proposition}
אם \([A,[A,B]]\,=\,[B,[A,B]]\,=\,0\) אזי מתקיים:
$$\left[A,f\left(B\right)\right]=\ \left[A,B\right]f^{\prime}\left(B\right)$$

\begin{definition}[אנטי קומוטטור]
$$\{A,B\}=AB+BA$$

\end{definition}
\section{יחס פלנק ויחס דה ברויי}

\begin{theorem}[יחס פלנק]
האנרגיה של פוטון קשור ע"י:
$$E = h \nu = \hbar \omega$$
כאשר \(h\) זה קבוע פלאנק, \(\hbar=\frac{h}{2\pi}\) קבוע פלאנק מצומצם, \(\nu\) זה תדירות ו-\(\omega\) זה תדירות זוויתית.

\end{theorem}
\begin{remark}
זו למעשה מסקנה ניסויונית מהאפקט הפוטואלקטרי, וקרינת גוף שחור. ולמעשה אומר כי רמות האנרגיות של האור מקוונטטות, ושלא ניתן יהיה

\end{remark}
\begin{remark}
זה המקור של כל התחום הזה - תורת הקוונטים זה בעברית נקרא תורת המנות - זה עוסק בערכים אשר הטווח שלהם בן מנייה.

\end{remark}
\begin{proposition}
פוטון חסר מסה מקיים:
$$p=\frac{h}{\lambda}=\hbar k=h\lambda$$

\end{proposition}
\begin{proof}
מיחסות אנו יודעים כי הגודל של ה-4 תנע יהיה אינווריאנטי ולכן:
$$E^{2}=p^{2}c^{2}+m_{0}^{\,\,2}c^{4}$$
כאשר עבור פוטון \(m_{0}=0\) ונקבל \(E^2=p^2c^2\). לכן:
$$E=pc=\hbar k\implies pc=\hbar \omega$$
כאשר אנחנו מניחים שהפוטון מתנהג כמו גל, ולכן ינוע במהירות חבורה. כלומר:
$$v_{g}=c= {\frac{\partial\omega}{\partial k}}={\frac{d\nu}{d\left( 1/\lambda \right)}}\implies \omega=ck$$
ולכן נקבל:
$$pc=\hbar \omega=\hbar ck\implies p=\hbar k=h\lambda$$

\end{proof}
\begin{theorem}[דה ברויי]
גם חלקיקים חסרי מסה יכולים להתנהג כגלים, ומקיימים \(p=h\lambda=\hbar k\)

\end{theorem}
\section{ספין}

\begin{definition}[ספין]
תכונה קוונטית חומר. ניתן למדוד ב-3 כיוונים, אשר נסמן אותם \(\sigma_{z},\sigma_{y},\sigma_{x}\)

\end{definition}
\begin{proposition}
עבור כל ערך של ספין(\(\sigma _{x},\sigma_{y},\sigma_{z}\)) הערכים האפשריים הם רק \(1,-1\).

\end{proposition}
היינו יכולים אולי לצפות שיהיו ערכים רציפים. אבל למעשה יש רק 2 ערכים דיסקרטים.

\begin{proposition}
כאשר מודדים ערך של ספין בכיוון מסויים(למשל \(z\)) ומקבלים 1, אז מדידות עוקבות יתנו את אותו הערך.

\end{proposition}
\begin{proposition}
אם נהפוך את הגלאי נקבל \(-1\). זה נכון גם עבור מדידות עוקבות.

\end{proposition}
\begin{remark}
אפשר לחשוב על זה בתור - המדידה הראשונה מקבעת את הערך. ומדידות נוספות מאשרות אותו.

\end{remark}
\begin{proposition}
כאשר מודדים ערך של ספין בכיוון \(z\) ולאחר מכן מודדים את הערך של הספין בכיוון \(x\). נקבל כי התוצאה של הספין בכיוון \(x\) תהיה או \(1\) או \(-1\) בהסתרות שווה לחלוטין. כלומר הממוצע לאחר מדידות רבות תהיה 0.

\end{proposition}
\begin{remark}
יש כאן קצת אינטואיציה קלאסית, כיוון שאם הספין הוא בכיוון \(z\) אז נצפה כי הערך שלו בכיוון \(x\) הוא 0.

\end{remark}
\begin{proposition}
כאשר מודדים את הערך של הספין בכיוון \(z\). ולאחר מכן מודדים את הערך של הספין בכיוון בין \(x\) ל-\(z\). נקבל כי התוצאה של הספין בכיוון הזה יהיה שוב או \(1\) או \(-1\). אך הפעם ההסתבריות לכל אפשרות לא תהיה שווה. אם הווקטור של הכיוון של המדידה הוא \(\hat{n}\) כאשר הכיוון של ציר \(z\) יהיה \(\hat{z}\) אז הממוצע לאחר הטלות רבות יהיה \(\hat{z}\hat{n}\).

\end{proposition}
\begin{remark}
שוב יש כאן קצת אינטואיציה קלאסית. בפיזיקה הקלאסית נקבל כי אם ווקטור בכיוון \(\hat{z}\) אז אם נחפש את הערך שלו על ציר בכיוון \(\hat{m}\) נקבל כי הוא יהיה \(\hat{z}\hat{m}\) - הערך הממוצע שמתקבל פה בפועל.

\end{remark}
\begin{proposition}
ניתן לייצג כל מצב ספין מרחב הילברט דו מימדי.

\end{proposition}
\begin{symbolize}
נרצה לייצג את כל הספינים האפשריים בעזרת ווקטורים. עבור \(\sigma_{z}=\pm 1\) נקבל את הווקטורים \(\ket{\uparrow_{z}}\) ו-\(\ket{\downarrow_{z}}\).
כאשר באופן דומה נייצג את \(\sigma_{x}=1\) ע"י \(\ket{\uparrow_{x}}\) ו-\(\sigma_{x}=-1\) ע"י \(\ket{\downarrow_{x}}\). כנל לגבי \(\sigma_{y}\) ע"י הווקטורים \(\ket{\uparrow_{y}}\) ו-\(\ket{\downarrow_{y}}\).

\end{symbolize}
נדרוש כי ווקטורים אלו יהיו מנורמלים. כלומר \(\braket{ \uparrow_{z} | \uparrow_{z} }=1\).

ניתן לייצג ווקטור(או מצב) במרחב הדו מימדי על ידי כל שתי ווקטורים באוסף הזה. לכן עבור המצב \(\ket{A}\) ניתן לכתוב:
$$\ket{A} = \alpha_{1}\ket{\uparrow_{z}} +\alpha_{2}\ket{\downarrow_{z}} \implies \alpha_{1}=\braket{  \uparrow_{z}|  A}\quad \alpha_{2}=\braket{ \downarrow | A }    $$
בפני עצמם, למספרים המרוכבים \(\alpha_{1},\alpha_{2}\) אין משמעות. אך לגודל שלהם יש משמעות חשובה.

\begin{proposition}
עבור \(\ket{A} = \alpha_{1}\ket{\uparrow_{z}} +\alpha_{2}\ket{\downarrow_{z}}\) ההסתרות של תוצאות המדידה יהיו:
$$P\left( \uparrow_{z} \right)=\left\lvert  \alpha_{1}  \right\rvert =\alpha_{1}^*\alpha_{1}^\qquad P\left( \downarrow_{z} \right)=\left\lvert  \alpha_{2}  \right\rvert =\alpha_{2}^{*}\alpha_{2}$$

\end{proposition}
\begin{corollary}
$$P\left( \uparrow_{z} \right)=\braket{ \uparrow_{z} | A }^{\dagger} \braket{ \uparrow_{z} | A }  =\braket{ A | \uparrow_{z} } \braket{ \uparrow_{z} | A } $$

\end{corollary}
\begin{proposition}
כיוון שלא יכול להיות גם במצב \(\ket{\uparrow_{z}}\) וגם \({} \ket{\downarrow_{z}} {}\) נדרוש כי \(\braket{ \uparrow_{z} | \downarrow_{z} }=0\). כלומר המצבים חד משמעיים ולכן אורתוגונאלים.

\end{proposition}
\begin{proposition}
כיוון שבוודאות נמצא או במצב \(\ket{\uparrow_{z}}\) או במצב \(\ket{\downarrow_{z}}\) נצפה כי סך ההסתרויות יהיה 1. נשים לב כי כיוון שהווקטורים האלו מנורמלים, נדרש גם ש-\(A\) יהיה מנורמל. לכן \(\lvert A \rvert=1\).

\end{proposition}
נזכור כי אמרנו כי אם המערכת הייתה במצב של \(\ket{\uparrow_{z}}\) נקבל כי יש הסתברות שווה שזה יהי ב-\(\ket{\uparrow_{x}}\) ו-ב-\(\ket{\downarrow_{y}}\). ולכן
$$\left\lvert  \braket{ \uparrow_{z} | \uparrow_{x} }  \right\rvert  =\frac{1}{2}\quad \left\lvert  \braket{ \downarrow_{z} | \uparrow_{x} }\right\rvert = \frac{1}{2}\implies    \ket{\uparrow_{x}} = \frac{1}{\sqrt{ 2 }}\ket{\uparrow_{z}} +\frac{1}{\sqrt{ 2 }}\ket{\downarrow_{z}} $$

כאשר יש חופש בחירה של המקדמים, כיוון שגם למשל \(\frac{i}{\sqrt{ 2 }},-\frac{1}{\sqrt{ 2 }}\) מקיימים את הדרישה שההסתברות היא חצי.
\textbf{הגדרה} חופש פאזה
כפל של המקדמים עם מספר על המעל היחידה \(z = e^{i\theta}\) לא משנה את ההסתברות.

\begin{remark}
חופש של הפאזה למעשה נובע מהחופש בחירה של הצירים.

\end{remark}
\begin{proposition}
$$\ket{\downarrow_{x}} = \frac{1}{\sqrt{ 2 }}\ket{\uparrow_{z}} -\frac{1}{\sqrt{ 2 }}\ket{\downarrow_{z}} $$

\end{proposition}
\begin{proof}
כעת נזכור כי המצב \(\ket{\uparrow_{x}}\) לא יכול להיות בו זמנית עם המצב \(\ket{\downarrow_{x}}\). ולכן נדרוש \(\braket{ \uparrow_{x} | \downarrow_{x} }=0\). מתנאי זה נקבל כי:
$$\ket{\uparrow_{x}} = \frac{1}{\sqrt{ 2 }}\ket{\uparrow_{z}} +\frac{1}{\sqrt{ 2 }}\ket{\downarrow_{z}} \implies 0 = \frac{1}{\sqrt{ 2 }}\braket{ \downarrow_{x} |\uparrow_{z}  }+\frac{1}{\sqrt{ 2 }}\braket{ \downarrow_{x} | \downarrow_{z} }\implies   \braket{ \downarrow_{x} | \downarrow_{z} } = - \braket{ \downarrow_{x} | \uparrow_{z} }$$
כאשר שוב נדרוש כי ההסתברות תהיה שווה לחצי שנקבל \(\ket{\downarrow_{x}}\). 

\end{proof}
\begin{proposition}
$$
\ket{\uparrow_{y}} = \frac{1}{\sqrt{ 2 }}\ket{\uparrow_{z}} +\frac{i}{\sqrt{ 2 }}\ket{\downarrow_{z}} \qquad \ket{\downarrow_{y}} = \frac{1}{\sqrt{ 2 }}\ket{\uparrow_{z}} -\frac{i}{\sqrt{ 2 }}\ket{\downarrow_{z}} $$

\end{proposition}
\begin{proof}
ראשית נדורש כי לא ייתכן ויהיה גם במצב \(\sigma_{y}=1\) וגם במצב \(\sigma_{y}=-1\). כלומר \(\braket{ \uparrow_{y} | \downarrow_{y} }=0\).
כעת נדרוש גם כמו מקודם כי אם נמדד את אחד הערכים האחרים \(\ket{\uparrow_{z}},\ket{\uparrow_{x}},\ket{\downarrow_{z}},\ket{\downarrow_{x}}\) נקבל:
$$\left\lvert  \braket{ \uparrow_{x} | \uparrow_{y} }   \right\rvert =\left\lvert  \braket{ \uparrow_{x} | \downarrow_{y} }   \right\rvert =\left\lvert  \braket{ \downarrow_{x} | \uparrow_{y} }   \right\rvert =\left\lvert  \braket{ \downarrow_{x} | \downarrow_{y} }   \right\rvert =\left\lvert  \braket{ \uparrow_{z} | \uparrow_{y} }   \right\rvert =\left\lvert  \braket{ \uparrow_{z} | \downarrow_{y} }   \right\rvert =\left\lvert  \braket{ \downarrow_{z} | \uparrow_{y} }   \right\rvert =\left\lvert  \braket{ \downarrow_{z} | \downarrow_{y} }   \right\rvert=\frac{1}{2}$$

\end{proof}
\begin{corollary}
תחת הבסיס של \(\ket{\uparrow_{z}}\) ו-\(\ket{\downarrow_{z}}\) מתקיים:
\begin{gather*}\ket{\uparrow_{x}} = \frac{1}{\sqrt{ 2 }}\ket{\uparrow_{z}} +\frac{1}{\sqrt{ 2 }}\ket{\downarrow_{z}}  \qquad \ket{\downarrow_{x}} = \frac{1}{\sqrt{ 2 }}\ket{\uparrow_{z}} -\frac{1}{\sqrt{ 2 }}\ket{\downarrow_{z}}  \\\ket{\uparrow_{y}} = \frac{1}{\sqrt{ 2 }}\ket{\uparrow_{z}} +\frac{i}{\sqrt{ 2 }}\ket{\downarrow_{z}} \qquad \ket{\downarrow_{y}} = \frac{1}{\sqrt{ 2 }}\ket{\uparrow_{z}} -\frac{i}{\sqrt{ 2 }}\ket{\downarrow_{z}} 
\end{gather*}

\end{corollary}
\begin{proposition}[ייצוג של הבסיס כווקטורים]
ניתן להציג את הספין כווקטורים אורתוגונאלים כרצונינו, ולכן הבחירה הטבעית תהיה:
$$\ket{\uparrow_{z}} =\begin{pmatrix}1 \\ 0\end{pmatrix}\qquad  \ket{\downarrow_{z}} =\begin{pmatrix}0 \\ 1
\end{pmatrix}$$

\end{proposition}
\begin{proposition}[ייצוג של מדידה של ספין כמטריצה]
$$\sigma_{z}=\begin{pmatrix}1 & 0 \\0 & -1\end{pmatrix}\qquad \sigma_{x}=\begin{pmatrix}0 & 1 \\1 & 0\end{pmatrix}\qquad \sigma_{y}=\begin{pmatrix}0 & i \\i & 0
\end{pmatrix}$$
כאשר מטריצות אלו נקראות מטריצות פאולי.

\end{proposition}
\begin{proof}
נרצה למצוא מטריצה \(\sigma_{z}\) ששקולה למדידה של הספין בכיוון \(z\). ראשית נדרוש:
\begin{gather*}\sigma _{z}\ket{\uparrow_{z}} =\ket{\uparrow_{z}}\implies \begin{pmatrix}\left( \sigma_z \right)_{11} & \left( \sigma_{z} \right)_{12} \\\left( \sigma_{z} \right)_{21} & \left( \sigma_z \right)_{22}\end{pmatrix}\begin{pmatrix}1\\0\end{pmatrix}=\begin{pmatrix}1\\0\end{pmatrix}\\\sigma_{z}\ket{\downarrow_{z}} =-\ket{\downarrow_{z}} \implies   \begin{pmatrix}\left( \sigma_{z} \right)_{11} & \left( \sigma_{z} \right)_{12} \\\left( \sigma_{z} \right)_{21} & \left( \sigma_{z} \right)_{22}\end{pmatrix}\begin{pmatrix}0\\1\end{pmatrix}=-\begin{pmatrix}0\\1\end{pmatrix}
\end{gather*}
כלומר נרצה שתי ערכים עצמיים שמייצגים ערכי מדידה - \(1,-1\). זוהי מערכת של 4 משוואות עם 4 נעלמים. נקבל לבסוף:
$$\sigma_{z}=\begin{pmatrix}1 & 0 \\0 & -1
\end{pmatrix}$$
כעת נשתמש בערכים שמצאנו מקודם כדי למצוא את המטריצת מדידה בכיוון \(y\) ו-\(x\).
\begin{gather*}\ket{\uparrow_{x}} = \frac{1}{\sqrt{ 2 }}\ket{\uparrow_{z}} +\frac{1}{\sqrt{ 2 }}\ket{\downarrow_{z}} \implies \begin{pmatrix}\left( \sigma_{x} \right)_{11} & \left( \sigma_{x} \right)_{12} \\\left( \sigma_{x} \right)_{21} & \left( \sigma_{x} \right)_{22}\end{pmatrix}\begin{pmatrix}\frac{1}{\sqrt{ 2 }} \\\frac{1}{\sqrt{ 2 }}\end{pmatrix}=\begin{pmatrix}\frac{1}{\sqrt{ 2 }} \\\frac{1}{\sqrt{ 2 }}\end{pmatrix} \\\ket{\downarrow_{x}} = \frac{1}{\sqrt{ 2 }}\ket{\uparrow_{z}} -\frac{1}{\sqrt{ 2 }}\ket{\downarrow_{z}} \implies \begin{pmatrix}\left( \sigma_{x} \right)_{11} & \left( \sigma_{x} \right)_{12} \\\left( \sigma_{x} \right)_{21} & \left( \sigma_{x} \right)_{22}\end{pmatrix}\begin{pmatrix}\frac{1}{\sqrt{ 2 }} \\-\frac{1}{\sqrt{ 2 }}\end{pmatrix}=-\begin{pmatrix}\frac{1}{\sqrt{ 2 }} \\-\frac{1}{\sqrt{ 2 }}\end{pmatrix} 
\end{gather*}
כאשר נקבל שוב מערכת של 4 משוואות עם 4 נעלמים, אשר תתן:
$$\sigma_{x}=\begin{pmatrix}0 & 1 \\1 & 0
\end{pmatrix}$$
כאשר באופן דומה נקבל:
$$\sigma_{y}=\begin{pmatrix}0 & i \\i & 0
\end{pmatrix}$$

\end{proof}
\begin{remark}
מטריצות פאולי ביחד עם מטריצת היחידה יוצרות בסיס למרחב המטריצות ה-\(2\times 2\).

\end{remark}
\section{מרחבים רציפים}

\begin{definition}[מרחב הילברט]
מרחב הילברט זה מרחב מכפלה פנימית שלם(כלומר כל סדרת קושי של איברים במרחב מתכנסת).

\end{definition}
\begin{definition}[בסיס של מרחב הילברט]
אוסף של ווקטורים אשר צירוף לינארי שלהם(לאו דווקא סופי) פורש את המרחב, וכל איבר יהיה בלתי תלוי לינארי ביתר האיברים.

\end{definition}
\begin{remark}
במתמטיקה הרבה פעמים דורשים כי בסיס יהיה כך שכל איבר הוא צירוף לינארי סופי של איברים. זה נובע ממהגדרה של צירוף לינארי - שאנו דורשים שיהיה סופי כיוון שצירוף לינארי אינסופי לא מוגדר עבור כל מרחב ווקטורי, וזה דורש לשנות את המבנה האלגברי. למה שאנחנו קוראים בסיס, מתמטיקאים קוראים סט שלם.

\end{remark}
\begin{proposition}
בדומה למרחב ווקטורי סופי, גם במרחב ווקטורי אין סופי לכל בסיס יש את אותה עוצמה.

\end{proposition}
\begin{definition}[מרחב הילברט רציף]
מרחב הילברט כאשר הקבוצה של האיברי בסיס שלו בעוצמת הרצף.

\end{definition}
\begin{definition}[פונקציית הגל]
בהנתן בסיס \(\left\{  \ket{x}  \right\}\) ניתן להציג כל \(\ket{\psi}\) במרחב הילברט ע"י צירוף לינארי אינסופי של איברים מ-\(\left\{  \ket{x}  \right\}\). נסמן את מקדמים הנרמול של האיבר \(\ket{x}\) ב-\(\psi(x)\) ולכן מתקיים:
$$\left|\psi\right\rangle=\int d x\psi\left(x\right)\left|x\right\rangle$$

\end{definition}
\begin{definition}[ווקטור עצמי]
עבור אופרטור \(X\), ווקטור עצמי \(\ket{x}\) של \(X\) יהיה ווקטור המקיים:
$$X\left|x\right\rangle=x\left|x\right\rangle$$
עבור \(x \in \mathbb{C}\) כלשהו.

\end{definition}
\begin{symbolize}
עבור ווקטור עצמי \(\ket{x}\) בדרך כלל נסמן ב-\(x\) את הע"ע המתאים.

\end{symbolize}
\begin{definition}[אורתוגונאליות]
ווקטורים \(x,x'\) נקראים אורתוגונאלים אם:
$$\left\langle  x|x^{\prime} \right\rangle=\begin{cases}0 & \ket{x} \neq \ket{x'} \\1 & \ket{x} =\ket{x'}  
\end{cases}$$
כאשר עבור ווקטורים עצמיים אורתוגונאלים נקבל כי:
$$\braket{ x | x' } =\delta(x-x')  $$

\end{definition}
\begin{proposition}[המשפט הספקטרלי במרחבי הילברט רציפים]
כאשר \(X\) אופרטור הרמיטי, הע"ע של האופרטור יוצרים בסיס אורתונורמלי למרחב הילברט, כך שהע"ע כולם ממשיים.

\end{proposition}
\begin{remark}
זה מקביל למערכת הסוף מימדית שבו \(\langle x_{n}|x_{m} \rangle=\delta_{mn}\). זה מראה שהדלתא של קרונקר ההרחבה הרציפה של הדלתא של דיראק.

\end{remark}
\begin{proposition}
פונקציית הגל מקיימת:
$$\psi\left(x\right)=\left\langle x|\psi\right\rangle=\int d x^{\prime}\psi\left(x^{\prime}\right)\left\langle x|x^{\prime}\right\rangle=\int d x^{\prime}\psi\left(x^{\prime}\right)\delta\left(x-x^{\prime}\right)$$

\end{proposition}
\begin{proposition}[תכונות של הפונקציית גל]
במרחב הילברט רציף מתקיים התכונות הבאות עבור פונקציות גל:

  \begin{itemize}
    \item \textbf{חד ערכית} - לכל ערך קיים ערך יחיד של הפונקציית גל. זה כיוון שייתכן הסתברות יחידה שהגל יהיה בערך מסויים.
    \item \textbf{חלקה -} הנגזרת הראשונה צריכה להיות חלקה. אחרת משוואת שרדינגר לא מוגדרת היטב(זה לא נכון כאשר יש פוטנציאל איסופי, לדוגמא במקרים של לדוגמא בור פוטנציאל אינסופי או פוטנציאל דלתא)
    \item \textbf{הריבוע שלה אינטגרבילי -} אחרת ההסתברות לא מוגדרת היטב.
  \end{itemize}
זה נובע מהתכונות הפיזיקלות של פונקציית הגל ולא מתכונות מתמטיות.

\end{proposition}
\begin{corollary}
ניתן להכליל את הדרישה שחלקה ושרציפה על ידי דרישת רציפות הנגזרת הלוגריתמית.

\end{corollary}
\begin{proposition}[נורמה של מצב בעזרת פונקציית גל]
הנורמה של מצב נתון על ידי:
$$\left\langle  \psi|\psi  \right\rangle = \int  \left\lvert  \psi(x)  \right\rvert ^2 \; \mathrm{d}x$$

\end{proposition}
\begin{proof}
\begin{gather*}\langle\psi|\psi\rangle=\left(\int dx'\left\langle x'\right|\overline{\psi\left(x'\right)}\right)\left(\int dx\psi\left(x\right)|x\rangle\right)=\int dxdx'\left\langle x'|x\right\rangle\overline{\psi\left(x'\right)}f\left(x\right)\\=\int dxdx'\delta\left(x-x'\right)\overline{\psi\left(x'\right)}\psi\left(x\right)=\int dx\left|\psi\left(x\right)\right|^{2} 
\end{gather*}

\end{proof}
\begin{definition}[פונקציית גל מנורמלת]
$$\int \left|\psi\left(x\right)\right|^{2}\;\mathrm{d}x=1$$

\end{definition}
\begin{proposition}[מכפלה פנימית של שתי מצבים בעזרת פונקציות הגל]
$$\braket{ \psi | \phi } =\int \overline{\phi(x)} \psi(x) \;\mathrm{d}x$$

\end{proposition}
\begin{proof}
$$ \begin{array}{rcl}\langle\phi|\psi\rangle&=&\left(\int dx'\left\langle x'\right|\overline{\phi\left(x'\right)}\right)\left(\int dx\psi\left(x\right)|x\rangle\right)=\int dxdx'\left\langle x'|x\right\rangle\overline{\phi\left(x'\right)}\psi\left(x\right)\\&=&\int dxdx'\delta\left(x-x'\right)\overline{\phi\left(x'\right)}\psi\left(x\right)=\int dx\overline{\phi\left(x\right)}\psi\left(x\right)\end{array}$$

\end{proof}
\begin{proposition}[ערך תצפית בעזרת הפונקציית גל]
$$\langle A\rangle=\int d x\,\langle\psi|x\rangle\langle x|A|\psi\rangle=\int d x\,\psi^{*}(x)\,(A\psi)(x)$$

\end{proposition}
\begin{example}[ערך תצפית של מיקום]
$$\langle x\rangle=\int_{-\infty}^{\infty}x\,|\psi(x)|^{2}\,d x$$

\end{example}
\begin{proposition}[אלמנטי מטריצה בעזרת פונקציית גל]
$$\langle\phi|A|\psi\rangle=\int d x\int d x'\,\phi^{*}(x)\langle x|A|x'\rangle\,\psi(x')$$

$$\langle\phi|A|\psi\rangle=\int d x\phi^{*}(x)\langle x|A|x\rangle\,\psi(x)$$

\end{proposition}
\begin{proposition}
בהנתן מרחב ווקטור \(V\) מעל \(\mathbb{C}\) וזוג אופרטורים \(A,B:V\to V\) אזי אם לכל \(\ket{v}\in V\) מתקיים:
$$\langle v|A|v \rangle =\langle v|B|v \rangle $$
אזי \(A=B\).

\end{proposition}
\begin{proof}
מספיק להראות כי לכל \(\ket{v},\ket{w}\) מתקיים \(\langle w|A|v \rangle=\langle w|B|v \rangle\) כיוון שאם כל אלמנטי המטריצה זהים האופרטורים יהיו זהים.
יהיו \(\ket{v},\ket{w}\in V\). מההנחה עבור \(\ket{\psi}=\ket{v}+\ket{w}\) ו-\(\ket{\phi}=\ket{v}+i\ket{w}\) מתקיים:
\begin{gather*}\left(\left\langle  v|+\left\langle  w|\right)A\left(|v \right\rangle+|w \right\rangle\right)=\left(\left\langle  v|+\left\langle  w|\right)B\left(|v \right\rangle+|w \right\rangle\right)\\\left(\left\langle  v|-i\left\langle  w|\right)A\left(|v \right\rangle+i|w \right\rangle\right)=\left(\left\langle  v|-i\left\langle  w|\right)B\left(|v \right\rangle+i|w \right\rangle\right) 
\end{gather*}
ניתן לפשט ולקבל משוואה עבור אלמנטי המטריצה:
\begin{gather*}\langle v|A|w\rangle+\langle w|A|v\rangle=\langle v|B|w\rangle+\langle w|B|v\rangle \\\langle v|A|w\rangle-\langle w|A|v\rangle=\langle v|B|w\rangle-\langle w|B|v\rangle
\end{gather*}
ולכן מחיבור המשוואות נקבל כמבוקש \(\langle v|A|w \rangle=\langle v|B|w \rangle\).

\end{proof}
\chapter{יסודות הפיזיקה הקוונטית}

\section{מצבים קוונטים וגדלים מדידים}

\subsection{מצבים קוונטים}

\begin{definition}[גודל פיזיקלי]
נקרא לגדלים כמו, תנע מהירות, מיקום, אנרגיה וספין גדלים פיזיקליים.

\end{definition}
\begin{definition}[מצב קוונטי]
מצב קוונטי זה אובייקט מתמטי שמתאר את הידע שלנו על מערכת קוונטית. ממנו ניתן לחלץ את המידע על הגדלים הפיזיקלים בכל רגע של זמן.

\end{definition}
\begin{remark}
במכניקה קלאסית אנחנו רגילים להתייחס לכל גודל פיזיקלי בתור אופייקט נפרד, אך פה מתייחסים להכל כאובייקט יחיד שמכיל את כל המידע על המערכת. הגודל ממכניקה הקלאסית שאפשר אולי לנסות להקביל עליו זה הלגרנג'יאן - אשר מכיל את כל המידע הרלוונטי על מערכת.

\end{remark}
\begin{theorem}[אופי המצב הקוונטי]
מצב קוונטי הוא ווקטור מנורמל במרחב הילברט.

\end{theorem}
\begin{corollary}
סכום של שתי מצבים קוונטים הוא מצב קוונטי חדש.

\end{corollary}
\begin{corollary}
ניתן להכפיל במצב קוונטי בסקלר ונקבל מצב קוונטי חדש.

\end{corollary}
\begin{corollary}
ניתן להציג את אותו מצב קוונטי בבסיסים שונים. המצב הקוונטי נשאר אותו דבר, אבל הייצוג שלו משתנה.

\end{corollary}
\begin{symbolize}
מסמנים מצב קוונטי ב-\(\ket{\psi}\).

\end{symbolize}
מערכות קוונטיות באופן כללי נמצאים בסופרפוזציה של מצבים קוונטים(כלומר סכום של ווקטורים), אשר "קורס" למצב יחיד אחרי מדידה. באופן כללי הקריסה היא הסתברותית.

\subsection{גדלים מדידים}

\begin{proposition}
גדלים פיזיקלים מיוצגים ע"י אופרטורים לינארים.

\end{proposition}
\begin{proposition}
הערכים האפשריים של מדידה הם הערכים העצמיים של האופטור \(\psi_{i}\), כאשר הווקטור העצמי המתאים יהיה חד משמעית \(\ket{\psi_{i}}\).

\end{proposition}
\begin{remark}
הכוונה בחד משמעית, זה שאם נמדד מצב קוונטי \(\ket{\psi_{i}}\) אשר ווקטור עצמי של אופרטור המדידה, אז ערך המדידה בוודאות יהיה \(\psi_{i}\). 

\end{remark}
\begin{proposition}
כאשר ערך הוא חד משמעי - הוא מייצג ווקטורים אורתוגונאילים.

\end{proposition}
\begin{proposition}[כלל בורן]
כאשר המערכת מייצגת את המצב הקוונטי \(\ket{\psi}\), ההסתברות למדוד את הערך עצמי \(\psi_{i}\) של הגודל המדיד \(L\) יהיה:
$$P\left( \psi_{i} \right)=\left\langle  \psi | \psi_{i}  \right\rangle \braket{ \psi_{i} | \psi }= \left\lvert  \braket{ \psi | \psi_{i} }  \right\rvert ^2 $$

\end{proposition}
\begin{theorem}[אופי הגדלים הפיזיקלים]
גדלים פיזיקלים הם אופרטורים הרמיטים.

\end{theorem}
עקרון זה מאוד נוח לנו כי זה נותן לנו את העובדות ההגיוניות:

\begin{enumerate}
  \item ע"ע ממשים - תוצאות מדידה אפשריות יהיו ממשיות. 


  \item ו"ע אורתוגונאלים - תוצאות מדידה הם חד משמעיות. 


  \item הו"ע פורשים את המרחב - כל מדידה תתן תוצאה, כלומר פעולת המדידה מוגדרת היטב. 


\end{enumerate}
\begin{theorem}[מדידה של גודל פיזיקלי]
מדידה של גודל פיזיקלי של מצב קוונטי שקולה להפעלה של אחד מהאופרטורי ההטלה על מצביו העצמיים של הגודל הפיזיקלי.

\end{theorem}
\section{מערכות מדידות}

\begin{theorem}[מדידה]
לאחר מדידה, הגודל הפיזיקלי קורס לאחד המצבים העצמיים שלו בצורה לא דיטרמניסטית ולא הפיכה, כאשר ההסתברות להיות במצב עצמי כלשהו נקבע ע"י כלל בורן.

\end{theorem}
\begin{proposition}
מדידה של המערכת שקול להפעלה של אופרטור הרמיטי.

\end{proposition}
\textbf{הסבר:}

\begin{enumerate}
  \item נצפה כי הערכים העצמיים יהיו ממשיים. אין משמעות לגודל פיזיקלי מרוכב. 


  \item נצפה כי יפרוש את המרחב כולו, אחרת אם לא נמצא - מה היה קורה אם לא היה ניתן לבטא בעזרת מצבים עצמיים? איזה ערך היה מתקבל במדידה.  


  \item נצפה כי לא יהיו תלויים אחד בשני, כלומר אם אם נמדד בערך אחד, זה אומר שכבר לא נמצא בסופרפוזיציה של הערכים האחרים - כלומר אורתוגונאלי עליהם. 


\end{enumerate}
\begin{proposition}[מדידה של גודל דיסקרטי]
עבור כמות סופית של מצבים עצמיים \(\left\{  \ket{n}  \right\}\), נקבל כי ההסתברות להיות במצב עצמי \(\ket{n}\) תהיה:
$$\mathrm{Prob}\left(n\right)=\left|\left\langle n|\psi\right\rangle\right|^{2}=\left|\psi_{n}\right|^{2}$$

\end{proposition}
\begin{proposition}[מדידה של גודל רציף]
עבור כמות רציפה של מצבים עצמיים, ההסתברות בנקודה היא 0, ולכן נדרש לחפש הסתברות בטווח. נקבל עבור \(x_{1}<x_{2}\) ופונקציית גל \(\psi(x)\) כי:
$$\mathrm{Prob}\left( x_{1}\leq x\leq x_{2} \right)= \int_{x_{1}}^{x_{2}} \lvert \psi(x) \rvert ^2\;\mathrm{d}x$$

\end{proposition}
\begin{remark}
בשתי המקרים נקבל כי זה פשוט האמפיטודה של מקדמי הפריסה.

\end{remark}
\begin{proposition}
הערך הנצפה של אופרטור \(Q\) אשר מייצג מדידה תחת בסיס מנורמל יהיה:
$$\left\langle Q\right\rangle=\left\langle\psi\right|Q\left|\psi\right\rangle$$

\end{proposition}
\begin{proof}
עבור המקרה הדיסקרטי ניתן לכתוב:
$$\left\langle Q\right\rangle={\sum_{n=1}^{d}}q_{n}\left|\left\langle n|\psi\right\rangle\right|^{2}={\sum_{n=1}^{d}}q_{n}\left\langle\psi|n\right\rangle\left\langle n|\psi\right\rangle$$
כאשר ניתן להוציא את האופרטור הלינארי ולנקבל:
$$\left\langle Q\right\rangle={\sum_{n=1}^{d}}q_{n}\left|\left\langle n|\psi\right\rangle\right|^{2}=\left\langle\psi\right|{\sum_{n=1}^{d}}q_{n}\left|n\right\rangle\left\langle n\right|\left|\psi\right\rangle$$
כאשר נזכור כי \(Q=\sum_{n=1}^{d}q_{n}\left|n\right\rangle\left\langle n\right|\). ולכן נקבל סה"כ:
$$\langle Q \rangle =\bra{\psi} Q\ket{\psi}  $$
ההוכחה עבור המקרה הרציף היא זהה.

\end{proof}
\begin{remark}
אין כאן תלות בבסיס, אבל כן נדרש שיהיה בבסיס מנורמל. בבסיס שאינו מנורמל נקבל:
$$\langle Q\rangle =\frac{\langle \psi |Q  |\psi \rangle }{\langle\psi|\psi\rangle}$$

\end{remark}
\begin{remark}
ערך התצפית לא חייב להיות מצב עצמי.

\end{remark}
\section{קידום בזמן}

\begin{definition}[המילטוניאן]
נקרא לאופרטור של האנרגיה ההמילטוניאן, ונסמנו ב-\(H\).

\end{definition}
\begin{proposition}
מתוך פירוק לאופני התנודה של האנרגיה נקבל:
$$H=\sum_{n=1}^{d}\!E_{n}\left|n\right\rangle\left\langle n\right|\implies H\left|n\right\rangle=E_{n}\left|n\right\rangle$$

\end{proposition}
\begin{proposition}
למערכת פיזיקלית יש שתי דרכים להתפתח בזמן:

  \begin{enumerate}
    \item מדידה - התפתחות לא דטרמינסטית ולא הפיכה. 


    \item כל עוד לא מדדנו, מצפים לאיזשהו הפתפתחות רציפה והפיכה של המערכת. 


  \end{enumerate}
\end{proposition}
כרגע אנחנו נתייחס למערכות ללא המדידה. כלומר מקרה 2.

\begin{proposition}[קידום בזמן]
לקדם את המצב קוונטי בזמן שקול ללהפעיל אופרטור לינארי.

\end{proposition}
\begin{proposition}[תכונות אופרטור קידום בזמן]
  \begin{enumerate}
    \item אופרטור אוניטרי - נדרש שהאופרטור קידום בזמן משמר הסתברות, לכן שולח איברים עם נורמה 1 לאיברים עם נורמה 1. 


    \item בזמן \(t_{0}\) אופרטור קידום בזמן יהיה הזהות. כלומר \(U(t_{0},t_{0})=Id\). 


    \item לקדם בזמן יקיים: 
$$\left|\psi\left(t_{1}\right)\right\rangle=U\left(t_{1},t_{0}\right)\left|\psi\left(t_{0}\right)\right\rangle$$


    \item הרכבה של אופרטורי קידום בזמן יקיים: 
$$\left|\psi\left(t_{2}\right)\right\rangle=U\left(t_{2},t_{1}\right)\left|\psi\left(t_{1}\right)\right\rangle=U\left(t_{2},t_{1}\right)U\left(t_{1},t_{0}\right)\left|\psi\left(t_{0}\right)\right\rangle \implies U\left(t_{2},t_{1}\right)U\left(t_{1},t_{0}\right)=U\left(t_{2},t_{0}\right)$$


  \end{enumerate}
\end{proposition}
\begin{proposition}
אופרטור הקידום בזמן יהיה:
$$U\left(t-t_{0}\right)=e^{-i H\left(t-t_{0}\right)/\hbar}$$

\end{proposition}
\begin{proof}
נרצה כי בזמן \(t_{0}\) נקבל כי האופרטור קידום בזמן הוא הזהות. לכן:
$$U(t_{0},t_{0})=1$$
וכן יקדם בזמן:
$$\left|\psi\left(t_{1}\right)\right\rangle=U\left(t_{1},t_{0}\right)\left|\psi\left(t_{0}\right)\right\rangle$$
וגם שעבור זמנים \(t_{0}<t_{1}<t_{2}\) מתקיים:
$$\left|\psi\left(t_{2}\right)\right\rangle=U\left(t_{2},t_{1}\right)\left|\psi\left(t_{1}\right)\right\rangle=U\left(t_{2},t_{1}\right)U\left(t_{1},t_{0}\right)\left|\psi\left(t_{0}\right)\right\rangle$$
ולכן:
$$U\left(t_{2},t_{1}\right)U\left(t_{1},t_{0}\right)=U\left(t_{2},t_{0}\right)$$
כיוון שאוניטרי, הע"ע נמצאים על מעגל היחידה המרוכב. ולכן ניתן לכתוב:
$$U\left(s\right)\left|n\right\rangle=e^{i\phi_{n}\left(s\right)}\left|n\right\rangle$$
כאשר ניתן לכתוב:
$$\mathcal{\phi}_{n}\left(s\right)=\sum_{k=0}^{\infty}a_{k}^{\left(n\right)}s^{k}$$
מהדרישה \(U(t_{0},t_{0})=1\) נקבל ש-\(\psi_{n}(s=0)=0\) ולכן \(a_{0}^{(n)}=0\). מדרישת ההרכבה נקבל:
$$U\left(s_{2}\right)U\left(s_{1}\right)=U\left(s_{1}+s_{2}\right)$$
ולכן:
$$\sum_{k=1}^{\infty}a_{k}^{(n)}\left(s_{1}^{k}+s_{2}^{k}\right)=\sum_{k=1}^{\infty}a_{k}^{(n)}\left(s_{1}+s_{2}\right)^{k}$$
כדי שיתקיים לכל \(s_{1},s_{2}\) כל מקדמי הפיתוח פרט ל-\(a_{1}^{(n)}\) חייבים להתאפס. נקבל:
$$U\left(t-t_{0}\right)|n\rangle=e^{i a_{1}^{\left(n\right)}\left(t-t_{0}\right)}\left|n\right\rangle$$
כעת ניתן להגדיר אופרטור התדר:
$$\Omega=\sum_{n=1}^{d}\omega_{n}\left|n\right\rangle\left\langle n\right|$$
ומכאן אופרטור הקידום בזמן יהיה:
$$U\left(t-t_{0}\right)=e^{-i\Omega\left(t-t_{0}\right)}$$
נזכור כי לפי פלנק מתקיים \(E = \hbar \omega\). ולכן נגדיר את אופני התנודה של האנרגיה(אנרגיות עצמיות) בתור:
$$E_{n}=\hbar\omega_{n}\,$$
ולכן נכתוב את אופרטור האנרגיה \(H=\hbar \Omega\). ולכן \(\Omega=\frac{H}{\hbar}\) ולכן:
$$\left|\psi\left(t\right)\right\rangle=e^{-i H\left(t-t_{0}\right)/\hbar}\left|\psi\left(t_{0}\right)\right\rangle$$

\end{proof}
\begin{proposition}[משוואת שרדינגר]
$$i\hbar\frac{\partial}{\partial t}\left|\psi\right\rangle=H\left|\psi\right\rangle$$

\end{proposition}
\begin{proof}
נכתוב:
$$\left|\psi\left(t\right)\right\rangle=e^{-i H\left(t-t_{0}\right)/\hbar}\left|\psi\left(t_{0}\right)\right\rangle$$
נגזור את שני צידי המשוואה לפי זמן ונקבל:
$$\frac{\partial}{\partial t}\left|\psi\left(t\right)\right\rangle=-\frac{i H}{\hbar}e^{-i H\left(t-t_{0}\right)/\hbar}\left|\psi\left(t_{0}\right)\right\rangle=-\frac{i H}{\hbar}\left|\psi\left(t\right)\right\rangle$$

\end{proof}
\begin{definition}[תמונת הייזנברג]
האופרטורים משתנים בזמן והמצבים קוונטים קבועים בזמן.

\end{definition}
\begin{definition}[תמונת שרדינגר]
האופרטורים קבועים בזמן והמצבים הקוונטים משתנים בזמן.

\end{definition}
שתי המצבים התמונות האלה שקולות לחלוטין.

\section{עקרון אי וודאות}

\begin{definition}[אי וודאות של אופרטור]
$$\triangle A = \sqrt{\left\langle\psi\right|\left(A-\left\langle A\right\rangle\right)^{2}\left|\psi\right\rangle}$$

\end{definition}
\begin{remark}
זה הסטיית תקן של הערכים - מדד של התפלגות הערכים סביב הממוצע. גודל נמצא בסופרפוזציה של ערכים סביב הערך הממוצע. ולכן אם נבצע ממדידה, ככל שהאי וודאות קטן יותר, יש סיכוי גבוה יותר שמדידה תתן ערך קרוב לערך המצופה. כאשר אם זה גדול, הערכים יהיו רחוקים יותר מהערך המצופה.

\end{remark}
\begin{proposition}
$$\triangle A=\sqrt{\left\langle\psi\right|A^{2}\left|\psi\right\rangle-2\left\langle A\right\rangle\left\langle\psi\right|A\left|\psi\right\rangle+\left\langle A\right\rangle^{2}\left\langle\psi|\psi\right\rangle}=\sqrt{\left\langle A^{2}\right\rangle-\left\langle A\right\rangle^{2}}$$

\end{proposition}
\begin{theorem}[אי הוודאות]
$$\triangle A \triangle B \geq \frac{1}{2}\left\lvert  \langle [A,B] \rangle   \right\rvert $$

\end{theorem}
\begin{proof}
נכתוב:
$$\Delta A\Delta B={\frac{1}{2}}[\Delta A,\Delta B]+{\frac{1}{2}}\{\Delta A,\Delta B\},$$
כאשר הקומוטטור הוא אופרטור אנטי הרמיטי, והאנטי קומוטטור הוא הרמיטי. כעת מתקיים:
$$\left\langle \Delta A\Delta B \right\rangle=\frac{1}{2}\overbrace{ \left\langle  \left[ \Delta A,\Delta B \right]  \right\rangle }^{ \text{מדומה} } +\frac{1}{2}\overbrace{ \left\langle \left\{ \Delta A,\Delta B \right\} \right\rangle }^{ \text{ממשי} }$$
ולכן אם נעלה בריבוע הגורם הממשי רק יגדיל, ולכן אם נזניח אותו, נקבל את האי שיוויון:
$$\left\lvert  \left\langle  \Delta A\Delta B  \right\rangle   \right\rvert ^2 \leq \frac{1}{4}\left\langle  \left[ \Delta A,\Delta B \right]  \right\rangle $$

\end{proof}
\begin{remark}
אם אנחנו במצב עצמי של \(A\), שעבורו \(\triangle A = 0\) אין שום וודאות לגבי ערכו של \(B\).

\end{remark}
\begin{proposition}[משפט אהרנפסט]
$$\frac{d\left<Q\right>}{d t}=\frac{i}{\hbar}\left<[H,Q]\right>+\left<\frac{\partial Q}{\partial t}\right>$$

\end{proposition}
\begin{proof}
נכתוב:
\begin{gather*}\frac{d}{dt}\langle Q \rangle =\frac{d}{dt}\left[ \bra{\psi} Q\ket{\psi}  \right]=\left( \frac{\mathrm{d} }{\mathrm{d} t} \bra{\psi} \right) Q\ket{\psi} +\bra{\psi} \left( \frac{\mathrm{d} }{\mathrm{d} t} Q \right)\ket{\psi} +\bra{\psi} Q\left( \frac{d}{dt}\ket{\psi}  \right) 
\end{gather*}
כאשר נשתמש במשפט שרודינגר ונקבל:
$$=\frac{i}{\hbar}\bra{\psi} HQ \ket{\psi} -\frac{i}{\hbar}\bra{\psi} QH \ket{\psi} +\bra{\psi} \left( \frac{\mathrm{d} }{\mathrm{d} t} Q \right)\ket{\psi} =\frac{i}{\hbar}\langle HQ-QH \rangle +\bra{\psi} \left( \frac{\mathrm{d} }{\mathrm{d} t} Q \right)\ket{\psi}   $$
כאשר נשתמש בהגדרת הקומוטטור ובערת הממוצע של התוכלת ונקבל:
$$\frac{\mathrm{d} }{\mathrm{d} t} \langle Q \rangle =\frac{i}{\hbar}\langle [H,Q] \rangle + \left\langle  \frac{\partial Q}{\partial t}   \right\rangle $$

\end{proof}
\begin{proposition}[משפט הלכסון המשותף]
שני אופרטורים ניתנים ללכסון משותף(כלומר בעלי אותם מצבים עצמיים אך ספקרטום שונה) אם"ם הם מתחלפים.

\end{proposition}
\begin{remark}
זה אומר שאם אנחנו נמצאים במצב עצמי של \(A\), הערך של \(B\) ידוע בוודאות אם"ם הם מתחלפים.

\end{remark}
זה נותן איזשהו אינטואיציה לגבי מקור אי הוודאות. אם לא לכסינים במשותף, אז אין בסיס משותף של ווקטורים עצמיים. לכן אם נמדוד קודם את \(A\) ואז את \(B\) נקבל כי הערך האפשרי של \(A\) משתנה, ולכן 

\section{אופרטור המיקום והתנע}

\begin{definition}[אופרטור המיקום]
המיקום הוא גודל מדיד, לכן ניתן לתאר אותו ע"י אופרטור הרמיטי \(X\). כיוון שהערכים האפשריים של \(X\) הם רציפים, הע"ע של האופרטור \(X\) הם רציפים.

\end{definition}
\begin{proposition}
הווקטורים העצמיים של האופרטור \(X\) הם פונקציות דלתא של המשתנה \(x\).

\end{proposition}
\begin{proposition}[הסתברות למצוא חלקיק בתחום]
ההסתברות למצוא חלקיק על הישר הממשי \(\mathbb{R}\) בתחום \(a\leq x \leq b\) יהיה:
$$P_{X}(\psi)([a,b])=\int_{a}^{b}|\psi|^{2}d\!\mathrm{x}$$

\end{proposition}
\begin{definition}[אופרטור הזזה במיקום]
אופרטור \(T\) המקיים:
$$T(a)\ket{x} = \ket{x+a}$$

\end{definition}
\begin{proposition}[תכונות של אופרטור הזזה]
  \begin{enumerate}
    \item אוניטרי - משימור הסתברות.  


    \item הרכבה של אופרטרי הזזה נותן אופרטור הזזה: 
$$T(a)T(b)\ket{x}=\ket{x+a+b}  $$


    \item האופרטור \(T(0)=Id\). כלומר לא לקדם שקול ללא לעשות כלום. 


  \end{enumerate}
\end{proposition}
\begin{remark}
אופרטור הזזה למעשה יוצר שינוי במקום, בדומה לאופרטור קידום בזמן.

\end{remark}
\begin{definition}[אופרטור התנע]
התנע הוא גודל מדיד לכן ניתן לתאר אותו ע"י אופרטור הרמיטי \(P\).

\end{definition}
\begin{proposition}[אופרטור התנע כיוצר של שינוי במיקום]
אופרטור התנע מקיים:
$$i\hbar \frac{\mathrm{d} }{\mathrm{d} x} \ket{x} = P\ket{x} $$

\end{proposition}
\begin{proof}
נרצה ליצור אופרטור אשר מקדם את המצב הקוונטי. כלומר אופרטור המקיים:
$$T(a)\ket{x} = \ket{x+a} $$
משימור הסתברות נקבל כי אפרטור זה אוניטרי. נסתכל על הפיתוח טיילור של קידום ב-\(dx\):
$$T(dx)=I+T(0)dx+O(dx^2)\implies T(dx)\ket{x} =\ket{x} +\dot{T}(0)\ket{x} dx+O(dx^2)\ket{x} =\ket{x+dx} $$
נעביר אגפים, נחלק ב-\(dx\) ונקבל:
$$\frac{\ket{x+dx}-\ket{x} }{dx}=\dot{T}(x)\ket{x} +\cancelto{ 0 }{ \frac{O(dx^2)}{dx} }\implies  \frac{d}{dx}\ket{x} =\dot{T}(0)\ket{x}  $$
כיוון ש-\(T\) הרמיטי, אז הנגזרת של \(T\) תהיה אנטי הרמיטית, ולכן אם נכפיל ב-\(i\) נקבל \(i\dot{T}\) אופרטור הרמיטי, אשר נסמן ב-\(P\). כלומר:
$$i\dot{T}(0)=P\implies \dot{T}=\frac{P}{i }\implies i\frac{d}{dx}\ket{x}  =  P \ket{x}  $$
כעת מטעמי יחידות ומכך שנרצה שיתאים לעקרון דה ברויי נדרש להכפיל ב-\(\hbar\) ולקבל:
$$i\hbar \frac{\mathrm{d} }{\mathrm{d} x} \ket{x} = P\ket{x} $$

\end{proof}
\begin{remark}
ניתן היה להוכיח באופן זהה לחלוטין להוכחה של אופרטור קידום בזמן, ונשתמש ביחס דה ברויי במקום יחס פלק. כדי להשיג את ה-\(\hbar\). או אלטרנטיבית, להראות בעזרת משוואת שרדינגר והקוונטיזציה ראשונה ניתן לקבל את הביטוי עבור אופרטור התנע של חלקיק חופשי.

\end{remark}
\begin{proposition}
מיחס השלמות של \(P\), מתקיים:
$$\int_\left\{  \ket{p}   \right\} \ket{p} \bra{p} =Id$$

\end{proposition}
\begin{proposition}
$$\langle x|p\rangle=\frac{1}{\sqrt{2\pi\hbar}}e^{i p x/\hbar}$$

\end{proposition}
\begin{proof}
אנו יודעים כי \({\hat{p}}=-i\hbar{\frac{d}{d x}}\) וכן כי הע"ע של התנע בבסיס המקום מקיים:
$$P\ket{p} =p\ket{p} \implies \bra{x} P\ket{p} =\bra{x} p\ket{p}\implies {\hat{p}}\psi_{p}(x)=p\psi_{p}(x) $$
כאשר סימנו \(\braket{ x | p }\) ב-\(\psi_{p}(x)\). זוהי משוואה דיפרנציאלית. נציב את אופרטור התנע ונקבל:
$$-i\hbar\frac{d}{d x}\psi_{p}(x)=p\psi_{p}(x)\implies \psi_{p}(x)=A e^{i p x/\hbar}$$
נקבל את \(A\) מהדרישה לנרמול. מתקיים:
$$\left\langle  p|p^{\prime} \right\rangle=\delta\left( p-p^{\prime} \right)\implies \int_{-\infty}^{\infty}\langle p|x\rangle\langle x|p^{\prime}\rangle\,d x=\delta(p-p^{\prime})$$
כאשר נציב את התוצאה שקיבלנו \(\braket{ p | x }=Ae^{ ipx/\hbar }\) ונקבל:
$$|A|^{2}\int_{-\infty}^{\infty}e^{i(p-p^{\prime})x/\hbar}\,d x=\delta(p-p^{\prime})$$
כאשר אנו יודעים כי עבור התמרת פורייה של 1 מתקיים:
$$\int_{-\infty}^{\infty}e^{i(p-p^{\prime})x/\hbar}\,d x=2\pi\hbar\delta(p-p^{\prime})$$
ולכן:
$$A={\frac{1}{\sqrt{2\pi\hbar}}}\implies \langle x|p\rangle={\frac{1}{\sqrt{2\pi\hbar}}}e^{i p x/\hbar}$$

\end{proof}
\begin{proposition}[התנע כהתמרת פורייה]
$$\psi(x)=\langle x|\psi\rangle=\int\!\!d p\;\langle x|p\rangle\langle p|\psi\rangle=\int\!\!d p\;{\frac{e^{i x p/\hbar}\tilde{\psi}(p)}{\sqrt{2\pi\hbar}}}$$
ולכן נקבל כי \(k=\frac{p}{\hbar}\) זה ההתמרת פורייה של המיקום.

\end{proposition}
\section{קוונטיזציה קנונית ראשונה}

\begin{definition}[קוונטיזציה קנונית ראשונה]
תהליך להעביר משוואות קלאסיות למשוואות גל קוונטיות. מעביר את המשתנים לאופרטורים הרמיטים הבאים:
$$x\mapsto \hat{X}\quad p\mapsto \hat{P}\quad \mathcal{H}\to \hat{H}\quad $$

\end{definition}
\begin{remark}
קיימת גם קוונטיזציה קוונטית שנייה, אשר בעיקר עוסקת בלהכליל את תורת השדות הקלאסית לעולם הקוונטי.

\end{remark}
\begin{definition}[עקרון ההתאמה]
עקרון שקבוע שהקוונטיציה הקוונטית היא הכללה של הפיזיקה הקלאסית, ובקנה מידה גדול נצפה לקבל את המשוואות הקלסיות המוכרות.

\end{definition}
\begin{proposition}
אם מופיע \(xp\) בצורה הקלאסית, נעביר אותו ל:
$$xp\mapsto \frac{1}{2}\left\{  \hat{X},\hat{P}  \right\}$$

\end{proposition}
הסבר לזה הוא שאופן כללי האופרטור \(XP\) אינו אופרטור הרמיטי, לכן לא ניתן להעביר אותו כמו שהוא, אבל ניתן לקבל מהם אופרטור הרמיטי בעזרת האנטי קומטטור:
$$.x p={\frac{1}{2}}\left(x p+p x\right)\longrightarrow{\frac{1}{2}}\left(X P+P X\right)={\frac{1}{2}}\left\{X,P\right\}$$

\begin{proposition}[קומוטטור של מיקום ותנע]
האופרטורים \(X,P\) מקיימים:
$$[X,P]=i\hbar$$

\end{proposition}
\begin{proof}
נכתוב את את משפט אהרנפסט על אופרטור המיקום כאשר המיקום לא תלוי מפורשות בזמן:
$${\frac{d\left\langle X\right\rangle}{d t}}={\frac{i}{\hbar}}\left\langle\left[H,X\right]\right\rangle$$
נחשב את יחס החילוף \([H,X]\) עבור המילטוניאן על חלקיק עם פואנציאל כללי:
$$.\left[H,X\right]=\left[{\frac{P^{2}}{2m}}+V\left(X\right),X\right]={\frac{1}{2m}}\left[P^{2},X\right]+\left[V\left(X\right),X\right]$$
כיוון שכל אופרטור חלופי עם עצמו, מתקיים \([V(X),X]=0\). וכן מתכונות הקומוטטור מתקיים:
$$\left[P^{2},X\right]=P\left[P,X\right]+\left[P,X\right]P$$
ולכן נקבל:
$$.\frac{d\left\langle X\right\rangle}{d t}=\frac{i}{2m\hbar}\left\langle P\left[P,X\right]\right\rangle+\frac{i}{2m\hbar}\left\langle\left[P,X\right]P\right\rangle$$
כדי שערך התצפית של המיקום יקיים את משוואות המילטון, כלומר יקיים:
$$.\frac{d\left\langle X\right\rangle}{d t}=\frac{i}{2m\hbar}\left\langle P\left[P,X\right]\right\rangle+\frac{i}{2m\hbar}\left\langle\left[P,X\right]P\right\rangle$$
ונשים לב כי אם \([P,X]=i\hbar\) נקבל שמתקיים.

\end{proof}
\begin{proposition}[קומוטטור של התנע עם פונקציה של המיקום]
$$[P,F(x)]=-i\hbar{\frac{d F}{d X}}$$

\end{proposition}
\begin{proof}
נפכיל את הקומוטטור על \(\phi\):
\begin{gather*}[P,F(x)]\phi=P\left( F(x)\phi \right)-F\left( P\phi \right)=\frac{\hbar}{i}\frac{\mathrm{d} }{\mathrm{d} x}\left( F(x)\phi \right) -F\left( \frac{\hbar}{i}\frac{\mathrm{d} \phi}{\mathrm{d} X}  \right)  
\end{gather*}
כאשר בעזרת נגזרת מכפלה נקבל:
$$=\frac{\hbar}{i}\left( \frac{\mathrm{d} F}{\mathrm{d} X}\phi +F\frac{\mathrm{d} \phi}{\mathrm{d} X}  -F\frac{\mathrm{d} }{\mathrm{d} X} \right)=\frac{\hbar}{i}\frac{\mathrm{d} F}{\mathrm{d} X}\phi =-i\hbar \frac{\mathrm{d} F}{\mathrm{d} X} \phi$$

\end{proof}
\begin{proposition}[סד"פ קוונטיזציה]
  \begin{enumerate}
    \item מיקום \(x\) יהפוך לאופרטור \(X\). 


    \item התנע \(p\) יהפוך לאופרטור \(P\). 


    \item האופרטורים אינם חילופיים, ומקיימים את יחס החילוף \([X,P]=i\hbar\). 


    \item כל פונקציה שלהם תהפוך לפונקציה של האופרטור, כולל ההמילטוניאן \(H\), כאשר אם מקבלים אופרטור לא הרמיטי, מבצעים סימטריזציה באמצעות האנטי קומוטטור. 


  \end{enumerate}
\end{proposition}
\begin{proposition}[אי וודאות קנונית של מיקום תנע]
$$\triangle X\triangle P\geq{\frac{\hbar}{2}}$$

\end{proposition}
\section{אופרטור המיקום והתנע}

\begin{definition}[אופרטור המיקום]
המיקום הוא גודל מדיד, לכן ניתן לתאר אותו ע"י אופרטור הרמיטי \(X\). כיוון שהערכים האפשריים של \(X\) הם רציפים, הע"ע של האופרטור \(X\) הם רציפים.

\end{definition}
\begin{proposition}
הווקטורים העצמיים של האופרטור \(X\) הם פונקציות דלתא של המשתנה \(x\).

\end{proposition}
\begin{proposition}[הסתברות למצוא חלקיק בתחום]
ההסתברות למצוא חלקיק על הישר הממשי \(\mathbb{R}\) בתחום \(a\leq x \leq b\) יהיה:
$$P_{X}(\psi)([a,b])=\int_{a}^{b}|\psi|^{2}d\!\mathrm{x}$$

\end{proposition}
\begin{proposition}[ערך מצופה של מיקום]
$$\langle X\rangle_{\psi}=\int_{\mathbb{R}}\mathbf{x}|\psi|^{2}d\mathbf{x}=\int_{\mathbb{R}}{\psi^{*}\mathbf{x}\psi\,d\mathbf{x}}$$

\end{proposition}
\begin{definition}[אופרטור הזזה]
אופרטור \(T\) המקיים:
$$T(a)\ket{x} = \ket{x+a}$$

\end{definition}
\begin{proposition}[תכונות של אופרטור הזזה]
  \begin{enumerate}
    \item אוניטרי - משימור הסתברות.  


    \item הרכבה של אופרטרי הזזה נותן אופרטור הזזה: 
$$T(a)T(b)\ket{x}=\ket{x+a+b}  $$


    \item האופרטור \(T(0)=Id\). כלומר לא לקדם שקול ללא לעשות כלום. 


  \end{enumerate}
\end{proposition}
\begin{remark}
אופרטור הזזה למעשה יוצר שינוי במקום, בדומה לאופרטור קידום בזמן.

\end{remark}
\begin{definition}[אופרטור התנע]
התנע הוא גודל מדיד לכן ניתן לתאר אותו ע"י אופרטור אוניטרי \(P\).

\end{definition}
\begin{proposition}[אופרטור התנע כיוצר של שינוי במיקום]
אופרטור התנע מקיים:
$$i\hbar \frac{\mathrm{d} }{\mathrm{d} x} \ket{x} = P\ket{x} $$

\end{proposition}
\begin{proof}
נרצה ליצור אופרטור אשר מקדם את המצב הקוונטי. כלומר אופרטור המקיים:
$$T(a)\ket{x} = \ket{x+a} $$
משימור הסתברות נקבל כי אפרטור זה אוניטרי. נסתכל על הפיתוח טיילור של קידום ב-\(dx\):
$$T(dx)=I+T(0)dx+O(dx^2)\implies T(dx)\ket{x} =\ket{x} +\dot{T}(0)\ket{x} dx+O(dx^2)\ket{x} =\ket{x+dx} $$
נעביר אגפים, נחלק ב-\(dx\) ונקבל:
$$\frac{\ket{x+dx}-\ket{x} }{dx}=\dot{T}(x)\ket{x} +\cancel{ \frac{O(dx^2)}{dx} }\implies  \frac{d}{dx}\ket{x} =\dot{T}(0)\ket{x}  $$
כיוון ש-\(T\) הרמיטי, אז הנגזרת של \(T\) תהיה אנטי הרמיטית, ולכן אם נכפיל ב-\(i\) נקבל \(i\dot{T}\) אופרטור הרמיטי, אשר נסמן ב-\(P\). כלומר:
$$i\dot{T}(0)=P\implies \dot{T}=\frac{P}{i }\implies i\frac{d}{dx}\ket{x}  =  P \ket{x}  $$
כעת מטעמי יחידות ומכך שנרצה שיתאים לעקרון דה ברויי נדרש להכפיל ב-\(\hbar\) ולקבל:
$$i\hbar \frac{\mathrm{d} }{\mathrm{d} x} \ket{x} = P\ket{x} $$

\end{proof}
\begin{remark}
ניתן היה להוכיח באופן זהה לחלוטין להוכחה של אופרטור קידום בזמן, ונשתמש ביחס דה ברויי במקום יחס פלק. כדי להשיג את ה-\(\hbar\).

\end{remark}
\begin{proposition}[תנע בבסיס המיקום]
$$\psi_{p}\left(x\right)=\left\langle x|p\right\rangle={\frac{1}{\sqrt{2\pi\hbar}}}e^{i p x/\hbar}$$

\end{proposition}
\begin{symbolize}
נשים לב כי \(P\ket{\psi}\) אומר שאנחנו מפעילים את האופרטור \(P\) על המצב הקוונטי \(\ket{\psi}\). כאשר \(\bra{x}P\ket{\psi}\) אומר שאנחנו מפעילים את האופרטור \(P\) על \(\ket{\psi}\) בבסיס המקום. לעומת זאת \(\bra{x}X\ket{x}\) זה הערך של "המטריצה" במיקום \(x,x\) כלומר על האלכסון.

\end{symbolize}
\begin{proposition}[אופרטורים בהצגת המקום]
\begin{gather*}\left\langle  x|X|\psi \right\rangle=x\psi(x)  & \langle x|X|x^{\prime}\rangle=x^{\prime}\delta(x-x^{\prime}) \\\langle x|P|\psi\rangle=-i\hbar\frac{\partial}{\partial x}\psi(x) & \langle x|P|x^{\prime}\rangle=-i\hbar{\frac{\partial}{\partial x}}\delta(x-x^{\prime})
\end{gather*}

\end{proposition}
\begin{proposition}[מעבר בין בסיס המיקום ובסיס התנע]
נעשה באמצעות טרנספורם פורייה:
$$\widetilde{\psi}(p)=\langle p|\int d x\psi(x)|x  \rangle =\int d x\psi\left(x)\left\langle p\right|x\right\rangle=\frac{1}{\sqrt{2\pi\hbar}}\int d x\psi\left(x\right)e^{-i p x/\hbar}$$
כאשר מהעבר מבסיס התנע לבסיס המיקום נעשה באמצעות טרנספורם פורייה הפוך:
$$\psi(x)=\langle x|\int d p\widetilde{\psi}(p)|p\rangle=\int d p\widetilde{\psi}(p) \langle x|p\rangle=\frac{1}{\sqrt{2\pi\hbar}}\int d p\widetilde{\psi}(p)e^{i p x/\hbar}$$

\end{proposition}
\begin{proposition}[אופרטורים בהצגת התנע]
\begin{gather*}\left\langle  p|X|\psi \right\rangle=i\hbar{\frac{\partial}{\partial p}}\tilde{\psi}(p)&\left\langle  p|X|p^{\prime} \right\rangle=i\hbar{\frac{\partial}{\partial p}}\delta\left( p-p^{\prime} \right)\\ \left\langle  p|P|\psi \right\rangle=p\tilde{\psi}(p)& \left\langle  p|P|p^{\prime} \right\rangle=p^{\prime}\delta\left( p-p^{\prime} \right) 
\end{gather*}

\end{proposition}
\begin{proposition}[משוואת שרדינגר בבסיס המיקום והתנע]
$$\psi(x,t)=g(t)f(x)=e^{-{\frac{i}{\hbar}}E t}f(x)$$

\end{proposition}
\chapter{פתרונות מיוחדים של משוואת שרדינגר}

\section{עוד על משוואת שרדינגר}

\begin{definition}[נרמול]
אנו דורשים שסך ההסתברות שחלקיק נמצא איפשהו יהיה 1:
$$\int_{-\infty}^{+\infty}|\Psi(x,t)|^{2}\;d x=1$$

\end{definition}
\begin{proposition}
משוואת שרדינגר לא מבטיחה שהפתרון מנורמל. אבל באופן כללי כאשר \(A=\int_{-\infty}^{+\infty}|\Psi(x,t)|^{2}\;d x\) סופי מתקיים:
$$\frac{1}{A}\int_{-\infty}^{+\infty}|\Psi(x,t)|^{2}\;d x=1$$
כאשר פתרון זה פותר בעצמו את משוואת שרדינגר מלינאריות. ולכן זה יהיה הפתרון הפיזיקלי.

\end{proposition}
\begin{proposition}
משוואת שרדינגר מבטיחה שהנרמול ישמר לאורך זמן, ולכן הפקטור נרמול \(A\) נשאר קבוע.

\end{proposition}
\begin{proof}
אנו יודעים כי מתקיים:
$${\frac{d}{d t}}\int_{-\infty}^{+\infty}|\Psi(x,t)|^{2}\;d x=\int_{-\infty}^{+\infty}{\frac{\partial}{\partial t}}\,|\Psi(x,t)|^{2}\;d x$$
כאשר:
$${\frac{\partial}{\partial t}}\left|\Psi\right|^{2}={\frac{\partial}{\partial t}}\left(\Psi^{*}\Psi\right)=\Psi^{*}{\frac{\partial\Psi}{\partial t}}+{\frac{\partial\Psi^{*}}{\partial t}}\Psi$$
כעת ממשוואת שרדינגר מתקיים:
\begin{gather*}{\frac{\partial\Psi}{\partial t}}={\frac{i\hbar}{2m}}{\frac{\partial^{2}\Psi}{\partial x^{2}}}-{\frac{i}{\hbar}}V\Psi \implies{\frac{\partial\Psi^{*}}{\partial t}}=-{\frac{i\hbar}{2m}}{\frac{\partial^{2}\Psi^{*}}{\partial x^{2}}}+{\frac{i}{\hbar}}V\Psi^{*}  \\\implies {\frac{\partial}{\partial t}}\left|\Psi\right|^{2}={\frac{i\hbar}{2m}}\left(\Psi^{*}{\frac{\partial^{2}\Psi}{\partial x^{2}}}-{\frac{\partial^{2}\Psi^{*}}{\partial x^{2}}}\Psi\right)={\frac{\partial}{\partial x}}\left[{\frac{i\hbar}{2m}}\left(\Psi^{*}{\frac{\partial\Psi}{\partial x}}-{\frac{\partial\Psi^{*}}{\partial x}}\Psi\right)\right]
\end{gather*}
כאשר כעת ניתן לחשב את האינטגרל המקורי:
$$\frac{d}{d t}\int_{-\infty}^{+\infty}|\Psi(x,t)|^{2}\ d x=\frac{i\hbar}{2m}\left.\left(\Psi^{*}\frac{\partial\Psi}{\partial x}-\frac{\partial\Psi^{*}}{\partial x}\Psi\right)\right|_{-\infty}^{+\infty}$$
כאשר נדרוש כי הפונקציית גל תדעך באינסוף, ונקבל:
$$\frac{d}{d t}\int_{-\infty}^{+\infty}|\Psi(x,t)|^{2}\;d x=0$$
כלומר האינטגרל קבוע, ואם מנורמל בזמן \(t=0\) ישאר מנורמל לאורך זמן.

\end{proof}
\begin{remark}
לעיתים קיימים פתרונות למשוואת שרדינגר שעבורם ההסתברות מתבדרת, ולכן לא ניתן לנרמל אותם. אלו פתרונות מתמטיים שאינם פיזיקאלים, ויש לפסול אותם.

\end{remark}
\subsection{הבדל בין משוואת שרדינגר התלויה בזמן והבלתי תלוייה בזמן}

המשוואת שרדינגר האמיתית היא:
$$i\hbar{\frac{\mathrm{d}}{\mathrm{d}t}}|\psi\rangle={\hat{H}}(t)|\psi\rangle$$
זה נכון תמיד, וניתן לפתור בעזרתה כל בעיה∞ במערכת סגורה. אך במקרה שההמילטוניאן(או מספיק במקרה שלנו, פוטנציאל) לא תלוי בזמן, ניתן לקבל פתרון בצורה קצרה יותר. אם נפריד את הפתרון לפתרון דינמי(תלוי בזמן), ולפתרון סטציונארי(לא תלוי בזמן), נקבל כי ניתן לפתור ישירות עבור החלק הזמני, ולקבל:
$$|\psi(t)\rangle=e^{-i E t/\hbar}|\psi_{0}\rangle$$
כאשר עבור החלק הסטציונארי נקבל:
$$H\ket{\psi_{0}} =E\ket{\psi_{0}} $$
מהמשוואה הסציונארית ניתן למצוא את \(E\) ואז להציב במשוואה הדינאמית. הבעיה זה שבדרך כלל יש הרבה ערכים סטציונארים, כאשר הערכים הסטציונארים הם יהיו למעשה הע"ע של ההמילטוניאן, ולכן מייצגים רמות אנרגיות שונות. לכל רמת אנרגיה יש למערכת הסתברות שונה להיות בה. ולכן נדרש לקדם כל רמת אנרגיה בנפרד בהתאם להסתרות שלה. ולכן נקבל כי ההתקדמות של הפונקציית גל תהיה:
$$|\psi(t)\rangle=\sum_{n}a_{n}e^{-i E_{n}t/\hbar}|n\rangle$$
או כאשר אנחנו במקרה רציף:
$$|\psi(t)\rangle=\int_{n}\mathrm{d}n\;a_{n}e^{-i E_{n}t/\hbar}|n\rangle$$

\subsection{מיון פתרונות של משוואת שרדינגר}

אם נכתוב את אופרטור התנע בעזרת אופרטור המיקום, ניתן לכתוב את משוואת שרדינגר הבלתי תלוייה בזמן בצורה הבאה:
$$-\,\frac{\hbar^{2}}{2m}\frac{d^{2}\psi}{d x^{2}}=(E-V(x))\psi(x)$$
כאשר סוג הפתרון של המשוואה הזאת יכול להיות או אוסצילטורי או אקספוננטי כתלות בסימן של \(E-V(x)\).

\begin{proposition}
אם \(E-V(x)>0\) נקבל כי הפתרון יהיה אקספוננט דועך או גדל. במקרה זה נקבל כי האנרגיה תהיה רציפה. במקרה זה פונקציית הגל תהיה
$$\psi(x)=\mathrm{e}^{(i p x)/\hbar} \qquad \tilde{p}=\sqrt{2m(E-V)}$$

\end{proposition}
\begin{proposition}
אם \(E-V(x)<0\) נקבל כי הפתרון יהיה אוסצילטורי, דבר זה יכניס אילוץ מחזורי אשר יגרום לכך שרמות האנרגיה יהיו דיסקרטיות. במקרה זה פונקציית הגל תהיה:
$$\psi=e^{ -\tilde{p}x/\hbar }\qquad \tilde{p}=\sqrt{ 2m(V-E) }$$

\end{proposition}
\begin{proposition}[פוטנציאל אינסופי]
כאשר \(V(x)=\infty\) נקבל כי הפתרון הסטציונרי יחידי יהיה \(\psi=0\).

\end{proposition}
בגלל מנהור, חלקיק יכול לעבור בור פוטנציאל, ולכן מה שבסוף משנה זה הפוטנציאל באינסוף.

\begin{definition}[מצב קשור]
כאשר האנרגיה של החלקיק קטן מהפוטנציאל באיסוף, כלומר כאשר \(E<V\left( -\infty \right),V\left( \infty \right)\). במצב זה החלקיק תקוע בתוך אזור ולא יכול להתפזר לאינסוף.

\end{definition}
\begin{definition}[מצב חופשי/מצב פיזור]
כאשר האנרגיה של החלקיק גדולה מהפוטנציאל באינסוף, כלומר \(E>V\left( -\infty \right),V\left( \infty \right)\). במקרה זה נקבל כי תמיד קיים הסתברות שהחלקיק יברח לאינסוף.

\end{definition}
\begin{remark}
קיימים פוטנציאלים שללא תלות ב-\(E\) נקבל כי המצב יהיה קשור(לדוגמא אוסצילטור הרמוני) ומצבים שללא תלות ב-\(E\) נקבל מצב חופשי(לדוגמא חלקיק חופשי) אך גם קיימים פוטנציאלים שכל אחד מהמצבים יכול להתקיים כתלות ב-\(E\), ולכן כדי לקבל פתרון מלא יש לבדוק את שתי המקרים.

\end{remark}
\begin{definition}[נרמול דלתא]
עבור מצבים שאינם ניתנים לנרמול, ניתן לדרוש נרמול של פונקציית דלתא:
$$\langle p|p^{\prime}\rangle=\delta(p-p^{\prime})$$

\end{definition}
\begin{remark}
זה לא מצב פיזיקלי כיוון שלא מנורמל. זה רק נהיה מצב פיזיקלי כאשר התנאי התחלה או שפה מאלצים את המצב להיות מנורמל. בכל זאת לעיתים נוח לנתח כך בעיות כדי לפתור אותם בצורה כללית.

\end{remark}
\subsection{צפיפות הסתברות וזרם הסתברות}

\begin{definition}[צפיפות הסתברות]
נסמן את צפיפות ההסתברות ב-\(x\) בזמן \(t\) ע"י:
$$\rho\left(x,t\right)\equiv\left|\psi\left(x,t\right)\right|^{2}$$

\end{definition}
\begin{definition}[זרם הסתברות]
$$j\left(x\right)=\frac\hbar{2i m}\left(\overline{{{\psi}}}\frac{\partial\psi}{\partial x}-\psi\frac{\partial\overline{{{\psi}}}}{\partial x}\right)=\frac\hbar m\mathrm{Im}\left(\overline{{{\psi}}}\frac{\partial\psi}{\partial x}\right)$$

\end{definition}
\begin{proposition}[משוואת הרציפות]
תחת האילוץ הפיזיקלי ש-\(V\) ממשי, נקבל כי:
$${\frac{\partial\rho}{\partial t}}+{\frac{\partial j}{\partial x}}=0$$

\end{proposition}
\begin{proof}
משוואת שרדינגר בבסיס המקום והצמוד שלה הם:
$$i\hbar\frac{\partial}{\partial t}\psi\left(x,t\right)=\left(-\frac{\hbar^{2}}{2m}\frac{\partial^{2}}{\partial x^{2}}+V\left(x\right)\right)\psi\left(x,t\right)\quad -\left.\hbar{\frac{\partial}{\partial t}}{\overline{{\psi}}}\left(x,t\right)=\left(-{\frac{\hbar^{2}}{2m}}{\frac{\partial^{2}}{\partial x^{2}}}+{\overline{{V}}}\left(x\right)\right){\overline{{\psi}}}\left(x,t\right)\right.$$
כאשר ע"י כפל המשוואה הראשונה ב-\(\overline{\psi}\) והשנייה ב-\(\psi\) נקבל כי:
$$.\hbar{\frac{\partial\rho}{\partial t}}=\hbar\left({\frac{\partial{\overline{{\psi}}}}{\partial t}}\psi+{\overline{{\psi}}}{\frac{\partial\psi}{\partial t}}\right)=-{\frac{\hbar^{2}}{2m}}\left({\overline{{\psi}}}{\frac{\partial^{2}\psi}{\partial x^{2}}}-\psi{\frac{\partial^{2}{\overline{{\psi}}}}{\partial x^{2}}}\right)+\left(V-{\overline{{V}}}\right)\rho$$
כאשר נציב את ההגדרות ונקבל:
$${\frac{\partial\rho}{\partial t}}+{\frac{\partial j}{\partial x}}=-{\frac{i}{\hbar}}\left(V-{\overline{{V}}}\right)\rho$$

\end{proof}
\section{חלקיק חופשי}

\begin{definition}[חלקיק חופשי]
חלקיק שבו הפוטנציאל הוא אפס. כלומר ההמילטוניאן מקיים:
$$\mathcal{H}= \frac{P^{2}}{2m}$$

\end{definition}
זוהי בעיה שטבעי לפתור אותה בבסיס התנע כיוון שההמילטוניאן תלוי ב-\(P\) בלבד, ואין השפעה של המיקום בפוטנציאל, לכן נקבל ישירות משוואה עבור התנע.

\begin{proposition}[אנרגיה בבסיס התנע]
ניתן להציג את האנרגיה בעזרת התנע בצורה הבאה:
$$E_{p}={\frac{p^{2}}{2m}}$$

\end{proposition}
\begin{remark}
נשים לב כי כל המצבים העצמיים של האנרגיה מנוונים(חוץ מעבור \(p=0\)), והספקטרום רציף. זה אומר כי לכל ע"ע של אנרגיה יש שתי ע"ע של התנע אשר מתאימים למצב עצמי של אנרגיה. מבחינה פיזיקלית, זה אומר כי עבור ערך מסויים של אנרגיה החלקיק יכול לנוע או ימינה או שמאלה. כפי שנראה בהמשך, הניוון הזה נובע מהסימטריה להיפוך מרחבי.

\end{remark}
\begin{proof}
נשים לב כי ההמילטוניאן הוא פונקציה של התנע, ולכן בפרט הקומוטטור מתאפס ולכן קיים בסיס משותף עבור התנע והאנרגיה. ניתן לכתוב את האנרגיה של המערכת בצורה \(E=\frac{p^{2}}{2m}\). זהו אנרגיה קבועה שתלוייה ב-\(p\). לכן משוואת שרדינגר תתן:
$$\mathcal{H}\ket{p} =\frac{p^{2}}{2m}\ket{p} $$
עבור מצב בבסיס התנע \(\ket{p}\).

\end{proof}
\begin{proposition}[משוואת הגל כפונקציה של התנע]
$$\ket{\psi(t)} = \int \mathrm{d}p \;\mathrm{Prob}(p) e^{-i E_{p}/\hbar}=\int\mathrm{d}p\;\psi(p,t)e^{ -ip^{2}t/2m\hbar  }$$
כאשר באופן כללי נדרש לבצע אינטגרציה לפי בסיס האנרגיה, אך כיוון שההמילטוניאן מתחלף עם התנע, קיים בסיס משותף וניתן לבצע אינטגרציה לפי בסיס זה.

\end{proposition}
\begin{proposition}[משוואת התנועה בבסיס המיקום]
$$\psi\left(x,t\right)=\left\langle x|\psi\left(t\right)\right\rangle={\frac{1}{\sqrt{2\pi\hbar}}}\int d p e^{i p x/\hbar}e^{-i{\frac{p^{2}}{2m\hbar}}t}{\psi}\left(p,t=0\right)$$

\end{proposition}
\begin{remark}
כדי שיהיה מצב פיזיקלי נדרש כי פונקציית הגל תהיה מנורמלת. לכן כדי שנקבל מצב פיזיקלי, נדרש כי תנאי ההתחלה יהיה גל מנורמל, כאשר במקרה זה הנרמול ישמר לאורך זמן. בכל זאת במקרים רבים (למשל פיזור) שימושי להשתמש במצב למרות שלא מנורמל כיוון שמהוות בסיס שלם.

\end{remark}
\begin{proposition}
ערך התצפית של התנע לא משתנה בזמן.

\end{proposition}
\begin{proof}
נובע מכך ש-\([H,P]=0\), ולכן מאהרנפסט הנגזרת של \(P\) לא משתנה בזמן. לחלופין ניתן להראות ישירות:
$$\left\langle\psi\left(t\right)\right|P\left|\psi\left(t\right)\right\rangle=\int d p p\left|\widetilde{\psi}\left(p,t\right)\right|^{2}=\int d p p\left|\widetilde{\psi}_{0}\left(p\right)\right|^{2}=\left\langle\psi_{0}\right|P\left|\psi_{0}\right\rangle=p_{0}$$

\end{proof}
\begin{corollary}
אי וודאות של התנע \(\triangle P\) לא משתנה בזמן.

\end{corollary}
\begin{proposition}[ערך התצפית של המיקום]
לפי משפט אהרנפסט מתקיים:
$$\frac{d x\left(t\right)}{d t}=\frac{d\left\langle X\right\rangle}{d t}=\frac{i}{\hbar}\left\langle\left[H,X\right]\right\rangle=\frac{\left\langle P\right\rangle}{m}=\frac{p_{0}}{m}$$
ולכן:
$$x\left(t\right)=x_{0}+{\frac{p_{0}}{m}}t$$

\end{proposition}
\begin{remark}
ניתן גם לחשוב על זה כמו מהירות החבורה של גל. אנו יודעים כי מהירות של גל תהיה בקרוב \(v_{g}=\left.\frac{\partial\omega}{\partial k}\right|_{k=k_{0}}\) כאשר מתקיים:
$$\omega={\frac{E}{\hbar}}={\frac{p^{2}}{2m\hbar}}={\frac{\hbar k^{2}}{2m}} \implies v_{g}=\left.{\frac{\partial\omega}{\partial k}}\right|_{k=k_{0}}={\frac{\hbar k_{0}}{m}}={\frac{p_{0}}{m}}$$

\end{remark}
\begin{proposition}[צפיפות ההסתברות]
צפיפות ההסתברות של גל מישורי נתון על ידי:
$$J={\frac{p}{m}}|\psi_{p}(x)|^{2}$$

\end{proposition}
\begin{proof}
נזכור כי:
$$J={\frac{\hbar}{m}}\mathrm{Im}\left(\psi^{*}{\frac{d\psi}{d x}}\right)$$
ולכן עבור גל מישורי \(\psi_{p}(x)=e^{ ipx/\hbar }\) נקבל \(J={\frac{p}{m}}|\psi_{p}(x)|^{2}.\)

\end{proof}
\section{בור פוטנציאל אינסופי}

\begin{definition}[בור בפוטנציאל אינסופי]
מערכת עם פוטנציאל מהצורה:
$$V(x)=\left\{{\begin{array}{l l}{{0}}&{{{0\leq x\leq a}}}\\ {{\infty,}}&{{\mathrm{otherwise}}}\end{array}}\right.$$

\end{definition}
\begin{proposition}
כאשר הפוטנציאל הוא \(\infty\), ההסתברות שהחלקיק ימצא במקום הוא 0. ולכן כיוון שפונקציית הגל צריכה להיות רציפה, נקבל כי באיזור זה \(\psi(x)=0\).

\end{proposition}
\begin{proposition}
בתוך הבור פוטנציאל, משוואת ערכים העצמיים של האנרגיה תהיה:
$$-\,\frac{\hbar^{2}}{2m}\frac{d^{2}\psi}{d x^{2}}=E\psi$$

\end{proposition}
\begin{proposition}
פונקציית הגל של בור בפוטנציאל אינסופי יהיו:
$$\psi_{n}(x)=\sqrt{\frac{2}{a}}\sin\left(\frac{n\pi}{a}x\right)$$
כאשר רמות האנרגיה יהיו:
$$E_{n}={\frac{n^{2}\pi^{2}\hbar^{2}}{2m a^{2}}}$$

\end{proposition}
\begin{proof}
ניתן לכתוב את משוואת שרדינגר של בור בפוטנציאל אינסופי בצורה:
$${\frac{d^{2}\psi}{d x^{2}}}=-k^{2}\psi \qquad k\equiv{\frac{\sqrt{2m E}}{\hbar}}$$
כאשר אנו יודעים את הפתרון של משוואה זו:
$$\psi(x)=A\sin k x+B\cos k x$$
מרציפות פונקציית הגל, נציב את התנאי שפה \(\psi(0)=\psi(a)=0\) ונקבל:
$$\psi(x)=A\,\sin k x \qquad k \in \left\{  \pi n\mid n \in \mathbb{Z}  \right\}$$
כאשר נוכל למצוא את \(A\) בעזרת דרישת הנרמול:
$$\int_{0}^{a}|A|^{2}\sin^{2}(k x)\ d x=|A|^{2}\,{\frac{a}{2}}=1\implies|A|^{2}={\frac{2}{a}}$$
ולכן פונקציית הגל תהיה:
$$\psi_{n}(x)=\sqrt{\frac{2}{a}}\sin\left(\frac{n\pi}{a}x\right)$$
כאשר ניתן גם למצוא את רמות האנרגיה ע"י שימוש בקשר \(E=\frac{k^{2}\hbar^{2}}{2m}\) ולקבל
$$E_{n}={\frac{\hbar^{2}k_{n}^{2}}{2m}}={\frac{n^{2}\pi^{2}\hbar^{2}}{2m a^{2}}}$$

\end{proof}
\begin{corollary}
ניתן לכתוב את פונקציית הגל כתלות בזמן ע"י:
$$\Psi_{n}(x,t)=\sqrt{\frac{2}{a}}\sin\left({\frac{n\pi}{a}}x\right)e^{-i\left( n^{2}\pi^{2}\hbar/2m a^{2} \right)t}\implies \Psi(x,t)=\sum_{n=1}^{\infty}c_{n}{\sqrt{\frac{2}{a}}}\sin\left({\frac{n\pi}{a}}x\right)e^{-i\,(n^{2}\pi^{2}\hbar/2m a^{2})t}$$

\end{corollary}
זה כיוון שאנו יודעים כי הגורם הזמני יהיה \(e^{ -iEt/\hbar }\).

\begin{example}
נמצא את פונקציית הגל כתלות בזמן של חלקיק בבור פוטנציאל אינסופי עם פונקציית גל התחלתית:
$$\Psi(x,0)=A x\,(a-x)\,,\quad(0\le x\le a)$$
כאשר \(A\) קבוע ופונקציית הגל תהיה 0 בשאר התחום.
נתחיל מלקבע את \(A\) בעזרת דרישת הנרמול:
$$1=\int_{0}^{a}|\Psi(x,0)|^{2}\,d x=|A|^{2}\int_{0}^{a}x^{2}\,(a-x)^{2}\,d x=|A|^{2}\,{\frac{a^{5}}{30}}\implies A=\sqrt{ \frac{30}{a^{5}} }$$
כעת נרצה לפרוש את התנאי התחלה המרחבי על הפונקציות העצמיות המרחביות ונקבל:
\begin{gather*}c_{n}=\sqrt{\frac{2}{a}}\int_{0}^{a}\sin\left(\frac{n\pi}{a}x\right)\sqrt{\frac{30}{a^{5}}}\,x\,\left(a-x\right)\,d x =\frac{4\sqrt{15}}{\left(n\pi\right)^{3}}\left[\cos\left(0\right)-\cos\left(n\pi\right)\right]=\left\{\begin{array}{l l}{{0,}}&{{n\;\mathrm{even},}}\\ {{8\sqrt{15}/\left(n\pi\right)^{3},}}&{{n\;\mathrm{odd}.}}\end{array}\right.
\end{gather*}
ולכן הפתרון הכולל יהיה:
$$\Psi(x,t)={\sqrt{\frac{30}{a}}}\left({\frac{2}{\pi}}\right)^{3}\sum_{n=1,3,5\ldots}{\frac{1}{n^{3}}}\sin\left({\frac{n\pi}{a}}x\right)e^{-i n^{2}\pi^{2}\hbar t/2m a^{2}}$$

\end{example}
\section{פוטנציאל מדרגה}

\begin{definition}[פוטנציאל מדרגה]
פוטנציאל מהצורה:
$$V(x)=\left\{{0\quad x\leq0}\atop{U\quad x>0}\right.$$
כאשר \(U>0\) זה קבוע.

\end{definition}
כאשר אנו רוצים לפתור את משוואת שרדינגר הבלתי תלוייה בזמן:
$${\frac{-\hbar^{2}}{2m}}\psi^{\prime\prime}+V(x)\psi=E\psi$$

\subsection{המקרה \(E<U\)}

נסמן:
$$k\equiv\sqrt{2m E/\hbar^{2}} \qquad \tilde{\kappa}\equiv\sqrt{2m(U-E)/\hbar^{2}}$$
כאשר משוואת שרדינגר בשתי התחומים תהיה:
$$\begin{cases}\psi^{\prime\prime}+k^{2}\psi=0 &x<0 \\\psi^{\prime\prime}-\tilde{\kappa}^{2}\psi=0 &x>0
\end{cases}$$
כאשר הפתרונות יהיו:
$$\psi=\begin{cases}e^{ ikx }+Ae^{ -ikx } & x<0 \\Be^{ -\tilde{\kappa}x } & x>0
\end{cases}$$
כאשר נדרוש רציפות פונקציית הגל והנגזרת ונקבל:
$$\begin{cases}1+A=B \\ik-ikA=-\tilde{\kappa}B
\end{cases}\implies A=\frac{k-i\tilde{\kappa}}{k+i\tilde{\kappa}},\quad B=\frac{2k}{k+i\tilde{\kappa}}$$

\subsection{המקרה \(E>U\)}

במקרה זה נגדיר:
$$k\equiv\sqrt{2m E/\hbar^{2}}\qquad \kappa\equiv{\sqrt{2m(E-U)/\hbar^{2}}}$$
כאשר משוואת שרדינגר בכל תחום תתן
$$\left\{\begin{array}{l l}{{\psi^{\prime\prime}+k^{2}\psi=0}}&{{x<0}}\\ {{\psi^{\prime\prime}+\kappa^{2}\psi=0}}&{{x>0}}\end{array}\right.$$
וכעת הפתרון יהיה
$$\psi=\begin{cases}e^{ ikx }+R e^{ -ikx } & x<0 \\Te^{ i\kappa x }
\end{cases}$$
ומתנאי לרצפיפות פונקציית הגל והנגזרת נקבל כי
$$\begin{cases}1+R=T \\ik-ikR=i\kappa T
\end{cases}\implies R=\frac{k-\kappa}{k+\kappa}\quad T=\frac{2k}{k+\kappa}$$

\section{פוטנציאל קופסא}

נסתכל על המקרה שבו הפוטנציאל יהיה מהצורה
$$V(x)=\left\{\begin{array}{l l}{{0}}&{{x\leq0}}\\ {{U}}&{{0<x<a}}\\ {{0}}&{{x\geq a\,.}}\end{array}\right.$$
כאשר ניתן לסמן
$$k\equiv{\sqrt{2m E/\hbar^{2}}} \qquad \tilde{\kappa}\equiv i\sqrt{2m(U-E)/\hbar^{2}}$$
ונקבל כי משוואת שרדינגר תהיה
$$\begin{cases}\psi^{\prime\prime}+k^{2}\psi=0 & x< 0\\\psi^{\prime\prime}-\tilde{\kappa}^{2}\psi=0 & 0<x<a \\\psi^{\prime\prime}+k^{2}\psi=0 & x> 0
\end{cases}$$
כאשר אנו יודעים כי הפתרונות יהיו:
$$\psi=\begin{cases}e^{ ikx }+ R e^{ ikx } & x<0 \\Ae^{ \kappa x + B e^{ -\kappa x } } & 0<x<a \\Te^{ ikx } & x>a
\end{cases}$$
וניתן שוב למצוא את הפתרונות מרציפות הפונקציה והנגזרת ונקבל לבסוף:
$$T=e^{-i k a}\left(\cosh\kappa a-i\frac{k^{2}-\kappa^{2}}{2k\kappa}\sinh\kappa a\right)^{-1}$$
כאשר ניתן לפתור נומרית.

\section{בור פוטנציאל סופי}

ראשית נסתכל על הטענה הבאה:

\begin{proposition}
אם הפוטנציאל \(V(x)\) הוא זוגי, אז פונקציית הגל \(\psi(x)\) תהיה בהכרח או זוגית או אי זוגית.

\end{proposition}
זה נובע מזה שאם \(\psi(x)\) מקיים את משוואת שרדינגר, אז גם \(\psi(-x)\) מקיים את המשוואה, ולכן גם \(\psi(x)+\psi(-x)\)(זוגי) ו-\(\psi(x)-\psi(-x)\)(אי זוגי) מקיימות את המשוואה. 

\begin{definition}[בור בפוטנציאל סופי]
מערכת עם פוטנציאל מהצורה:
$$V(x)=\left\{\begin{array}{l l}{{-V_{0},}}&{{-a\le x\le a,}}\\ {{0,}}&{{\vert x\vert>a,}}\end{array}\right.$$
כאשר \(V_{0}\) קבוע חיובי.

\end{definition}
\begin{proposition}
בתחום \(x<-a\) נקבל:
$$\psi(x)=B e^{\kappa x}\qquad \kappa\equiv{\frac{\sqrt{-2m E}}{\hbar}}\qquad (x<-a)$$
כאשר בתחום \(-a<x<a\) נקבל:
$$\psi(x)=C\sin(l x)+D\cos(l x)\qquad l\equiv{\frac{\sqrt{2m\left(E+V_{0}\right)}}{\hbar}}\;\;\;\;(-a<x<a)$$
כאשר בתחום \(x>a\) נקבל:
$$\psi(x)=F e^{-\kappa x}\quad(x>a)$$

\end{proposition}
זה נובע פשוט מפתרון משוואת שרדינגר בכל תחום. 

\begin{proposition}
אם נסמן:
$$z\equiv l a,\quad\mathrm{and}\,\,\,z_{0}\equiv\frac{a}{\hbar}\sqrt{2m\,V_{0}}$$
הפתרונות הזוגיים יקיימו
$$\tan z={\sqrt{(z_{0}/z)^{2}-1}}$$

\end{proposition}
\begin{proof}
אם נדרוש זוגיות נקבל:
$$\psi\left(x\right)=\left\{\begin{array}{l l}{{F e^{-\kappa x},}}&{{\left(x\,>a\right),}}\\ {{D\cos(l x)\,,}}&{{\left(0<x\,<a\right),}}\\ {{\psi\left(-x\right),}}&{{\left(x\,<0\right).}}\end{array}\right.$$
וכעת מספיק לפתור עבור תנאי שפה אחד. מרציפות של \(\psi\) ושל הנגזרת שלה ב-\(x=a\), נקבל:
$$\begin{cases}F e^{-\kappa a}=D\cos(l a) \\-\,\kappa\,F e^{-\kappa\,a}=-l D\sin(l a)
\end{cases}\implies \kappa=l\tan(l a)$$
כאשר ניתן לסמן:
$$z\equiv l a,\quad\mathrm{and}\,\,\,z_{0}\equiv\frac{a}{\hbar}\sqrt{2m\,V_{0}}.$$
כאשר מתקיים:
$$\left(\kappa^{2}+l^{2}\right)\,=\,2m\,V_{0}/\hbar^{2}\qquad \kappa a\;=\;\sqrt{z_{0}^{2}-z^{2}},$$
ולכן ניתן לכתוב:
$$\tan z={\sqrt{(z_{0}/z)^{2}-1}}$$

\end{proof}
כאשר זוהי משוואת טרנזינדנטית עבור \(z\). ניתן לפתור נומרית.

\section{פוטנציאל דלתא}

\begin{definition}[פוטנציאל דלתא]
עבור קבוע ממשי \(\alpha\) פוטנציאל דלתא יהיה פוטנציאל מהצורה:
$$V(x)=-\alpha \delta(x)$$
ולכן משוואת שרדינגר תהיה
$$-\;\frac{\hbar^{2}}{2m}\frac{d^{2}\psi}{d x^{2}}-\alpha\delta(x)\;\psi=E\psi$$

\end{definition}
נשים לב כי ייתקבל מצב פיזור עבור \(E>0\) ומצב קשור עבור \(E<0\), ולכן יש לבדוק את שתי המקרים.

נשים לב כי עבור פוטנציאל דלתא \(\psi(x)\) עדיין רציף כיוון שזה אינטגרל כפול על הפוטנציאל שהוא לכל היותר \(\delta(x)\). לעומת זאת הנגזרת של \(\psi\) לא יהיה רציפה כיוון שאינטגרל בודד על פונקציית דלתא - שמחזירה פונקציית מדרגה. לכן נראה ראשית את הטענה הבאה:

\begin{proposition}[הקפיצה בנגזרת של פוטנציאל דלתא]
עבור פוטנציאל דלתא מהצורה \(V(x)=-\alpha \delta(x)\) נקבל כי הקפיצה בנגזרת של פונקציית הגל תהיה:
$$\Delta\left(\frac{d\psi}{d x}\right)=-\frac{2m\alpha}{\hbar^{2}}\psi\left(0\right)$$

\end{proposition}
\begin{proof}
ניקח אינטגרל על קטע קטן סביב 0 של משוואת שרדינגר:
$$-\frac{\hbar^{2}}{2m}\underbrace{ \int_{-\epsilon}^{+\epsilon}\frac{d^{2}\psi}{d x^{2}}\,d x }_{ 1 }+\underbrace{ \int_{-\epsilon}^{+\epsilon}V(x)\:\psi(x)\:d x }_{ 2 }=\underbrace{ { E}\int_{-\epsilon}^{+\epsilon}\psi(x)\,\,d x }_{ 3 }$$
ניקח את הגבול כאשר \(\varepsilon\to 0\). נשים לב כי אינטגרל 1 זה פשוט הנגזרת \(\frac{\mathrm{d} \psi}{\mathrm{d} x}\), כאשר אינטגרל 3 זה אינטגרל של פונקצייה חסומה בקטע עם אורך שואף לאפס, ולכן מתאפס. ניתן להעביר אגפים ולקבל:
$$\Delta\left({\frac{d\psi}{d x}}\right)\equiv\operatorname*{lim}_{\epsilon\to0}\left({\frac{\partial\psi}{\partial x}}\left|_{+\epsilon}-{\frac{\partial\psi}{\partial x}}\right|_{-\epsilon}\right)={\frac{2m}{\hbar^{2}}}\operatorname*{lim}_{\epsilon\to0}\int_{-\epsilon}^{+\epsilon}\,V(x)\,\psi(x)\,d x$$
כאשר כעת ניתן להציב \(V(x)=-\alpha \delta(x)\) ולקבל את הטענה.

\end{proof}
\begin{proposition}[הפתרון עבור מצב קשור]
כאשר \(E<0\) נקבל כי פתרון הכללי למשוואת שרדינגר יהיה מורכב ממצב קשור יחיד:
$$\psi(x)=\frac{\sqrt{m\alpha}}{\hbar}e^{-m\alpha|x|/\hbar^{2}}\;\;\;\;E=-\frac{m\alpha^{2}}{2\hbar^{2}}$$

\end{proposition}
\begin{proof}
כיוון ש-\(E>0\) ניתן לכתוב את משוואת שרדינגר(הבלתי תלוייה בזמן) בצורה הבאה:
$${\frac{d^{2}\psi}{d x^{2}}}=-{\frac{2m E}{\hbar^{2}}}\psi=\kappa^{2}\psi \qquad \kappa\equiv{\frac{\sqrt{-2m E}}{\hbar}}$$
כאשר הפתרון הכללי למשוואה זו תהיה:
$$\psi(x)=A e^{-\kappa x}+B e^{\kappa x}$$
נדרש לפצל לתחומים הרציפים \(x<0,x>0\). עבור \(x>0\) נקבל מטעמי התבדרות:
$$\psi(x)=Be^{\kappa x}$$
כאשר באופן דומה עבור \(x>0\) נקבל:
$$\psi(x)=Fe^{ -\kappa x }$$
כאשר ניתן להשתמש ברציפות של \(\psi\)(נשים לב כי \(\psi'\) לא רציף בגלל שהפוטנציאל אינסופי בנקודה). מכאן נקבל את התוצאה:
$$\psi(x)=\begin{cases}Be^{ \kappa x } & x\leq 0 \\Be^{ -\kappa x } & x\geq 0
\end{cases}\qquad \kappa^{2}\psi \qquad \kappa\equiv{\frac{\sqrt{-2m E}}{\hbar}}$$
כאשר נותר להשתמש בתנאי של הקפיצה בנגזרת:
$$\begin{cases}\frac{\mathrm{d} \psi}{\mathrm{d} x} =-B\kappa e^{ -\kappa x } & x>0 \\\frac{\mathrm{d} \psi}{\mathrm{d} x} =B\kappa e^{ -\kappa x } & x<0
\end{cases}$$
ולכן הקפיצה ב-\(x=0\) תהיה:
$$\Delta\left( \frac{\mathrm{d} \psi}{\mathrm{d} x}  \right)=-B\kappa-\left( +B\kappa \right)=-2B\kappa\overset{!}{=} -\frac{2m\alpha}{\hbar^{2}}\psi\left(0\right)=-\frac{2m\alpha}{\hbar^{2}}B$$
כאשר השתמשנו בזה ש-\(\psi(0)=B\), ולכן האנרגיות המותרות יהיו:
$$\kappa = \frac{m\alpha}{\hbar^{2}}\implies E=-\frac{\hbar^{2}\kappa^{2}}{2m}=-\frac{m\alpha^{2}}{2\hbar^{2}}$$
כאשר כעת מדרישת הנרמול נקבל:
$$\int_{-\infty}^{+\infty}|\psi(x)|^{2}\,d x=2\,|B|^{2}\int_{0}^{\infty}e^{-2\kappa x}\,d x={\frac{|B|^{2}}{\kappa}}=1\implies B=\sqrt{ \kappa }=\frac{\sqrt{ m\alpha }}{\hbar}$$
ונקבל כי ללא תלות ב-"עוצמה" \(\alpha\) תמיד יהיה מצב קשור יחיד:
$$\psi(x)=\frac{\sqrt{m\alpha}}{\hbar}e^{-m\alpha|x|/\hbar^{2}};\;\;\;\;E=-\frac{m\alpha^{2}}{2\hbar^{2}}$$

\end{proof}
\begin{proposition}[פתרון עבור מצב חופשי]
כאשר \(E<0\) נקבל כי הפתרון יהיה:
$$\psi(x)=\begin{cases}A e^{i k x}+B e^{-i k x} & x<0 \\F e^{i k x}+G e^{-i k x} & x> 0
\end{cases}$$
כאשר מרציפות וקפיצה בנגזרת נקבל:
$$F+G=A+B\qquad ik(F-G-A+B)=-\frac{2m\alpha}{\hbar^{2}}(A+B)$$
כאשר נקבל 5 משתנים ושתי אילוצים, ולכן נדרשים תנאים נוספים כדי לפתור את הבעיה.

\end{proposition}
\begin{proof}
משוואת שרדינגר תהיה:
$${\frac{d^{2}\psi}{d x^{2}}}=-{\frac{2m E}{\hbar^{2}}}\psi=-k^{2}\psi \qquad k\equiv{\frac{\sqrt{2m E}}{\hbar}}$$
כאשר הפתרון הכללי יהיה:
$$\psi(x)=\begin{cases}A e^{i k x}+B e^{-i k x} & x<0 \\F e^{i k x}+G e^{-i k x} & x> 0
\end{cases}$$
כאשר הפעם לא ניתן לפסול אף פתרון כיוון שאלו אקספוננטיים מרוכבים, ואינם מתבדרים. מרציפות נקבל:
$$F+G=A+B$$
הנגזרות יהיו:
$$\begin{cases}\frac{\mathrm{d} \psi}{\mathrm{d} x} =i k\left(F e^{i k x}-G e^{-i k x}\right) \\\frac{\mathrm{d} \psi}{\mathrm{d} x} =i k\left(A e^{i k x}-B e^{-i k x}\right)
\end{cases}$$
כאשר מהקפיצה בנגזרת ב-\(x=0\) נקבל:
$$ik(F-G-A+B)=-\frac{2m\alpha}{\hbar^{2}}(A+B)$$
כאשר נשים לב כי לא ניתן לקבל תנאים נוספים מנרמול למשל כי זה מצב לא מנורמל.

\end{proof}
\begin{proposition}[בהנתן גל המתקדם ימינה]
כאשר נתון גל המתקדם ימינה(כלומר מגיע משמאל) ההסתברות שיעבור את המחסום פוטנציאל T וההסתברות שיוחזר \(R\) יהיו:
$$R=\frac{1}{1+\left(2\hbar^{2}E/m\alpha^{2}\right)},\;\;\;\;T=\frac{1}{1+(m\alpha^{2}/2\hbar^{2}E)}$$
כאשר משימור הסתברות \(R+T=1\).

\end{proposition}
\begin{proof}
כיוון שנתון גל המתקדם ימינה, אנו יודעים כי \(G=0\). כעת ניקח את הפתרון עבור הקפיצה בנגזרת מהטענה הקודמת הקודמת כאשר נסמן \(\beta \equiv \frac{m\alpha}{\hbar^{2}k}\) ונקבל:
$$F-G=A\left(1+2i\beta\right)-B\left(1-2i\beta\right)$$
כאשר ניתן לקבל מזה:
$$B=\frac{i\beta}{1-i\beta}A,\quad F=\frac{1}{1-i\beta}A$$
כאשר ההסתברות שיעבור את המחסום פוטנציאל יהיה היחס היחסי בין ההסתברויות  שיהיה בכל חלק:
$$R\equiv\frac{|B|^{2}}{|A|^{2}}=\frac{\beta^{2}}{1+\beta^{2}}$$
כאשר מקדם ההחזרה יהיה:
$$T\equiv\frac{|F|^{2}}{|A|^{2}}=\frac{1}{1+\beta^{2}}$$
כאשר נשים לב כי מתקיים \(R+T=1\), ובהצבת הערך של \(\beta\) נקבל את הטענה.

\end{proof}
\section{אוסצילטור הרמוני קוונטי}

\begin{definition}[פוטנציאל הרמוני]
פוטנציאל מהצורה:
$$V(x)={\frac{1}{2}}m\omega^{2}X^{2}$$

\end{definition}
\begin{definition}[אוסצילטור הרמוני קוונטי]
מערכת שבה יש פוטנציאל הרמוני, כלומר מערכת שההמילטוניאן שלה יהיה:
$$H={\frac{p^{2}}{2m}}+{\frac{m\omega^{2}x^{2}}{2}}$$

\end{definition}
ניתן לפתור מערכת זו בשתי דרכים - הדרך הראשונה בעזרת טור חזקות(נקבל את משוואת הרמיט) והדרך השנייה זה בעזרת אופרטורי העלה והרדה

\begin{symbolize}
נגדיר אופרטורים חסרי יחידות ונסמן אותם עם כובע:
$$X=\sqrt{\frac{\hbar}{m\omega}}\hat{X}\,\,\,;\,\,\,P=\sqrt{\hbar m\omega}\hat{P}\,\,\,;\,\,\,\mathcal{H}=\hbar\omega\hat{\mathcal{H}}$$

\end{symbolize}
ונקבל כעת כי ההמילטוניאן שלנו יהיה:
$$\hat{\mathcal{H}}=\frac{1}{2}\left(\hat{P}^{2}+\hat{X}^{2}\right)$$
זה מזכיר השלמה לריבוע מרוכבת. נראה אם באמת זה מה שמתקיים:
$$\left( \hat{X}-i\hat{P} \right)\left( \hat{X}+i\hat{P} \right)=\hat{X}^2-i\hat{P}\hat{X}+i\hat{X}\hat{P} + \hat{P}^2=\hat{X}^2 +\hat{P}^2 -i\left[ \hat{X},\hat{P} \right]$$
כאשר אומנם לא קיבלנו השלמה לריבוע מושלמת בגלל החוסר חילופיות, אבל ניתן למצוא ביטוי בעזרת הקומוטטור כאשר מתקיים \(\left[ \hat{X},\hat{P} \right]=\frac{i}{2}I\) ולכן נקבל:
$$\hat{\mathcal{H}}=\left( \hat{X}+i\hat{P} \right)\left( \hat{X}-i\hat{P} \right)+\frac{1}{2}I$$

\begin{definition}[אופרטורי העלה והורדה]
$$a=\left( \hat{X}+i\hat{P} \right)=\sqrt{\frac{m\omega}{2\hbar}}\left(x+\frac{i p}{m\omega}\right)\quad a^{\dagger}=\left( \hat{X}-i\hat{P} \right)=\sqrt{\frac{m\omega}{2\hbar}}\left(x-\frac{i p}{m\omega}\right)$$
לעיתים אופרטור הורדה \(a\) נקרא אופרטור השמדה ואופרטור העלה \(a^{\dagger}\) נקרא אופרטור יצירה.

\end{definition}
\begin{proposition}[יחסי חילוף של אופרטורי העלה והורדה]
$$\left[a,a^{\dagger}\right]=\left(\frac{1}{2\hbar}\right)\left(-i[x,p]+i[p,x]\right)=1$$

\end{proposition}
\begin{definition}[אופרטור המספר]
$$N=a^{\dagger}a=\left(\frac{m\omega}{2\hbar}\right)\left(x^{2}+\frac{p^{2}}{m^{2}\omega^{2}}\right)+\left(\frac{i}{2\hbar}\right)[x,p]=\frac{H}{\hbar \omega}-\frac{1}{2}$$

\end{definition}
\begin{remark}
אופרטורי העלה והורדה הם לא הרמיטים בפני עצמם, כאשר אופרטור המספר הוא כמובן יהיה אופרטור הרמיטי כמכפלה של אופרטור עם הצמוד ההרמיטי שלו.

\end{remark}
\begin{proposition}
$$H=\hbar\omega(N+\frac{1}{2})$$

\end{proposition}
\begin{corollary}
קיבלנו כי \(H\) היא פונקציה של \(N\), לכן הקומוטטור שלהם מתאפס וניתן ללכסן אותם במשותף.

\end{corollary}
\begin{proposition}
הערכים העצמיים של האנרגיות יהיו:
$$E_{n}=\left( n+{{\frac{1}{2}}} \right)\hbar\omega$$

\end{proposition}
\begin{proof}
נרצה למצוא את הערכים העצמיים של \(N\ket{n}=n\ket{n}\). כיוון שלכסינים במשותף, ניתן להציב \(N\mapsto H\) וכן \(\hbar \omega \left( N+\frac{1}{2} \right)\mapsto \hbar \omega\left( n+\frac{1}{2} \right)\) ונקבל:
$$H\ket{n} =\hbar \omega\left( n+\frac{1}{2} \right)\ket{n} $$
ונקבל את הע"ע של האנרגיה המובקשים.

\end{proof}
\begin{proposition}
$$Na^{\dagger}\ket{ n} =(n+1)a^{\dagger}\ket{n} \qquad Na\ket{n} =(n-1)a\ket{n} $$

\end{proposition}
\begin{proof}
נשים לב כי:
\begin{gather*}\left[N,a\right]=\left[a^{\dagger}a,a\right]=a^{\dagger}\left[a,a\right]+\left[a^{\dagger},a\right]a=-a  \\\left[ N,a^{\dagger} \right]=\left[ a^{\dagger}a,a^{\dagger} \right]=a^{\dagger}\left[ a,a^{\dagger} \right]+\left[ a^{\dagger},a^{\dagger} \right]a = a^{\dagger}
\end{gather*}
נזכור כי מהגדרת הקומוטטור \(AB=[A,B]+BA\) ולכן נקבל:
\begin{gather*}N a^{\dagger}|n\rangle=\left( \left[ N,a^{\dagger} \right]+a^{\dagger}N \right)|n\rangle=(n+1)a^{\dagger}|n\rangle  \\N a|n\rangle=([N,a]+a N)|n\rangle=(n-1)a|n\rangle
\end{gather*}

\end{proof}
\begin{remark}
נקבל כי אם \(\ket{n}\) פתרון של משוואת שרדינגר, אז גם \(a\ket{n}\) יהיה פתרון עם רמת אנרגיה נמוכה יותר. באופן דומה גם \(a^{\dagger}\ket{n}\) יהיה פתרון עם רמת אנרגיה גובהה יותר.

\end{remark}
\begin{proposition}
$$a|n\rangle=\sqrt{n}|n-1\rangle \qquad  a^{\dagger}|n\rangle=\sqrt{n+1}|n+1\rangle$$

\end{proposition}
נשים לב כי מתקיים \(a|n\rangle=c|n-1\rangle\) כלומר הווקטור העצמי \(\ket{n}\) שמתאים לע"ע \(n\) ולווקטור עצמי \(\ket{n-1}\) שמתאים לערך עצמי \(n-1\) נבדלים בכפל בסקלר. כעת נשים לב כי:
$$\left\langle  n|a^{\dagger}a|n \right\rangle=|c|^{2}\implies \bra{n} N\ket{n} =\bra{n} n\ket{n} =n=\lvert c \rvert ^2$$
כאשר ניקח את \(c\) להיות חיובי ונקבל:
$$a|n\rangle=\sqrt{n}|n-1\rangle \qquad  a^{\dagger}|n\rangle=\sqrt{n+1}|n+1\rangle$$

\begin{proposition}
$$a^k\ket{n} =\sqrt{ \frac{n!}{(n-k)!} }\ket{n-k} $$
כאשר מוגדר על עוד \(n-k>0\) בגלל דרישת חיוביות הנורמה:
$$n=\langle n|N|n\rangle=(\langle n|a^{\dagger}\rangle\cdot(a|n\rangle)\geq0$$

\end{proposition}
\begin{corollary}
המצב יסוד של האוסצילטור ההרמוני יהיה:
$$E_{0}={\frac{1}{2}}\hbar\omega$$

\end{corollary}
\begin{corollary}
$$\left|n\right\rangle=\left[\frac{(a^{\dagger})^{n}}{\sqrt{n!}}\right]\left|0\right\rangle$$

\end{corollary}
\begin{proposition}
לא קיים רמת אנרגיה מקסימלת באסצילטור ההרמוני

\end{proposition}
\begin{proof}
נניח בשלילה שקיים. כלומר קיים קיים רמת אנרגיה שביצוע האופרטור העלה יתן לי שהאהרגיה מתאפסת. לכן:
$$a^{\dagger}\ket{H} =0\implies aa^{\dagger}\ket{H} =0\implies a^{\dagger}a\ket{H} +1\ket{H} =0\implies a^{\dagger}a\ket{H} =-1\ket{H} $$
כאשר השתמשנו ביחס חילוף נקבל כי \(-1\) הוא ערך עצמי - וזה פתרון לא פיזיקלי - לא ייתכן אנרגיה שלילית.

\end{proof}
\begin{proposition}
מיחסי האורטונורמליות מתקיים:
$$\langle n^{\prime}|a|n\rangle=\sqrt{n}\delta_{n^{\prime},n-1},\quad\langle n^{\prime}|a^{\dagger}|n\rangle=\sqrt{n+1}\delta_{n^{\prime},n+1}.$$

\end{proposition}
\begin{proposition}[אופרטורי התנע והמיקום בבסיס האנרגיה]
\begin{gather*}\left\langle  n^{\prime}|x|n \right\rangle=\sqrt{\frac{\hbar}{2m\omega}}\left( \sqrt{n}\delta_{n^{\prime},n-1}+\sqrt{n+1}\delta_{n^{\prime},n+1} \right)  \\\langle n^{\prime}|p|n\rangle=i\sqrt{\frac{m\hbar\omega}{2}}(-\sqrt{n}\delta_{n^{\prime},n-1}+\sqrt{n+1}\delta_{n^{\prime},n+1})
\end{gather*}

\end{proposition}
\begin{proposition}[מצבים עצמיים של האנרגיה בבסיס המקום]
$$\langle x^{\prime}|n\rangle=\left(\frac{1}{\pi^{1/4}\sqrt{2^{n}n!}}\right)\left(\frac{1}{x_{0}^{n+1/2}}\right)\left(x^{\prime}-x_{0}^{2}\frac{d}{d x^{\prime}}\right)^{n}\exp{\left[-\frac{1}{2}\left(\frac{x^{\prime}}{x_{0}}\right)^{2}\right]}$$

\end{proposition}
\begin{proof}
נסתכל על המצב היסוד המוגדר ע"י \(a\ket{0}=\ket{0}\). ע"י ההגדרה של אופרטור ההורדה, נקבל:
$$\langle x^{\prime}|a|0\rangle=\sqrt{\frac{m\omega}{2\hbar}}\langle x^{\prime}|\left(x+\frac{i p}{m\omega}\right)|0\rangle=0.$$
כאשר ניתן לכתוב את זה בתור משוואה דיפרנציאלית:
$$\left(x^{\prime}+x_{0}^{2}\frac{d}{d x^{\prime}}\right)\left\langle  x^{\prime}|0 \right\rangle=0\qquad x_{0}\equiv{\sqrt{\frac{\hbar}{m\omega}}}$$
כאשר באופן כללי הפתרון של המשוואה תהיה:
$$\langle x^{\prime}|0\rangle=\left(\frac{1}{\pi^{1/4}\sqrt{x_{0}}}\right)\exp{\left[-\frac{1}{2}\left(\frac{x^{\prime}}{x_{0}}\right)^{2}\right]}$$
בעזרת אופרטור העלה ניתן לקבל:
\begin{gather*}\left\langle  x^{\prime}|1 \right\rangle=\left\langle  x^{\prime}|a^{\dagger}|0 \right\rangle=\left(\frac{1}{\sqrt{2}x_{0}}\right)\left(x^{\prime}-x_{0}^{2}\frac{d}{d x^{\prime}}\right)\left\langle  x^{\prime}|0 \right\rangle  \\\langle x^{\prime}|2\rangle=\left(\frac{1}{\sqrt{2}}\right)\langle x^{\prime}|(a^{\dagger})^{2}|0\rangle=\left(\frac{1}{\sqrt{2!}}\right)\left(\frac{1}{\sqrt{2}x_{0}}\right)^{2}\left(x^{\prime}-x_{0}^{2}\frac{d}{d x^{\prime}}\right)^{2}\langle x^{\prime}|0\rangle
\end{gather*}
כאשר אם נמשיך את זה באופן כללי נקבל את המבוקש.

\end{proof}
\begin{proposition}[ערך התצפית של המיקום והתנע בריבוע במצב יסוד]
$$\left\langle x^{2}\right\rangle=\frac\hbar{2m\omega}=\frac{x_{0}^{2}}2 \qquad \left\langle p^{2}\right\rangle=\frac{\hbar m\omega}{2}$$

\end{proposition}
\begin{proof}
מתקיים:
$$x^{2}=\left(\frac{\hbar}{2m\omega}\right)\left(a^{2}+a^{\dagger2}+a^{\dagger}a+a a^{\dagger}\right)$$
כאשר אם ניקח את הערך המצופה, כל הערכים פרט לאחרון יתאפס.

\end{proof}
\begin{proposition}[ערך התצפית של האנרגיה הקינטית והפוטנציאלית]
$$\left\langle\frac{p^{2}}{2m}\right\rangle=\frac{\hbar\omega}{4}=\frac{\langle H\rangle}{2}\qquad \left\langle\frac{m\omega^{2}x^{2}}{2}\right\rangle=\frac{\hbar\omega}{4}=\frac{\langle H\rangle}{2}$$

\end{proposition}
\begin{proposition}[אי וודאות במצב יסוד]
$$\left\langle \left( \triangle x \right)^{2} \right\rangle=\langle x^{2}\rangle={\frac{\hbar}{2m\omega}}\qquad  \langle(\triangle p)^{2}\rangle=\langle p^{2}\rangle=\frac{\hbar m\omega}{2}$$
כאשר מתקיים:
$$\langle(\triangle x)^{2}\rangle\langle(\triangle p)^{2}\rangle={\frac{\hbar^{2}}{4}}$$
כאשר נשים לב כי זו האי וודאות המינימלית האפשרית לפי עקרון האי וודאות.

\end{proposition}
\begin{proof}
נזכור כי מתקיים:
$$\begin{array}{c c l}{{x^{2}}}={{\displaystyle\frac{\hbar}{2m\omega}\left( a+a^{\dagger} \right)^{2}}}\end{array}\qquad p^{2}\;\;=\;\;-\frac{m\omega\hbar}2(a-a^{\dagger})^{2}$$
ולכן ניתן לחשב את הערך תצפית בצורה ישירה:
$$\begin{array}{l c l}{{\langle0|(a+a^{\dagger})(a+a^{\dagger})|0\rangle}}&{{=}}&{{\langle0|a a^{\dagger}|0\rangle=1}}\\ {{\langle0|(a-a^{\dagger})(a-a^{\dagger})|0\rangle}}&{{=}}&{{-\langle0|a a^{\dagger}|0\rangle=-1}}\end{array}$$
ונקבל:
$$\langle x^{2}\rangle_{0}\langle p^{2}\rangle_{0}=-\frac{\hbar^{2}}{4}1(-1)=\frac{\hbar^{2}}{4}$$
כאשר כיוון שהערך תצפית של אופרטור המקום והתנע הוא 0, נקבל:
$$\langle(\triangle x)^{2}\rangle_{0}\langle(\triangle p)^{2}\rangle_{0}=\frac{\hbar^{2}}{4}$$

\end{proof}
\begin{proposition}[אי וודאות עבור רמת אנרגיה כללית]
$$\left\langle \left( \triangle x \right)^{2} \right\rangle\left\langle \left( \triangle p \right)^{2} \right\rangle=\left(n+\frac{1}{2}\right)^{2}\hbar^{2}=\frac{\hbar^{2}}{4}(2n+1)^{2}$$

\end{proposition}
\begin{proof}
מתקיים:
\begin{gather*}\left\langle  n|\left( a+a^{\dagger} \right)\left( a+a^{\dagger} \right)|n \right\rangle=\left\langle  n|a a^{\dagger}+a^{\dagger}a|n \right\rangle=\left\langle  n|2a^{\dagger}a+\left[ a,a^{\dagger} \right]|n \right\rangle=2n+1  \\\langle n|(a-a^{\dagger})(a-a^{\dagger})|n\rangle\ \ =-(2n+1)
\end{gather*}
ולכן נקבל באותו אופן כמו עבור מצב היסוד:
$$\langle(\triangle x)^{2}\rangle_{n}\langle(\triangle p)^{2}\rangle_{n}=\frac{\hbar^{2}}{4}(2n+1)^{2}$$

\end{proof}
\subsection{מצבים קוהרנטיים}

ראינו כי המצב היסוד של בסיס המספר הוא יהיה עם אי וודאות מינימלית. נרצה למצוא איפיון לתכונה זו.
\textbf{הגדרה} מצב קוהרנטי
זהו מצב קוונטי \(\ket{\alpha}\) שמוגדר בתור המצב עצמי הייחודי של אופרטור ההשמדה \(a\). נשים לב כי כיוון ש-\(a\) לא הרמיטי, המצב הקוהרנטי ולא בהכרח ממשי, וקיים לו אמפליטודה ופאזה מרוכבת.

\begin{proposition}
המצב הקוהרנטי הוא המצב שבו האי וודואות היא מינימלי.

\end{proposition}
\begin{proof}
אנו יודעים כי מצב קוהרנטי מקיים:
$$a|\alpha\rangle=\alpha|\alpha\rangle \implies \langle\alpha|a^{\dagger}a|\alpha\rangle=|\alpha|^{2}$$
בנוסף מתקיים:
$$\begin{array}{c}{{\langle\alpha|(a+a^{\dagger})|\alpha\rangle=(\alpha+\alpha^{\star})}}\\ {{\langle\alpha|(a-a^{\dagger})|\alpha\rangle=(\alpha-\alpha^{\star})}}\\ {{\langle\alpha|(a+a^{\dagger})(a+a^{\dagger})|\alpha\rangle=(\alpha+\alpha^{\star})^{2}+1}}\\ {{\langle\alpha|(a-a^{\dagger})(a-a^{\dagger})|\alpha\rangle=(\alpha-\alpha^{\star})^{2}-1}}\end{array}$$
כאשר מההגדרה של \(x,p\) בעזרת אופרטורי סולם ולינאריות התוחלת נקבל:
$$\begin{array}{l}{{\langle(\Delta x)^{2}\rangle_{\alpha}=\langle x^{2}\rangle_{\alpha}-\langle x\rangle_{\alpha}^{2}={\frac{\hbar}{2m\omega}}}}\\ {{\langle(\Delta p)^{2}\rangle_{\alpha}=\langle p^{2}\rangle_{\alpha}-\langle p\rangle_{\alpha}^{2}={\frac{\hbar m\omega}{2}}}}\end{array}$$
ולכן נקבל:
$$\langle(\Delta x)^{2}\rangle_{\alpha}\langle(\Delta p)^{2}\rangle_{\alpha}=\frac{\hbar}{4}$$

\end{proof}
\begin{proposition}[מצבים קוהרנטים בבסים המספר]
המצב הקוהרנטי שווה:
$$|\alpha\rangle=e^{-{\frac{1}{2}}|\alpha|^{2}}\sum_{n=0}^{\infty}{\frac{\alpha^{n}}{\sqrt{n!}}}|n\rangle$$

\end{proposition}
\begin{proof}
בבסיס המספר נקבל כי ניתן לכתוב את המצב הקוהרנטי בצורה:
$$|\alpha\rangle=\sum_{n}c_{n}|n\rangle=\sum_{n}|n\rangle\langle n|\alpha\rangle$$
כאשר אנו יודעים כי:
$$|n\rangle=\frac{\left( a^{\dagger} \right)^{n}}{\sqrt{n!}}|0\rangle\implies \langle n|\alpha\rangle={\frac{\alpha^{n}}{\sqrt{n!}}}\langle0|\alpha\rangle$$
ולכן:
$$|\alpha\rangle=\langle0|\alpha\rangle\sum_{n=0}^{\infty}\frac{\alpha^{n}}{\sqrt{n!}}|n\rangle$$
כאשר ניתן למצוא את הקבוע \(\left\langle  0|\alpha  \right\rangle\) בעזרת נרמול:
$$1=\sum_{n}\left\langle \alpha|n \right\rangle\left\langle  n|\alpha \right\rangle=|\left\langle 0|\alpha \right\rangle|^{2}\sum_{m=0}^{\infty}{\frac{|\alpha|^{2m}}{m!}}=|\left\langle 0|\alpha \right\rangle|^{2}e^{|\alpha|^{2}}\implies\langle0|\alpha\rangle=e^{-{\frac{1}{2}}|\alpha|^{2}}$$

\end{proof}
\begin{proposition}
מצב קוהראנטי מקיים:
$$|\alpha\rangle=e^{-{\frac{1}{2}}|\alpha|^{2}+\alpha a^{\dagger}}|0\rangle=e^{\alpha a^{\dagger}-\alpha^{\star}a}|0\rangle$$

\end{proposition}
\begin{proof}
נציב את המצב הקוהרנטי שקיבלנו מהטענה הקודמת בביטוי \(|n\rangle=\frac{(a^{\dagger})^{n}}{\sqrt{n!}}|0\rangle\) ונקבל:
$$\sum_{n=0}^{\infty}\frac{\alpha^{n}}{\sqrt{n!}}|n\rangle=\sum_{n=0}^{\infty}\frac{\alpha^{n}}{n!}(a^{\dagger})^{n}|0\rangle=e^{\alpha a^{\dagger}}|0\rangle$$
כאשר מכאן נקבל:
$$|\alpha\rangle=e^{-\frac12|\alpha|^{2}+\alpha a^{\dagger}}|0\rangle=e^{\alpha a^{\dagger}-\alpha^{\star}a}|0\rangle$$

\end{proof}
\begin{proposition}
כאשר מבצעים אינטגרציה על המרחב כולו מתקיים:
$$\int\frac{d^{2}\alpha}{\pi}\,|\alpha\rangle\langle\alpha|=1$$

\end{proposition}
\begin{proof}
מהביטוי שפיתחנו עבור מצב קוהרנטי מתקיים:
$$\int d^{2}\alpha\,|\alpha\rangle\langle\alpha|=\int d^{2}\alpha\,e^{-|\alpha|^{2}}\sum_{m,n}\frac{(\alpha^{\star})^{n}\alpha^{m}}{\sqrt{n!\,m!}}|m\rangle\langle n|$$
כאשר נבצע אינטגרציה פולרית, כלומר נכתוב \(d^{2}\alpha=d\phi\,d r\,r\) ונקבל שאגף ימין יהיה:
$$\begin{array}{c}{{\displaystyle{\int d^{2}\alpha\,e^{-\left|\alpha\right|^{2}}(\alpha^{\star})^{n}\alpha^{m}=\int_{0}^{\infty}d r\,r e^{-r^{2}}r^{m+n}\int_{0}^{2\pi}d\phi\,e^{i(m-n)\phi}}}}\\ {{{=2\pi\delta_{m,n}\displaystyle{\frac{1}{2}}\int_{0}^{\infty}d r^{2}\,\bigl(r^{2}\bigr)^{m}e^{-r^{2}}=\pi m!\delta_{m,n}}}}\end{array}$$
כאשר נקבל סה"כ:
$$\int d^{2}\alpha\,|\alpha\rangle\langle\alpha|=\pi\sum_{n}|n\rangle\langle n|=\pi \implies \int\frac{d^{2}\alpha}{\pi}\,|\alpha\rangle\langle\alpha|=1$$

\end{proof}
אנחנו אומרים שקבוצת המצבים הקוהרנטיים \(\left\{  \ket{\alpha}  \right\}\) מקיים שלמות יתר, כלומר גם אם נוריד ממנו איברים, עדיין יקיים המרחב שיוצר יהיה שלם.

\begin{proposition}[הצגת המקום של מצב קוהרנטי]
מתקיים:
$$\psi_{\alpha}\left( x^{\prime} \right)=\left( \frac{m\omega}{\pi\hbar} \right)^{1/4}e^{\frac{i}{\hbar}\langle p\rangle_{\alpha}x^{\prime}-\frac{m\omega}{2\hbar}\left( x^{\prime}-\langle x\rangle_{\alpha} \right)^{2}}$$

\end{proposition}
\begin{proof}
כיוון שהתוחלת של מצב קוהרנטי הוא מינימלי מתקיים:
$$\mathrm{Re}\left( \alpha \right)=\left\langle \alpha|\frac{a+a^{\dagger}}{2}|\alpha \right\rangle=\sqrt{\frac{m\omega}{2\hbar}}\left\langle \alpha|x|\alpha \right\rangle\implies \langle x\rangle_{\alpha}={\sqrt{\frac{\hbar}{m\omega}}}{\sqrt{2}}\Re\left( \alpha \right)$$
וגם:
$$\mathrm{Im}\left( \alpha \right)=\left\langle \alpha|\frac{a-a^{\dagger}}{2i}|\alpha \right\rangle=\frac{1}{\sqrt{2m\omega\hbar}}\left\langle \alpha|p|\alpha \right\rangle \implies \langle p\rangle_{\alpha}={\sqrt{m\omega\hbar}}\sqrt{2}\Im\left( \alpha \right)$$
כעת נקבל:
\begin{gather*}a\left|\alpha\right\rangle=\alpha\left|\alpha\right\rangle\implies\left\langle x^{\prime}\right|a\left|\alpha\right\rangle=\alpha\langle x^{\prime}|\alpha\rangle\implies{\frac{1}{\sqrt{2}}}\left\langle x^{\prime}\right|\left({\hat{X}}+i{\hat{P}}\right)|\alpha\rangle=\alpha\langle x^{\prime}|\alpha\rangle \\\langle x^{\prime}|\,\hat{X}\,|\alpha\rangle+i\,\langle x^{\prime}|\,\hat{P}\,|\alpha\rangle=\sqrt{2}\alpha\langle x^{\prime}|\alpha\rangle\implies x^{\prime}\psi_{\alpha}\left(x^{\prime}\right)+\frac{\partial}{\partial x^{\prime}}\psi_{\alpha}\left(x^{\prime}\right)=\sqrt{2}\alpha\psi_{\alpha}\left(x^{\prime}\right) \\\frac{\partial}{\partial x^{\prime}}\psi_{\alpha}\left(x^{\prime}\right)=\left(-x,+\sqrt{2}\alpha\right)\psi_{\alpha}\left(x^{\prime}\right)\implies\psi_{\alpha}\left(x^{\prime}\right)=C e^{-\frac{x^{\prime2}}{2}+\sqrt{2}\alpha x^{\prime}}
\end{gather*}
ונקבל:
$$\psi_{\alpha}\left(x^{\prime}\right)=\pi^{-\frac{1}{4}}e^{-R e^{2}\left(\alpha\right)}e^{-\frac{1}{2}\left(x^{\prime}-\sqrt{2}R e(\alpha)\right)^{2}+i\sqrt{2}I m(\alpha)x^{\prime}}$$
כאשר כעת ניתן להציב את הערכים עבור החלק הממשי והמדומה.

\end{proof}
\begin{proposition}
מצב קוהרנטי נשאר מצב קוהרנטי תחת התפתחות בזמן, כך שאם מצב התחלתי נתון הוא \(\ket{\psi(t=0)}=\ket{\alpha}\) מתקיים:
$$\left|\psi\left(t\right)\right\rangle=e^{-\frac{i\omega t}{2}}\left|\alpha e^{-i\omega t}\right\rangle$$

\end{proposition}
\begin{proof}
התפתחות בזמן נתון ע"י אופרטור קידום בזמן \(U\). נקבל:
$$|\alpha,t\rangle={\mathcal{U}}(t,0)|\alpha(0)\rangle=e^{-{\frac{i}{\hbar}}H t}|\alpha(0)\rangle=e^{-{\frac{i}{\hbar}}H t}e^{-{\frac{1}{2}}|\alpha(0)|^{2}}\sum_{n}{\frac{(\alpha(0))^{n}}{\sqrt{n!}}}|n\rangle$$
כאשר \(\ket{n}\) הם הע"ע של ההמילטוניאן, ולכן:
$$|\alpha,t\rangle=e^{-{\frac{1}{2}}|\alpha(0)|^{2}}\sum_{n}{\frac{(\alpha(0))^{n}}{\sqrt{n!}}}e^{-{\frac{i}{\hbar}}\omega\hbar(n+{\frac{1}{2}})t}{\frac{(a^{\dagger})^{n}}{\sqrt{n!}}}|0\rangle$$
ונקבל:
$$\begin{array}{c}{{\vert\alpha,t\rangle=e^{-\frac{1}{2}\vert\alpha(0)\vert^{2}}e^{-\frac{i}{2}\omega t}\sum_{n}\frac{(\alpha(0)e^{-i\omega t}a^{\dagger})^{n}}{n!}\vert0\rangle=}}\\ {{\exp{\left(-\frac{1}{2}\vert\alpha(0)\vert^{2}-\frac{i}{2}\omega t+\alpha(0)e^{-i\omega t}a^{\dagger}\right)}\vert0\rangle}}\end{array}$$
כאשר אם נשוואה לביטוי \(\ket{\alpha}=e^{ \alpha a^{\dagger}-\alpha^{*}a }\)  נשים לב כי:
$$|\alpha,t\rangle=e^{-{\frac{i}{2}}\omega t}|e^{-i\omega t}\alpha(0)\rangle=|\alpha(t)\rangle$$

\end{proof}
\begin{proposition}
התוחלת של המצבים הקוהרנטיים מקיימים את ההתנהגות הקלאסית של אוסצליטור הרמוני. כלומר:
$$\frac{\mathrm{d} }{\mathrm{d} t} \langle P(t) \rangle =-m\omega^{2}\langle X(t) \rangle \qquad \langle P \rangle =m \langle X \rangle  $$

\end{proposition}
\begin{proof}
מהטענה הקודמת נקבל:
$$\alpha(t)=e^{-i\omega t}\alpha(0)\ \ \Rightarrow\ \ \frac{d}{d t}\alpha(t)=-i\omega\alpha(t)$$
כאשר ניתן להשוואת רכיבים ולקבל:
$$\begin{array}{l}{{\frac{d}{d t}\mathrm{Re}\left( \alpha \right)=\omega\mathrm{Im}\left( \alpha \right)}}\qquad  {{\frac{d}{d t}\mathrm{Im}\left( \alpha \right)=-\omega \mathrm{Re}\left( \alpha \right)}}\end{array}$$
נגדיר ערכי תצפית:
$$x(t)=\left\langle \alpha(t)|x|\alpha,t \right\rangle \qquad p(t)=\langle\alpha(t)|p|\alpha,t\rangle$$
ונקבל:
$$\begin{cases}{\begin{array}{l}{{\frac{d}{d t}}x(t)={\sqrt{{\frac{\hbar}{2m\omega}}}}2{\frac{d}{d t}}\mathrm{Re}\left( \alpha \right)={\sqrt{\frac{\hbar}{2m\omega}}}2\omega \mathrm{Im}\left( \alpha \right)={\frac{p(t)}{m}}}\end{array}} \\\textstyle{\frac{d}{d t}}p(t)={\dot{\imath}}{\sqrt{\frac{m\hbar\omega}{2}}}(-2i){\frac{d}{d t}}\mathrm{Im}\left( \alpha \right)=-{\sqrt{\frac{m\hbar\omega}{2}}}2\omega \mathrm{Re}\left( \alpha \right)=-m\omega^{2}x(t) \\
\end{cases}$$
כאשר ניתן לכתוב את המשוואות האלה באופן שקול ולקבל את המשוואות המוכרות:
$$\begin{cases}p(t)=m{\frac{d}{d t}}x(t)=m v(t) \\{\begin{array}{l}{{\frac{d}{d t}}p(t)=-m\omega^{2}x(t)}\end{array}}
\end{cases}$$

\end{proof}
\chapter{ריבוי דרגות חופש}

\section{מכפלה טנזורית}

נניח כעת שיש לנו שתי דרגות חופש בלתי תלויות \(A,B\). מדידת גודל פיזיקלית של אחת לא אמורה להשפיע על השנייה, ולכן נצפה כי \([O_{A},O_{B}]=0\).

\begin{definition}[מכפלה טנזורית של מרחבי הילברט]
מרחבי הילברט \(\mathcal{H}_{A},\mathcal{H}_{B}\) בלתי תלויות, ניתן להגדר מרחב חדש \(\mathcal{H}=\mathcal{H}_{A}\times \mathcal{H}_{B}\) כאשר מימד המרחב החדש יהיה כפל מרחבי הילברט.

\end{definition}
\begin{definition}[מצב מכפלה]
אם \(\ket{\psi}_{A}\in \mathcal{H}_{A}\) ו-\(\ket{\psi}_{B}\in \mathcal{H}_{B}\) נגדיר את המצב מכפלה שלהם בתור האיבר
$$\ket{\psi} _{A}\otimes \ket{\psi} _{B} \in \mathcal{H}_{A}\times \mathcal{H}_{B}$$
כאשר הפעולה שמחברת ביניהם נקראת המכפלה הטנזורית בין המצבים

\end{definition}
\begin{symbolize}
לעיתים משמיטים את הסימון של מכפלה הטנזורית, ולפעמים אפילו כותבים אותם ביחד באותו קאט. כלומר:
$$\ket{\psi} _{A}\otimes \ket{\psi} _{B}=\ket{\psi_{A}} \ket{\psi} _{B}=\ket{\psi_{A}\psi_{B}} $$

\end{symbolize}
\begin{proposition}
מכפלה טנזורית של מצבים מקיימת בילינאריות. כלומר לכל \(c \in \mathbb{C}\) מתקיים:
\begin{gather*}c\left(\left|\psi\right\rangle_{A}\otimes\left|\phi\right\rangle_{B}\right)=\left(c\left|\psi\right\rangle_{A}\right)\otimes\left|\phi\right\rangle_{B}=\left|\psi\right\rangle_{A}\otimes\left(c\left|\phi\right\rangle_{B}\right) \\\left(\left|\psi\right\rangle_{A}+\left|\psi^{\prime}\right\rangle_{A}\right)\otimes\left|\phi\right\rangle_{B}=\left|\psi\right\rangle_{A}\otimes\left|\phi\right\rangle_{B}+\left|\psi^{\prime}\right\rangle_{A}\otimes\left|\phi\right\rangle_{B} \\|\psi\rangle_{A}\otimes(|\phi\rangle_{B}+|\phi^{\prime}\rangle_{B})=|\psi\rangle_{A}\otimes|\phi\rangle_{B}+|\psi\rangle_{A}\otimes|\phi^{\prime}\rangle_{B}
\end{gather*}

\end{proposition}
\begin{definition}[מכפלה פנימית של מצבי מכפלה]
נגדיר:
$$\left(\left\langle\psi^{\prime}\right|_{A}\otimes\left\langle\phi^{\prime}\right|_{B}\right)\left(\left|\psi\right\rangle_{A}\otimes\left|\phi\right\rangle_{B}\right)=\left\langle\psi^{\prime}|\psi\right\rangle\left\langle\phi^{\prime}|\phi\right\rangle$$

\end{definition}
\begin{proposition}[בסיס אורתונורמלי של מצבי מכפלה]
יהי \(\mathcal{H}_{A}\) מרחב הילברט ממימד \(d_{A}\) עם בסיס אורתונורמלי \(\{\left|e_{i}\right\rangle\}_{i=1}^{d_{A}}\) ו-\(\mathcal{H}_{B}\) מרחב הילברט עם בסיס אורתונורמלי \(\bigl\{\bigl|f_{j}\bigr\rangle\bigr\}_{j=1}^{d_{B}},\). אזי הקבוצה
$$\left\{\left|e_{i}\right\rangle_{A}\otimes\left|f_{j}\right\rangle_{B}\right\}$$
היא בסיס אורתונורמלי של המצב מכפלה \(\mathcal{H}_{A}\times \mathcal{H}_{B}\) ממימד \(d=d_{A}d_{B}\).

\end{proposition}
כאשר האורתונורמליות נובעת מההגדרה של המכפלה הפנימית:
$$\left(\left\langle e_{k}\right|_{A}\otimes\left\langle f_{l}\right|_{B}\right)\left(\left|e_{i}\right\rangle_{A}\otimes\left|f_{j}\right\rangle_{B}\right)=\left\langle e_{k}|e_{i}\right\rangle\left\langle f_{l}|f_{j}\right\rangle=\delta_{k i}\delta_{l j}$$

\begin{corollary}
ניתן לכתוב מצב קוונטי בבסיס של המרחב מכפלה בצורה הבאה:
$$\left|\Psi\right\rangle=\sum_{i=1}^{d_{A}}\sum_{j=1}^{d_{B}}\Psi_{i j}\left|e_{i}\right\rangle_{A}\otimes\left|f_{j}\right\rangle_{B}$$
כאשר אם מנורמל מתקיים:
$$\sum\sum_{i=1\,j=1}^{d_{A}\,\,\,\,d_{B}}|\Psi_{i j}|^{2}=1$$

\end{corollary}
\begin{theorem}[היסוד החמישי]
מרחב הילברט של מערכת קוונטית המוכבת שתי מערכות \(A,B\) בעלות מרחב הילברט \(\mathcal{H}_{A},\mathcal{H}_{B}\) בהתאמה נתון ע"י מרחב המכפלה הטנזורי \(\mathcal{H}=\mathcal{H}_{A}\times \mathcal{H}_{B}\).

\end{theorem}
\begin{definition}[מכפלה טנזורית של אופרטור]
יהי \(O_{A}\) אופרטור שפועל על \(\mathcal{H}_{A}\) ויהי \(O_{B}\) אופרטור שפועל על \(\mathcal{H}_{B}\). נגדיר \(O=O_{A}\otimes O_{B}\) בתור האופרטור שפועל על מרחב המכפלה \(\mathcal{H}=\mathcal{H}_{A}\times \mathcal{H}_{B}\) בצורה הבאה:
$$\left({ O}_{A}\otimes{ O}_{B}\right)\left(\left|\psi\right\rangle_{A}\otimes\left|\phi\right\rangle_{B}\right)=\left({\cal O}_{A}\left|\psi\right\rangle_{A}\right)\otimes\left({\cal O}_{B}\left|\phi\right\rangle_{B}\right)$$
כלומר מצב כללי המוגדר ע"י \(\left|\Psi\right\rangle=\sum_{i=1}^{d_{A}}\sum_{j=1}^{d_{B}}\Psi_{i j}\left|e_{i}\right\rangle_{A}\otimes\left|f_{j}\right\rangle_{B}\) מקיים:
$${ O}\left|\Psi\right\rangle=\sum_{i=1}^{d_{A}}\sum_{j=1}^{d_{B}}\Psi_{i j}\left({ O}_{A}\left|e_{i}\right\rangle_{A}\right)\otimes\left({ O}_{B}\left|f_{j}\right\rangle_{B}\right)$$

\end{definition}
נזכור כי במכניקה אנליטית, כאשר יש לנו מערכת של שתי דרגות בלתי תלויות עם המילטוניאנים \(\mathcal{H}_{A},\mathcal{H}_{B}\) אז נקבל כי ההמילטוניאן של המערכת הכוללת תהיה הסכום שלהם \(\mathcal{H}=\mathcal{H}_{A}+\mathcal{H}_{B}\). בעולם הקוונטי המצב נדרש להתאים את המימדים, ולכן נדרש לתאר עם מכפלה טנזורית:

\begin{proposition}
כאשר יש שתי מרחבי הילברט בלתי תלויים \(\mathcal{H}_{A},\mathcal{H}_{B}\) עם המילטוניאנים \(H_{A},H_{B}\) אזי ההמילטוניאן \(H\) של המצב מכפלה \(\mathcal{H}=\mathcal{H}_{A}\times \mathcal{H}_{B}\) יהיה:
$${ {H}}={ H}_{A}\otimes{\bf1}_{B}+{\bf1}_{A}\otimes{ H}_{B}$$

\end{proposition}
\section{חלקיק בשלוש מימדים}

למעשה בחלק הזה נגדיר מחדש את מה שאנחנו יודעים בצורה תלת מימדית.

\begin{proposition}[יחסי החילוף הקנונים]
כאשר אנחנו מוסיפים מימדים נוספים נקבל אופרטורים \(X_{i},P_{i}\) אשר מקיימים את יחסי החילוף הקנונים:
$$\begin{array}{c}{{[X_{i},P_{j}]=i\hbar\delta_{i j}}}\qquad  {{[X_{i},X_{j}]=0}}\qquad  {{[P_{i},P_{j}]=0}}\end{array}$$

\end{proposition}
\begin{definition}[מיקום בשלוש מימדים]
$$\left|\vec{r}\right\rangle=\left|x,y,z\right\rangle=\left|x\right\rangle\otimes\left|y\right\rangle\otimes\left|z\right\rangle$$

\end{definition}
\begin{proposition}[יחס שלמות בשלוש מימדים]
$$\mathbf{1}=\int d^{3}r\left|\vec{r}\right\rangle\left\langle\vec{r}\right|$$

\end{proposition}
\begin{proposition}[תנע בבסיס המקום]
$$\left\langle\vec{r}\right|P_{x_{i}}\left|\vec{r}^{\prime}\right\rangle=-i\hbar\delta\left(x-x^{\prime}\right)\delta\left(y-y^{\prime}\right)\delta\left(z-z^{\prime}\right){\frac{\partial}{\partial x^{\prime}_{i}}}$$

\end{proposition}
\begin{definition}[תנע בשלוש מימדים]
$$|{\textbf{p}}\rangle=|p_{x},p_{y},p_{z}\rangle=|p_{x}\rangle\otimes|p_{y}\rangle\otimes|p_{z}\rangle$$

\end{definition}
\begin{proposition}[הטלה של המיקום על התנע]
$$\left\langle  \vec{r}|\vec{p}  \right\rangle =\langle x|p_{x}\rangle\,\langle y|p_{y}\rangle\,\langle z|p_{z}\rangle=\frac{1}{\left(2\pi\hbar\right)^{3/2}}e^{i{\vec p}\cdot{\vec r}/\hbar}$$

\end{proposition}
\begin{definition}[פונקציית גל תלת מימדית]
$$\psi\left(\vec{r}\right)=\left\langle \vec{r}|\psi \right\rangle  \qquad \psi\left( \vec{p} \right)=\left\langle  \vec{p}|\psi  \right\rangle $$

\end{definition}
\begin{proposition}
המעבר בין פונקציית הגל בבסיס המקום לפונקציית הגל בבסיס התנע תהיה בעזרת התמרת פורייה תלת מימדי:
\begin{gather*}\widetilde{\psi}\left({\vec p}\right)=\langle {\vec p}|\psi \rangle=\langle {\vec p}|\int d^{3}r\,|{\vec r} \rangle\,\left\langle {\vec r}|\psi \right\rangle =\int d^{3}r\left\langle\vec{p}|\vec{r}\right\rangle\left\langle\vec{r}|\psi\right\rangle={\frac{1}{(2\pi\hbar)^{3/2}}}\int d^{3}r e^{-i\vec{p}\cdot\vec{r}/\hbar}\psi\left(\vec{r}\right)
\end{gather*}

\end{proposition}
\begin{definition}[אופרטור הצפיפות הסתברות והזרם הסתברות בתלת מימד]
$$P=-i\hbar \bar{\nabla} \qquad j\left( \vec{r},t \right)={\frac{\hbar}{2i m}}\left(\overline{{{\psi}}}\nabla\psi-\psi\nabla\overline{{{\psi}}}\right)={\frac{\hbar}{m}}\mathrm{Im}\left(\overline{{{\psi}}}\nabla\psi\right)$$

\end{definition}
\begin{proposition}[משוואת הרציפות]
$${\frac{\partial\rho}{\partial t}}+{\vec{\nabla}}\cdot\vec{j}=0$$

\end{proposition}
\section{אופרטור הצפיפות ושזירה}

\begin{definition}[צבר של מצבים קוונטים]
מערכת שבה יש לנו אוסף של מצבים קוונטים כאשר יש לנו הסתברות(קלאסית) שונה להיות בכל מצב קוונטי.

\end{definition}
\begin{definition}[אופרטור הצפיפות]
עבור צבר של מצבים קוונטים \(\left\{  \ket{\psi_{\mu}}  \right\}\) כאשר ההסתברות להיות במצב קוונטי \(\ket{\psi_{\mu}}\) יהיה \(p_{\mu}\) נגדיר את מטריצת הצפיפות להיות:
$$\rho=\sum_{\mu=1}^{N}p_{\mu}\left|\psi_{\mu}\right\rangle\left\langle\psi_{\mu}\right|$$

\end{definition}
\begin{definition}[עקבה של אופרטור]
נגדיר את העקבה של אופרטור בצורה הבאה:
$${\mathrm{Tr}}\left[Q\right]=\sum_{n}\left\langle n\right|Q\left|n\right\rangle$$
כאשר נשים לב כי כאשר האופרטור הוא אופרטור שנוצר ממכפלה טנזורית העקבה תחזיר אופרטור.

\end{definition}
\begin{proposition}
ערך התצפית של אופרטור כלשהו \(M\) יהיה:
$$\langle M\rangle=\mathrm{Tr}\left[\rho M\right]=\mathrm{Tr}\left[M\rho\right]$$

\end{proposition}
\begin{proof}
$$ \begin{aligned}\left\langle M\right\rangle&=\sum_{\mu=1}^{N}p_{\mu}\left\langle\psi_{\mu}\right|M\left|\psi_{\mu}\right\rangle=\sum_{\mu=1}^{N}p_{\mu}\left\langle\psi_{\mu}\right|\sum_{n}\left|n\right\rangle\left\langle n\right|M\left|\psi_{\mu}\right\rangle\\&=\sum_{n}\sum_{\mu=1}^{N}p_{\mu}\left\langle n\right|\left(M\left|\psi_{\mu}\right\rangle\left\langle\psi_{\mu}\right|\right)\left|n\right\rangle=\sum_{n}\left\langle n\right|M\left(\sum_{\mu=1}^{N}p_{\mu}\left|\psi_{\mu}\right\rangle\left\langle\psi_{\mu}\right|\right)\left|n\right\rangle\\&=\sum_{n}\left\langle n\right|M\rho\left|n\right\rangle=\mathrm{Tr}\left[M\rho\right]=\mathrm{Tr}\left[\rho M\right]\end{aligned}$$

\end{proof}
\begin{definition}[עקבה חלקית ומטריצת צפיפות מצומצמת]
יהי \(\mathcal{H}_{A},\mathcal{H}_{B}\) מרחבי הילברט בלתי תלויים עם מימדים \(d_{A},d_{B}\) ואיברי בסיס \(\{\left|e_{i}\right\rangle_{A}\}_{i=1}^{d_{A}},\left\{\left|f_{j}\right\rangle_{B}\right\}_{j=1}^{d_{B}}\) כך ש-\(\mathcal{H}=\mathcal{H}_{A}\times \mathcal{H}_{B}\). 
$$\rho_{A}=\sum_{j}\left\langle f_{j}\right|_{B}\rho\left|f_{j}\right\rangle_{B}\equiv\mathrm{Tr}_{B}\left[\rho\right]$$

\end{definition}
ההסבר להגדרה זה שאם נרצה להסתכל על אופרטור \(M_{A}\) שפועל רק על \(A\) אזי נקבל:
$$\begin{array}{r c l}{{\left<M_{A}\right>=\mathrm{Tr}\left[M_{A}\rho\right]}}&{{=}}&{{\displaystyle\sum_{i,j}\left<e_{i}\right|_{A}\otimes\left<f_{j}\right|_{B}\rho\left(M_{A}\left|e_{i}\right>_{A}\right)\otimes\left|f_{j}\right>_{B}}}\\ {{}}&{{=}}&{{\displaystyle\sum_{i}\left<e_{i}\right|_{A}\left(\sum_{j}\left<f_{j}\right|_{B}\rho\left|f_{j}\right>_{B}\right)M_{A}\left|e_{i}\right>_{A}}}\end{array}$$

כעת נסתכל על מספר תכונות של מטריצת הצפיפות
\textbf{טענה} הרמיטיות
אופרטור הצפיפות הוא אופרטור הרמיטי. כלומר \(\rho ^{\dagger}=\rho\).

\begin{proposition}[חיוביות]
מטריצת הצפיפות היא חיובית. כלומר לכל מצב קוונטי \(\ket{\psi}\) מתקיים:
$$\forall\left|\psi\right\rangle\;\;\left\langle\psi\right|\rho\left|\psi\right\rangle\geq0$$

\end{proposition}
\begin{proposition}[נרמול]
העקבה של אופרטור הצפיפות הוא 1.

\end{proposition}
\begin{proof}
$$T r\left(\rho\right)=\sum_{i}\left\langle\psi_{i}\right|\rho\left|\psi_{i}\right\rangle=\sum_{i}\sum_{n}p_{n}\langle\psi_{i}|\psi_{n}\rangle\langle\psi_{n}|\psi_{i}\rangle=\sum_{n}\sum_{i}p_{n}\langle\psi_{n}|\psi_{i}\rangle\langle\psi_{i}|\psi_{n}\rangle=\sum_{n}p_{n}=1$$

\end{proof}
\begin{proposition}
העקבה של ריבוע אופרטור הצפיפות הוא קטן או שווה ל-1. 

\end{proposition}
\begin{proof}
$$T r\left(\rho^{2}\right)=\sum_{i,n,m}p_{n}p_{m}\langle\psi_{i}|\psi_{m}\rangle\langle\psi_{m}|\psi_{n}\rangle\langle\psi_{n}|\psi_{i}\rangle=\sum_{n,m}p_{n}p_{m}\delta_{n m}=\sum_{n}p_{n}^{2}\leq1$$

\end{proof}
\begin{definition}[מצב טהור]
כאשר בצבר שלנו כל האובייקטים עם מצב קוונטי יחיד \(\ket{\psi}\). במקרה זה ניתן לכתוב \(\rho=\ket{\psi}\bra{\psi}\).

\end{definition}
\begin{proposition}
מצב טהור הוא אופרטור הטלה. לכן מתקיים \(\rho^{2}=\rho\) וכן מתקיים:
$$\mathrm{Tr}\left( \rho \right)=\mathrm{Tr}\left( \rho^{2} \right)=1$$

\end{proposition}
\begin{proposition}[מצב מעורב]
כאשר יש לנו צבר המכיל מצבים קוונטים שונים. מטריצת הצפיפות תהיה:
$$\rho_{\mathrm{mix}}\,=\,\sum_{\mathrm{i}}\,p_{\mathrm{i}}\,\rho_{\mathrm{i}}^{\mathrm{pure}}\,=\,\sum_{\mathrm{i}}\,p_{\mathrm{i}}\,\left|\,\psi_{\mathrm{i}}\,\right>\left<\,\psi_{\mathrm{i}}\,\right|$$

\end{proposition}
\begin{proposition}
עבור מצב מעורב נקבל כי העקבה של מטריצת הצפיפות בריבוע קטנה מ-1. כלומר \(\mathrm{Tr}\left( \rho^{2} \right)<1\).

\end{proposition}
\begin{proof}
$$\begin{array}{r c l}{{\mathrm{Tr}\,\rho_{\mathrm{mix}}^{2}}}&{{=}}&{{\displaystyle\sum_{\mathrm{n}}\,\langle\,n\,|\sum_{\mathrm{i}}\,\sum_{\mathrm{j}}\,p_{\mathrm{i}}\,p_{\mathrm{j}}\,|\,\psi_{\mathrm{i}}\,\rangle\,\langle\,\psi_{\mathrm{i}}\,|\,\psi_{\mathrm{j}}\,\rangle\,\langle\,\psi_{\mathrm{j}}\,|\,n\,\rangle\,=}}\\ {{}}&{{=}}&{{\displaystyle\sum_{\mathrm{i}}\,\sum_{\mathrm{j}}\,p_{\mathrm{i}}\,p_{\mathrm{j}}\,\langle\,\psi_{\mathrm{i}}\,|\,\psi_{\mathrm{j}}\,\rangle\,\langle\,\psi_{\mathrm{j}}\,|\,\sum_{\mathrm{n}}\,|\,n\,\rangle\,\langle\,n\,|\,\psi_{\mathrm{i}}\,\rangle\,=}}\\ {{}}&{{=}}&{{\displaystyle\sum_{\mathrm{i}}\,\sum_{\mathrm{j}}\,p_{\mathrm{i}}\,p_{\mathrm{j}}\,|\,\langle\,\psi_{\mathrm{i}}\,|\,\psi_{\mathrm{j}}\,\rangle\,|^{2}\,=}}\\ {{}}&{{=}}&{{\displaystyle\sum_{\mathrm{i}}\,p_{\mathrm{i}}^{2}\,<\,\sum_{\mathrm{i}}\,p_{\mathrm{i}}\,=\,1\;.}}\end{array}$$

\end{proof}
כאשר נשים לב כי הערך של \(\mathrm{Tr}\left( \rho^{2} \right)\) זה מדד טוב לכמה מעורב מצב. ככל שזה יותר קטן - המצב יותר מעורב - ככל שזה יותר קרוב ל-1 - המצב יותר טהור. אופרטור מעורב מקסימלית של מעורב מקסימלית של מערכת ממימד \(d\) יקיים \(\mathrm{Tr}\left( \rho^{2} \right)=\frac{1}{d}>0\).

\subsection{שזירה}

\begin{definition}[מצב מכפלה]
יהיו \(\mathcal{H}_{A},\mathcal{H}_{B}\) מרחבים הילברט. מצב \(\ket{\Psi}\in \mathcal{H}_{A}\times \mathcal{H}_{B}\) נקרא מצב מכפלה אם קיימים \(\ket{\psi}_{A}\in A,\ket{\psi}_{B}\in B\) כך שמתקיים:
$$\ket{\Psi} =\ket{\psi} _{A}\otimes \ket{\psi} _{B}$$

\end{definition}
\begin{definition}[מצב שזור]
מצב שאינו מצב מכפלה. כלומר עבור \(\mathcal{H}_{A},\mathcal{H}_{B}\) מרחבי הילברט. מצב \(\ket{\Psi}\in \mathcal{H}_{A}\times \mathcal{H}_{B}\) נקרא מצב שזור אם לא קיימים \(\ket{\psi}_{A}\in A,\ket{\psi}_{B}\in B\) כך שמתקיים:
$$\ket{\Psi} =\ket{\psi} _{A}\otimes \ket{\psi} _{B}$$

\end{definition}
כלומר במצב מכפלה נקבל כי שתי המרחבי הילברט לא תלויים אחד בשני לחלוטין, כאשר אנחנו לא במצב מכפלה, תהיה איזשהי תלות בין מרחבי ההילברט. כלומר אם אנחנו יודעים משהו על מרחב הילברט אחד אז זה נותן לנו מידע על מרחב הילברט אחר.

\begin{definition}[שזירה]
שזירה בין שתי מערכות אומרת לנו עד כמה אי ידיעת מצבה של מערכת אחת מונעת מאיתנו לדעת את מצבה של השנייה.

\end{definition}
\begin{definition}[אנטרופיית שאנון]
דרך לכמת את כמות האי ידיעה במערכת. אם הסתבריות נתונות ע"י \(\{ p_{i} \}\) אנטרופיית שאנון מוגדרת:
$$S=-\sum_{i}p_{i}\ln p_{i}$$

\end{definition}
\begin{proposition}
המערכת עם שזירה מקסימלית תהיה עם אנטרופיה מקסימלית, וזה קורה כאשר ההתפלגות אחידה. מצב זה יהיה מהצורה:
$$\left|\psi\right\rangle={\frac{1}{\sqrt{D}}}{\sum_{\mu=1}^{D}{\left|\mu\right\rangle}_{A}\otimes{\left|\mu\right\rangle}_{B}}$$
כאשר מטריצת הצפיפות המצומצמת תהיה:
$$\rho_{A}=\mathrm{Tr}_{B}\left[\rho\right]=\sum_{\mu=1}^{D}\left\langle\mu\right|_{B}\rho\left|\mu\right\rangle_{B}=\frac{1}{D}{\bf1}$$

\end{proposition}
\chapter{סימטרייה}

\section{מבוא לתורת ההצגות של חבורות}

\begin{definition}[חבורה]
קבוצה \(A\) עם פעולה \(\cdot:A\times A\to A\)  אשר מקיימת:

  \begin{enumerate}
    \item קיום איבר זהות \(1\) אשר מקיים: 
$$\forall a \in A\quad a \cdot 1 = 1 \cdot a = 1$$


    \item לכל איבר \(a \in A\) קיים \(a^{-1} \in A\) כך ש: 
$$a \cdot a^{-1} = a ^{-1} \cdot a = 1$$


    \item אסוצייטיביות. 


  \end{enumerate}
\end{definition}
\begin{remark}
לרוב נשמיט את הסימן של הכפל.

\end{remark}
\begin{definition}[חבורה אבלית]
חבורה אשר קומוטטיבית.

\end{definition}
\begin{definition}[הצגה של חוברה]
יהי \(G\) חבורה, ו-\(V\) מרחב ווקטורי. הומומורפיזם \(\rho:G\to GL(V)\) כאשר \(GL(V)\) מייצג את החבורת אוטומורפיזמים של \(V\)(כלומר כל העתקות הלינארית ההפיכות על \(V\)) נקראת הצגה של החבורה \(G\).

\end{definition}
\begin{example}[ההצגה הטריוויאלית]
נדרש רק שיהיה הומומורפיזם, ולכן לא נדרש חח"ע ועל. לכן ההצגה הטריוויאלית \(e\mapsto \mathbb{1}\) עם \(V=\mathbb{C}^{n}\) יהיה של כל חבורה למרות שרק ממפה איבר יחיד. וכן נשים לב כי ניתן להגדיר הצגה זו עם מטריצה יחידה מכל גודל.

\end{example}
\begin{example}[הצגה של תמורה]
ניתן את החבורה \(S_{n}\) על ידי מטריצת תמורה על \(V=\mathbb{F} _2\)(או כל שדה שמכיל את 0 ו-1). למשל ניתן להציג את התמורה \((1,2,3)\) על ידי:
$$T((1,2,3))=\begin{pmatrix}0&0&1\\ 1&0&0\\ 0&1&0\end{pmatrix}$$

\end{example}
\begin{proposition}
כיוון שכל חבורה מגודל \(n\) היא תת חבורה של \(S_{n}\)(משפט קיילי) וכל תמורה ספית ניתן לייצג כמטריצה, לכל חבורה יש הצגה של החבורה בעזרת תמורות.

\end{proposition}
\begin{remark}
הצגה זו נקראת לפעמים ההצגה הרגילה.

\end{remark}
דבר זה אבל מעלה 2 שאלות:

\begin{enumerate}
  \item האם ההצגה היא יחידה. 


  \item האם ואיך ניתן לפשט את ההצגה? (מטריצה מגודל \(n!\) היא תהיה גדולה עם 0 או 1 בערכים שלה מרגיש לא יעיל). 


\end{enumerate}
\begin{example}
עבור כל תמורה למשל ניתן ליצור הצגה על ידי הסימן של התמורה. זה יהיה הומומורפיזם של "מטריצה" מגודל 1 שיהיה או 1 או \(-1\).

\end{example}
\begin{definition}[מימד של הצגה]
עבור הצגה \(R:G\to GL_{n}\left( \mathbb{F}  \right)\) המימד של ההצגה יהיה \(n\). כלומר זה יהיה גודל המטריצות שבתמונה.

\end{definition}
\begin{example}
עבור חבורת הממשיים עם פעולת החיבור ניתן להגדיר הצגה על ידי אקספוננט כיוון שמתקיים:
$$e^{ u+v }=e^{ u }e^{ v }$$
ולכן המומומורפיזם, כאשר זה יהיה הצגה ממימד 1. ניתן להגדיר גם על ידי הפעולה הבאה:
$$D(u)=\left(\begin{array}{l l}{{1}}&{{0}}\\ {{u}}&{{1}}\end{array}\right)$$
כיוון שאכן מתקיים:
$$D(u+v)=D(u)D(v),$$
וזה יהיה הצגה מגודל 2.

\end{example}
\begin{proposition}
אם \(g\mapsto R(g)\) הצגה אזי גם \(g\mapsto UR(g)U^{-1}\) יהיה הצגה. הצגות כאלה אשר נבדלות ביחס דמיון לעיתים נקראים הצגות שקולות.

\end{proposition}
\begin{proposition}
מכל שתי הצגות \(\left( \rho_{1},V_{1} \right),\left( \rho_{2},V_{2} \right)\) ניתן להרכיב הצגה שלישית על ידי לקחת המרחב המתקבל מהסכום הישר \(V_{1} \oplus V_{2}\) של שתי המרחבים ביחד עם ההומומורפיזם המוגדר על ידי:
$$\rho(av_{1}+bv_{2})=a\rho_{1}(v_{1})+b\rho_{2}(v_{2})$$
כאשר \(v_{1} \in V_{1}\) ו-\(v_{2} \in V_{2}\)(ניתן להציג כל ווקטור בצורה הזאת מתכונות של הסכום הישר)

\end{proposition}
\begin{corollary}
אם \(R_{1}\) ו-\(R_{2}\) הצגות אז גם הפונקציה:
$$g \mapsto \begin{pmatrix}R_{1}(g) & 0 \\0 & R_{2}(g)
\end{pmatrix}$$
תהיה הצגה.

\end{corollary}
\begin{reminder}[תת מרחב אינווריאנטי]
תת מרחב \(U\leq V\) נקרא אינווריאנטי תחת העתקה \(T:V\to V\) אם לכל \(v \in U\) מתקיים \(Tv \in U\). כלומר זהו מרחב שההעתקה משאירה אותה במרחב. כאשר נזכור כי המרחב כולו ומרחב האפס הם המרחבים שתמיד \(T\)-אינווריאנטיים.

\end{reminder}
\begin{definition}[תת יצוג]
יהי \(R\) ייצוג על מרחב ווקטורי \(V\). תת מרחב \(U\subset V\) אשר אינווריאנטי תחת כל אופרטור \(R(g)\) כאשר \(g\in G\) יקרא תת ייצוג.

\end{definition}
\begin{corollary}
תת ייצוג מייצג תת חבורה במטריצה המקורית.

\end{corollary}
\begin{definition}[הצגה בלתי פריקה]
זוהי תהיה הצגה שהתת ייצוג היחיד שלה יהיה התת יצוג הטריוויאלי והיצוג עצמו.

\end{definition}
\begin{corollary}
עבור הצגה בלתי פריקה אין בסיס שעבורה כל \(g\in G\) בחבורה תהיה מטריצת בלוקים. כלומר אין לה תת מרחב אינווריאנטי.

\end{corollary}
\begin{proposition}
עבור כל חבורה סופית כל הצגה בלתי פריקה תהיה סופית.

\end{proposition}
\begin{remark}
בפועל ניתן להכליל טענה זו גם עבור חבורות אינסופיות אך נדרש להגדיר חבורה טופולוגית ולדרוש כי הקבוצה תהיה קומפקטית תחת הטופולוגיה.

\end{remark}
\begin{definition}[הצגה אוניטרית]
הצגה \(T\) שעבורה לכל \(g \in G\) נקבל \(T(g)\) אוניטרית.

\end{definition}
\begin{proposition}
כל הצגה מימד סופי של חבורה סופית תהיה שקולה להצגה אונטרית, כלומר ניתן להביא אותה לצורה אוניטרית על ידי טרנספורמציית דמיון.

\end{proposition}
\begin{remark}
עבור חבורה אינסופית לא בהכרח נקבל כי קיים הצגה אוניטרית עבורה. לדוגמא עבור החבורה \(R\) עם חיבור עם ההצגה:
$$T(u)={\left(\begin{array}{l l}{1}&{0}\\ {u}&{1}\end{array}\right)}\,\qquad u\in\mathbb{R}$$
נקבל:
$$\begin{array}{c}{{T^{\dagger}(u)T(u)=\left(\begin{array}{c c}{{1}}&{{u}}\\ {{0}}&{{1}}\end{array}\right)\left(\begin{array}{c c}{{1}}&{{0}}\\ {{u}}&{{1}}\end{array}\right)}} {{=\left(\begin{array}{c c}{{1+u^{2}}}&{{u}}\\ {{u}}&{{1}}\end{array}\right)}}\end{array}$$
ולמעשה נדרש תנאי נוסף כדי שיהיה נכון עבור חבורה אינסופית, שעבורה נדרש חבורה טופולוגית.

\end{remark}
\begin{proposition}
כל הצגה מימד סופי של חבורה טופולוגית אינסופית קומפקטית תהיה שקולה להצגה אונטרית, כלומר ניתן להביא אותה לצורה אוניטרית על ידי טרנספורמציית דמיון.

\end{proposition}
\begin{proposition}
כל הצגה אוניטרית תהיה סכום ישר של הצגות בלתי פריקות.

\end{proposition}
\begin{corollary}
כל הצגה של חבורה סופית או חבורה טופולוגית קומפקטית תהיה ניתנת לפירוק לסכום ישר של הצגות בלתי פריקות.

\end{corollary}
\begin{lemma}[הראשונה של שור]
יהי \(T:G\to GL(V)\) ו-\(S:G\to GL(U)\) הצגות בלתי פריקות. אם מטריצה \(A\) היא כך ש:
$$A T(g)=S(g)A\qquad\forall g\in G,$$
אז או ש-\(A=0\) או ש-\(A\) מטריצה הפיכה ולכן שתי ההצגות הם שקולות.

\end{lemma}
\begin{proof}
נניח \(\dim T=n\) ו-\(\dim S = m\). נניח כי קיים תת מרחב \(W\subset V\) כך ש-\(Aw=0\) אם"ם \(w \in W\). יהי \(P\) מטריצת ההטלה על התת מרחב \(W\subset V\) אזי \(AP=0\). מזה נקבל:
$$A T(g)P=S(g)A P=0.$$
אבל אז נקבל כי \(T(g)P \in W\) כלומר \(T(f)w \in W\) לכל \(w \in W\). זה אומר ש-\(W\) הוא תת מרחב \(T\) אינווריאנטי, ולכן נקבל כי או ש-\(W=0\) ואז \(A=0\) או ש-\(W=V\) אז זה אומר כי לכל \(0 \neq v \in V\) נקבל \(Av \neq 0\).
ניתן לבצע את אותו התהליך על \(W' \in U\) כך ש-\(uA=0\) אם"ם \(u \in W'\). נקבל כי או ש-\(A=0\) או ש-\(uA\) לא אפס לכל \(0\neq u \in U\).
נותרנו עם שתי מקרים. או ש-\(A=0\) או ש-\(Av\) ו-\(uA\) הם אף פעם לא אפס עבור \(u\in U, v\in V\) אשר שונים מ-0. כעת נראה שהמקרה השני שקול לכך שמטריצה ריבועית הפיכה. אם מספר השורות קטן ממספר העמודות, קיים \(v \in V\) כך ש-\(Av=0\), בסתירה. באופן דומה, אם מספר העמודות קטן ממספר השורות, קיים \(u \in U\) כך ש-\(uA=0\), בסתירה. לכן \(A\) ריבועית. והיא תהיה הפיכה כיוון שהגרעין שלה לא טריוויאלי.

\end{proof}
\begin{lemma}[השנייה של שור]
יהי \(T:G\to GL(V)\) הצגה בפלתי פריקה של חבורה סופית \(G\). אם מטריצה \(A\) מתחלפת עם \(T(g)\) לכל \(g \in G\), כלומר אם:
$$A T(g)=T(g)A\qquad\forall g\in G,$$
נקבל \(A=\lambda I\) עבור \(\lambda \in \mathbb{C}\) כלשהו. כלומר \(A\) יהיה כפולה של המטריצת יחידה

\end{lemma}
\begin{proof}
הלמה השנייה היא מסקנה ישירה של הראשונה. ל-\(A\) חייב להיות לפחות ערך עצמי אחד. לכן \(A-\lambda I\) תהיה עם דטרמיננטה 1 ולכן לא תהיה הפיכה. כליוון ש-\(AT(g)=T(g)A\) נקבל:
$$(A-\lambda I)T(g)=T(g)(A-\lambda I)$$

\end{proof}
\begin{remark}
אומנם ניסחנו והוכחנו עבור חבורות סופיות. הטענות(הלמה הראשונה והשנייה של שור) יהיו נכונות גם עבור חבורות טופולוגיות קומפקטויות.

\end{remark}
\begin{corollary}
בתנאי הלמה השנייה של שור(חבורה סופית או חבורה טופולוגית קומפקטית) כל ההצגות הבלתי פריקות של חבורה \underline{אבלית} יהיו ממימד 1.

\end{corollary}
\begin{proof}
תהי \(T:G\to GL(V)\) הצגה בלתי פריקה של חבורה אבלית סופית \(G\). אזי לכל \(a \in G\) נקבל כי \(T(a)\) מתחלף עם כל \(T(g)\). לכן לפי הלמה השנייה של שור נקבל \(T(a)=\lambda I\) עבור \(\lambda \in \mathbb{C}\) וזה נכון לכל \(a \in G\) ולכן \(T(a)\) היא כפולה סקלארית של היחידה. כעת הדרך היחידה ש-\(T\) תהיה אי פריקה היא הם היא חד מימדית.

\end{proof}
\begin{theorem}[האורתוגונאליות]
יהי \(T:G\to GL(V)\) ו-\(S:G\to GL(V)\) הצגות בלתי פריקות של חבורה סופית \(G\). נסמן ב-\(T(g)_{ij}\) וב-\(S(g)_{ij}\) את אלמנטי המטריצה של המטריצות המתאימות. אזי:

  \begin{enumerate}
    \item אם \(S\) ו-\(G\) הם לא שקולות: 
$$\sum_{g\in G}T^{\dagger}(g)_{i j}S(g)_{k l}=0.$$


    \item אם \(S\) ו-\(G\) שקולות נסמן את מימד ההצגה ב-\(d\) ונקבל: 
$$\sum_{g\in G}T^{\dagger}(g)_{i j}T(g)_{k l}=\frac{|G|}{d}\delta_{i l}\delta_{j k},$$


  \end{enumerate}
\end{theorem}
\begin{corollary}
לחבורה סופית תהיה מספר סופי של הצגות בלתי פריקות לא שקולות כיוון שכל אחת צריכה להיות אורתוגונאלית לכל האחרות.

\end{corollary}
\begin{summary}
  \begin{itemize}
    \item חבורה היא קבוצה עם פעולה על הקבוצה אשר אסוצייטיבית, סגורה להופכי ומכילה יחידה.
    \item עבור חבורה \(\left( G,\cdot \right)\) צירוף של מרחב ווקטורי ביחד עם הומומורפיזם \(\rho:G\to GL(V)\) נקרא הצגה של \(G\).
    \item שתי הצגות אשר מייצגות את אותה ההצגה בבסיסים שונים נקראות הצגות שקולות, ונבדלות ביחס דמיון.
    \item מכל שתי הצגות ניתן להרכיב הצגה שלישית על ידי הסכום הישר של המרחבים. כאשר כל הצגה תהווה תת מרחב אינוורינטי.
    \item הצגה ללא תתי מרחבים אינווריאנטים לא טריוויאלים נקראים הצגות בלתי פריקות.
    \item מהלמה הראשונה של שור אם יש שתי הצגות בלתי פריקות כך שמטריצה \(A\) מתחלפת עם שתינן לכל איבר בחבורה, נקבל כי אם ההצגות לא שקולות \(A=0\).
    \item מהלמה השנייה של שור, עבור הצגה בלתי פריקה אם יש מטריצה שמתחלפת איתה לכל איבר בחבורה המטריצה הזאת תהיה פרופורציונית ליחידה.
    \item משפט האורתוגונאליות נותן סוג של יחס אורתוגונאליות עבור ההצגות הבלתי פריקות.
  \end{itemize}
\end{summary}
\section{סימטריות}

\begin{definition}[סימטריה]
אוטומורפיזם(טרנספורמציה המשמרת מבנה של המרחב ווקטורי) על המרחב אשר אינו משנה את חוקי הפיזיקה.

\end{definition}
\begin{proposition}[סימטרייה במערכת קוונטית]
אופורטור מעבר קורדינטות \(U\) אשר משמר הסתברות. כלומר אם \(\ket{\psi'}=U\ket{\psi}\) אזי אם:
$$\left\lvert  \braket{ \phi | \psi }   \right\rvert ^{2}= \left\lvert  \braket{ \phi' | \psi' }   \right\rvert ^{2}$$
אז הטרנפורמציה \(U\) נקראת טרנספורמציה סימטרית.

\end{proposition}
\begin{theorem}[ווגנר]
אם טרנספורמציה \(U\) היא טרנספורמציה סימטרית אז מתקיים אחד התנאים הבאים:

  \begin{enumerate}
    \item האופרטור \(U\) יהיה אופרטור לינארי ואוניטרי: 
$$\ket{\psi} \mapsto \ket{\psi'} =U\ket{\psi} \qquad U^{\dagger} U = Id $$


    \item האופרטור \(U\) יהיה אופרטור אנטי אוניטרי ואנטי לינארי: 
$$\ket{\psi} \mapsto \ket{\psi'} =U\ket{\psi} \qquad \braket{ U\phi | U\psi } =\braket{ \psi | \phi } $$


  \end{enumerate}
\end{theorem}
\begin{definition}[סימטרייה רציפה]
סימטריה הניתנת להצגה על ידי אופרטור אוניטרי וקשורה ליחידה על ידי משתנה רציף. כלומר פונקציה רציפה \(U_{\varepsilon}\) אשר מקיימת \(U_{\varepsilon=0}=\mathrm{Id}\).

\end{definition}
\begin{example}
אופרטור הזזה יהיה סימטריה רציפה. אי ביצוע הזזה שקול ללא לעשות כלום ולכן \(U_{\varepsilon=0}=\mathrm{Id}\) כאשר ביצוע הזזה עצמה זה פעולה רציפה - אפשר להזיז בכל מספר ממשי.

\end{example}
\begin{proposition}
ניתן לקרב סימטריה רציפה בעזרת טור טיילור. כך שמתקיים:
$$U_{\varepsilon}= \mathbb{ 1} +i\varepsilon T+O\left( \varepsilon^{2} \right)$$

\end{proposition}
\begin{corollary}
מתקיים תחת הקירוב מסדר ראשון:
$$T=T^{\dagger}$$
כלומר \(T\) יהיה אופרטור הרמיטי.

\end{corollary}
\begin{proof}
נכתוב:
$$U_{\varepsilon}\approx \mathbb{1} +i\varepsilon T$$
כאשר מדרישת האוניטריות נקבל:
$$U_{\varepsilon}^{\dagger}U_{\varepsilon}=\mathbb{1} \implies \left( \mathbb{1} -i\varepsilon T^{\dagger} \right)\left( \mathbb{1} +i\varepsilon T \right)=\mathbb{1} -i\varepsilon T^{\dagger}+i\varepsilon T=\mathbb{1} -i\varepsilon\left( T^{\dagger}-T \right)\overset{!}{=}  \mathbb{1} $$
ולכן נקבל \(T^{\dagger}=T\).

\end{proof}
\begin{proposition}
אופרטור של סימטריה רציפה יוצר חבורה תחת פעולת ההרכב/הכפלת מטריצות. כלומר קיים חבורה \(G\) כך שאם \(U_{\varepsilon_{1}},U_{\varepsilon_{2}}\in G\) אזי
$$U_{\varepsilon_{1}}\circ U_{\varepsilon_{2}}\in G$$

\end{proposition}
\begin{corollary}
ניתן לבצע כמות גבוהה של טרנספורמציות רציפות קטנות כך שהקרוב לסדר ראשון יהיה יעיל. כלומר עבור \(\varepsilon> 0\) לאו דווקא קטן, ניתן לקחת \(N>0\) מספיק גדול כך ש-\(\frac{\varepsilon}{N}\) יהיה קטן מספיק, כך שניתן לכתוב:
$$U_{\varepsilon} = \left( \mathbb{1} +i \frac{\varepsilon}{N}T \right)^{N}\xrightarrow[N\to \infty]{} U_{\varepsilon}=e^{ i\varepsilon T }$$

\end{corollary}
\begin{definition}[יוצר של טרנספורמציה רציפה]
אופרטור \(T\) כך ש:
$$U_{\varepsilon}=e^{ i\varepsilon T }$$
כלומר זהו אופרטור אשר יוצר את החבורה של הסימטריה.

\end{definition}
\begin{proposition}[מציאת יוצר של סימטריה]
נניח כי \(U_{\varepsilon}=e^{ \theta T }\) אזי:
$$\frac{d}{d\varepsilon}U_{\varepsilon}=Te^{ \varepsilon T }=TU_{\varepsilon}\implies T=\left( \frac{\mathrm{d} }{\mathrm{d} \varepsilon} U_{\varepsilon} \right)U_{\varepsilon}^{-1}$$

\end{proposition}
\begin{proposition}[דרך נוספת למציאת יוצר של חבורה]
נקרב גם את \(U_{\varepsilon}\) לסדר ראשון ונקבל:
$$T= \frac{U_{\varepsilon}-\mathbb{1} }{i\varepsilon}$$

\end{proposition}
\begin{example}[מציאת יוצרים ל-\(SO(4)\)]
נרשום את המטריצות סיבוב ונקרב אותם לסדר ראשון:
\begin{gather*}R_{xy}=R_{12}=\left[\begin{matrix}\cos{\left(\theta \right)} & - \sin{\left(\theta \right)} & 0 & 0\\\sin{\left(\theta \right)} & \cos{\left(\theta \right)} & 0 & 0\\0 & 0 & 1 & 0\\0 & 0 & 0 & 1\end{matrix}\right] \approx \left[\begin{matrix}1 & - \theta & 0 & 0\\\theta & 1 & 0 & 0\\0 & 0 & 1 & 0\\0 & 0 & 0 & 1\end{matrix}\right]\\R_{xz}=R_{13}=\left[\begin{matrix}\cos{\left(\theta \right)} & 0 & - \sin{\left(\theta \right)} & 0\\0 & 1 & 0 & 0\\\sin{\left(\theta \right)} & 0 & \cos{\left(\theta \right)} & 0\\0 & 0 & 0 & 1\end{matrix}\right] \approx \left[\begin{matrix}1 & 0 & - \theta & 0\\0 & 1 & 0 & 0\\\theta & 0 & 1 & 0\\0 & 0 & 0 & 1\end{matrix}\right]\\R_{xw}=R_{14}=\left[\begin{matrix}\cos{\left(\theta \right)} & 0 & 0 & - \sin{\left(\theta \right)}\\0 & 1 & 0 & 0\\0 & 0 & 1 & 0\\\sin{\left(\theta \right)} & 0 & 0 & \cos{\left(\theta \right)}\end{matrix}\right] \approx \left[\begin{matrix}1 & 0 & 0 & - \theta\\0 & 1 & 0 & 0\\0 & 0 & 1 & 0\\\theta & 0 & 0 & 1\end{matrix}\right] \\R_{yz}=R_{23}=\left[\begin{matrix}1 & 0 & 0 & 0\\0 & \cos{\left(\theta \right)} & - \sin{\left(\theta \right)} & 0\\0 & \sin{\left(\theta \right)} & \cos{\left(\theta \right)} & 0\\0 & 0 & 0 & 1\end{matrix}\right]\approx \left[\begin{matrix}1 & 0 & 0 & 0\\0 & 1 & - \theta & 0\\0 & \theta & 1 & 0\\0 & 0 & 0 & 1\end{matrix}\right]\\R_{yw}=R_{24}=\left[\begin{matrix}1 & 0 & 0 & 0\\0 & \cos{\left(\theta \right)} & 0 & - \sin{\left(\theta \right)}\\0 & 0 & 1 & 0\\0 & \sin{\left(\theta \right)} & 0 & \cos{\left(\theta \right)}\end{matrix}\right] \approx \left[\begin{matrix}1 & 0 & 0 & 0\\0 & 1 & 0 & - \theta\\0 & 0 & 1 & 0\\0 & \theta & 0 & 1\end{matrix}\right]\\R_{zw}=R_{34}=\left[\begin{matrix}1 & 0 & 0 & 0\\0 & 1 & 0 & 0\\0 & 0 & \cos{\left(\theta \right)} & - \sin{\left(\theta \right)}\\0 & 0 & \sin{\left(\theta \right)} & \cos{\left(\theta \right)}\end{matrix}\right] \approx \left[\begin{matrix}1 & 0 & 0 & 0\\0 & 1 & 0 & 0\\0 & 0 & 1 & - \theta\\0 & 0 & \theta & 1\end{matrix}\right]\\
\end{gather*}
כעת ניתן למצוא את היוצר על ידי \(T=\frac{R-Id}{i\theta}\) וקירוב מסדר ראשון ולקבל:
\begin{gather*}T_{12}= \frac{R_{12} - Id}{i\theta } = \left[\begin{matrix}0 & i & 0 & 0\\- i & 0 & 0 & 0\\0 & 0 & 0 & 0\\0 & 0 & 0 & 0\end{matrix}\right] \qquad T_{13}= \frac{R_{13} - Id}{i\theta } = \left[\begin{matrix}0 & 0 & i & 0\\0 & 0 & 0 & 0\\- i & 0 & 0 & 0\\0 & 0 & 0 & 0\end{matrix}\right]\\T_{14}= \frac{R_{14} - Id}{i\theta } = \left[\begin{matrix}0 & 0 & 0 & i\\0 & 0 & 0 & 0\\0 & 0 & 0 & 0\\- i & 0 & 0 & 0\end{matrix}\right] \qquad T_{23}= \frac{R_{23} - Id}{i\theta } = \left[\begin{matrix}0 & 0 & 0 & 0\\0 & 0 & i & 0\\0 & - i & 0 & 0\\0 & 0 & 0 & 0\end{matrix}\right]\\T_{24}= \frac{R_{24} - Id}{i\theta} = \left[\begin{matrix}0 & 0 & 0 & 0\\0 & 0 & 0 & i\\0 & 0 & 0 & 0\\0 & - i & 0 & 0\end{matrix}\right]\qquad T_{34}= \frac{R_{34} - Id}{i\theta} = \left[\begin{matrix}0 & 0 & 0 & 0\\0 & 0 & 0 & 0\\0 & 0 & 0 & i\\0 & 0 & - i & 0\end{matrix}\right]\\
\end{gather*}

\end{example}
\begin{proposition}[בייקר קמפבל האסדורף]
$$e^{X}\circ e^ {Y}=e^{X+Y+{\frac{1}{2}}[X,Y]+{\frac{1}{12}}[X,[X,Y]]-{\frac{1}{12}}[Y,[X,Y]]+\ldots}$$

\end{proposition}
\begin{proposition}[הרחבה למספר משתנים]
ניתן לתאר אופרטור אוניטרי רציף במקרה הרב מימדי על ידי מספר משתנים רציפים - \(U_{\varepsilon_{1},\varepsilon_{2},\dots, \varepsilon_{n}}\). במקרה זה נקבל כי הקרוב בעזרת טור חזקות תהיה מהצורה:
$$U_{\varepsilon_{1},\varepsilon_{2},\dots, \varepsilon_{n}}= \mathbb{1} + i\sum_{a=1}^{n} \varepsilon_{a}T_{a}+O\left( \varepsilon_{a} \right)$$
וכעת יהיה מספר יוצרים, וניתן יהיה לכתוב:
$$U_{\varepsilon_{1},\varepsilon_{2},\dots, \varepsilon_{n}} = e^{  i\sum_{a=1}^{n} \varepsilon_{a}T_{a}}=\prod _{a=1}^{n}e^{ i\varepsilon_{a}T_{a} }$$
כאשר נשים לב כי גם כעת היוצר יהיה אופרטור הרמיטי.

\end{proposition}
\begin{proposition}
יוצרים של סימטריה יוצרים מרחב ווקטורי.

\end{proposition}
\begin{summary}
  \begin{itemize}
    \item בקוונטים, טרנספורמציית סימטריה היא אופרטור \(U\) המקיים \(\left\lvert  \braket{ \phi | \psi }   \right\rvert ^{2}= \left\lvert  \braket{ U\phi | U\psi }   \right\rvert ^{2}\).
    \item לפי משפט ווגנר כל טרנספורמציית סימטריה תהיה או אוניטרית ולינארית או אנטי אוניטרית ואנטי לינארית, כאשר אנטי לינאריות רלוונטית רק במקרה של היפוך בזמן.
    \item סימטריה רציפה היא טרנספורמציה סימטרית \(U\left( \varepsilon \right)\) רציפה כך ש-\(U\left( \varepsilon \right)=\mathbb{1}\).
    \item יוצר של סימטריה רציפה יהיה אופרטור \(T\) אשר מקיים \(U\left( \varepsilon \right)=e^{ i\varepsilon T }\).
  \end{itemize}
\end{summary}
\section{וריאציה}

\begin{definition}[ווריאציה]
$$\delta S=S\left( x+\varepsilon \eta \right)-S(x)$$

\end{definition}
\begin{proposition}
הפעולה אינווריאנטיות להוספת נגזרת שלמה לפי זמן.

\end{proposition}
\begin{proof}
\begin{gather*}S=\int \mathcal{L}\left( q,\dot{q}, t \right) \;\mathrm{d} t  \\\delta S = \delta\int \mathcal{L}\left( q,\dot{q} ,t\right) \;\mathrm{d} t= \int \delta L\left( q,\dot{q},t \right)  \;\mathrm{d} t 
\end{gather} $$
כאשר ניתן היה להכניס את הווריאציה כיוון שהאינטגרל הוא לפי זמן והווריאציה לפי מרחב.
כעת נניח כי \(\delta L = \frac{\mathrm{d} G}{\mathrm{d} t}\) כלומר היא איזושהי נגזרת שלמה לפי זמן. ונקבל:
$$\delta S = \int \frac{\mathrm{d} G}{\mathrm{d} t}  \;\mathrm{d} t \equiv \text{const} $$

\end{proof}
\begin{reminder}[משוואת אויילר לגרנג']
הונקציונאל הסציונארי מקיים:
$$\frac{\partial L}{\partial q_{i}}-\frac{\mathrm{d}}{\mathrm{d}t}\left( \frac{\partial L}{\partial \dot{q}_{i}}  \right)=0$$

\end{reminder}
\begin{theorem}[נטר]
קיום של סימטרייה רציפה גורר חוק שימור.

\end{theorem}
\begin{proof}
נניח כי נתון הפעולה:
$$S\left( q_{i},\dot{q}_{i} \right)=\int \mathcal{L}\left( q_{i},\dot{q}_{i},t \right)\;\mathrm{d}t$$
אם נבצע שינוי אינפיניטסימלי:
$$q_{i}\to q_{i}+\varepsilon F_{i}\left( \left\{  q_{i} \},\{ q_{j} \},t  \right\} \right)$$
כאשר \(F\) יהיה היוצר של טרנספורמציית סימטריה. כעת נגדיר \(G\) כך ש-\(\delta \mathcal{L}=\frac{\partial G}{\partial t}\) ומתקיים \(G|_{t=t_{1}}=G|_{t=t_{2}}=0\) ונדרוש:
$$0=\delta S = \int \delta \mathcal{L}\;\mathrm{d}t=\int\frac{\partial G}{\partial t} \;\mathrm{d}t = G|_{q(t_{1})}^{q(t_{2})}$$
כאשר:
$$\frac{\partial G}{\partial t} = \delta \mathcal{L}\left( q_{i},\dot{q}_{i},t \right)=\sum_{i}\frac{\partial \mathcal{L}}{\partial q_{i}} \delta q_{i}+\frac{\partial L}{\partial \dot{q}_{i}} \delta \dot{q}_{i}=\varepsilon \sum_{i}\frac{\partial \mathcal{L}}{\partial q_{i}} F_{i}+\frac{\partial \mathcal{L}}{\partial \dot{q}_{i}} \dot{F}_{i}$$
ולכן נקבל:
$$\frac{\partial G}{\partial t} =\sum_{i}\frac{\partial }{\partial t} \left( \frac{\partial \mathcal{L} }{\partial \dot{q}_{i}}  \right) F_{i}+\frac{\partial \mathcal{L}}{\partial \dot{q}_{i}}\dot{F_{i}} $$
ולכן:
$$\frac{\mathrm{d} }{\mathrm{d} t} \underbrace{ \left( \sum_{i} \frac{\partial L}{\partial \dot{q}_{i}} F_{i}-\tilde{G} \right) }_{ Q }=0$$
כאשר \(Q\) יהיה הגודל השמור.

\end{proof}
\section{היפוך במקום}

\begin{definition}[היפוך במקום חד מימדי]
אופרטור \(\Pi\) אשר מקיים:
$$\Pi \psi(x)=\psi(-x)$$

\end{definition}
\begin{proposition}
אופרטור ההיפוך במקום הוא הרמיטי ואוניטרי, וכן הערכים העצמיים שלו הם \(\pm 1\).

\end{proposition}
\begin{proof}
נובע ישירות מכך שההופכי של עצמו(היפוך פעמיים יחזיר לאותו מצב). מזה גם נובע שהערכים העצמיים שלו הם \(\pm 1\) כיוון שמתקיים \(\Pi^{2}=\mathbb{1}\) ולכן \(\Pi^{2}-\mathbb{1}=0\) נקבל כי הפולינום המינימלי מתחלק ב-\(x^{2}-1\) וזה חייב גם להיות הפולינום המינימלי כי לא ייתכן כי מסדר ראשון. ולכן הערכים העצמיים הם השורשים \(\pm 1\).

\end{proof}
\begin{definition}
מצבים עצמיים של אופרטור ההיפוך המרחבי עם ערך עצמי \(1\) נקראים מצבים זוגיים או סימטריים ומצבים עצמיים עם ערך עצמי \(-1\) נקראים אי זוגיים או אנטי סימטריים.

\end{definition}
\begin{proposition}
ההיפוך המרחבי של אופרטור \(Q\) יהיה \(\Pi ^{\dagger}Q \Pi\). 

\end{proposition}
\begin{proposition}
אופרטור המיקום והתנע הם אי זוגיים תחת היפוך מרחבי. כלומר מקיימים:
$$\Pi ^{\dagger} X\Pi=-X\qquad \Pi ^{\dagger}P \Pi=-P$$

\end{proposition}
\begin{corollary}
עבור אופרטור כללי אשר מורכב מאופרטורי המיקום והתנע \(Q(X,P)\) מתקיים:
$$\Pi ^{\dagger}Q(X,P)\Pi = Q(-X,-P)$$

\end{corollary}
\begin{proof}
נובע מכך שכל גורם חיבורי שמופיע \(X\) או \(P\) מתהפך מהטענה בקודמת וכל גורם כפלי ניתן להכניס יחידה באמצא(לדוגמא \(\Pi XP \Pi ^{\dagger}=\Pi X\Pi ^{\dagger}\Pi P \Pi ^{\dagger}=(-X)(-P)\)).

\end{proof}
\begin{definition}[מערכת סימטרית להיפוך מרחבי]
מערכת אשר אינווריאנטית תחת היפוך מרחבי, כלומר \(\Pi ^{\dagger}H\Pi=H\), או לחלופין מקיימת \(\left[ H,\Pi \right]=0\).

\end{definition}
\begin{remark}
עבור חלקיק חופשי בפוטנציאל חד מימדי היפוך מרחבי אומר כי הפוטנציאל היא פונקציה זוגית \(V(x)=V(-x)\).

\end{remark}
\begin{proposition}
אם למערכת יש סימטריה להיפוך מרחבי נקבל כי לכסינה במשותף עם האופרטור ההיפוך המרחבי ולכן:
$$\hat{\Pi}\,\psi_{n}(x)=\psi_{n}(-x)=\pm\,\psi_{n}(x)\,,$$

\end{proposition}
\begin{proposition}
אם למערכת יש היפוך מרחבי אז לפי משפט אנרנפסט:
$$\frac{d}{d t}\left<\Pi\right>=\frac{i}{\hbar}\left<\left[\hat{H},\hat{\Pi}\right]\right>=0$$
ולכן הזוגיות נשמרת לאורך זמן(אם הפונקציה זוגית תשאר זוגית ואם אי זוגית אז תשאר אי זוגית).

\end{proposition}
\begin{example}[אוסצילטור הרמוני]
נסתכל עם אוסצילטור הרמוני חד מימדי עם המילטוניאן מהצורה:
$$H=\frac{1}{2m}p^{2}+\frac{1}{2}m\omega^{2}x^{2}$$
נבנה את המרחב הילברט בעזרת אופרטורי העלה \(a^{\dagger}=X-\frac{i}{m\omega}P\) והורדה \(a \sim \left( X+\frac{i}{m\omega}P \right)\). נזכור כי \(a\ket{0}=0\) ו-\(\ket{n}\sim\left( a^{\dagger} \right)^{n}\ket{0}\).
ההמילטוניאן אינווריאנטי לזוגיות(אכן \(\left[ H,\Pi \right]=0\)), ולכן הזוגיות מוגדרת היטב. כיוון שאופרטור היצירה לינארי ב-\(X\) ו-\(P\) נקבל:
$$\Pi a^{\dagger}\Pi=-a^{\dagger}$$
כלומר הזוגיות של המצב \(\ket{n+1}\) יהיה:
$$\Pi \ket{n+1} =\Pi a^{\dagger}\ket{n} =-a^{\dagger}\Pi \ket{n} \implies \eta_{n+1}=-\eta_{n}$$
כאשר \(\eta_{n}\) מסמן מסמן את הזוגיות של הרמה המעורערת ה-\(n\). נשים לב כי הזוגיות מתחלפת בכל רמה מעורערת. לכן מספיק לדעת את הזוגיות של רמה נתונה. עבור מצב היסוד נקבל:
$$\psi_{0}(x)=\langle x|0\rangle\sim\exp\left(-{\frac{m\omega x^{2}}{2\hbar}}\right)$$
כאשר נשים לב כי לא משתנה תחת שיקוף מרחבי, ולכן \(\eta_{0}=1\) ומזה נקבל כי \(\eta_{n}=(-1)^{n}\).

\end{example}
\begin{definition}[היפוך מרחבי תלת מימדי]
כעת מוגדר עבור פונקציית גל ווקטורית:
$$\hat{\Pi}\psi({\bf r})=\psi(-{\bf r})$$

\end{definition}
\begin{corollary}
בדיוק כמו במקרה החד מימדי מתקיים:
$$\hat{\Pi}^{\dagger}\,\hat{\mathbf{r}}\,\hat{\Pi}=-\hat{\mathbf{r}}\qquad \hat{\Pi}^{\dagger}\,\hat{\bf p}\,\hat{\Pi}=-\hat{\bf p}$$
כאשר כל שילוב שלהם מקיים:
$$\hat{\Pi}^{\dagger}\hat{Q}(\hat{\bf r},\hat{\bf p})\;\hat{\Pi}=\hat{Q}(-\hat{\bf r},-\hat{\bf p})$$

\end{corollary}
\begin{proposition}
תנע זוויתי נשמר תחת היפוך מרחבי. כלומר:
$$\Pi ^{\dagger}L\Pi=L$$

\end{proposition}
\begin{proof}
$$\hat{\mathbf{L}}^{\prime}=\hat{\Pi}^{\dagger}\,\hat{\mathbf{L}}\,\hat{\Pi}=\hat{\mathbf{r}}^{\prime}\times\hat{\mathbf{p}}^{\prime}=\left(-\hat{\mathbf{r}}\right)\times\left(-\hat{\mathbf{p}}\right)=\hat{\mathbf{r}}\times\hat{\mathbf{p}}=\hat{\mathbf{L}}$$

\end{proof}
\begin{proposition}[היפוך מרחבי בקורדינטות כדוריות]
בקורדינטות כדוריות היפוך במרחב מקיים:
$$\left( r,\theta,\phi \right)\mapsto\left( r,\,\pi-\theta,\,\phi+\pi \right)$$

\end{proposition}
\begin{example}
עבור הרמיוניות ספריות בקורדינטות כדוריות מתקיים:
$$Y_{\ell}^{m}(\pi-\theta,\phi+\pi)=(-1)^{\ell}\,Y_{\ell}^{m}(\theta,\phi)$$
זאת כיוון שניתן לכתוב:
$$Y_{\ell}^{m}(\theta,\phi)=N\,P_{\ell}^{m}(\cos\theta)\,e^{i m\phi}$$
אם מבצעים את אופרטור ההיפוך המרחבי נקבל \(\theta \mapsto \pi-\theta\) ו-\(\phi \mapsto \phi+\pi\). הפולינומי לג'נדר המוכללים מקיימים:
$$P_{\ell}^{m}(-\cos\theta)=(-1)^{\ell+m}P_{\ell}^{m}(\cos\theta)$$
כאשר החלק האזיומטלי מקיים:
$$e^{i m(\phi+\pi)}=e^{i m\phi}e^{i m\pi}=e^{i m\phi}(-1)^{m}$$
כלומר סה"כ:
$$(-1)^{\ell+m}\cdot(-1)^{m}=(-1)^{\ell}$$

\end{example}
\begin{remark}
טנזורים ספריים כללים(כאשר הרמיוניות ספריות הם מקרה פרטי שלהם) מדרגה \(k\) מתווסף פאקטור של \((-1)^{k}\).

\end{remark}
\begin{definition}[זוגיות תחת היפוך מרחבי]
אופרטור נקרא זוגי תחת היפוך מרחבי אם לא משתנה. 

\end{definition}
\begin{remark}
קוראים לאובייקט אשר עובר סיבוב וגם היפוך במרחב כמו \(\mathbf{x}\) ווקטור, וכן לאובייקט אשר עובר סיבוב כמו \(\mathbf{x}\) אך היפוך במרחב בצורה שונה פסודו ווקטור(כלומר אובייקט כמו תנע זוויתי אשר מקיים \(\Pi V\Pi=+V\)).

\end{remark}
\begin{example}
השדה החשמלי הוא ווקטור כאשר השדה המגנטי הוא פסאודו ווקטור כיוון שתחת היפוך מרחבי מתקיים:
$$E\mapsto -E\qquad B\mapsto +B$$

\end{example}
\begin{summary}
  \begin{itemize}
    \item אופרטור \(\Pi\) נקרא היפוך במרחב והופך את המיקום של החלקיקים. בפרט \(\Pi \psi(x)=\psi(-x)\).
    \item לכל אופרטור \(Q(X,P)\) מתקיים:
$$Q(X,P)\Pi^{\dagger}Q(X,P)\Pi=Q(-X,-P)$$
    \item תנע זוויתי לא משתנה תחת היפוך מרחבי.
    \item בקורדינטות ספריות היפוך מרחבי נתון על ידי \(\left( r,\theta,\phi \right)\mapsto\left( r,\,\pi-\theta,\,\phi+\pi \right)\).
  \end{itemize}
\end{summary}
\section{היפוך בזמן}

\subsection{היפוך בזמן הפיזיקה קלאסית}

\begin{definition}[מערכת סימטרית להיפוך בזמן]
מערכת פיזיקלית אשר מקיימת שעבור כל פתרון \(\mathbf{x}(t)\) גם \(\mathbf{x}(-t)\) יהיה פתרון.

\end{definition}
\begin{example}
מערכת מהצורה \(m\ddot{\mathbf{x}}=-\bar{\nabla}V(\mathbf{x})\) תהיה אינווריאנטית תחת היפוך בזמן, כאשר מערכת מרוסנת מהצורה:
$$m{\ddot{\mathbf{x}}}=-\nabla V(\mathbf{x})-\gamma{\dot{\mathbf{x}}}$$
לא תהיה אינווריאנטית להיפוך בזמן.

\end{example}
\begin{remark}
נרצה כי כל מערכת המילטונית תהיה אינווריאנטית להיפוך זמן. לכן למשל אם נסתכל על חליקיק עם מטען \(q\) הנע בשדה מגנטי משוואות התנועה נתונות על ידי:
$$m{\ddot{\mathbf{x}}}=q{\dot{\mathbf{x}}}\times\mathbf{B}$$
כאשר זו מערכת המילטוניאנית אשר אינה אינווריאנטית להיפוך בזמן כאשר נניח כי המטען \(q\) והשדה המגנטי \(\vec{B}\) לא משתנים עם היפוך בזמן. לכן המוסכמה היא שאנחנו מחליפים את הסימן של \(\vec{B}\) תחת היפוך בזמן כאשר המטען לא משתנה. נדרש היה לדרוש כי המטען משנה סימן תחת היפוך בזמן והשדה המגנטי לא והפיזיקה הייתה זהה.

\end{remark}
\begin{corollary}
תחת היפוך בזמן נקבל:
$$t\mapsto-t\quad \mathbf{x}(t)\mapsto \mathbf{x}(-t)\quad \mathbf{p}(t)\mapsto-\mathbf{p}(-t)$$
כאשר השדות החשמליים והמגנטיים יקיימו:
$$\mathbf{E}\mapsto\mathbf{E}\qquad \mathbf{B}\mapsto-\mathbf{B}$$

\end{corollary}
\subsection{אופרטור אנטי אוניטרי}

\begin{definition}[אופרטור אנטי לינארי]
עבור \(\alpha,\beta \in \mathbb{C}\) אופרטור \(B\) נקרא אנטי לינארי אם מקיים:
$$B(\alpha|\psi_{1}\rangle+\beta|\psi_{2}\rangle)=\alpha^{\star}B|\psi_{1}\rangle+\beta^{\star}B|\psi_{2}\rangle$$

\end{definition}
\begin{corollary}
אופרטור אנטי אוניטרי לא מתחלף עם סקלארים מרוכבים ולמעשה מקיים:
$$B\alpha=\alpha^{\star}B$$

\end{corollary}
\begin{definition}[ברא עבור אופרטורים אנטי לינארים]
$$(\langle\phi|B\rangle|\psi\rangle=[\langle\phi|(B|\psi\rangle)]^{\star}$$

\end{definition}
\begin{definition}[אופרטור אנטי אוניטרי]
אופרטור אנטי לינארי אשר בנוסף מקיים:
$$B^{\dagger}B=B B^{\dagger}=\mathbb{1} $$

\end{definition}
\subsection{היפוך בזמן בפיזיקה קוונטית}

\begin{reminder}[משוואת שרדינגר התלויה בזמן]
$$i\hbar{\frac{\partial\psi}{\partial t}}=H\psi$$

\end{reminder}
\begin{remark}
נשים לב כי אם ההמילטוניאן סימטרי להיפוך בזמן אז היינו מצפים כי מצב יתקדם בזמן באופן המתאים להיפוך בזמן. אבל משוואת שרדינגר היא מסדר ראשון בזמן, ולכן אם \(\psi(t)\) פתרון אז \(\psi(-t)\) באופן כללי הוא לא פתרון נוסף.

\end{remark}
\begin{proposition}
אם \(\psi(t)\) פתרון אז \(\psi^{*}(-t)\) יהיה פתרון.

\end{proposition}
\begin{definition}[אופרטור ההיפוך בזמן]
אופרטור \(\mathcal{T}\) אנטי לינארי ואנטי אוניטרי אשר מייצג היפוך בזמן.

\end{definition}
\begin{proposition}
נדרוש שאופרטור ההיפוך בזמן מקיים:
$$\mathcal{T} i H=-i H\mathcal{T}$$

\end{proposition}
\begin{proof}
אנו יודעים כי מצב מתקדם בזמן על ידי:
$$\left|\psi(t)\right\rangle=e^{-i H t/\hbar}\left|\psi(0)\right\rangle$$
כאשר אם ניקח את המצב \(\mathcal{T}\ket{\psi(0)}\) ונקדם אותו קדימה בזמן ונדרוש שיהיה שווה להיפוך בזמני של במצב \(\ket{\psi(0)}\) אשר מקודם אחורה בזמן נקבל:
$$\mathcal{T}\,e^{+i H t/\hbar}\,|\psi(0)\rangle=e^{-i H t/\hbar}\,\mathcal{T}\,|\psi(0)\rangle$$
נפתח לזמן אינפוניטסימלי ונקבל:
$$\mathcal{T} i H=-i H\mathcal{T}$$

\end{proof}
\begin{remark}
הדבר הטבעי שנראה שאפשר לעשות כרגע זה לצמצם את ה-\(i\), אבל זה לא נכון! אופרטור אנטי אונטרי אינו מתחלף עם סקלארים מרוכבים.

\end{remark}
\begin{corollary}
אם נשתמש באנטי לינאריות של ההיפוך בזמן נקבל:
$$[\mathcal{T} ,H]=0$$

\end{corollary}
\begin{remark}
עבור אופרטורים לינארים זה אומר שיש לנו גודל שמור. כיוון שזה לא אופרטור לינארי זה לא המקרה כאן, אבל עדיין מאפשר לנו להגיד דברים מעניינים.

\end{remark}
\begin{proposition}[חלקיק חסר ספין]
עבור חלקיק חסר ספין עם המילטוניאן מהצורה \(H={\frac{\mathbf{p}^{2}}{2m}}+V(\mathbf{x})\) נקבל כי מתקיים:
$$T\psi(\mathbf{x})=\psi^*(\mathbf{x})\qquad\mathcal{T} \ket{p} =\ket{-p} $$

\end{proposition}
\begin{proof}
נזכור כי אופרטור נקבע ביחידות לפי לאן ששולח את איברי הבסיס שלו. נסתכל על בסיס המיקום ונדרוש:
$$\mathcal{T}|\mathbf{x}\rangle=|\mathbf{x}\rangle\implies \mathcal{T} \alpha \ket{x} =\alpha^{*}\ket{x} $$
ניתן כעת לפתח מצב כללי על ידי:
$$|\psi\rangle=\int d^{3}x~|{\bf x}\rangle\langle{\bf x}|\psi\rangle=\int d^{3}x~\psi({\bf x})|{\bf x}\rangle$$
וכעת נקבל כי היפוך בזמן נתון על ידי:
$$\mathcal{T}|\psi\rangle=\int d^{3}x\,\mathcal{T}\psi({\bf x})|{\bf x}\rangle=\int d^{3}x\,\,\psi^{\star}({\bf x})\mathcal{T}|{\bf x}\rangle=\int d^{3}x\,\,\psi^{\star}({\bf x})|{\bf x}\rangle$$
ניתן באופן דומה לכתוב את הבסיס התנע ולקבל:
$$|{\bf p}\rangle=\int d^{3}x~e^{i{\bf p}\cdot{\bf x}}|{\bf x}\rangle$$
כאשר אם נפעיל היפוך בזמן נקבל:
$$\mathcal{T}|{\bf p}\rangle=\int d^{3}x~\mathcal{T} e^{i{\bf p}\cdot{\bf x}}|{\bf x}\rangle=\int d^{3}x~e^{-i{\bf p}\cdot{\bf x}}|{\bf x}\rangle=|-{\bf p}\rangle$$
ניתן לעשות זאת גם על האופרטורים ולקבל:
$$\hat{\bf x}=\int d^{3}\!x\ {\bf x}|{\bf x}\rangle\langle{\bf x}|\implies\mathcal{T}\hat{\bf x}\mathcal{T}=\int d^{3}\!x\ {\bf x}\mathcal{T}|{\bf x}\rangle\langle{\bf x}|\mathcal{T}=\hat{\bf x}$$
וגם:
$$\hat{\bf p}=\int d^{3}p\ {\bf p}|{\bf p}\rangle\langle{\bf p}|\implies\mathcal{T}\hat{\bf p}\mathcal{T}=\int d^{3}p\ {\bf p}\mathcal{T}|{\bf p}\rangle\langle{\bf p}|\mathcal{T}=-\hat{\bf p}$$

\end{proof}
\begin{corollary}
עבור חלקיק חסר ספין עם פוטנציאל מרכזי(\(V(\mathbf{x})=V(\lvert \mathbf{x} \rvert)\)) מתקיים בנוסף:
$$\mathcal{T} L\mathcal{T} ^{-1} =-L$$
כאשר עבור פונקציית הגל נקבל \(\psi_{n l m}^{\star}({\bf x})=(-1)^{m}\psi_{n l,-m}({\bf x})\)

\end{corollary}
\begin{proof}
במקרה של פוטנציאל מרכזי מטעמי סימטריה נקבל כי \(\mathbf{L}\) נשמר, וכן כיוון ש-\(\mathbf{L}=\mathbf{x}\times \mathbf{p}\) נקבל כי מקבל סימן מינוס תחת היפוך בזמן. כעת ניתן לכתוב את הפונקציית הגל בבסיס על ידי:
$$\psi_{n l m}({\bf x})=R_{n l}(r)Y_{l m}(\mathcal{T},\phi)$$
כאשר ניתן לקחת את החלק הרדיאלי כך שממשית, ולכתוב את ההרמוניות הספריות על ידי:
$$Y_{l m}(\mathcal{T},\phi)\,=\,e^{i m\phi}P_{l}^{m}(\cos\mathcal{T})$$
מההגדרה של הפונקציות נקבל:
$$\psi_{n l m}^{\star}({\bf x})=(-1)^{m}\psi_{n l,-m}({\bf x})$$

\end{proof}
\begin{proposition}[היפוך בזמן של ספין]
ספין עובר היפוך זמני על ידי:
$$\mathcal{T}{\bf S}\mathcal{T}^{-1}=-{\bf S}$$

\end{proposition}
\begin{proof}
נראה עבור חלקיק עם ספין חצי. נזכור כי \(\mathbf{S}=\frac{\hbar}{2}\boldsymbol\sigma\). המרחב הילברט הוא מרחב דו מימדי ולכן ניקח עבורו את הבסיס של ווקטורים עצמיים של \(S_{z}\):
$$\left|+\right>={\binom{1}{0}}\quad{\mathrm{and}}\quad\left|-\right>={\binom{0}{1}}$$
כך שמתקיים \(S_{z}\ket{\pm}=\pm \frac{\hbar}{2}\ket{\pm}\). הפעולה של היפוך בזמן נתונה על ידי:
$$\mathcal{T}|+\rangle=i|-\rangle\,\,\,\,\,\,,\,\,\,\,\,\mathcal{T}|-\rangle=-i|+\rangle$$
נשים לב כי כאשר נפעיל פעמיים על הבסיס נקבל:
\begin{gather*}\mathcal{T}^{2}|+\rangle=\mathcal{T}(i|-\rangle)=-i\mathcal{T}|-\rangle=-|+\rangle \\\mathcal{T}^{2}|-\rangle=\mathcal{T}(-i|+\rangle)=i\mathcal{T}|+\rangle=-|-\rangle
\end{gather*}
ולכן \(\mathcal{T}^{2}=-\mathbb{1}\). כמו כן נשים לב כי מתקיים:
\begin{gather*}\bra{+} \mathcal{T} =i\bra{-} \qquad\bra{-} \mathcal{T} =-i\bra{+}  \\\mathcal{T}^{\dagger}|+\rangle=-i|-\rangle\quad,\quad\mathcal{T}^{\dagger}|-\rangle=i|+\rangle
\end{gather*}
ונשים לב כי \(\mathcal{T}^{\dagger}=-\mathcal{T}\). נפעיל כעת על כל רכיב של אופרטור הספין:
\begin{gather*}S_{x}=|+\rangle\langle-|+|-\rangle\langle+|\implies\mathcal{T} S_{x}\mathcal{T}^{\dagger}=-S_{x} \\S_{z}=|+\rangle\langle+|-|-\rangle\langle-|\implies\mathcal{T} S_{z}\mathcal{T}^{\dagger}=-S_{z} \\S_{y}\,=\,-i|+\rangle\langle-|+i|-\rangle\langle+|\implies\mathcal{T} S_{y}\mathcal{T}^{\dagger}=-S_{y}
\end{gather*}

\end{proof}
\begin{proposition}
עבור ספין שלם מתקיים \(\mathcal{T}^{2}=\mathbb{1}\) כאשר עבור ספין חצי שלם מתקיים \(\mathcal{T}^{2}=-\mathbb{1}\).
\textbf{הגדרה} היפוך בזמן
אופרטור אנטי לינארי ואנטי ואוניטרי המוגדר על ידי \(t\mapsto -t\).

\end{proposition}
\begin{corollary}
כיוון שאנטי לינארי מתקיים:
$$T\alpha=\alpha^{*}T$$

\end{corollary}
\begin{proposition}
תחת היפוך בזמן \(T:t\to -t\), גורמים שאינם תלוים בזמן לא משתנים, למשל מיקום לא ישתנה - \(TXT^{-1}=X\). לעומת זאת גורמים אשר תלוים בזמן מסדר ראשון(למשל שווים לנגזרת לפי זמן כמו מהירות, תנע, תנע זוויתי וספין) יהפכו סימן, כלומר:
$$TXT^{-1}=X \qquad TPT^{-1}=-P \qquad TLT^{-1} = -L \qquad TST^{-1}=-S$$

\end{proposition}
\begin{proposition}
אופרטור הקידום בזמן \(U(t)\) עובר תחת היפוך בזמן ל-\(U^{\dagger}(t)\). כלומר \(T U(t)T^{-1}=U^{\dagger}(t)\)

\end{proposition}
\begin{example}
נתון כי לרמת אנרגיה \(E_{n}\) של ההמילטוניאן \(H\) יש מצב עצמי יחיד \(\ket{\psi_{n}}\). נראה כי אם \(\Pi\) מתחלף עם ההמילטוניאן אז \(\ket{\psi_{n}}\) הוא מצב עצמי של \(\Pi\) עם ערך עצמי \(\pm 1\).
נתחיל מכך שמתקיים \(H\ket{\psi_{n}}=E_{n}\ket{\psi_{n}}\) כך שמתקיים:
$$\Pi H\ket{\psi_{n}} =H\left( P\ket{\psi_{n}}  \right)=E_{n}\left( \Pi\ket{\psi_{n}}  \right)$$
כאשר כיוון ש-\(\ket{\psi_{n}}\) לא מנוון אין מצב עם אנרגיה \(E_{n}\) אבל הביטוי אומר ש-\(\Pi \ket{\psi_{n}}\) הוא מצב עם \(E_{n}\) ולכן הם זהים עד כדי פאזה ומתקיים:
$$\Pi \ket{\psi_{n}} =\alpha \ket{\psi_{n}} \implies \Pi^{2}\ket{\psi_{n}}=\alpha^{2}\ket{\psi_{n}}\implies \alpha=\pm 1$$

\end{example}
\begin{proposition}
יהי \(Q\) אופרטור אינווראנטי לזמן(כלומר אופרטור המקיים \([H,Q]=0\)). אזי אם \(\ket{\psi_{n}}\) מצב עצמי של \(H\) אז גם \(Q\ket{\psi_{n}}\) מצב עצמי של \(H\).

\end{proposition}
\begin{proof}
$${\hat{H}} \left|\psi_{n}^{\prime}\right\rangle={\hat{H}}\left({\hat{Q}}\ |\psi_{n}\rangle\right)={\hat{Q}}\,{\hat{H}}\ |\psi_{n}\rangle={\hat{Q}}\,E_{n}\ |\psi_{n}\rangle=E_{n}\left({\hat{Q}}\ |\psi_{n}\rangle\right)=E_{n}\ \left|\psi_{n}^{\prime}\right\rangle$$

\end{proof}
\begin{remark}
סימטריה כזו היא המקור של רוב הניוון בפיזיקה, אך לרוב קשה לזהות את הסימטריה המתאימה.

\end{remark}
\begin{example}
יהי \(\psi_{n\ell m}\) מצב עצמי של פוטנציאל מרכזי עם אנרגיה \(E_{n}\). כיוון שהמילטוניאן מרכזי מתחלף עם כל רכיב של \(\mathbf{L}\) נקבל כי גם \(L_{+},L_{-}\) מתחלפים עם ההמילטוניאן. לכן:
$$\left(\hat{H}\,\hat{L}_{\pm}-\hat{L}_{\pm}\,\hat{H}\right)\,\psi_{n\ell m}=0,$$
כלומר:
$$\hat{H}\,\hat{L}_{\pm}\,\psi_{n\ell m}=\hat{L}_{\pm}\,\hat{H}\,\psi_{n\ell m}=E_{n}\,\hat{L}_{\pm}\,\psi_{n\ell m}$$
או לחלופין:
$$\hat{H}\,\psi_{n\ell m\pm1}=E_{n}\,\psi_{n\ell m\pm1}$$
וגם \(\psi_{n\ell m\pm 1}\) מצב עצמי. ולכן הסימטריה הזו לסיבובים מסבירה את הניוון של \(m\). אכן אם אנחנו מבטלים את הסימטריה הזו על ידי שדה חשמלי או מגנטי אנחנו מאבדים את הניוון.

\end{example}
\begin{example}[שבירת סימטריה כמקור לאיבוד ניוון]
עבור למשל בעיית אטום המימן ניתן להוסיף הפרעה של שדה חשמלי אשר הורסת את הזוגיות המרחבית של ההמילטוניאן. זה גורם לאיבוד של הניוון.

\end{example}
\begin{summary}
  \begin{itemize}
    \item היפוך בזמן הופך את הסימן של גדלים התלויים בזמן בצורה אי זוגית, כמו תנע, תנע זוויתי וספין כאשר משאירים את האברים שלא תלויים בזמן כמו מיקום. כלומר:
$$x\mapsto x\quad v\mapsto -v\quad a\mapsto a\quad L\mapsto -L\quad t\mapsto-t$$
    \item השדה החשמלי והמגנטי עוברים היפוך בזמן על ידי:
$$\vec{E}\mapsto \vec{E}\qquad \vec{B}\mapsto-\vec{B}$$
  \end{itemize}
\end{summary}
\chapter{תנע זוויתי}

\section{חבורת הסיבובים}

\begin{proposition}
אופרטור הסיבוב יוצר חבורה, אשר ב-3 מימדים נקראת \(SO(3)\).

\end{proposition}
\begin{proposition}[היוצרים של SO(3)]
יהיו מהצורה:
$$J_{1}=\left(\begin{array}{l l l}{{0}}&{{0}}&{{0}}\\ {{0}}&{{0}}&{{i}}\\ {{0}}&{{-i}}&{{0}}\end{array}\right),\quad J_{2}=\left(\begin{array}{l l l}{{0}}&{{0}}&{{-i}}\\ {{0}}&{{0}}&{{0}}\\ {{i}}&{{0}}&{{0}}\end{array}\right),\quad J_{3}=\left(\begin{array}{l l l}{{0}}&{{i}}&{{0}}\\ {{-i}}&{{0}}&{{0}}\\ {{0}}&{{0}}&{{0}}\end{array}\right)$$

\end{proposition}
\begin{proof}
ניתן לכתוב כל איבר \(O \in SO(3)\) על ידי יוצר \(J\) כך שמתקיים \(O=e^{ \theta J }\).
נזכור כי \(O\) צריך להקיים \(\det O=1\) ומאורתוגונאליות:
$${ O}^{T}{ O}=\mathrm{e}^{\theta J^{T}}\mathrm{e}^{\theta J}\stackrel{!}{=}1 \implies e^{ \theta(J^{T}+J) }=1\implies J^{T}+J=0\implies J^{T}=-J$$
ולכן יהיה אנטי סימטרי. מטריצה כללית כזו תהיה מהצורה:
$$\omega=\left(\begin{array}{c c c}{{0}}&{{-\omega_{12}}}&{{-\omega_{13}}}\\ {{+\omega_{12}}}&{{0}}&{{-\omega_{23}}}\\ {{+\omega_{13}}}&{{+\omega_{23}}}&{{0}}\end{array}\right)=\sum_{i<j}i\omega_{i j}X_{i j}$$
נשים לב כי זהו מרחב ווקטורי ממימד 3, לכן מספיק למצוא 3 מטריצות אשר יהיו בסיס למרחב הזה. אם נדרוש בנוסף כי יהיה הרמיטי ניתן למצוא את הבסיס הבא:
$$X_{12}=\left(\begin{array}{l l l}{{0}}&{{i}}&{{0}}\\ {{-i}}&{{0}}&{{0}}\\ {{0}}&{{0}}&{{0}}\end{array}\right),\quad X_{23}=\left(\begin{array}{l l l}{{0}}&{{0}}&{{0}}\\ {{0}}&{{0}}&{{i}}\\ {{0}}&{{-i}}&{{0}}\end{array}\right),\quad X_{13}=\left(\begin{array}{l l l}{{0}}&{{0}}&{{i}}\\ {{0}}&{{0}}&{{0}}\\ {{-i}}&{{0}}&{{0}}\end{array}\right).$$
כאשר מקיים את היחס חילוף:
$$[X_{i j},X_{l k}]=-i\left(\delta_{j l}X_{i k}-\delta_{i l}X_{j k}-\delta_{i k}X_{l j}+\delta_{j k}X_{l i}\right)$$
נשים לב כי זהו בסיס למרחב היוצרים, ולא יוצר בעצמו. ולכן ניתן להגדיר:
$$J_{1}=X_{23},\quad J_{2}=-X_{13},\quad J_{3}=X_{12}$$

\end{proof}
\begin{corollary}
היוצרים של \(SO(3)\) מקיימים את היחס חילוף:
$$[J_{1},J_{2}]=i J_{3},\quad[J_{2},J_{3}]=i J_{1},\quad[J_{3},J_{1}]=i J_{2},$$
כלומר בעזרת סימון לוי ציווטה נקבל:
$$\left[J_a,J_{b}\right]=i\epsilon_{a b c}J_{c}$$

\end{corollary}
\begin{proposition}
האופרטור:
$$J^{2}=\sum_{a}J_{a}J_{a}=J_{1}^{2}+J_{2}^{2}+J_{3}^{2}$$
מתחלף עם כל היוצרים של החבורה \(J_{1},J_{2},J_{3}\).

\end{proposition}
\begin{proof}
נכפיל את \(\left[J_a,J_{b}\right]=i\epsilon_{a b c}J_{c}\) ב-\(J_{a}\) משמאל ונסכום על \(a\). נחזור על התהליך מימין. נקבל:
\begin{gather*}{{J^{2}J_{c}-J_{a}J_{c}J_{a}}}{{=}}{{i\epsilon_{a c b}J_{a}J_{b}}}\\ {{J_{a}J_{c}J_{a}-J_{c}J^{2}}}{{=}}{{i\epsilon_{a c b}J_{b}J_{a}}} 
\end{gather*}
אם נסכום את שתי המשוואות האלו ומאנטיסימטריה של טנזור לוי ציוויטה \(\varepsilon_{ijk}\) נקבל:
$$[J^{2},J_{c}]=0$$
ניתן לחזור על התהליך ולקבל שהיחס חילוף מתאפס עבור כל אחד מהיוצרים.

\end{proof}
\begin{definition}[הבסיס הקרטזי]
נגדיר משתנים חדשים \(J_{x},J_{y},J_{z}\) באופן הבא:
$$J_{x}=J_{1} \qquad J_{y}=J_{2} \qquad  J_{z}=J_{3}$$

\end{definition}
\begin{definition}[הבסיס הפולארי]
נגדיר משתנים חדשים \(J_{0},J_{+},J_{-}\) באופן הבא:
$$J_{\pm}=J_{x}\pm i J_{y}\qquad J_{0}=J_{z}$$
כאשר האופרטורים \(J_{\pm}\) נקראים אופרטורי סולם, ואינם הרמיטים אבל כן מתקיים \(J_{\pm}^{\dagger}=J_{\mp}\).

\end{definition}
\begin{corollary}
בבסיס הפולארי נקבל את היחסי חילוף:
\begin{gather*}{{\left[J_{+},J_{-}\right]}}{{=}}{{2J_{0}}}\\ {{\left[J_{0},J_{\pm}\right]}}{{=}}{{\pm J_{\pm}}}  
\end{gather*}

\end{corollary}
\begin{proposition}[חסם עם הערכים העצמיים]
כיוון ש-\(J^{2}\) ו-\(J_{0}\) מתחלים קיים בסיס של ווקטורים עצמיים משותפים. נסמן בסיס זה ב-\(\left\{  \ket{ab}  \right\}\) כך שמתקיים:
$$J^{2}|a b\rangle=a|a b\rangle\qquad J_{0}|a b\rangle=b|a b\rangle$$
אזי \(a\geq b^{2}\).

\end{proposition}
\begin{proof}
נשים לב כי ניתן לכתוב:
$$J^{2}=J_{0}^{2}+\frac{1}{2}\left(J_{-}J_{+}+J_{+}J_{-}\right)$$
וכעת:
$$J^{2}-J_{0}^{2}=\frac{1}{2}\left(J_{+}^{\dagger}J_{+}+J_{+}J_{+}^{\dagger}\right)$$
ולכן נקבל:
$$\langle a b|J^{2}-J_{0}^{2}|a b\rangle=\langle a b|\frac{1}{2}\left(J_{+}^{\dagger}J_{+}+J_{+}J_{+}^{\dagger}\right)|a b\rangle\geq0.$$
כאשר כיוון ש-\(J^{2}\ket{ab}=a\ket{ab}\) ו-\(J_{0}\ket{ab}=b\ket{ab}\) נקבל:
$$\langle a b|J^{2}-J_{0}^{2}|a b\rangle=\bra{ab}  a-b^{2}\ket{ab} =a-b^{2}\geq0\;\Rightarrow\;a\ge b^{2}$$

\end{proof}
\begin{proposition}
האופרטור העלה \(J_{+}\) מופעל על המצב \(\ket{a,b}\) יהיה מצב עצמי של \(J_{0}\). כלומר:
$$J_{0}J_{+}\ket{a,b} =(b+1)J_{+}\ket{a,b} $$

\end{proposition}
\begin{proof}
$$J_{0}J_{+}|a b\rangle=\left(J_{+}J_{0}+[J_{0},J_{+}]\right)|a b\rangle\\ =J_{+}(J_{0}+1)|a b\rangle=\left(b+1\right)J_{+}|a b\rangle$$

\end{proof}
\begin{corollary}
באופן זהה מתקיים:
$$J_{0}J_{-}\ket{a,b} =(b-1)\ket{a,b} $$
ולכן ניתן לכתוב:
$$J_{\pm}|a b\rangle=C_{\pm}|a b\pm1\rangle$$

\end{corollary}
\begin{corollary}
כיוון שיש ערך מינימלי ומקסימלי של \(b\) נקבל כי:
$$J_{+}|a b_{\mathrm{max}}\rangle=0,\quad\mathrm{and}\quad J_{-}|a b_{\mathrm{min}}\rangle=0$$

\end{corollary}
\begin{corollary}
הערכים עצמיים \(a\) צריכים להיות מהצורה:
$$a=b_{\operatorname*{max}}(b_{\operatorname*{max}}+1)$$
או:
$$a=b_{\operatorname*{min}}(b_{\operatorname*{min}}-1)$$

\end{corollary}
\begin{proof}
נשים לב כי:
$$J_{-}J_{+}=J_{x}^{2}+J_{y}^{2}-i\left(J_{y}J_{x}-J_{x}J_{y}\right)=J^{2}-J_{0}^{2}-J_{0},$$
כעת אם נפעיל אופרטור זה על מצב \(\ket{a, b_{\text{max}}}\) נקבל:
$$J_{-}J_{+}|a b_{\mathrm{max}}\rangle=(J^{2}-J_{0}^{2}-J_{0})|a b_{\mathrm{max}}\rangle=(a-b^{2}-b)|a b_{\mathrm{max}}\rangle$$
כיוון שהמצב \(\ket{a,b_{\max}}\) אינו אפס, ואנחנו אנחנו מבצעים אופרטור העלה מעל הערך המקסימלי נדרש כי הערך יתאפס:
$$a=b_{\operatorname*{max}}(b_{\operatorname*{max}}+1)$$
כאשר אופן דומה מהתנאי \(J_{-}\ket{a,b_{\text{min}}}=0\) ומהיחס:
$$J_{+}J_{-}=J^{2}-J_{0}^{2}+J_{0}$$
נקבל:
$$a=b_{\operatorname*{min}}(b_{\operatorname*{min}}-1)$$

\end{proof}
\begin{corollary}
כיוון שקיים \(k \in \mathbb{Z}\) כך ש-\(b_{\max}=b_{\text{min}}+k\) נקבל:
\begin{gather*}b_{\min}\left( b_{\min}-1 \right)=\left( b_{\min}+k \right)\left( b_{\min}+k+1 \right)\implies \\ b_{\min}^{2}-b_{\min}=b_{\min}^{2}+b_{\min}+k(k+1)+2kb_{\min}\\ \implies -2b_{\min}(k+1)=k(k+1) 
\end{gather*}
כלומר נקבל:
$$b_{\min}=-\frac{k}{2}\qquad b_{\max}=\frac{k}{2}$$
ולכן \(b_{\text{min}}=-b_{\text{max}}\) והערך של \(b_{\text{max}}\) יהיה חצי ממספר שלם.

\end{corollary}
\begin{symbolize}
נסמן \(m=b\) וכן \(j=b_{\text{max}}\) ולכן \(a=j(j+1)\). 

\end{symbolize}
\begin{corollary}
לכל \(j\) הערכים המותרים של \(m\) יהיו:
$$m=\underbrace{-j,-j+1,\ldots,j-1,j}_{2j+1\;{\mathrm{states}}}$$

\end{corollary}
\begin{proposition}[אורטורי העלה והורדה]
האופרטורים \(J_{-}\) ו-\(J_{+}\) הם למעשה אופרטורי העלה והורדה של התנע הזוויתי, כך שמתקיים:
\begin{gather*}J_{+}|j,m\rangle=\sqrt{(j-m)(j+m+1)}\hbar|j,m+1\rangle\\J_{-}|j,m\rangle=\sqrt{(j+m)(j-m+1)}\hbar|j,m-1\rangle 
\end{gather*}

\end{proposition}
\begin{proof}
עבור \(J_{\pm}\) נשים לב כי:
$$\begin{array}{c}{{\langle j,m|J_{+}^{\dagger}J_{+}|j,m\rangle=\langle j,m|(J^{2}-J_{z}^{2}-\hbar J_{z})|j,m\rangle}}\\ {{=\hbar^{2}[j(j+1)-m^{2}-m].}}\end{array}$$
כאשר אנו יודעים כי \(J_{+}\ket{ j,m}\) צריך להיות שווה ל-\(\ket{j,m+1}\) עד כדי קבוע נרמול. כלומר:
$$J_{+}|j,m\rangle=c_{j m}^{+}|j,m+1\rangle$$
כאשר ע"י השוואה עם הביטוי של הערך מטריצה של \(J_{+}^{\dagger}J_{+}\) נקבל:
$$\begin{array}{c}{{|c_{j m}^{+}|^{2}=\hbar^{2}[j(j+1)-m(m+1)]}}\\ {{=\hbar^{2}(j-m)(j+m+1).}}\end{array}$$
כלומר קיבלנו את \(c_{jm}^{+}\) עד כדי פאזה. נהוג לבחור \(c_{jm}^{+}\) להיות ממשי וחיובי ולכן:
$$J_{+}|j,m\rangle=\sqrt{(j-m)(j+m+1)}\hbar|j,m+1\rangle$$
על ידי ביצוע תהליך דומה ניתן לקבל:
$$J_{-}|j,m\rangle=\sqrt{(j+m)(j-m+1)}\hbar|j,m-1\rangle$$

\end{proof}
נראה את המצבים האפשריים עבור ערכים שונים של \(j\):

\begin{table}[htbp]
  \centering
  \begin{tabular}{|cccc|}
    \hline
    \(j\) & \(j(j+1)\) & \(m\) & \(\ket{j,m}\) \\ \hline
    \(0\) & \(0\) & \(0\) & \(0\) \\ \hline
    \(\frac{1}{2}\) & \(\frac{1}{2}\cdot \frac{3}{2}=\frac{3}{4}\) & \(-\frac{1}{2},\frac{1}{2}\) & \(\ket{\frac{1}{2},-\frac{1}{2}},\ket{\frac{1}{2},\frac{1}{2}}\) \\ \hline
    \(1\) & \(1\cdot 2 = 2\) & \(-1,0,1\) & \(\ket{1,-1},\ket{1,0},\ket{1,1}\) \\ \hline
  \end{tabular}
\end{table}
\begin{summary}
  \begin{itemize}
    \item היוצרים של סיבובים ב-3 מימדים הם:
$$J_{x}=\left(\begin{array}{l l l}{{0}}&{{0}}&{{0}}\\ {{0}}&{{0}}&{{i}}\\ {{0}}&{{-i}}&{{0}}\end{array}\right),\quad J_{y}=\left(\begin{array}{l l l}{{0}}&{{0}}&{{-i}}\\ {{0}}&{{0}}&{{0}}\\ {{i}}&{{0}}&{{0}}\end{array}\right),\quad J_{z}=\left(\begin{array}{l l l}{{0}}&{{i}}&{{0}}\\ {{-i}}&{{0}}&{{0}}\\ {{0}}&{{0}}&{{0}}\end{array}\right)$$
    \item האופרטור \(J^{2}=J_{x}^{2}+J_{y}^{2}+J_{z}^{2}\) מתחלף עם כל היוצרים.
    \item הבסיס הפולארי מוגדר על ידי:
$$J_{+}=J_{x}+ i J_{y}\qquad J_{-}=J_{x}- i J_{y}\qquad J_{0}=J_{z}$$
    \item האופרטור \(J^{2}\) מתחלף עם \(J_{0}\) ולכן קיים עבורם בסיס משותף \(\{ \ket{ab} \}\).
    \item הערכים העצמיים חסומים כך שמתקיים \(b_{\text{max}}=-b_{\min}=\frac{k}{2}\) עבור \(k \in \mathbb{N}\). ולכן ניתן לסמן \(j=b_{\text{max}}\) ו-\(m=b\) ולקבל כי הבסיס המתחלף של \(J^{2}\) ו-\(J_{0}\) יהיה מהצורה \(\left\{  \ket{j(j+1),m}  \right\}\) אך בדרך כלל מוסמן פשוט \(\ket{j,m}\).
    \item עבור \(j\) נתון יש \(2j+1\) ערכים מותרים של \(m\).
    \item האופרטורי סולם \(J_{+}\) ו-\(J_{-}\) יקיימו:
$$J_{\pm} = \sqrt{ \left( j\mp m \right)\left( j \pm m + 1 \right) } \hbar \ket{j,m+1} $$
  \end{itemize}
\end{summary}
\section{אלמנטי מטריצת הסיבוב}

\begin{proposition}[אלמנטי המטריצות של היוצרים של התנע הזוויתי]
\begin{gather*}\left\langle  j^{\prime},m^{\prime}|{\bf J}^{2}|j,m \right\rangle=j(j+1)\hbar^{2}\delta_{j^{\prime}j}\delta_{m^{\prime}m}\\\left\langle  j^{\prime},m^{\prime}|J_{z}|j,m \right\rangle=m\hbar\delta_{j^{\prime}j}\delta_{m^{\prime}m}\\ \left\langle  j^{\prime},m^{\prime}|J_{\pm}|j,m \right\rangle=\sqrt{\left( j\mp m \right)\left( j\pm m+1 \right)}\hbar\delta_{j^{\prime}j}\delta_{m^{\prime},m\pm1} 
\end{gather*}

\end{proposition}
\begin{proof}
עבור \(J^{2}\) נקבל מכך שמתקיים:
$${\bf J}^{2}|j,m\rangle=j(j+1)\hbar^{2}|j,m\rangle$$
ולכן
$$ \left\langle  j^{\prime},m^{\prime}|{\bf J}^{2}|j,m \right\rangle=\left\langle  j^{\prime},m^{\prime}|j(j+1)\hbar^{2}|j,m \right\rangle=j(j+1)\hbar^{2} \langle j^{\prime},m^{\prime}| j,m \rangle=j(j+1)\hbar^{2}\delta_{j^{\prime}j}\delta_{m^{\prime}m}$$
ניתן לקבל באופן דומה עבור \(J_{z}\) בעזרת הע"ע:
$$J_{z}|j,m\rangle=m\hbar|j,m\rangle$$
עבור \(J_{\pm}\) אנו יודעים כי:
\begin{gather*}J_{+}|j,m\rangle=\sqrt{(j-m)(j+m+1)}\hbar|j,m+1\rangle\\ J_{-}|j,m\rangle=\sqrt{(j+m)(j-m+1)}\hbar|j,m-1\rangle 
\end{gather*}
ולכן אלמנטי המטריצה \(J_{\pm}\) יהיו:
$$\langle j^{\prime},m^{\prime}|J_{\pm}|j,m\rangle=\sqrt{(j\mp m)(j\pm m+1)}\hbar\delta_{j^{\prime}j}\delta_{m^{\prime},m\pm1}$$

\end{proof}
\begin{definition}[מטריצת אלמנטי מטריצת הסיבוב]
עבור סיבוב \(R\) אשר מוגדר על ידי \(\hat{n}\) ו-\(\phi\) ניתן להגדיר את אלמנטי המטריצה על ידי:
$${\mathcal{D}}_{m^{\prime}m}^{(j)}(R)=\langle j,m^{\prime}|\exp\left(\frac{-i{\bf J}\!\cdot\!{\hat{\bf n}}\phi}{\hbar}\right)|j,m\rangle$$
כאשר אלמנטי המטריצה לעיתים נקראות פונקציות ויגנר. 

\end{definition}
\begin{proposition}
האלמנט \({\mathcal{D}}(R)|j,m\rangle\) יהיה ווקטור עצמי של \(J^{2}\) עם ערך עצמי \(j(j+1)\hbar\) ולכן:
$${{{\bf J}^{2}\mathcal{D}(R)|j,m\rangle=\mathcal{D}(R){\bf J}^{2}|j,m\rangle}} {{=j(j+1)\hbar^{2}[\mathcal{D}(R)|j,m\rangle]}}$$

\end{proposition}
\begin{remark}
האופרטור \({\mathcal{D}}_{m^{\prime}m}^{(j)}(R)\) מחזיר מטריצה \((2j+1)\times (2j+1)\) אשר נקראת ההצגה הבלתי פריקה של האופרטור. ולכן מטריצת סיבוב במרחב הילברט שלא בהכרח מאפיין על ידי ערך \(j\) יחיד יכול להיות מבוא לצורת בלוקים ריבועיות מגודל \(2j+1\) על האלכסון אשר יהיו מיוצגות על ידי \(\mathcal{D}^{(j)}_{m'm}\).

\end{remark}
\begin{proposition}[תכונות של מטריצת הסיבוב המאופיינת על ידי \(j\) יחיד]
מטריצות הסיבוב אשר מאפיינות על ידי \(j\) יחיד יוצרות חבורה, כלומר הזהות תהיה כזו,  \((\mathcal{D}_{m^{\prime\prime}m^{\prime}}^{(j)})^{-1}\) תהיה מטריצת סיבוב אשר מופיינת על ידי \(j\) יחיד, וכן משמר כפל:
$$\sum_{m^{\prime}}\mathcal{D}_{m^{\prime\prime}m^{\prime}}^{(j)}(R_{1})\mathcal{D}_{m^{\prime}m}^{(j)}(R_{2})=\mathcal{D}_{m^{\prime\prime}m}^{(j)}(R_{1}R_{2})$$
כמו כן אוניטרי:
$$\mathcal{D}_{m^{\prime}m}\big(R^{-1}\big)=\mathcal{D}_{m m^{\prime}}^{*}(R)$$

\end{proposition}
\begin{proposition}
ניתן לכתוב את מטריצת הסיבוב המאופיינת על ידי \(j\) יחיד על ידי:
\begin{gather*}\mathcal{D}_{m^{\prime}m}^{(j)}\left( \alpha,\beta,\gamma \right)=\left\langle  j,m^{\prime}|\exp\left(\frac{-iJ_{z}\alpha}{\hbar}\right)\exp\left(\frac{-iJ_{y}\beta}{\hbar}\right)\exp\left(\frac{-iJ_{z}\gamma}{\hbar}\right)|j,m \right\rangle\\=e^{-i\left( m^{\prime}\alpha+m\gamma \right)}\left\langle  j,m^{\prime}|\exp\left(\frac{-iJ_{y}\beta}{\hbar}\right)|j,m \right\rangle 
\end{gather*}
כאשר לעיתים קוראים להצגת המטריצה "ההצגה הבלתי פריקה" של אופרטור הסיבוב.

\end{proposition}
\begin{symbolize}
החלק הלא טריוויאלי הוא רק הסיבוב סביב ציר ה-\(y\) אשר מערבב ערכי \(m\) שונים. ולכן נוח להגדיר מטריצה חדשה \(d^{(j)}\left( \beta \right)\) על ידי:
$$d_{m^{\prime}m}^{(j)}(\beta)\equiv\langle j,m^{\prime}|\exp\left(\frac{-i J_{y}\beta}{\hbar}\right)|j,m\rangle$$

\end{symbolize}
\begin{example}
עבור \(j=\frac{1}{2}\) נצפה למטריצה ריבועית מגודל \(2\cdot \frac{1}{2}+1=2\) וכן מתקבל:
\begin{gather*}\exp\left(\frac{-i\sigma_{3}\alpha}{2}\right)\exp\left(\frac{-i\sigma_{2}\beta}{2}\right)\exp\left(\frac{-i\sigma_{3}\gamma}{2}\right)=\\=\begin{pmatrix}e^{-i\alpha/2}&0\\ 0&e^{i\alpha/2}\end{pmatrix}\begin{pmatrix}\cos\left( \beta/2 \right)&-\sin\left( \beta/2 \right)\\ \sin\left( \beta/2 \right)&\cos\left( \beta/2 \right)\end{pmatrix}\begin{pmatrix}e^{-i\gamma/2}&0\\ 0&e^{i\gamma/2}\end{pmatrix}=\\=\begin{pmatrix}e^{-i\left( \alpha+\gamma \right)/2}\cos\left( \beta/2 \right)&-e^{-i\left( \alpha-\gamma \right)/2}\sin\left( \beta/2 \right)\\ e^{i\left( \alpha-\gamma \right)/2}\sin\left( \beta/2 \right)&e^{i\left( \alpha+\gamma \right)/2}\cos\left( \beta/2 \right)\end{pmatrix} 
\end{gather*}
כאשר נקבל כי:
$$d^{1/2}=\left(\begin{array}{c c}{{\cos\left(\frac{\beta}{2}\right)}}&{{-\sin\left(\frac{\beta}{2}\right)}}\\ {{\sin\left(\frac{\beta}{2}\right)}}&{{\cos\left(\frac{\beta}{2}\right)}}\end{array}\right)$$

\end{example}
\begin{example}
ניתן גם לפתור בפורש עבור \(j=1\). ניתן לכתוב \(J_{y}=\frac{(J_{+}-J_{-})}{2i}\) כאשר מהביטוי שפיתחנו עבור האלמנטי מטריצות המתאימות ניתן לקבל:
$$d^{(1)}(\beta)=\left(\begin{array}{c c c}{{\left(\frac{1}{2}\right)\left(1+\cos\beta\right)}}&{{-\left(\frac{1}{\sqrt{2}}\right)\sin\beta}}&{{\left(\frac{1}{2}\right)\left(1-\cos\beta\right)}}\\  {{\left(\frac{1}{\sqrt{2}}\right)\sin\beta}}&{{\cos\beta}}&{{-\left(\frac{1}{\sqrt{2}}\right)\sin\beta}}\\ {{\left(\frac{1}{2}\right)\left(1-\cos\beta\right)}}&{{\left(\frac{1}{\sqrt{2}}\right)\sin\beta}}&{{\left(\frac{1}{2}\right)\left(1+\cos\beta\right)}}\end{array}\right)$$

\end{example}
ניתן לראות שככל שנבצע את זה עבור \(j\) גבוה יותר התהליך יהיה יותר קשה.

\begin{proposition}[נוסחאת וויגנר]
עבור \(j\) כללי ניתן לקבל נוסחה מפורשת עבור \(d^{(j)}_{m',m}\):
\begin{gather*}d_{m^{\prime}m}^{(j)}\,\left( \beta \right)\;=\;\sum_{k}(-1)^{k+m^{\prime}-m}\frac{\sqrt{(j+m)!(j-m)!\left( j+m^{\prime} \right)!\left( j-m^{\prime} \right)!}}{\left( j-m^{\prime}-k \right)!(j+m-k)!\left( k+m^{\prime}-m \right)!k!}\\ \times\;\left(\cos\frac{\beta}{2}\right)^{2j+m-m^{\prime}-2k}\left(\sin\frac{\beta}{2}\right)^{m^{\prime}-m+2k} 
\end{gather*}

\end{proposition}
\begin{summary}
  \begin{itemize}
    \item אלמטי המטריצה של יוצרי התנע הזוויתי יהיו:
\begin{gather*}\left\langle  j^{\prime},m^{\prime}|{\bf J}^{2}|j,m \right\rangle=j(j+1)\hbar^{2}\delta_{j^{\prime}j}\delta_{m^{\prime}m}\qquad  \left\langle  j^{\prime},m^{\prime}|J_{z}|j,m \right\rangle=m\hbar\delta_{j^{\prime}j}\delta_{m^{\prime}m}\\ \left\langle  j^{\prime},m^{\prime}|J_{\pm}|j,m \right\rangle=\sqrt{\left( j\mp m \right)\left( j\pm m+1 \right)}\hbar\delta_{j^{\prime}j}\delta_{m^{\prime},m\pm1} 
\end{gather*}
    \item אלמנטי המטריצה של אופרטור הסיבוב לעיתים נקראת מטריצת וויגנר ומוגדרת על ידי:
$${\mathcal{D}}_{m^{\prime}m}^{(j)}(R)=\langle j,m^{\prime}|\exp\left(\frac{-i{\bf J}\!\cdot\!{\hat{\bf n}}\phi}{\hbar}\right)|j,m\rangle$$
    \item ניתן להגדיר מטריצה ווגנר מצומצמת
$$d_{m^{\prime}m}^{(j)}(\beta)\equiv\langle j,m^{\prime}|\exp\left(\frac{-i J_{y}\beta}{\hbar}\right)|j,m\rangle$$
כאשר קיים עבורו ביטוי מפורש.
  \end{itemize}
\end{summary}
\section{תנע זוויתי אורביטלי}

\begin{reminder}[אופרטור הסיבוב]
עבור סיבוב \(R \in SO(3)\) קיים אופרטור אוניטרי \(U(R)\) אשר מקיים:
$$U(R)|\mathbf{x}\rangle=|R\mathbf{x}\rangle$$
כאשר \(U(R)\) היא הצגה של החבורה \(SO(3)\), ולכן בפרט מתקיים:
$$U(R_{1})U(R_{2})=U(R_{1}R_{2})\qquad U\left( \mathbb{1}  \right)=1\qquad U(R^{-1})=U(R)^{-1}$$

\end{reminder}
\begin{proposition}[השפעה של סיבוב על פונקציית הגל]
נניח כי פונקציית הגל \(\psi\left( \mathbf{x} \right)\) עובר לאחר סיבוב ל-\(\psi'\left( \mathbf{x} \right)\). אזי מתקיים:
$$\psi^{\prime}({\bf x})=\left(U(R)\psi\right)({\bf x})=\psi(R^{-1}{\bf x})$$

\end{proposition}
\begin{proof}
ניתן לכתוב את המצב על ידי:
$$|\psi\rangle=\int d^{3}{\bf x}\,|{\bf x}\rangle\langle{\bf x}|\psi\rangle=\int d^{3}{\bf x}\,|{\bf x}\rangle\psi({\bf x})$$
כאשר אם נפעיל את \(U(R)\) על שתי האגפים נקבל:
$$|\psi^{\prime}\rangle=U(R)|\psi\rangle\implies|\psi^{\prime}\rangle=\int d^{3}{\bf x}\,|{\cal R}{\bf x}\rangle\psi\left( {\bf x} \right)$$
אם נבצע החלפת משתנה נקבל:
$$d^{3}{\bf x}^{\prime}=(\operatorname*{det}{\sf R})d^{3}{\bf x}=d^{3}{\bf x},$$
ולכן נקבל:
$$|\psi^{\prime}\rangle=\int d^{3}{\bf x}\,|{\bf x}\rangle\psi(R^{-1}{\bf x})$$
כאשר אם נפעיל את \(\bra{\mathbf{x}}\) נקבל את הנדרש.

\end{proof}
\begin{definition}[תנע זוויתי אורביטלי]
היוצר של שינוי בזווית. כלומר עבור סיבוב כללי סביב ציר \(\hat{n}\) בזווית \(\theta\) המוגדר על ידי:
$$R\left( \mathbf{\hat{n}},\theta \right)=\mathsf{I}+\theta\mathbf{\hat{n}}\cdot \mathbf{J}$$
כאשר \(\theta\ll 1\) האופרטור האוניטרי המתאים יהיה:
$$U(R)=U(\mathbf{\hat{n}},\theta)=1-{\frac{i}{\hbar}}\theta\mathbf{\hat{n}}\cdot\mathbf{L}$$

\end{definition}
\begin{proposition}
האופרטור התנע הזוויתי האורביטלי מקיים:
$$\mathbf{L}=\mathbf{x}\times\mathbf{p}$$

\end{proposition}
\begin{proof}
ראשית נזכור כי מתקיים:
$$R({\hat{\mathbf{n}}},\theta)^{-1}=\mathbf{l}-\theta{\hat{\mathbf{n}}}\cdot\mathbf{J},$$
מהמשוואה \(\left(U(R)\psi\right)({\bf x})=\psi(R^{-1}{\bf x})\) והביטוי עבור \(R\left( \hat{n},\theta \right)\) נקבל:
$$\Big(1-{\frac{i}{\hbar}}\theta\mathbf{\hat{n}}\cdot\mathbf{L}\Big)\psi(\mathbf{x})=\psi\big((1-\theta\mathbf{\hat{n}}\cdot\mathbf{J})\mathbf{x}\big)=\psi(\mathbf{x}-\theta\mathbf{\hat{n}}\times\mathbf{x})=\psi(\mathbf{x})-\theta(\mathbf{\hat{n}}\times\mathbf{x})\cdot\nabla\psi,$$
כאשר לקחנו רק את האיברים מסדר ראשון ב-\(\theta\). ה-\(\theta\) מתבטל מהגורמים מסדר ראשון ונקבל:
$$({\hat{\mathbf{n}}}\cdot\mathbf{L})\psi=-i\hbar({\hat{\mathbf{n}}}{\times}\mathbf{x})\cdot\nabla\psi=({\hat{\mathbf{n}}}{\times}\mathbf{x})\cdot\mathbf{p}\,\psi={\hat{\mathbf{n}}}\cdot(\mathbf{x}{\times}\mathbf{p})\psi,$$
כאשר כיוון ש-\(\hat{n}\) ו-\(\psi\) שרירותיים נקבל:
$$\mathbf{L}=\mathbf{x}{\times}\mathbf{p}$$

\end{proof}
\begin{remark}
ניתן היה לחלופין להגדיר את התנע הזוויתי בתור \(\mathbf{L}=\mathbf{x}\times \mathbf{p}\) ולהראות כי היוצר של הסיבוב בעזרת היחסי חילוף.

\end{remark}
\begin{corollary}
התנע הזוויתי הקוונטי מוגדר באופן דומה לתנע הקלאסי:
$$\hat{\vec{L}}=\hat{\vec{R}}\times\hat{\vec{P}}=-i\hbar\hat{\vec{R}}\times\vec{\nabla}$$
זהו למעשה אופרטור ווקטורי עם שלושה רכיבים, כך שהרכיבים של האופרטור בקורדינטות קרטזיות יהיו:
\begin{gather*}\hat{L}_{x}=\hat{Y}\hat{P}_{z}-\hat{Z}\hat{P}_{y}=-i\hbar\left(\hat{Y}\frac{\partial}{\partial z}-\hat{Z}\frac{\partial}{\partial y}\right)\\ \hat{L}_{y}=\hat{Z}\hat{P}_{x}-\hat{X}\hat{P}_{z}=-i\hbar\left(\hat{Z}\frac{\partial}{\partial x}-\hat{X}\frac{\partial}{\partial z}\right)\\\hat{L}_{z}=\hat{X}\hat{P}_{y}-\hat{Y}\hat{P}_{x}=-i\hbar\left(\hat{X}\frac{\partial}{\partial y}-\hat{Y}\frac{\partial}{\partial x}\right) 
\end{gather*}

\end{corollary}
\begin{proposition}[יחסי חילוף של התנע הזוויתי]
$$[\hat{L}_{x},\hat{L}_{y}]=i\hbar\hat{L}_{z}\qquad[\hat{L}_{y},\hat{L}_{z}]=i\hbar\hat{L}_{x}\qquad[\hat{L}_{z},\hat{L}_{x}]=i\hbar\hat{L}_{y}$$

\end{proposition}
\begin{proof}
נראה עבור \([L_{x},L_{y}]\). מתקיים:
\begin{gather*}[L_{x},L_{y}]=[y p_{z}-z p_{y},z p_{x}-x p_{z}]=\left[y p_{z},z p_{x}\right]+\left[z p_{y},x p_{z}\right]=\\=y p_{x}[p_{z},z]+p_{y}x[z,p_{z}]=i\hbar(x p_{y}-y p_{x})=i\hbar L_{z} 
\end{gather*}

\end{proof}
\begin{corollary}
ניתן מהיחסי חילוף לקבל כי התנע הזוויתי האורביטלי הוא היוצר של השינוי בזווית.

\end{corollary}
\begin{remark}
ניתן לסכם את חוקי החילוף על ידי:
$$\hat{\mathbf{L}}\times\hat{\mathbf{L}}=i\hbar\hat{\mathbf{L}}$$
או בעזרת לווי צ'יווטה:
$$[L_{i},L_{j}]=i\varepsilon_{i j k}\hbar L_{k}$$

\end{remark}
\begin{proposition}[הצגת תנע זוויתי בקורדינטות כדוריות]
\begin{gather*}\hat{L}_{x}=i\hbar\Bigg(\sin\phi\frac{\partial}{\partial\theta}+\cot\theta\cos\phi\frac{\partial}{\partial\phi}\Bigg)\\\hat{L}_{y}=i\hbar\Bigg{(}-\cos\phi\frac{\partial}{\partial\theta}+\cot\theta\sin\phi\frac{\partial}{\partial\phi}\Bigg{)}\\\hat{L}_{z}=-i\hbar\frac{\partial}{\partial\phi} 
\end{gather*}

\end{proposition}
\begin{proof}
ניתן להראות זאת בעזרת מטריצות יעקובי של מעברי קורדינטות.

\end{proof}
\begin{proposition}[אופרטורי העלה והורדה]
$$L_{\pm}=L_{x}\pm i L_{y}=-i\hbar\,e^{\pm i\phi}\Big(\pm i\frac{\partial}{\partial\theta}-\cot\theta\frac{\partial}{\partial\phi}\Big),$$

\end{proposition}
\begin{definition}[אופרטור התנע בריבוע]
$$\hat{L}^{2}=\hat{L}_{x}^{2}+\hat{L}_{y}^{2}+\hat{L}_{z}^{2}$$
לעיתים נקרא אופרטור קסימיר.

\end{definition}
\begin{proposition}
כל האיברים בבסיס הקרטזי מתחלפים עם \(L^{2}\).

\end{proposition}
\begin{proof}
נראה למשל עבור \(L_{x}\):
$$\begin{array}{l}{{\left[L^{2},L_{x}\right]=\left[L_{x}^{2},L_{x}\right]+\left[L_{y}^{2},L_{x}\right]+\left[L_{z}^{2},L_{x}\right]}}\\ {{\quad=L_{y}\left[L_{y},L_{x}\right]+\left[L_{y},L_{x}\right]L_{y}+L_{z}\left[L_{z},L_{x}\right]+\left[L_{z},L_{x}\right]L_{z}}}\\ {{\quad=L_{y}\left(-i\hbar L_{z}\right)+\left(-i\hbar L_{z}\right)L_{y}+L_{z}\left(i\hbar L_{y}\right)+\left(i\hbar L_{y}\right)L_{z}}}{{=0}}\end{array}$$

\end{proof}
\begin{proposition}
אופרטור קסימיר מקיים:
$$L^{2}=\frac{1}{2}\big(L_{+}L_{-}+L_{-}L_{+}\big)+L_{z}^{2}=-\hbar^{2}\Big[\frac{1}{\sin\theta}\frac{\partial}{\partial\theta}\Big(\sin\theta\frac{\partial}{\partial\theta}\Big)+\frac{1}{\sin^{2}\theta}\frac{\partial^{2}}{\partial\phi^{2}}\Big]=-\hbar \bar{\nabla}^2 _{\theta, \phi}$$

\end{proposition}
\begin{corollary}
ניתן לכתוב את התנע בעזרת אופרטור קסימיר:
$${\bf p}^{2}\psi=-\hbar^{2}\nabla^{2}\psi=-\hbar^{2}\frac{1}{r^{2}}\frac{\partial}{\partial r}\Big(r^{2}\frac{\partial\psi}{\partial r}\Big)+\frac{L^{2}}{r^{2}}\psi$$

\end{corollary}
\begin{proof}
אנו יודעים כי:
$$\nabla^{2}={\frac{1}{r^{2}}}{\frac{\partial}{\partial r}}\left(r^{2}{\frac{\partial}{\partial r}}\right)+{\frac{1}{r^{2}}}\left[{\frac{1}{\sin\theta}}{\frac{\partial}{\partial\theta}}\left(\sin\theta{\frac{\partial}{\partial\theta}}\right)+{\frac{1}{\sin^{2}\theta}}{\frac{\partial^{2}}{\partial\phi^{2}}}\right]=\frac{1}{r^{2}}\frac{\partial }{\partial r}\left( r^{2}\frac{\partial }{\partial r}  \right)-\frac{\hbar^{2}}{r^{2}}L^{2} $$
ולכן \(-\hbar^{2}\nabla^{2}\psi=-\hbar^{2}\frac{1}{r^{2}}\frac{\partial}{\partial r}\Big(r^{2}\frac{\partial\psi}{\partial r}\Big)+\frac{L^{2}}{r^{2}}\psi\).

\end{proof}
\begin{proposition}
ניתן לכתוב את אופרטור קסימיר בעזרת התנע הזוויתי והמיקום בצורה הבאה:
$${\bf L}^{2}={\bf x}^{2}{\bf p}^{2}-\left({\bf x}{\cdot}{\bf p}\right)^{2}+i\hbar{\bf x}{\cdot}{\bf p}$$

\end{proposition}
\begin{proof}
\begin{gather*}{\bf L}^{2}=\sum_{i j l m k}\varepsilon_{i j k}x_{i}p_{j}\varepsilon_{l m k}x_{l}p_{m}=\sum_{i j l m}\left( \delta_{i l}\delta_{j m}-\delta_{i m}\delta_{j l} \right)x_{i}p_{j}x_{l}p_{m}=\\=\sum_{i j l m}\left[ \delta_{i l}\delta_{j m}x_{i}\left( x_{l}p_{j}-i\hbar\delta_{j l} \right)p_{m}-\delta_{i m}\delta_{j l}x_{i}p_{j}\left( p_{m}x_{l}+i\hbar\delta_{l m} \right) \right]=\\={\bf x}^{2}{\bf p}^{2}-i\hbar{\bf x}\cdot{\bf p}-\sum_{i j l m}\delta_{i m}\delta_{j l}\left[ x_{i}p_{m}\left( x_{l}p_{j}-i\hbar\delta_{j l} \right)+i\hbar\delta_{l m}x_{i}p_{j} \right]= \\=\mathbf{x}^{2}\mathbf{p}^{2}-\left(\mathbf{x}{\cdot}\mathbf{p}\right)^{2}+i\hbar\mathbf{x}{\cdot}\mathbf{p}
\end{gather*}

\end{proof}
\begin{summary}
  \begin{itemize}
    \item סיבוב \(R \in SO(3)\) מיוצג על ידי אופרטור אוניטרי \(U(R)\ket{\mathbf{x}}=\ket{R\mathbf{x}}\). אופרטורים אלו מקיימים:
$$U(R_{1})U(R_{2})=U(R_{1}R_{2})\quad U(1)=1\quad U(R^{-1})=U(R)^{-1}$$
    \item הפונקציית גל תחת תחת סיבוב \(\psi'\left( \mathbf{x} \right)\) נתונה על ידי \(\psi\left( R^{-1}\mathbf{x} \right)\).
    \item התנע הזוויתי האורביטלי \(\mathbf{L}\) מוגדר על ידי היוצר של סיבוב מרחבי \(U(R)\approx1-{\frac{i}{\hbar}}\theta\mathbf{\hat{n}}\cdot\mathbf{L}\).
    \item ניתן לכתוב את התנע הזוויתי האורביטלי בצורה אופרטורית - \(\mathbf{L}=\mathbf{x}\times\mathbf{p}=-i\hbar\,\mathbf{x}\times\nabla\).
    \item מקיים את היחס חילוף \([L_{i},L_{j}]=i\hbar\varepsilon_{i j k}L_{k}\), כאשר אופרטור קסימיר \(L^{2}\) מתחלף עם כל הערכים
    \item ניתן לכתוב את הרכיבים בקורדינטות קרטזיות בצורה הבאה:
$${{L_{x}=-i\hbar\left(y\frac{\partial}{\partial z}-z\frac{\partial}{\partial y}\right)}}\quad \;\; {{L_{y}=-i\hbar\left(z\frac{\partial}{\partial x}-x\frac{\partial}{\partial z}\right)}}\;\;\quad  {{L_{z}=-i\hbar\left(x\frac{\partial}{\partial y}-y\frac{\partial}{\partial x}\right)}}$$
    \item ניתן לכתוב את הרכיבים בקורדינטות כדוריות בצורה הבאה:
$$L_{z}=-i\hbar\frac{\partial}{\partial\phi}\;\;\quad L_{\pm}=L_{x}\pm i L_{y}\;\;\quad L^{2}=-\hbar^{2}\left[\frac{1}{\sin\theta}\frac{\partial}{\partial\theta}\left(\sin\theta\frac{\partial}{\partial\theta}\right)+\frac{1}{\sin^{2}\theta}\frac{\partial^{2}}{\partial\phi^{2}}\right]$$
כאשר \(L_{\pm}\) נקראים אופרטורי סולם ומשומשים להעלות ולהוריד את התנע הזוויתי.
    \item ניתן לכתוב את התנע בעזרת אופרטור קסימיר והמיקום:
$${\bf p}^{2}\psi=-\hbar^{2}\nabla^{2}\psi=-\hbar^{2}\frac{1}{r^{2}}\frac{\partial}{\partial r}\Big(r^{2}\frac{\partial\psi}{\partial r}\Big)+\frac{L^{2}}{r^{2}}\psi$$
כאשר ניתן גם לכתוב את אופרטור קסימיר בעזרת המיקום והתנע:
$${\bf L}^{2}={\bf x}^{2}{\bf p}^{2}-\left({\bf x}{\cdot}{\bf p}\right)^{2}+i\hbar{\bf x}{\cdot}{\bf p}$$
  \end{itemize}
\end{summary}
\section{בעיית אטום המימן}

\begin{definition}[אטום המימן]
האטום הפשוט ביותר. מורכב מגרעין טעון חיובית עם מסת מנוחה \(m_{p}\,=\,1.6726231\times10^{-27}\,\mathrm{kg},\) וכן אלקטרון טעון שלילית עם מסה \(m_{e}\;=\;9.1093897\,\times\,10^{-31}\,\mathrm{kg}\).

\end{definition}
\begin{proposition}[משוואת שרדינגר עבור אטום המימן]
משוואת שרדינגר לא תלויה בזמן, כלומר מהצורה:
$$\Delta\psi\left(\vec{r}\right)+\frac{2m_{0}}{\hbar^{2}}\left(E-V\left(\vec{r}\right)\right)\ \psi\left(\vec{r}\right)=0$$
הפוטנציאל יהיה מהצורה של פוטנציאל קולון:
$$V\left(\vec{r}\right)=-\frac{e_{0}^{2}}{4\,\pi\,\varepsilon_{0}\,\left|\vec{r}\right|}$$
כאשר המסה של המערכת תהיה המסה המצומצמת:
$$m_{0}=\frac{m_{p}\ \cdot\ m e}{m_{p}+m_{e}}$$

\end{proposition}
\begin{reminder}[לפלסיאן בקורדינטות כדוריות]
$$\Delta\psi=\frac{\partial^{2}\psi}{\partial r^{2}}+\frac{2}{r}\;\frac{\partial\psi}{\partial r}+\frac{1}{r^{2}}\left[\frac{1}{\sin\theta}\;\frac{\partial}{\partial\theta}\left(\sin\theta\frac{\partial\psi}{\partial\theta}\right)+\frac{1}{\sin^{2}\theta}\;\frac{\partial^{2}\psi}{\partial\;\varphi^{2}}\right]$$

\end{reminder}
\begin{proposition}[פירוק בעיית המימן בעזרת הפרדת משתנים]
אם נפריד למשתנים מהצורה:
$$\psi\left(r,\theta,\varphi\right)=R\left(r\right)\Theta\left(\theta\right)\Phi\left(\varphi\right)$$
נקבל שלושה משוואות. עבור הרכיב הרדיאלי נקבל:
$${\frac{d^{2}R}{d r^{2}}}+{\frac{2}{r}}\;{\frac{d R}{d r}}+\left({\frac{2m_{0}E}{\hbar^{2}}}+{\frac{m_{0}e_{0}^{2}}{2\pi\varepsilon_{0}\hbar^{2}r}}-{\frac{\alpha}{r^{2}}}\right)R=0$$
עבור הרכיב האזימוטלי נקבל:
$${\frac{1}{\sin\theta}}\;{\frac{d}{d\theta}}\left(\sin\theta{\frac{d\Theta}{d\theta}}\right)+\left(\alpha-{\frac{m^{2}}{\sin^{2}\theta}}\right)\Theta=0$$
עבור הרכיב הפולארי נקבל:
$$\frac{d^{2}\Phi}{d\varphi^{2}}+m^{2}\Phi=0$$
כאשר \(\alpha,m\) הם קבועים לא ידועים.

\end{proposition}
\begin{proof}
הרעיון הוא להציב את הפונקציה, להעביר אגפים כך שאגף אחד תלוי ב-\(R\) והאגף השני תלוי ב-\(\Phi,\theta\).  כעת נשוואת כל אחת מהמשוואות לקבוע, לאחר מכן להעביר שוב אגפים כך שאגף אחד תלוי ב-\(\theta\) בלבד ואגף שני תלוי ב-\(\Phi\) בלבד. כעת יש לנו שלוש המשוואות ושתי קבועים לא ידועים.

\end{proof}
\begin{proposition}[פתרון הרכיב הפולארי]
$$\Phi= C e^{ im\varphi }\qquad \varphi \in \mathbb{Z}$$

\end{proposition}
\begin{proposition}[פתרון הרכיב האזימוטלי]
הפתרון יהיה:
$$\Phi=P_{\ell}^{m}(\cos\left( \theta ) \right)\qquad  m \in \{ -\ell,-\ell+1, \dots, -1,0,1,\dots, \ell-1, \ell \}$$
כאשר:
$$\alpha=\ell\left(\ell+1\right)\qquad  \ell \in \mathbb{N} \cup \{ 0 \}$$

\end{proposition}
\begin{proof}
עבור הצבה \(\xi=\cos(\theta)\) נקבל:
$$\left(1-\xi^{2}\right)\frac{d^{2}\Theta}{d\xi^{2}}-2\xi\frac{d\Theta}{d\xi}+\left(\alpha-\frac{m^{2}}{1-\xi^{2}}\right)\Theta=0$$
כאשר זוהי משוואה לג'נדר המוכללת. קיים פתרון חסום רק כאשר \(\xi \in [-1,1]\) ו-\(\alpha\) הוא מספר שלם כך ש-\(\alpha = \ell\left( \ell+1 \right)\) כאשר \(\ell\) מספר שלם אי שלילי. ניתן לפתור בעזרת טור חזקות ולקבל את הפולינומי לג'נדר המוכללים.

\end{proof}
\begin{remark}
משתי משוואות האלה למעשה קיבלנו את הערכים עבור \(m\) ו-\(\alpha\).

\end{remark}
\begin{corollary}
ניתן לכתוב את המכפלה של החלק הזוויתי והרדיאלי בעזרת ההרמוניות הספריות \(Y_{\ell m}\left( \theta,\varphi \right)\)

\end{corollary}
\begin{proposition}[פתרון המשוואה הרדיאלית]
המשוואה הרדיאלית תהיה מהצורה:
$$R_{n}l(r)=N_{n l}\exp(-r/n a_{0})(2r/n a_{0})^{l}L_{n+l}^{2l+1}(2r/n a_{0}),$$
כאשר \(a_{0}={\frac{4\pi\epsilon_{0}\,\hbar^{2}}{m_{e}\,e^{2}}}\) נקרא רדיוס בוהר ופקטור הנרמול יהיה:
$$N_{n l}=-\sqrt{\left(\frac{2}{n a_{0}}\right)^{3}\frac{(n-l-1)!}{2n[(n+l)!]^{3}}}$$

\end{proposition}
\begin{proof}
לאחר הבצבת הביטויים עבור \(\alpha\) ו-\(m\) נקבל כי המשוואה הרדיאלית תהיה מהצורה:
$${\frac{d^{2}R}{d r^{2}}}+{\frac{2}{r}}\;{\frac{d R}{d r}}+\left({\frac{2m_{0}E}{\hbar^{2}}}+{\frac{m_{0}e_{0}^{2}}{2\pi\varepsilon_{0}\hbar^{2}r}}-{\frac{l\left(l+1\right)}{r^{2}}}\right)\;R=0$$
כעת בנוסף לפוטנציאל קולון, יש את התוספת של הפוטנציאל הצנטריפוגאלי:
$$E_{r o t}=\frac{\hbar^{2}}{2m_{0}}\frac{\ell\left(\ell+1\right)}{r^{2}}=\frac{\vec{L}^{2}}{2m_{0}r^{2}}$$
כלומר למעשה ניתן להתייחס לבעיה כאילו מסתובב חלקיק עם תנע זוויתי המקיים:
$$\left\lvert  \vec{L}  \right\rvert = \sqrt{ \ell\left( \ell+1 \right)\hbar }$$
ומומנט אינרציה של \(I=m_{0}r^{2}\)(כמו של חלקיק נקודתי המסתובב במרחק \(r\)). נסמן את סך האנרגיה להיות \(E\). נסתכל על הפתרון למשוואה בשתי מקרי קצה:

  \begin{enumerate}
    \item כאשר המרחק גדול מאוד, במקרה זה נקבל כי המשוואה תהיה מהצורה: 
$$\frac{d^{2}R}{d r^{2}}+\frac{2m_{0}E}{\hbar^{2}}\ R=0$$
ולכן הפתרון יהיה מהצורה:
$$u\sim A\exp(-\left(\sqrt{\frac{2m|E|}{\hbar^{2}}}r\right)+B\exp\left(\sqrt{\frac{2m|E|}{\hbar^{2}}}r\right)$$
כאשר \(r\to 0\) נקבל משוואה מהצורה:
$${\frac{d^{2}R}{d r^{2}}}-{\frac{\ell\left( \ell+1 \right)}{r^{2}}}R=0$$
כאשר הפתרון יהיה:
$$R = A r^{\ell+1}+Br ^{-\ell}$$
ולכן כדי שיקיים את שתי המקרי קצה נכתוב פתרון בצורה כללית על ידי:
$$C r^{l+1}\exp(-\alpha r)G(r)$$
כאשר \(G(r)\) הוא טור חזקות מהצורה \(G(r)=\sum A_{i}r^{i}\) ואם נציב במשוואה נקבל את נוסחאת הנסיגה:
$$A_{k}=-2A_{k-1}\frac{\frac{m e^{2}}{\hbar^{2}}-(l+k)\sqrt{-\frac{2m E}{\hbar^{2}}}}{(l+k)(l+k+1)-l(l+1)}$$
ולמעשה נקבל כי הפתרון מתבדר אלה אם היחס נסיגה הזה מתאפס החל מאיבר מסויים, כלומר אם מתקבל כי \(G(r)\) הוא פולינום. זה יקרה רק אם המונה ביחס נסיגה יהיה יתאפס. כלומר כאשר:
$$\frac{m e^{2}}{\hbar^{2}}=(l+k)\sqrt{-\frac{2m E}{\hbar^{2}}}.$$
ואם נגדיר \(n=\left( \ell+k \right)\) נקבל:
$$E=-{\frac{m e^{4}}{2n^{2}\hbar^{2}}}$$
הפולינומים האלה הם פולינומי לג'נדר המוכללים \(L_{q}^{p}\) כאשר אם נחבר את הכל נקבל כי התוצאה של המשוואה הרדיאלית תהיה מהצורה:
$$R_{n}l(r)=N_{n l}\exp\left( -\frac{r}{n a_{0} }\right)\left( \frac{2r}{n a_{0}} \right)^{l}L_{n+l}^{2l+1}\left( \frac{2r}{n a_{0}} \right)$$
כאשר פקטור הנרמול יהיה:
$$N_{n l}=-\sqrt{\left(\frac{2}{n a_{0}}\right)^{3}\frac{(n-l-1)!}{2n[(n+l)!]^{3}}}$$
  \end{enumerate}
\end{proof}
\begin{corollary}[רמות אנרגיה]
מההוכחה הקודמת ראינו כי הרמות האנרגיה האפשריות יהיו תלויות ב-\(n\) ומקיימות:
$$ E_{n} = -{\frac{m_{0} e_{0}^{4}}{2n^{2}\hbar^{2}}} = -\frac{e_{0}^{2}}{8\pi\varepsilon_{0}a_{0}n^{2}} = -\frac{13.6 , \mathrm{eV}}{n^{2}} $$

\end{corollary}
\begin{definition}[המספרים הקוונטים]
  \begin{itemize}
    \item הקבוע \(n \in \mathbb{N}\) נקרא המספר הקוונטי היסודי וקובע את רמת האנרגיה.
    \item הקבוע \(\ell \in \left\{  0,1,2,\dots ,n-1  \right\}\) נקרא המספר הקוונטי האזימוטלי וקובע את התנע הזוויתי האורביטלי.
    \item הקבוע \(m \in \left\{  -\ell+1,-\ell, \dots, 0, \dots\ell,\ell+1  \right\}\) נקרא המספר הקוונטי המגנטי קובע את האורינציה המרחבית של התנע הזוויתי ואת הרכיב ה-\(z\) של התנע הזוויתי \(L_{z}=m\hbar\).
  \end{itemize}
\end{definition}
\begin{remark}
הקבוע \(\ell\) קבוע את הצורה של האורביטל. לעיתים מסמנים \(\ell=0\) ב-\(s\), את \(\ell=1\) ב-\(p\), את \(\ell=2\) ב-\(d\) ואת \(\ell=3\) ב-\(f\) בהתאמה.

\end{remark}
\begin{corollary}
הפתרון הכולל של הפונקציית גל של אטום המימן נתונה על ידי:
$$ \psi_{n\ell m}\left( r, \theta, \varphi \right) = N_{n\ell} A_{\ell m}  \frac{1}{\sqrt{2\pi}} \overbrace{\left( \frac{2r}{n a_0} \right)^\ell e^{-r/na_0} L_{n-\ell-1}^{2\ell+1} \left( \frac{2r}{n a_0} \right) }^{ R(r) } \overbrace{ P_{\ell}^{|m|} \left( \cos \theta \right) }^{ \Theta\left( \theta \right) } \overbrace{ e^{im\varphi} }^{ \Phi\left( \varphi \right) } $$
כאשר \(N_{n l}=-\sqrt{\left(\frac{2}{n a_{0}}\right)^{3}\frac{(n-l-1)!}{2n[(n+l)!]^{3}}}\) הפקטור נרמול של החלק הרדיאלי\((R(r))\), \(A_{\ell m} = \sqrt{ \frac{(2\ell+1)}{2} \frac{(\ell-|m|)!}{(\ell+|m|)!} }\) הפאקטור נרמול של הגורם הפולארי \(\Theta\left( \theta \right)\) ו-\(\frac{1}{\sqrt{2\pi}}\) הפאקטור נרמול של הגורם האזימוטלי(\(\Phi\left( \varphi \right)\)).

\end{corollary}
\begin{remark}
ניתן לחלופין להשתמש בהגדרה של ההרמוניות הספריות, לשלב את \(\Theta\left( \theta \right)\Phi\left( \varphi \right)\) ולקבל \(\psi_{n\ell m}(r,\theta,\varphi)=R_{n\ell}(r)\;Y_{\ell}^{m}(\theta,\varphi),\) כאשר פאקטור הנרמול כבר נכנס בהגדרה של ההרמוניות הספריות.

\end{remark}
\begin{corollary}[ניוון אנרגיות]
באטום המימן הרמות אנרגיה תלויות רק מספר הקוונטי היסודי \(n\). לכל \(n\) יש \(n\) ערכים שונים של \(0\leq \ell \leq n-1\). עבור כל \(\ell\) יש \(2\ell+1\) ערכים שונים של \(m\). ולכן הניוון האנרגטי יהיה:
$$\sum_{\ell=0}^{n-1} (2\ell+1) = n^2$$

\end{corollary}
\begin{remark}
הניוון האנרגטי יכול להעלם תחת הפעלה של שדה מגנטי(אפקט זימן) או שדה חשמלי(אפקט שטרק).

\end{remark}
\begin{proposition}[מצב היסוד]
המצב היסוד של אטום המימן יהיה:
$$\psi_{100}(r,\theta,\varphi)=\frac{1}{\sqrt{\pi a_{0}^{3}}}\,e^{-r/a_{0}}$$

\end{proposition}
\begin{proof}
אנו יודעים כי הפונקציית גל הכוללת תהיה:
$$\psi_{n\ell m}(r,\theta,\varphi)=N_{n\ell}\,A_{\ell m}\,\frac{1}{\sqrt{2\pi}}\left(\frac{2r}{n a_{0}}\right)^{\ell}e^{-r/(n a_{0})}L_{n-\ell-1}^{\,2\ell+1}\left(\frac{2r}{n a_{0}}\right)P_{\ell}^{|m|}(\cos\theta)e^{i m\varphi},$$
כאשר המצב יסוד זה הערך המינימלי ומתקבל עבור \(n=1,\ell=0,m=0\). נציב ונקבל עבור פאקטורי הנרמול:
$$N_{10}=-\sqrt{\left(\frac{2}{a_{0}}\right)^{3}\frac{(0)!}{2\cdot1\left(1!\right)^{3}}}=-\frac{2}{a_{0}^{3/2}}\qquad A_{00}=\sqrt{\frac{2\cdot0+1}{2}\frac{(0)!}{(0)!}}=\sqrt{\frac{1}{2}}.$$
כאשר בנוסף מתקיים \(L^{1}_{0}(x)=1\) ו-\(P_{0}^{0}\left( \cos \theta \right)=1\). נציב בשאר הביטוי ונקבל את הטענה.

\end{proof}
\begin{lemma}[כללי הברירה של אלמנטי המטריצה של האופרטור Z]
אלמנטי המטריצה \(\bra{n'\ell m'}Z\ket{n\ell m}\) מקיימים:

  \begin{enumerate}
    \item מתאפסים כאשר \(m'\neq m\). 


    \item מתאפסים כאשר \(\ell'-\ell\) הוא זוגי. 


  \end{enumerate}
\end{lemma}
\begin{proof}
ראשית נפשט את הבעיה בעזרת שתי כללי ברירה:

  \begin{enumerate}
    \item אנו יודעים כי האופרטור \(Z\) בקואורדינטות ספריות יהיה \(Z=r\cos \theta\). וכן אנו יודעים כי: 
$$Y_{10}\left( \theta,\varphi \right)=\sqrt{\frac{3}{4\pi}}\cos\theta\implies Z=r\sqrt{\frac{4\pi}{3}}Y_{10}\left( \theta,\varphi \right)$$
אנו יודעים כי באופן כללי:
$$ \bra{n'l'm'}Z\ket{n\ell m} = \int \psi_{n'l'm'}^*\left( \vec{r} \right)  Z  \psi_{n\ell m}\left( \vec{r} \right) , \mathrm{d}^3r $$
כאשר אם נציב \(Z=r\cos \theta\) ו-\(\psi_{n\ell m}\left( r,\theta,\varphi \right)=R_{n\ell}(r)Y_{\ell m}\left( \theta,\varphi \right)\) נקבל:
$$ \bra{n'l'm'}Z\ket{n\ell m} \propto \int R_{n'l'}^*(r) R_{n\ell}(r) r^3 dr \int Y_{l'm'}^*\left( \theta, \varphi \right)  \cos \left( \theta \right)  Y_{\ell m}\left( \theta, \varphi \right)  \overbrace{ \sin \theta\mathrm{d}\theta \mathrm{d}\varphi }^{ \mathrm{d}\Omega } $$
עבור האינטגרל הזוויתי נקבל:
$$ I_{\text{angular}} = \int Y_{l'm'}^*\left( \theta, \varphi \right)  \cos \left( \theta \right)  Y_{\ell m}\left( \theta, \varphi \right)  \mathrm{d}\Omega = \int\limits_0^{2\pi} \!\!\!\int\limits_0^{\pi} Y_{l'm'}^*\left( \theta, \varphi \right)  \cos \left( \theta \right)  Y_{\ell m}\left( \theta, \varphi \right)  \sin \left( \theta  \right) \mathrm{d}\theta  \mathrm{d}\varphi $$
וכיוון שיש סימטריה אזימוטלית, החלק האזימוטלי מקיים מאורתוגונאליות הטור הפורייה המרוכב:
$$ I_{\varphi} = \int\limits_0^{2\pi} e^{-im'\varphi} e^{im\varphi} \mathrm{d}\varphi = \int\limits_0^{2\pi} e^{i(m-m')\varphi} \mathrm{d}\varphi =2\pi \delta_{m,m'}$$
כלומר אלמנט המטריצה מתאפס כאשר \(m\neq m'\). 


    \item נשתמש כעת בזוגיות מרחבית - כלומר איך המצב משתנה תחת היפוך במרחב. הזוגיות המרחבית של מצב \(\ket{n\ell m}\) של אטום המימן נקבע על ידי \(\ell\) כאשר הזויות תהיה \((-1)^{\ell}\). האופרטור \(Z=r\cos \theta\) הוא אי זוגי כיוון שתחת היפוך המרחב מתקיים \(r \mapsto -r\) או תחת סימטריה כדורית: 
$$r \mapsto r\qquad \theta \mapsto \pi-\theta \qquad \varphi \mapsto \varphi+\pi$$
ונקבל כעת:
$$Z=r\cos \theta \mapsto r\cos\left( \pi-\theta \right)=-r\cos \theta=-Z$$
או באופן שקול ל-\(Y_{10}\) יש זוגיות מרחבית \((-1)^{1}=-1\) ולכן גם ל-\(Z\) כיוון שפרופורציונאליים. כעת כדי שאלמנטי המטריצה \(\bra{n'\ell'm'}Z\ket{n\ell m}\) לא יהיו אפס נדרש כי האינטגרנד יהיה זוגי בסך הכל. הזוגיות של האינטגרנד יהיה מכפלת הזוגיות של \(\psi^{*}_{n'\ell'm'},Z,\psi_{n,\ell,m}\) כאשר הזוגיות של פונקציות הגל יקיימו \(\psi_{n\ell m}=(-1)^{\ell}\) ו-\(\psi^{*}_{n'\ell',m'}=(-1)^{\ell}\). לכן הזוגיות הכוללת של האינטגרנד תהיה אפס רק אם:
$$ (-1)^{\ell'} \times (-1) \times (-1)^{\ell} = +1 \implies (-1)^{\ell' + \ell + 1} = +1 \implies \ell'+\ell +1\in \mathbb{Z} / 2\mathbb{Z}\implies\ell'+\ell \in \mathbb{Z} / 2\mathbb{Z}+1$$
כלומר \(\ell'+\ell\) יהיה אי זוגי. 


  \end{enumerate}
\end{proof}
\begin{example}
נרצה למצוא את אלמנטי המטריצה של \(Z\) עבור \(n=2\). כלומר את \(\bra{2l'm'}Z\ket{2lm}\). כללי הברירה נותנים \(\Delta m=m'-m=0\) ו-\(\Delta \ell=\pm 1\) הם הערכים היחידים שלא יכולים להיות אפס. לכן יש רק שתי אלמנטי מטריצה שצריך להתייחס עליהם - \(\bra{210}Z\ket{200}\) ו-\(\bra{200}Z\ket{210}\). מתקיים:
$$ \bra{210}Z\ket{200} = \int_0^\infty R_{21}(r) R_{20}(r) r^3 dr \int Y_{10}^*(\theta, \varphi) \cos \theta Y_{00}(\theta, \varphi) d\Omega $$
כאשר האינטגרל הזוויתי נותן:
$$ \int Y_{10}^*\left( \theta, \varphi \right) \cos \left( \theta \right) Y_{00}\left( \theta, \varphi \right) d\Omega = \frac{1}{\sqrt{3}} $$
והאינטגרל הרדיאלי נותן:
$$ \int_0^\infty R_{21}(r) R_{20}(r) r^3 dr = -3\sqrt{3} a_0 $$
ולכן:
$$\bra{210}Z\ket{200} = \frac{1}{\sqrt{3}} \times (-3\sqrt{3} a_0) = -3 a_0 $$
כאשר כיוון ש-\(Z\) הרמיטי וממשי נקבל:
$$\bra{200}Z\ket{210} = (\bra{210}Z\ket{200})^* = -3 a_0 $$
כאשר כל שאר אלמנטי המטריצה הם אפס.

\end{example}
\begin{summary}
  \begin{itemize}
    \item אטום המימן הוא האטום הפשוט ביותר, ומתואר על ידי המשוואת שרודינגר עם סימטריה רדיאלית תחת פוטנציאל קולומבי:
$$ \bar{\nabla}^2 \psi\left(\vec{r}\right)+\frac{2m_{0}}{\hbar^{2}}\left(E+\frac{e_{0}^{2}}{4\pi\varepsilon_{0}\left|\vec{r}\right|}\left(\vec{r}\right)\right) \psi\left(\vec{r}\right)=0 $$
    \item נפתור על ידי הפרדת משתנים לחלק הרדיאלי, פולארי ואזימוטלי \(\psi \left(r,\theta,\varphi\right) =R\left(r\right) \Theta\left(\theta\right) \Phi\left(\varphi\right)\).
    \item הפתרון מאופיין על ידי שלושה מספרים שלמים - המספר הקוונטי היסודי \(1\leq n\) אשר קובע את רמת האנרגיה, המספר הקוונטי האזימוטלי \(0\leq \ell \leq n-1\) אשר קבוע את התנע הזוויתי האורביטלי \(|\vec{L}| = \sqrt{\ell(\ell+1)}\hbar\) והמספר הקוונטי המגנטי \(-\ell \leq m\leq \ell\) אשר קובע את הרכיב ה-\(z\) של התנע הזוויתי האורביטלי.
    \item האנרגיות האפשריות מקוונטטות ונתונות על ידי:
$$ E_{n} = -{\frac{m_{0} e_{0}^{4}}{2n^{2}\hbar^{2}}} = -\frac{e_{0}^{2}}{8\pi\varepsilon_{0}a_{0}n^{2}} = -\frac{13.6 \; \mathrm{eV}}{n^{2}} $$
    \item פונקציית הגל הכוללת נתונה על ידי:
$$ \psi_{n\ell m}\left( r, \theta, \varphi \right) = N_{n\ell} A_{\ell m} \frac{1}{\sqrt{2\pi}}  \overbrace{\left( \frac{2r}{n a_0} \right)^\ell e^{-r/na_0} L_{n-\ell-1}^{2\ell+1} \left( \frac{2r}{n a_0} \right) }^{ R(r) } \overbrace{ P_{\ell}^{|m|} \left( \cos \theta \right) }^{ \Theta\left( \theta \right) } \overbrace{ e^{im\varphi} }^{ \Phi\left( \varphi \right) } $$
כאשר \(N_{n l}=-\sqrt{\left(\frac{2}{n a_{0}}\right)^{3}\frac{(n-l-1)!}{2n[(n+l)!]^{3}}}\) הפקטור נרמול של החלק הרדיאלי\((R(r))\), \(A_{\ell m} = \sqrt{ \frac{(2\ell+1)}{2} \frac{(\ell-|m|)!}{(\ell+|m|)!} }\) הפאקטור נרמול של הגורם הפולארי \(\Theta\left( \theta \right)\) ו-\(\frac{1}{\sqrt{2\pi}}\) הפאקטור נרמול של הגורם האזימוטלי(\(\Phi\left( \varphi \right)\)).
    \item ניתן לכתוב על ידי ההרמוניות הספריות כך שפונקציית הגל תהיה מהצורה \(\psi_{n\ell m}\left( r,\theta,\varphi \right)=R_{n\ell}(r)Y_{\ell m}\left( \theta \varphi \right)\).
  \end{itemize}
\end{summary}
\chapter{שדה חשמלי ומגנטי}

\section{טרנספורמציית כיול}

\begin{proposition}
משוואות התנועה בשדה אלקטרומגנטי נתונות על ידי:
$$m{\ddot{x}}=q\,v\times B+q E$$

\end{proposition}
\begin{proof}
נכתוב את הלגרנג'יאן הקלאסי ביחידות SI:
$${\mathcal{L}}={\frac{1}{2}}m{\dot{x}}^{2}+q{\dot{x}}\cdot A-q\phi$$
מאויילר לגרנג' נקבל את משוואת התנועה:
$$m{\ddot{x}}+q{\dot{A}}=q{\frac{\partial({\dot{x}}\cdot A)}{\partial x}}-q\nabla\phi$$
ולכן:
$$\dot{\mathbf{A}}=\left({\frac{\partial}{\partial t}}+{\dot{\mathbf{x}}}\cdot\mathbf{\nabla}\right)\mathbf{A}$$
נציב במשוואת התנועה:
$$m{\ddot{x}}=q\left(\mathbf{\nabla}({\dot{x}}\cdot\mathbf{A})-({\dot{x}}\cdot\mathbf{\nabla})\mathbf{A}\right)-q\left(\mathbf{\nabla}\phi+{\frac{\partial\mathbf{A}}{\partial t}}\right)$$
כאשר אם נשתמש בזהות:
$$\begin{array}{c}{{\left[\dot{\mathbf{x}}\times\left(\mathbf{\nabla}\times\mathbf{A}\right)\right]_{i}=\epsilon_{i j k}\dot{\mathbf{x}}_{j}\left(\epsilon_{k l m}\partial_{l}\mathbf{A}_{m}\right)}}\\ {{=\left(\delta_{i l}\delta_{j m}-\delta_{i m}\delta_{j l}\right)\dot{\mathbf{x}}_{j}\partial_{l}\mathbf{A}_{m}}}\\ {{=\partial_{i}\left(\dot{\mathbf{x}}\cdot\mathbf{A}\right)-\left(\dot{\mathbf{x}}\cdot\mathbf{\nabla}\right)\mathbf{A}_{i}\,,}}\end{array}$$
ניתן לכתוב את אויילר לגרנג' על ידי:
$$m{\ddot{x}}=q\,{\dot{x}}\times(\nabla\times A)-q\left(\nabla\phi+{\frac{\partial A}{\partial t}}\right)$$
נכניס את האלקטרומגנטי והחשמלי כך שנקבל:
$$E=-\,\nabla\phi-\frac{\partial A}{\partial t}\;\;\;\;B=\nabla\times A$$
נשים לב כי שיחזרנו את משוואות התנועה בשדה אלקטרומגנטי:
$$m{\ddot{x}}=q\,v\times B+q E$$

\end{proof}
\begin{proposition}[המילטוניאן בשדה אלקטרומגנטי]
ההמילטוניאן בשדה אלקטרומגנטי יהיה מהצורה:
$$H=\frac{1}{2m}\left(p-q A\right)^{2}+q\phi$$

\end{proposition}
\begin{proof}
אנו יודעים כי ההמילטוניאן נתון על ידי:
$$H=p\cdot\dot{x}-{\mathcal{L}}$$
כאשר \(p\) זה התנע הצמוד קנוני ל-\(x\):
$$p={\frac{\partial{\mathcal{L}}}{\partial{\dot{x}}}}=m{\dot{x}}+q A$$
אם נציב את הקשר:
$${\dot{x}}={\frac{1}{m}}\left(p-q A\right)$$
נקבל:
$${{H=p\cdot\frac{1}{m}\left(p-q A\right)}} {{-\left[\frac{1}{2}m\left(\frac{1}{m}\left(p-q A\right)\right)^{2}+q\frac{1}{m}\left(p-q A\right)\cdot A-q\phi\right]}}$$
לאחר פישוט נקבל את הטענה

\end{proof}
\begin{proposition}[משוואת שרדינגר עבור חלקיק בשדה אלקטרומגנטי]
נתונה על ידי:
$$i\hbar{\frac{\partial}{\partial t}}\Psi(x,t)=\left({\frac{1}{2m}}\left(-i\hbar\mathbf{\nabla}-q\mathbf{A}\right)^{2}+q\phi\right)\Psi(x,t)$$

\end{proposition}
\begin{corollary}[משוואת הרציפות]
מתקיים:
$${\frac{\partial\rho}{\partial t}}+\nabla\cdot J=0$$
כאשר:
\begin{gather*}\rho=|\Psi(x,t)|^{2}\\ J=\frac{-i\hbar}{2m}\left[\Psi^{*}\left(\mathbf{\nabla}-\frac{iq}{\hbar}\mathbf{A}\right)\Psi-\left(\left(\mathbf{\nabla}+\frac{iq}{\hbar}\mathbf{A}\right)\mathbf{\nabla}^{*}\right)\Psi\right]=\\=\frac{-i\hbar}{2m}\left(\Psi^{*}\mathbf{\nabla}\Psi-\left( \mathbf{\nabla}\Psi^{*} \right)\Psi\right)-\frac{q}{m}\mathbf{A}\Psi^{*}\Psi 
\end{gather*}
.

\end{corollary}
\begin{definition}[טרנספורמציית כיול]
טרנספורמציה אשר ניתן לבצע על הפוטנציאל החשמלי והמגנטי ולא משנה את משוואות התנועה.

\end{definition}
\begin{proposition}
טרנספורמציית כיול מתבצעת על ידי הטרנספורמציה:
$$\begin{array}{l}{{A\mapsto A^{\prime}=A+\nabla\!\alpha(x,t)}}\\ {{\phi\mapsto\phi^{\prime}=\phi-\frac{\partial}{\partial t}\alpha(x,t)}}\end{array}$$

\end{proposition}
\begin{proposition}
כיול מוסיף נגזרת שלמה בזמן ללגרנג'יאן ולכן לא משנה את משוואות התנועה.

\end{proposition}
\begin{proof}
$$\begin{array}{c}{{{\mathcal{L}}\rightarrow{\mathcal{L}}^{\prime}=\frac{1}{2}m{\dot{x}}^{2}+q{\dot{x}}\cdot A^{\prime}-q\phi^{\prime}}}\\ {{{}={\mathcal{L}}+q{\dot{x}}\cdot\nabla\alpha+q\frac{\partial}{\partial t}\alpha(x,t)}}\\ {{{}={\mathcal{L}}+q\frac{d\alpha}{d t}}}\end{array}$$

\end{proof}
\begin{proposition}
ההמילטוניאן תחת טרנספורמציית כיול תהיה מהצורה:
$$H^{\prime}=\frac{1}{2m}\left(p-q A-q\nabla\alpha\right)^{2}+q\phi-q\frac{\partial\alpha}{\partial t}$$
אשר אומנם משתנה תחת טרנספורמציית כיול, אך משוואות התנועה הנובעות ממנה אינם משתנות.

\end{proposition}
\begin{corollary}
משוואת שרדינגר תחת טרנספורמציית כיול תהיה מהצורה:
$$i\hbar{\frac{\partial}{\partial t}}\Psi^{\prime}(x,t)=\left({\frac{1}{2m}}\left(-i\hbar\nabla-q A^{\prime}\right)^{2}+q\phi^{\prime}\right)\Psi^{\prime}(x,t)$$
כאשר \(\Psi'(x,t)\) פותר את משוואת שרדינגר בכיול החדש.

\end{corollary}
\begin{corollary}
פונקציית הגל תחת הטרנספורמציית כיול תהיה:
$$\Psi^{\prime}(x,t)=\exp\left(\frac{i q}{\hbar}\alpha(x,t)\right)\Psi(x,t)$$

\end{corollary}
\begin{corollary}
עבור טרנספורמציית כיול מתקיים:
$$|\Psi^{\prime}(x,t)|^{2}=|\Psi(x,t)|^{2}$$

\end{corollary}
\begin{summary}
  \begin{itemize}
    \item ההמילטוניאן של חלקיק בשדה האלקטרומגנטי נתון על ידי:
$$H={\frac{1}{2m}}(p-q A)^{2}+q\phi.$$
כך שמשוואת שרדינגר תהיה:
$$i\hbar{\frac{\partial}{\partial t}}\Psi(x,t)=\left[{\frac{1}{2m}}\left(-i\hbar\nabla-q A\right)^{2}+q\phi\right]\Psi(x,t)$$
    \item טרנספורמציית כיול היא טרנספורמציה מהצורה:
$$A\to A^{\prime}=A+\nabla\alpha(x,t),\quad\phi\to\phi^{\prime}=\phi-\frac{\partial}{\partial t}\alpha(x,t)$$
כאשר תחת טרנספורמציית כיול פונקציית הגל משתנה על ידי:
$$\Psi^{\prime}(x,t)=\exp\left({\frac{i q}{\hbar}}\alpha(x,t)\right)\Psi(x,t)$$
  \end{itemize}
\end{summary}
\section{בעיית לנדאו ואהרונוב-בוהם}

\begin{definition}[כיול לנדאו]
$$\phi=0\;,\quad A=-B_{0}\,y\,\hat{x}$$

\end{definition}
\begin{definition}[בעיית לאנדאו]
הבעיה של מציאת רמות האנרגיה של חלקיק בשדה מגנטי אחיד הנתון על ידי \(\mathbf{B}=B_{0}\hat{z}\).

\end{definition}
\begin{proposition}
תחת בעיית לאנדאו מתקיים:
$$B=\nabla\times A=B_{0}\hat{z}\;.$$
כך שמשוואת שרדינגר תהיה:
$$-\frac{\hbar^{2}}{2m}\left[\left(\frac{\partial}{\partial x}+i\frac{q B_{0}}{\hbar}y\right)^{2}+\frac{\partial^{2}}{\partial y^{2}}+\frac{\partial^{2}}{\partial z^{2}}\right]\Psi(\mathbf{r})=E\Psi(\mathbf{r})$$

\end{proposition}
\begin{remark}
נשים לב כי הבחירה של הכיול שוברת את הסימטריה להזזה מרחבית באופן חלקי - ההמילטוניאן תלוי כעת ב-\(y\).

\end{remark}
\begin{proposition}
הרמות אנרגיה של בעיית לנדאו נתונות על ידי
$$E_{n,k_{z}}=\hbar\omega_{B}\big(n+\frac{1}{2}\big)+\frac{\hbar^{2}k_{z}^{2}}{2m}$$
כך שהגורם הראשון נקרא רמות לנדאו. הפונקציית הגל תהיה:
$$\varphi_{n}(y)=\frac{1}{\pi^{1/4}a_{B}^{1/2}\sqrt{2^{n}n!}}\exp\left(-\frac{(y-y_{0})^{2}}{2a_{B}^{2}}\right)H_{n}\left(\frac{y-y_{0}}{a_{B}}\right)$$
כאשר \(H_{n}(x)\) הם פולינומי הרמיט ו-\(a_{B}=\sqrt{ \frac{\hbar}{m\omega_{B}} }\) הוא אורך אופייני.

\end{proposition}
\begin{proof}
ההמילטוניאן מתחלף עם \(p_{x},p_{z}\) ולכן נחפש פתרון מהצורה:
$$\Psi(\mathbf{r})=e^{i(k_{x}x+k_{z}z)}\varphi(y)$$
נציב במשוואת שרדינגר ונקבל:
$$-\frac{\hbar^{2}}{2m}\left[\left(i k_{x}+i\frac{q B_{0}}{\hbar}y\right)^{2}+\frac{\partial^{2}}{\partial y^{2}}-k_{z}^{2}\right]\varphi(y)=E\varphi(y)\;.$$
כלומר:
$$-\frac{\hbar^{2}}{2m}\varphi^{\prime\prime}(y)+\frac{\hbar^{2}}{2m}\left(k_{x}+\frac{q B_{0}}{\hbar}y\right)^{2}\varphi(y)=\left(E-\frac{\hbar^{2}k_{z}^{2}}{2m}\right)\varphi(y)$$
אם נגדיר את הקבועים:
$$y_{0}=-\frac{\hbar k_{x}}{q B_{0}}\,,\quad\omega_{B}=\frac{|q|B_{0}}{m}\,,\quad\tilde{E}=E-\frac{\hbar^{2}k_{z}^{2}}{2m}$$
נקבל את המשוואות:
$$-\frac{\hbar^{2}}{2m}\varphi^{\prime\prime}(y)+\frac{m\omega_{B}^{2}}{2}\left(y-y_{0}\right)^{2}\varphi(y)=\tilde{E}\varphi(y)$$
שזה בדיוק המשוואה של אוסצילטור הרמוני החד מימדי הממורכז סביב \(y=y_{0}\), האנרגיות יהיו:
$$\tilde{E}_{n}=\hbar\omega_{B}(n+\frac{1}{2})$$
ולכן:
$$E_{n,k_{z}}=\hbar\omega_{B}\big(n+\frac{1}{2}\big)+\frac{\hbar^{2}k_{z}^{2}}{2m}$$
כאשר פונקציית הגל תהיה הפונקציית הגל המתאימה של אוסצילטור הרמוני.

\end{proof}
\begin{remark}
המצב הקוונטי נקבע לחלוטין על ידי \(k_{x},k_{z},n\) כאשר הרמות אנרגיה נקבעות רק על ידי \(k_{z},n\), כלומר בלתי תלויות ב-\(k_{x}\). לכן כיוון ש-\(k_{x}\) רציף הניוון של כל רמת לאנדאו היא אינסופית. אם אבל למשל נתחום את הבעיה שלנו בקופסא באורך \(L\), נקבל כי \(k_{x}\) נהיה דיסקרטי:
$$k_{x}=\frac{2\pi}{L_{x}}\,n_{x},\quad n_{x}\in\mathbb{Z}\,.$$
כאשר בנוסף מוגבל על ידי האילוץ:
$$0\leq y_{0}\leq L_{y}\quad\Leftrightarrow\quad0\leq k_{x}\leq{\frac{|q|B_{0}L_{y}}{\hbar}}$$
אחרת המרכז של \(\varphi_{n}(y)\) מחוץ לקופסא התחומה. באופן שקול:
$$0\leq n_{x}\leq\frac{|q|B_{0}L_{y}L_{x}}{2\pi\hbar}=\frac{|q|\Phi_{B}}{2\pi\hbar}$$
כאשר \(\Phi_{B}=B_{0}L_{y}L_{x}\) הוא השטף המגנטי החודר את התחום שבו החלקיק נע. ולכן הניוון יהיה נתון על ידי:
$$\mathrm{degeneracy}=\frac{|q|\Phi_{B}}{2\pi\hbar}=\frac{L_{y}L_{x}}{2\pi a_{B}^{2}}\gg1$$

\end{remark}
\begin{definition}[בעיית אהרונוב בוהר]
נניח כי חלקיק מאולץ לנוע בטבעת ברדיוס \(r=b\) מסביב לציר \(\hat{z}\). כמו כן נניח כי נמקם סולנואיד אינסופי עם זרם אחיד \(I\) ברדיוס \(a< b\).

\end{definition}
\begin{lemma}
הפוטנציאל הווקטורי נתון על ידי:
$$A=\begin{cases} \frac{B_{0}r}{2}\hat{\theta} & r\leq a\\\frac{B_{0}a^{2}}{2r}\hat{\theta} & r>a\end{cases}=\begin{cases} \frac{B_{0}r}{2}\hat{\theta} & r\leq a\\\frac{\Phi_{B}}{2\pi r}\hat{\theta} & r>a
\end{cases}$$

\end{lemma}
\begin{proof}
נובע ישירות מכך שהפוטנציאל הווקטורי נתון על ידי:
$$(\nabla\times A)_{z}={\frac{1}{r}}\left({\frac{\partial(r A_{\theta})}{\partial r}}-{\frac{\partial A_{r}}{\partial\theta}}\right)={\begin{cases}B_{0}&{\mathrm{if}}\;\;r\leq a\\ 0&{\mathrm{if}}\;\;r>a\end{cases}}$$

\end{proof}
\begin{proposition}
ההמילטוניאן נתון על ידי:
$$H=\frac{1}{2m}\left[-\hbar^{2}\nabla^{2}+q^{2}A^{2}+2i\hbar q A\cdot\nabla\right]$$

\end{proposition}
\begin{corollary}
משוואת שרדינגר נתונה על ידי:
$$\frac{1}{2m}\left[-\frac{\hbar^{2}}{b^{2}}\frac{d^{2}}{d\theta^{2}}+\left(\frac{q\Phi_{B}}{2\pi b}\right)^{2}+i\frac{\hbar q\Phi_{B}}{\pi b^{2}}\frac{d}{d\theta}\right]\Psi(\theta)=E\Psi(\theta)$$

\end{corollary}
\begin{proof}
כיוון שהתנועה מאולצת על מעגל נקבל \(\nabla\rightarrow\hat{\theta}(1/b)(d/d\theta)\). נציב את ההמילטוניאן והפוטנציאל הווקטורי ונקבל את המשוואה.

\end{proof}
\begin{proposition}
פתרון בעיית אהרונוב בוהר נתון על ידי אנרגיות:
$$E_{n}=\frac{\hbar^{2}}{2m b^{2}}\left(n-\frac{q\Phi_{B}}{2\pi\hbar}\right)^{2},\quad n\in\mathbb{Z}$$
ופונקציות גל:
$$\Psi=A e^{i\lambda\theta}\qquad \lambda=\frac{q\Phi_{B}}{2\pi\hbar}\pm\frac{b}{\hbar}\sqrt{2m E}$$

\end{proposition}
\begin{proof}
ניתן לכתוב את משוואת שרדינגר בצורה:
$$\frac{d^{2}\Psi}{d\theta^{2}}+2i\beta\frac{d\Psi}{d\theta}-\epsilon\Psi=0$$
כאשר:
$$\beta=\frac{q\Phi_{B}}{2\pi\hbar}\quad\mathrm{and}\quad\epsilon=\frac{2m b^{2}E}{\hbar^{2}}-\beta^{2}$$
כאשר פונקציות הגל יהיו:
$$\Psi=A e^{i\lambda\theta}\qquad \lambda=\beta\pm\sqrt{\beta^{2}+\epsilon}=\beta\pm\frac{b}{\hbar}\sqrt{2m E}$$
ורמות אנרגיה:
$$E_{n}=\frac{\hbar^{2}}{2m b^{2}}\left(n-\frac{q\Phi_{B}}{2\pi\hbar}\right)^{2},\quad n\in\mathbb{Z}$$

\end{proof}
\begin{summary}
  \begin{itemize}
    \item בעיית לנדאו זה כשחלקיק נמצא בשדה מגנטי אחיד \(\mathbf{B}=B_{0}\hat{z}\).
    \item עבור הכיול \(\phi=0\) ו-\(A=-B_{0}y\hat{x}\) נקבל \(B=\bar{\nabla} \times A=B_{0}\hat{z}\) ומשוואת שרדינגר תהיה:
$$-\frac{\hbar^{2}}{2m}\left[\left(\frac{\partial}{\partial x}+i\frac{q B_{0}}{\hbar}y\right)^{2}+\frac{\partial^{2}}{\partial y^{2}}+\frac{\partial^{2}}{\partial z^{2}}\right]\Psi(\mathbf{r})=E\Psi(\mathbf{r}).$$
    \item הפתרון יהיה מהצורה \(\Psi(\mathbf{r})=e^{i(k_{x}x+k_{z}z)}\varphi(y),\) כאשר נקבל בעזרת משוואת שרדינגר משוואת אוסצילטור הרמוני:
$$-\frac{\hbar^{2}}{2m}\varphi^{\prime\prime}(y)+\frac{m\omega_{B}^{2}}{2}(y-y_{0})^{2}\varphi(y)=\tilde{E}\varphi(y),$$
כאשר:
$$y_{0}=-\frac{\hbar k_{x}}{q B_{0}},\quad\omega_{B}=\frac{|q|B_{0}}{m},\quad\tilde{E}=E-\frac{\hbar^{2}k_{z}^{2}}{2m}$$
    \item הרמות אנרגיה יקיימו:
$$E_{n,k_{z}}=\hbar\omega_{B}\Big(n+\frac{1}{2}\Big)+\frac{\hbar^{2}k_{z}^{2}}{2m}$$
    \item בעיית אהרונוב בוהם זה כשחלקיק נע לאורך טבעת ברדיוס \(b\) מסביב לסולנויד אינסופי ברדיוס \(a<b\).
    \item מחוץ לסולנואיד הפוטנציאל הווקטורי נתון על ידי \(A=\frac{\Phi_{B}}{2\pi r}\,\hat{\theta}\).
    \item משוואת שרדינגר תהיה:
$$H=\frac{1}{2m}\left[-\hbar^{2}\nabla^{2}+q^{2}A^{2}+2i\hbar q A\cdot\nabla\right]$$
לאחר שמאלצים תנועה בטבעת נקבל:
$$\frac{1}{2m}\left[-\frac{\hbar^{2}}{b^{2}}\frac{d^{2}}{d\theta^{2}}+\left(\frac{q\Phi_{B}}{2\pi b}\right)^{2}+i\frac{\hbar q\Phi_{B}}{\pi b^{2}}\frac{d}{d\theta}\right]\Psi(\theta)=E\Psi(\theta)$$
    \item הספקטרום אנרגיה תהיה:
$$E_{n}=\frac{\hbar^{2}}{2m b^{2}}\left(n-\frac{q\Phi_{B}}{2\pi\hbar}\right)^{2},\quad n\in\mathbb{Z},$$
כאשר פונקציות הגל יהיו:
$$\Psi=A e^{i\lambda\theta}\qquad \lambda=\frac{q\Phi_{B}}{2\pi\hbar}\pm\frac{b}{\hbar}\sqrt{2m E}$$
  \end{itemize}
\end{summary}
\section{פאזה גאומטרית}

\begin{proposition}[המשפט האדיאבטי]
מערכת פיזיקלי תשאר במצב עצמי הרגעי שלו אם הפרעה פועלת לאט מספיק ויש רווח בין הערכים עצמיים ושאר הספקטרום.
כלומר אם ב-\(t=0\) מתקיים \(H_{0}\ket{\phi}=E_{0}\ket{\phi}\) כאשר המצב ההתחלתי יהיה \(\ket{\psi(0)}=\ket{\phi}\) אז הפתרון למשוואה:
$$i\hbar{\frac{\partial}{\partial t}}|\psi(t)\rangle=H(t)|\psi(t)\rangle$$
מקיימת:
$$|\psi(t)\rangle=e^{i\theta_{n}(t)}\,e^{i\gamma_{n}(t)}\,|\phi_{n}(t)\rangle$$
כאשר \(\ket{\phi_{n}(t)}\) הוא המצב עצמי הרגעי של \(H(t)\) של המצב עצמי המשתנה \(E_{n}(t)\) המקיים:
$$H(t)\ket{\psi(t)} =E_{n}(t)\ket{\psi(t)} $$

\end{proposition}
\begin{remark}
יש הרבה משפטים אדיאבטים, זה רק אחד מהם המשויך ל-Born ו-Fock. 

\end{remark}
\begin{remark}
נדרש רק שהמצב עצמי שאנחנו מתעניינים בו לא שווה לאף מצב עצמי אחר.

\end{remark}
\begin{proof}
נניח כי מתקיים משוואת הערכים עצמיים הרגעית:
$$H(t)\left|\phi_{n}(t)\right\rangle=E_{n}(t)\left|\phi_{n}(t)\right\rangle$$
תחת הנרמול \(\langle\phi_{m}(t)|\phi_{n}(t)\rangle=\delta_{m n}\). משוואת שרדינגר התלויה בזמן נתונה על ידי:
$$i\hbar\,\frac{\partial}{\partial t}|\Psi(t)\rangle=H(t)|\Psi(t)\rangle$$
נניח פתרון מהצורה:
$$|\Psi(t)\rangle=\sum_{n}c_{n}(t)\,|\phi_{n}(t)\rangle\,e^{i\theta_{n}(t)}$$
כאשר \(\theta_{n}(t)\) נקראת הפאזה הדינאמית ומוגדרת על ידי:
$$\theta_{n}(t)=-\frac{1}{\hbar}\int_{0}^{t}E_{n}(t^{\prime})\,d t^{\prime}$$
אם נגזור את הניחוש שלנו נקבל:
$$\frac{\partial}{\partial t}|\Psi(t)\rangle=\sum_{n}\Bigl[\dot{c}_{n}(t)|\phi_{n}(t)\rangle+c_{n}(t)|\dot{\phi}_{n}(t)\rangle+i\,c_{n}(t)\,\dot{\theta}_{n}(t)|\phi_{n}(t)\rangle\Bigr]e^{i\theta_{n}(t)}$$
כיוון שמהמשפט היסודי של האינפי מתקיים \(\dot{\theta}_{n}(t)=-\frac{E_{n}(t)}{\hbar}\) נקבל לאחר הכפלה ב-\(i\hbar\):
$$i\hbar\,\frac{\partial}{\partial t}|\Psi(t)\rangle=\sum_{n}\Bigl[i\hbar\,\dot{c}_{n}(t)|\phi_{n}(t)\rangle+i\hbar\,c_{n}(t)|\dot{\phi}_{n}(t)\rangle-c_{n}(t)E_{n}(t)|\phi_{n}(t)\rangle\Bigr]e^{i\theta_{n}(t)}$$
מצד שני מתקיים ממשוואת שרדינגר התלויה בזמן זה יהיה שווה לביטוי:
$$H(t)|\Psi(t)\rangle=\sum_{n}c_{n}(t)E_{n}(t)|\phi_{n}(t)\rangle\,e^{i\theta_{n}(t)}$$
כאשר הגורמים \(E_{n}(t)\) מתבטלים ולכן:
$$\sum_{n}\Bigl[i\hbar\,\dot{c}_{n}(t)|\phi_{n}(t)\rangle+i\hbar\,c_{n}(t)|\dot{\phi}_{n}(t)\rangle\Bigr]e^{i\theta_{n}(t)}=0$$
נטיל על מצב עצמי רגעי \(\bra{\phi_{m}(t)}\) ונקבל:
$$i\hbar\,\dot{c}_{m}(t)+i\hbar\,c_{m}(t)\langle\phi_{m}(t)|\dot{\phi}_{m}(t)\rangle+i\hbar\,\sum_{n\neq m}c_{n}(t)\langle\phi_{m}(t)|\dot{\phi}_{n}(t)\rangle\,e^{i(\theta_{n}(t)-\theta_{m}(t))}=0$$
נחלק ב-\(i\hbar\) ונקבל:
$$\dot{c}_{m}(t)+c_{m}(t)\langle\phi_{m}(t)|\dot{\phi}_{m}(t)\rangle+\sum_{n\neq m}c_{n}(t)\langle\phi_{m}(t)|\dot{\phi}_{n}(t)\rangle\,e^{i(\theta_{n}(t)-\theta_{m}(t))}=0$$
כעת נבצע קירוב. בגבול האדיאבטי ההפרש פאזה:
$$\theta_{n}(t)-\theta_{m}(t)=-\frac{1}{\hbar}\int_{0}^{t}[E_{n}(t^{\prime})-E_{m}(t^{\prime})]\,d t^{\prime}$$
מבצעת אוסצילציות מהירות עבור \(n\neq m\) ולכן הערכים הלא אלכסוניים\(\left( n\neq m \right)\) שונים מ-0. נקבל תחת קירוב זה:
$$\dot{c}_{m}(t)\approx-c_{m}(t)\langle\phi_{m}(t)|\dot{\phi}_{m}(t)\rangle.$$
כאשר הפתרון של משוואה זו תהיה:
$$c_{m}(t)=c_{m}(0)\,\exp\left[-\int_{0}^{t}\langle\phi_{m}(t^{\prime})|\dot{\phi}_{m}(t^{\prime})\rangle\,d t^{\prime}\right]$$
נגדיר את הפאזה הגאומטרית על ידי:
$$\gamma_{m}(t)=i\int_{0}^{t}\langle\phi_{m}(t^{\prime})|\dot{\phi}_{m}(t^{\prime})\rangle\,d t^{\prime}$$
וניתן כעת לכתוב:
$$c_{m}(t)=c_{m}(0)\,e^{i\gamma_{m}(t)}$$

\end{proof}
\begin{definition}[פאזה גאומטרית]
$$\gamma_{m}(t)\equiv i\int_{0}^{t}\langle m|\dot{m}\rangle d t^{\prime}.$$

\end{definition}
\begin{proposition}
$$\gamma_{m}(T)=i\oint_{c}\langle m|\nabla_{R}m\rangle d{\bf R}$$

\end{proposition}
\begin{proof}
נשים לב כי:
$$0=\frac{d}{d t}(\left<m|m\right>)=\left<\dot{m}|m\right>+\left<m|\dot{m}\right>=2\mathrm{Re}\,\left(\left<m|\dot{m}\right>\right)$$
ולכן \(\gamma\) ממשי. נסתכל על מצב שבו ההמילטוניאן תלוי באוסף של פרמטרים \(\mathbf{R}(t)\). במקרה זה נקבל:
$${\frac{\partial\varphi_{m}}{\partial t}}=\nabla_{R}\varphi\cdot{\frac{d R}{d t}}$$
כאשר הפאזה הגאומטרית תהיה מהצורה:
$$\gamma_{m}(t)=i\int_{0}^{t}\langle m|\nabla_{R}m\rangle\frac{d\mathbf{R}}{d t}d t=i\int_{i}^{f}\langle m|\nabla_{R}m\rangle d\mathbf{R}.$$
כך שלאחר זמן \(T\) הההמילטוניאן חוזר לצורתו המקורית, והשינוי בפאזה הגאומטרית תהיה:
$$\gamma_{m}(T)=i\oint_{c}\langle m|\nabla_{R}m\rangle d R.$$

\end{proof}
\begin{corollary}
אם \(\mathbf{R}\) חד מימדי הפאזה הגאומרטית תהיה אפס.

\end{corollary}
\begin{example}[אהרונוב בורם]
נסתכל על האפקט של חלקיק טעין המאולץ לנועב בקופסא קטנה במרחק \(T\) מסלונאויד על ידי פוטנציאל \(V\left( \mathbf{r}-\mathbf{R} \right)\) כאשר המצבים העצמיים פותרים את משוואת שרדינגר:
$$\left[{\frac{1}{2m}}\left(-i\hbar\nabla-q A(\mathbf{r})\right)^{2}+V(\mathbf{r}-\mathbf{R})\right]\varphi_{n}=E_{n}\varphi_{n}.$$
כאשר אם נציב $$\varphi_{n}=e^{i S(\mathbf{r})}\varphi_{n}^{\prime}$$
עם $$S(\mathbf{r})={\frac{q}{\hbar}}\int_{\mathbf{R}}^{\mathbf{r}}A(\mathbf{r}^{\prime})\cdot d\mathbf{r}^{\prime}$$
אז נקבל כי \(\varphi'_{n}\) מקיימת את משוואת שרדינגר עבור \(A=0\):
$$\left[-\frac{1}{2m}\nabla^{2}+V(\mathbf{r}-\mathbf{R})\right]\varphi_{n}^{\prime}=E_{n}\varphi_{n}^{\prime}.$$
כאשר נשים לב כי \(\varphi'_{n}\) היא פונקציה של \(\mathbf{r-R}\). ניתן כעת להזיז את הקופסא לאט מסביב לסלונואיד ולקבל:
$$\nabla_{R}\varphi_{n}=\nabla_{R}\left[e^{i S}\varphi_{n}^{\prime}(\mathbf{r}-\mathbf{R})\right]=\left[-i{\frac{q}{\hbar}}A^{i S}\varphi_{n}^{\prime}(\mathbf{r}-\mathbf{R})+e^{i S}\nabla_{R}\varphi_{n}^{\prime}(\mathbf{r}-\mathbf{R})\right]$$
כאשר נמצא כי:
$$\langle n\nabla_{R}n\rangle=-i\frac{q}{\hbar}A.$$$$\gamma_{n}=\frac{q}{\hbar}\oint\mathbf{A}\cdot d\mathbf{R}=\frac{q}{\hbar}\int\mathbf{B}\cdot d\mathbf{a}=\frac{q\Phi_{B}}{\hbar}.$$

\end{example}
\section{אטום מימן בשדה מגנטי}

\begin{definition}[המילטונואין של אטום מימן בשדה מגנטי]
$$H=\frac{1}{2m_{e}}\left(p+e A\right)^{2}-e\phi(r)$$
כאשר \(\phi(r)=\frac{e}{4\pi\varepsilon r}\) הוא הפוטנציאל החשמלי ו-\(A\) הוא הפוטנציאל המגנטי.

\end{definition}
\begin{proposition}
ניתן לבחור את הכיול:
$$A(r)=-\frac{1}{2}r\times B$$
ולקבל:
$$H=H_{0}+H_{P}+H_{D}$$
כאשר \(H_{0}=\frac{p^{2}}{2m_{e}}-e\phi(r)\) השדה המגנטי של חלקיק חופשי. 
$$H_{P}=-\frac{e}{4m_{e}}\left(\mathbf{p}\cdot(\mathbf{r}\times\mathbf{B})+(\mathbf{r}\times\mathbf{B})\cdot\mathbf{p}\right)$$
יהיה השדה המגנטי הפראמגנטי - הולך כמו \(R\) ו:
$$H_{D}=\frac{e^{2}}{8m_{e}}\left(\mathbf{r}\times\mathbf{B}\right)^{2}$$
יהיה הרכיב הדיאמגנטי - הולך כמו \(R^{2}\).

\end{proposition}
\begin{proof}
\begin{gather*}\left(\mathbf{\nabla}\times\mathbf{A}\right)_{i}=-{\frac{1}{2}}\epsilon_{i j k}\partial_{j}\left(\mathbf{r}\times\mathbf{B}\right)_{k}=-{\frac{1}{2}}\epsilon_{i j k}\partial_{j}\left(\epsilon_{k l m}r_{l}B_{m}\right)=-{\frac{1}{2}}\epsilon_{i j k}\epsilon_{k l m}\delta_{j l}B_{m}\\=-\frac{1}{2}\epsilon_{i j k}\epsilon_{k j m}B_{m}=\frac{1}{2}\epsilon_{i j k}\epsilon_{m j k}B_{m}=\frac{1}{2}\left(2\delta_{i m}\right)B_{m}=B_{i} 
\end{gather*}
ונקבל כעת:
$$H=\frac{1}{2m_{e}}\left(p-\frac{e}{2}r\times B\right)^{2}-e\phi(r)$$
ונקבל אחרי פתיחת סוגריים:
$$H=\frac{p^{2}}{2m_{e}}-e\phi(r)-\frac{e}{4m_{e}}\left(\mathbf{p}\cdot(\mathbf{r}\times\mathbf{B})+(\mathbf{r}\times\mathbf{B})\cdot\mathbf{p}\right)+\frac{e^{2}}{8m_{e}}\left(\mathbf{r}\times\mathbf{B}\right)^{2}=H_{0}+H_{P}+H_{D}$$
בקירוב של שדה מגנטי חלש ניתן להזניח

\end{proof}
\begin{proposition}
החלק הפאראמגנטי יהיה שווה ל:
$$H_{P}=-M\cdot B$$
כאשר:
$$M=\frac{\mu_{B}}{\hbar}\vec{L} \qquad \mu_{B}=-\frac{e\hbar}{2m_{e}}$$

\end{proposition}
\begin{proof}
בשדה חלש ניתן להזניח את \(H_{D}\). נמצא ביטוי ל-\(H_{P}\). נשים לב כי מתקיים:
$$\mathbf{p}\cdot(\mathbf{r}\times\mathbf{B})=p_{i}\epsilon_{i j k}r_{j}B_{k}=B_{k}\epsilon_{k i j}p_{i}r_{j}=\mathbf{B}\cdot(\mathbf{p}\times\mathbf{r})$$
וכן:
$$(\mathbf{r}\times\mathbf{B})\cdot\mathbf{p}=\epsilon_{i j k}r_{j}B_{k}p_{i}=-\epsilon_{k j i}B_{k}r_{j}p_{i}=-\mathbf{B}\cdot(\mathbf{r}\times\mathbf{p})$$
ולכן ניתן לכתוב \({\bf\nabla}\cdot{\bf\nabla}p\times{\bf\nabla}r=r\times p=-L\) ולקבל:
$$H_{P}=\frac{e}{2m_{e}}L\cdot B=-\frac{\mu_{B}}{\hbar}L \cdot B$$
כאשר נזכור כי קלאסית מתקיים:
$$\mathbf{m}=I a{\hat{n}}={\frac{Q}{\Delta T}}\pi r^{2}{\hat{n}}={\frac{-e}{2\pi r/v}}\pi r^{2}{\hat{n}}=-{\frac{e v r}{2}}{\hat{n}}=-{\frac{e}{2m_{e}}}\mathbf{L}$$
כאשר \(\vec{m}\) זה המומנט דיפולי. ולכן אם נציב חזרה נקבל:
$$H_{P}=-M\cdot B$$
ואם נציב חזרה בהמילטוניאן נקבל את המבוקש.

\end{proof}
\begin{corollary}
בשדות חלשים ניתן להזניח את החלק הדיאמגנטי ולכן נקבל:
$$H\approx H_{0}+H_{P}=\frac{p^{2}}{2m_{e}}-e\phi(r)-\frac{\mu_{B}}{\hbar}L_{z}B$$

\end{corollary}
\begin{remark}
נשים לב כי ההמילטוניאן אינו סימטרי יותר לסיבובים שאינם במישור \(xy\) וכי לא מתחליף עם \(L_{x},L_{y}\), אך עדיין מתחלף עם \(L^{2},L_{Z}\).

\end{remark}
\begin{corollary}
המצבים \(\ket{n,\ell ,m}\) עדיין המצבים העצמיים של האנרגיה, ולכן:
$$H|n,l,m\rangle=\left(H_{0}+H_{P}\right)|n,l,m\rangle=\left(E_{n}^{(0)}-\mu_{B}B m\right)|n,l,m\rangle$$

\end{corollary}
\begin{proposition}
החלק הדיאמגנטי מקיים:
$$H_{D}=\frac{e^{2}}{8m_{e}}B^{2}r_{\perp}^{2}$$
כאשר \(r_{\perp}\) זה ההיטל של \(r\) על המישור הניצב לשדה המגנטי.

\end{proposition}
\begin{proof}
\begin{gather*}H_{D}=\frac{e^{2}}{8m_{e}}\left(\mathbf{r}\times\mathbf{B}\right)^{2}=\frac{e^{2}}{8m_{e}}\epsilon_{i j k}r_{j}B_{k}\epsilon_{i l m}r_{l}B_{m}=\\=\frac{e^{2}}{8m_{e}}r_{j}r_{l}B_{k}B_{m}\left(\delta_{j l}\delta_{k m}-\delta_{j m}\delta_{k l}\right)=\frac{e^{2}}{8m_{e}}\left(r_{l}r_{l}B_{k}B_{k}-r_{j}r_{l}B_{j}B_{l}\right)=\\=\frac{e^{2}}{8m_{e}}\left(r^{2}B^{2}-\left(r\cdot B\right)^{2}\right)=\frac{e^{2}}{8m_{e}}\left(B^{2}\left(r^{2}-r_{||}^{2}\right)\right)=\frac{e^{2}}{8m_{e}}B^{2}r_{\perp}^{2} 
\end{gather*}

\end{proof}
\begin{proposition}
החלק הדיאמגנטי יצאה מאינטגרציה בין המומהט במושרה לשדה החיצוני. כלומר:
$$H_{D}=\int _{0}^{B}M_{\text{ind}} \;\mathrm{d} B $$

\end{proposition}
\begin{proof}
נזכור כי התנע המכני בשדה האלקטרו מגנטי(אשר שונה מהתנע הקנוני) שווה:
$$\mathbf{\Pi}=p+e A$$
ולכן התנע הזוויתי המכני יהיה
$$\mathbf{\Lambda}=\mathbf{r}\times\mathbf{\Pi}=\mathbf{L}+e\mathbf{r}\times\mathbf{A}$$
כלומר קיימת תרמה נוספת למומנט המגנטי התלויה בשדה עצמו, זהו המומנט המגנטי המושרה
$$M_{i n d}=-\frac{e}{2m_{e}}\left(e\mathbf{r}\times\mathbf{A}\right)=\frac{e^{2}}{4m_{e}}\mathbf{r}\times\left(\mathbf{r}\times\mathbf{B}\right)=\frac{e^{2}}{4m_{e}}\left(\left(\mathbf{r}\cdot\mathbf{B}\right)\mathbf{r}-r^{2}\mathbf{B}\right)$$
כאשר העבודה הנעשית על מנת להתנגד להכנסת השדה המגנטי החיצוני היא
$$-\int_{0}^{B}M_{i n d}\cdot d B=-\frac{e^{2}}{4m_{e}}\int_{0}^{B}\left(r_{\parallel}^{2}B-r^{2}B\right)d B=\frac{e^{2}}{4m_{e}}\left(\frac{B^{2}}{2}\left(r^{2}-r_{\parallel}^{2}\right)\right)=\frac{e^{2}}{8m_{e}}B^{2}r_{\perp}^{2}$$
וזהו אכן \(H_{D}\)

\end{proof}
\begin{proposition}[סימטריית LRL]
תחת מערכת זו הגודל:
$$A=p\times L-\frac{\mu e^{2}}{r}r$$
יהיה גודל שמור זה נקרא סימטריית Laplace-Runge-Lentz

\end{proposition}
\begin{proof}
$${\frac{d}{d t}}\left(\mathbf{p}\times\mathbf{L}\right)={\dot{\mathbf{p}}}\times\mathbf{L}=\mathbf{F}\times\mathbf{L}=\left(-{\frac{e^{2}}{r^{3}}}\mathbf{r}\right)\times\left(\mathbf{r}\times\mu{\dot{\mathbf{r}}}\right)={\frac{e^{2}\mu}{r^{3}}}\left(r^{2}{\dot{\mathbf{r}}}-\left(\mathbf{r}\cdot{\dot{\mathbf{r}}}\right)\mathbf{r}\right)$$
כאשר בגלל שמתקיים
$$\mathbf{r}\cdot{\dot{\mathbf{r}}}={\frac{1}{2}}{\frac{d}{d t}}\left(\mathbf{r}\cdot\mathbf{r}\right)={\frac{1}{2}}{\frac{d}{d t}}\left(r^{2}\right)=r{\dot{\mathbf{r}}}$$
נקבל
$${\frac{d}{d t}}\left(\mathbf{p}\times\mathbf{L}\right)={\frac{e^{2}\mu}{r^{2}}}\left(r{\dot{\mathbf{r}}}-{\dot{\mathbf{r}}}\mathbf{r}\right)={\frac{d}{d t}}\left({\frac{\mu e^{2}}{r}}\mathbf{r}\right)\Longrightarrow{\frac{d\mathbf{A}}{d t}}=0$$

\end{proof}
\begin{summary}
  \begin{itemize}
    \item ההמילטוניאן של אטום המימן בשדה מגנטי נתון על ידי:
$$H=\frac{1}{2m_{e}}\left(p+e A\right)^{2}-e\phi(r)=H_{0}+H_{P}+H_{D}$$
כאשר תחת פוטנציאל קולון \(\phi(r)=\frac{3}{4\pi \epsilon r}\) ובחירת הכיול הסימטרי \(A(r)=-\frac{1}{2}\,r\times B\).
    \item הרכיב הלא מופרע יהיה המילטוניאן אטום המימן:
$$H_{0}=\frac{p^{2}}{2m_{e}}-e\phi(r)$$
    \item הרכיב הפרארמגנטי נתון הל ידי:
$$H_{P}=-{\frac{e}{4m_{e}}}\left[\mathbf{p}\cdot\left( \mathbf{r}\times\mathbf{B} \right)+\left( \mathbf{r}\times\mathbf{B} \right)\cdot\mathbf{p}\right]=-M\cdot B\qquad M=\frac{\mu_{B}}{\hbar}\vec{L}\quad\mu_{B}=-\frac{e\hbar}{2m_{e}}$$
    \item הרכיב הדיאמגנטי יהיה:
$$H_{D}=\frac{e^{2}}{8m_{e}}\left(\mathbf{r}\times\mathbf{B}\right)^{2}=\frac{e^{2}}{8m_{e}}B^{2}r_{\perp}^{2}$$
  \end{itemize}
\end{summary}
\section{תהודה מגנטית}

\begin{reminder}[פרסציה ונוטציה]
\includegraphics[width=0.8\textwidth]{diagrams/svg_1.svg}
\end{reminder}
\begin{definition}[פרסציה לרמור]
התופעה שבה המומנט המגנטי \(\vec{m}\) מבצע פרסציה מסביב לשדה מגנטי חיצוני \(\vec{B}_{0}\).

\includegraphics[width=0.8\textwidth]{diagrams/svg_2.svg}
\end{definition}
נסתכל ראשית על המקרה הקלאסי.

\begin{reminder}[מומנט מגנטי]
הגורם הדיפולי של הפוטנציאל המגנטי נתון על ידי:
$$\vec{A}\left( \vec{x} \right)={\frac{1}{2c|\vec{x}|^3}}\vec{x} \times\int\vec{x}^{\prime}\,\times\,\vec{J\left( x^{\prime} \right)}\ d^{3}x^{\prime}$$
כאשר נגדיר את המומנט דיפולי להיות:
$$\vec{m}={\frac{1}{2c}}\int\vec{x}^{\prime}\,\times\,\vec{J}\left( x^{\prime} \right)\ d^{3}x^{\prime}$$
כך שמקיים עבור שדה מגנטי חיצוני \(B_{0}\):
$$\tau=\mathbf{m}\times\mathbf{B}_{0}$$
כאשר למעשה המומנט המגנטי בעצמו יוצר שדה מגנטי הנתון על ידי:
$$\vec{B}\left( \vec{r} \right)=\frac{\mu_{0}}{4\pi}\left[ \frac{3\left( \vec{m}\cdot \vec{r} \right)\hat{r}-\vec{m}}{r^{3}} \right]$$

\end{reminder}
\begin{proposition}
המומנט המגנטי פרופורציונאלי לתנע הזוויתי:
$$\mathbf{m}=\gamma\mathbf{J}$$

\end{proposition}
\begin{reminder}
$$\frac{d\mathbf{J}}{d t}=\tau$$

\end{reminder}
\begin{corollary}[פרסציה תחת שדה מגנטי קבוע]
$${\frac{d\mathbf{J}}{d t}}=(\gamma\mathbf{J})\times\mathbf{B}_{0}$$
כלומר יש לנו תנועת פרסציה של \(\vec{J}\) סביב \(B_{0}\) עם גודל:
$$\omega_{L}=|\gamma|B_{0}$$

\end{corollary}
\begin{proposition}
אם נוסיף שדה מגנטי התלוי בזמן מהצורה:
$$\mathbf{B}_{1}(t)=B_{1}\cos(\omega t){\hat{x}}+B_{1}\sin(\omega t){\hat{y}}$$
המומנט המגנטי יקיים:
$$\frac{d{\bf m}}{d t}=\gamma{\bf m}\times\left({\bf B}_{0}+{\bf B}_{1}(t)\right).$$

\end{proposition}
\begin{proof}
ניתן לכתוב:
$$\mathbf{B}_{\mathrm{total}}=\mathbf{B}_{0}+\mathbf{B}_{1}(t),$$
כאשר \(\vec{B}_{0}=B_{0}\hat{z}\). המומנט הכולל יהיה:
$$\tau={\bf m}\times{\bf B}_{\mathrm{total}}.$$
כאשר אם \(\vec{m}=\gamma \vec{J}\) נקבל:
$${\frac{d\mathbf{m}}{d t}}=\gamma\mathbf{m}\times\mathbf{B}_{\mathrm{total}}.$$
כאשר אם נחזיר חזרה את \(\vec{B}_{\mathrm{total}}\) נקבל:
$${\frac{d\mathbf{m}}{d t}}=\gamma\mathbf{m}\times\left(\mathbf{B}_{0}+\mathbf{B}_{1}(t)\right)$$

\end{proof}
\begin{corollary}
אם נעבור למערכת מסתובבת נקבל:
$$\left({\frac{d\mathbf{m}}{d t}}\right)_{\mathrm{rot}}=\gamma \vec{m} \times \vec{B}_{\text{eff}}\qquad B_{\text{eff}}=\frac{1}{\gamma}\left(\left( \omega-\omega_{0} \right)\hat{z^{\prime}}-\omega_{1}\hat{x^{\prime}}\right)$$
כאשר \(\vec{\omega}=\omega \hat{z}\), \(\omega=\left\lvert  \gamma  \right\rvert B_{0}\) ו-\(\omega=-\gamma B_{1}\).

\end{corollary}
\begin{proof}
נגדיר את המעבר קורדינטות הבאה:
$$\begin{array}{c}{{\hat{x}^{\prime}=\cos(\omega t)\hat{x}+\sin(\omega t)\hat{y},}}\\ {{\hat{y}^{\prime}=-\sin(\omega t)\hat{x}+\cos(\omega t)\hat{y},}}\\ {{\hat{z}^{\prime}=\hat{z}.}}\end{array}$$
כאשר במערכת מסתובבת כללית מתקיים:
$$\left({\frac{d\mathbf{v}}{d t}}\right)_{\mathrm{lab}}=\left({\frac{d\mathbf{v}}{d t}}\right)_{\mathrm{rot}}+{\boldsymbol{\omega}}\times\mathbf{v},$$
כאשר \(\vec{\omega}=\omega \hat{z}\). אם נציב את הביטוי עבור מערכת המעבדה מהטענה הקודמת נקבל:
$$\left(\frac{d\mathbf{m}}{d t}\right)_{\text{rot}}=\frac{d\mathbf{m}}{d t}-\mathbf{\omega}\times\mathbf{m}=\mathbf{m}\times\left( \gamma\mathbf{B}_{0}+\gamma\mathbf{B}_{1}(t)+\mathbf{\omega} \right)\equiv\gamma\mathbf{m}\times\mathbf{B}_{e f f}$$
כאשר:
$$B_{\text{eff}}=\frac{1}{\gamma}\left(\left( \Delta w \right)\hat{z^{\prime}}-\omega_{1}\hat{x^{\prime}}\right) \qquad  \Delta \omega= \omega-\omega_{0}\qquad \omega_{1}=-\gamma B_{1}$$

\end{proof}
\begin{definition}[תדירות אפקטיבית]
\end{definition}
\begin{remark}
נשים לב כי אם \(\Delta \omega \approx 0\), כלומר \(\omega \approx |\gamma|B_{0}\) נקבל כי השדה האפקטיבי יהיה בניצב לשדה \(B_{0}\), והמומנט המגנטי ימשוך לכיוון \(-\hat{z}\). זה יהיה רסוננס והאמפלידות הפרססיה תהיה מקסימלית.

\end{remark}
\section{אינטרפרטציה קוונטית}

\begin{proposition}
ההמילטוניאן אינטרקציה יהיה שווה לאנרגיה שנובע מהמומנט המגנטי:
$$H=-\mathbf{m}\cdot\mathbf{B}_{0}=-\gamma\mathbf{J}\cdot\mathbf{B}_{0}$$
כאשר אם נניח כי \(\vec{B}_{0}\) פועל לאורך ציר \(z\) נקבל
$$H=-\gamma B_{0}J_{z}$$

\end{proposition}
\begin{corollary}
מטיעונים של הקוונטיזיציה של התנע הזוויתי שלא באמת למדנו נקבל כי האופרטור \(H\) יהיה:
$$H=-\gamma_{s p i n}{\mathbf{S}}\cdot{\mathbf{B}}\qquad\gamma_{s p i n}=g_{e}{\frac{\mu_{B}}{\hbar}}\approx2{\frac{\mu_{B}}{\hbar}}$$
כאשר \(g_{e}\) הוא קבוע.

\end{corollary}
\begin{proposition}[קידום בזמן של פרסציית לרמור]
במקרה ללא כוח מגנטי חיצוני נוסף התלוי בזמן נקבל:
$$|\psi(t)\rangle=\sum_{m}c_{m}e^{-i\omega_{L}m t}|m\rangle$$

\end{proposition}
\begin{proof}
נציב את ההמילטוניאן הקודם כאשר המצבים העצמיים של \(J_{z}\) נתונים על ידי \(\ket{m}\) והם יהיו מצבים סטציונארים בזמן עם אנרגיה:
$$E_{m}=-\gamma B_{0}m.$$
ולכן מאופרטור הקידום בזמן נקבל את הטענה.

\end{proof}
\begin{proposition}
כאשר כמו במקרה הקלאסי מפעילים שדה חיצוני תלוי בזמן, עבור חלקיקים בעל ספין \(\frac{1}{2}\) כמו אלקטרון ההמילטוניאן יהיה מהצורה:
$$H=-\gamma_{p e i n}{\mathbf{S}}\cdot{\mathbf{B}}=-\gamma_{p e i n}{\mathbf{S}}\cdot({\mathbf{B}}_{0}+{\mathbf{B}}_{1})=\omega_{0}S_{z}+\omega_{1}\left(\cos(\omega t)S_{x}+\sin(\omega t)S_{y}\right)$$

\end{proposition}
\begin{proof}
נעבור לבסיס המצבים העצמיים של \(S_{z}\). נקבל:
$$|S={\frac{1}{2}},m_{s}={\frac{1}{2}}\rangle=|\uparrow\rangle={\binom{1}{0}}\qquad|S={\frac{1}{2}},m_{s}=-{\frac{1}{2}}\rangle=|\downarrow\rangle={\binom{0}{1}}$$
כאשר אנו יודעים כי \(\vec{S}=\frac{\hbar}{2}\vec{\sigma}\). לכן אם נכתוב את ההמילטוניאן בבסיס של מטריצות פאולי נקבל:
$$H=\frac{\hbar}{2}\left(\begin{array}{c c}{{\omega_{0}}}&{{\omega_{1}\left(\cos(\omega t)-i\sin(\omega t)\right)}}\\ {{\omega_{1}\left(\cos(\omega t)+i\sin(\omega t)\right)}}&{{-\omega_{o}}}\end{array}\right)=\frac{\hbar}{2}\left(\begin{array}{c c}{{\omega_{0}}}&{{\omega_{1}e^{-i\omega t}}}\\ {{\omega_{1}e^{i\omega t}}}&{{-\omega_{0}}}\end{array}\right)$$
כעת נגדיר מצב כללי במרחב הילברט מהצורה $$|\psi(t)\rangle=c_{\uparrow}(t)|\uparrow\rangle+c_{\downarrow}(t)|\downarrow\rangle$$
כאשר נציב במשוואת שרדינגר התלוייה בזמן ונקבל:
$$\begin{pmatrix}\dot{c}_{\uparrow}(t)\\ \dot{c}_{\downarrow}(t)\end{pmatrix}=-\frac{i}{2}\begin{pmatrix}\omega_{0}&\omega_{1}e^{-i\omega t}\\ \omega_{1}e^{i\omega t}&-\omega_{0}\end{pmatrix}\begin{pmatrix}c_{\uparrow}(t)\\ c_{\downarrow}(t)\end{pmatrix}$$
ניתן לפתור זאת על ידי לכסון או לחלופין על ידי הדרך הבאה:
נגדיר את המצב המסתובב סביב ציר \(z\) באופן הבא:
$$|\psi^{\prime}(t)\rangle=e^{\frac{i}{\hbar}\omega t S_{z}}|\psi(t)\rangle$$
כאשר כעת נגזור את המצב לפי זמן:
$$i\hbar{\frac{d}{d t}}|\psi^{\prime}(t)\rangle=-\omega S_{z}e^{{\frac{i}{\hbar}}\omega t S_{z}}|\psi(t)\rangle+e^{{\frac{i}{\hbar}}\omega t S_{z}}H|\psi(t)\rangle=-\omega S_{z}|\psi^{\prime}(t)\rangle+e^{{\frac{i}{\hbar}}\omega t S_{z}}H e^{-{\frac{i}{\hbar}}\omega t S_{z}}|\psi^{\prime}(t)\rangle$$
כאשר זה יהיה שווה ל-\(H'\ket{\psi'(t)}\) עבור:
$$H^{\prime}\,=\,e^{\frac{i}{\hbar}\omega t S_{z}}H e^{-\,\frac{i}{\hbar}\omega t S_{z}}\,-\,\omega S_{z}$$
תחת סיבוב נקבל:
$$S_{x}^{\prime}=e^{\frac{i}{\hbar}\omega t S_{z}}S_{x}e^{-\frac{i}{\hbar}\omega t S_{z}}=\frac{\hbar}{2}\begin{pmatrix}e^{\frac{i}{2}\omega t}&0\\ 0&e^{-\frac{i}{2}\omega t}\end{pmatrix}\begin{pmatrix}0&1\\ 1&0\end{pmatrix}\begin{pmatrix}e^{-\frac{i}{2}\omega t}&0\\ 0&e^{\frac{i}{2}\omega t}\end{pmatrix}.$$$$=\frac{\hbar}{2}\begin{pmatrix}e^{\frac{i}{2}\omega t}&0\\ 0&e^{-\frac{i}{2}\omega t}\end{pmatrix}\begin{pmatrix}0&e^{\frac{i}{2}\omega t}\\ e^{-\frac{i}{2}\omega t}&0\end{pmatrix}=$$$${\frac{\hbar}{2}}\left(\begin{matrix}0&e^{i\omega t}\\ e^{-i\omega t}&0\end{matrix}\right)={\frac{\hbar}{2}}\left(\begin{matrix}0&\cos(\omega t)+i\sin(\omega t)\\ \cos(\omega t)-i\sin(\omega t)&0\end{matrix}\right)=S_{x}\cos(\omega t)-S_{y}\sin(\omega t)$$
וכעת:
\begin{gather*}S_{y}^{\prime}=e^{\frac{i}{\hbar}\omega t S_{z}}S_{y}e^{-\frac{i}{\hbar}\omega t S_{z}}=\frac{\hbar}{2}\begin{pmatrix}e^{\frac{i}{2}\omega t}&0\\ 0&e^{-\frac{i}{2}\omega t}\end{pmatrix}\begin{pmatrix}0&-i\\ i&0\end{pmatrix}\begin{pmatrix}e^{-\frac{i}{2}\omega t}&0\\ 0&e^{\frac{i}{2}\omega t}\end{pmatrix}\\ \frac{\hbar}{2}\left(\begin{array}{c c}{{e^{\frac{i}{2}\omega t}}}&{{0}}\\ {{0}}&{{e^{-\frac{i}{2}\omega t}}}\end{array}\right)\left(\begin{array}{c c}{{0}}&{{-i e^{\frac{i}{2}\omega t}}}\\ {{i e^{-\frac{i}{2}\omega t}}}&{{0}}\end{array}\right)=\\={\frac{\hbar}{2}}\left({\begin{array}{l l}{0}&{-i e^{i\omega t}}\\ {i e^{-i\omega t}}&{0}\end{array}}\right)={\frac{\hbar}{2}}\left({\begin{array}{l l}{0}&{-i\cos\left( \omega t \right)+\sin\left( \omega t \right)}\\ {i\cos\left( \omega t \right)+\sin\left( \omega t \right)}&{0}\end{array}}\right)=\\=S_{x}\sin\left( \omega t \right)+S_{y}\cos\left( \omega t \right)+S_{z}\sin\left( \omega t \right)+S_{z}\sin\left( \omega t \right) 
\end{gather*}
כאשר כיוון שמתקיים \(S_{z}=S'_{z}\) נקבל כי ההמילטוניאן האפקטיבי יהיה:
$$H^{\prime}=e^{\frac{i}{\hbar}\omega t S_{z}}\left(\omega_{0}S_{z}+\omega_{1}\left(\cos(\omega t)S_{x}+\sin(\omega t)S_{y}\right)\right)e^{-\frac{i}{\hbar}\omega t S_{z}}-\omega S_{z}=$$$$\omega_{0}S_{z}+\omega_{1}S_{x}\cos^{2}(\omega t)-\omega_{1}S_{y}\cos(\omega t)\sin(\omega t)+\omega_{1}S_{z}\sin^{2}(\omega t)+\omega_{1}S_{y}\cos(\omega t)\sin(\omega t)-\omega S_{z}=0,$$$$S_{z}\left(\omega_{0}-\omega\right)+S_{x}\omega_{1}=-S_{z}\Delta\omega+S_{x}\omega_{1}=\frac{\hbar}{2}\begin{pmatrix}-\Delta\omega&\omega_{1}\\ \omega_{1}&\Delta\omega\end{pmatrix}$$

\end{proof}
נלכסן את המטריצה ונקבל שהאנרגיות העצמיות יהיו:
$$E_{\pm}= \pm \frac{\hbar}{2}\sqrt{ \Delta \omega^{2}+\omega_{1}^{2} }$$
ולכן המצבים העצמיים יהיו:
$$|\psi_{+}^{\prime}\rangle=\cos\frac{\theta}{2}|\uparrow\rangle+\sin\frac{\theta}{2}|\downarrow\rangle\qquad|\psi_{-}^{\prime}\rangle=-\sin\frac{\theta}{2}|\uparrow\rangle+\cos\frac{\theta}{2}|\downarrow\rangle$$
כאשר:
$$\cos\theta=\frac{-\Delta\omega}{\sqrt{\Delta\omega^{2}+\omega_{1}^{2}}}\qquad\sin\theta=\frac{\omega_{1}}{\sqrt{\Delta\omega^{2}+\omega_{1}^{2}}}$$

\begin{corollary}[נוסחאת רבי]
ההתסברות להיות במצב \(\ket{\downarrow}\) עם ספין למטה לאחר זמן \(t\) יהיה:
$$P(t)=\frac{\omega_{1}^{2}}{\Delta\omega^{2}+\omega_{1}^{2}}\sin^{2}\left(\frac{t}{2}\sqrt{\Delta\omega^{2}+\omega_{1}^{2}}\right) $$

\end{corollary}
\begin{proof}
נקדם את המצב בזמן:
\begin{gather*}|\psi^{\prime}(t)\rangle=e^{-\frac{i}{\hbar}H^{\prime}t}|\uparrow\rangle=e^{-\frac{i}{\hbar}H^{\prime}t}\left(\cos\frac{\theta}{2}|\psi_{+}^{\prime}\rangle-\sin\frac{\theta}{2}|\psi_{-}^{\prime}\rangle\right)=\\= e^{-\frac{i}{\hbar}E_{+}t}\cos\frac{\theta}{2}|\psi_{+}^{\prime}\rangle-e^{-\frac{i}{\hbar}E_{-}t}\sin\frac{\theta}{2}|\psi_{-}^{\prime}\rangle=\\=e^{-\frac{i}{\hbar}E_{-}t}\cos\frac{\theta}{2}\left(\cos\frac{\theta}{2}|\uparrow\rangle+\sin\frac{\theta}{2}|\downarrow\rangle\right)-e^{\frac{i}{\hbar}E_{-}t}\sin\frac{\theta}{2}\left(-\sin\frac{\theta}{2}|\uparrow\rangle+\cos\frac{\theta}{2}|\downarrow\rangle\right)=\\= \left(e^{-\frac{i}{\hbar}E t}\cos^{2}{\frac{\theta}{2}}+e^{\frac{i}{\hbar}E t}\sin^{2}{\frac{\theta}{2}}\right)|\uparrow\rangle+\left(e^{-\frac{i}{\hbar}E t}\cos{\frac{\theta}{2}}\sin{\frac{\theta}{2}}-e^{\frac{i}{\hbar}E t}\sin{\frac{\theta}{2}}\cos{\frac{\theta}{2}}\right)|\downarrow\rangle 
\end{gather*}
כאשר ההסתברות למצב \(\ket{\downarrow}\) תהיה:
\begin{gather*}c_{\downarrow}=|\left\langle \psi^{\prime}(t)|\downarrow \right\rangle|^{2}=|\frac{1}{2}\sin\theta\left(-2i\sin\frac{E t}{\hbar}\right)|^{2}=\sin^{2}\theta\sin^{2}\frac{E t}{\hbar}=\\=\frac{\omega_{1}^{2}}{\Delta\omega^{2}+\omega_{1}^{2}}\sin^{2}\left(\frac{t}{2}\sqrt{\Delta\omega^{2}+\omega_{1}^{2}}\right) 
\end{gather*}

\end{proof}
\begin{remark}
מנוסחה זו ניתן לחלץ בעזרת ניסוי את \(\gamma_{\text{spin}}\).

\end{remark}
\begin{summary}
  \begin{itemize}
    \item אם פועל שדה מגנטי \(B_{0}\) ומפעילים שדה מגנטי שתלוי בזמן \(\mathbf{B}_{1}(t)=B_{1}\cos(\omega t){\hat{x}}+B_{1}\sin(\omega t){\hat{y}}\) נקבל
$$\mathbf{B}_{\mathrm{lab}}(t)=\mathbf{B}_{0}+\mathbf{B}_{1}(t)$$
    \item אם נעבור למערכת המסתובבת פועל שדה קבוע הפועל במערכת המסתובבת:
$$\mathbf{B}_{\mathrm{eff}}=\Delta\omega\hat{z}^{\prime}-\omega_{1}\hat{x}^{\prime}$$
    \item במקרה הקלאסי נקבל כי \(\vec{m}\) מבצע פרסציה מסביב ל-\(B_{\text{eff}}\). במקרה הקוונטי ערך התצפית \(\left\langle  \vec{S}  \right\rangle\) מבצע פרסציה סביב \(B_{\mathrm{eff}}\).
    \item בשתי המקרים תדירות הפרסציה תהיה \(\omega_{\mathrm{eff}}=\sqrt{ \Delta \omega^{2}+\omega_{1}^{2} }\). כאשר נזכור כי \(\omega_{\mathrm{eff}}=\gamma|\mathbf{B}_{\mathrm{eff}}|\).
    \item עבור \(\omega=\omega_{0}\) נקבל רסוננס בשתי המקרים, כאשר אמפליטדת הפרסציה מקסימלית.
  \end{itemize}
\end{summary}
\chapter{ספין}

\section{חיבור תנע זוויתי}

\begin{reminder}
כאשר התנע הזוויתי \(\vec{J}\) של מערכת נשמרת, אז הגודל שלו \(\vec{J}^{2}\) מתחלף עם ההמילטוניאן.

\end{reminder}
\begin{proposition}
לעיתים יש שתי מרחבי הילברט שונים עם תנע זוויתי \(J_{1},J_{2}\). כאשר \(J_{1}^{2}\) ו-\(J_{2}^{2}\) לא מתחלפים בפני עצמם עם ההמילטוניאן, אבל אם מסתכלים על הסכום שלהם:
$$J_{1}\otimes \mathbb{1} +\mathbb{1} \otimes J_{2}\equiv J_{3}$$
נקבל כי הגודל שלו \(J_{3}^{2}\) מתחלף עם ההמילטוניאן. ונרצה ללכסן אותו.

\end{proposition}
\begin{definition}[בסיס לא מצומד - uncoupled basis]
יהי \(J_{1},J_{2}\) אופרטורי תנע זוויתי כך ש-\(J_{1}\otimes \mathbb{1} +\mathbb{1} \otimes J_{2}\equiv J_{3}\). הבסיס הלא מצומד של \(J_{3}\) יהיה:
$$\left\{  \ket{j_{1},m_{1}} \otimes \ket{j_{2},m_{2}}   \right\}\equiv \left\{  \ket{j_{1}j_{2};m_{1}m_{2}}   \right\}$$
כאשר \(\ket{j_{1},m_{1}}\) ו-\(\ket{j_{2},m_{2}}\) הם המצבים העצמים של \(J_{1}^{2}\) ו-\(J_{2}^{2}\) בהתאמה. לעיתים נקרא "בסיס המכפלה".

\end{definition}
\begin{definition}[בסיס מצומד - coupled basis]
יהי \(J_{1},J_{2}\) אופרטורי תנע זוויתי כך ש-\(J_{1}\otimes \mathbb{1} +\mathbb{1} \otimes J_{2}\equiv J_{3}\). הבסיס המצומד של \(J_{3}\) יהיה:
$$\left\{  \ket{JM}   \right\}$$
כאשר הרבה פעמים כותבים:
$$\left\{  \ket{j_{1},j_{2};JM}   \right\}$$
כדי שיהיה אפשר לעבור בצורה הפיכה בין הבסיס הלא מצומד לבסיס המצומד.

\end{definition}
\begin{proposition}[מעבר מהבסיס המצומד ללא מצומד]
$$|j_{1}j_{2};j m\rangle=\sum_{m_{1}}\sum_{m_{2}}\langle j_{1}j_{2};m_{1}m_{2}|j_{1}j_{2}j m\rangle|j_{1}j_{2};m_{1}m_{2}\rangle$$
כאשר הטרנספורמצייה ההופכית תקיים:
$$|j_{1}j_{2};m_{1}m_{2}\rangle=\sum_{j}\sum_{m}\langle j_{1}j_{2};m_{1}m_{2}|j_{1}j_{2}j m\rangle|j_{1}j_{2};j m\rangle$$

\end{proposition}
\begin{definition}[מקדמי קלבש גורדון]
המקדמים של המעבר בסיס בין הבסיס המצומד לבסיס הלא מצומד. כלומר:
$$\langle j_{1}j_{2};m_{1}m_{2}|j_{1}j_{2}j m\rangle$$
כאשר כיוון שאנחנו מגדירים אותם כך שממשיים מתקיים:
$$\langle j_{1}j_{2};m_{1}m_{2}|j_{1}j_{2}j m\rangle=\langle j_{1}j_{2}j m|j_{1}j_{2};m_{1}m_{2}\rangle$$

\end{definition}
\begin{proposition}[כללי הברירה]
$$m=m_{1}+m_{2} \qquad |j_{1}-j_{2}|\leq j\leq j_{1}+j_{2}$$

\end{proposition}
\begin{proof}
עבור הכלל הראשון מספיק להראות:
$$\left(J_{z}-J_{1z}-J_{2z}\right)|j_{1}j_{2};j m\rangle=0$$
כאשר מכיין ש-\(J_{z}=J_{1z}+J_{2z}\). נכפיל משמאל במצב \(\bra{j_{1},j_{2};m_{1}m_{2}}\) ונקבל:
$$\left\langle j_{1}j_{2};m_{1}m_{2}|\left(J_{z}-J_{1z}-J_{2z}\right)|j_{1}j_{2};j m\right\rangle=\left(m-m_{1}-m_{2}\right)\left\langle j_{1}j_{2};m_{1}m_{2}|j_{1}j_{2}j m\right\rangle=0$$
וניתן להסיק מזה כי:
$$m-m_{1}-m_{2}=0\implies m=m_{1}+m_{2}$$

\end{proof}
\begin{proposition}[אופרטור הורדה והעלה של הסכום]
$$J_{-}\equiv J_{1-}+J_{2-}=J_{1x}-i J_{1y}+J_{2x}-i J_{2y}=(J_{1x}+J_{2x})-i\left(J_{1y}+J_{2y}\right)=J_{x}-i J_{y}$$
כאשר מתקיים באופן זהה \(J_{+}=J_{1+}+J_{2+}\). כך שמתקיים:
\begin{gather*}J_{+}|j m\rangle=\hbar\sqrt{\left(j-m\right)\left(j+m+1\right)}|j,m+1\rangle\\ J_{-}|j m\rangle=\hbar\sqrt{(j+m)\,(j-m+1)}|j,m-1\rangle 
\end{gather*}

\end{proposition}
\begin{proposition}
אופרטור המקיים:
$$P|j_{1}j_{2};m_{1}m_{2}\rangle=|j_{1}j_{2};m_{2}m_{1}\rangle$$

\end{proposition}
\section{מציאת מקדמי קלבש גורדן}

\begin{proposition}[אלגוריתם למציאת מקדמי קלבש גורדון]
  \begin{enumerate}
    \item התחלה מהמצב בעל הערכים הגבוהים ביותר של \(j\) ושל \(m\): 
    \item המצב עם \(j_{\text{max}}=j_1+j_2\) ו-\(m_{\text{max}}=j_{\text{max}}\) נלקח ישירות מהבסיס הלא מצומד:
$$|j_1j_2;j=j_{\text{max}},m=j_{\text{max}}\rangle=|j_1j_2;m_1=j_1,m_2=j_2\rangle$$


    \item הפעלת אופרטור ההורדה \(J_-\) באופן חוזר: 


    \item אופרטור ההורדה מוגדר כ:
$$J_-=J_{1-}+J_{2-}$$
    \item השתמשו בו כדי להוריד את הערך של \(m\) במצב הבא:
     $$     J_-|j_1j_2;jm\rangle=\hbar\sqrt{(j+m)(j-m+1)}|j_1j_2;j,m-1\rangle
     $$
    \item תהליך זה מייצר את כל המצבים במרחב \(j=j_{\text{max}}\).


    \item מעבר למרחב עם \(j\) נמוך יותר: 


    \item לאחר שסיימתם את מרחב \(j=j_{\text{max}}\), עברו לערך \(j=j_{\text{max}}-1\).
    \item התחילו מהמצב העליון עם \(m=j\), וכתבו אותו כקומבינציה לינארית של בסיס לא מצומד:
$$|j_1j_2;j=j_{\text{max}}-1,m=j_{\text{max}}-1\rangle=\alpha|j_1j_2;m_1=j_11,m_2=j_2\rangle+\beta|j_1j_2;m_1=j_1,m_2=j_2-1\rangle     $$


    \item חוזרים על תהליך ההורדה במרחב החדש: 


    \item השתמשו ב-\(J_-\) כדי להוריד את \(m\) וליצור את שאר המצבים במרחב \(j\) החדש.


    \item המשיכו לערכים נמוכים יותר של \(j\): 


    \item חזרו על התהליך עבור \(j=j_{\text{max}}-2,j_{\text{max}}-3,\dots,|j_1-j_2|\).


    \item בדיקת אורתוגונליות ונרמול: 


    \item ודאו שכל המצבים \(|j_1j_2;jm\rangle\) מספקים את תנאי האורתונורמליות:
$$     \langle j_1j_2;jm|j_1j_2;j'm'\rangle=\delta_{jj'}\delta_{mm'}     $$
  \end{enumerate}
\end{proposition}
\begin{example}[חיבור תנ"ז \(s=\frac{1}{2}\) עם \(l=1\)]
נתחיל מהמצב העליון שמקיים:
$$|1,{\frac{1}{2}};j={\frac{3}{2}},m={\frac{3}{2}}\rangle=|1,{\frac{1}{2}};m_{l}=1,m_{s}={\frac{1}{2}}\rangle$$
כאשר נפעיל עליו את האופרטור \(J_{-}=L_{-}+S_{-}\) ונקבל:
$$\sqrt{3}|1,\frac{1}{2};j=\frac{3}{2},m=\frac{1}{2}\rangle=\sqrt{2}|1,\frac{1}{2};m_{l}=0,m_{s}=\frac{1}{2}\rangle+\sqrt{1}|1,\frac{1}{2};m_{l}=1,m_{s}=-\frac{1}{2}\rangle$$
ולכן:
$$|1,{\frac{1}{2}};j={\frac{3}{2}},m={\frac{1}{2}}\rangle={\sqrt{\frac{2}{3}}}|1,{\frac{1}{2}};m_{l}=0,m_{s}={\frac{1}{2}}\rangle+{\sqrt{\frac{1}{3}}}|1,{\frac{1}{2}};m_{l}=1,m_{s}=-{\frac{1}{2}}\rangle$$
כאשר אם נפעיל שוב את אופרטור ההורדה נקבל:
\begin{gather*}{\sqrt{\left({\frac{3}{2}}+{\frac{1}{2}}\right)\left({\frac{3}{2}}-{\frac{1}{2}}+1\right)}}|1,{\frac{1}{2}};j={\frac{3}{2}},m=-{\frac{1}{2}}\rangle=\\ =\sqrt{\frac{2}{3}}\sqrt{2}|1,\frac{1}{2};m_{l}=-1,m_{s}=\frac{1}{2}\rangle+\sqrt{\frac{2}{3}}\sqrt{1}|1,\frac{1}{2};m_{l}=0,m_{s}=-\frac{1}{2}\rangle+ \\+\sqrt{\frac{1}{3}}\sqrt{2}|1,\frac{1}{2};m_{l}=0,m_{s}=-\frac{1}{2}\rangle 
\end{gather*}
כאשר הפעלה של \(S_{-}\) על המצב עם \(m_{s}=-\frac{1}{2}\) מאפסת אותו. נסדר את התוצאות:
$$|1,{\frac{1}{2}};j={\frac{3}{2}},m=-{\frac{1}{2}}\rangle={\sqrt{\frac{1}{3}}}|1,{\frac{1}{2}};m_{l}=-1,m_{s}={\frac{1}{2}}\rangle+{\sqrt{\frac{2}{3}}}|1,{\frac{1}{2}};m_{l}=0,m_{s}=-{\frac{1}{2}}\rangle$$

\end{example}
ניתן לחלופין למצוא את המקדמים בעזרת טבלה:

 Created with Inkscape (http://www.inkscape.org/) \includegraphics[width=0.8\textwidth]{diagrams/svg_3.svg}
\begin{summary}
השלבים למציאת מקדמי קלבש-גורדן הם:

  \begin{enumerate}
    \item להתחיל ממצב \(j=m=j_{1}+j_{2}\) כאשר מצב זה נפרס על ידי מצב אחד ממצבי המכפלה ולכן הוא שווה לו. 


    \item להפעיל את אופרטור ההורדה \(J_{-}=J_{1-}+J_{2-}\) על מצב זה \(2j\) פעמים כדי לקבל את כל המצבים בתת מרחב זה. 


    \item אם מגיעים לתת מרחב \(j=\lvert j_{1}-j_{2} \rvert\) סיימנו, אחרת נעבור לתת מרחב הבא \(j\to j-1\) ולכתוב את המצב העליון שלו \(m=j\) כקומבינציה של מצבים בבסיס המכפלה המקיימים \(m_{1}+m_{2}=m\). 


    \item נדרוש כי מצב זה יהיה אורתוגונאלי לכל שאר המצבים בבסיס הכולל בעלי אותו ערך \(m\), וכן נדרוש נרמול. 


    \item נפתור את המערכת משוואות ונקבל את המקדמים עבור המצב משלב 3. 


    \item נחזור לשלב 2. 


  \end{enumerate}
\end{summary}
\section{סימון 3j}

\begin{definition}[סימון 3j]
$$\begin{pmatrix}j_{1} & j_{2} & J \\m_{1} & m_{2} & M 
\end{pmatrix}= \frac{(-1)^{j_{1}-j_{2}-m_{2}}}{\sqrt{ 2J+1 }}\left\langle  j_{1}m_{1}j_{2}m_{2}\mid J,-M  \right\rangle \in \mathbb{R}$$

\end{definition}
\begin{remark}
המוטיבציה מאוחרי זה זה העובדה שמתקיים:
$$\sum_{m_{1}}\sum_{m_{2}}\sum_{m_{3}}\ket{j_{1},m_{1}} \ket{j_{2},m_{2}} \ket{j_{3},m_{3}} \begin{pmatrix}j_{1} & j_{2} & j_{3} \\m_{1} & m_{2} & m_{3}
\end{pmatrix}=\ket{00} $$

\end{remark}
\begin{proposition}[כללים של סימון 3j]
  \begin{enumerate}
    \item כל ערך \(m_{i}\) יכול לקבל רק את הערכים המתאים לפי כללי הברירה: 
$$m_{i}\in \left\{  -j_{i},-j_{i}+1 ,\dots ,j_{i}-1, j_{i}  \right\}$$
כלומר זה אילוץ על השורה התחתונה לפי הערך שנמצא מעל האיבר.


    \item נדרש שוב מכללי הברירה עבור השורה התחתונה \(m_{1}+m_{2}+m_{3}=0\). 


    \item מכללי הברירה נדרש: 
$$\lvert j_{1}-j_{2} \rvert \leq j_{3}\leq j_{1}+j_{2}$$


    \item מתקיים גם \(j_{1}+j_{2}+j_{3} \in \mathbb{Z}\). 


  \end{enumerate}
\end{proposition}
\begin{proposition}[חילופי עמודות]
עבור מספר זוגי של חילופי עמודות הערך לא משתנה:
$$\begin{pmatrix}j_{1} & j_{2} & j_{3} \\m_{1} & m_{2} & m_{3}\end{pmatrix}=\begin{pmatrix}j_{2} & j_{3} & j_{1} \\m_{2} & m_{3} & m_{1}
\end{pmatrix}=\dots$$
עבור מספר אי זוגי של חילופי עמודות נקבל:
$$\begin{pmatrix}j_{1} & j_{2} & j_{3} \\m_{1} & m_{2} & m_{3} \end{pmatrix}=(-1)^{j_{1}+j_{2}+j_{3}} \begin{pmatrix}j_{2} & j_{1} & j_{3} \\m_{2} & m_{1} & m_{3}
\end{pmatrix}=\dots$$

\end{proposition}
\begin{proposition}[היפוך המספרים המגנטיים]
$$\begin{pmatrix}j_{1} & j_{2} & j_{3} \\m_{1} & m_{2} & m_{3}\end{pmatrix}= (-1)^{j_{1}+j_{2}+j_{3}}\begin{pmatrix}j_{1} & j_{2} & j_{3} \\-m_{1} & -m_{2} & -m_{3}
\end{pmatrix}$$
זה נובע מהיפוך בזמן כיוון שתחת היפוך בזמן המספר המגנטי \(m\) מקבל סימן מינוס.

\end{proposition}
הרעיון בסימון \(3j\) זה שזה מאפשר לנו למצוא מקדמי קלבש גורדון בעזרת סימטריה בעזרת היחס:
$$\left\langle  j_{1}m_{1}j_{2}m_{2} \mid JM \right\rangle=(-1)^{j_{1}-j_{2}-m_{2}}\sqrt{ 2J+1 }\begin{pmatrix}j_{1} & j_{2} & J \\m_{1} & m_{2} & -M
\end{pmatrix} $$

\begin{remark}
ניתן לכתוב בעזרת סימון \(3j\) את משפט וויגנר אקרט:
$$\left\langle  j_{1}m_{1}\mid T_{q}^{(k)}\mid j_{2}m_{2}  \right\rangle =(-1)^{j_{1}-m_{1}}\begin{pmatrix}j_{1} & k & j_{2} \\-m_{1} & q & m_{2} 
\end{pmatrix}\langle j_{1}|| T^{(k)}||j_{2} \rangle $$

\end{remark}
\section{ספין}

\begin{definition}[ספין]
סוג של תנע זוויתי שלא נוצר מסיבוב, אלה מתכונה פנימית של החומר. כדי להבדיל מתנע זוויתי רגיל, במקום לסמן ב-\(J\) מסומן ב-\(S\).

\end{definition}
\begin{proposition}
לחלקיק עם ספין \(S\) יש \(2S+1\) תתי רמות של אנרגיה.

\end{proposition}
\begin{proposition}
לכל חלקיק אלמטרי יש ערך ייחודי עבור מספר קוונטי \(S\). לחלק יש מספרים שלמים (\(0,1,2,3,\dots\)) ולחלק יש חצי שלמים \(\left( \frac{1}{2},\frac{3}{2},\frac{5}{2},\dots \right)\).

\end{proposition}
\begin{definition}[פרמיונים ובוזונים]
לחלקיקים אשר יש ערכי ספין חצי שלמים נקרא פרמיונים ולחלקיקים אשר יש ספין שלם נקרא בוזונים.

\end{definition}
\begin{corollary}
לפרמיונים(כמו אלקטרון למשל) לא יכול להיות אנרגיה אפסית! גם באפס המוחלט הספין שלהם יהיה \(\frac{1}{2}\) ויהיה להם תנע זוויתי.

\end{corollary}
\begin{remark}
למעשה יש לחלקיק שתי רכיבים של תנע זוויתי. רכיב שנוצר מהסיבוב ורכיב של הספין. לכן התנע הזוויתי הכולל של המערכת תהיה הצימוד של שתי הספינים האלה \(S\otimes J\).

\end{remark}
\begin{proposition}
עבור מערכת שמכילה שתי רכיבים של תנע זוויתי \(J_{1},J_{2}\) התנע הזוויתי של המערכת הכוללת תהיה נתונה על ידי \(\sqrt{J_{3}(J_{3}+1) }\hbar\) כאשר \(J_{3}\) יכול לקבל ערכים רק מהקבוצה הבאה:
$$\left\{  \lvert J_{1} -J_{2}\rvert   , \lvert J_{1}-J_{2} \rvert +1,\dots \lvert J_{1}+J_{2} \rvert \right\}$$
כלומר התנע הזוויתי החדש יכול לקבל את ההפרש של הערכי תנע זוויתי, את הסכום שלהם, או כל ערך שלם ביניהם.

\end{proposition}
\begin{example}
עבור שתי מערכות ספין חצי(כלומר עם \(S_{1}=S_{2}=\frac{1}{2}\))  נקבל \(\lvert S_{1}-S_{2} \rvert=0\) ו-\(\lvert S_{1}+S_{2} \rvert=1\). ולכן יש רק שתי ערכים אפשריים עבור התנע הזוויתי. מצב Singlet \(S_{3}=0\) שבו שתי החלקיקים "מבטלים" אחד את השני(\(\uparrow\downarrow\)) ולכן יש רק את המצב \(M=0\), ומצב Triplet \(S_{3}=1\) שבו שתי החלקיקים מחזקים אחד את השני \(\left( \uparrow\uparrow \right)\) שבו יש שלושה ערכים אפשריים עבור \(M\) - \(M=-1,0,1\).

\end{example}
\section{אופרטורים טנזוריים}

\begin{definition}[אופרטור סקלארי]
אופרטור שלא משתנה תחת סיבובים. זה שקול ללהגיד כי האופרטור \(V\)  מתחלף עם התנע הזוויתי \([L_{i},V]=0\).

\end{definition}
\begin{proposition}[תנאי שקול לאופרטור סקלארי]
אם אופרטור \(f\) מקיים:
$${{\left[\hat{L}_{z},\,\hat{f}\right]=0}}\qquad {{\left[\hat{L}_{\pm},\,\hat{f}\right]=0}} \qquad {{\left[\hat{L}^{2},\,\hat{f}\right]=0.}}$$

\end{proposition}
\begin{definition}[אופרטורים ווקטורים]
קבוצה של שלושה אופרטורים \(V_{1},V_{2},V_{3}\) אשר מקיימות:
$$T(R)V_{i}T^{\dagger}(R)=\sum_{j=1}^{3}R_{j i}V_{j}$$
נקראות שלושת אופרטורי ווקטורים, כאשר \(T(R)=\exp\left( -i\sum_{i=1}^{3}\frac{\beta_{i}J_{i}}{\hbar} \right)\) זה הייצוג האונטירי של אופרטור הסיבוב. כלומר זהו אופרטור שתחת סיבובים מתנהגים באותו דרך כמו ווקטור מיקום. 

\end{definition}
\begin{proposition}[תנאי שקול לאופרטור ווקטורי]
ניתן להגדיר אופרטור ווקטורי באופן שקול בעזרת היחסי החילוף. כלומר אוסף \(V_{1},V_{2},V_{3}\) של אופרטורים נקראים אופרטורים ווקטרים אם מקיימות:
$$\left[\hat{L}_{i},\,\hat{V}_{j}\right]=i\,\hbar\,\epsilon_{i j k}\,\hat{V}_{k},$$

\end{proposition}
\begin{example}
פגשנו שלושה אופרטורים ווקטורים כאלה:
$$\left[\hat{L}_{i},\hat{r}_{j}\right]=i\,\hbar\,\epsilon_{i j k}\hat{r}_{k},\qquad\left[\hat{L}_{i},\hat{p}_{j}\right]=i\,\hbar\,\epsilon_{i j k}\,\hat{p}_{k},\qquad\left[\hat{L}_{i},\hat{L}_{j}\right]=i\,\hbar\,\epsilon_{i j k}\,\hat{L}_{k}$$

\end{example}
\begin{definition}[טנזורים קרטזים]
זוהי הכללה של אופרטורים ווקטורים. טנזור מסדר \(n\) יהיה אובייקט עם \(n\) אינדקסים אשר עובר טרנספורמציית סיבוב בצורה הבאה:
$$T_{ij \dots n}\mapsto R_{ii'}R_{jj'} \dots R_{nn'} T_{i' j' \dots n'}$$
כאשר \(R\) הם טרנספורמציית סיבוב.

\end{definition}
\begin{remark}
באופן כללי טנזורים קרטזים הם פריקים. כלומר ניתן לפרק אותם להצגות קטנות יותר אשר עוברים לעצמם תחת סיבוב(כלומר אינוונטרים תחת אופרטור הסיבוב).

\end{remark}
\begin{definition}[אופרטורים ספרים בלתי פריקים]
אופרטור \(T_{q}^{(k)}\) אשר עוברים טרנספורמציה תחת סיבוב בצורה הבאה:
$$T(R)T_{q}^{(k)}T^{\dagger}(R) =\sum_{q'=-k}^{k}D^{(k)}_{qq'}(R)T_{q'}^{(k)}$$
כאשר \(T(R)\) זה ההצגה האוניטרית של אופרטור הסיבוב \(R\) ו-\(D_{q'q}^{(k)}\) אלמנטי מטריצת ווגינר. זהו למעשה אוסף של \(2k+1\) אופרטורים.

\end{definition}
\begin{remark}
נשים לב כי אופרטור ספרי בלתי פריק מסדר אפס הוא אופרטור סקלרי ואופרטור ספרי בלתי פריק מסדר 1 הוא אופרטור ווקטורי.

\end{remark}
\begin{remark}
נשים לב כי הערכים של \(q\) יהיו בין \(-k\) ל-\(k\). כלומר עבור טנזור מסדר 0 יתקיים \(q=0\) כאשר טנזור מסדר 1 נקבל כי האפשרויות עבור \(q\) יהיו \(q=-1,0,1\).

\end{remark}
\begin{example}[ההרמוניות הספריות]
ניתן להסתכל על הטנזור הספרי:
$$T_{q}^{(k)}(\vec{V})=Y_{l=k}^{m=q}(\vec{V})$$
כאשר \(Y_{l=k}^{m=q}\left( \vec{V} \right)\) זה למעשה כמו \(Y_{l}^{m}\left( \theta,\phi \right)\) כאשר ניתן להציג את הזווית בתור ווקטור יחידה על ספרת היחידה ולכתוב \(Y_{l}^{m}\left( \hat{n} \right)\) ואז להחליף את \(\hat{n}\) באופרטור ולקבל \(Y_{l}^{m}\left( \vec{V} \right)\). בפרט עבור \(k=1\) נקבל:
$$T_{0}^{(1)}=V_{z}\quad,\quad T_{\pm1}^{(1)}=\mp\frac{1}{\sqrt{2}}(V_{x}\pm i V_{y})$$
כלומר אוסף של ווקטורים \(V_{x},V_{y},V_{z}\) יכולים להפוך לטנזורים ספרים לפי היחס לעיל. ניתן גם לקבל את היחס ההפוך:
$$V_{x}=\frac{1}{\sqrt{2}}(T_{-1}^{(1)}-T_{1}^{(1)})\quad,\quad V_{y}=\frac{i}{\sqrt{2}}(T_{-1}^{(1)}+T_{1}^{(1)})\quad,\quad V_{z}=T_{0}^{(1)}$$

\end{example}
\begin{remark}
זה בעצם מסביר למה אנחנו אוהבים את הבסיס הפולארי. זה כיוון בהם התנע הזוויתי הוא טנזור ספרי בלתי פריק.

\end{remark}
\begin{example}[דיאד dyadic]
אופרטור המוגדר על ידי \(W_{ij}=U_{i}V_{j}\) כאשר \(1\leq i,j\leq 3\) יהיה דיאד. יש עבורו סה"כ \(3\times 3 = 9\) רכיבים בלתי תלויים. זהו למעשה הצגה פריקה. ניתן לכתוב:
$$U_{i}V_{j}={\frac{\vec{U}\cdot\vec{V}}{3}}\delta_{i j}+{\frac{(U_{i}V_{j}-U_{j}V_{i})}{2}}+\left({\frac{U_{i}V_{j}+U_{j}V_{i}}{2}}-{\frac{\vec{U}\cdot\vec{V}}{3}}\delta_{i j}\right)$$
כאשר האיבר הראשון יהיה טנזור ספרי עם \(k=0\). הרכיב הזה הוא טנזור סקלארי ולכן מכיל רכיב אחד בלתי תלוי . הגורם השני יהיה טנזור ספרי עם \(k=1\). זהו טנזור ווקטורי ולכן מכיל שלושה רכיבים בלתי תלויים והגורם השלישי יהיה טנזור ספרי עם \(k=2\) ולכן מכיל 5 רכיבים בלתי תלויים. אכן קיבלנו כי המימדים של הפירוקים הבלתי פריקים מתאימים:
$$1+3+5=9$$
ניתן כעת לכתוב את הרכיבים הבלתי פריקים בצורה הבאה:
\begin{gather*}T_{0}^{(0)}=-\frac{\vec{U}\cdot\vec{V}}{3}=\frac{U_{1}V_{-1}+U_{-1}V_{1}-U_{0}V_{0}}{3}\\ T_{q}^{(1)}=\frac{\left( \vec{U}\times\vec{V} \right)_{q}}{i\sqrt{2}}\quad,\quad q=-1,0,1\\ T_{\pm2}^{(2)}=U_{\pm1}V_{\pm1}\\ T_{\pm1}^{(2)}=\frac{U_{\pm1}V_{0}+U_{0}V_{\pm1}}{\sqrt{2}}\\ T_{0}^{(2)}=\frac{U_{1}V_{-1}+U_{-1}V_{1}+2U_{0}V_{0}}{\sqrt{6}} 
\end{gather*}

\end{example}
\begin{proposition}[תנאי שקול לאופרטור ספרי בלתי פריק]
אוסף של \(2k+1\) אופרטורים \(A^{\left( \lambda \right)}\) כאשר \(-k\leq \lambda \leq k\) אשר מקיימים:
$$\left[J_{z},A_{\mu}^{(\lambda)}\right]=\hbar\mu A_{\mu}^{(\lambda)}\quad\left[J_{\pm},A_{\mu}^{(\lambda)}\right]=\hbar\sqrt{\left(\lambda\mp\mu\right)\left(\lambda\pm\mu+1\right)}A_{\mu\pm1}^{(\lambda)}$$

\end{proposition}
\begin{proposition}
אם \(X_{q}^{(k_{1})}\) ו-\(Z_{q_{2}}^{(k_{2})}\) הם טנזורים ספרים בלתי פריקים מדגרות \(k_{1}\) ו-\(k_{2}\) בהתאמה נקבל:
$$T_{q}^{(k)}=\sum_{q_{1}=-k_{1}}^{k_{1}}\sum_{q_{2}=-k_{2}}^{k_{2}}\langle k_{1}k_{2};q_{1}q_{2}|k_{1}k_{2};k q\rangle X_{q_{1}}^{(k_{1})}Z_{q_{2}}^{(k_{2})}$$
יהיה טנזור ספרי בלתי פריק. כאשר \(\left\langle  k_{1}k_{2};q_{1}q_{2}|k_{1}k_{2};kq  \right\rangle\) הם מקדמי קלבש גורדן. ניתן גם להפוך את הטענה ולקבל:
$$X_{q_{1}}^{(k_{1})}Z_{q_{2}}^{(k_{2})}=\sum_{k=|k_{1}-k_{2}|}^{k_{1}+k_{2}}\sum_{q=-k}^{k}\langle k_{1}k_{2};k q|k_{1}k_{2};q_{1}q_{2}\rangle T_{q}^{(k)}$$

\end{proposition}
\begin{summary}
  \begin{itemize}
    \item אופרטור סקלארי הוא אופרטור \(V\) אשר אינווריאנטי תחת סיבוב, כלומר מתחלף עם כל הרכיבים של התנע הזוויתי \([L_{i},V]=0\).
    \item אופרטור ווקטורי הוא אוסף של שלושה אופרטורים \(V_{x},V_{y},V_{z}\) אשר תחת סיבוב עוברים כמו ווקטור:
$$T(R)V_{i}\,T^{\dagger}(R)=\sum_{j=1}^{3}R_{j i}V_{j}.$$
או באופן שקול מקיימות \([L_{i},V_{j}]=i\hbar\epsilon_{ijk}V_{k}\) כאשר אופרטורי המיקום, התנע והתנע זוויתי הם כאלה.
    \item אופרטורים טנזורים ספרים בלתי פריקים נם אוסף של \(2k+1\) טנזורים \(T^{(k)}_{q}\) אשר עוברים סיבוב בצורה הבאה:
$$T(R)\,T_{q}^{(k)}\,T^{\dagger}(R)=\sum_{q^{\prime}=-k}^{k}D_{q q^{\prime}}^{(k)}(R)\,T_{q^{\prime}}^{(k)}$$
לדוגמא האוסף:
$$T_{0}^{(1)}=V_{z},\quad T_{\pm1}^{(1)}=\mp\frac{1}{\sqrt{2}}(V_{x}\pm i V_{y}).$$
    \item עבור שתי טנזורים ספרים \(X^{(k_{1})}\) ו-\(Z^{(k_{2})}\) ניתן להרכיב טנזור ספרי בלתי פריק חדש בעזרת מקדמי קלבש גורדון:
$$T_{q}^{(k)}=\sum_{q_{1},q_{2}}\langle k_{1}\,k_{2};q_{1}\,q_{2}|k_{1}\,k_{2};k\,q\rangle\,X_{q_{1}}^{(k_{1})}Z_{q_{2}}^{(k_{2})}.$$
  \end{itemize}
\end{summary}
\section{וויגנר-אקארט}

\begin{theorem}[ווגנר-אקרט]
יהי \(T_{q}^{(k)}\) טנזור ספרי. אלמנטי המטריצה של הטנזור ביחס לבסיס התנע יקיים:
$$\langle\alpha^{\prime},j^{\prime}m^{\prime}|T_{q}^{(k)}|\alpha,j m\rangle=\langle j k;m q|j k;j^{\prime}m^{\prime}\rangle\frac{\langle\alpha^{\prime},j^{\prime}||T^{(k)}||\alpha,j\rangle}{\sqrt{2j+1}}$$
כאשר הרכיב \(\langle\alpha^{\prime},j^{\prime}||T^{(k)}||\alpha,j\rangle\) נקרא אלמנטי המטריצה המצומצמת וזה סימון שמדגיש את זה שלא תלוי ביתר הגורמים(\(m,m',q\)).

\end{theorem}
\begin{remark}
ניתן להשתמש במשפט זה כדי לחשב את \(\langle\alpha^{\prime},j^{\prime}m^{\prime}|T_{q}^{(k)}|\alpha,j m\rangle\) עבור שילובים שונים של \(m,m',q'\) כאשר יודעים את התוצאה רק עבור אחד מהם.

\end{remark}
\begin{corollary}
מכללי הברירה של מקדמי קלבש גורדון כדי שהמקדמים לא יהיו אפס נדרש:
\begin{gather*}m^{\prime}=q+m\quad\iff\quad\Delta m\equiv m^{\prime}-m=q\\ |j-k|\leq j^{\prime}\leq j+k\quad\iff\quad|\Delta j|\equiv|j^{\prime}-j|\leq k 
\end{gather*}

\end{corollary}
\begin{example}[אטום מימן עם שדה חשמלי]
נתון אטום מימן במצב \(\ket{n=3,\ell=2,m}\) אשר מופעל עליו שדה חשמלי \(E\hat{z}\). בעזרת חישוב ישיר ניתן למצוא:
$$\braket{ 310 | Ez|320 } =E\int_{0}^{2\pi}  \, d\varphi \int_{0}^{\pi} \sin \theta \, d\theta Y_{20}  \overbracket{ \cos\left( \theta \right) }^{ z }Y_{10}^{*}\braket{ 3 | 3 } =\dots=\frac{3E}{\sqrt{ 15 }}$$
כיוון שמתקיים:
$$z=\cos\theta=\sqrt{\frac{4\pi}{3}}\,Y_{10}(\theta,\varphi)=\sqrt{\frac{4\pi}{3}}\,T_{0}^{(1)}$$
ניתן לכתוב:
$$E\,z=E\sqrt{\frac{4\pi}{3}}\,T_{0}^{(1)}$$
כעת בעזרת וויגנר אקארט מתקיים:
$$\langle3,1,m^{\prime}|T_{0}^{(1)}|3,2,m\rangle=\langle2,1;m,0|1,m^{\prime}\rangle\,\frac{\langle3,1||T^{(1)}||3,2\rangle}{\sqrt{5}}$$
ולכן:
$$\langle3,1,m^{\prime}|E\,z|3,2,m\rangle=E\sqrt{\frac{4\pi}{3}}\,\langle2,1;m,0|1,m^{\prime}\rangle\,\frac{\langle3,1||T^{(1)}||3,2\rangle}{\sqrt{5}}$$
עבור המקרה של \(m=0,m'=0\) נקבל:
$$\langle3,1,0|E\,z|3,2,0\rangle=E\sqrt{\frac{4\pi}{3}}\,\langle2,1;0,0|1,0\rangle\,\frac{\langle3,1||T^{(1)}||3,2\rangle}{\sqrt{5}}$$
כאשר ניתן למצוא בעזרת טבלה כי \(\langle2,1;0,0|1,0\rangle=-\sqrt{\frac{1}{10}}\).  נדרוש שיוויון בין הערך שהתקבל מחישוב ישיר לערך שמתקבל מוויגנר אקארט כדי למצוא את אלמנט המטריצה המצומצם:
$$E{\sqrt{\frac{4\pi}{3}}}\left(-{\sqrt{\frac{1}{10}}}\right){\frac{\langle3,1||T^{(1)}||3,2\rangle}{\sqrt{5}}}={\frac{3E}{\sqrt{15}}}\implies\left\langle {3},{1}||T^{(1)}||{3},{2} \right\rangle=-\frac{{3}\sqrt{10}}{2\sqrt{\pi}}$$
ולכן הביטוי עבור אלמנט כללי נתון על ידי:
$$\langle3,1,m^{\prime}|E\,z|3,2,m\rangle=-E\sqrt{\frac{4\pi}{3}}\,\langle2,1;m,0|1,m^{\prime}\rangle\,\frac{3\sqrt{10}}{2\sqrt{\pi}\sqrt{5}}$$

\end{example}
\begin{corollary}[טנזור סקלארי]
עבור טנזור מסדר אפס, כלומר מהצורה \(T_{0}^{(0)}=S\) נקבל:
$$\langle\alpha^{\prime},j^{\prime}m^{\prime}|S|\alpha,j m\rangle=\delta_{j j^{\prime}}\delta_{m m^{\prime}}\frac{\langle\alpha^{\prime}j^{\prime}||S||\alpha j\rangle}{\sqrt{2j^{\prime}+1}}$$
כיוון ש-\(S\) פועל על \(\langle  \alpha,jm|\) כמו הוספה של תנע זוויתי אפס, ולכן אופרטור סקלארי לא יכול לשנות את הערכים של \(j,m\)

\end{corollary}
\begin{proof}
אנו יודעים ממשפט וויגנר אקארט כי:
$$\langle\alpha^{\prime},j^{\prime},m^{\prime}|T_{q}^{(k)}|\alpha,j,m\rangle=\langle j\,k;m\,q|j^{\prime},m^{\prime}\rangle\,\frac{\langle\alpha^{\prime},j^{\prime}||T^{(k)}||\alpha,j\rangle}{\sqrt{2j+1}}$$
כאשר עבור אופרטור סקלארי \(k=0\). ולכן גם ה-\(q\) המותר היחיד יהיה \(q=0\). כאשר המקדם קלבש ג'ורדון \(\langle j 0;m 0|j',m' \rangle\) הוא לא אפס אם"ם \(j'=j\) וגם \(m'=m\). למעשה ניתן לכתוב:
$$\langle j\,0;m\,0|j,m\rangle=\frac{1}{\sqrt{2j+1}}$$
כאשר אם נציב משפט נקבל:
$$\langle\alpha^{\prime},j,m|S|\alpha,j,m\rangle=\frac{1}{\sqrt{2j+1}}\,\frac{\langle\alpha^{\prime},j||S||\alpha,j\rangle}{\sqrt{2j+1}}=\frac{\langle\alpha^{\prime},j||S||\alpha,j\rangle}{2j+1}.$$
כאשר בעזרת הכתיבה הסטנרטית ניתן לספוח \(\frac{1}{\sqrt{ 2j+1 }}\) לתוך הגורם המוצומצם ולקבל:
$$\langle\alpha^{\prime},j^{\prime},m^{\prime}|S|\alpha,j,m\rangle=\delta_{j j^{\prime}}\,\delta_{m m^{\prime}}\,\frac{\langle\alpha^{\prime},j^{\prime}||S||\alpha,j\rangle}{\sqrt{2j^{\prime}+1}}$$

\end{proof}
\begin{example}
נמצא את \(\langle r^{2} \rangle\) עבור כל הארבע מצבים המנוונים עם \(n=2\) של אטום המימן. מהטענה נקבל עבור המצבים עם \(\ell=1\) כי:
$$\left\langle2\,1\,1\,{|}\,r^{2}\,{|}\,2\,1\,1\right\rangle=\left\langle2\,1\,0\,{|}\,r^{2}\,{|}\,2\,1\,0\right\rangle=\left\langle2\,1\,-1\,{|}\,r^{2}\,{|}\,2\,1\,-1\right\rangle\equiv\left\langle2\,1\,{\big|\big|}\,r^{2}\,{\big|\big|}\,2\,1\right\rangle$$
כאשר כדי לחשב את אלמנט המטריצה המצומצם מספיק לבחור כל אחד מהערכים האלה:
$$\left\langle2\,1\,{\big|\big|}\,r^{2}\,{\big|\big|}\,2\,1\right\rangle=\left\langle2\,1\,0\left|\,r^{2}\,\right|\,2\,1\,0\right\rangle=\int r^{2}\;|\psi_{210}(r)|^{2}\;d^{3}\mathbf{r}=\int_{0}^{\infty}r^{4}~|R_{21}(r)|^{2}~d r\,\int\Big|Y_{1}^{0}(\theta,\phi)\Big|^{2}~d\Omega.$$
כיוון שההרמוניות הספריות מנורמלות האינטגרל הזוויתי יהיה 1. כאשר עבור הרכיב הרדיאלי נקבל:
$$\langle2\,1\,\|\,r^{2}\,\left\|\,2\,1\right\rangle=\int_{0}^{\infty}r^{4}\,\frac{1}{24\,a^{3}}\,\frac{r^{2}}{a^{2}}\,e^{-r/a}\,d r=30\,a^{2}$$
זה קבוע את שלושת ערכי התצפית, כאשר ערך התצפית האחרון יהיה:
\begin{gather*}\langle 2\,0\|r^{2}\,\|\,2\,0 \rangle=\langle 2\,0\,0\,|r^{2}\,|\,2\,0\,0 \rangle=\int r^{2}\;|\psi_{210}(r)|^{2}\;d^{3}\mathbf{r}=\\=\int_{0}^{\infty}r^{4}~|R_{21}(r)|^{2}~d r\,\int\left|Y_{1}^{0}(\theta,\phi)\right|^{2}\,d\Omega= \int_{0}^{\infty}r^{4}\,{\frac{1}{2\,a^{3}}}\left(1-{\frac{1}{2}}\,{\frac{r}{a}}\right)^{2}\,e^{-r/a}\,d r= 42a^{2}
\end{gather*}
כעת עבור למשל המצב \(|\psi\rangle={\frac{1}{\sqrt{2}}}\;(|200\rangle-i\;|211\rangle)\) ניתן לכתוב:
\begin{gather*}\langle\psi\left|\left.r^{2}\right|\psi\right\rangle=\frac{1}{2}\left(\left\langle200\right|+i\left\langle211\right|\right)\,r^{2}\left(\left|200\right\rangle-i\,\left|211\right\rangle\right)= \\={\frac{1}{2}}\,(\left\langle200\left|\,r^{2}\,\right|200\right\rangle+i\,\left\langle211\left|\,r^{2}\,\right|200\right\rangle-i\,\left\langle200\left|\,r^{2}\,\right|211\right\rangle\left.+\left\langle211\left|\left.r^{2}\right|211\right\rangle\right\rangle\right.
\end{gather*}
כאשר מהטענה שתיים מאלמנטי המטריצה מתאפסים ולכן:
$$\left\langle\psi\left|\,r^{2}\,\right|\,\psi\right\rangle=\frac{1}{2}\left(\left\langle20\left\|\,r^{2}\,\right\|\,20\right\rangle+\left\langle21\left\|\,r^{2}\,\right\|\,21\right\rangle\right)=36\,a^{2}.$$

\end{example}
\begin{corollary}[משפט ההטלה]
עבור אופרטור ווקטורי \(V_{q}\) מתקיים:
$$\langle\alpha^{\prime},j m^{\prime}|V_{q}|\alpha,j m\rangle=\frac{\langle\alpha^{\prime},j m|\mathbf{J}\cdot\mathbf{V}|\alpha,j m\rangle}{\hbar^{2}j(j+1)}\langle j m^{\prime}|J_{q}|j m\rangle$$
כאשר \(J_{q}\) הם הרכיבים של ההצגה הספרית של התנע הזוויתי:
$$J_{\pm1}=\mp\frac{1}{\sqrt{2}}(J_{x}\pm i J_{y})=\mp\frac{1}{\sqrt{2}}J_{\pm},\quad J_{0}=J_{z}.$$

\end{corollary}
\begin{proof}
ממשפט וויגנר עבור הטנזורים הספרים \(V_{q}\) ו-\(J_{q}\) מתקיים:
$$\langle\alpha^{\prime},j,m^{\prime}|V_{q}|\alpha,j,m\rangle=\langle j,1;m,q|j,m^{\prime}\rangle\,\frac{\langle\alpha^{\prime},j||V||\alpha,j\rangle}{\sqrt{2j+1}}$$$$\langle j,m^{\prime}|J_{q}|j,m\rangle=\langle j,1;m,q|j,m^{\prime}\rangle\,\frac{\langle j||J||j\rangle}{\sqrt{2j+1}}$$
כאשר ניתן להראות כי:
$$\langle j||J||j\rangle=\hbar\sqrt{j(j+1)(2j+1)}.$$
כעת נסתכל על המכפלה הסקלרית \(\mathbf{J}\cdot \mathbf{V}\). מתקיים:
$$\langle\alpha^{\prime},j,m|\mathbf{J}\cdot\mathbf{V}|\alpha,j,m\rangle={\frac{\langle\alpha^{\prime},j||V||\alpha,j\rangle\,\langle j||J||j\rangle}{\sqrt{2j+1}}}\,{\frac{1}{\sqrt{2j+1}}},$$
כלומר:
$$\langle\alpha^{\prime},j||V||\alpha,j\rangle=\frac{\langle\alpha^{\prime},j,m|\mathbf{J}\cdot\mathbf{V}|\alpha,j,m\rangle}{\hbar\,\sqrt{j(j+1)(2j+1)}}$$
ולכן ניתן להשוואת בין הערכים ולקבל:
$$\langle\alpha^{\prime},j,m^{\prime}|V_{q}|\alpha,j,m\rangle=\frac{\langle\alpha^{\prime},j,m|\mathbf{J}\cdot\mathbf{V}|\alpha,j,m\rangle}{\hbar^{2}\,j(j+1)}\,\langle j,m^{\prime}|J_{q}|j,m\rangle$$

\end{proof}
\begin{example}[כללי ברירה אופרטור ווקטורי]
יהי \(\mathbf{V}\) אופרטור ווקטורי. נסתכל על \(\hat{V}_{\pm}\equiv\hat{V}_{x}\pm i\,\hat{V}_{y}\). נקבל את היחסי חילוף:
\begin{gather*}{{\left[\hat{L}_{z},\,\hat{V}_{z}\right]=0}}\qquad {{\left[\hat{L}_{z},\,\hat{V}_{\pm}\right]=\pm\,\hbar\,\hat{V}_{\pm}}} \qquad {{\left[\hat{L}_{\pm},\,\hat{V}_{\pm}\right]=0}}\\ {{\left[\hat{L}_{\pm},\,\hat{V}_{z}\right]=\mp\,\hbar\,\hat{V}_{\pm}}}\qquad {{\left[\hat{L}_{\pm},\,\hat{V}_{\mp}\right]=\pm\,2\,\hbar\,\hat{V}_{z}}} 
\end{gather*}
עבור למשל \(\left[\hat{L}_{z},\,\hat{V}_{\pm}\right]=\pm\,\hbar\,\hat{V}_{\pm}\) נקבל עבור אלמנטי המטריצה:
$$\left\langle n^{\prime}\,\ell^{\prime}\,m^{\prime}\,\right|\,\hat{L}_{z}\,\hat{V}_{\pm}\,\left|\,n\,\ell\,m\right\rangle-\left\langle n^{\prime}\,\ell^{\prime}\,m^{\prime}\,\right|\,\hat{V}_{\pm}\,\hat{L}_{z}\,\left|\,n\,\ell\,m\right\rangle=\pm\hbar\left\langle n^{\prime}\,\ell^{\prime}\,m^{\prime}\,\right|\,\hat{V}_{\pm}\,\left|\,n\,\ell\,m\right\rangle$$
וכיוון שהמצבים העצמיים שלנו הם המצבים עצמיים של \(L_{z}\) נקבל:
$$\left[m^{\prime}-(m\pm1)\right]\,\left\langle n^{\prime}\,\ell^{\prime}\,m^{\prime}\,\right|\,\hat{V}_{\pm}\,\left|\,n\,\ell\,m\right\rangle=0$$
כלומר אם נקבל כי אלמנטי המטריצה מתאפסים כל עוד לא מתקיים \(m'=\pm 1\). ניתן להראות באופן דומה עבור \(V_{z}\) ולקבל:
$$\begin{array}{l l}{{\left<n^{\prime}\,\ell^{\prime}\,m^{\prime}\,\right|\,\hat{V}_{+}\,\left|\,n\,\ell\,m\right>=0}}&{{\ \ \mathrm{unless~}m^{\prime}=m+1}}\\ {{\left<n^{\prime}\,\ell^{\prime}\,m^{\prime}\,\right|\,\hat{V}_{z}\,\left|\,n\,\ell\,m\right>=0}}&{{\ \ \mathrm{unless~}m^{\prime}=m}}\\ {{\left<n^{\prime}\,\ell^{\prime}\,m^{\prime}\,\right|\,\hat{V}_{-}\,\left|\,n\,\ell\,m\right>=0}}&{{\ \ \mathrm{unless~}m^{\prime}=m-1.}}\end{array}$$

\end{example}
\begin{remark}
למעשה כאשר אנחנו במרחב עם \(j\) זהה, אז אופרטור ווקטורי \(\mathbf{V}\) נקבע לחלוטין על ידי ההטלה שלו על \(\mathbf{J}\).

\end{remark}
\begin{example}[שימוש של ווגנר-אקרט למציאת כללי ברירה]
נסתכל על האופרטור:
$$V=a\,\hat{X}\hat{Y}\;,$$
כאשר \(\hat{X},\hat{Y}\) הם רכיבי המיקום של אופרטור \(\mathbf{\hat{R}}\). אנו יודעים כי באופן כללי מתקיים:
$$U_{x}V_{y}=\frac{i}{2}(T_{-2}^{(2)}-T_{2}^{(2)}+\sqrt{2}\ T_{0}^{(1)})$$
כאשר \(T^{(2)}_{\pm 2}\) הם רכיבים של טנזור ספרי בלתי פריק מסדר 2 ו-\(T_{0}^{(1)}\) הרכיב של טנזור ספרי בלתי פריק מסדר 1(אופרטור ווקטורי).
עבור המקרה שלנו \(U=V=\mathbf{\hat{R}}\) ולכן \(U_{x}=\hat{X}\) ו-\(V_{y}=\hat{Y}\) הרכיב \(T_{0}^{(1)}\) הוא רכיב אנטי סימטרי ולכן מתבטל(נזכור כי \(T_{q}^{(1)}=\frac{\left( \vec{U}\times\vec{V} \right)_{q}}{i\sqrt{2}}\) ובפרט כיוון ש-\(\mathbf{U}=\mathbf{V}\) מתאפס). לכן נקבל:
$$XY=\frac{i}{2}(T_{-2}^{(2)}-T_{2}^{(2)})$$
מהדרישה של הכללי ברירה של קלבש גורדון נקבל מהגורם הראשון מווגנאר אקרט:
$$\langle\alpha^{\prime},j^{\prime},m^{\prime}|T_{-2}^{(2)}|\alpha,j,m\rangle=\langle j\,2;m\,(-2)|j^{\prime},m^{\prime}\rangle\,\frac{\langle\alpha^{\prime},j^{\prime}||T^{(2)}||\alpha,j\rangle}{\sqrt{2j+1}},$$
ולכן מכללי הברירה של מקדמי קלבש גורדון נקבל:
$$\Delta m = -2\quad \left\lvert  \Delta l  \right\rvert \leq 2$$
כאשר עבור הגורם השני נקבל מווגנר אקארט כי:
$$\langle\alpha^{\prime},j^{\prime},m^{\prime}|T_{2}^{(2)}|\alpha,j,m\rangle=\langle j\,2;m\,2|j^{\prime},m^{\prime}\rangle\,\frac{\langle\alpha^{\prime},j^{\prime}||T^{(2)}||\alpha,j\rangle}{\sqrt{2j+1}},$$
ולכן מכללי הברירה של מקדמי קלבש גורדון נקבל:
$$\Delta m=2,\,|\Delta l|\leq2$$
כאשר מטעמי זוגיות כיוון שלאופרטור \(xy\) יש זוגיות חיובית, נדרש \(\left\lvert  \Delta l  \right\rvert=0,2,4,\dots\)
ולכן סה"כ נקבל כי מתקיים:
$$|\Delta m|=2\qquad|\Delta l|=0,2$$

\end{example}
\begin{example}
נמצא את הערך המינימלי של \(j\) במצב עצמי של תנע זוויתי כולל \(\ket{jm}\) ואת הערכים העצמיים כך שבאופן כללי ערך תצפית של אופרטור \(O_{m_{0}}^{j_{0}}\) בעל ספין \(j_{0}\) לא יתאפס. מוויגנר אקארט כיוון שהאלמנט מטריצה מצומצם לא אפס מתקיים:
$$m=m+m_{0}\ ,\quad|j_{0}-j|\leq j\leq j+j_{0}\ .$$
ולכן \(m_{0}=0\) ו-\(j\geq \frac{j_{0}}{2}\) כך ש-\(j_{\min}=\frac{j_{0}}{2}\).

\end{example}
\begin{summary}
  \begin{itemize}
    \item משפט ווגינר אקארט אומר כי ניתן לכתוב את אלמנטי המטריצה של טנזור ספרי בלתי פריק על ידי מכפלה של מקדמי קלבש גורדון בגורם שלא תלוי ב-\(m,m',q\) בצורה הבאה:
$$\langle\alpha^{\prime},j^{\prime},m^{\prime}|T_{q}^{(k)}|\alpha,j,m\rangle=\langle j\,k;m\,q|j^{\prime}\,m^{\prime}\rangle\,\frac{\langle\alpha^{\prime},j^{\prime}||T^{(k)}||\alpha,j\rangle}{\sqrt{2j+1}}$$
    \item אלמנטי המטריצה של קלבש גורדון גוררים כי:
$$m^{\prime}=m+q\quad\mathrm{and}\quad|j-k|\leq j^{\prime}\leq j+k.$$
    \item אופרטור סקלארי אינו יכול לשנות את המספרים הקוונטים ולכן אלמנטי המטריצה שלו הם לא אפס רק כאשר \(j'=j\) ו-\(m'=m\) ונתון על ידי:
 $$\langle\alpha^{\prime},j^{\prime}m^{\prime}|S|\alpha,j m\rangle=\delta_{j j^{\prime}}\delta_{m m^{\prime}}\frac{\langle\alpha^{\prime}j^{\prime}||S||\alpha j\rangle}{\sqrt{2j^{\prime}+1}}$$
    \item אופרטור ווקטורי \(\mathbf{V}\) נקבע ביחידות על ידי ההטלה שלו על אופרטור התנע הזוויתי בבסיס הפלארי בצורה הבאה:
 $$\langle\alpha^{\prime},j m^{\prime}|V_{q}|\alpha,j m\rangle=\frac{\langle\alpha^{\prime},j m|\mathbf{J}\cdot\mathbf{V}|\alpha,j m\rangle}{\hbar^{2}j(j+1)}\langle j m^{\prime}|J_{q}|j m\rangle$$
  \end{itemize}
\end{summary}
\chapter{תורת ההפרעות והמבנה הדק}

\section{לא מנוון לא תלוי בזמן}

\begin{definition}[תורת ההפרעות]
שיטה המאפשרת לנו למצוא את הפתרון בצורה מקורבת של מערכת עם המילטוניאן מהצורה:
$$H= H_{0}+\lambda V\qquad \lambda\ll 1$$
כאשר אנחנו יודעים את הפתרון של מערכת עם ההמילטוניאן \(H_{0}\). כלומר זוהי שיטה למצוא איך מערכות אשר דומות למערכות שאנחנו מכירים מתקדמות בזמן, בלי שנצטרך לדעת לחשב במפורש.

\end{definition}
נפתור כעת בצורה כללית יותר ואז נחזור להפרעה מסדר ראשון שלנו.

\begin{symbolize}
ניתן לפתח כטור טיילור את הגורמים לפי \(\lambda\):
\begin{gather*}{{\hat{H}}}={{\hat{H}^{(0)}+\lambda\hat{H}^{(1)}+\lambda^{2}\hat{H}^{(2)}+\cdots}}\\ {{\psi_{n}}}={{\psi_{n}^{(0)}+\lambda\psi_{n}^{(1)}+\lambda^{2}\psi_{n}^{(2)}+\cdots}}\\{{E_{n}}}={{E_{n}^{(0)}+\lambda E_{n}^{(1)}+\lambda^{2}E_{n}^{(2)}+\cdots}}
\end{gather*}
כאשר נקרא ל-\(E_{n}^{(k)}\) לתיקון מסדר \(k\) של האנרגיה ול-\(\psi_{n}^{(k)}\) לתיקון מסדר \(k\) של הפונקציית גל. כאשר עבור המקרה שלנו נסמן \(H^{(1)} \equiv V\) וכן עבור \(i\geq 2\) נקבל \(H^{(i)}=0\).

\end{symbolize}
\begin{proposition}[תיקון לאנרגיה]
כאשר אנחנו במערכת לא מנוונת ולא תלויה בזמן מתקיים:
$$\begin{aligned}\hat{H}^{(0)}\psi_{n}^{(0)} & =E_{n}^{(0)}\psi_{n}^{(0)} \\\left( \hat{H}^{(0)}-E_{n}^{(0)} \right)\psi_{n}^{(1)} & =\left( E_{n}^{(1)}-\hat{V} \right)\psi_{n}^{(0)} \\\left( \hat{H}^{(0)}-E_n^{(0)} \right)\psi_n^{(2)} & =\left( E_n^{(2)}-\hat{H}^{(2)} \right)\psi_n^{(0)}+\left( E_n^{(1)}-\hat{V} \right)\psi_n^{(1)}\\&\dots
\end{aligned}$$

\end{proposition}
\begin{proof}
נרצה לפתור את הבעית ערכים עצמיים \(H\psi_{n}=E_{n}\psi_{n}\) או לחלופין \((H-E_{n})\psi_{n}=0\).
נכתוב את האנרגיה וההמילטוניאן כטור ונאחד את כל הגורמים לפי החזקה של \(\lambda\):
\begin{gather*}\left\{ \hat{H}^{(0)}\psi_{n}^{(0)}\,-\,E_{n}^{(0)}\psi_{n}^{(0)} \right\}+\\+\lambda\left\{ \hat{H}^{(0)}\psi_{n}^{(1)}+\hat{H}^{(1)}\psi_{n}^{(0)}-E_{n}^{(0)}\psi_{n}^{(1)}-E_{n}^{(1)}\psi_{n}^{(0)} \right\}+\\+\lambda^{2}\left\{ \hat{H}^{(0)}\psi_{n}^{(2)}+\hat{H}^{(1)}\psi_{n}^{(1)}+\hat{H}^{(2)}\psi_{n}^{(0)}-E_{n}^{(0)}\psi_{n}^{(2)}-E_{n}^{(1)}\psi_{n}^{(1)}-E_{n}^{(2)}\psi_{n}^{(0)} \right\}+\\+\dots = 0
\end{gather*}
מיחידות הטור חזקות נדרש כי כל גורם יתאפס, ולכן:
$$\begin{aligned}\hat{H}^{(0)}\psi_{n}^{(0)} & =E_{n}^{(0)}\psi_{n}^{(0)} \\\left( \hat{H}^{(0)}-E_{n}^{(0)} \right)\psi_{n}^{(1)} & =\left( E_{n}^{(1)}-\hat{H}^{(1)} \right)\psi_{n}^{(0)} \\\left( \hat{H}^{(0)}-E_n^{(0)} \right)\psi_n^{(2)} & =\left( E_n^{(2)}-\hat{H}^{(2)} \right)\psi_n^{(0)}+\left( E_n^{(1)}-\hat{V} \right)\psi_n^{(1)}\\&\dots
\end{aligned}$$

\end{proof}
\begin{symbolize}
נסמן כעת \(\psi_{n}^{(k)}\equiv \ket{n^{(k)}}\).

\end{symbolize}
\begin{proposition}[תיקון ראשון לאנרגיה]
כאשר אנחנו במערכת לא מנוונת ולא תלויה בזמן נקבל:
$$E_{n}^{(1)}=\langle n^{(0)}|\hat{V}|n^{(0)}\rangle$$

\end{proposition}
\begin{proof}
נסתכל על המשוואה:
$$(\hat{H}^{(0)}-E_{n}^{(0)})|n^{(1)}\rangle=(E_{n}^{(1)}-\hat{V})|n^{(0)}\rangle$$
כאשר נכפיל משמאל ב-\(\bra{n^{(0)}}\) ונקבל:
$$\begin{aligned}\langle n^{(0)}|(\hat{H}^{(0)}-E_{n}^{(0)})|n^{(1)}\rangle & =\langle n^{(0)}|(E_{n}^{(1)}-\hat{V})|n^{(0)}\rangle \\\langle n^{(0)}|\hat{H}^{(0)}|n^{(1)}\rangle-E_{n}^{(0)}\langle n^{(0)}|n^{(1)}\rangle & =E_{n}^{(1)}\langle n^{(0)}|n^{(0)}\rangle-\langle n^{(0)}|\hat{V}|n^{(0)}\rangle \\E_n^{(0)}\langle n^{(0)}|n^{(1)}\rangle-E_n^{(0)}\langle n^{(0)}|n^{(1)}\rangle & = E_n^{(1)}-\langle n^{(0)}|\hat{V}|n^{(0)}\rangle \\\mathrm{0} & = E_n^{(1)}-\langle n^{(0)}|\hat{V}|n^{(0)}\rangle
\end{aligned}$$
כאשר השתמשנו בזה שמתקיים:
$$\langle n^{(0)}|\hat{H}^{(0)}|n^{(1)}\rangle=\langle(\hat{H}^{(0)}n^{(0)})|n^{(1)}\rangle=\langle(E_{n}^{(0)}n^{(0)})|n^{(1)}\rangle=E_{n}^{(0)}\langle n^{(0)}|n^{(1)}\rangle$$
ולכן:
$$E_{n}^{(1)}=\langle n^{(0)}|\hat{V}|n^{(0)}\rangle$$

\end{proof}
\begin{example}[חישוב תיקון מסדר ראשון של פוטנציאל הרמוני מוזז]
נחשב את התיקון מסדר ראשון של אוסילטור הרמוני אשר מרכז הפוטנציאל מוזז מ-0 ל-\(\ell\). ההמילטוניאן הלא מוזז של אוסילטור הרמוני הוא כידוע:
$$\hat{H}^{(0)}=-\frac{\hbar^{2}}{2m}\frac{d^{2}}{d x^{2}}+\frac{1}{2}k\hat{x}^{2}$$
כאשר ההמילטוניאן מוזז של האוסילטור ההרמוני יהיה:
$$\begin{array}{r c l}{{\hat{H}}}&=&{{-\frac{\hbar^{2}}{2m}\frac{d^{2}}{d x^{2}}+\frac{1}{2}k(\hat{x}-l)^{2}}}\\ {{}}&{{=}}&{{-\frac{\hbar^{2}}{2m}\frac{d^{2}}{d x^{2}}+\frac{1}{2}k\hat{x}^{2}-l k\hat{x}+l^{2}\frac{1}{2}k}}\\ {{}}&{{=}}&{{\hat{H}^{(0)}+l\hat{V}+l^{2}\hat{H}^{(2)}}}\end{array}$$
כאשר הגדרנו את הגורם אשר המקדם של \(l\) תהיות \(V\equiv-k\hat{x}\) ו-\(H^{(2)}=\frac{1}{2}k\). נשים לב כי \(H^{(2)} = \frac{1}{2}k\) הוא גורם מסדר שני ב-\(l\) ואינו תלוי באופרטור \(\hat{x}\). לכן לפי הטענה הקודמת הקירוב מסדר ראשון של האנרגיה תהיה:
$$E_{n}^{(1)}=\langle n^{(0)}|\hat{V}|n^{(0)}\rangle=-k\langle n^{(0)}|\hat{x}|n^{(0)}\rangle$$
כאשר אנו יודעים כי עבור אוסילטור הרמוני בבסיס המספר מתקיים \(\left\langle  n^{(0)}|\hat{x}  |n^{(0)}\right\rangle=0\) ולכן התיקון מסדר ראשון יהיה אפס.

\end{example}
\begin{proposition}[תיקון מסדר ראשון של הפונקציית גל]
עבור מערכת לא מנוונת ולא תלויה בזמן נקבל:
$$|n^{(1)}\rangle=\sum_{k\neq n}|k^{(0)}\rangle\frac{\langle k^{(0)}|\hat{V}|n^{(0)}\rangle}{E_{n}^{(0)}-E_{k}^{(0)}}=\sum_{k\neq n}|k^{(0)}\rangle\frac{V_{k n}}{E_{n}^{(0)}-E_{k}^{(0)}}$$

\end{proposition}
\begin{proof}
נתחיל מהמשוואה:
$$(\hat{H}^{(0)}-E_{n}^{(0)})|n^{(1)}\rangle=(E_{n}^{(1)}-\hat{V})|n^{(0)}\rangle$$
נכפיל משמאל ב-\(\bra{k^{(0)}}\) כאשר \(k\neq n\) ונקבל:
\begin{gather*}{{\left\langle  k^{(0)}|\hat{H}^{(0)}-E_{n}^{(0)}|n^{(1)} \right\rangle}}={{\left\langle  k^{(0)}|E_{n}^{(1)}-\hat{V}|n^{(0)} \right\rangle}}\\ {{(E_{k}^{(0)}-E_{n}^{(0)})\langle k^{(0)}|n^{(1)}\rangle}}={{-\left\langle  k^{(0)}|\hat{V}|n^{(0)} \right\rangle}}\\{{\langle k^{(0)}|n^{(1)}\rangle}}={{\frac{\left\langle  k^{(0)}|\hat{V}|n^{(0)} \right\rangle}{E_{n}^{(0)}-E_{k}^{(0)}}}}
\end{gather*}
נעביר את \(\ket{n^{(1)}}\) בסיס ונקבל:
$$|n^{(1)}\rangle=\hat{1}|n^{(1)}\rangle=\sum_{k}|k^{(0)}\rangle\langle k^{(0)}|n^{(1)}\rangle$$
ולכן:
$$|n^{(1)}\rangle=\sum_{k\neq n}|k^{(0)}\rangle\frac{\langle k^{(0)}|\hat{V}|n^{(0)}\rangle}{E_{n}^{(0)}-E_{k}^{(0)}}=\sum_{k\neq n}|k^{(0)}\rangle\frac{V_{k n}}{E_{n}^{(0)}-E_{k}^{(0)}}$$

\end{proof}
\begin{proposition}[תיקון מסדר שני לאנרגיה]
$$E_{n}^{(2)}=H_{n n}^{(2)}+\sum_{k\neq n}\frac{V_{n k}V_{k n}}{E_{n}^{(0)}-E_{k}^{(0)}}$$

\end{proposition}
\begin{proof}
מהמשוואה:
$${\left( \hat{H}^{(0)}-E_{n}^{(0)} \right)\ket{n^{(2)}} =\left( E_{n}^{(2)}-\hat{H}^{(2)} \right)\ket{n^{(0)}} +\left( E_{n}^{(1)}-\hat{V} \right)\ket{n^{(1)}} }$$
כאשר אם נכפיל את אגף שמאל ב-\(\ket{n^{(0)}}\) נקבל:
\begin{gather*}\left\langle  n^{(0)}|\hat{H}^{(0)}-E_{n}^{(0)}|n^{(2)} \right\rangle=\left\langle  n^{(0)}|E_{n}^{(2)}-\hat{H}^{(2)}|n^{(0)} \right\rangle+\left\langle  n^{(0)}|E_{n}^{(1)}-\hat{V}|n^{(1)} \right\rangle\\ 0={{E_{n}^{(2)}-\left\langle  n^{(0)}|\hat{H}^{(2)}|n^{(0)} \right\rangle-\left\langle  n^{(0)}|\hat{V}|n^{(1)} \right\rangle}}
\end{gather*}
כאשר השתמשנו בארתוגונאליות \(\langle n^{(0)}|n^{(1)} \rangle=0\). נבודד את \(E_{n}^{(2)}\):
$$E_{n}^{(2)}=\langle n^{(0)}|\hat{H}^{(2)}|n^{(0)}\rangle+\langle n^{(0)}|\hat{V}|n^{(1)}\rangle=H_{n n}^{(2)}+\langle n^{(0)}|\hat{V}|n^{(1)}\rangle$$
כלומר:
$$E_{n}^{(2)}=H_{n n}^{(2)}+\sum_{k\neq n}\frac{V_{n k}V_{k n}}{E_{n}^{(0)}-E_{k}^{(0)}}$$

\end{proof}
\begin{example}[מערכת שתי רמות כללית]
מערכת שתי רמות כללית תהיה עם המילטוניאן מהצורה:
$${H}^{0}=\left(\begin{array}{c c}{{E_{a}^{0}}}&{{0}}\\ {{0}}&{{E_{b}^{0}}}\end{array}\right)$$
כאשר ניתן לכתוב הפרעה כללית על ידי:
$${V}=\lambda\left(\begin{array}{c c}{{V_{a a}}}&{{V_{a b}}}\\ {{V_{b a}}}&{{V_{b b}}}\end{array}\right)$$
כאשר \(V_{ba}=V_{ab}^{*}\) ו-\(V_{aa},V_{bb}\) ממשיים, כך ש-\(H\) הרמיטי. נחשב ראשית את התיקונים מסדר ראשון לאנרגיה:
$$E_{a}^{(1)}=\left\langle  a^{0}|\mathrm{V}|a^{0} \right\rangle=\lambda V_{a a} \qquad E_{b}^{(1)}=\left\langle  b^{0}|\mathrm{V}|b^{0} \right\rangle=\lambda V_{b b}$$
כאשר התיקונים מסדר שני לאנרגיה יהיו:
\begin{gather*}E_{a}^{(2)}=\sum_{k\neq a}\frac{|\left\langle  k^{0}|\mathrm{V}|a^{0} \right\rangle|^{2}}{E_{a}^{0}-E_{k}^{0}}=\frac{|\lambda V_{a b}|^{2}}{E_{a}^{0}-E_{b}^{0}}\\ E_{b}^{(2)}=\sum_{k\neq b}\frac{|\left\langle  k^{0}|\mathrm{V}|b^{0} \right\rangle|^{2}}{E_{b}^{0}-E_{k}^{0}}=\frac{|\lambda V_{a b}|^{2}}{E_{b}^{0}-E_{a}^{0}}
\end{gather*}
ולכן סה"כ נקבל כי האנרגיה עד כדי תיקון מסדר שני תהיה:
$$E_{a}\approx E_{a}^{0}+\lambda V_{a a}+\lambda^{2}\frac{|V_{a b}|^{2}}{E_{a}^{0}-E_{b}^{0}}\qquad E_{b}\approx E_{b}^{0}+\lambda V_{b b}+\lambda^{2}\frac{|V_{a b}|^{2}}{E_{b}^{0}-E_{a}^{0}}$$
כעת נבצע קירוב למצבים העצמיים מסדר ראשון. נקבל מהצבה בנוסחה:
$$|a\rangle\approx|a^{0}\rangle+\frac{\lambda V_{a b}^{*}}{E_{a}^{0}-E_{b}^{0}}|b^{0}\rangle  \quad |b\rangle\approx|b^{0}\rangle-\frac{\lambda V_{a b}}{E_{a}^{0}-E_{b}^{0}}|a^{0}\rangle$$

\end{example}
\begin{proposition}[תקפות תורת הפרעות]
כדי שניתן להשתמש בתורת הפרעות נדרש כי התיקון יהיו קטנים. למעשה נדרש:
$$\left|\frac{\langle\phi_{m}\,|\,V\mid\phi_{n}\rangle}{E_{n}^{(0)}-E_{m}^{(0)}}\right|\ll1\qquad(n\neq m)$$
תנאי זה מבטיח שהתיקון לאנרגיה ולפונקציית הגל מסדר ראשון יהיה קטן ביחס לאנרגיה ולפונקציית גל מסדר אפס, ובכך מצדיק את הקירוב הטורי.

\end{proposition}
\begin{example}[פוטנציאל הרמוני תחת שדה מגנטי]
נתון חלקיק עם מטען \(q\) ומסה \(m\) בפוטנציאל הרמוני חד מימדי עם תדירות \(\omega\) אשר נתון תחת שדה מגנטי חלש \(\mathcal{E}\) בכיוון \(x\). האנרגיה נתונה על ידי:
$$\hat{H}=\hat{H}_{0}+\hat{V}=-\frac{\hbar}{2m}\frac{d^{2}}{d X^{2}}+\frac{1}{2}m\omega^{2}\hat{X}^{2}+q\mathcal{E}\hat{X}.$$
ניתן לחשב את האנרגיות העצמיות של ההמילטוניאן נבצורה מדוייקת (כלומר ללא תורת הפרעות) על ידי החלפת משתנה \(\hat{y}=\hat{X}+\frac{q\mathcal{E}}{m\omega^{2}}\):
$$\hat{H}=-\frac{\hbar^{2}}{2m}\frac{d^{2}}{d y^{2}}+\frac{1}{2}m\omega^{2}\hat{y}^{2}-\frac{q^{2}\mathcal{E}^{2}}{2m\omega^{2}}.$$
כאשר זה המילטוניאן של אוסצילטור המרמוני קוונטי רגיל אשר מחסר ממינו קבוע. ולכן האנרגיות העצמיות יהיו:
$$E_{n}=\left(n+\frac{1}{2}\right)\hbar\omega-\frac{q^{2}\mathcal{E}^{2}}{2m\omega^{2}}.$$
כעת נשוואה לפתרון בעזרת תורת הפרעות. נשים לב כי \(E_{n}^{(1)}=a\bra{n}X\ket{n}=0\) כיוון שערך התצפית של המיקום באוסצליטור הרמוני הוא אפס. לכן נחשב את התיקון מסדר שני:
$$E_{n}^{(2)}=q^{2}\mathcal{E}^{2}\sum_{m\neq n}\frac{\left|\langle m\mid\hat{X}\mid n\rangle\right|^{2}}{E_{n}^{(0)}-E_{m}^{(0)}}$$
נזכור כי \(E_{n}^{(0)}=\left( n+\frac{1}{2} \right)\hbar \omega\) כאשר בעזרת היחסים:
\begin{gather*}\left\langle  n+1\mid\hat{X}\mid n \right\rangle\;=\;\sqrt{n+1}\sqrt{\frac{\hbar}{2m\omega}}\qquad E_{n}^{(0)}-E_{n-1}^{(0)}\;=\;\hbar\omega\left\langle  n-1\mid\hat{X}\mid n \right\rangle=\sqrt{n}\sqrt{\frac{\hbar}{2m\omega}}\qquad E_{n}^{(0)}-E_{n+1}^{(0)}=-\hbar\omega
\end{gather*}
ניתן לכתוב את הקירוב מסדר שני בצורה הבאה:
$$E_{n}^{(2)}\,=\,q^{2}{\mathcal E}^{2}\left[\,\frac{|\left\langle  n+1\mid\hat{X}\mid n \right\rangle|^{2}}{E_{n}^{(0)}-E_{n+1}^{(0)}}+\frac{|\left\langle  n-1\mid\hat{X}\mid n \right\rangle|^{2}}{E_{n}^{(0)}-E_{n-1}^{(0)}}\,\right]=-\frac{q^{2}\mathcal{E}^{2}}{2m\omega^{2}}$$
ולכן עד כדי סדר שני האנרגיה נתונה על ידי:
$$E_{n}=E_{n}^{(0)}+E_{n}^{(1)}+E_{n}^{(2)}=\left(n+\frac{1}{2}\right)\hbar\omega-\frac{q^{2}\mathcal{E}^{2}}{2m\omega^{2}}.$$
אשר שווה לחלוטין לפתרון המדויק.

\end{example}
\begin{proposition}[שימוש בזוגיות לאיפוס סדר ראשון]
נניח כי יש לנו המילטוניאן מהצורה \(H=H_{0}+V\) כאשר \(V\) קטן כך ש-\(H_{0}\) אינווראינטי לשיקוף מרחבי(כלומר \(\Pi H_{0} \Pi ^{-1}=H_{0}\)), ו-\(V\) הפרעה אשר אי זוגית לשיקוף מרחבי(כלומר \(\Pi V\Pi ^{-1}=-V\)). אזי הגורם מסדר ראשון של ההפרעה מתאפס.

\end{proposition}
\begin{proof}
מההנחה, \(\left[ H_{0},\Pi \right]=0\). ולכן המצבים העצמיים של \(H_{0}\) הם עם זוגיות מוגדרת. כלומר מתקיים אחד מהאפשרויות עבור כל מצב:
$$\Pi \ket{n} =\ket{n} \quad \text{or}\quad \Pi \ket{n} =-\ket{n} $$
עבור הפרעה מסדר ראשון מתקיים:
$$E^{(1)}_{n}=\braket{ n | V|n } =\braket{ n|\Pi ^{-1} \Pi V\Pi ^{-1}\Pi  | n }= \braket{ n | (-V)|n } $$
כאשר השתמשנו בכך שתחת היפוך מרחב \(\Pi V\Pi ^{-1} = -V\) וכן \(\Pi \ket{n}=\left( \pm 1 \right)\ket{n}\) ונקבל תוספת סימן זהה עבור \(\bra{n}\Pi ^{-1}\) כאשר \(\left( \pm 1 \right)\cdot\left( \pm 1 \right)=1\). כעת קיבלנו:
$$\braket{ n|V | n } =-\braket{ n|V | n } \implies \braket{ n|V | n } =0$$

\end{proof}
\begin{summary}
  \begin{itemize}
    \item תורת ההפרעות הבלתי תלוי בזמן מתייחסת להמילטוניאנים מהצורה:
$$H=H^{(0)}+\lambda V,\quad\lambda\ll1$$
כאשר \(H^{(0)}\) הוא המילטוניאן שערכים העצמיים שלו ידועים, ו-\(V \equiv H^{(1)}\) מסמן את ההפרעה מסדר ראשון.
    \item ניתן לפתח את האנרגיות העצמיות והפונקציות גל בטור חזקות של \(\lambda\):
$$\begin{array}{l} {{\psi_{n}=\psi_{n}^{(0)}+\lambda\,\psi_{n}^{(1)}+\lambda^{2}\,\psi_{n}^{(2)}+\cdots}}\\ {{E_{n}=E_{n}^{(0)}+\lambda\,E_{n}^{(1)}+\lambda^{2}\,E_{n}^{(2)}+\cdots}}\end{array}$$
    \item הגורם מסדר אפס מקיים:
$$H^{(0)}\psi_{n}^{(0)}=E_{n}^{(0)}\psi_{n}^{(0)}$$
    \item הגורם מסדר ראשון מקיים \(\big(H^{(0)}-E_{n}^{(0)}\big)\psi_{n}^{(1)}=\big(E_{n}^{(1)}-V\big)\psi_{n}^{(0)}\) ומכאן בעזרת האורתוגונאליות המצבים הלא מופרעים התיקון לאנרגיה תהיה:
$$E_{n}^{(1)}=\langle\psi_{n}^{(0)}|V|\psi_{n}^{(0)}\rangle$$
והתיקון לפונקציית גל תהיה:
$$\psi_{n}^{(1)}=\sum_{k\neq n}\frac{\langle\psi_{k}^{(0)}|V|\psi_{n}^{(0)}\rangle}{E_{n}^{(0)}-E_{k}^{(0)}}\,\psi_{k}^{(0)}$$
    \item מסדר שני התיקון לאנרגיה יקיים:
$$E_{n}^{(2)}=\sum_{k\neq n}\frac{|\langle\psi_{k}^{(0)}|V|\psi_{n}^{(0)}\rangle|^{2}}{E_{n}^{(0)}-E_{k}^{(0)}}$$
  \end{itemize}
\end{summary}
\section{מנוונת שאינו תלוי בזמן}

\begin{symbolize}
אם להמילטוניאן יש \(d\) מצבים עם אנרגיות \(E_{n}^{(0)}\) נסמן את הקירוב מסדר 0 על ידי:
$$\psi_{n,i}^{(0)}=|(n,i)^{(0)}\rangle,\;\;\;i=1,\ldots,d$$

\end{symbolize}
הרעיון זה שההפרעה \(\lambda\) יכולה לבטל את הניוון, ואז ניתן לפתור בצורה הרגילה.

\begin{proposition}
כיוון שיש ניוון כל צירוף לינארי של מצבים \(\ket{(n,i)^{(0)}}\) גם כן יהיה מצב עצמי של המערכת.

\end{proposition}
\begin{definition}[בחירה "טובה" של מצבים]
וקטורים עצמיים אשר תחת ההפרעה מלכסנת את התת מרחב המנוון כאשר לוקחים את הגבול \(\lambda\to 0\).

\end{definition}
\begin{example}
נסתכל על מסה \(m\) בפוטנציאל אוסצילטורי דו מימדי מהצורה:
$$H^{0}=\frac{p^{2}}{2\,m}+\frac{1}{2}\,m\,\omega^{2}\left(x^{2}+y^{2}\right)$$
כאשר נתונה ההפרעה:
$$V=\epsilon\,m\,\omega^{2}\,x\;y.$$
המצב המעורער הראשון (\(E^{(0)}=2\hbar \omega\)) מכיל ניוון מסדר שתיים, כאשר בסיס אחד לניוון הזה הוא:
$$\psi_{a}^{0}=\psi_{0}(x)\,\,\psi_{1}(y)=\sqrt{\frac{2}{\pi}}\,\frac{m\,\omega}{\hbar}\,\,y\,e^{-\frac{m\,\omega}{2\,\hbar}\left(x^{2}+y^{2}\right)}\,.$$
וגם:
$$\psi_{b}^{0}=\psi_{1}(x)\ \psi_{0}(y)=\sqrt{\frac{2}{\pi}}\,\frac{m\,\omega}{\hbar}\,x\,e^{-\frac{m\,\omega}{2\,\hbar}\left(x^{2}+y^{2}\right)}$$
כאשר \(\psi_{1},\psi_{0}\) הם אוסצילטורים הרמונים חד מימדיים. ניתן לפתור את הבעיה הזו בצורה מדוייקת על ידי מעבר למערכת מסובבת:
$$x^{\prime}=\frac{x+y}{\sqrt{2}}\qquad y^{\prime}=\frac{x-y}{\sqrt{2}}\,.$$
כאשר במונחי הקורדינטות החדשות נקבל:
$$H=\frac{1}{2\,m}\bigg(\frac{\partial^{2}}{\partial x^{\prime2}}+\frac{\partial^{2}}{\partial y^{\prime2}}\bigg)+\frac{1}{2}\,m\,(1+\epsilon)\,\,\omega^{2}\,x^{\prime2}+\frac{1}{2}\,m\,(1-\epsilon)\,\,\omega^{2}\,y^{\prime2}.$$
זה מתאים לשתי אוסצילטורים הרמונים חד מימדיים \(\psi_{m}^{+}\) ו-\(\psi_{m}^{-}\) עם תדירות \(\omega_{\pm}=\sqrt{ 1\pm \varepsilon }\omega\). הרמות הראשונות יהיו:
$$E_{m n}=\left(m+\frac{1}{2}\right)\,\hbar\,\omega_{+}+\left(n+\frac{1}{2}\right)\,\hbar\,\omega_{-}$$
ניקח כעת את הגבול כאשר \(\varepsilon\to 0\) של הרמת אנרגיה הראשונה ונקבל:
$$\operatorname*{lim}_{\epsilon\rightarrow0}\psi_{01}(x)=\operatorname*{lim}_{\epsilon\rightarrow0}\psi_{0}^{+}(x^{\prime})\ \psi_{1}^{-}(y^{\prime})=\psi_{0}\bigg(\frac{x+y}{\sqrt{2}}\bigg)\ \psi_{1}\bigg(\frac{x-y}{\sqrt{2}}\bigg)\ .$$$$=\sqrt{\frac{2}{\pi}}\,\frac{m\,\omega}{\hbar}\,\frac{x-y}{\sqrt{2}}\,e^{-\frac{m\,\omega}{2\hbar}\left(x^{2}+y^{2}\right)}=\frac{-\psi_{a}^{0}+\psi_{b}^{0}}{\sqrt{2}}.$$
ובאופן דומה:
$$\operatorname*{lim}_{\epsilon\rightarrow0}\psi_{10}(x)=\frac{\psi_{a}^{0}+\psi_{b}^{0}}{\sqrt{2}}.$$
ולכן המצבים ה-"טובים" יהיו:
$$\psi_{\pm}^{0}\equiv\frac{1}{\sqrt{2}}\left(\psi_{b}^{0}\pm\psi_{a}^{0}\right)$$

\end{example}
\begin{remark}
לא תמיד קיימים מצבים כאלה. כאשר לא ייתכן וניתן לפתור על ידי תורת ההפרעות המנוונת מסדר שני.

\end{remark}
\begin{symbolize}
$$W_{i j}\equiv\left<\psi_{i}^{0}\left|\,V\,\right|\,\psi_{j}^{0}\right>$$

\end{symbolize}
\begin{remark}
המטריצת \(W\) למעשה מייצגת איך ההפרעה משפיעה על איברי הבסיס של \(H_{0}\). לדוגמא אם \(W_{10}=0\) זה מראה כי האיבר \(\ket{0}\) לא משתנה ל-\(\ket{1}\) בעקיבות ההפרעה. לעומת זאת אם \(W_{10}=1\) זה אומר כי ההפרעה תשנה את \(\ket{0}\) ל-\(\ket{1}\). רק תחת בסיס שבו \(W\) אלכסוני נקבל כי המערכת מתנהגת כאילו לא מנוונת - ההפרעה תשאיר כל מצב עצמי באותו מצב עצמי.

\end{remark}
\begin{proposition}
כאשר קיימים מצבים טובים, הם יהיו הווקטורים העצמיים של \(W\). כלומר נקבל:
$$W \vec{v}=E^{(1)}\vec{v}$$

\end{proposition}
\begin{proposition}
כאשר יש נוון מסדר 2 נקבל כי התיקון מסדר ראשון לאנרגיה תהיה:
$$E_{\pm}^{1}=\frac{1}{2}\left[\,W_{a a}+\,W_{b b}\pm\sqrt{\,(\,W_{a a}-\,W_{b b})^{2}+4\,|\,W_{a b}|^{2}\,}\right]$$

\end{proposition}
\begin{proof}
נניח ונתונה משוואת שרודינגר \(H\psi=E\psi\) עם \(H=H^{(0)}+\lambda V\) וכן כרגיל נסמן:
$$E=E^{0}+\lambda E^{1}+\lambda^{2}E^{2}+\cdots,\quad\psi=\psi^{0}+\lambda\psi^{1}+\lambda^{2}\psi^{2}+\cdots.$$
הצב את הפיתוח לטור במשוואת שרדינגר ונאחד לפי חזקות של \(\lambda\) כך שנקבל:
$$H^{0}\psi^{0}+\lambda\left(V\psi^{0}+H^{0}\psi^{1}\right)+\cdots=E^{0}\psi^{0}+\lambda\left(E^{1}\psi^{0}+E^{0}\psi^{1}\right)+\cdots.$$
כאשר \(H^{0}\psi^{0}=E^{0}\psi^{0}\) ולכן נקבל עבור הגודם של \(\lambda^{1}\):
$$H^{0}\psi^{1}+V\psi^{0}=E^{0}\psi^{1}+E^{1}\psi^{0}.$$
ניקח את המכפלה הפנימית עם \(\psi^{0}_{a}\) ונקבל:
$$\left\langle\psi_{a}^{0}\left|H^{0}\psi^{1}\right\rangle+\left\langle\psi_{a}^{0}\right|V\psi^{0}\right\rangle=E^{0}\left\langle\psi_{a}^{0}\left|\psi^{1}\right\rangle+E^{1}\left\langle\psi_{a}^{0}\right|\psi^{0}\right\rangle$$
כיוון ש-\(H^{0}\) הרמיטי הגורם נראשון משמאל מבטל את הגורם הראשון מימין. על ידי שימוש באורותונרליות נקבל:
$$\alpha\left\langle\psi_{a}^{0}\left|\right.V\left|\right.\psi_{a}^{0}\right\rangle+\beta\left\langle\psi_{a}^{0}\left|\right.V\left|\right.\psi_{b}^{0}\right\rangle=\alpha E^{1},$$
כלומר:
$$\alpha\,W_{a a}+\beta\,W_{a b}=\alpha\,E^{1},$$
באופן דומה בעזרת מכפלה פנימית של \(\psi_{b}^{0}\) נקבל:
$$\alpha\,W_{b a}+\beta\,W_{b b}=\beta E^{1}$$
מללכסן את \(W\) עבור מערכת של שתי רמות נקבל את הטענה.

\end{proof}
\begin{example}
נמצא את הרמות האנרגיה ה-"טובות" של הדוגמא הקודמת ללא לפתור בצורה מפורשת. נחשב את \(W\).
$$W_{a a}=\int\int\psi_{a}^{0}(x,y)\ V\,\psi_{a}^{0}(x,y)\ d x\,d y$$$$=\epsilon\,m\,\omega^{2}\,\int|\psi_{0}(x)|^{2}\,\,x\,d x\,\int|\psi_{0}(y)|^{2}\,\,y\,d y=0$$
כיוון שהאינטגרנד של שתי האינטגרלים הם פונקציות אי זוגיות. באופן דומה \(W_{bb}=0\). כעת נחשב את \(W_{ab}\):
\begin{gather*}W_{a b}=\iint\psi_{a}^{0}(x,y)\ V\,\psi_{b}^{0}(x,y)\ \mathrm{d} x\,\mathrm{d} y\\=\epsilon\,m\,\omega^{2}\,\int\psi_{0}(x)\ x\ \psi_{1}(x)\ \mathrm{d} x\,\int\psi_{1}(y)\ y\ \psi_{0}(y)\ \mathrm{d} y 
\end{gather*}
כאשר ניתן לכתוב \(x=\sqrt{ \frac{\hbar}{2m\omega} }(a_{+}+a_{-})\) ונקבל:
$$W_{a b}=\epsilon\,m\,\omega^{2}\left[\int\,\psi_{0}(x)\,\sqrt{\frac{\hbar}{2\,m\,\omega}}\,(a_{+}+a_{-})\,\,\psi_{1}(x)\,\,d x\right]^{2}$$$$=\epsilon\,\frac{\hbar\,\omega}{2}\,\left[\int\psi_{0}(x)\,\,\psi_{0}(x)\,\,d x\right]^{2}=\epsilon\,\frac{\hbar\,\omega}{2}.$$
ולכן המטרצה \(W\) תהיה:
$$W=\epsilon\,\frac{\hbar\,\omega}{2}\left(\begin{array}{l l}{{0}}&{{1}}\\ {{1}}&{{0}}\end{array}\right)$$
כאשר הווקטורים העצמיים יהיו:
$${\frac{1}{\sqrt{2}}}{\left(\begin{array}{l}{1}\\ {1}\end{array}\right)}\;\;\mathrm{and}\;\;{\frac{1}{\sqrt{2}}}{\left(\begin{array}{l}{-1}\\ {1}\end{array}\right)}$$
ולכן המצבים הטובים יהיו:
$$\psi_{\pm}^{0}=\frac{1}{\sqrt{2}}\left(\psi_{b}^{0}\pm\psi_{a}^{0}\right),$$
כאשר הערך העצמי(שזה התיקון הראשון לאנרגיה) יהיה:
$$E^{1}=\pm\,\epsilon\,\frac{\hbar\,\omega}{2}$$

\end{example}
\begin{proposition}
יהי \(A\) אופרטור הרמיטי אשר מתחלף עם \(H^{0}\) ועם \(V\). אם \(\psi_{a}^{(0)}\) ו-\(\psi_{b}^{(0)}\) הם פונקציות עצמיות של \(A\) עם ערכים עצמיים יחודיים:
$$A\psi_{a}^{0}=\mu\psi_{a}^{0},\quad A\psi_{b}^{0}=\nu\psi_{b}^{0},\quad\mathrm{and}\;\mu\neq\nu,$$
אזי \(\psi_{a}^{0}\) ו-\(\psi_{b}^{0}\) המצבים "טובים".

\end{proposition}
\begin{proof}
כיוון ש-\(H\left( \lambda \right)=H^{0}+\lambda V\) ו-\(A\) מתחלף קיים בסיס משותף של מצבים עצמיים \(\psi_{\gamma}\left( \lambda \right)\) כך שמתקיים:
$$H(\lambda)\ \psi_{\gamma}(\lambda)=E(\lambda)\ \psi_{\gamma}(\lambda)\ \ \mathrm{and}\ A\ \psi_{\gamma}(\lambda)=\gamma\ \psi_{\gamma}(\lambda)\ .$$
כאשר כיוון ש-\(A\) הרמיטי נקבל:
$$\left\langle\psi_{a}^{0}\left|\right.A\psi_{\gamma}(\lambda)\right\rangle=\left\langle A\psi_{a}^{0}\left|\right.\psi_{\gamma}(\lambda)\right\rangle$$$$\gamma\,\left\langle\psi_{a}^{0}\,\right|\,\psi_{\gamma}(\lambda)\rangle=\mu^{*}\,\left\langle\psi_{a}^{0}\,\right|\,\psi_{\gamma}(\lambda)\rangle$$$$\left(\gamma-\mu\right)\left\langle\psi_{a}^{0}\right|\psi_{\gamma}(\lambda)\rangle=0,$$
זה נכון לכל ערך של \(\lambda\), ובפרט אם ניקח את הגבול \(\lambda\to 0\) נקבל:
$$\left\langle\psi_{a}^{0}\left|\;\psi_{\gamma}(0)\right\rangle=0\;\mathrm{unless}\;\;\gamma=\mu,\right.$$
ובאופן דומה:
$$\left\langle\psi_{b}^{0}\left|\right.\psi_{\gamma}(0)\right\rangle=0\mathrm{~unless~}\gamma=\nu.$$
כאשר המצבים הטובים הם צירופים לינארים של

\end{proof}
\begin{example}
עבור המערכת בשתי דוגמאות הקודמות נשים לב כי האופרטור \(D\) של סיבוב ב-\(45^\circ\) אשר מחליף בין \(x\) ל-\(y\) מקיים:
$$\begin{array}{l}{{D\,\psi_{a}^{0}(x,\,y)=\psi_{a}^{0}(y,x)=\psi_{b}^{0}(x,\,y)\,,}}\\ {{D\,\psi_{b}^{0}(x,\,y)=\psi_{b}^{0}(y,x)=\psi_{a}^{0}(x,\,y)\,.}}\end{array}$$
נשים לב שהמצבים העצמיים שלנו הם לא מצבים עצמיים של \(D\), אך ניתן לבנות כאלה שהם כן בעזרת הצירוף הלינארי שלהם:
$$\psi_{\pm}^{0}\equiv\pm\psi_{a}^{0}+\psi_{b}^{0}.$$
וכעת:
$$D\left(\pm\psi_{a}^{0}+\psi_{b}^{0}\right)=\pm\,D\,\psi_{a}^{0}+D\,\psi_{b}^{0}=\pm\psi_{b}^{0}+\psi_{a}^{0}=\pm\left(\pm\psi_{a}^{0}+\psi_{b}^{0}\right)$$
כאשר מהטענה אלו מצבים טובים.

\end{example}
\begin{remark}
תחת הבסיס הזה אשר מתחלף עם \(H^{0}\) ו-\(V\) נקבל כי \(W\) תהיה אלכסונית.

\end{remark}
\begin{remark}
השיטה הזו היא כללית. עבור למשל מערכת עם ניוון 3 נחשב:
$$W_{i j}=\left(\psi_{i}^{0}\left|\;V\;\right|\;\psi_{j}^{0}\right)$$
כאשר מוצאים את המצבים והערכים העצמיים של \(W\) על ידי פתרון:
$$\left(\begin{array}{c c c}{{W_{a a}}}&{{W_{a b}}}&{{W_{a c}}}\\ {{W_{b a}}}&{{W_{b b}}}&{{W_{b c}}}\\ {{W_{c a}}}&{{W_{c b}}}&{{W_{c c}}}\end{array}\right)\left(\begin{array}{c}{{\alpha}}\\ {{\beta}}\\ {{\gamma}}\end{array}\right)=E^{1}\left(\begin{array}{c}{{\alpha}}\\ {{\beta}}\\ {{\gamma}}\end{array}\right)$$
כאשר מצבים "טובים" יהיו מצבים מהצורה:
$$\psi^{0}=\alpha\,\psi_{a}^{0}+\beta\,\psi_{b}^{0}+\gamma\,\psi_{c}^{0}.$$

\end{remark}
\begin{example}
נתון ההמילטוניאן:
$$H=V_{0}\left(\begin{array}{c c c}{{(1-\epsilon)}}&{{0}}&{{0}}\\ {{0}}&{{1}}&{{\epsilon}}\\ {{0}}&{{\epsilon}}&{{2}}\end{array}\right)$$
כאשר \(V_{0}\) קבוע ו-\(\varepsilon\ll 1\). ראשית נמצא את הפתרונות המדוייקים עבור \(\varepsilon=0\). נקבל מיידית את הערכים העצמיים:
$$\lambda_{1}=1\quad\lambda_{2}=1\quad\lambda_{3}=2$$
כאשר נשים לב כי יש ריבוי 2 עבור הערך עצמי 1. הווקטורים העצמיים המתאימים יהיו:
$$\psi_{1}^{(0)}=e_{1}=\begin{pmatrix}1\\ 0\\ 0\end{pmatrix}\quad\psi_{2}^{(0)}=e_{2}=\begin{pmatrix}0\\ 1\\ 0\end{pmatrix}\quad\psi_{3}^{(0)}=e_{3}=\begin{pmatrix}0\\ 0\\ 1\end{pmatrix}$$
ניתן לכתוב את ההפרעה על ידי:
$$V=H-H^{(0)}=V_{0}\left(\begin{array}{c c c}{{(1-\epsilon)}}&{{0}}&{{0}}\\ {{0}}&{{1}}&{{\epsilon}}\\ {{0}}&{{\epsilon}}&{{2}}\end{array}\right)-V_{0}\left(\begin{array}{c c c}{{1}}&{{0}}&{{0}}\\ {{0}}&{{1}}&{{0}}\\ {{0}}&{{0}}&{{2}}\end{array}\right)=V_{0}\left(\begin{array}{c c c}{{-\epsilon}}&{{0}}&{{0}}\\ {{0}}&{{0}}&{{\epsilon}}\\ {{0}}&{{\epsilon}}&{{0}}\end{array}\right)$$
נדרש לחשב את אלמנטי המטריצה של \(V\) בתת מרחב המנוון הנפרש על ידי \(\left\{  \psi_{1}^{(0)},\psi^{(0)}_{2}  \right\}\). נסמן \(W_{ij}=\braket{ \psi_{i}^{(0)} | V|\psi_{j}^{(0)} }\) ונחשב את אלמנטי \(W\) במפורש:
\begin{gather*}W_{11}=\langle \psi_{1}^{(0)}|V|\psi_{1}^{(0)} \rangle=(e_{1})^{\dagger}V(e_{1})=\left(1\quad0\quad0\right)V_{0}\left(\begin{array}{l l l}{{-\epsilon}}&{{0}}&{{0}}\\ {{0}}&{{0}}&{{\epsilon}}\\ {{0}}&{{\epsilon}}&{{0}}\end{array}\right)\left(\begin{array}{l}{{1}}\\ {{0}}\\ {{0}}\end{array}\right)=-\epsilon V_{0}\\ W_{22}=\langle \psi_{2}^{(0)}|V|\psi_{2}^{(0)} \rangle=(e_{2})^{\dagger}V(e_{2})=\left(\begin{array}{l l l}{{0}}&{{1}}&{{0}}\end{array}\right)V_{0}\left(\begin{array}{l l l}{{-\epsilon}}&{{0}}&{{0}}\\ {{0}}&{{0}}&{{\epsilon}}\\ {{0}}&{{\epsilon}}&{{0}}\end{array}\right)\left(\begin{array}{l}{{0}}\\ {{1}}\\ {{0}}\end{array}\right)=0\\ W_{12}=\langle \psi_{1}^{(0)}|V|\psi_{2}^{(0)} \rangle=(e_{1})^{\dagger}V(e_{2})=\left(1\quad0\quad0\right)V_{0}\left(\begin{array}{l l l}{{-\epsilon}}&{{0}}&{{0}}\\ {{0}}&{{0}}&{{\epsilon}}\\ {{0}}&{{\epsilon}}&{{0}}\end{array}\right)\left(\begin{array}{l}{{0}}\\ {{1}}\\ {{0}}\end{array}\right)=0\\ W_{21}=\langle \psi_{2}^{(0)}|V|\psi_{1}^{(0)} \rangle=(e_{2})^{\dagger}V(e_{1})=\left(\begin{array}{l l l}{{0}}&{{1}}&{{0}}\end{array}\right)V_{0}\left(\begin{array}{l l l}{{-\epsilon}}&{{0}}&{{0}}\\ {{0}}&{{0}}&{{\epsilon}}\\ {{0}}&{{\epsilon}}&{{0}}\end{array}\right)\left(\begin{array}{l}{{1}}\\ {{0}}\\ {{0}}\end{array}\right)=0
\end{gather*}
ולכן:
$$W=\left(\begin{array}{c c}{{-\epsilon V_{0}}}&{{0}}\\ {{0}}&{{0}}\end{array}\right)$$
כיוון ש-\(W\) כבר אלכסונית נקבל כי הערכים עצמיים יהיו \(E_{1}^{(1)}=-\varepsilon V_{0}\) ו-\(E_{2}^{(1)}=0\). האנרגיות העצמיות של הרמות המנוונות יהיו:
\begin{gather*}E_{1}\approx\lambda_{1}^{(0)}V_{0}+E_{1}^{(1)}=V_{0}-\epsilon V_{0}=V_{0}\left( 1-\epsilon \right)\\ E_{2}\approx\lambda_{2}^{(0)}V_{0}+E_{2}^{(1)}=V_{0}+0=V_{0}\\
\end{gather*}
כאשר הרמת אנרגיה הלא מנוונת תהיה:
\begin{gather*}E_{3}^{(1)}=\langle \psi_{3}^{(0)}|V|\psi_{3}^{(0)} \rangle=(e_{3})^{\dagger}V(e_{3})=\left(0\quad0\quad1\right)V_{0}\left(\begin{array}{c c c}{{-\epsilon}}&{{0}}&{{0}}\\ {{0}}&{{0}}&{{\epsilon}}\\ {{0}}&{{\epsilon}}&{{0}}\end{array}\right)\left(\begin{array}{c}{{0}}\\ {{0}}\\ {{1}}\end{array}\right)=0\\ \implies E_{3}\approx\lambda_{3}^{(0)}V_{0}+E_{3}^{(1)}=2V_{0}+0=2V_{0}
\end{gather*}

\end{example}
\begin{example}[אפקט שטרק]
נמצא בעזרת תורת הפרעות מנוונת מסדר ראשון את הרמות אנרגיה על אטום מימן עם \(n=2\) אשר תחת שדה חשמלי אחיד חלש \(\mathcal{E}\) הפועל לאורך ציר \(z\). הבסיס עבור המצב המנוון \(n=2\) יהיה עם ניוון \(2^{2}=4\) ויסומן בצורה הבאה:
$$|1\rangle=|200\rangle,\;|2\rangle=|211\rangle,\;|3\rangle=|210\rangle,\;\ |4\rangle=|21-1\rangle$$
וכן ההפרעה תהיה מהצורה:
$$V=-\vec{d}\cdot\vec{\mathcal{E}}=e\vec{r}\cdot\vec{\mathcal{E}}=e\mathcal{E}Z$$
נרצה לחשב את אלמנטי מטריצת \(W=\bra{2l'm'}V\ket{2ml}=e\mathcal{E}\bra{2l'm'}Z\ket{2lm}\). כפי שראינו אלמנטי המטריצה \(\bra{2\ell'm'}Z\ket{2\ell m}\) יתאפסו כאשר \(m'\neq m\) וכאשר \(\ell'-\ell = \pm 1\). ולכן נקבל כי \(\bra{200}Z\ket{210} = \bra{210}Z\ket{200} = -3 a_0\) הם אלמנטי המטריצה היחידים של \(Z\) שאינם מתאפסים. ולכן עבור \(W\) נקבל כי האלמנטים היחידים שלא מתאפסים יהיו:
$$W_{13}=W_{31}=\bra{200}e\mathcal{E}Z\ket{210}=e\mathcal{E} (-3a_{0})=-3e\mathcal{E} a_{0}$$
ולכן:
$$W=\begin{pmatrix}W_{11}&W_{12}&W_{13}&W_{14}\\ W_{21}&W_{22}&W_{23}&W_{24}\\ W_{31}&W_{32}&W_{33}&W_{34}\\ W_{41}&W_{42}&W_{43}&W_{44}\end{pmatrix}=\begin{pmatrix}0&0&-3e\mathcal{E}a_{0}&0\\ 0&0&0&0\\ -3e\mathcal{E}a_{0}&0&0&0\\ 0&0&0&0\end{pmatrix}$$
נדרש ללכסן את \(W\). הערכים העצמיים המתקבלים יהיו הפתרונות של \(\det \left( W-E^{(1)}\mathbb{1} \right)=0\) כאשר נקבל:
$$E^{(1)}=-3e\mathcal{E}a_{0},0,0,3e\mathcal{E}a_{0}.$$
ולכן התיקון מסדר ראשון לאנרגיה ה-0 יהיה:
\begin{gather*}E_{1}=E_{2}^{(0)}+E_{1}^{(1)}=-\frac{R_{y}}{4}-3e{\mathcal{E}}a_{0}\\ E_{2}=E_{2}^{(0)}+E_{2}^{(1)}=-\frac{R_{y}}{4}+3e{\mathcal E}a_{0}\\ E_{3}=E_{2}^{(0)}+E_{3}^{(1)}=-\frac{R_{y}}{4}+0=-\frac{R_{y}}{4}\\ E_{4}=E_{2}^{(0)}+E_{4}^{(1)}=-\frac{R_{y}}{4}+0=-\frac{R_{y}}{4}
\end{gather*}
כאשר השתמשנו בנוסחה של רמות אנרגיה של אטום המימן עבור \(E_{2}^{(0)}=-\frac{R_{y}}{n^{2}}=-\frac{\mu e^{4}}{2\hbar^{2}}\). המצבים הטובים יהיו הווקטורים העצמיים של \(W\):
\begin{gather*}E_{1}^{(1)}=-3e{\mathcal{E}}a_{0}\implies\psi_{1}^{(0)}={\frac{1}{\sqrt{2}}}\left( |200\rangle+|210\rangle \right)\\ E_{2}^{(1)}=3e{\mathcal E}a_{0}\implies\psi_{2}^{(0)}={\frac{1}{\sqrt2}}\left( -|200\rangle+|210\rangle \right)\\ E_{3}^{(1)}=0\implies\psi_{3}^{(0)}=|211\rangle\\ E_{4}^{(1)}=0\implies\psi_{4}^{(0)}=|21-1\rangle
\end{gather*}

\end{example}
\begin{summary}
  \begin{itemize}
    \item כאשר יש להמילטוניאן לא מופרע \(H^{(0)}\) ניוון, הפרעה \(V\) יכולה להעלים את הניוון, ולגרום לרמות אנרגיה להתפצל.
    \item כאשר יש ניוון, כל צירוף לינארי של מצבים מנוונים הם מצבים עצמיים של \(H^{(0)}\). צירוף "טוב" הם הצירוף הלינארי במרחב המנוון אשר מלכסן את ההפרעה במרחב המנוון מסדר ראשון.
    \item לכן כדי למצוא את המצבים ה-"טובים" נדרש ללכסן את \(W_{ij}=\bra{\psi_{i}^{(0)}}V\ket{\psi_{j}^{(0)}}\) כאשר \(\psi_{i}^{(0)}\) ו-\(\psi_{j}^{(0)}\) הם המצבים מנוונים מסדר אפס. התיקונים לאנרגיה מסדר ראשון \(E^{(1)}\) יהיו הערכים העצמיים, כאשר המצבים ה-"טובים" יהיו הווקטורים העצמיים.
  \end{itemize}
\end{summary}
\section{המבנה הדק}

הפתרון המדויק של אטום המימן הוא לא בדיוק מה שמצאנו. יש עוד שתי אפקטים קטנים שניתן להתייחס עליהם בקירוב על ידי תורת ההפרעות - הצימוד Spin-Orbit שזה הגורם שנובע מהאינטראקציה של הספין עם התנע הזוויתי האורביטלי, וכן תיקון יחסותי.

\begin{definition}[צימוד Spin-Orbit]
אינטראקציה בין המומנט המגנטי של הספין של האלקטרון \(\vec{\mu}_{S}=-\frac{e\vec{S}}{m_{e}c}\) והשדה המגנטי האורביטלי של הפרוטון \(\vec{B}\). כלומר התנע הזוויתי האורביטלי והספין לא לגמרי בלתי תלויים, ויש איזשהו איבר אינטראקציה שלהם.

\end{definition}
\begin{lemma}
ניתן לכתוב:
$${\bf L}\cdot{\bf S}={ L}_{0}S_{0}+\frac{1}{2}\left({ L}_{+}S_{-}+{ L}_{-}S_{+}\right)$$

\end{lemma}
\begin{proposition}[אנרגיה של צימוד Spin-Orbit]
האנרגיה המקבלת מהצימוד תהיה:
$$\hat{H}_{S O}=\frac{1}{2m_{e}^{2}c^{2}}\frac{1}{r}\frac{d\hat{V}}{d r}\hat{\vec{S}}\cdot\hat{\vec{L}}$$
כאשר \(V(r)\) נובע מפוטנציאל קולומבי מהצורה \(V(r)=-e\phi(r)\).

\end{proposition}
\begin{proof}
נפתח בעזרת אלקטרודינמיקה קלאסית ובסוף נעביר הכל לאופרטורים. נסתכל על השדה המגנטי של אלקטרון הנע במהירות \(\vec{v}\) במסלול מעגלי. ניתן לעבור למערכת המוחה של האלקרטון, שבו הפרוטון זז במהירות \(-\vec{v}\) במסלול מעגלי סביב האלקטרון. השדה המגנטי שאלקטרון חווה יהיה:
$$\vec{B}=-\frac{1}{c}\vec{v}\times\vec{E}=-\frac{1}{m_{e}c}\vec{p}\times\vec{E}=\frac{1}{m_{e}c}\vec{E}\times\vec{p},$$
כאשר \(\vec{p}=m_{e}\vec{v}\) זה התנע הקווי של האלקטרון ו-\(\vec{E}\) זה השדה החשמלי הנוצר מהשדה קולמבי של הפרטון. אם נניח כי יש פוטנציאל קולומבי \(V(r)=-e\phi(r)\) נקבל כי השדה החשמלי יהיה:
$$\vec{E}(\vec{r})=-\vec{\nabla}\phi(r)=\frac{1}{e}\vec{\nabla}V(r)=\frac{1}{e}\frac{\vec{r}}{r}\frac{d V}{d r}$$
ולכן השדה המגנטי של הגרעין במערכת המנוחה של האלקטרון תהיה:
$${\vec{B}}={\frac{1}{m_{e}c}}{\vec{E}}\times{\vec{p}}={\frac{1}{e m_{e}c}}{\frac{1}{r}}{\frac{d V}{d r}}{\vec{r}}\times{\vec{p}}={\frac{1}{e m_{e}c}}{\frac{1}{r}}{\frac{d V}{d r}}{\vec{L}},$$
כאשר כרגיל \(\vec{L}=\vec{r}\times \vec{p}\) זה התנע הזוויתי האורביטלי של האלקטרון. ולכן האנרגיה שנובעת מהדיפול המגנטי של האלקטרון בשדה המגנטי יהיה:
$${\hat{H}}_{S O}=-{\vec{\mu}}_{S}\cdot{\vec{B}}={\frac{e}{m_{e}c}}{\vec{S}}\cdot{\vec{B}}={\frac{1}{m_{e}^{2}c^{2}}}{\frac{1}{r}}{\frac{d V}{d r}}{\vec{S}}\cdot{\vec{L}}.$$
חזרה למערכת למערכת המעבדה תקטין את האנרגיה בפאקטור של 2(פרסציית טומאס) . ולכן נקבל כי אנרגיית האינטרקציה במערכת המעבדה תהיה:
$$\hat{H}_{S O}=\frac{1}{2m_{e}^{2}c^{2}}\frac{1}{r}\frac{d V}{d r}\vec{S}\cdot\vec{L}$$
כאשר כעת ניתן להעביר את \(S\) ו-\(L\) לאופרטורים ולקבל את הטענה.

\end{proof}
\begin{corollary}
עבור אטום המימן \(V(r)=-\frac{e^{2}}{r}\) ולכן \(\frac{\mathrm{d} V}{\mathrm{d} r}=\frac{e^{2}}{r^{2}}\) ונקבל:
$$\hat{H}_{S O}=\frac{e^{2}}{2m_{e}^{2}c^{2}}\frac{1}{r^{3}}\hat{S}\cdot\hat{L}.$$

\end{corollary}
\begin{proposition}[תרומה של אינטרקציית Spin-Orbit]
התיקון הראשון של האנרגיה הנובע מאינטרקציית Spin-Orbit יהיה:
$$E_{S O}^{(1)}=\frac{|E_{n}^{(0)}|a^{2}}{n}\left[\frac{j(j+1)-l(l+1)-\frac{3}{4}}{l(l+1)(2l+1)}\right]$$
כאשר \(\alpha=\frac{e^{2}}{\hbar c}\approx \frac{1}{137}\) נקרא קבוע המבנה הדק.

\end{proposition}
\begin{proof}
נכתוב את ההמליטוניאן הכולל:
$$\hat{H}=\frac{\hat{\vec{p}}^{2}}{2m_{e}}-\frac{e^{2}}{r}+\frac{e^{2}}{2m_{e}^{2}c^{2}r^{3}}\hat{\vec{S}}\cdot\hat{\vec{L}}=\hat{H}_{0}+\hat{H}_{S O},$$
נרצה לבטא את \(\vec{S}\cdot \vec{L}\) בעזרת הגדלים של התנע \(J^{2},L^{2},S^{2}\). בשביל זה נזכור כי ההגדרה \(J=L+S\) וכן אם נבצע מכפלה פנימית עם עצמו נקבל:
$$J\cdot J=(L+S)\cdot(L+S)=L\cdot L+S\cdot S+2L\cdot S$$
כאשר במונחי אופרטורים נקבל:
$$\hat{J}^{2}=\hat{L}^{2}+\hat{S}^{2}+2\hat{S}\cdot\hat{L}\implies\hat{S}\cdot\hat{L}=\frac{1}{2}\left( \hat{J}^{2}-\hat{L}^{2}-\hat{S}^{2} \right)$$
כעת נחשב את התיקון מסדר ראשון \(E_{SO}^{(1)}\). נציב את הביטוי עבור \(S\cdot J\) בהמילטוניאן ונקבל:
$${\hat{H}}_{S O}={\frac{e^{2}}{2m_{e}^{2}c^{2}r^{3}}}{\hat{S}}\cdot{\hat{L}}={\frac{e^{2}}{2m_{e}^{2}c^{2}r^{3}}}\cdot{\frac{1}{2}}({\hat{J}}^{2}-{\hat{L}}^{2}-{\hat{S}}^{2})={\frac{e^{2}}{4m_{e}^{2}c^{2}r^{3}}}({\hat{J}}^{2}-{\hat{L}}^{2}-{\hat{S}}^{2})$$
ניקח את הערך תצפית של \(\hat{H}_{SO}\) של המצב \(\ket{n\ell jm}\) ונקבל:
$$E_{S O}^{(1)}=\langle n l j m|\hat{H}_{S O}|n l j m\rangle=\left\langle n l j m\left|\frac{e^{2}}{4m_{e}^{2}c^{2}r^{3}}(\hat{J}^{2}-\hat{L}^{2}-\hat{S}^{2})\right|n l j m\right\rangle$$
נפצל לרכיב רדיאלי של \(r^{3}\) ולרכיב זוויתי של \(J^{2},L^{2},S^{2}\) ונקבל:
$$E_{S O}^{(1)}=\frac{e^{2}}{4m_{e}^{2}c^{2}}\left\langle n l\left|\frac{1}{r^{3}}\right|n l\right\rangle\left\langle n l j m\left|(\hat{J}^{2}-\hat{L}^{2}-\hat{S}^{2})\right|n l j m\right\rangle$$
כאשר נשתמש בזה ש-\(\ket{n\ell jm}\) הם מצבים עצמיים של \(J^{2},L^{2},S^{2}\). כלומר שמתקיים:
$$\hat{J}^{2}|n l j m\rangle=j(j+1)\hbar^{2}|n l j m\rangle \qquad \hat{L}^{2}|n l j m\rangle=l(l+1)\hbar^{2}|n l j m\rangle$$
כאשר כיוון שאלקטרונים עם ספין חצי מתקיים:
$$\hat{S}^{2}|n l j m\rangle=s(s+1)\hbar^{2}|n l j m\rangle={{\frac{3}{4}}}\hbar^{2}|n l j m\rangle$$
ולכן הערך תצפית של החלק הזוויתי יהיה:
\begin{gather*}\left\langle n l j m\left|\left( \hat{J}^{2}-\hat{L}^{2}-\hat{S}^{2} \right)\right|n l j m\right\rangle=\left\langle  n l j m|j(j+1)\hbar^{2}-l(l+1)\hbar^{2}-s(s+1)\hbar^{2}|n l j m \right\rangle =\\=\hbar^{2}[j(j+1)-l(l+1)-s(s+1)]=\hbar^{2}\left[ j(j+1)-l(l+1)-\frac{3}{4} \right] 
\end{gather*}
כאשר ערך תצפית הרדיאלי ניתן יהיה נתון על ידי:
$$\left\langle n l\left|\frac{1}{r^{3}}\right|n l\right\rangle=\frac{2}{n^{3}l(l+1)(2l+1)a_{0}^{3}}$$
כאשר לאחר אלגברה ניתן לקבל את הביטוי:
$$E_{S O}^{(1)}=\frac{|E_{n}^{(0)}|\alpha^{2}}{2l(l+1)(2l+1)n}\left[j(j+1)-l(l+1)-\frac{3}{4}\right]$$

\end{proof}
\begin{definition}[תיקון יחסותי של אטום המימן]
למרות שהאפקטיים היחסותיים באטום המימן הוא קטן, הם עדיין ניתנים להבחנה.

\end{definition}
\begin{proposition}[התיקון היחסותי של אטום המימן]
התיקון מסדר ראשון של האנרגיה נתונה על ידי:
$$E_{R}^{(1)}=-\frac{\alpha^{4}m_{e}c^{2}}{8n^{4}}\left(\frac{8n}{2l+1}-3\right)=-\frac{\alpha^{2}|E_{n}^{(0)}|}{4n^{2}}\left(\frac{8n}{2l+1}-3\right)$$

\end{proposition}
\begin{proof}
האנרגיה הקינטית היחסותית של האלקטרון נתונה על ידי:
$$E_{k}=\sqrt{ p^{2}c^{2}+m_{e}^{2}c^{4} }-m_{e}c^{2}$$
כאשר אם נקרב נקבל:
$$\sqrt{\hat{p}^{2}c^{2}+m_{e}^{2}c^{4}}-m_{e}c^{2}\simeq\frac{\hat{p}^{2}}{2m_{e}}-\frac{\hat{p}^{4}}{8m_{e}^{3}c^{2}}+\cdot\cdot\cdot.$$
כאשר מכלילים את הגורם הזה בהמילטוניאן נקבל:
$$\hat{H}=\frac{\hat{p}^{2}}{2m_{e}}-\frac{e^{2}}{r}-\frac{\hat{p}^{4}}{8m_{e}^{3}c^{2}}=\hat{H}_{0}+\hat{H}_{R},$$
מתקיים:
$$E_{R}^{(1)}=\langle n l j m_{j}\mid\hat{H}_{R}\mid n l j m_{j}\rangle=-\frac{1}{8m_{e}^{3}c^{2}}\left\langle n l j m_{j}\left|\hat{p}^{4}\right|n l j m_{j}\right\rangle$$
כאשר אנו יודעים כי:
$$\left\langle n l j m_{j}\left|\hat{P}^{4}\right|n l j m_{j}\right\rangle=\frac{m_{e}^{4}e^{8}}{\hbar^{4}n^{4}}\left(\frac{8n}{2l+1}-3\right)=\frac{a^{4}m_{e}^{4}c^{4}}{n^{4}}\left(\frac{8n}{2l+1}-3\right)$$
ואם נציב ערך זה נקבל:
$$E_{R}^{(1)}=-\frac{\alpha^{4}m_{e}c^{2}}{8n^{4}}\left(\frac{8n}{2l+1}-3\right)=-\frac{\alpha^{2}|E_{n}^{(0)}|}{4n^{2}}\left(\frac{8n}{2l+1}-3\right)$$

\end{proof}
\begin{definition}[המבנה הדק של האטום]
מודל האטום אשר כולל בתוכו את הצימוד ספין-אורביטאל וכן את התיקון היחסותי.

\end{definition}
\begin{corollary}[המילטוניאן המבנה הדק]
ההמילטוניאן של המבנה הדק נתון על ידי:
$$\hat{H}=\hat{H}_{0}+\hat{H}_{F S}=\hat{H}_{0}+\left(\hat{H}_{S O}+\hat{H}_{R}\right)=\frac{\hat{p}^{2}}{2m_{e}}-\frac{e^{2}}{r}+\left(\frac{e^{2}}{2m_{e}^{2}c^{2}r^{3}}\hat{S}\cdot\hat{L}-\frac{\hat{p}^{4}}{8m_{e}^{3}c^{2}}\right)$$

\end{corollary}
\begin{corollary}
$$E_{n j}=E_{n}^{(0)}+E_{F S}^{(1)}=E_{n}^{(0)}\left[1+\frac{\alpha^{2}}{4n^{2}}\left(\frac{4n}{j+\frac{1}{2}}-3\right)\right]$$

\end{corollary}
\begin{proof}
ניתן לכתוב את התיקון לאנרגיה של המבנה הדק על ידי:
$$E_{F S}^{(1)}=E_{S O}^{(1)}+E_{R}^{(1)}=\frac{\alpha^{4}m_{c}c^{2}}{8n^{4}}\left(3-\frac{4n}{j+\frac{1}{2}}\right)=\frac{\alpha^{2}E_{n}^{(0)}}{4n^{2}}\left(\frac{4n}{j+\frac{1}{2}}-3\right)$$
כאשר \(E_{n}^{(0)}=-\frac{\alpha^{2}m_{e}c^{2}}{2n^{2}}\) ו-\(j=\ell+\frac{1}{2}\).

\end{proof}
\begin{summary}
  \begin{itemize}
    \item האינטראקציה בין התנע הזוויתי האורביטלי והספין בפוטנציאל קולומבי ניתן לתיאור על ידי:
$$\hat{H}_{S O}=\frac{1}{2m_{e}^{2}c^{2}}\frac{1}{r}\frac{d V}{d r}\;\hat{S}\cdot\hat{L}=\frac{e^{2}}{2m_{e}^{2}c^{2}}\frac{1}{r^{3}}\,\hat{S}\cdot\hat{L}.$$
כאשר ניתן למצוא את התיקון סדר ראשון על ידי מציאת הערך תצפית של \(\ket{n\ell jm}\) ושימוש ביחס 
$$\hat{S}\cdot\hat{L}=\frac{1}{2}\left(\hat{J}^{2}-\hat{L}^{2}-\hat{S}^{2}\right)$$
    \item התיקון היחסותי נובע ופיתוח האנרגיה הקינטית היחסותית:
$$E_{k}=\sqrt{p^{2}c^{2}+m_{e}^{2}c^{4}}-m_{e}c^{2}\simeq\frac{p^{2}}{2m_{e}}-\frac{p^{4}}{8m_{e}^{3}c^{2}}+\cdots,$$
אשר מובילה לגורם נוסף בהמילטוניאן \(\hat{H}_{R}=-\frac{\hat{p}^{4}}{8m_{e}^{3}c^{2}}\) כך שהתיקון מסדר ראשון נתון על ידי:
$$E_{R}^{(1)}=-\frac{\alpha^{4}m_{e}c^{2}}{8n^{4}}\left(\frac{8n}{2\ell+1}-3\right)=-\frac{\alpha^{2}|E_{n}^{(0)}|}{4n^{2}}\left(\frac{8n}{2\ell+1}-3\right)$$
כאשר \(\alpha=\frac{e^{2}}{\hbar c}\) קבוע המבנה הדק.
    \item אם נשלב את שתי האפקטים נקבל את המילטונאן הבנה הדק:
$$\hat{H}=\hat{H}_{0}+\hat{H}_{F S}=\frac{\hat{p}^{2}}{2m_{e}}-\frac{e^{2}}{r}+\left(\frac{e^{2}}{2m_{e}^{2}c^{2}r^{3}}\,\hat{S}\cdot\hat{L}-\frac{\hat{p}^{4}}{8m_{e}^{3}c^{2}}\right)$$
כך שסך התיקון נתון על ידי:
$$E_{n j}=E_{n}^{(0)}\left[1+\frac{\alpha^{2}}{4n^{2}}\left(\frac{4n}{j+\frac{1}{2}}-3\right)\right]$$
עם \(E_{n}^{(0)}=-\frac{\alpha^{2}m_{e}c^{2}}{2n^{2}},j=\ell\pm\frac{1}{2}\).
  \end{itemize}
\end{summary}
\section{תמונת האינטרקציה}

\begin{reminder}[תמונת שרדינגר]
המצבים הקוונטים תלויים בזמן, אבל האופרטורים לא. ומתקיימת משוואת שרדינגר:
$$i\hbar{\frac{\mathrm{d}}{\mathrm{d} t}}|\psi(t)\rangle={\hat{H}}|\psi(t)\rangle$$
כאשר ניתן לחלופין לתאר את הקידום בזמן של המצב על ידי אופרטור קידום בזמן \(U(t,t_{0})\) אשר מקיים:
$$|\psi(t)\rangle=\hat{U}(t,t_{0})|\psi(t_{0})\rangle$$
כאשר \(\hat{U}(t,t_{0})=e^{-i(t-t_{0})\hat{H}/\hbar}.\) הוא אופרטור אוניטרי אשר מקיים:
$$\begin{array}{c}{{\hat{U}(t,t)=I,}}\\ {{\hat{U}^{\dagger}(t,t_{0})=\hat{U}^{-1}(t,t_{0})=\hat{U}(t_{0},t),}}\\ {{\hat{U}(t_{1},t_{2})\hat{U}(t_{2},t_{3})=\hat{U}(t_{1},t_{3}).}}\end{array}$$

\end{reminder}
\begin{definition}[תמונת היזנברג]
דרך להסתכל על מערכות כך שאין תלות הזמנים של מצבים קוונטים, ורק האופרטורים משתנים בזמן. כלומר כל ווקטור בתמונת היזנברג הוא קפוא בזמן.

\end{definition}
\begin{proposition}
ניתן לקבל ווקטור \(\ket{\psi(t)}_{H}\) בתמונת היזנברג מווקטור \(\ket{\psi(t)}\) מתמונת שרדינגר על ידי:
$$|\psi(t)\rangle_{H}=\hat{U}^{\dagger}(t)|\psi(t)\rangle=e^{ itH/\hbar }\ket{\psi(t)} =|\psi(0)\rangle$$
כלומר ווקטור \(\ket{\psi(t)}_{H}\) בתמונת היזנברג בכל זמן \(t\) יהיה שווה לווקטור \(\ket{\psi(0)}\) בתמונת שרדינגר. ואכן לא משתנה בזמן.

\end{proposition}
\begin{corollary}
עבור כל ווקטור בתמונת היזנברג מתקיים:
$${\frac{d}{d t}}|\psi\rangle_{H}=0$$

\end{corollary}
\begin{proposition}
הערך תצפית של אופרטור \(A\) זהה בשתי התמונות. זה תוצאה שחייבת להיות נכונה כיוון שהפיזיקה לא משתנה בין שתי התמונות וערך התצפית זה תוצאה פיזיקלית.

\end{proposition}
\begin{proof}
$$\langle\psi(t)|\hat{A}|\psi(t)\rangle=\langle\psi(0)|e^{i t H/\hbar}\hat{A}e^{-i t H/\hbar}|\psi(0)\rangle=\langle\psi(0)|\hat{A}_{H}(t)|\psi(0)\rangle={}_{H}\langle\psi|\hat{A}_{H}(t)|\psi\rangle_{H}$$
כאשר השתמשנו בזה ש-\(\ket{\psi}_{H}=\ket{\psi(0)}\).

\end{proof}
\begin{proposition}[משוואת היזנברג]
ניתן לתאר קידום בזמן בתמונת היזנברג על ידי:
$$\frac{d\hat{A}_{H}}{d t}=\frac{1}{i\hbar}\left[\hat{A}_{H},\hat{H}\right]$$

\end{proposition}
\begin{proof}
\begin{gather*}\frac{d\hat{A}_{H}(t)}{d t}\,=\,\frac{\partial\hat{U}^{\dagger}(t)}{d t}\hat{A}\hat{U}_{(t)}+\hat{U}^{\dagger}(t)\hat{A}\frac{\partial\hat{U}(t)}{\partial t}=\\=-\frac{1}{i\hbar}\hat{U}^{\dagger}\hat{H}\hat{U}\hat{U}^{\dagger}\hat{A}\hat{U}+\frac{1}{i\hbar}\hat{U}^{\dagger}\hat{A}\hat{U}\hat{U}^{\dagger}\hat{H}\hat{U}=\,\frac{1}{i\hbar}\left[\hat{A}_{H},\hat{U}^{\dagger}\hat{H}\hat{U}\right] 
\end{gather*}
כאשר כיוון ש-\(U\) ו-\(H\) מתחלפים נקבל את הטענה.

\end{proof}
\begin{definition}[תמונת האינטרקציה]
שימושי לתאר תופעות אשר תלויות בזמן. בתמונה זו גם המצב הקוונטי וגם האופרטורים משתנים בזמן. עבור המילטוניאן מהצורה \(H(t)=H_{0}+V(t)\) כאשר \(H_{0}\) לא תלוי בזמן ב-\(V\) יכול להיות תלוי בזמן המצב הקוונטי נתון על ידי:
$$|\psi(t)\rangle_{I}=e^{i t\hat{H}_{0}/\hbar}|\psi(t)\rangle$$

\end{definition}
\begin{proposition}[משוואת שרדינגר בתמונת האינטרקציה]
מצבים קוונטים בתמונת האינטרקציה מקיימים:
$$i\hbar\frac{d|\psi(t)\rangle_{I}}{d t}=\hat{V}_{I}(t)|\psi(t)\rangle_{I}$$

\end{proposition}
\begin{proof}
נמצא ראשית את אופרטור הקידום בזמן בתמונת האינטרקציה:
\begin{gather*}i\hbar\frac{d|\psi(t)\rangle_{I}}{d t}\;=\;-\hat{H}_{0}e^{i t\hat{H}_{0}/\hbar}|\psi(t)\rangle+e^{i t\hat{H}_{0}/\hbar}\left(i\hbar\frac{d|\psi(t)\rangle}{d t}\right)\\=\,-\hat{H}_{0}|\psi(t)\rangle_{I}+e^{i t\hat{H}_{0}/\hbar}\hat{H}|\psi(t)\rangle 
\end{gather*}
כיוון ש-\(H(t)=H_{0}+V(t)\) וכן מתקיים:
$$e^{i\,H_{0}t/\hbar}\hat{V}=\left(e^{i\,t\,\hat{H}_{0}/\hbar}\hat{V}e^{-i\,t\,\hat{H}_{0}/\hbar}\right)e^{i\,t\,\hat{H}_{0}/\hbar}=\hat{V}_{I}(t)e^{i\,t\,\hat{H}_{0}/\hbar},$$
כאשר \(\hat{V}_{I}(t)=e^{i t\hat{H}_{0}/\hbar}\hat{V}e^{-i t\hat{H}_{0}/\hbar},\) נקבל:
$$i\hbar\frac{d|\psi(t)\rangle_{I}}{d t}=-\hat{H}_{0}|\psi(t)\rangle_{I}+\hat{H}_{0}e^{i t\hat{H}_{0}/\hbar}|\psi(t)\rangle+\hat{V}_{I}(t)e^{i t\hat{H}_{0}/\hbar}|\psi(t)\rangle,$$
כלומר:
$$i\hbar\frac{d|\psi(t)\rangle_{I}}{d t}=\hat{V}_{I}(t)|\psi(t)\rangle_{I}$$

\end{proof}
\begin{proposition}[אופרטורים בתמונת האינטרקציה]
ניתן לכתוב את האופרטורים בתמונת האינטרקציה בעזרת האופרטורים בתמונת שרדינגר על ידי:
$$\hat{A}_{I}(t)=e^{i\hat{H}_{0}t/\hbar}\hat{A}e^{-i\hat{H}_{0}t/\hbar}$$

\end{proposition}
\begin{proposition}[משוואת האיזנברג בתמונת האינטרקציה]
האופרטורים בתמונת האינטרקציה מקיימים:
$$\frac{d\hat{A}_{I}(t)}{d t}=\frac{1}{i\hbar}\left[\hat{A}_{I}(t),\;\hat{H}_{0}\right]$$

\end{proposition}
\begin{corollary}
בתמונת האינטראקציה קידום בזמן של המצב נקבע על ידי \(V_{I}\) כאשר קידום בזמן של האופרטור נקבע על ידי \(H_{0}\).

\end{corollary}
\begin{proposition}
אופרטור הקידום בזמן בתמונת האינטרקציה מקיים:
$$i\hbar\frac{d\hat{U}_{I}(t,t_{i})}{d t}=\hat{V}_{I}(t)\hat{U}_{I}(t,t_{i})$$

\end{proposition}
\begin{proof}
אופרטור הקידום בזמן מקיים:
$$\mid\Psi(t)\rangle={U}(t,t_{i})\mid\Psi(t_{i})\rangle$$
כך שניתן לכתוב בתמונת האינטרקציה:
$$\mid\Psi(t)\rangle_{I}=e^{i\,t\hat{H}_{0}/\hbar}\mid\Psi(t)\rangle=e^{i\,t\hat{H}_{0}/\hbar}\hat{U}(t,t_{i})\mid\Psi(t_{i})\rangle=e^{i\,t\hat{H}_{0}/\hbar}\hat{U}(t,t_{i})e^{-i\,t\hat{H}_{0}t/\hbar}\mid\Psi(t_{i})\rangle_{I}$$
כלומר:
$$\mid\Psi(t)\rangle_{I}=\hat{U}_{I}(t,t_{i})\mid\Psi(t_{i})\rangle_{I},$$
כאשר אופרטור הקידום בזמן בתמונת האינטרקציה מתקבל על ידי הקידום בזמן של אופרטור הקידום בזמן 
$$\hat{U}_{I}(t,t_{i})=e^{i t\hat{H}_{0}/\hbar}\hat{U}(t,t_{i})e^{-i\hat{H}_{0}t_{i}/\hbar}$$
כעת נציב את \(\ket{\Psi(t)}_{I}\) במשוואת שרודינגר של תמונת האינטרקציה ונקבל:
$$i\hbar\frac{d\hat{U}_{I}(t,t_{i})}{d t}=\hat{V}_{I}(t)\hat{U}_{I}(t,t_{i})$$

\end{proof}
\begin{corollary}
עבור התנאי התחלה \(U_{I}(t_{i},t_{i})\) הפתרון של \(U_{I}\) נתון על ידי המשוואה האינטרגלית:
$$\hat{U}_{I}(t,t_{i})=U_{I}(t_{i},t_{i}) -\frac{i}{\hbar}\int_{t_{i}}^{t}\hat{V}_{I}\left( t^{\prime} \right)\hat{U}_{I}\left( t^{\prime},t_{i} \right)\,d t^{\prime}$$

\end{corollary}
\begin{proof}
נתחיל מהמשוואה הדיפרנציאלית:
$$\frac{d}{d t}\hat{U}_{I}(t,t_{i})=-\frac{i}{\hbar}\hat{V}_{I}(t)\hat{U}_{I}(t,t_{i}).$$
ניקח אינטגרל על שתי האגפים מ-\(t_{i}\) ל-\(t\). עבור הנגזרת(אגף שמאל) נקבל:
$$\int_{t_{i}}^{t}{\frac{d}{d t^{\prime}}}{\hat{U}}_{I}\left( t^{\prime},t_{i} \right)\,d t^{\prime}={\hat{U}}_{I}(t,t_{i})-{\hat{U}}_{I}(t_{i},t_{i}) $$
עבור אגף שמאל נקבל:
$$-\frac{i}{\hbar}\int_{t_{i}}^{t}\hat{V}_{I}(t^{\prime})\hat{U}_{I}(t^{\prime},t_{i})\,d t^{\prime}.$$
ניתן להעביר אגפים ולקבל:
$$\hat{U}_{I}(t,t_{i})=U_{I}(t_{i},t_{i}) -\frac{i}{\hbar}\int_{t_{i}}^{t}\hat{V}_{I}\left( t^{\prime} \right)\hat{U}_{I}\left( t^{\prime},t_{i} \right)\,d t^{\prime}.$$

\end{proof}
\begin{corollary}
אופרטור הקידום בזמן בתמונת האינטראקציה \(U_{I}(t,t_{i})\) תחת התנאי התחלה \(U_{I}(t_{i},t_{i})=\mathbb{1}\) נתון על ידי:
$$\hat{U}_{I}(t,t_{i})=\mathbb{1} -\frac{i}{\hbar}\int_{t_{i}}^{t}\hat{V}_{I}\left( t^{\prime} \right)\hat{U}_{I}\left( t^{\prime},t_{i} \right)\,\mathrm{d}t^{\prime}$$

\end{corollary}
\begin{proposition}
הקירוב מסדר ראשון של אופרטור הקידום בזמן יתקבל על ידי הצבה \(U_{I}(t',t_{i})=\mathbb{1}\) בתוך האינטגרל כך שנקבל:
$$\hat{U}_{I}^{(1)}(t,t_{i})=\mathbb{1} -\left( i/\hbar \right)\int_{t_{i}}^{t}\hat{V}_{I}\left( t^{\prime} \right)\,\mathrm{d}t^{\prime}$$

\end{proposition}
\begin{proposition}
הקירוב מסדר שני של אופרטור קידום בזמן מתקבל על ידי הצבה \(U_{I}(t',t_{i})=U_{I}^{(1)}(t',t_{i})\) בתוך האינטגרל כך שנקבל:
$$\hat{U}_{I}^{(2)}(t,t_{i})=\mathbb{1} -\frac{i}{\hbar}\int_{t_{i}}^{t}\hat{V}_{I}\left( t^{\prime} \right)\,\mathrm{d}t^{\prime}+\left(-\frac{i}{\hbar}\right)^{2}\int_{t_{i}}^{t}\hat{V}_{I}(t_{1})\,\mathrm{d}t_{1}\int_{t_{i}}^{t_{1}}\hat{V}_{I}(t_{2})\,\mathrm{d}t_{2}$$

\end{proposition}
\begin{corollary}
ניתן לחזור על התהליך על ידי הצבה \(U_{I}(t',t_{i})=U_{I}^{(2)}(t',t_{i})\) בתוך האינטגרל וכן הלאה כך שנקבל במקרה הכללי:
\begin{gather*}\hat{U}_{I}(t,t_{i})=\mathbb{1} -\frac{i}{\hbar}\int_{t_{i}}^{t}\hat{V}_{I}\left( t^{\prime} \right)\mathrm{d}t^{\prime}+\left(-\frac{i}{\hbar}\right)^{2}\int_{t_{i}}^{t}\hat{V}_{I}(t_{1})\mathrm{d}t_{1}\int_{t_{i}}^{t_{1}}\hat{V}_{I}(t_{2})\mathrm{d}t_{2}+\cdots \\ +\;\left(-\frac{i}{\hbar}\right)^{n}\!\!\int_{t_{i}}^{t}\hat{V}_{I}(t_{1})\mathrm{d}t_{1}\!\int_{t_{i}}^{t_{1}}\hat{V}_{I}(t_{2})\mathrm{d}t_{2}\!\int_{t_{i}}^{t_{2}}\hat{V}_{I}(t_{3})\mathrm{d}t_{3}\cdots\int_{t_{i}}^{t_{n-1}}\hat{V}_{I}(t_{n})\mathrm{d}t_{n}+\cdots 
\end{gather*}
כאשר טור זה נקרא טור דייסון(Dyson series).

\end{corollary}
\begin{summary}
  \begin{itemize}
    \item בתמונת שרדינגר האופרטורים לא משתנים בזמן והמצבים הקוונטים משתנים בזמן לפי משוואת שרדינגר - \(i\hbar{\frac{d}{d t}}|\psi(t)\rangle={\hat{H}}|\psi(t)\rangle\).
    \item בתמונת היזנברג המצבים הקוונטים קפואים בזמן והאופרטורים משתנים בזמן לפי משוואת איזנברג - \(\frac{d\hat{A}_{H}}{d t}=\frac{1}{i\hbar}\left[\hat{A}_{H},\hat{H}\right]\).
    \item בתמונת האינטראקציה גם המצבים הקוונטים וגם האופרטורים משתנים בזמן.
    \item עבור המילטוניאן מהצורה \(H(t)=H_{0}+V(t)\) כאשר \(H_{0}\) לא תלוי בזמן ניתן לכתוב מצב קוונטי בתמונת שרדינגר על ידי \(|\psi(t)\rangle_{I}=e^{i t\hat{H}_{0}/\hbar}|\psi(t)\rangle\).
    \item בתמונת האינטראקציה המצבים מתקדמים בזמן על ידי \(i\hbar\frac{d|\psi(t)\rangle_{I}}{d t}=\hat{V}_{I}(t)|\psi(t)\rangle_{I}\) והאופרטורים מתקדמים בזמן כך שמקיימים \(\frac{d\hat{A}_{I}(t)}{d t}=\frac{1}{i\hbar}\left[\hat{A}_{I}(t),\;\hat{H}_{0}\right]\).
    \item ניתן לכתוב את אופרטור הקידום בזמן בתמונת האינטראקציה \(U_{I}(t,t_{i})\) בעזרת סדרת דייסון:
$$U_{I}(t,0)=1-\frac{i\lambda}{\hbar}\int_{0}^{t}V_{I}(t_{1})\,d t_{1}+\left(-\frac{i\lambda}{\hbar}\right)^{2}\int_{0}^{t}d t_{1}\int_{0}^{t_{1}}d t_{2}\,V_{I}(t_{1})V_{I}(t_{2})+\cdots$$
  \end{itemize}
\end{summary}
\section{תורת ההפרעות התלויה בזמן}

\begin{definition}[תורת ההפרעות התלויה בזמן]
שיטה המאפשרת לנו למצוא את הפתרון בצורה מקורבת של מערכת עם המילטוניאן מהצורה:
$$H= H_{0}+ \lambda V(t)\qquad \lambda\ll 1$$
כאשר אנחנו יודעים את הפתרון של מערכת עם המילטוניאן \(H_{0}\), ו-\(V(t)\) הפרעה קטנה.

\end{definition}
\begin{symbolize}
נסמן את המצבים העצמיים של \(H_{0}\) - כלומר הרכיב שלא תלוי בזמן ב-\(\ket{\psi_{n}}\) ואת האנרגיות העצמיות המתאימות ב-\(E_{n}\) כך שמתקיים:
$$\hat{H}_{0}|\psi_{n}\rangle=E_{n}|\psi_{n}\rangle$$
כאשר ללא ההפרעה המצבים מתקדמים בזמן בצורה הבאה:
$$|\Psi(t)\rangle=\sum_{n}c_{n}(t)\,e^{-i E_{n}t/\hbar}|\psi_{n}\rangle$$
כך שאם המערכת מתחילה במצה \(\ket{\psi_{i}}\) אז \(c_{n}(0)=\delta_{n,i}\).

\end{symbolize}
\begin{symbolize}
ניתן לכתוב את המקדמים \(c_{n}(t)\) על ידי טור חזקות לפי \(\lambda\):
$$c_{n}(t)=c_{n}^{(0)}(t)+\lambda\,c_{n}^{(1)}(t)+\lambda^{2}\,c_{n}^{(2)}(t)+\cdots$$

\end{symbolize}
כדי להתמודד עם בתלות הזמנית של \(V(t)\) תוך שימור הקידום של \(H_{0}\) נרצה לעבור לתמונת האינטראקציה.

\begin{reminder}
המצב \(\ket{\Psi_{I}}\) בתמונת האינטרקציה נתון על ידי:
$$\left|\Psi_{I}(t)\right\rangle=e^{i H_{0}t/\hbar}\left|\Psi(t)\right\rangle$$

\end{reminder}
\begin{corollary}
המצב \(\ket{\Psi(t)}\) בתמונת האינטראקציה נתון על ידי:
$$|\Psi_{I}(t)\rangle=\sum_{n}c_{n}(t)\,|\psi_{n}\rangle$$

\end{corollary}
\begin{proof}
אם נעביר את המצב \(|\Psi(t)\rangle=\sum_{n}c_{n}(t)\,e^{-i E_{n}t/\hbar}|\psi_{n}\rangle\) לתמונת האינטראקציה נקבל:
$$|\Psi_{I}(t)\rangle=e^{i H_{0}t/\hbar}\sum_{n}c_{n}(t)e^{-i E_{n}t/\hbar}|\psi_{n}\rangle$$
כאשר נזכור כי מצב עצמי מקיים \(e^{iH_{0} t/\hbar}=e^{ iE_{n}t/\hbar }\) ולכן הגורמים האקספוננציאלים מצטמצמים ונקבל:
$$|\Psi_{I}(t)\rangle=\sum_{n}c_{n}(t)\,|\psi_{n}\rangle$$

\end{proof}
\begin{symbolize}
ניתן לכתוב את \(c_{n}(t)\) ו- \(\ket{\Psi_{I}(t)}\) בתור טור חזקות לפי \(\lambda\):
\begin{gather*}c_{n}(t)=c_{n}^{(0)}(t)+\lambda\,c_{n}^{(1)}(t)+\lambda^{2}\,c_{n}^{(2)}(t)+\cdots\\|\Psi_{I}(t)\rangle=|\Psi_{I}^{(0)}(t)\rangle+\lambda\,|\Psi_{I}^{(1)}(t)\rangle+\lambda^{2}\,|\Psi_{I}^{(2)}(t)\rangle+\cdots  \end{gather}
$$
כאשר \(\ket{\Psi^{(i)}_{I}(t)}\) נקרא התיקון ה-\(i\) לפונקציית גל.

\end{symbolize}
\begin{corollary}
מהשוואת המקדמים נקבל:
$$|\Psi_{I}^{(i)}(t)\rangle=\sum_{n}c_{n}^{(i)}(t)|\psi_{n}\rangle$$
ולכן בתמונת שרדינגר:
$$|\Psi^{(i)}(t)\rangle=\sum_{n}c_{n}^{(i)}(t)e^{-i E_{n}t}|\psi_{n}\rangle$$

\end{corollary}
\begin{proof}
אם מציבים:
$$c_{n}(t)=c_{n}^{(0)}(t)+\lambda\,c_{n}^{(1)}(t)+\lambda^{2}\,c_{n}^{(2)}(t)+\cdots$$
בביטוי:
$$|\Psi_{I}(t)\rangle=\sum_{n}c_{n}(t)|\psi_{n}\rangle$$
ניתן לפרק לטור חזקות ולהשוואות את החזקות של \(|\Psi_{I}^{(0)}(t)\rangle+\lambda\,|\Psi_{I}^{(1)}(t)\rangle+\lambda^{2}\,|\Psi_{I}^{(2)}(t)\rangle+\cdots\).

\end{proof}
\begin{proposition}
מתקיים:
$$i\hbar\,\frac{d}{\mathrm{d} t}|\Psi_{I}^{(0)}(t)\rangle=0$$
כאשר עבור \(i\geq 0\) נקבל:
$$i\hbar\,\frac{d}{\mathrm{d} t}|\Psi_{I}^{(i+1)}(t)\rangle=V_{I}(t)\,|\Psi_{I}^{(i)}(t)\rangle$$

\end{proposition}
\begin{proof}
משוואת שרדינגר בתמונת האינטראקציה מקיימת:
$$i\hbar\,\frac{d}{\mathrm{d} t}|\Psi_{I}(t)\rangle=\lambda\,V_{I}(t)\,|\Psi_{I}(t)\rangle$$
כאשר אם נציב את הפיתוח נקבל:
$$i\hbar\,\frac{d}{dt}\Big{(}|\Psi_{I}^{(0)}(t)\rangle+\lambda|\Psi_{I}^{(1)}(t)\rangle+\lambda^{2}\,|\Psi_{I}^{(2)}(t)\rangle+\cdots\Big{)}=\lambda\,V_{I}(t)\Big{(}|\Psi_{I}^{(0)}(t)\rangle+\lambda\,|\Psi_{I}^{(1)}(t)\rangle+\lambda^{2}\,|\Psi_{I}^{(2)}(t)\rangle+\cdots\Big{)}$$
ונאחד גורמים לפי \(\lambda\). הגורם ה-0 מקיים:
$$i\hbar\,\frac{d}{\mathrm{d} t}|\Psi_{I}^{(0)}(t)\rangle=0$$
כאשר הגורם ה-1 באגף שמאל יהיה \(i\hbar \frac{\mathrm{d} }{\mathrm{d} t}\ket{\Psi_{I}^{(2)}(t)}\) כאשר באגף ימין נקבל \(V_{I}(t)\ket{\Psi_{I}^{(1)}(t)}\) ולכן נקבל:
$$i\hbar\,\frac{d}{\mathrm{d} t}|\Psi_{I}^{(2)}(t)\rangle=V_{I}(t)\,|\Psi_{I}^{(1)}(t)\rangle.$$
הדפוס הזה ממשיך לחזקות גבוהות יותר.

\end{proof}
\begin{proposition}[קירוב סדר אפס]
תחת קירוב סדר אפס נקבל:
$$|\Psi_{I}^{(0)}(t)\rangle=|\psi_{i}\rangle \qquad c_{n}^{(0)}(t)=c_{n}(0)=\delta_{n,i}$$

\end{proposition}
\begin{proof}
ראינו כי \(i\hbar\,\frac{d}{\mathrm{d} t}|\Psi_{I}^{(0)}(t)\rangle=0\) ולכן:
$$|\Psi_{I}^{(0)}(t)\rangle=|\Psi_{I}^{(0)}(0)\rangle$$
כאשר כיוון שהמערכת במצב התחלתי \(\ket{\psi_{i}}\) נקבל \(\ket{\Psi_{I}^{(0)}}=\ket{\psi_{i}}\) כלומר \(c_{n}^{(0)}(t)=c_{n}(0)=\delta_{n,i}\).

\end{proof}
\begin{proposition}[קירוב סדר ראשון]
תחת קירוב סדר ראשון נקבל:
$$c_{n}^{(1)}(t)=-\frac{i}{\hbar}\int_{0}^{t}V_{n i}\left( t^{\prime} \right)\,e^{i\omega_{n i}t^{\prime}}\,\mathrm{d} t^{\prime}$$
ולכן:
$$|\Psi_{I}^{(1)}(t)\rangle=-\frac{i}{\hbar}\int_{0}^{t}\mathrm{d} t^{\prime}\,V_{I}(t^{\prime})|\psi_{i}\rangle=\sum_{n}c_{n}^{(1)}(t)e^{-i E_{n}t/\hbar}|\psi_{n}\rangle$$

\end{proposition}
\begin{proof}
ראינו כי מתקיים:
$$i\hbar\,\frac{d}{\mathrm{d} t}|\Psi_{I}^{(1)}(t)\rangle=V_{I}(t)|\Psi_{I}^{(0)}(t)\rangle.$$
אם נטיל על \(\bra{\psi_{n}}\) כאשר נזכור כי \(\ket{\Psi_{I}^{(0)}(t)}=\ket{\psi_{i}}\) מהטענה הקודמת נקבל:
$$(*)\qquad i\hbar\,\frac{d}{\mathrm{d} t}\left[c_{n}^{(1)}(t)\right]=\left\langle \psi_{n}|V_{I}(t)|\psi_{i} \right\rangle$$
כאשר עבור אגף שמאל נזכור כי \(V_{I}(t)=e^{i H_{0}t/\hbar}\,V(t)\,e^{-i H_{0}t/\hbar}\) ולכן:
$$\langle\psi_{n}|V_{I}(t)|\psi_{i}\rangle=\langle\psi_{n}|e^{i H_{0}t/\hbar}V(t)e^{-i H_{0}t/\hbar}|\psi_{i}\rangle$$
כאשר כיוון ש-\(H_{0}\ket{\psi_{n}}=E_{n}\ket{\psi_{n}}\) ו-\(H_{0}\ket{\psi_{i}}=E_{i}\ket{\psi_{i}}\) נקבל:
$$\langle\psi_{n}|V_{I}(t)|\psi_{i}\rangle=e^{i(E_{n}-E_{i})t/\hbar}\langle\psi_{n}|V(t)|\psi_{i}\rangle\equiv V_{n i}(t)e^{i\omega_{n i}t}$$
כאשר סימנו \(\omega_{ni}=\frac{E_{n}-E_{i}}{\hbar}\). נפעיל על אגף ימין של \((*)\) נגזרת מכפלה ונקבל:
$$i\hbar\,\frac{d}{\mathrm{d} t}c_{n}^{(1)}(t)=V_{n i}(t)e^{i\omega_{n i}t}$$
כיוון שאנו מניחים כי ב-\(t=0\) אין תיקון מסדר ראשון מתקיים \(c_{n}^{(1)}(t)=0\) ולכן:
$$c_{n}^{(1)}(t)=\frac{1}{i\hbar}\int_{0}^{t}V_{n i}\left( t^{\prime} \right)\,e^{i\omega_{n i}t^{\prime}}\,\mathrm{d} t^{\prime}\implies c_{n}^{(1)}(t)=-\frac{i}{\hbar}\int_{0}^{t}V_{n i}\left( t^{\prime} \right)\,e^{i\omega_{n i}t^{\prime}}\,\mathrm{d} t^{\prime}$$

\end{proof}
\begin{example}[הפרעה גאוסיאנית לפוטנציאל הרמוני]
נניח חלקיק עם מסה \(m\) הנע בפוטנציאל חד מימד. ההמילטוניאן של מערכת זו נתונה על ידי:
$$H_{0}=\frac{p^{2}}{2m}+\frac{1}{2}m\omega^{2}x^{2}$$
המערכת נתונה להפרעה התלויה בזמן \(V(t)=A_{0}xe^{ -t^{2} / 2 }\) אשר פועל מזמן \(t=-\infty\) עד \(t=\infty\). נתון כי המערכת מתחילה ב-\(t=-\infty\) במצב היסוד \(\ket{0}\). נחשב את ההסתברות מסדר ראשון שיעבור למצב \(\ket{1}\) ב-\(t=\infty\). אמפליטודת המעבר נתונה על ידי:
$$c_{0\rightarrow1}^{(1)}=\frac{-i}{\hbar}\int_{-\infty}^{+\infty}d t^{\prime}\bra{1} V\ket{0} (t^{\prime})e^{i\omega_{10}t^{\prime}}$$
נדרש לחשב את אלמנטי המטריצה של \(V\):
$$V_{10}=\bra{1} A_{0}Xe^{ -t^{2}/\tau^{2} }\ket{0} =\bra{1} X\ket{0} A_{0}e^{ -t^{2}/\tau^{2} }=\left( -\frac{i}{\hbar}\int_{-\infty}^{\infty} A_{0}e^{ -t^{2}/\tau^{2} }e^{ i\omega_{10}t } \, dt \right)\braket{ 1|X | 0 }$$
כאשר נזכור כעת כי \(E_{n}=\frac{1}{2}\hbar \omega+n\hbar \omega\). ולכן:
$$\omega_{10}=\frac{E_{1}-E_{0}}{\hbar}=\hbar \frac{\omega}{\hbar}=\omega$$
ולכן נקבל התמרת פורייה של גאוסיאן. את \(\braket{ 1|X | 0 }\) ניתן לחשב באופן נאייבי:
$$\langle 1|X|0 \rangle =\int_{-\infty}^{\infty} x\psi_{1}^{*}(x)\psi_{0}(x) \, \mathrm{d}x $$
או לחלופין להשתמש באופרטורי העלה והורדה \(a=\frac{X-iP}{\sqrt{ 2 }}\) ו-\(a^{\dagger}=\frac{X+iP}{\sqrt{ 2 }}\) כך שמתקיים \(a^{\dagger}\ket{n}=\sqrt{ n+1 }\ket{n+1}\) ו-\(a\ket{n}=\sqrt{ a }\ket{n-1}\) ולקבל:
$$\braket{ 1 | X | 0 } =\sqrt{ \frac{\hbar}{2m\omega} }\braket{ 1 | a+a^{\dagger}|0 } =\sqrt{ \frac{\hbar}{2m\omega} }  $$
כעת כל מה שנותר זה להכפיל בין ההתמרת פורייה של גאוסיאן לערך תצפית:
$$\!\!\!\!\!c_{0\rightarrow1}\!=\!\frac{-iA_{0}}{\hbar}\!\sqrt{\frac{\hbar}{2m\omega}}\!\!\!\int\limits_{-\infty}^{+\infty}\!\!\!d t^{\prime}e^{-t^{2}/\tau^{2}}\!e^{i\omega_{10}t^{\prime}}\!\!=\!-i A_{0}\tau\!\sqrt{\frac{\pi}{2m\hbar\omega}}e^{-\frac{\omega^{2}\tau^{2}}{4}}$$
ונקבל לאחר העלה בריבוע \(P_{0\rightarrow1}=|A_{0}|^{2}\frac{\pi\tau^{2}}{2m\hbar\omega}e^{-\omega^{2}\tau^{2}/2}\).

\end{example}
\begin{reminder}[טור דייסון]
אופרטור הקידום בזמן בתמונת האינטראקציה עבור פוטנציאל \(\lambda V\) יהיה:
$$U_{I}(t,0)=1-\frac{i\lambda}{\hbar}\int_{0}^{t}V_{I}(t_{1})\,\mathrm{d} t_{1}+\left(-\frac{i\lambda}{\hbar}\right)^{2}\int_{0}^{t}\mathrm{d} t_{1}\int_{0}^{t_{1}}\mathrm{d} t_{2}\,V_{I}(t_{1})V_{I}(t_{2})+\cdots$$

\end{reminder}
\begin{proposition}[קשר של טור דייסון למקדמים \(c_{n}^{(i)}(t)\)]
אם נפעיל את טור דייסון על \(\ket{\Psi_{I}(0)}=\ket{\psi_{i}}\) נקבל את \(\ket{\Psi_{I}(t)}\) כאשר אם נפריד לפי חזקות נקבל ישירות את הפונקציות גל:
\begin{gather*}|\Psi_{I}^{(0)}(t)\rangle=1\,|\psi_{i}\rangle=|\psi_{i}\rangle\\ |\Psi_{I}^{(1)}(t)\rangle=-\frac{i\lambda}{\hbar}\int_{0}^{t}\mathrm{d} t_{1}\,V_{I}(t_{1})|\psi_{i}\rangle\\|\Psi_{I}^{(2)}(t)\rangle=\left(-\frac{i\lambda}{\hbar}\right)^{2}\int_{0}^{t}\mathrm{d} t_{1}\int_{0}^{t_{1}}\mathrm{d} t_{2}\,V_{I}(t_{1})V_{I}(t_{2})|\psi_{i}\rangle 
\end{gather*}
כאשר ניתן כעת להטיל על \(\bra{\psi_{n}}\) ולקבל את המקדמים:
\begin{gather*}c_{n}^{(0)}(t)=\delta_{n,i}\\ c_{n}^{(1)}(t)=-\frac{i}{\hbar}\int_{0}^{t}\mathrm{d} t_{1}\,V_{n i}(t_{1})e^{i\omega_{n i}t_{1}}\\c_{n}^{(2)}(t)=\left(-\frac{i}{\hbar}\right)^{2}\sum_{m}\int_{0}^{t}\mathrm{d} t_{1}\int_{0}^{t_{1}}\mathrm{d} t_{2}\,V_{n m}(t_{1})e^{i\omega_{m n}t_{1}}V_{m i}(t_{2})e^{i\omega_{m i}t_{2}} 
\end{gather*}

\end{proposition}
\begin{definition}[הסתברות מעבר]
ההסתברות לעבור ממצב לא מופרע \(\ket{\psi_{i}}\) למצב לא מופרע אחר \(\ket{\psi_{f}}\).

\end{definition}
\begin{proposition}
הסתברות המעבר תהיה:
$$P_{i f}(t)=\left|c_{f}^{(0)}+c_{f}^{(1)}(t)+c_{f}^{(2)}(t)+\cdots\right|^{2}$$
כאשר \(\omega_{fi}=\frac{1}{\hbar}\left(\left\langle \psi_{f}|\hat{H}_{0}|\psi_{f} \right\rangle-\left\langle \psi_{i}|\hat{H}_{0}|\psi_{i} \right\rangle\right)\) וכן:
$$c_{f}^{(0)}=\langle\psi_{f}|\psi_{i}\rangle=\delta_{f,i}\qquad c_{f}^{(1)}(t)=-\frac{i}{\hbar}\int_{0}^{t}\langle\psi_{f}|\hat{V}(t^{\prime})|\psi_{i}\rangle e^{i\sigma_{f i}t^{\prime}}\mathrm{d}t^{\prime}\qquad  \ldots.$$

\end{proposition}
\begin{proof}
ניתן לקבל את הסתברות המעבר בעזרת סדרת דייסון:
\begin{gather*}P_{i f}(t)=\left|\left\langle \psi_{f}|\hat{U}_{I}(t,t_{i})|\psi_{i} \right\rangle\right|^{2}=\left|\left\langle \psi_{f}|\psi_{i} \right\rangle-\frac{i}{\hbar}\int_{0}^{t}e^{i\omega_{f i}t^{\prime}}\left\langle \psi_{f}|\hat{V}\left( t^{\prime} \right)|\psi_{i} \right\rangle\,\mathrm{d}t^{\prime}\right.\\ \left.+\,\left(-\frac{i}{\hbar}\right)^{2}\sum_{n}\int_{0}^{t}e^{i\omega_{f n}t_{1}}\,\left\langle \psi_{f}|\hat{V}(t_{1})|\psi_{n} \right\rangle\,\mathrm{d}t_{1}\!\!\int_{0}^{t_{1}}\!e^{i\omega_{n}t_{2}}\,\langle \psi_{n}|\hat{V}(t_{2})|\psi_{i} \rangle\,\mathrm{d}t_{2}+\cdots\right|^{2} 
\end{gather*}
כאשר משתמשים בכך שמתקיים:
$$\langle\psi_{f}|\hat{V}_{I}(t^{\prime})|\psi_{i}\rangle=\langle\psi_{f}| e^{i H_{0}t^{\prime}/\hbar}\hat{V}(t^{\prime})e^{-i H_{0}t^{\prime}/\hbar}|\psi_{i}\rangle=\langle\psi_{f}| V(t^{\prime})|\psi_{i}\rangle\exp\left(i\omega_{f i}t^{\prime}\right)$$
כאשר תדירות המעבר בין רמה \(i\) ל-\(f\) תקיים:
$$\omega_{f i}=\frac{E_{f}-E_{i}}{h}=\frac{1}{h}\left(\langle\psi_{f}|\hat{H}_{0}|\psi_{f}\rangle-\langle\psi_{i}|\hat{H}_{0}|\psi_{i}\rangle\right)$$
ניתן לכתוב כעת בעזרת המקדמים \(c_{n}(t)\) ולקבל את הטענה.

\end{proof}
\begin{corollary}
הסתברות המעבר עבור הפרעה מסדר ראשון תהיה:
$$P_{i f}(t)=\left|-\frac{i}{\hbar}\int_{0}^{t}\langle\psi_{f}|\hat{V}(t^{\prime})|\psi_{i}\rangle e^{i\omega_{f i}t^{\prime}}\mathrm{d}t^{\prime}\right|^{2}$$

\end{corollary}
\begin{definition}[קצב המעבר]
ההסתברות מעבר ליחידת זמן. \(\mathrm{d}P_{if}=\Gamma_{if}\mathrm{d}t\).

\end{definition}
\begin{example}[הסתברות מעבר עבור הפרעה קבועה]
כאשר \(V\) אינו תלוי בזמן נקבל:
$$P_{i f}(t)=\frac{1}{\hbar^{2}}\left| \langle\psi_{f} | \hat{V} | \psi_{i}\rangle\int_{0}^{t}e^{i\omega_{f i}t^{\prime}}\mathrm{d}t^{\prime}\right|^{2}\!\!=\!\frac{1}{\hbar^{2}}\left| \langle\psi_{f} | \hat{V} | \psi_{i}\rangle\right|^{2}\left| \frac{e^{i\omega_{f i}t}-1}{\omega_{f i}}\right|^{2}$$
כאשר כיוון ש-\(|e^{i\theta}-1|^{2}=4\sin^{2}(\theta/2),\) נקבל:
$$P_{i f}(t)=\frac{4\left| \langle\psi_{f} | \hat{V} | \psi_{i}\rangle\right|^{2}}{\hbar^{2}\omega_{f i}^{2}}\sin^{2}\left(\frac{\omega_{f i}t}{2}\right)$$
כאשר \(t\to \infty\) ניתן להשתמש ביחס:
$$\operatorname*{lim}_{t\rightarrow\infty}\frac{\sin^{2}(y t)}{\pi y^{2}t}=\delta(y)$$
ולקבל:
$$P_{i f}(t)=\frac{2\pi t}{\hbar}\left| \langle\psi_{f} | \hat{V} | \psi_{i}\rangle\right|^{2}\delta(E_{f}-E_{i})$$
כך שקצב המעבר יהיה:
$$\Gamma_{i f}=\frac{P_{i f}(t)}{t}=\frac{2\pi}{\hbar}\left|\left<\psi_{f}|\hat{V}|\psi_{i}\right>\right|^{2}\delta(E_{f}-E_{i})$$

\end{example}
\begin{proposition}[כלל הזהב של פרמי]
אם \(\rho(E_{f})\) זה צפיפות המצבים של המצב הסופי, אז קצב המעבר הכולל נתון על ידי:
$$\Gamma_{i f}=\frac{2\pi}{\hbar}\left|\langle \psi_{f}|\hat{V}|\psi_{i} \rangle\right|^{2}\rho(E_{i})=\frac{2\pi}{\hbar}\lvert V_{if} \rvert ^{2}\rho(E_{f})$$

\end{proposition}
\begin{proof}
$$\Gamma_{i f}=\int{\frac{P_{i f}(t)}{t}}\rho(E_{f})\,\mathrm{d}E_{f}={\frac{2\pi}{\hbar}}|\ \langle\psi_{f}\ |\ \hat{V}\ |\ \psi_{i}\rangle|^{2}\int\rho(E_{f})\delta(E_{f}-E_{i})\,\mathrm{d}E_{f}$$
כלומר:
$$\Gamma_{i f}=\frac{2\pi}{\hbar}\left|\langle\psi_{f}|\hat{V}|\psi_{i}\rangle\right|^{2}\rho(E_{i})$$

\end{proof}
\begin{example}[מציאת ביטוי דיפרנציאלי לחתך פעולה]
נמצא ביטוי דיפרנציאלי לחתך פעולה של חלקיק עם מסה \(m\) אשר עובר פיזור מפוטנציאל \(V\left( \mathbf{r} \right)\). החלקיק מגיע עם מספר גל \(\mathbf{k'}\) ולאחר הפיזור המספר הגל משתנה ל-\(\mathbf{k}\). נבצע טריק שבו נכתוב את החלקיק כאילו כלוא בקופסא באורך \(l\), ואז ניקח את הגבול \(l\to \infty\)(בפועל נקבל כי כלל לא תלוי בגודל הקופסא). זאת כדי שיהיה אפשר לנרמל את הפונקציות גל. לכן ניתן לכתוב:
$$\psi_{i}=\frac{1}{\sqrt{l^{3}}}\,e^{i\,{\bf k^{\prime}\cdot r}}\;\mathrm{and}\;\;\psi_{f}=\frac{1}{\sqrt{l^{3}}}\,e^{i\,{\bf k\cdot r}}$$
כאשר מתנאי השפה המחזוריים נקבל שלמים \(n_{x},n_{y},n_{z}\) כך שמתקיים:
$${\bf k}=\frac{2\,\pi}{l}\left(n_{x}\,\hat{t}+n_{y}\,\hat{\jmath}+n_{z}\,\hat{k}\right)$$
ההפרעה שלנו זה פוטנציאל הפיזור \(H'=V\left( \mathbf{r} \right)\) כאשר אלמנטי המטריצה הרלוונטים יהיו:
$$V_{f i}=\int\psi_{f}^{*}(r)\ V({\bf r})\ \psi_{i}({\bf r})\ \mathrm{d}^{3}{\bf r}=\frac{1}{l^{3}}\ \int e^{i\,({\bf k^{\prime}-k})\cdot{\bf r}}\ V({\bf r})\ \mathrm{d}^{3}{\bf r}\,.$$
מספר המצבים בין אנרגיה \(E\) ל-\(E+dE\) תגדל באופן ריבועי במרחב ה-\(k\) ביחס לזווית מרחבית. כלומר יהיה שווה \(k^{2}\mathrm{d}k\mathrm{\mathrm{d}\Omega}\) כאשר מספר המצבים יהיה:
$$\rho(E)\;\mathrm{d} E={\frac{k^{2}\mathrm{d} k\mathrm{d}\Omega}{\left(2\pi l\right)^{3}}}=\left({\frac{l}{2\,\pi}}\right)^{3}\,k^{2}\,{\frac{\mathrm{d} k}{\mathrm{d} E}}\mathrm{d} E\,\mathrm{d}\Omega$$
כאשר כיוון ש-\(E=\frac{\hbar^{2}k^{2}}{2m}\) נקבל:
$$\rho(E)=\left(\frac{l}{2\,\pi}\right)^{3}\,\frac{\sqrt{2\,m^{3}\,E}}{\hbar^{3}}\;\mathrm{d}\Omega$$
ולכן מכלל הזהב של פרמי קצב הפיזור לזווית מרחבית נתון על ידי:
$$R_{i\to \mathrm{d}\Omega}=\frac{2\,\pi}{\hbar}\,\frac{1}{l^{6}}\,\left|\int e^{i\,({\bf k^{\prime}-k})\cdot{\bf r}}\,V({\bf r})\ \mathrm{d}^{3}{\bf r}\right|^{2}\left(\frac{l}{2\,\pi}\right)^{3}\,\frac{\sqrt{2\,m^{3}\,E_{f}}}{\hbar^{3}}\,\mathrm{d}\Omega$$
כאשר החתך פעולה מקיים:
$$\frac{d\sigma}{\mathrm{d}\Omega}=\frac{R_{i\to \mathrm{d}\Omega}}{J_{i}\,\mathrm{d}\Omega}$$
וכן צפיפות ההסתברות של גל מהצורה \(\psi_{i}=Ae^{ i\mathbf{k'\cdot r'} }\) נתון על ידי:
$$J_{i}=|A|^{2}\ v=\frac{1}{l^{3}}\,\frac{\hbar k^{\prime}}{m}$$
ולכן נקבל:
$$\frac{d\sigma}{\mathrm{d}\Omega}=\left|-\frac{m}{2\,\pi\,\hbar^{2}}\int e^{i\,({\bf k}-{\bf k}^{\prime})\cdot{\bf r}}\,V({\bf r})\,\,\mathrm{d}^{3}{\bf r}\right|^{2}$$

\end{example}
\begin{remark}
זה ביטוי זהה למה שמתקבל מקירוב של תורת הפיזור הנקרא קירוב בורן.

\end{remark}
\begin{summary}
  \begin{itemize}
    \item עבור המילטוניאן מהצורה \(H=H_{0}+\lambda V(t)\) עם מצבים עצמיים ידועים \(\left\{  \ket{\psi_{n}}  \right\}\) של \(H_{0}\) כאשר \(\lambda\ll 1\) ניתן לפתח מצב כטור על ידי:
$$|\Psi(t)\rangle=\sum_{n}c_{n}(t)\,e^{-i E_{n}t/\hbar}|\psi_{n}\rangle$$
    \item ניתן לפתח את המקדמים לפי סדרים של \(\lambda\) כך שמתקיים:
$$c_{n}(t)=c_{n}^{(0)}(t)+\lambda\,c_{n}^{(1)}(t)+\lambda^{2}\,c_{n}^{(2)}(t)+\cdots$$
כאשר פונקציית הגל המתוקנת בתמונת שרדינגר תהיה נתונה על ידי:
$$|\Psi^{(i)}(t)\rangle=\sum_{n}c_{n}^{(i)}(t)e^{-i E_{n}t}|\psi_{n}\rangle$$
    \item התיקון ה-\(i\) למקדמים הראשונים יהיו:
\begin{gather*}c_{n}^{(0)}(t)=\delta_{n,i}\qquad  c_{n}^{(1)}(t)=-\frac{i}{\hbar}\int_{0}^{t}\mathrm{d} t_{1}\,V_{n i}(t_{1})e^{i\omega_{n i}t_{1}}\\c_{n}^{(2)}(t)=\left(-\frac{i}{\hbar}\right)^{2}\sum_{m}\int_{0}^{t}\mathrm{d} t_{1}\int_{0}^{t_{1}}\mathrm{d} t_{2}\,V_{n m}(t_{1})e^{i\omega_{m n}t_{1}}V_{m i}(t_{2})e^{i\omega_{m i}t_{2}} 
\end{gather*}
כאשר \(\omega_{ni}=(E_{n}-E_{i}) / \hbar\).
    \item ההסתברות לעבור ממצב \(\ket{\psi_{i}}\) ל-\(\ket{\psi_{f}}\) תהיה:
$$P_{i f}(t)=\left|\sum_{k=0}^{\infty}c_{f}^{(k)}(t)\right|^{2}$$
כאשר ניתן לקחת מסדר ראשון רק את \(c_{f}^{(0)}\) ו-\(c^{(1)}_{f}\).
    \item כלל הזהב של פרמי אומר כי עבור \(t\to \infty\) נקבל עבור מצב יחיד כי קצב המעבר(הסתברות ליחידת זמן) תהיה:
$$\Gamma_{i f}=\frac{2\pi}{\hbar}\left|\langle\psi_{f}|V|\psi_{i}\rangle\right|^{2}\delta(E_{f}-E_{i})$$
כך שעבור כמות גדולה של מצבים אם למצב הסופי יש צפיפות מצבים \(\rho(E_{f})\) נקבל:
$$\Gamma_{i f}=\frac{2\pi}{\hbar}\left|\langle\psi_{f}|V|\psi_{i}\rangle\right|^{2}\rho(E_{f})$$
  \end{itemize}
\end{summary}
\section{שימושים של תורת ההפרעות התלויה בזמן}

\begin{definition}[מדרגה קוונטית]
מערכת עם המילטוניאן \(H(t)=H_{0}+V(t)\) כאשר \(V(t)\) היא הפרעה קטנה הפועלת רק לזמן סופי. בזמן \(t \to -\infty\) אנחנו מתחלים במצב \(\ket{\Psi_{i}}\) של \(H_{0}\) עם אנרגיה \(E_{i}\) כאשר אנחנו רוצים למצוא את ההסתברות לעבור למצב \(\ket{\Psi_{f}}\) עם אנרגיה \(E_{f}\) לאחר שההפרעה הסתיימה(כלומר \(t \to \infty\)).

\end{definition}
\begin{proposition}
ההסתברות לעבור ממצב \(i\) ל-\(f\) במערכת של מדרגה קוונטית תהיה:
$$P(i\to f)=|c_{f}(\infty)|^{2}={\frac{1}{\hbar^{2}}}\left|\int_{-\infty}^{\infty}V_{f i}(t^{\prime})e^{i\omega_{f i}t^{\prime}}d t^{\prime}\right|^{2}$$
כאשר נשים לב כי הביטוי באינטגרל זה התמרת פורייה.

\end{proposition}
\begin{proof}
אנו יודעים כי הקידום בזמן של המערכת תהיה מהצורה:
$$|\Psi(t)\rangle=\sum_{n}c_{n}(t)e^{-i E_{n}t/\hbar}|\Psi_{n}\rangle,$$
כאשר \(c_{n}(t)\) הם אמפליטודות ההסתברות של המצבים \(\ket{\Psi_{n}}\). כדי למצוא את ההפרעה מסדר ראשון נחשב:
$$c_{f}^{(1)}(\infty)=-\frac{i}{\hbar}\int_{-\infty}^{\infty}V_{f i}(t^{\prime})e^{i\omega_{f i}t^{\prime}}d t^{\prime},$$
כאשר 
$$V_{f i}(t)=\langle\Psi_{f}|V(t)|\Psi_{i}\rangle$$
ו-\(\omega_{fi}=\frac{E_{f}-E_{i}}{\hbar}\). ההסתברות לעבור ממצב \(\ket{\Psi_{i}}\) ל-\(\ket{\Psi_{f}}\) הוא:
$$P(i\to f)=|c_{f}(\infty)|^{2}={\frac{1}{\hbar^{2}}}\left|\int_{-\infty}^{\infty}V_{f i}(t^{\prime})e^{i\omega_{f i}t^{\prime}}d t^{\prime}\right|^{2}$$
כאשר נשים לב כי זהו ההתמרת פורייה של \(V_{fi}\).  לכן יש מעבר כאשר \(V(t)\) מכיל תדירות המתאימה ל-\(\omega_{fi}\). כעת נרצה לחשב את ההסתברות להשאר ב-\(\ket{\Psi_{i}}\). מתקיים:
$$c_{i}(\infty)\approx1-\frac{i}{\hbar}\int_{-\infty}^{\infty}V_{i i}(t^{\prime})d t^{\prime}+\mathcal{O}(V^{2}).$$
כאשר זה גורם מרוכב לחלוטין כיוון ש-\(V_{ii}(t)\) הוא ממשי(ערכי האלכסון של הרמיטי). מתי שמחשבים את \(\left\lvert  c_{i}\left( \infty \right)  \right\rvert^{2}\)

\end{proof}
\begin{example}[מדרגה גאוסיאנית]
נניח \(V(t)=V_{0}e^{-t^{2}/\tau^{2}}\) גאוסיאן ברוחב \(\tau\). אזי אם נניח \(V_{fi}(t)=V_{0}e^{ -t^{2} / \tau^{2} }\) ונקבל כי ההתמרת פורייה תהיה:
$$\int_{-\infty}^{\infty}V_{f i}(t)e^{i\omega_{f i}t}d t=V_{0}\sqrt{\pi}\tau e^{-\omega_{f i}^{2}\tau^{2}/4}.$$
ולכן הסתברות המעבר תהיה:
$$P(i\to f)=\frac{\pi\tau^{2}|V_{0}|^{2}}{\hbar^{2}}e^{-\omega_{f i}^{2}\tau^{2}/2}.$$
כאשר נשים לב כי ההסתברות מקסימלית כאשר \(\omega_{fi}\tau\ll 1\) ומינימלית כאשר \(\omega_{fi}\tau \gg 1\).

\end{example}
\begin{definition}[דעיכה קוונטית - Quantum Quench]
מערכת עם המילטוניאן \(H(t)=H_{0}+V(t)\) כאשר \(H_{0}\) הוא בלתי תלוי בזמן ו-\(V(t)\) היא הפרעה קטנה שמתחילה ב-0 עבור \(t\to -\infty\) ומתכנסת אסימפטוטית ל-\(V_{0}\) ב-\(t\to \infty\). המצב ההתחלתי של המערכת הוא המצב עצמי \(\ket{\Psi_{i}}\) של \(H_{0}\) כאשר המטרה היא למצוא את המצב הסופי \(\ket{\Psi(t)}\) ואת הסתברויות המעבר בזמן \(t\to \infty\).

\end{definition}
\begin{remark}
לא ניתן לפתור בשיטה הקודמת כיוון שבאינסוף \(V(t)\to V_{0} \neq 0\) ולכן לא קיים התמרת פורייה.

\end{remark}
\begin{proposition}
באינסוף מתקיים:
$$c_{f}^{(1)}(\infty)=\frac{i}{\hbar\omega_{f i}}\int_{-\infty}^{\infty}\frac{\partial V_{f i}}{\partial t}e^{i\omega_{f i}t}d t.$$

\end{proposition}
\begin{proof}
נבצע אינטגרציה בחלקים:
$$c_{f}^{(1)}(\infty)=-\frac{i}{\hbar}\int_{-\infty}^{\infty}V_{f i}(t)e^{i\omega_{f i}t}d t=-\frac{V_{f i}(t)e^{i\omega_{f i}t}}{\hbar\omega_{f i}}\Big|_{-\infty}^{\infty}+\frac{i}{\hbar\omega_{f i}}\int_{-\infty}^{\infty}\frac{\partial V_{f i}}{\partial t}e^{i\omega_{f i}t}d t.$$
כאשר כל עוד \(V(t)\) הוא חלק נקבל כי הגורם הראשון מתאפס ולגורם השני קיים התמרת פורייה.

\end{proof}
\begin{corollary}
הסתברות המעבר תהיה:
$$P(i\to f)=\frac{1}{\hbar^{2}\omega_{f i}^{2}}\left|\int_{-\infty}^{\infty}\frac{\partial V_{f i}}{\partial t}e^{i\omega_{f i}t}d t\right|^{2}$$

\end{corollary}
\begin{remark}
הפרשנות הפיזיקלית זה שההסתברות במקרה זה תלויה בקצב השינוי - אם אנחנו משנים את האנרגיה בצורה קווזיסטטית, אז נשאר באותו מצב, לעומת זאת אם נשנה את המערכת ממש מהר, אז ההסתברות לעבור למצב אחר יעלה משמעותית.

\end{remark}
\begin{corollary}
המצב הקוונטי החדש יהיה בקירוב סדר ראשון:
$$|\Psi_{i}\rangle_{\mathrm{static}}\approx|\Psi_{i}\rangle+\sum_{f\neq i}{\frac{\langle\Psi_{f}|V_{0}|\Psi_{i}\rangle}{E_{i}-E_{f}}}|\Psi_{f}\rangle$$

\end{corollary}
\begin{example}
עבור \(V(t)=V_{0}\cdot \Theta(t)\) נקבל \(\frac{\partial }{\partial t}V_{fi}=V_{fi}\delta(t)\). אמפליטודת המעבר תהיה:
$$c_{f}^{(1)}(\infty)=\frac{i V_{f i}}{\hbar\omega_{f i}}$$
ולכן הסתברות המעבר תהיה:
$$P(i\to f)=\frac{|V_{f i}|^{2}}{\hbar^{2}\omega_{f i}^{2}}$$

\end{example}
\begin{definition}[הפרעה מונוכרומטית]
הפרעה אשר מבצעת אוסצילציות בתדר יחיד \(\omega\). כלומר עבור המילטוניאן \(H=H_{0}+V(t)\) הפרעה מהצורה:
$$V(t)=U e^{-i\omega t}+U^{\dagger}e^{i\omega t},$$
כאשר \(U\) לא תלוי בזמן.

\end{definition}
\begin{proposition}
אמפליטודת המעבר למצב \(\ket{\Psi_{f}}\) תהיה:
$$c_{f i}(t)=-\frac{i}{\hbar}\left[U_{f i}\frac{e^{i(\omega_{f i}-\omega)t}-1}{\omega_{f i}-\omega}+U_{f i}^{*}\frac{e^{i(\omega_{f i}+\omega)t}-1}{\omega_{f i}+\omega}\right]$$

\end{proposition}
\begin{proof}
אנו יודעים כי באופן כללי זה יהיה:
$$c_{f i}(t)=-\frac{i}{\hbar}\int_{0}^{t}V_{f i}(t^{\prime})e^{i\omega_{f i}t^{\prime}}d t^{\prime},$$
כאשר $$\omega_{f i}=(E_{f}-E_{i})/\hbar.$$ולכן אם נציב את הפוטנציאל נקבל את הטענה.

\end{proof}
\begin{proposition}[תנאי רסוננס]
עבור \(\omega_{fi}\approx \omega\) ועבור \(\omega_{fi}\approx-\omega\) נקבל מעבר.

\end{proposition}
\begin{proposition}[הסתברות מעבר]
עבור \(E_{f}\approx E_{i}+\hbar \omega\) נקבל:
$$P(i\to f)=|c_{f i}(t)|^{2}=\frac{4|U_{f i}|^{2}}{\hbar^{2}}\frac{\sin^{2}\left((\omega_{f i}-\omega)t/2\right)}{(\omega_{f i}-\omega)^{2}}$$

\end{proposition}
\begin{proposition}[כלל הזהב של פרמי להפרעות מונוכרומטיות]
בזמנים ארוכים אם נשתמש בזה ש-
$$\operatorname*{lim}_{t\rightarrow\infty}\,\frac{\sin^{2}(a t)}{\pi a^{2}t}=\delta(a)$$
נקבל כי קצב המעבר יהיה:
$$\Gamma(i\to f)=\frac{2\pi}{\hbar}|U_{f i}|^{2}\delta(E_{f}-E_{i}-\hbar\omega)$$

\end{proposition}
\begin{corollary}
אם נתונה צפיפות המצבים \(\rho(E_{f})\) ניתן לכתוב:
$$\Gamma=\frac{2\pi}{\hbar}\int|U_{f i}|^{2}\rho(E_{f})\delta(E_{f}-E_{i}-\hbar\omega)d E_{f}=\frac{2\pi}{\hbar}|U_{f i}|^{2}\rho(E_{i}+\hbar\omega)$$

\end{corollary}
\begin{remark}
זה דומה לכלל הזהב של פרמי. ההבדל הוא שבגלל ההפרעה המונוכרומטית קיבלנו כי בצפיפות מצבים יש הזזה.

\end{remark}
\begin{definition}[המילטוניאן אינטראקציה עם שדה קרינה]
אלקטרון(כלומר עם מטען \(q=-e\)) פועל בשדה אלקטרומגנטי עם ההמילטוניאן:
$$H={\frac{1}{2m}}\left(\mathbf{p}-q\mathbf{A}\right)^{2}-{\frac{q}{m}}\mathbf{s}\cdot\mathbf{B}+q\phi,$$
כאשר \(\mathbf{A}\) זה הפוטנציאל הווקטורי, \(\phi\) זה הפוטנציאל הסקלארי, \(B=\bar{\nabla} \times \mathbf{A}\) השדה המגנטי ו-\(\mathbf{s}\) זה הספין של האלקטרון. בנוסף נעבוד בכיול קולון - כולמר \(\bar{\nabla} \cdot \mathbf{A}=0\), ונניח שאנחנו באיזור הקרינה ולכן \(\phi=0\).

\end{definition}
\begin{proposition}[קירוב שדה חלש]
עבור שדה חלש מספיק נקבל \(\mathbf{A}^{2}\) זניח ונקבל:
$$H={\frac{{\bf p}^{2}}{2m}}\,-\,{\frac{q}{m}}{\bf A}\cdot{\bf p}\,-\,{\frac{q}{m}}{\bf s}\cdot{\bf B}.$$

\end{proposition}
\begin{proposition}
ניתן להניח כי הפוטנציאל הווקטורי הוא גל מישורי:
$$\mathbf{A}(\mathbf{r},t)=2A_{0}{\boldsymbol{\epsilon}}\cos(\mathbf{k}_{\gamma}\cdot\mathbf{r}-{\boldsymbol{\omega}}t)=A_{0}\epsilon\left(e^{i(\mathbf{k}_{\gamma}\cdot\mathbf{r}-\omega t)}+e^{-i(\mathbf{k}_{\gamma}\cdot\mathbf{r}-\omega t)}\right)$$
כאשר \(\epsilon\) וקטור הפולאריזציה ו-\(\omega=c|k_{\gamma}|\) התדירות הזוויתית.

\end{proposition}
\begin{corollary}
ניתן לכתוב את איבר האינטראקציה על ידי:
$$V(t)=-{\frac{q}{m}}\mathbf{A}\cdot\mathbf{p}=-{\frac{q A_{0}}{m}}\mathbf{\epsilon}\cdot\mathbf{p}\left(e^{i(\mathbf{k}_{\gamma}\cdot\mathbf{r}-\omega t)}+e^{-i(\mathbf{k}_{\gamma}\cdot\mathbf{r}-\omega t)}\right)$$

\end{corollary}
\begin{corollary}
אמפליטודת המעבר תהיה:
$$c_{fi}(t)=-\frac{i}{\hbar}\frac{qA_{0}}{m}\epsilon\cdot\left[\langle f|e^{i\mathbf{k}_{\gamma}\cdot\mathbf{r}}\mathbf{p}|i\rangle\frac{e^{i(\omega_{fi}-\omega)t}-1}{\omega_{fi}-\omega}+\langle f|e^{-i\mathbf{k}_{\gamma}\cdot\mathbf{r}}\mathbf{p}|i\rangle\frac{e^{i(\omega_{fi}+\omega)t}-1}{\omega_{fi}+\omega}\right]$$

\end{corollary}
\begin{proof}
נובע כרגיל מהצבה ישרה של הביטוי של ההפרעה בנוסחה
$$c_{f i}(t)=-\frac{i}{\hbar}\int_{0}^{t}\langle f|V(t^{\prime})|i\rangle e^{i\omega_{f i}t^{\prime}}d t^{\prime}$$
כאשר \(\omega_{f i}=(E_{f}-E_{i})/\hbar\).

\end{proof}
\begin{lemma}
עבור קירוב דיפולי (אורך הגל \(\lambda\) גודל ממש מגודל האטום) ניתן לקרב \(e^{ i\mathbf{k}_{\gamma}\cdot \mathbf{r} }\approx 1\) ולקבל:
$$\langle f|e^{i\mathbf{k}_{\gamma}\cdot\mathbf{r}}\mathbf{p}|i\rangle\approx\epsilon\cdot\langle f|\mathbf{p}|i\rangle.$$
כאשר על ידי שימוש בקומוטטור \(\left[ \mathbf{r},H_{0} \right]=i\hbar \mathbf{p} / m\) נקבל:
$$\langle f|\mathbf{p}|i\rangle=i m\omega_{f i}\langle f|\mathbf{r}|i\rangle.$$

\end{lemma}
\begin{corollary}
תחת הקירוב הדיפולי נקבל:
$$c_{f i}(t)=-\frac{q A_{0}}{\hbar}\epsilon\cdot\langle f|\mathbf{r}|i\rangle\left[\frac{e^{i(\omega_{f i}-\omega)t}-1}{\omega_{f i}-\omega}+\frac{e^{i(\omega_{f i}+\omega)t}-1}{\omega_{f i}+\omega}\right].$$

\end{corollary}
\begin{corollary}
הסתברות המעבר תהיה:
$$P(i\to f)=|c_{f i}(t)|^{2}=\frac{4q^{2}A_{0}^{2}}{\hbar^{2}}|\epsilon\cdot\langle f|\mathbf{r}|i\rangle|^{2}\frac{\sin^{2}\left[(\omega_{f i}-\omega)t/2\right]}{(\omega_{f i}-\omega)^{2}}$$
כאשר בזמנים ארוכים ניתן להשתמש ב-\(\lim_{ t \to \infty }\sin \frac{^2(at)}{\pi a^{2}t}=\delta(a)\) ולקבל:
$$\Gamma(i\to f)=\frac{2\pi}{\hbar}\left(\frac{q A_{0}}{m}\right)^{2}|\epsilon\cdot\langle f|\mathbf{r}|i\rangle|^{2}\delta(E_{f}-E_{i}-\hbar\omega)$$

\end{corollary}
\begin{summary}
  \begin{itemize}
    \item עבור מערכת של הפרעה סופית \(V(t)\) הסתברות המעבר ממצב \(\ket{\Psi_{i}}\) למצב \(\ket{\Psi_{f}}\) נתונה על ידי:
$$P(i\to f)={\frac{1}{\hbar^{2}}}\left|\int_{-\infty}^{\infty}V_{f i}(t)e^{i\omega_{f i}t}d t\right|^{2}$$
למשל עבור גאוסיאן \(V(t)=V_{0}e^{ -t^{2}/\tau^{2} }\) נקבל מקסימום הסתברות כאשר \(\omega_{fi}\tau\ll 1\) - רסוננס.
    \item עבור הפרעה \(V(t)\) אשר עוברת מ-0 ל-\(V_{0}\) נקבל כי הסתברות המעבר תלויה בנגזרת הזמנית של ההפרעה:
$$P(i\to f)=\frac{1}{\hbar^{2}\omega_{f i}^{2}}\left|\int_{-\infty}^{\infty}\frac{\partial V_{f i}}{\partial t}e^{i\omega_{f i}t}d t\right|^{2}$$
כאשר דוגמא לכך היא פונקציית מדרגה \(V(t)=V_{0}\Theta(t)\) כאשר ההסתברות תהיה \(\frac{|V_{fi}|^{2}}{(\hbar \omega_{fi})^{2}}\).
    \item עבור הפרעה מונוכרומטית מהצורה \(V(t)=U e^{-i\omega t}+U^{\dagger}e^{i\omega t}\) נקבל מעברים ברסיננס כאשר \(\omega_{fi}\approx\pm \omega\) כאשר במקרה זה:
$$P(i\to f)=\frac{4|U_{f i}|^{2}}{\hbar^{2}}\frac{\sin^{2}[(\omega_{f i}-\omega)t/2]}{(\omega_{f i}-\omega)^{2}}$$
    \item עבור שדה חשמלי עם המילטוניאן מהצורה \(H\approx \frac{p^{2}}{2m}-\frac{q}{m}\mathbf{A\cdot p}\) תחת קירוב דיפול \(e^{ i\mathbf{k\cdot r} }=1\) נקבל:
$$\Gamma(i\to f)=\frac{2\pi}{\hbar}\left(\frac{q A_{0}}{m}\right)^{2}|\epsilon\cdot\langle f|\mathbf{r}|i\rangle|^{2}\delta(E_{f}-E_{i}-\hbar\omega)$$
  \end{itemize}
\end{summary}
\chapter{חלקיקים זההים}

\section{זוגות של חלקיקים זהים}

\begin{definition}[חלקיקים זההים]
שתי חלקיקים נקראים זהים אם כל התכונות הפנימיות שלהם(מסה, מטען, ספין וכו') הם זהים לגמרי, כך שלא ניתן לבנות ניסוי אשר יכול להבחין ביניהם

\end{definition}
\begin{example}
כל אלקטרון ביקום זהה לכל אלקטרון אחר. זה גם נכון לגבי כל חלקיק אלמנטרי ידוע, כמו קוורק, פוטון וכו'.

\end{example}
\begin{remark}
העובדה החלקיקים הם זהים לא אומר שנמצאים באותו מצב קוונטי. ייתכן הבדלים עדיין במיקום, בתנע וכו'.

\end{remark}
\begin{proposition}
עבור שתי חלקיקים זהים במערכת, ניתן להחליף ביניהם, והפיזיקה לא תשתנה. כלומר אם נחליף במשוואה את הפונקציות גל של שתי חלקיקים, הפתרון לא ישתנה.

\end{proposition}
\begin{symbolize}
נכתוב \(\ket{\phi_{a}(i)}\) כדי לסמן שחלקיק \(i\) נמצא במצב \(\phi_{a}\). ניתן באופן שקול לכתוב \(\ket{\phi_{a}}_{i}\).

\end{symbolize}
\section{זוגות של חלקיקים}

\begin{definition}[אופרטור ההחלפה]
אופרטור \(P_{12}\) המקיים:
$$P_{12}\Bigl(|\phi_{a}\rangle_{1}\otimes|\phi_{b}\rangle_{2}\Bigr)=|\phi_{b}\rangle_{1}\otimes|\phi_{a}\rangle_{2}$$

\end{definition}
\begin{corollary}
אופרטור ההחלפה הוא הרמיטי ומקיים \(P^{2}_{12}=\mathbb{1}\). בפרט הערכים העצמיים שלו הם \(\pm 1\).

\end{corollary}
\begin{proposition}
הפיזיקה אינוורינטית תחת אופרטור החילוף על חלקיקים זהים, כלומר:
$$[H,P_{12}]=0$$
ולכן בפרט לכסינה במשותף עם ההמילטונאן. ולכן כל מצב קוונטי יהיה עם ערך עצמי של \(1\) או \(-1\) תחת אופרטור ההחלפה. ונשאר במצב זה לאורך זמן.

\end{proposition}
\begin{definition}[בוזונים ופרמיונים]
  \begin{enumerate}
    \item בוזון - חלקיק אשר המצב הקוונטי שלו עם ערך עצמי \(1\) תחת אופרטור ההחלפה. כלומר מקיים \(P_{12}\ket{\Psi}=+\ket{\Psi}\). 


    \item פרמיון - חלקיק אשר המצב הקוונטי שלו עם ערך עצמי \(-1\) תחת אופרטור ההחלפה. כלומר קיים \(P_{12}\ket{\Psi}=-\ket{\Psi}\). 


  \end{enumerate}
\end{definition}
\begin{proposition}
ניתן להגדיר מחלקת שקילות תחת החלפת חלקיקים:
$$|\phi_{a}\rangle_{1}\otimes|\phi_{b}\rangle_{2}\sim|\phi_{b}\rangle_{1}\otimes|\phi_{a}\rangle_{2}.$$
כאשר ההנחה של חלקיקים זהים שכל איבר במחלקת שקילות מתאר את אותו מצב פיזיקלי. 

\end{proposition}
\begin{remark}
למעשה זה אומר שאני יכול להחליף את הסדר של המכפלה הטנזורית, ואומנם אני אקבל אובייקט מתמטי שונה, אך זה מייצג את אותו מצב פיזיקלי. 

\end{remark}
\begin{definition}[מסמטר ואנטי ומסמטר עבור שתי חלקיקים]
עבור מערכת של שתי חלקיקים המסמטר \(S\) והאנטי מסמטר \(A\) מוגדרים על ידי:
$$S=\frac{1}{2}\Big(\mathbb{1} +P_{12}\Big)\qquad A=\frac{1}{2}\Big(\mathbb{1} -P_{12}\Big)$$

\end{definition}
\begin{remark}
אם נזכור כי המצבים הסימטרים והאנטי סימטרים יוצרים תת מרחב ווקטורי אז \(A\) ו-\(S\) הם למעשה אופרטורי ההטלה על התת מרחב הזה. 

\end{remark}
\begin{proposition}
עבור שתי בוזונים \(\ket{\phi_{a}}_{1}\) ו-\(\ket{\phi_{b}}_{2}\) המצב הקוונטי במצב מכפלה יהיה:
$$|\Psi_{S}\rangle=S{\Bigl(}|\phi_{a}\rangle_{1}\otimes|\phi_{b}\rangle_{2}{\Bigr)}={\frac{1}{\sqrt{2}}}{\Bigl(}|\phi_{a}\rangle_{1}\otimes|\phi_{b}\rangle_{2}+|\phi_{b}\rangle_{1}\otimes|\phi_{a}\rangle_{2}{\Bigr)}$$
כאשר עבור פרמיונים המצב הקוונטי במצב מכפלה יהיה:
$$|\Psi_{A}\rangle=\frac{1}{\sqrt{2}}\Bigl(|\phi_{a}\rangle_{1}\otimes|\phi_{b}\rangle_{2}-|\phi_{b}\rangle_{1}\otimes|\phi_{a}\rangle_{2}\Bigr)$$

\end{proposition}
\begin{remark}
למעשה המסמטר והאנטי מסמטר בוחר את הנציג של המחלקת שקילות אשר משמר את \(P_{12}\). 

\end{remark}
\begin{proposition}[עקרון האיסור של פאולי]
לא ייתכן ויהיו שתי פרמיונים באותו מצב קוונטי

\end{proposition}
\begin{proof}
נניח בשלילה ש-\(\psi=\ket{\varphi}\ket{\varphi}\) הוא מערכת פרמיונית של שתי מצבים באותו מצב קוונטי. תחת אופרטור החילוף נקבל:
$$\hat{P}_{12}|\varphi\rangle|\varphi\rangle=|\varphi\rangle|\varphi\rangle=-|\varphi\rangle|\varphi\rangle,$$
כאשר השתמשנו באנטי סימטריה של פרמיונים. לכן מתקיים:
$$|\varphi\rangle|\varphi\rangle=0$$
כלומר לא ייתכן כי יש שתי מצבים קוונטים זהים. 

\end{proof}
\begin{proposition}
עבור זוג של חלקיקים זהים היפוך מרחבי סביב מרכז המסה שקול לאופרטור ההחלפה.

\end{proposition}
\section{אוסף כללי של חלקיקים זהים}

נכליל את מה שעשינו עבור זוגות עבור כל אוסף של \(N\) חלקיקים בעזרת תורת החבורות.

\begin{reminder}
אוסף של \(N\) חלקיקים יהיה איבר במרחב:
$${\mathcal{H}}_{\mathrm{total}}={\mathcal{H}}_{1}\otimes{\mathcal{H}}_{2}\otimes\cdots\otimes{\mathcal{H}}_{N},$$
מהצורה:
$$|\varphi_{i_{1}}\varphi_{i_{2}}\cdots\varphi_{i_{N}}\rangle\equiv|\varphi_{i_{1}}\rangle_{1}\otimes|\varphi_{i_{2}}\rangle_{2}\otimes\cdots\otimes|\varphi_{i_{N}}\rangle_{N}$$

\end{reminder}
\begin{definition}[פעולה של איבר בחבורת הסימטריות]
נגדיר את הפעולה של החבורת התמורות \(S_{N}\) על \(H_{\text{total}}\) על ידי שינוי השמות של החלקיקים. כלומר לכל פרמוטציה \(g \in S_{N}\) קיים אופרטור ואונטרי \(P_{g}\) כך שמקיים:
$$P_{g}\left|\varphi_{i_{1}}\varphi_{i_{2}}\cdots\varphi_{i_{N}}\right\rangle=\left|\varphi_{i_{g(1)}}\varphi_{i_{g(2)}}\cdots\varphi_{i_{g(N)}}\right\rangle$$

\end{definition}
\begin{remark}
האופרטור \(P_{g}\) זה פשוט המטריצה שמתאימה לתמורה \(g\). הסיבה היחידה שמשתמשים בזה במקום ב-\(g\) זה שהקבוצות לא מתאימות, הרי \(P_{g}\) פועלת על המרחב הילברט. במתמטיקה הרבה פעמים משתמשים בסימון \(g.\ket{\psi}\) כאשר ה-. מסמנת את זה שזה פועל בהתאם לחוקיות הרצויה, ושזה לא כפל רגיל.

\end{remark}
\begin{reminder}
כאשר חבורה \(G\) פועלת על קבוצה \(X\) ניתן לחלק את הקבוצה \(X\) לאוסף של מחלקות שקילות כך ששתי איברים \(x_{1},x_{2} \in X\) נמצאים באותה מחלקת שקילות רק אם קיים \(g \in G\) כך ש-\(x_{1}=g.x_{2}\). מחלקות שקילות אלו נקראות מסלולים. כמו כן אם \(x\) נציג של מחלקת שקילות נסמן את המחלקת שקילות ב-\(\mathcal{O}(x_{i})\) כך שעבור נציגים של כל המסלולים \(\{ x_{i} \}\) מתקיים:
$$X=\bigsqcup_{i} \mathcal{O} (x_{i})$$

\end{reminder}
\begin{corollary}
שתי איברים \(\ket{\psi},\ket{\psi '}\in \mathcal{H}\) נמצאים באותו מסלול אם קיים \(g \in S_{N}\) כך ש-\(\ket{\psi'}=P_{g}\ket{\psi}\). כלומר:
$${\mathcal{O}}(|\psi\rangle)=\{P_{g}|\psi\rangle:g\in S_{N}\}$$
כאשר כל שתי איברים באותו מסלול עבור חלקיקים זהים מייצגים את אותו מצב פיזיקלי.

\end{corollary}
\begin{reminder}
ניתן לפרק כל תמורה לאוסף של חילופים. הסימן של התמורה(\(\mathrm{sgn}\)) תהיה שווה \(1\) אם יש כמות זוגית של חילופים ול-\(-1\) אם יש כמות אי זוגית של חילופים.

\end{reminder}
\begin{proposition}
כאשר יש לי אוסף של \(N\) בוזונים התת מרחב הפיזיקלי יהיה:
$${\mathcal{H}}_{\text{bosons}}=\left\{\,v\in{\mathcal{H}}_{\mathrm{total}}:P_{g}\,v=v\,\,\mathrm{for~all}\,\,g\in S_{N}\,\right\}$$
כיוון זה אוסף של כל המצבים שאינם משתנים תחת כל החילופים.

\end{proposition}
\begin{proposition}
כאשר יש לי אוסף של \(N\) פרמיונים התת המרחב בפיזיקלי יהיה:
$${\mathcal{H}}_{\mathrm{fermions}}=\{\,v\in{\mathcal{H}}_{\mathrm{total}}:P_{g}\,v=\mathrm{sgn}(g)\,v\,\mathrm{for~all}\,g\in S_{N}\,\}.$$
כלומר זה אוסף כל המצבים שמחליפים סימן תחת כל חילוף.

\end{proposition}
כעת כמו מקודם נגדיר את המסמטר והאנטי מסמטר על ידי אופרטורי ההטלה על המרחבים הפיזיקליים אשר אינם משנים את המחלקת שקילות המתאימה. כלומר זה יחזיר את המצב אשר מייצג את אותו מצב הפיזיקלי(כלומר נמצא באותו מסלול) וגם נמצא בתת מרחב הפיזיקלי אשר אינווריאנטי לחילוף החלקיקים.

\begin{definition}[מסמטר ואנטי מסמטר]
עבור מערכת של \(N\) חלקיקים המסמטר \(S\) והאנטי מסמטר \(A\) מוגדרים על ידי:
$$S={\frac{1}{N!}}\sum_{g\in S_{N}}P_{g}\qquad A=\frac{1}{N!}\sum_{g\in S_{N}}\mathrm{sgn}(g)\,P_{g}$$
כלומר זה יהיה האופרטור שנדרש להפעיל על האופורטור במצב מכפלה כדי שיש במסמטר. 

\end{definition}
\begin{remark}
עבור \(N>2\) נקבל \(A+S\neq \mathbb{1}\) ולכן לא כל מצב הוא או סימטרי או אנטי סימטרי.

\end{remark}
\begin{remark}
משהו נחמד לשים לב עליו זה שאם איבר הוא על האלכסון של מטריצת תמורה אז הוא נקודת שבת, ולכן עבור מטריצת תמורה העקבה של אופרטור זה כמות נקודות השבת. לכן אם ניקח את העקבה של שתי האגפים של המסמטר ניתן לכתוב:
$$\mathrm{Tr}(S)=\frac{1}{N!}\sum_{g\in S_{N}}\mathrm{Tr}(P_{g})=\frac{1}{N!}\sum_{g \in S_{N}}|\mathrm{fix(g)|}=\frac{1}{|S_{N}|}\sum_{g \in S_{N}}|\mathrm{fix(g)|}$$
כאשר קיבלנו בדיוק את הביטוי מהלמה של ברנסייד עבור כמות המסלולים! ולכן העקבה של המסמטר יהיה כמות המסלולים בפעולה. כלומר כמות המצבים הפיזיקלים של המערכת הבוזונית! ניתן להראות באופן דומה עבור האנטי מסמטר:
$$\mathrm{Tr}(A)=\frac{1}{N!}\sum_{g\in S_{N}}\mathrm{sgn(g)}\cdot\mathrm{Tr}(P_{g})=\frac{1}{N!}\sum_{g \in S_{N}}\mathrm{sgn}(g)|\mathrm{fix(g)|}=\frac{1}{|S_{N}|}\sum_{g \in S_{N}}\mathrm{sgn(g)}|\mathrm{fix(g)|}$$

\end{remark}
\begin{proposition}
המסמטר הוא אופרטור הטלה אורתוגונאלי על המרחב הפיזיקלי של הבוזונים והאנטי מסמטר הם האופרטורי ההטלה על התת מרחב הפיזיקלי של הפרמיונים.

\end{proposition}
\begin{proof}
נראה עבור המסמטר. ראשית נראה כי אופרטור הטלה. כלומר ש-\(S^{2}=S\). מתקיים:
$$S^{2}=\left(\frac{1}{N!}\sum_{g\in S_{N}}P_{g}\right)\left(\frac{1}{N!}\sum_{h\in S_{N}}P_{h}\right)=\frac{1}{(N!)^{2}}\sum_{g,h\in S_{N}}P_{g}P_{h}.$$
כאשר כיוון ש-\(P_{g}P_{h}=P_{gh}\)(מהדרישות של פעולה על חבורה) כאשר ההעתקה \((g,h)\mapsto gh\) מ-\(S_{N}\times S_{N}\to S_{N}\) היא על כאשר כל גורם מופיע בדיוק \(N!\) פעמים נקבל:
$$S^{2}={\frac{1}{(N!)^{2}}}\cdot N!\sum_{k\in S_{N}}P_{k}={\frac{1}{N!}}\sum_{k\in S_{N}}P_{k}=S.$$
כעת כדי להראות כי הטלה אורתוגונאלי נראה כי הרמיטי:
$$S^{\dagger}=\frac{1}{N!}\sum_{g\in S_{N}}P_{g}^{\dagger}=\frac{1}{N!}\sum_{g\in S_{N}}P_{g^{-1}}.$$
כיוון שההעתקה \(g\mapsto g^{-1}\) היא העתקה הפיכה מ-\(S_N\to S_{N}\) נקבל:
$$S^{\dagger}=\frac{1}{N!}\sum_{g\in S_{N}}P_{g}=S$$
לבסוף כדי להראות שהמרחב שמוטל עליו הוא המרחב הפיזיקלי של הבוזונים(\(\left\{\,v\in{\mathcal{H}}_{\mathrm{total}}:P_{g}\,v=v\,\,\mathrm{for~all}\,\,g\in S_{N}\,\right\}\)) נשתמש בזה שמצב כללי מהמרחב הזה לא משתנה תחת המסמטר.

\end{proof}
\begin{example}[ספירת מצבים אנטי סימטרים]
נמצא כמה מצבים אנטי סימטרים יש במערכת של 3 חלקיקים עם ספין \(S=1\). נכתוב מצב כללי על ידי:
$$|1,m_{1}\rangle\otimes|1,m_{2}\rangle\otimes|1,m_{3}\rangle=|m_{1}m_{2}m_{3}\rangle$$
כאשר כדי למצוא את התת מרחב האנטי סימטרי נרצה להטיל את איברי הבסיס \(\left\{  \ket{m_{1},m_{2},m_{3}}\mid m_{1},m_{2},m_{3} \in\{ -1,0,1 \} \right\}\) על המרחב האנטי סימטרי על ידי האנטי מסמטר \(A=\frac{1}{N!}\sum_{g\in S_{N}}\mathrm{sgn}(g)\,P_{g}\). נשים לב כי אם יש לנו שתי ערכי \(m\) זהים נקבל כי נבדלים על ידי מחזור אחד ולכן כל מחזור יהיה זהה למינוס אותו המחזור(החילוף הלא משנה כלום הזה יוסיף סימן מינוס) ולכן נקבל 0. כלומר המצב היחיד שנדרש לבדוק זה המצב שבו מצב מכיל את כל ערכי \(m\) האפשריים. כיוון שיש מצב אחד כזה עד כדי תמורה נדרש להפעיל את האנטי מסמטר רק על מצב אחד כזה:
$$A\ket{-1,0,1} ={\frac{1}{\sqrt{6}}}{\Big[}\mid+1,0,-1\rangle-\mid0,+1,-1\rangle-\mid+1,-1,0\rangle+\mid0,-1,+1\rangle+\mid-1,+1,0\rangle-\mid-1,0,+1\rangle{\Big]}$$
ניתן לחלופין אם רק מעניין אותנו לספור את המצבים להשתמש בלמה של ברנסייד:
$$\mathrm{Tr}(A)=\frac{1}{N!}\sum_{g \in S_{N}}\mathrm{sgn}(g)|\mathrm{fix(g)|}=\text{Num of Orbits}$$
ניתן לסווג לשלושה קטגוריות:

  \begin{itemize}
    \item מחזור של היחידה - יש מצב אחד כזה. לא משנה כלום ולכן כל איבר בבסיס הוא שבת. ומוסיף לסכום \(27 \times 1\).
    \item מחזור של חילוף - יש 3 מצבים כאלה - מתוכם יהיה שבת רק אם שתי האיברים המוחלפים שווים ואז יש גם דרגת חופש של האיבר השלישי. לכן סה"כ יש 9 איברי שבת לכל חילוף כזה. אך כיוון שזה תמורה אם סימן שלילי מוסיף לסכום \(-9 \times 3 = -27\)
    \item מחזור של שלוש - יש 2 מצבים כאלה. מצב כזה יהיה שבת רק אם שלושת האיברים שווים. יש שלושה מצבים שבם כולם שווים ולכן יש שלושה איברי שבת. סה"כ מוסיף לסכום \(2\times 3 = 6\).
  \end{itemize}
נקבל כעת כי סה"כ כמות המצבים יהיו:
$$\frac{1}{6}\lvert 27-27+6 \rvert =1$$
ואכן קיבלנו כי יש רק מצב אחד כזה.

\end{example}
\begin{proposition}[דטרמיננטת סלאטור - Slator]
עבור אוסף של \(N\) פרמיונים חישוב האופרטור האנטי מסומטר \(A\) המוגדר על ידי:
$$A\left(\left|\psi_{1}\right\rangle\otimes\cdots\otimes\left|\psi_{N}\right\rangle\right)=\frac{1}{N!}\sum_{g\in S_{N}}\operatorname{sgn}(g)\,P_{g}\left(\left|\psi_{1}\right\rangle\otimes\cdots\otimes\left|\psi_{N}\right\rangle\right).$$
שקול לחישוב הדטרימינטטה הבאה:
$$\Psi(x_{1},\ldots,x_{N})=\frac{1}{\sqrt{N!}}\left|\begin{array}{cccc}{{\psi_{1}(x_{1})}}&{{\psi_{1}(x_{2})}}&{{\cdots}}&{{\psi_{1}(x_{N})}}\\ {{\psi_{2}(x_{1})}}&{{\psi_{2}(x_{2})}}&{{\cdots}}&{{\psi_{2}(x_{N})}}\\ {{\vdots}}&{{\vdots}}&{{\ddots}}&{{\vdots}}\\ {{\psi_{N}(x_{1})}}&{{\psi_{N}(x_{2})}}&{{\cdots}}&{{\psi_{N}(x_{N})}}\end{array}\right|$$

\end{proposition}
\begin{proof}
כלל ליבניץ לדטרמיננטה אומר כי:
$$\operatorname*{det}(A)=\sum_{\tau\in S_{n}}\operatorname{sgn}(\tau)\prod_{i=1}^{n}a_{i\tau(i)}=\sum_{\sigma\in S_{n}}\operatorname{sgn}(\sigma)\prod_{i=1}^{n}a_{\sigma(i)i}$$
ולכן נובע ישירות.

\end{proof}
\begin{summary}
  \begin{itemize}
    \item מערכת של \(N\) חלקיקים מתואר על ידי איבר במרחב \({\mathcal{H}}_{\mathrm{total}}={\mathcal{H}}_{1}\otimes{\mathcal{H}}_{2}\otimes\cdots\otimes{\mathcal{H}}_{N}\).
    \item אופרטור \(P_{g}\) מחליף את השמות של החלקיקים. כאשר עבור חלקיקים זהים כל החליופים האפשריים של מצב קוונטים מייצגים מצב פיזיקלי.
    \item בוזונים לכל תמורה \(g\) מקיימים \(P_{g}\ket{\psi}=\ket{\psi}\) כאשר פרמיונים לכל מצב מקיימים \(P_{g}\ket{\psi}=\mathrm{sgn}(g)\ket{\psi}\).
    \item המסמטר ואנטי מסמטר מטילים את התת מרחב הפיסיקלי:
$$S={\frac{1}{N!}}\sum_{g\in S_{N}}P_{g}\qquad A=\frac{1}{N!}\sum_{g\in S_{N}}\mathrm{sgn}(g)\,P_{g}$$
  \end{itemize}
\end{summary}
\section{צימוד ספין}

\begin{symbolize}
כאשר הפונקציית גל היא פונקציה מצומדת של הפונקציית גל והסביבה ניתן לכתוב:
$$\Psi_{t}(x_{i},s_{i})=\chi(s_{i})\,\psi(x_{i})$$
כאשר \(\psi(x_{i})\) היא הפונקציית גל המרחבית ו-\(\chi(s_{i})\) הוא הפונקציית גל של הספין

\end{symbolize}
\begin{symbolize}
עבור בוזונים נסמן \(\chi_{+},\psi_{+}\) כאשר פונקציית הגל סימטרית ו-\(\chi_{-},\psi_{-}\) כאשר פונקציית הגל אנטי סימטרית.

\end{symbolize}
\begin{proposition}
  \begin{itemize}
    \item עבור בוזונים מתקיים \(\Psi_{t}=\chi_{+}\ \psi_{+}\) או \(\Psi_{t}=\chi_{-}\ \psi_{-}\).
    \item עבור פרמיונים מתקיים \(\Psi_{t}=\chi_{-}\ \psi_{+}\) או \(\Psi_{t}=\chi_{+}\ \psi_{-}\)
  \end{itemize}
\end{proposition}
\begin{proof}
נובע מיידית מהדרישה שפונקציית הגל הכוללת תהיה סימטרית עבור בוזונים ואנטי סימטרית עבור פרמיונים.

\end{proof}
\begin{proposition}[אינווריאנטיות של \(P_{12}\) עם \(J_{\pm}\)]
אופרטור החילוף חלקיקים מקיים \(\left[ P_{12},J_{\pm} \right]=0\) כאשר \(J_{\pm}\) זה אופרטורי העלה והורדה של התנע הזוויתי.

\end{proposition}
\begin{proof}
נסתכל על שתי חלקיקים זהים במצב \(|j\,j;m_{1}\,m_{2}\rangle\) כאשר אופרטורי העלה והורדה נתונים על ידי \(J_{\pm}=J_{1\pm}+J_{2\pm}\). הפעולה של זה על מצב נתונה על ידי:
$$J_{\pm}\,|j\,j;m_{1}\,m_{2}\rangle=c_{j\,m_{1}}^{\pm}\,|j\,j;m_{1}\pm1,\,m_{2}\rangle+c_{j\,m_{2}}^{\pm}\,|j\,j;m_{1},\,m_{2}\pm1\rangle,$$
כאשר \(c_{j\,m}^{+}=\sqrt{(j-m)(j+m+1)})\) הם המקדמים של אופרטורי העלה והורדה. פעולת החילוף מקיימת \(P_{12}\,|j\,j;m_{1}\,m_{2}\rangle=|j\,j;m_{2}\,m_{1}\rangle\) ולכן:
$$\begin{array}{c}{{P_{12}\,J_{\pm}\,|j\,j;m_{1}\,m_{2}\rangle=P_{12}\Big[c_{j\,m_{1}}^{\pm}\,|j\,j;m_{1}\pm1,\,m_{2}\rangle+c_{j\,m_{2}}^{\pm}\,|j\,j;m_{1},\,m_{2}\pm1\rangle\Big]}}\\ {{=c_{j\,m_{1}}^{\pm}\,|j\,j;m_{2},\,m_{1}\pm1\rangle+c_{j\,m_{2}}^{\pm}\,|j\,j;m_{2}\pm1,\,m_{1}\rangle.}}\end{array}$$
לעומת זאת אם נפעיל קודם את אופרטור החילוף נקבל:
$$\begin{array}{c}{{J_{\pm}\,P_{12}\,|j\,j;m_{1}\,m_{2}\rangle=J_{\pm}\,|j\,j;m_{2}\,m_{1}\rangle}}\\ {{=c_{j\,m_{2}}^{\pm}\,|j\,j;m_{2}\pm1,\,m_{1}\rangle+c_{j\,m_{1}}^{\pm}\,|j\,j;m_{2},\,m_{1}\pm1\rangle.}}\end{array}$$
כאשר נשים לב כי שתי הביטויים זהים. ולכן \(P_{12}J_{\pm}=J_{\pm}P_{12}\).

\end{proof}
\begin{corollary}
כיוון שכל תמורה מורכבת מחילופים כל תמורה תתחלף עם אופרטורי העלה והורדה.

\end{corollary}
\begin{corollary}
אופרטור ההורדה והעלה משמרים זוגיות לחילופים. כלומר אם אופרטור הוא סימטרי לחילופים אז תחת אופרטור הורדה ישאר סימטרי ואם אופרטור אנטי סימטרי לחילופים תחת אופרטור ההורדה ישאר אנטי סימטרי.

\end{corollary}
\begin{example}[ערך עצמי מקסימלי סימטרי]
נתונים שני חלקיקים עם תנע זוויתי \(J_{1},J_{2}\). המצבים המתאימים במרחבים שלהם יהיו \(\left\{  \ket{j_{1},m_{1}}  \right\}\) ו-\(\left\{  \ket{j_{2},m_{2}}  \right\}\) כאשר כיוון של חלקיקים זהים מתקיים \(j_{1}=j_{2}\). נראה כי במרחב המצבים העצמיים \(J^{2}=J_{1}^{2}+J_{2}^{2}\) נקבל כי ערך עצמי של \(2j(2j+1)\hbar^{2}\) יהיה סימטרי להחלפה. מכללי הברירה של הצימוד מתקיים:
$$0=|j_{1}-j_{2}|\leq J\leq |j_{1}+j_{2}|=2j$$
כלומר הערך עצמי המקסימלי יהיה \(2j(2j+1)\hbar^{2}\) שזה בדיוק הערך עצמי שעליו אנחנו מסתכלים. נסתכל על מצב עם \(M\) מקסימלי. כלומר המצב עצמי המתאים יהיה:
$$|J_{\mathrm{max}}\,M_{\mathrm{max}}\rangle=|J_{\mathrm{max}}\,J_{\mathrm{max}}\rangle=|j\,j\rangle\otimes|j\,j\rangle $$
שזה בפירוש סימטרי. כעת כיוון שאופרטור ההורדה מתחלף עם אופרטור ההחלפה נקבל כי הזוגיות לא תשתנה תחת הורדה של \(M\) ולכן הטענה תהיה נכונה לכל \(M\leq M_{\max}\).

\end{example}
\begin{example}
נתונים שני חלקיקים עם תנע זוויתי \(J_{1},J_{2}\). המצבים המתאימים במרחבים שלהם יהיו \(\left\{  \ket{j_{1},m_{1}}  \right\}\) ו-\(\left\{  \ket{j_{2},m_{2}}  \right\}\) כאשר כיוון של חלקיקים זהים מתקיים \(j_{1}=j_{2}\). נראה כי במרחב המצבים העצמיים \(J^{2}=J_{1}^{2}+J_{2}^{2}\) מצב עצמי של \(J^{2}\) עם ערך עצמי של \(2j(2j-1)\hbar^{2}\) יהיה אנטי סימטרי להחלפה. זה נובע מכך שזה הערך  המתאים ל-\(J=2j-1\). נשים לב כי:
$$J_{-}|2j\,2j\rangle=\tilde{N}\Big(|j\,j-1\rangle\otimes|j\,j\rangle+|j\,j\rangle\otimes|j\,j-1\rangle\Big)$$
ולכן כיוון שהמצב \(\ket{2j-1\;2j-1}\) יהיה אורתוגונאלי למצב \(J_{-}|2j\,2j\rangle\) המצב המתאים יהיה:
$$|2j-1\;2j-1\rangle=N{\Bigl(}|j\,j-1\rangle\otimes|j\,j\rangle-|j\,j\rangle\otimes|j\,j-1\rangle{\Bigr)}$$
בפרט ניתן לראות במפורש כי \(|2j-1\;2j-1\rangle\) אנטי סימטרי. דרך יותר ישירה להראות זאת זה בעזרת מקדמי קלבש גורדון. אנו יודעים כי:
$$|J,M\rangle=\sum_{m_{1},m_{2}}\langle j,m_{1};j,m_{2}|J,M\rangle\,|j,m_{1}\rangle\otimes|j,m_{2}\rangle$$
כאשר החלפה של \(m_{1},m_{2}\) גורם להיפוך בסימן ולכן ניתן לכתוב:
$$|2j-1,2j-1\rangle=N{\Big(}|j,j-1\rangle\otimes|j,j\rangle-|j,j\rangle\otimes|j,j-1\rangle{\Big)}$$
וזה בפירוש מצב אנטי סימטרי.

\end{example}
\begin{proposition}
עבור מערכת של שתי חלקיקים אופרטור החילוף המרחבי שקול לאופרטור ההחלפה.

\end{proposition}
\begin{corollary}
אם נעבור למערכת מרכז המסה:
$$\vec{r}=\vec{x}_{1}-\vec{x}_{2},\quad\psi\left( \vec{x}_{1},\vec{x}_{2} \right)=\psi\left( \vec{r} \right),\quad\Pi\psi\left( \vec{r} \right)=\psi\left( -\vec{r} \right).$$
ולכן:
$$\psi(\vec{r})=\pm\psi(-\vec{r})\quad\Leftrightarrow\quad\psi(\vec{r})\propto Y_{\ell}^{m}(\hat{r}),\quad Y_{\ell}^{m}(\hat{r})=(-1)^{\ell}Y_{\ell}^{m}(-\hat{r}).$$
כלומר החלק המרחבי של הפונקציית גל הוא סימטרי או אנטי סימטרי אם \(\ell\) זוגי או אי זוגי.

\end{corollary}
\begin{proposition}[מערכת של חלקיק עם ספין חצי]
ניתן לתאר מערכת של אלקטרון בעזרת איבר במרחב הילברט:
$${\mathcal{H}}_{1}={\mathcal{H}}_{1}^{\mathrm{space}}\otimes{\mathcal{H}}_{1}^{\mathrm{spin}}:\big\{|\psi_{1}^{\mathrm{space}}\rangle\big\}\otimes\Big\{\Big|\psi_{1}^{\mathrm{spin}}\Big\rangle\Big\}$$
כאשר \(\ket{\psi_{1}^{\mathrm{space}}}\) זה החלק המרחבי ו-\(\ket{\psi_{1}^{\mathrm{spin}}}\) זה החלק של הספין(\(\ket{\uparrow}\) או \(\ket{\downarrow}\))

\end{proposition}
\begin{corollary}[מערכת של שתי חלקיקים עם ספין חצי]
ניתן לתאר מערכת של שתי אלקטרונים בעזרת המרחב הילברט:
\begin{gather*}|\psi\rangle\in\mathcal{H}=\mathcal{H}_{1}\otimes\mathcal{H}_{2}:\big{\{}|\psi^{\mathrm{space}}\rangle\big{\}}\otimes\bigg{\{}\big{|}\psi^{\mathrm{spin}}\big{\rangle}\bigg{\}}=\\=\mathcal{H}^{\mathrm{space}}\otimes\mathcal{H}^{\mathrm{spin}}:\big{\{}|\psi_{1}^{\mathrm{space}}\psi_{2}^{\mathrm{space}}\rangle\big{\}}\otimes\bigg{\{}\big{|}\psi_{1}^{\mathrm{spin}}\psi_{2}^{\mathrm{spin}}\big{\rangle}\bigg{\}} 
\end{gather*}

\end{corollary}
\begin{definition}[סינגלט וטריפלט]
עבור חלקיק של ספין \(\frac{1}{2}\) נקבל כי עבור \(s=0\) יש מצב 1 הנקרא סינגלט ועבור \(s=1\) יש שלושה מצבים ולכן נקרא טריפלט.

\end{definition}
\begin{proposition}
מצבי הסינגלט יהיו:
$$|\chi\rangle=\frac{1}{\sqrt{2}}[|\uparrow\downarrow\rangle-|\downarrow\uparrow\rangle]$$
כאשר מצבי הטריפלט יהיו:
$$|\chi\rangle=\frac{1}{\sqrt{2}}\left[ |\uparrow\downarrow\rangle+|\downarrow\uparrow\rangle \right]\qquad |\chi\rangle=|\uparrow\uparrow\rangle\qquad |\chi\rangle=|\downarrow\downarrow\rangle$$

\end{proposition}
\begin{proof}
ניתן להראות ישירות בעזרת מקדמי קלבש גורדון. נתחיל מאיברי הבסיס הלא מצומד \(\{ \ket{\uparrow\uparrow},\ket{\uparrow\downarrow},\ket{\downarrow\uparrow},\ket{\downarrow\downarrow} \}\). עבור \(S=0\) נקבל את מצב הסינגלט בבסיס המצומד:
$$|0,0\rangle=\frac{1}{\sqrt{2}}\Big(|\uparrow\downarrow\rangle-|\downarrow\uparrow\rangle\Big)$$
כאשר עבור \(S=1\) נקבל את הטריפלט:
$$|1,1\rangle=|\uparrow\uparrow\rangle \qquad |1,0\rangle=\frac{1}{\sqrt{2}}\Bigl(|\uparrow\downarrow\rangle+|\downarrow\uparrow\rangle\Bigr)\qquad |1,-1\rangle=|\downarrow\downarrow\rangle$$
ניתן לחלופין להגיע לזה מטעמי סימטרייה. נתחיל כמו מקודם מהבסיס \(\{ \ket{\uparrow\uparrow},\ket{\uparrow\downarrow},\ket{\downarrow\uparrow},\ket{\downarrow\downarrow} \}\). נרצה למצוא בסיס למרחב הסמטרי ולמרחב האנטי סימטרי. ונפעיל את האנטי מסמטר על מצב כלשהו שלא סימטרי ונקבל:
$${\mathcal{A}}|\uparrow\downarrow\rangle={\frac{1}{2}}{\Big(}|\uparrow\downarrow\rangle-|\downarrow\uparrow\rangle{\Big)}$$
כאשר לאחר נרמול נקבל את המצב סינגלט:
$$|0,0\rangle=\frac{1}{\sqrt{2}}\Big(|\!\uparrow\downarrow\rangle-|\!\downarrow\uparrow\rangle\Big)$$
נשים לב כעת כי המצבים:
$$|1,1\rangle=|\!\uparrow\uparrow\rangle,\quad|1,-1\rangle=|\!\downarrow\downarrow\rangle.$$
כבר סימטריים. לכן מספיק להפעיל את המסמטר על אחד המצבים עם \(m=0\):
$${\mathcal{S}}|\!\uparrow\downarrow\rangle={\frac{1}{2}}{\Big(}|\!\uparrow\downarrow\rangle+|\!\downarrow\uparrow\rangle{\Big)}.$$

\end{proof}
\begin{corollary}
מצבי הסינגלט הם אנטי סימטרים כאשר מצבי הטריפלט סימטריים.

\end{corollary}
\begin{corollary}
עבור מערכת של שתי חלקיקים עם ספין חצי, מתקיים אחד מהאפשריויות הבאות:

  \begin{enumerate}
    \item פונקציית הגל סימטרית(\(\ell\) זוגי במערכת מרכז המסה) כאשר הספין אנטי סימטרי(סינגלט - \(S=0\)). 


    \item פונקציית הגל אנטי סימטרית(\(\ell\) אי זוגי במערכת מרכז המסה) כאשר הספין סימטרי(טריפלט - \(S = 1\)). 


  \end{enumerate}
\end{corollary}
\begin{definition}[ספין מוגדר היטב]
ספין אשר משתמר לאורך זמן. כלומר \([H,S^{2}]=0\). במקרה זה הסימטריזציה גורר כי המיקום והספין בלתי תלויים.

\end{definition}
\begin{proposition}
אם הספין מוגדר היטב אז ניתן להפריד את הפונקציית גל לגורם של ספין ולגורם מרחבי, כך שבהינתן הזוגיות של גורם מרחבי אנו יכולים לדעת את הזוגיות של הספין.

\end{proposition}
\begin{proposition}[זוגיות של ספין]
הזוגיות של ספין כולל \(S\) המורכב מחיבור של ספינים עם ספין \(s\) נתון על ידי \((-1)^{S-2s}\).

\end{proposition}
\begin{proof}
מהזהות הידועה של סימטריה להחלפת של קלבש גורדון:
$$\langle j_{1}\,m_{1}\,j_{2}\,m_{2}\mid J\,M\rangle=(-1)^{j_{1}+j_{2}-J}\langle j_{2}\,m_{2}\,j_{1}\,m_{1}\mid J\,M\rangle.$$
נקבל כעת כי:
$$\langle ssm_{1}m_{2}|ssSM \rangle =(-1)^{S-2s}\langle ssm_{2}m_{1}|ssSM \rangle $$
ובפרט עבור המצב הגבוה ביותר נקבל \(S=2s\) ולכן המצב סימטרי.

\end{proof}
\begin{corollary}
עבור המצב הגבוה ביותר נקבל כי סימטרי, זה כיוון שעבור מצב זה \(S=s+s=2S\) ולכן הזוגיות תהיה \((-1)^{S-S}=(-1)^{0}=1\).

\end{corollary}
\begin{summary}
  \begin{itemize}
    \item ניתן לכתוב את פונקציית הגל הכולל כשיש ספין על ידי \(\Psi_{t}(x_{i},s_{i})=\chi(s_{i})\,\psi(x_{i})\).
    \item עבור מערכת של בוזונים $$\Psi_{t}=\chi_{+}\,\psi_{+}\,\mathrm{or}\,\Psi_{t}=\chi_{-}\,\psi_{-}.$$
כאשר עבור פרמיונים:
$$\Psi_{t}=\chi_{-}\,\psi_{+}\,\mathrm{or}\,\Psi_{t}=\chi_{+}\,\psi_{-}.$$
    \item במערכת מרכז המסה של שתי חלקיקים:
$$\psi({\vec{r}})=\pm\psi(-{\vec{r}}),$$
כאשר הזוגיות תהיה \((-1)^{\ell}\).
    \item עבור חלקיק עם ספין חצי נקבל מצב סינגלט אנטי סימטרי:
  \end{itemize}
\end{summary}
\chapter{תורת הפיזור}

\section{תורת פיזור קלאסית}

\begin{definition}[בעיית פיזור]
בעיה מצורה הבאה:

 Created with Inkscape (http://www.inkscape.org/) \includegraphics[width=0.8\textwidth]{diagrams/svg_4.svg}
\end{definition}
כלומר יש איזשהו שטף של חלקיקים, שעובר דרך איזשהו חומר אשר מורכב מחלקיקים. כל חלקיק ניתן לתיאור על ידי שטח אפקטיבי כך שאם חלקיק עובר דרך השטח הזה מבצע אינטראקציה עם החלקיק ומשנה את מסלולו, כאשר אם לא עובר דרך השטח האפקטיבי הזה אז אינו מבצע אינטראקציה. לאחר החומר יש חיישן שיכול לזהות את החומרים ששינו מסלולם כתוצאה מהפיזור.

\begin{definition}[חתך פעולה - cross section]
השטח האפקטיבי של כל חלקיק. מסומן בדרך כלל ב-\(\sigma\). באופן כללי מוגדר על ידי:

\end{definition}
\begin{corollary}
אם נניח שיש \(N_{T}\) חלקיקים ויש \(A_{T}\) שטח כולל ושטח חתך \(\sigma\) תוך ההנחה המקלה שאין חפיפה בין החלקיקים הסיכוי לראקציה יהיה
$$P=\frac{N_{T}\cdot \sigma}{A_{T}}$$

\end{corollary}
\begin{proposition}
מספר החלקיקים \(N_{T}\) מקיים:
$$N_{T}=n_{T}\cdot A_{T} \cdot d_{T}$$
כאשר \(d_{T}\) זה העובי, \(n_{T}\) זה הצפיפות ו-\(A_{T}\) זה השטח.

\end{proposition}
\begin{corollary}
אם נציב את \(N_{T}\) בביטוי עבור ההסתברות נקבל:
$$P=\frac{\sigma n_{T}A_{T}d_{T}}{A_{T}}=\sigma n_{T}d_{T}$$

\end{corollary}
\begin{definition}[שטף החלקיקים הנכנסים]
כמות החלקיקים לשטח לזמן. כלומר:
$$\phi_{p}=\frac{\mathrm{Number~of~particles}}{\mathrm{Area}\times\mathrm{Time}}$$
כאשר היחידות יהיו \(\left[ \phi_{p} \right]=\left[ \frac{1}{L^{2}t} \right]\).

\end{definition}
\begin{definition}[קצב מניות בגלאי]
$$R_{D}=\overbrace{ \phi_{D}\cdot A_{T} }^{ \text{Current} }\cdot \overbrace{ \sigma \cdot n_{T}\cdot d_{T} }^{ \text{Reaction Prob} }$$

\end{definition}
\begin{definition}[קצב מניות לחלקיק]
כמות המניות שיש בגלאי לכל חלקיק במערכת:
$$\Gamma=\frac{R_{D}}{N_{T}}=\phi_{p}\sigma$$

\end{definition}
\begin{corollary}
בהינתן קצב מנויות לחלקיק ניתן לקבל את החתך פעולה:
$$\sigma=\frac{\Gamma}{\phi_{p}}$$

\end{corollary}
\begin{proposition}
בהנתן הפיזור על זווית מרחבית \(\Delta \Omega\) מתקיים:
$$R_{D}=\phi_{p}N_{T}\frac{\mathrm{d} \sigma}{\mathrm{d} \Omega} \Delta \Omega$$

\end{proposition}
\begin{summary}
  \begin{itemize}
    \item החתך פעולה זה השטח האפקטיבי של חלקיק אשר קובע את ההסתברות לאינטראקציה
    \item ההסתברות לאינטראקציה נתונה על ידי \(P=\sigma n_{T}d_{T}\) כאשר \(n_{T}\) זה צפיפות החלקיקים בחומר ו-\(d_{T}\) זה עובי החומר.
    \item שטף החלקיקים הנכנס זה כמות החלקיקים לשטח לזמן. 
    \item הקצב מניות בגלאי מוגדר על ידי \(R_{D}=\phi_{p}\cdot A_{T}\cdot \sigma \cdot n_{T}\cdot d_{T}\).
    \item הקצב מניות לחלקיק מוגדר על ידי \(\Gamma=\phi_{p}\sigma\).
    \item עבור פיזור לזווית מרחבית \(\Delta \Omega\) מתקיים \(R_{D}=\phi_{p}N_{T}\frac{d\sigma}{d\Omega}\Delta\Omega\).
  \end{itemize}
\end{summary}
\section{תורת פיזור קוונטית}

\begin{proposition}
בתמונה הקוונטית של פיזור, נתאר את השטף חלקיקים על ידי גל מישורי:
$$\Phi_{k}(x)=\langle x|k\rangle=\frac{1}{\sqrt{V}}e^{i k\cdot x}$$
כאשר \(V\) הוא הנפח.

\end{proposition}
\begin{proposition}
לאחר הפיזור, הפונקציית גל תהיה הסכום של הגל המישורי הנכנס והגל כדורי מפוזר:
$$\Psi_{k}(x)=\Phi_{k}(x)+\Psi_{s c}(x)=\frac{1}{\sqrt{{V}}}\left(e^{i k\cdot x}+f_{k}(\hat{x})\frac{e^{i k r}}{r}\right)$$
כאשר \(f_{k}\left( \hat{x} \right)\) זה אמפליטודת הפיזור אשר מכיל את המידע על איך הגל מתפזר לכיוונים שונים.

\end{proposition}
\begin{reminder}
שטף ההסתברות של גל נתון על ידי:
$$J=-\frac{i\hbar}{2m}\left(\Psi^{*}\nabla\Psi-\Psi\nabla\Psi^{*}\right)$$
כך שמקיים \(\bar{\nabla} \cdot \vec{J}=0\).

\end{reminder}
\begin{proposition}
עבור \(r\to \infty\) נקבל:
$$J_{\mathrm{sc}}=\frac{v}{{V}}\frac{|f_{k}\left( \hat{x} \right)|^{2}}{r^{2}}\hat{r}$$
כאשר \(v\) זה המהירות של החלקיק.

\end{proposition}
\begin{proof}
ממשוואת שרדינגר עבור \(r\) גדול נקבל
$$H=\left[-\frac{\hbar^{2}}{2m}\nabla^{2}+V(x)\right]\,\xrightarrow[r\to\infty]{}\,-\frac{\hbar^{2}}{2m}\left(\frac{\partial^{2}}{\partial r^{2}}+\frac{2}{r}\frac{\partial}{\partial r}\right)$$
כאשר אם נתעלם מהגורמים הדועכים לינארית נקבל
$$H\Psi_{s c}=\frac{\hbar^{2}k^{2}}{2m}\Psi_{s c}$$
כאשר נקבל כעת כי
$$J_{s c}=\frac{1}{{V}}\frac{\hbar k}{m}\frac{|f_{k}(\hat{x})|^{2}}{r^{2}}\hat{r}=\frac{v}{{V}}\frac{|f_{k}(\hat{x})|^{2}}{r^{2}}\hat{r}$$

\end{proof}
\begin{corollary}
המספר החלקיקים המתפזרים ליחידת זמן עבור זווית מרחבית \(\mathrm{d}\Omega\) תהיה:
$$d R=J_{s c}\,d A={\frac{1}{V}}\,{\frac{v\,|f({\hat{r}})|^{2}}{r^{2}}}\,(r^{2}\,d\Omega)={\frac{v}{V}}\,|f({\hat{r}})|^{2}\,d\Omega$$

\end{corollary}
\begin{reminder}
במקרה הקלאסי הגדרנו \(\Gamma=\frac{R_{D}}{N_{T}}\) בתור הקצב תגובות לחלקיק, ומשם קיבלנו את החתך פעולה על ידי \(\sigma=\frac{\Gamma}{\phi_{p}}\) כאשר \(\phi_{p}\) זה השטף הנכנס.

\end{reminder}
\begin{corollary}
מתקיים:
$$\frac{d\sigma}{d\Omega}=\frac{d R/d\Omega}{\phi_{p}}$$

\end{corollary}
\begin{proposition}
מתקיים:
$$\frac{d\sigma}{d\Omega}=|f(\hat{x})|^{2}$$

\end{proposition}
\begin{proof}
עבור גל נכנס מהצורה:
$$\Phi(\mathbf{r})={\frac{1}{\sqrt{V}}}e^{i\mathbf{k}\cdot\mathbf{r}}$$
הזרם הסתברות יהיה \(\phi_{p}=\frac{v}{V}\) כך שמתקיים:
$$\frac{d R}{d\Omega}=\frac{v}{V}\,|f(\hat{r})|^{2}$$
ולכן מהמסקנה:
$$\frac{d\sigma}{d\Omega}=\frac{\frac{v}{V}\,|f(\hat{r})|^{2}}{\frac{v}{V}}=|f(\hat{r})|^{2}$$

\end{proof}
\begin{corollary}
החתך פעולה הכולל לזווית יהיה:
$$\sigma_{t o t}=\int d\Omega\,|f_{k}(\hat{x})|^{2}$$

\end{corollary}
\begin{summary}
  \begin{itemize}
    \item נתאר את החלקיקים הנכנסים בתור פונקצית גל מישורית \(\Phi_{k}(x)=\frac{1}{\sqrt{ V }}e^{ i\mathbf{k}\cdot \mathbf{x} }\).
    \item המצב המפוזר יהיה סופרפוזציה של הגל הנכנס וגלים כדוריים יוצאים:
$$\Psi_{k}(x)=\Phi_{k}(x)+\frac{f_{k}(\hat{x})}{\sqrt{V}}\,\frac{e^{i k r}}{r}$$
כאשר \(f_{k}\left( \hat{x} \right)\) מבטא את התלוי הזוויתית של הפיזור.
    \item החתך פעולה כתלות בזווית המרחבית מקיים \(\frac{\mathrm{d} \sigma}{\mathrm{d} \Omega}=\left\lvert  f\left( \hat{x} \right)  \right\rvert^{2}\).
    \item עבור הגל המתפזר כאשר \(r\to \infty\) מתקיים:
$$J_{\mathrm{sc}}={\frac{v}{V}}{\frac{|f({\hat{x}})|^{2}}{r^{2}}}{\hat{r}}$$
    \item הקצב פיזור חלקיקים ליחידת זמן ביחס לזווית מרחבית נתונה על ידי:
$$d R=\frac{v}{V}|f(\hat{x})|^{2}d\Omega$$
  \end{itemize}
\end{summary}
\section{פיזור במרחב אינסופי}

\begin{proposition}
ניתן לכתוב את משוואת שרדונגר בצורה של משוואת הלמהודס:
$$\left(\nabla^{2}+k^{2}\right)\psi(\mathbf{r})=Q(\mathbf{r}),$$
כאשר:
$$k\equiv\frac{\sqrt{2m E}}{\hbar}\quad\mathrm{and}\quad Q({\bf r})\equiv\frac{2m}{\hbar^{2}}\,V({\bf r})\,\psi({\bf r}).$$

\end{proposition}
\begin{proof}
נובע מסידור מחדש של משוואת שרודנגר הבלתי תלויה בזמן:
$$-\frac{\hbar^{2}}{2m}\nabla^{2}\psi\left( \mathbf{r} \right)+V\left( \mathbf{r} \right)\psi\left( \mathbf{r} \right)=E\,\psi\left( \mathbf{r} \right)\implies-\nabla^{2}\psi\left( {\bf r} \right)+\frac{2m}{\hbar^{2}}\,V\left( {\bf r} \right)\psi\left( {\bf r} \right)=\frac{2m E}{\hbar^{2}}\,\psi\left( {\bf r} \right)$$
כאשר אם נגדיר:
$$k^{2}\equiv\frac{2m E}{\hbar^{2}},\quad\mathrm{and}\quad Q({\bf r})\equiv\frac{2m}{\hbar^{2}}\,V({\bf r})\psi({\bf r}),$$
נקבל את הטענה.

\end{proof}
\begin{reminder}[פונקציית גרין של משוואת הלמהודס]
פונקציה \(G\left( \mathbf{r} \right)\) אשר מקיימת:
$$\left(\nabla^{2}+k^{2}\right)G(\mathbf{r})=\delta^{3}(\mathbf{r})$$

\end{reminder}
\begin{proposition}[פונקציית גרין של משוואת הלמהולדס]
הביטוי:
$$\psi(\mathbf{r})=\int G(\mathbf{r}-\mathbf{r}_{0})\,Q(\mathbf{r}_{0})\,d^{3}\mathbf{r}_{0}.$$
יהיה פתרון של משוואת שרדינגר.

\end{proposition}
\begin{proof}
\begin{gather*}\left(\nabla^{2}+k^{2}\right)\psi\left(\mathbf{r}\right)=\int\left[\left(\nabla^{2}+k^{2}\right)G\left( \mathbf{r}-\mathbf{r}_{0} \right)\right]Q\left( \mathbf{r}_{0} \right)\,d^{3}\mathbf{r}_{0}=\\=\int\delta^{3}\left( {\bf r}-{\bf r}_{0} \right)\,Q\left( {\bf r}_{0} \right)\,d^{3}{\bf r}_{0}=Q\left( {\bf r} \right). 
\end{gather*}

\end{proof}
\begin{proposition}[ביטוי מפורש עבור הפונקציית גרין]
נתין לכתוב במפורש את הפונקציית גרין של משוואת הלמהולדס עם תנאי שפה דועכים באינסוף על ידי:
$$G\left( \mathbf{r} \right)=-\frac{e^{i k |\mathbf{r}-\mathbf{r}'|}}{4\pi |\mathbf{r}-\mathbf{r}'|}$$

\end{proposition}
\begin{proof}
נכתוב את \(G\left( \mathbf{r} \right)\) בעזרת התמרת פורייה:
$$(*)\quad G\left( {\bf r} \right)=\frac{1}{\left(2\pi\right)^{3/2}}\int e^{i{\bf s\cdot r}}g\left( {\bf s} \right)\,d^{3}{\bf s}.$$
כעת:
$$\left(\nabla^{2}+k^{2}\right)G(\mathbf{r})={\frac{1}{(2\pi)^{3/2}}}\int\left[\left(\nabla^{2}+k^{2}\right)e^{i\mathbf{s}\cdot\mathbf{r}}\right]g(\mathbf{s})\,d^{3}\mathbf{s}.$$
כאשר נשתמש בזה שמתקיים \(\bar{\nabla}^2e^{ i\mathbf{s}\cdot \mathbf{r} }=-s ^{2}e^{ i\mathbf{s}\cdot \mathbf{r} }\) וגם \(\delta^{3}({\bf r})=\frac{1}{(2\pi)^{3}}\int e^{i{\bf s\cdot r}}\,d^{3}{\bf s}\) ונציב במשוואה:
$${\frac{1}{(2\pi)^{3/2}}}\int\left(-s^{2}+k^{2}\right)e^{i{\bf s\cdot r}}g\;({\bf s})\;d^{3}{\bf s}={\frac{1}{(2\pi)^{3}}}\int e^{i{\bf s\cdot r}}\,d^{3}{\bf s}.$$
כלומר קיבלנו כי:
$$g({\bf s})=\frac{1}{(2\pi)^{3/2}\left(k^{2}-s^{2}\right)}.$$
נציב בחזרה ב-\((*)\) ונקבל:
$$G({\bf r})=\frac{1}{(2\pi)^{3}}\int e^{i{\bf s\cdot r}}\frac{1}{\left(k^{2}-s^{2}\right)}\,d^{3}{\bf s}.$$
כעת נשים לב כי \(\mathbf{r}\) מקובע(האינטגרל לא תלוי בו), לכן נגדיר מערכת קורדינטות כדוריות כאשר \(\mathbf{r}\) הוא ציר ה-\(\hat{z}\)(הציר הפולארי). כעת נקבל כי:
$$\mathbf{r}\cdot \mathbf{s}=s r\cos \theta$$
כאשר האינטגרל על \(\varphi\) נהיה טריוויאלי ולכן מוסיף פשוט פקטור של \(2\pi\). האינטגרל ב-\(\theta\) נותן:
$$\int_{0}^{\pi}e^{i s r\cos\theta}\sin\theta\,d\theta=-\frac{e^{i s r\cos\theta}}{i s r}\bigg|_{0}^{\pi}=\frac{2\sin(s r)}{s r}.$$
ולכן נקבל:
$$G(\mathbf{r})={\frac{1}{(2\pi)^{2}}}{\frac{2}{r}}\int_{0}^{\infty}{\frac{s\sin(s r)}{k^{2}-s^{2}}}d s={\frac{1}{4\pi^{2}r}}\int_{-\infty}^{\infty}{\frac{s\sin(s r)}{k^{2}-s^{2}}}d s.$$
זה לא אינטגרל פשוט. נרצה לפתור אותו על ידי פיצול לשתי אינטגרלים ושימוש באינטגרציה מרוכבת ומשפט השארית של קושי. ראשית נכתוב:
$$G(\mathbf{r})={\frac{i}{8\pi^{2}r}}\bigg\{\underbrace{ \int_{-\infty}^{\infty}{\frac{s e^{i s r}}{\left(s-k\right)\left(s+k\right)}}d s }_{ I_{2} }-\underbrace{ \int_{-\infty}^{\infty}{\frac{s e^{-i s r}}{\left(s-k\right)\left(s+k\right)}}d s }_{ I_{2} }\bigg\}=\frac{i}{8\pi^{2}r}\left(I_{1}\,-\,I_{2}\right)$$
כאשר נזכור כי משפט השארית של קושי אומר כי:
$$\oint\frac{f(z)}{(z-z_{0})}d z=2\pi i f(z_{0}),$$
וניתן למצוא מסלולים מתאימים כך שמתקיים:
$$I_{1}=\oint\left[\frac{s e^{i s r}}{s+k}\right]\frac{1}{s-k}\,d s=2\pi i\left[\frac{s e^{i s r}}{s+k}\right]\biggr\vert_{s=k}=i\pi e^{i k r}$$
וכן:
$$I_{2}=-\oint\left[\frac{s e^{-i s r}}{s-k}\right]\frac{1}{s+k}\,d s=-2\pi i\left[\frac{s e^{-i s r}}{s-k}\right]\biggr|_{s=-k}=-i\pi\,e^{i k r}$$
ולכן נקבל סה"כ:
$$G({\bf r})=\frac{i}{8\pi^{2}r}\left[\left(i\pi e^{i k r}\right)-\left(-i\pi e^{i k r}\right)\right]=-\frac{e^{i k r}}{4\pi r}$$
כאשר זה למעשה נכון מרחבית עבור \(r \mapsto |\mathbf{r}-\mathbf{r'}|\).

\end{proof}
\begin{corollary}[פתרון כללי]
הפתרון הכללי של משוואת שרדינגר תהיה מהצורה:
$$\psi\left(\mathbf{r}\right)=\psi_{0}(\mathbf{r})-{\frac{m}{2\pi\hbar^{2}}}\int{\frac{e^{i k\left|\mathbf{r}-\mathbf{r}_{0}\right|}}{\left|\mathbf{r}-\mathbf{r}_{0}\right|}}V\left(\mathbf{r}_{0}\right)\psi\left(\mathbf{r}_{0}\right)d^{3}\mathbf{r}_{0}$$

\end{corollary}
\begin{remark}
כאשר \(r\to \infty\) נקבל כמו שמצופה מפיזור סכום של הגל הנכנס עם גל כדורי מתפשט.

\end{remark}
\begin{definition}[מצבים נכנסים - In States]
מצבים אשר בזמן \(t\to -\infty\) יהיו חופשיים, כלומר ללא השפעה של הפוטנציאל פיזור.

\end{definition}
\begin{remark}
מתאר הפועל את חלקיקים לפני שהם נכנסים לפוטנציאל של הפיזור. חלקיקים אלו אינם מושפעים מהפוטנציאל פיזור ולכן רק יכולו את הרכיב הקינטי.

\end{remark}
\begin{symbolize}
מסמנים את המצבים האלה ב-\(\psi^{\text{in}}\) או \(\psi^{\text{in}}_{k}\) אם רוצים להדגיש שמדובר בגל מונוכרומטי עם אורך גל \(k\). לעיתים מסומן גם ב-\(\psi^{(+)}\).

\end{symbolize}
\begin{corollary}
מצבים נכנסים מקיימים:
$$H\Psi_{\bf k}^{\mathrm{in}}=\frac{{\hbar^{2}\bf k}^{2}}{2m}\Psi_{\bf k}^{\mathrm{in}}$$

\end{corollary}
\begin{corollary}
עבור מספר גל \(\mathbf{k}\) מתקיים:
$$\psi(\mathbf{r})=(2\pi\hbar)^{-3/2}e^{i\mathbf{k}\cdot\mathbf{r}}+\int G(\mathbf{r},\mathbf{r}^{\prime})V(\mathbf{r}^{\prime})\psi(\mathbf{r}^{\prime})d^{3}\mathbf{r}^{\prime}$$

\end{corollary}
\begin{proposition}
הפונקציית גרין מקיימת:
$$G(\mathbf{r},\mathbf{r^{\prime}})=\lim_{ \varepsilon \to 0 } \langle\mathbf{r}|{\frac{1}{E-H_{0}+i\epsilon}}|\mathbf{r^{\prime}}\rangle$$

\end{proposition}
\begin{proof}
אנו יודעים כי \(E=\frac{\hbar^{2}k^{2}}{2m}\) ולכן ניתן לכתוב:
$$E-\frac{\hbar^{2}q^{2}}{2m}+i\epsilon=\frac{\hbar^{2}}{2m}\Big(k^{2}-q^{2}+\frac{2m}{\hbar^{2}}i\epsilon\Big)$$
כאשר נגדיר \(\epsilon'=\frac{2m}{\hbar^{2}}\epsilon\). כעת מתקיים:
$$\langle{\bf r}|\frac{1}{E-H_{0}+i\epsilon}|{\bf r^{\prime}}\rangle=\frac{2m}{\hbar^{2}}\int\frac{d^{3}q}{(2\pi)^{3}}\,\frac{e^{i{\bf q\cdot(r-r^{\prime})}}}{k^{2}-q^{2}+i\epsilon^{\prime}}.$$
ונדרש רק לחשב את האינטגרל. נסמן \(\mathbf{R}=\mathbf{r}-\mathbf{r'}\). נעבור לקורדינטות כדוריות כך שציר ה-\(\mathbf{z}\) יהיה ציר ה-\(\mathbf{R}\). לכן מתקיים \(\mathbf{q}\cdot \mathbf{R}=qR\cos \theta\) ולכן:
$$I(\mathbf{R})={\frac{1}{(2\pi)^{2}}}\int_{0}^{\infty}d q\,{\frac{q^{2}}{k^{2}-q^{2}+i\epsilon^{\prime}}}\int_{0}^{\pi}d\theta\,\sin\theta\,e^{i q R\cos\theta}.$$
כאשר החלק הזוויתי נותן:
$$\int_{0}^{\pi}\sin\theta\,e^{i q R\cos\theta}d\theta=\frac{2\sin(q R)}{q R}.$$
ולכן:
$$I(\mathbf{R})={\frac{1}{(2\pi)^{2}R}}\int_{0}^{\infty}d q\,{\frac{q\,\sin(q R)}{k^{2}-q^{2}+i\epsilon^{\prime}}}.$$
בעזרת אינטגרציה מרוכבת כיוון שאין סינגולריות על ציר הממשי נבצע אינטגרל על הציר הממשי וחצי המישור העליון. נקבל \(I\left( \mathbf{R} \right)=-\frac{e^{ ikR }}{4\pi R}\) ולכן:
$$\langle\mathbf{r}|{\frac{1}{E-H_{0}+i\epsilon}}|\mathbf{r}^{\prime}\rangle={\frac{2m}{\hbar^{2}}}\left(-{\frac{e^{i k|\mathbf{r}-\mathbf{r}^{\prime}|}}{4\pi|\mathbf{r}-\mathbf{r}^{\prime}|}}\right)$$
כאשר קיבלנו את הפונקציית גרין(הפאקטורי נרמול פחות מעניינים אותנו, ותלויים בהגדרה של פונקציית גרין).

\end{proof}
\begin{corollary}[משוואת ליפמאן שכווינגר - Lippmann Schwinger]
$$|\psi_{\bf k}^{i n}\rangle=|{\bf k}\rangle+\frac{1}{E-H_{0}+i\epsilon}V|\psi_{\bf k}^{i n}\rangle$$

\end{corollary}
\begin{proof}
ניתן לכתוב את הפתרון הכללי על ידי:
$$\psi\left( \mathbf{r} \right)=\braket{ \mathbf{r} | \mathbf{k} } +\int G\left( \mathbf{r},\mathbf{r}^{\prime} \right)V\left( \mathbf{r}^{\prime} \right)\psi\left( \mathbf{r}^{\prime} \right)d^{3}\mathbf{r}^{\prime}$$
יהי \(\varepsilon\) שרירותי אינפיניטסימלי. נשתמש בביטוי \(G(\mathbf{r},\mathbf{r^{\prime}})=\langle\mathbf{r}|{\frac{1}{E-H_{0}+i\epsilon}}|\mathbf{r^{\prime}}\rangle\) ונקבל:
$$\langle\mathbf{r}|\psi_{\mathbf{k}}^{i n}\rangle=\langle\mathbf{r}|\mathbf{k}\rangle+\int\langle\mathbf{r}|{\frac{1}{E-H_{0}+i\epsilon}}|\mathbf{r}^{\prime}\rangle\langle\mathbf{r}^{\prime}|V|\psi_{\mathbf{k}}^{i n}\rangle d^{3}\mathbf{r}^{\prime}$$
נשים לב כי יש \(\ket{\mathbf{r'}}\bra{\mathbf{r'}}\) פונקציית דלתא בתוך האינטגרל. לכן ניתן "לבטל" את האינטגרל ולקבל:
$$\langle\mathbf{r}|\psi_{\mathbf{k}}^{i n}\rangle=\langle\mathbf{r}|\mathbf{k}\rangle+\langle\mathbf{r}|{\frac{1}{E-H_{0}+i\epsilon}}V|\psi_{\mathbf{k}}^{i n}\rangle$$
כאשר אם נסיר את הבסיס המיקום נקבל:
$$|\psi_{\bf k}^{i n}\rangle=|{\bf k}\rangle+\frac{1}{E-H_{0}+i\epsilon}V|\psi_{\bf k}^{i n}\rangle$$

\end{proof}
\begin{summary}
  \begin{itemize}
    \item ניתן לכתוב את משוואת שרודינגר בצורה אינטגרלית בתור קונבולוציה עם הפונקציית גרין \(G\left( \mathbf{r} \right)=-\frac{e^{i k |\mathbf{r}-\mathbf{r}'|}}{4\pi |\mathbf{r}-\mathbf{r}'|}\):
$$\psi({\bf r})\;=\;\psi_{0}({\bf r})\;-\;\frac{m}{2\pi\hbar^{2}}\,\int\frac{e^{i k|{\bf r}-{\bf r}^{\prime}|}}{|{\bf r}-{\bf r}^{\prime}|}V({\bf r}^{\prime})\,\psi({\bf r}^{\prime})\,d^{3}{\bf r}^{\prime}$$
    \item המצבים נכנסים(In states) הם מצבים אשר ב-\(t\to -\infty\) מתנהגים כאילו לא משפעים מהפוטנציאל, ולכן יהיו גלים מישוריים ומקיימים:
$$\left(H-E\right)\psi_{\bf k}^{\mathrm{in}}\;=\;0$$
    \item ניתן להציג את הפונקציית גרין בצורה מטריציונית על ידי:
$$G(\mathbf{r},\mathbf{r^{\prime}})=\lim_{ \varepsilon \to 0 } \langle\mathbf{r}|{\frac{1}{E-H_{0}+i\epsilon}}|\mathbf{r^{\prime}}\rangle$$
כאשר ה-\(+i\epsilon\) נועד כדי להבטיח הפיכות ותנאי שפה.
    \item ניתן לתאר את המצבים הנכנסים (In states) בזמן סופי על ידי משוואת Lippmann-Schwinger:
$$|\psi_{\bf k}^{\mathrm{in}}\rangle\ =\ |{\bf k}\rangle\ +\ \frac{1}{E-H_{0}+i\epsilon}\,V\,|\psi_{\bf k}^{\mathrm{in}}\rangle$$
  \end{itemize}
\end{summary}
\section{אופרטור T והקירוב האופטי}

\begin{reminder}
ניתן לכתוב את הגל היוצא של פיזור עבור \(r\) גדול על ידי סופרפוזיציה של הגל המישורי הנכנס והגל הכדורי המתפזר:
$$\psi(r,\theta)\approx A\left\{e^{i k z}+f(\theta)\frac{e^{i k r}}{r}\right\}$$
כאשר \(f\left( \theta \right)\) נקרא אמפליטדת הפיזור.

\end{reminder}
\begin{definition}[איזור שדה רחוק]
נניח כי \(V_{0}\left( \mathbf{r_{0}} \right)\). כלומר ניתן להניח כי הפוטנציאל הוא 0 מחוץ לתחום סופי. התחום \(|\mathbf{r}|\gg |\mathbf{r_{0}}|\) יהיה איזור קירוב בורן.

\end{definition}
\begin{remark}
זה דומה להגדרה של איזור הקרינה בתורה האלקטרומגנטית.

\end{remark}
\begin{lemma}
בתחום של איזור קירוב בורן מתקיים:
$$\frac{e^{i k|\mathbf{r}\,\cdot\,\mathbf{r}_{0}|}}{|\mathbf{r}-\mathbf{r}_{0}|}\approx\frac{e^{i k r}}{r}e^{-i\mathbf{k}\cdot\mathbf{r}_{0}}$$

\end{lemma}
\begin{proof}
כיוון ש-\(\frac{r_{0}}{r}\ll 1\) מתקיים:
$$|\mathbf{r}-\mathbf{r}_{0}|^{2}=r^{2}+r_{0}^{2}-2\mathbf{r}\cdot\mathbf{r}_{0}\approx r^{2}\left(1-2{\frac{\mathbf{r}\cdot\mathbf{r}_{0}}{r^{2}}}\right)$$
ולכן:
$$|\mathbf{r}-\mathbf{r}_{0}|\approx r-{\hat{r}}\cdot\mathbf{r}_{0}.$$
כעת נסמן \(\mathbf{k}=k\hat{r}\). מתקיים:
$$e^{i k|\mathbf{r}-\mathbf{r}_{0}|}\approx e^{i k r}e^{-i\mathbf{k}\cdot\mathbf{r}_{0}}\implies\frac{e^{i k|\mathbf{r}\cdot\mathbf{r}_{0}|}}{|\mathbf{r}-\mathbf{r}_{0}|}\approx\frac{e^{i k r}}{r}e^{-i\mathbf{k}\cdot\mathbf{r}_{0}}$$

\end{proof}
\begin{remark}
זה זהה לקירוב בגלים שנקרא הקירוב הפארא-אקסיאלי.

\end{remark}
\begin{proposition}
אמפליטדת הפיזור מקיימת:
$$f(\theta,\phi)=-\frac{m}{2\pi\hbar^{2}A}\int e^{-i{\bf k\cdot r_{0}}}V({\bf r_{0}})\psi({\bf r_{0}})\,d^{3}{\bf r_{0}}.$$

\end{proposition}
\begin{proof}
נזכור כי:
$$\psi\left(\mathbf{r}\right)=\psi_{0}(\mathbf{r})-{\frac{m}{2\pi\hbar^{2}}}\int{\frac{e^{i k\left|\mathbf{r}-\mathbf{r}_{0}\right|}}{\left|\mathbf{r}-\mathbf{r}_{0}\right|}}V(\mathbf{r}_{0})\psi\left(\mathbf{r}_{0}\right)d^{3}\mathbf{r}_{0},$$
כאשר במקרה שלנו(פיזור) נקבל כי \(\psi_{0}(r)\) הוא מצב נכנס(in-state) ולכן גל מישורי, ויהיה מהצורה \(\psi_{0}\left( \mathbf{r} \right)=Ae^{ ikz }\). ולכן בעזרת הלמה הקודמת נקבל:
$$\psi\left(\mathbf{r}\right)\approx A e^{i k z}-\frac{m}{2\pi\hbar^{2}}\frac{e^{i k r}}{r}\int e^{-i\mathbf{k}\cdot\mathbf{r}_{0}}V\left(\mathbf{r}_{0}\right)\psi\left(\mathbf{r}_{0}\right)d^{3}\mathbf{r}_{0}.$$
כאשר נזהה כי:
$$f(\theta,\phi)=-\frac{m}{2\pi\hbar^{2}A}\int e^{-i{\bf k\cdot r_{0}}}V({\bf r_{0}})\psi({\bf r_{0}})\,d^{3}{\bf r_{0}}$$$$\sigma_{\text{tot}}\left( \mathbf{k} \right)=\frac{4\pi}{k}\mathrm{Im}\left( f_{k}\left( \theta=0 \right) \right)$$\textbf{הערה}
זה הביטוי המדוייק. אין פה שום קירוב עדיין חוץ מההנחה הרגילה של פיזור שאנחנו מסתכלים על \(r\) גדול.

\end{proof}
\begin{corollary}
ניתן לכתוב בצורה מטריציונית ולקבל:
$$f(k,k^{\prime})=-\frac{m}{2\pi\hbar^{2}}(2\pi\hbar)^{3}\langle k^{\prime}|V|\Psi_{k}^{\mathrm{in}}\rangle$$

\end{corollary}
\begin{definition}[אופרטור T]
אופרטור המקיים:
$$V|\Psi_{k}^{\mathrm{in}}\rangle=T|k\rangle$$

\end{definition}
\begin{corollary}
$$T=V+VGT=V+\left(E_{k}-H_{0}+i\epsilon\right)^{-1}T$$

\end{corollary}
\begin{proof}
אנו יודעים ממשוואת Lippmann-Schwinger כי מתקיים:
$$|\psi_{\bf k}^{i n}\rangle=|{\bf k}\rangle+\frac{1}{E-H_{0}+i\epsilon}V|\psi_{\bf k}^{i n}\rangle$$
אם נכפיל ב-\(V\) נקבל:
$$T|\mathbf{k}\rangle=V|\mathbf{k}\rangle+V\left(E_{k}-H_{0}+i\epsilon\right)^{-1}T|\mathbf{k}\rangle$$

\end{proof}
\begin{corollary}
$$f(k,k^{\prime})=-\frac{m}{2\pi\hbar^{2}}(2\pi\hbar)^{3}\langle k^{\prime}|T|k\rangle$$

\end{corollary}
\begin{proof}
נובע מכך שמתקיים:
$$f(k,k^{\prime})=-\frac{m}{2\pi\hbar^{2}}(2\pi\hbar)^{3}\langle k^{\prime}|V|\Psi_{k}^{\mathrm{in}}\rangle$$

\end{proof}
\begin{proposition}[המשפט האופטי]
החתך פעולה מקיים:
$$\sigma_{\text{tot}}\left( \mathbf{k} \right)=\frac{4\pi}{k}\mathrm{Im}\left( f_{k}\left( \theta=0 \right) \right)$$

\end{proposition}
\begin{proof}
מהמטריצת \(T\) מתקיים:
$$f_{k}\left( \theta=0 \right)=f(k,k)=-\frac{m}{2\pi\hbar^{2}}\left( 2\pi\hbar \right)^{3}\langle k|T|k\rangle$$
כאשר ממשוואת Lippmann-Schwinger נקבל:
$$\mathrm{Im}\,\langle k|T|k\rangle=\mathrm{Im}\,\langle k|V|\Psi_{k}^{i n}\rangle=\mathrm{Im}\,\left[\left(\langle\Psi_{\,\,k}^{i n}|-\langle\Psi_{\,\,k}^{i n}|V\frac{1}{E-H_{0}+i\epsilon}\right)V|\Psi_{\,\,k}^{i n}\rangle\right]$$
כאשר כעת ניתן לכתוב:
$$\int_{-\infty}^{\infty}d x{\frac{f(x)}{x-x_{0}}}=\operatorname*{lim}_{\delta\to0}\left[\int_{-\infty}^{-\delta}d x{\frac{f(x)}{x-x_{0}}}+\int_{\delta}^{\infty}d x{\frac{f(x)}{x-x_{0}}}\right]$$
ולשים לב כי:
\begin{gather*}\int_{-\infty}^{\infty}d E\frac{f(E)}{E-E_{0}+i\epsilon}=\lim_{\delta\to0}\left[\int_{-\infty}^{-\delta}\!d E\frac{f(E)}{E-E_{0}+i\epsilon}+\int_{\epsilon}\!d E\frac{f(E)}{E-E_{0}+i\epsilon}+\int_{\delta}^{\infty}\!d E\frac{f(E)}{E-E_{0}+i\epsilon}\right]=\\=\int_{-\infty}^{\infty}d E\frac{f(E)}{E-E_{0}+i\epsilon}+i\pi f(E_{0}) 
\end{gather*}
כאשר \(c\) זו מסילה של חצי מעגל סביב \(-\delta\) ל-\(\delta\) על הציר הממשי. להשלים.

\end{proof}
\begin{remark}
המשפט האופטי הוא משפט מאוד עדין. לא נכון להשתמש בו כאשר נתון קירוב של \(f\), רק כאשר נתון את \(f\) בצורה מדוייקת.

\end{remark}
\begin{summary}
  \begin{itemize}
    \item אופרטור ה-\(T\) מוגדר על ידי \(V|\psi_{\bf k}^{\mathrm{in}}\rangle=T|k\rangle\) כך שמקיים:
$$T=V+V\frac{1}{E-H_{0}+i\epsilon}T$$
    \item ניתן לכתוב את הפיזור בעזרת אופרטור ה-\(T\) על ידי:
$$f(k,k^{\prime})=-\frac{m}{2\pi\hbar^{2}}(2\pi\hbar)^{3}\langle k^{\prime}|T|k\rangle.$$
    \item המשפט האופטי אומר כי:
$$\sigma_{\mathrm{tot}}(k)=\frac{4\pi}{k}\mathrm{Im}\,f_{k}(\theta=0)$$
  \end{itemize}
\end{summary}
\section{קירוב בורן}

\begin{proposition}
ניתן לקרב את \(T\) בצורה הבאה:
$$T=V+V{\frac{1}{E-H_{0}+i\varepsilon}}V+V{\frac{1}{E-H_{0}+i\varepsilon}}V{\frac{1}{E-H_{0}+i\varepsilon}}V+\cdots$$

\end{proposition}
\begin{proof}
המטריצת \(T\) מקיימת:
$$T=V+V\,G_{0}\,T$$
כאשר ניתן לכתוב בצורה איטרטיבית, נציב את הביטוי הזה בתוך עצמו:
$$T=V+V\,G_{0}\Big(V+V\,G_{0}\,T\Big)=V+V\,G_{0}\,V+V\,G_{0}\,V\,G_{0}\,T.$$
ניתן להמשיך כך הלאה ולקבל:
$$T=V+V\,G_{0}\,V+V\,G_{0}\,V\,G_{0}\,V+V\,G_{0}\,V\,G_{0}\,V\,G_{0}\,V+\cdots=\sum_{n=1}^{\infty}V(G_{0}V)^{n-1}$$

\end{proof}
\begin{definition}[קירוב בורן מסדר ראשון]
לקחת את האיבר הראשון בטור של ה-\(T\). כלומר לדרוש \(T=V\) או באופן שקול \(\ket{k}=\ket{\psi_{0}}\). 

\end{definition}
\begin{proposition}[קירוב בורן מסדר ראשון של אמפליטודת הפיזור]
כאשר הפוטנציאל פיזור נמוך ביחס לאנרגיה הקינטית ניתן לכתוב:
$$f\left( \mathbf{k},\mathbf{k'} \right)\approx-\frac{m}{2\pi\,\hbar^{2}}\int e^{i\,\left( {\bf k^{\prime}-k} \right)\cdot{\bf r_{0}}}V\left( {\bf r_{0}} \right)\,\mathrm{d}^3{\bf r_{0}}$$
כלומר אמפליטודת פיזור תהיה התמרת פורייה של \(V\left( \mathbf{r} \right)\) ביחס למשתנה \(\mathbf{k'-k}\).

\end{proposition}
\begin{proof}
נשתמש בזה שמתקיים:
$$\psi(\mathbf{r}_{0})\approx\psi_{0}(\mathbf{r}_{0})=A e^{i k z_{0}}=A e^{i\mathbf{k^{\prime}}\cdot\mathbf{r}_{0}}$$
כאשר \(\mathbf{k}'\equiv k\hat{z}\). 

\end{proof}
\begin{corollary}
עבור אנרגיות פיזור נמוכות במיוחד, האקספוננט הוא בקירוב 1 ולכן:
$$f(\theta,\phi)\approx-\frac{m}{2\pi\hbar^{2}}\int\,V({\bf r})\,\mathrm{d}^3{\bf r}$$

\end{corollary}
\begin{example}
עבור הפוטנציאל
$$V({\bf r})=\left\{\begin{array}{l l}{{V_{0}}}&{{r\leq a}}\\ {{0}}&{{r>a}}\end{array}\right.$$
נקבל עבור אמפליטודת הפיזור אינטגרל על קבוע בתחום סופי ולכן:
$$f(\theta,\phi)\approx-\frac{m}{2\pi\hbar^{2}}\,V_{0}\left(\frac{4}{3}\pi a^{3}\right)$$
כלומר החתך פעולה יהיה:
$$\frac{d\sigma}{d\Omega}=|f|^{2}\approx\left(\frac{2m V_{0}a^{3}}{3\hbar^{2}}\right)^{2}$$
ולכן החתך פעולה המלא יהיה:
$$\sigma\approx4\pi\left(\frac{2m V_{0}a^{3}}{3\hbar^{2}}\right)^{2}$$

\end{example}
\begin{corollary}
עבור פוטנציאל עם סימטריה ספרית מתקיים:
$$f\left( \theta \right)\approx-\frac{2m}{\hbar^{2}\kappa}\int_{0}^{\infty}r\,V(r)\sin\left( \boldsymbol\kappa r \right)\,\mathrm{d} r$$
כאשר \({} \boldsymbol \kappa = \mathbf{k'}-\mathbf{k}\).

\end{corollary}
\begin{proof}
מתקיים:
$$\left(\mathbf{k}^{\prime}-\mathbf{k}\right)\cdot\mathbf{r}_{0}=\kappa r_{0}\cos\theta_{0}$$
ולכן:
$$f(\theta)\approx-\frac{m}{2\pi\,\hbar^{2}}\int e^{i K r_{0}\cos\theta_{0}}\,V(r_{0})r_{0}^{2}\sin\theta_{0}\,d r_{0}\,d\theta_{0}\,d\phi_{0}$$
כאשר האינטגרל לפי \(\phi_{0}\) מביא \(2\pi\), האינטגרל לפי \(\theta_{0}\) מביא 1 ולכן נקבל:
$$f(\theta)\approx-\frac{2m}{\hbar^{2}\kappa}\int_{0}^{\infty}r\,V(r)\sin(\boldsymbol\kappa r)\,d r$$

\end{proof}
\begin{example}[פוטנציאל יוקאווה]
נתון פוטנציאל מהצורה:
$$V(r)=\beta{\frac{e^{-\mu r}}{r}}$$
כאשר קירוב בוהר נותן:
$$f\left( \theta \right)\approx-\frac{2m\beta}{\hbar^{2}\kappa}\int_{0}^{\infty}e^{-\mu r}\sin\left( \kappa r \right)\,\mathrm{d} r=-\frac{2m\beta}{\hbar^{2}\left(\mu^{2}+\kappa^{2}\right)}$$
כאשר כעת החתך פעולה מקיים:
$$\frac{d\sigma}{d\Omega}=\left(\frac{2m\beta}{\hbar^{2}}\frac{1}{\mu^{2}+q^{2}}\right)^{2}$$

\end{example}
\begin{proposition}[טור בוהר של הפונקציות גל]
$$\psi=\psi_{0}+\int g V\psi_{0}+\iint g V g V\psi_{0}+\iiint g V g V g V\psi_{0}+\cdots$$

\end{proposition}
\begin{proof}
אנו יודעים כי הצורה האינטגרלית של משוואת שרדינגר היא:
$$\psi\left(\mathbf{r}\right)=\psi_{0}\left( \mathbf{r} \right)+\int g\left( \mathbf{r}-\mathbf{r}_{0} \right)\,V\left( \mathbf{r}_{0} \right)\psi\left(\mathbf{r}_{0}\right)\mathrm{d}^{3}\mathbf{r}_{0}$$
כאשר \(g\) היא הפונקצית גרין, אשר נתונה על ידי(נכניס לה קבועים מטעמי נוחות):
$$g({\bf r})\equiv-\frac{m}{2\pi\hbar^{2}}\frac{e^{i k r}}{r}$$
כלומר ניתן לכתוב:
$$\psi=\psi_{0}+\int g\,V\,\psi$$
כאשר ניתן להציב את הפתרון על ידי פונקציית גרין עבור ה-\(\psi\) באינטגרל ולקבל:
$$\psi=\psi_{0}+\int g V\psi_{0}+\iint g V g V\psi$$
וניתן להמשיך ככה הלאה ולקבל את הטור.

\end{proof}
\begin{definition}[קירוב בורן מסדר שני]
כאשר לוקחים את האיבר השני מבטור עבור \(T\). כלומר כאשר מתקיים:
$$T=V+V\frac{1}{E-H_{0}+i\varepsilon}V$$
כך שנוכל להגיד בקירוב:
$$f({\bf k^{\prime},k})\approx f^{(1)}({\bf k^{\prime},k})+f^{(2)}({\bf k^{\prime},k})$$

\end{definition}
\begin{proposition}
הרכיב השני של הקירוב נתון על ידי:
$$f^{(2)}=-\frac{1}{4\pi}\frac{2m}{\hbar^{2}}\int \mathrm{d}^3x^{\prime}\int \mathrm{d}^3x^{\prime\prime}e^{-i{\bf k^{\prime}}\cdot{\bf x^{\prime}}}V({\bf x^{\prime}})\times\left[\frac{2m}{\hbar^{2}}G_{+}({\bf x^{\prime},x^{\prime\prime}})\right]V({\bf x^{\prime\prime}})e^{i{\bf k\cdot x^{\prime\prime}}}$$

\end{proposition}
\begin{proposition}[פיזור ממספר חלקיקים]
נניח כי פיזור מחלקיק נתון על ידי פוטנציאל \(V_{0}\). כעת נניח כי יש \(N\) חלקיקים זהים הממוקמים ב-\(\vec{x}_{1},\vec{x}_{2},\dots,\vec{x}_{N}\). אם אמפליטודת הפיזור של חלקיק יחיד יהיה \(f_{0}\left( \mathbf{q} \right)\) אזי אמפליטודת הפיזור הכוללת תהיה:
$$f(\mathbf{q})=f_{0}(\mathbf{q})\sum_{i=1}^{N}e^{i\mathbf{q}\cdot\mathbf{x}_{i}}$$
כך שחתך הפעולה הכולל יהיה:
$${\frac{d\sigma}{d\Omega}}=|f(\mathbf{q})|^{2}=|f_{0}(\mathbf{q})|^{2}\left|\sum_{i=1}^{N}e^{i\mathbf{q}\cdot\mathbf{x}_{i}}\right|^{2}$$

\end{proposition}
\begin{proof}
הפוטנציאל פיזור נתון כעת על ידי:
$$V\left( \vec{r} \right)=\sum_{i=1}^{N}V_{0}\left( \vec{r}-\vec{x}_{i} \right)$$
ולכן קירוב בורן של הפיזור נתון על ידי:
$$f\left( \vec{q} \right)=-\frac{m}{2\pi \hbar^{2}}\int e^{-i\vec{q}\cdot \vec{r}}\left( \sum_{i=1}^{N}V_{0}\left( \vec{r}-\vec{x}_{i} \right) \right) \;\mathrm{d} ^{3}r$$
כאשר ניתן להחליף את הסכום ואינטגרל:
$$f(\mathbf{q})=\sum_{i=1}^{N}\left(-{\frac{m}{2\pi\hbar^{2}}}\int d^{3}r\,e^{-i\mathbf{q}\cdot\mathbf{r}}V_{0}(\mathbf{r}-\mathbf{x}_{i})\right)$$
נבצע חילוף משתנים \(\vec{r}'=\vec{r}-\vec{x}_{i}\) כך ש-\(\vec{r}=\vec{r}+\vec{x}_{i}\) ו-\(d^{3}{r}=d^{3}r'\). נקבל עבור הגורם בסוגריים:
$$-{\frac{m}{2\pi\hbar^{2}}}\int d^{3}r^{\prime}\,e^{-i{\bf q}\cdot({\bf r}^{\prime}+{\bf x}_{i})}{ V}_{0}({\bf r}^{\prime})=-{\frac{m}{2\pi\hbar^{2}}}\int d^{3}r^{\prime}\,e^{-i{\bf q}\cdot{\bf r}^{\prime}}e^{-i{\bf q}\cdot{\bf x}_{i}}V_{0}({\bf r}^{\prime})$$
ניתן להוציא את הגורם \(e^{ -\vec{q}\cdot \vec{x}_{i} }\) ולקבל:
$$e^{-i{\bf q}\cdot{\bf x}_{i}}\left(-{\frac{m}{2\pi\hbar^{2}}}\int d^{3}r^{\prime}\,e^{-i{\bf q}\cdot{\bf r^{\prime}}}V_{0}({\bf r^{\prime}})\right)$$
כאשר נזהה:
$$f_{0}({\bf q})=-{\frac{m}{2\pi\hbar^{2}}}\int d^{3}r^{\prime}\,e^{-i{\bf q}\cdot{\bf r^{\prime}}}V_{0}({\bf r^{\prime}})$$
ונקבל:
$$f(\mathbf{q})=\sum_{i=1}^{N}e^{-i\mathbf{q}\cdot\mathbf{x}_{i}}f_{0}(\mathbf{q})=f_{0}(\mathbf{q})\sum_{i=1}^{N}e^{-i\mathbf{q}\cdot\mathbf{x}_{i}}$$

\end{proof}
\begin{summary}
  \begin{itemize}
    \item ניתן לפתח את אופרטור הפיזור \(T\) בטור על ידי:
$$T=V+V\,\frac{1}{E-H_{0}+i\varepsilon}\,V+V\,\frac{1}{E-H_{0}+i\varepsilon}\,V\,\frac{1}{E-H_{0}+i\varepsilon}\,V+\cdots$$
    \item עבור קירוב בורן מסדר ראשון נקבל כי אמפליטדת הפיזור נתון על ידי התמרת פורייה:
$$f\left( {\bf k},{\bf k^{\prime}} \right)\approx-\frac{m}{2\pi\hbar^{2}}\int e^{i\left( {\bf k^{\prime}}-{\bf k} \right)\cdot{\bf r}}\,V\left( {\bf r} \right)\,\mathrm{d}^{3}{\bf r}$$
    \item תחת סימטריה ספרית ניתן לכתוב:
$$f \left( \mathbf{k,k'} \right)\approx -\frac{m}{2\pi \hbar^{2}}\int r V\left( \mathbf{r} \right)\sin\left( \boldsymbol\kappa r \right)\;\mathrm{d} r $$
    \item קירוב בורן מסדר ראשון מתקשר לאופרטור פיזור \(T\) באופן הבא:
$$f({\bf k},{\bf k^{\prime}})=-\frac{m}{2\pi\hbar^{2}}(2\pi\hbar)^{3}\langle{\bf k^{\prime}}|T|{\bf k}\rangle$$
  \end{itemize}
\end{summary}
\section{גלים חלקיים}

נתעסק כעת בשיטה לפתור את משוואת שרודינגר עבור פיזור עם סימטריה ספרית תוך שימוש בשימור תנע זוויתי. משומש למשל כאשר יש לנו פוטנציאל מרכזי ובפרט תנע זוויתי נשמר. 

\begin{reminder}
ניתן לכתוב את הגל היוצא של פיזור עבור \(r\) גדול על ידי סופרפוזיציה של הגל המישורי הנכנס והגל הכדורי המתפזר:
$$\psi(r,\theta)\approx A\left(e^{i k z}+f(\theta)\frac{e^{i k r}}{r}\right)$$
כאשר \(f\left( \theta \right)\) נקרא אמפליטודת הפיזור.

\end{reminder}
\begin{proposition}
ניתן לכתוב את הפונקציית גל הכדורית היוצאת על ידי:
$$\psi(\mathbf{r})=\sum_{l=0}^{\infty}{\frac{u_{l}(r)}{r}}P_{l}(\cos\theta)$$
כאשר ניתן לכתוב את הפונקציית גל המישורי הנכנס על ידי:
$$e^{i k z}=\sum_{l=0}^{\infty}(2l+1)i^{l}j_{l}(k r)P_{l}(\cos\theta)$$
כאשר \(j_{l}(kr)\) הם פונקציות בסל ספריות.

\end{proposition}
\begin{proposition}
עבור פוטנציאל \(V(r)\) עם סימטריה ספרית החלק הרדיאלי מקיים:
$$\frac{d^{2}u_{l}(r)}{d r^{2}}+\left[k^{2}-\frac{l(l+1)}{r^{2}}-\frac{2m}{\hbar^{2}}V(r)\right]u_{l}(r)=0,$$
כאשר \(u_{l}=rR_{l}(r)\) ו-\(R(r)\) הוא הרכיב הרדיאלי.

\end{proposition}
\begin{remark}
למעשה הפכנו משוואה התלת מימדית לאוסף של משוואות חד מימדיות(אחד לכל \(l\)). השימור של תנע זוויתי למעשה אומר כי כל \(l\) משפיע באופן בלתי תלוי.

\end{remark}
\begin{proposition}
במרחקים גדולים מהפיזור(\(r\to \infty,V(r)\to 0\)) נקבל צירוף לינארי של הפונקציות בסל:
$$R_{l}(r)\sim A_{l}j_{l}(k r)+B_{l}n_{l}(k r)$$

\end{proposition}
\begin{lemma}
כאשר \(kr\to \infty\) מתקיים:
$$j_{l}(k r)\sim\frac{1}{k r}\sin\left(k r-\frac{l\pi}{2}\right),\quad n_{l}(k r)\sim-\frac{1}{k r}\cos\left(k r-\frac{l\pi}{2}\right)$$

\end{lemma}
\begin{proposition}
$$R_{l}(r)\sim\frac{1}{k r}\left[A_{l}\sin\left(k r-\frac{l\pi}{2}\right)-B_{l}\cos\left(k r-\frac{l\pi}{2}\right)\right]$$

\end{proposition}
\begin{corollary}
ניתן לכתוב את האת החלק הרדיאלי בעזרת הפרש פאזה \(\delta_{l}\):
$$R_{l}(r)\sim\frac{1}{k r}\sin\left(k r-\frac{l\pi}{2}+\delta_{l}\right) \qquad \tan\delta_{l}=\frac{B_{l}}{A_{l}}$$

\end{corollary}
\begin{remark}
ההפרש פאזה אומר לנו כמה הפוטנציאל מקדם או מעקב את הגל לעומת גל מישורי.

\end{remark}
\begin{proposition}[פיזור בגלים חלקיים]
אמפליטודת הפיזור נתונה על ידי:
$$f(\theta)=\frac{1}{k}\sum_{l=0}^{\infty}(2l+1)\,e^{i\delta_{l}}\sin\delta_{l}\,P_{l}(\cos\theta)$$
או באופן שקול אם נגדיר:
$$f_{l}(k)=\frac{e^{2i\delta_{l}}-1}{2i k}$$
ניתן לכתוב:
$$f(\theta)=\sum_{l=0}^{\infty}(2l+1)\,f_{l}(k)\,P_{l}(\cos\theta)$$

\end{proposition}
\begin{proposition}
החתך פעולה בסימטריה כדורית תקיים:
$$\sigma_{\mathrm{tot}}=\frac{4\pi}{k^{2}}\sum_{l=0}^{\infty}(2l+1)\sin^{2}\delta_{l}$$

\end{proposition}
\begin{proof}
נזכור כי:
$$\frac{d\sigma}{d\Omega}=|f(\theta)|^{2}.$$
ולכן:
$$\frac{d\sigma}{d\Omega}=\frac{1}{k^{2}}\left|\sum_{l=0}^{\infty}(2l+1)e^{i\delta_{l}}\sin\delta_{l}\,P_{l}(\cos\theta)\right|^{2}$$
וכעת:
$$\sigma_{\mathrm{tot}}=\int d\Omega\,\frac{d\sigma}{d\Omega}=\frac{1}{k^{2}}\int d\Omega\left|\sum_{l=0}^{\infty}(2l+1)e^{i\delta_{l}}\sin\delta_{l}\,P_{l}(\cos\theta)\right|^{2}$$
כאשר הגרום היחיד שתלוי בזווית הוא הפולינום לג'נדר. ונשתמש ביחס:
$$\int d\Omega\,P_{l}(\cos\theta)P_{l^{\prime}}(\cos\theta)=\delta_{l l^{\prime}}\,\frac{4\pi}{2l+1}$$
ולכן למעשה האינטגרל רק נותן פקטור \(\frac{4\pi}{2l+1}\) ונקבל:
$$\sigma_{\mathrm{tot}}=\frac{1}{k^{2}}\sum_{l=0}^{\infty}(2l+1)^{2}\sin^{2}\delta_{l}\,\frac{4\pi}{2l+1}=\frac{4\pi}{k^{2}}\sum_{l=0}^{\infty}(2l+1)\sin^{2}\delta_{l}$$

\end{proof}
\begin{example}[הוכחה של המשפט האופטי במקרה של סימטריה ספרית]
נראה כי מתקיים המשפט האופטי:
$$\sigma_{\mathrm{tot}}={\frac{4\pi}{k}}\,\mathrm{Im}\ f(0)$$
אנו יודעים כי עבור פוטנציאל מרכזי מתקיים:
$$f(\theta)=\frac{1}{k}\sum_{l=0}^{\infty}(2l+1)e^{i\delta_{l}}\sin\delta_{l}\,P_{l}(\cos\theta)$$
עבור \(\theta=0\) נקבל \(P_{l}\left( \cos \theta \right)=1\) ולכן:
$$f(0)=\frac{1}{k}\sum_{l=0}^{\infty}(2l+1)e^{i\delta_{l}}\sin\delta_{l}=\sum_{l=0}^{\infty}(2l+1)f_{l}$$
כאשר:
$$f_{l}=\frac{e^{i\delta_{l}}\sin\delta_{l}}{k}=\frac{\sin \delta_{l}\cos \delta_{l}+i\sin ^{2}\delta_{l}}{k}\implies\mathrm{Im}\;f_{l}=\frac{\sin^{2}\delta_{l}}{k}$$
ולכן:
$$\mathrm{Im}\ f(0)=\sum_{l=0}^{\infty}(2l+1)\,\mathrm{Im}\ f_{l}=\frac{1}{k}\sum_{l=0}^{\infty}(2l+1)\sin^{2}\delta_{l}$$
כאשר החתך פעולה מקיים:
$$\sigma_{\mathrm{tot}}=\int{\frac{\mathrm{d}\sigma}{\mathrm{d}\Omega}}\,\mathrm{d}\Omega={\frac{4\pi}{k^{2}}}\sum_{l=0}^{\infty}(2l+1)\sin^{2}\delta_{l}$$
ונשים לב כי עד כדי פקטור של \(\frac{4\pi}{k}\) נקבל את החלק המדומה של \(f(0)\)! לכן:
$$\sigma_{\mathrm{tot}}=\frac{4\pi}{k}\,\mathrm{Im}\ f(0)$$

\end{example}
\begin{summary}
  \begin{itemize}
    \item שיטת הגלים החלקים זה שיטה לפתור את משוואת שרדינגר בסימטריה ספרית תוך שימוש בשימור תנע זוויתי.
    \item ניתן לכתוב את הגלים היוצאים על ידי:
$$\psi(\mathbf{r})=\sum_{l=0}^{\infty}{\frac{u_{l}(r)}{r}}P_{l}(\cos\theta),$$
כאשר ניתן לכתוב את הגלים הנכנסים בעזרת פונקציות בסל ספריות:
$$e^{i k z}=\sum_{l=0}^{\infty}(2l+1)i^{l}j_{l}(k r)P_{l}(\cos\theta),$$
    \item המשוואה הרדיאלית עבור \(u_{l}(r)=rR_{l}(r)\) נתונה על ידי:
$$\frac{d^{2}u_{l}(r)}{d r^{2}}+\left[k^{2}-\frac{l(l+1)}{r^{2}}-\frac{2m}{\hbar^{2}}V(r)\right]u_{l}(r)=0$$
    \item בגבול האסימפטוטי הפתרון הרדיאלי יהיה \(R_{l}(r)\sim A_{l}j_{l}(k r)+B_{l}n_{l}(k r)\) כאשר ניתן לקרב ולקבל:
$$R_{l}(r)\sim\frac{1}{k r}\sin\left(k r-\frac{l\pi}{2}+\delta_{l}\right),\quad\tan\delta_{l}=\frac{B_{l}}{A_{l}}.$$
    \item אמפליטודת הפיזור נתונה על ידי:
$$f(\theta)=\frac{1}{k}\sum_{l=0}^{\infty}(2l+1)e^{i\delta_{l}}\sin\delta_{l}P_{l}(\cos\theta)$$
כאשר חתך החתך הכולל יהיה:
$$\sigma_{\mathrm{tot}}=\frac{4\pi}{k^{2}}\sum_{l=0}^{\infty}(2l+1)\sin^{2}\delta_{l}.$$
  \end{itemize}
\end{summary}
\chapter{שאלות}

\section{הגדרות - יסודות הקוונטים}

?
נקרא לאופרטור של האנרגיה ההמילטוניאן, ונסמנו ב-\(H\).

מה ההגדרה של תמונת הייזנברג?
?
האופרטורים משתנים בזמן והמצבים קוונטים קבועים בזמן.

מה ההגדרה של תמונת שרדינגר?
?
האופרטורים קבועים בזמן והמצבים הקוונטים משתנים בזמן.

מה ההגדרה של גודל פיזיקלי?
?
נקרא לגדלים כמו, תנע מהירות, מיקום, אנרגיה וספין גדלים פיזיקליים.

מה ההגדרה של מצב קוונטי?
?
מצב קוונטי זה אובייקט מתמטי שמתאר את הידע שלנו על מערכת קוונטית. ממנו ניתן לחלץ את המידע על הגדלים הפיזיקלים בכל רגע של זמן.

מה אומר משפט אופי המצב הקוונטי?
?
מצב קוונטי הוא ווקטור מנורמל במרחב הילברט.

מה אומר משפט אופי הגדלים הפיזיקלים?
?
גדלים פיזיקלים הם אופרטורים הרמיטים.

מה אומר משפט מדידה של גודל פיזיקלי?
?
מדידה של גודל פיזיקלי של מצב קוונטי שקולה להפעלה של אחד מהאופרטורי ההטלה על מצביו העצמיים של הגודל הפיזיקלי.

מה ההגדרה של אי וודאות של אופרטור?
?
$$\triangle A = \sqrt{\left\langle\psi\right|\left(A-\left\langle A\right\rangle\right)^{2}\left|\psi\right\rangle}$$

מה אומר משפט אי הוודאות?
?
$$\triangle A \triangle B \geq \frac{1}{2}\left\lvert  \langle [A,B] \rangle   \right\rvert $$

מה אומר משפט מדידה?
?
לאחר מדידה, הגודל הפיזיקלי קורס לאחד המצבים העצמיים שלו בצורה לא דיטרמניסטית ולא הפיכה, כאשר ההסתברות להיות במצב עצמי כלשהו נקבע ע"י כלל בורן.

מה ההגדרה של קוונטיזציה קנונית ראשונה?
?
תהליך להעביר משוואות קלאסיות למשוואות גל קוונטיות. מעביר את המשתנים לאופרטורים הרמיטים הבאים:
$$x\mapsto \hat{X}\quad p\mapsto \hat{P}\quad \mathcal{H}\to \hat{H}\quad $$

מה ההגדרה של עקרון ההתאמה?
?
עקרון שקבוע שהקוונטיציה הקוונטית היא הכללה של הפיזיקה הקלאסית, ובקנה מידה גדול נצפה לקבל את המשוואות הקלסיות המוכרות.

מה ההגדרה של אופרטור המיקום?
?
המיקום הוא גודל מדיד, לכן ניתן לתאר אותו ע"י אופרטור הרמיטי \(X\). כיוון שהערכים האפשריים של \(X\) הם רציפים, הע"ע של האופרטור \(X\) הם רציפים.

מה ההגדרה של אופרטור הזזה?
?
אופרטור \(T\) המקיים:
$$T(a)\ket{x} = \ket{x+a}$$

מה ההגדרה של אופרטור התנע?
?
התנע הוא גודל מדיד לכן ניתן לתאר אותו ע"י אופרטור אוניטרי \(P\).

מה ההגדרה של אופרטור המיקום?
?
המיקום הוא גודל מדיד, לכן ניתן לתאר אותו ע"י אופרטור הרמיטי \(X\). כיוון שהערכים האפשריים של \(X\) הם רציפים, הע"ע של האופרטור \(X\) הם רציפים.

מה ההגדרה של אופרטור הזזה במיקום?
?
אופרטור \(T\) המקיים:
$$T(a)\ket{x} = \ket{x+a}$$

מה ההגדרה של אופרטור התנע?
?
התנע הוא גודל מדיד לכן ניתן לתאר אותו ע"י אופרטור הרמיטי \(P\).

\section{הגדרות - ספין}

?
יהי \(J_{1},J_{2}\) אופרטורי תנע זוויתי כך ש-\(J_{1}\otimes \mathbb{1} +\mathbb{1} \otimes J_{2}\equiv J_{3}\). הבסיס הלא מצומד של \(J_{3}\) יהיה:
$$\left\{  \ket{j_{1},m_{1}} \otimes \ket{j_{2},m_{2}}   \right\}\equiv \left\{  \ket{j_{1}j_{2};m_{1}m_{2}}   \right\}$$
כאשר \(\ket{j_{1},m_{1}}\) ו-\(\ket{j_{2},m_{2}}\) הם המצבים העצמים של \(J_{1}^{2}\) ו-\(J_{2}^{2}\) בהתאמה. לעיתים נקרא "בסיס המכפלה".

מה ההגדרה של בסיס מצומד - coupled basis?
?
יהי \(J_{1},J_{2}\) אופרטורי תנע זוויתי כך ש-\(J_{1}\otimes \mathbb{1} +\mathbb{1} \otimes J_{2}\equiv J_{3}\). הבסיס המצומד של \(J_{3}\) יהיה:
$$\left\{  \ket{JM}   \right\}$$
כאשר הרבה פעמים כותבים:
$$\left\{  \ket{j_{1},j_{2};JM}   \right\}$$
כדי שיהיה אפשר לעבור בצורה הפיכה בין הבסיס הלא מצומד לבסיס המצומד.

מה ההגדרה של מקדמי קלבש גורדון?
?
המקדמים של המעבר בסיס בין הבסיס המצומד לבסיס הלא מצומד. כלומר:
$$\langle j_{1}j_{2};m_{1}m_{2}|j_{1}j_{2}j m\rangle$$
כאשר כיוון שאנחנו מגדירים אותם כך שממשיים מתקיים:
$$\langle j_{1}j_{2};m_{1}m_{2}|j_{1}j_{2}j m\rangle=\langle j_{1}j_{2}j m|j_{1}j_{2};m_{1}m_{2}\rangle$$

מה ההגדרה של אופרטור סקלארי?
?
אופרטור שלא משתנה תחת סיבובים. זה שקול ללהגיד כי האופרטור \(V\)  מתחלף עם התנע הזוויתי \([L_{i},V]=0\).

מה ההגדרה של אופרטורים ווקטורים?
?
קבוצה של שלושה אופרטורים \(V_{1},V_{2},V_{3}\) אשר מקיימות:
$$T(R)V_{i}T^{\dagger}(R)=\sum_{j=1}^{3}R_{j i}V_{j}$$
נקראות שלושת אופרטורי ווקטורים, כאשר \(T(R)=\exp\left( -i\sum_{i=1}^{3}\frac{\beta_{i}J_{i}}{\hbar} \right)\) זה הייצוג האונטירי של אופרטור הסיבוב. כלומר זהו אופרטור שתחת סיבובים מתנהגים באותו דרך כמו ווקטור מיקום.

מה ההגדרה של טנזורים קרטזים?
?
זוהי הכללה של אופרטורים ווקטורים. טנזור מסדר \(n\) יהיה אובייקט עם \(n\) אינדקסים אשר עובר טרנספורמציית סיבוב בצורה הבאה:
$$T_{ij \dots n}\mapsto R_{ii'}R_{jj'} \dots R_{nn'} T_{i' j' \dots n'}$$
כאשר \(R\) הם טרנספורמציית סיבוב.

מה ההגדרה של אופרטורים ספרים?
?
אופרטור \(T_{q}^{(k)}\) אשר עוברים טרנספורמציה תחת סיבוב בצורה הבאה:
$$T(R)T_{q}^{(k)}T^{\dagger}(R) =\sum_{q'=-k}^{k}D^{(k)}_{qq'}(R)T_{q'}^{(k)}$$
כאשר \(T(R)\) זה ההצגה האוניטרית של אופרטור הסיבוב \(R\) ו-\(D_{q'q}^{(k)}\) אלמנטי מטריצת ווגינר. זהו למעשה אוסף של \(2k+1\) אופרטורים.

מה אומר משפט ווגנר-אקרט?
?
יהי \(T_{q}^{(k)}\) טנזור ספרי. אלמנטי המטריצה של הטנזור ביחס לבסיס התנע יקיים:
$$\langle\alpha^{\prime},j^{\prime}m^{\prime}|T_{q}^{(k)}|\alpha,j m\rangle=\langle j k;m q|j k;j^{\prime}m^{\prime}\rangle\frac{\langle\alpha^{\prime},j^{\prime}||T^{(k)}||\alpha,j\rangle}{\sqrt{2j+1}}$$
כאשר הרכיב \(\langle\alpha^{\prime},j^{\prime}||T^{(k)}||\alpha,j\rangle\) נקרא אלמנטי המטריצה המצומצמת וזה סימון שמדגיש את זה שלא תלוי ביתר הגורמים(\(m,m',q\)).

מה ההגדרה של סימון 3j?
?
$$\begin{pmatrix}j_{1} & j_{2} & J \\m_{1} & m_{2} & M 
\end{pmatrix}= \frac{(-1)^{j_{1}-j_{2}-m_{2}}}{\sqrt{ 2J+1 }}\left\langle  j_{1}m_{1}j_{2}m_{2}\mid J,-M  \right\rangle \in \mathbb{R}$$

\section{הגדרות - פתרונות מיחדות של שרדינגר}

?
פוטנציאל מהצורה:
$$V(x)={\frac{1}{2}}m\omega^{2}X^{2}$$

מה ההגדרה של אוסצילטור הרמוני קוונטי?
?
מערכת שבה יש פוטנציאל הרמוני, כלומר מערכת שההמילטוניאן שלה יהיה:
$$H={\frac{p^{2}}{2m}}+{\frac{m\omega^{2}x^{2}}{2}}$$

מה ההגדרה של אופרטורי העלה והורדה?
?
$$a=\left( \hat{X}+i\hat{P} \right)=\sqrt{\frac{m\omega}{2\hbar}}\left(x+\frac{i p}{m\omega}\right)\quad a^{\dagger}=\left( \hat{X}-i\hat{P} \right)=\sqrt{\frac{m\omega}{2\hbar}}\left(x-\frac{i p}{m\omega}\right)$$
לעיתים אופרטור הורדה \(a\) נקרא אופרטור השמדה ואופרטור העלה \(a^{\dagger}\) נקרא אופרטור יצירה.

מה ההגדרה של אופרטור המספר?
?
$$N=a^{\dagger}a=\left(\frac{m\omega}{2\hbar}\right)\left(x^{2}+\frac{p^{2}}{m^{2}\omega^{2}}\right)+\left(\frac{i}{2\hbar}\right)[x,p]=\frac{H}{\hbar \omega}-\frac{1}{2}$$

מה ההגדרה של מצב קוהרנטי?
?
זהו מצב קוונטי \(\ket{\alpha}\) שמוגדר בתור המצב עצמי הייחודי של אופרטור ההשמדה \(a\). נשים לב כי כיוון ש-\(a\) לא הרמיטי, המצב הקוהרנטי ולא בהכרח ממשי, וקיים לו אמפליטודה ופאזה מרוכבת.

מה ההגדרה של חלקיק חופשי?
?
חלקיק שבו הפוטנציאל הוא אפס. כלומר ההמילטוניאן מקיים:
$$\mathcal{H}= \frac{P^{2}}{2m}$$

מה ההגדרה של פוטנציאל דלתא?
?
עבור קבוע ממשי \(\alpha\) פוטנציאל דלתא יהיה פוטנציאל מהצורה:
$$V(x)=-\alpha \delta(x)$$
ולכן משוואת שרדינגר תהיה
$$-\;\frac{\hbar^{2}}{2m}\frac{d^{2}\psi}{d x^{2}}-\alpha\delta(x)\;\psi=E\psi$$

מה ההגדרה של פוטנציאל מדרגה?
?
פוטנציאל מהצורה:
$$V(x)=\left\{{0\quad x\leq0}\atop{U\quad x>0}\right.$$
כאשר \(U>0\) זה קבוע.

מה ההגדרה של נרמול?
?
אנו דורשים שסך ההסתברות שחלקיק נמצא איפשהו יהיה 1:
$$\int_{-\infty}^{+\infty}|\Psi(x,t)|^{2}\;d x=1$$

מה ההגדרה של מצב קשור?
?
כאשר האנרגיה של החלקיק קטן מהפוטנציאל באיסוף, כלומר כאשר \(E<V\left( -\infty \right),V\left( \infty \right)\). במצב זה החלקיק תקוע בתוך איזור ולא יכול להתפזר לאינסוף.

מה ההגדרה של מצב חופשי/מצב פיזור?
?
כאשר האנרגיה של החלקיק גדולה מהוטנציאל באינסוף, כלומר \(E>V\left( -\infty \right),V\left( \infty \right)\). במקרה זה נקבל כי תמיד קיים הסתברות שהחלקיק יברח לאינסוף.

מה ההגדרה של צפיפות הסתברות?
?
נסמן את צפיפות ההסתברות ב-\(x\) בזמן \(t\) ע"י:
$$\rho\left(x,t\right)\equiv\left|\psi\left(x,t\right)\right|^{2}$$

מה ההגדרה של זרם הסתברות?
?
$$j\left(x\right)=\frac\hbar{2i m}\left(\overline{{{\psi}}}\frac{\partial\psi}{\partial x}-\psi\frac{\partial\overline{{{\psi}}}}{\partial x}\right)=\frac\hbar m\mathrm{Im}\left(\overline{{{\psi}}}\frac{\partial\psi}{\partial x}\right)$$

מה ההגדרה של בור בפוטנציאל אינסופי?
?
מערכת עם פוטנציאל מהצורה:
$$V(x)=\left\{{\begin{array}{l l}{{0}}&{{{0\leq x\leq a}}}\\ {{\infty,}}&{{\mathrm{otherwise}}}\end{array}}\right.$$

מה ההגדרה של בור בפוטנציאל סופי?
?
מערכת עם פוטנציאל מהצורה:
$$V(x)=\left\{\begin{array}{l l}{{-V_{0},}}&{{-a\le x\le a,}}\\ {{0,}}&{{\vert x\vert>a,}}\end{array}\right.$$
כאשר \(V_{0}\) קבוע חיובי.

\section{הגדרות - תנע זוויתי}

?
האטום הפשוט ביותר. מורכב מגרעין טעון חיובית עם מסת מנוחה
$$m_{p}\,=\,1.6726231\times10^{-27}\,\mathrm{kg},$$
וכן אלקטרון טעון שלילית עם מסה:
$$m_{e}\;=\;9.1093897\,\times\,10^{-31}\,\mathrm{kg}.$$

מה ההגדרה של מכפלה פנמימת כדורית?
?
$$\langle f|g\rangle=\int d\Omega\,f(\theta,\phi)^{*}g(\theta,\phi)$$
כאשר \(d\Omega=\sin \theta d\theta d\phi\).

מה ההגדרה של הבסיס הקרטזי?
?
נגדיר משתנים חדשים \(J_{x},J_{y},J_{z}\) באופן הבא:
$$J_{x}=J_{1} \qquad J_{y}=J_{2} \qquad  J_{z}=J_{3}$$

מה ההגדרה של הבסיס הפולארי?
?
נגדיר משתנים חדשים \(J_{0},J_{+},J_{-}\) באופן הבא:
$$J_{\pm}=J_{x}\pm i J_{y}\qquad J_{0}=J_{z}$$
כאשר האופרטורים \(J_{\pm}\) נקראים אופרטורי סולם, ואינם הרמיטים אבל כן מתקיים \(J_{\pm}^{\dagger}=J_{\mp}\).

מה ההגדרה של תנע זוויתי אורביטלי?
?
היוצר של שינוי בזווית. כלומר עבור סיבוב כללי סביב ציר \(\hat{n}\) בזווית \(\theta\) המוגדר על ידי:
$$R(\mathbf{\hat{n}},\theta)=\mathsf{I}+\theta\mathbf{\hat{n}}\cdot\mathsf{J}$$
כאשר \(\theta\ll 1\) האופרטור האוניטרי המתאים יהיה:
$$U(R)=U(\mathbf{\hat{n}},\theta)=1-{\frac{i}{\hbar}}\theta\mathbf{\hat{n}}\cdot\mathbf{L}$$\textbf{טענה}
האופרטור התנע הזוויתי האורביטלי מקיים:
$$\mathbf{L}=\mathbf{x}\times\mathbf{p}$$

מה ההגדרה של אופרטור התנע בריבוע?
?
$$\hat{L}^{2}=\hat{L}_{x}^{2}+\hat{L}_{y}^{2}+\hat{L}_{z}^{2}$$
לעיתים נקרא אופרטור קסימיר.

מה ההגדרה של מטריצת אלמנטי מטריצת הסיבוב?
?
עבור סיבוב \(R\) אשר מוגדר על ידי \(\hat{n}\) ו-\(\phi\) ניתן להגדיר את אלמנטי המטריצה על ידי:
$${\mathcal{D}}_{m^{\prime}m}^{(j)}(R)=\langle j,m^{\prime}|\exp\left(\frac{-i{\bf J}\!\cdot\!{\hat{\bf n}}\phi}{\hbar}\right)|j,m\rangle$$
כאשר אלמנטי המטריצה לעיתים נקראות פונקציות ויגנר.

\section{הגדרות ריבוי דרגות חופש}

?
$$\left|\vec{r}\right\rangle=\left|x,y,z\right\rangle=\left|x\right\rangle\otimes\left|y\right\rangle\otimes\left|z\right\rangle$$

מה ההגדרה של תנע בשלוש מימדים?
?
$$|{\textbf{p}}\rangle=|p_{x},p_{y},p_{z}\rangle=|p_{x}\rangle\otimes|p_{y}\rangle\otimes|p_{z}\rangle$$

מה ההגדרה של פונקציית גל תלת מימדית?
?
$$\psi\left(\vec{r}\right)=\left\langle \vec{r}|\psi \right\rangle  \qquad \psi\left( \vec{p} \right)=\left\langle  \vec{p}|\psi  \right\rangle $$

מה ההגדרה של אופרטור הצפיפות הסתברות והזרם הסתברות בתלת מימד?
?
$$P=-i\hbar \bar{\nabla} \qquad j\left( \vec{r},t \right)={\frac{\hbar}{2i m}}\left(\overline{{{\psi}}}\nabla\psi-\psi\nabla\overline{{{\psi}}}\right)={\frac{\hbar}{m}}\mathrm{Im}\left(\overline{{{\psi}}}\nabla\psi\right)$$

מה ההגדרה של מכפלה טנזורית של מרחבי הילברט?
?
מרחבי הילברט \(\mathcal{H}_{A},\mathcal{H}_{B}\) בלתי תלויות, ניתן להגדר מרחב חדש \(\mathcal{H}=\mathcal{H}_{A}\times \mathcal{H}_{B}\) כאשר מימד המרחב החדש יהיה כפל מרחבי הילברט.

מה ההגדרה של מצב מכפלה?
?
אם \(\ket{\psi}_{A}\in \mathcal{H}_{A}\) ו-\(\ket{\psi}_{B}\in \mathcal{H}_{B}\) נגדיר את המצב מכפלה שלהם בתור האיבר
$$\ket{\psi} _{A}\otimes \ket{\psi} _{B} \in \mathcal{H}_{A}\times \mathcal{H}_{B}$$
כאשר הפעולה שמחברת ביניהם נקראת המכפלה הטנזורית בין המצבים

מה ההגדרה של מכפלה פנימית של מצבי מכפלה?
?
נגדיר:
$$\left(\left\langle\psi^{\prime}\right|_{A}\otimes\left\langle\phi^{\prime}\right|_{B}\right)\left(\left|\psi\right\rangle_{A}\otimes\left|\phi\right\rangle_{B}\right)=\left\langle\psi^{\prime}|\psi\right\rangle\left\langle\phi^{\prime}|\phi\right\rangle$$

מה אומר משפט היסוד החמישי?
?
מרחב הילברט של מערכת קוונטית המוכבת שתי מערכות \(A,B\) בעלות מרחב הילברט \(\mathcal{H}_{A},\mathcal{H}_{B}\) בהתאמה נתון ע"י מרחב המכפלה הטנזורי \(\mathcal{H}=\mathcal{H}_{A}\times \mathcal{H}_{B}\).

מה ההגדרה של מכפלה טנזורית של אופרטור?
?
יהי \(O_{A}\) אופרטור שפועל על \(\mathcal{H}_{A}\) ויהי \(O_{B}\) אופרטור שפועל על \(\mathcal{H}_{B}\). נגדיר \(O=O_{A}\otimes O_{B}\) בתור האופרטור שפועל על מרחב המכפלה \(\mathcal{H}=\mathcal{H}_{A}\times \mathcal{H}_{B}\) בצורה הבאה:
$$\left({ O}_{A}\otimes{ O}_{B}\right)\left(\left|\psi\right\rangle_{A}\otimes\left|\phi\right\rangle_{B}\right)=\left({\cal O}_{A}\left|\psi\right\rangle_{A}\right)\otimes\left({\cal O}_{B}\left|\phi\right\rangle_{B}\right)$$
כלומר מצב כללי המוגדר ע"י \(\left|\Psi\right\rangle=\sum_{i=1}^{d_{A}}\sum_{j=1}^{d_{B}}\Psi_{i j}\left|e_{i}\right\rangle_{A}\otimes\left|f_{j}\right\rangle_{B}\) מקיים:
$${ O}\left|\Psi\right\rangle=\sum_{i=1}^{d_{A}}\sum_{j=1}^{d_{B}}\Psi_{i j}\left({ O}_{A}\left|e_{i}\right\rangle_{A}\right)\otimes\left({ O}_{B}\left|f_{j}\right\rangle_{B}\right)$$

מה ההגדרה של צבר של מצבים קוונטים?
?
מערכת שבה יש לנו אוסף של מצבים קוונטים כאשר יש לנו הסתברות(קלאסית) שונה להיות בכל מצב קוונטי.

מה ההגדרה של אופרטור הצפיפות?
?
עבור צבר של מצבים קוונטים \(\left\{  \ket{\psi_{\mu}}  \right\}\) כאשר ההסתברות להיות במצב קוונטי \(\ket{\psi_{\mu}}\) יהיה \(p_{\mu}\) נגדיר את מטריצת הצפיפות להיות:
$$\rho=\sum_{\mu=1}^{N}p_{\mu}\left|\psi_{\mu}\right\rangle\left\langle\psi_{\mu}\right|$$

מה ההגדרה של עקבה של אופרטור?
?
נגדיר את העקבה של אופרטור בצורה הבאה:
$${\mathrm{Tr}}\left[Q\right]=\sum_{n}\left\langle n\right|Q\left|n\right\rangle$$
כאשר נשים לב כי כאשר האופרטור הוא אופרטור שנוצר ממכפלה טנזורית העקבה תחזיר אופרטור.

מה ההגדרה של עקבה חלקית ומטריצת צפיפות מצומצמת?
?
יהי \(\mathcal{H}_{A},\mathcal{H}_{B}\) מרחבי הילברט בלתי תלויים עם מימדים \(d_{A},d_{B}\) ואיברי בסיס \(\{\left|e_{i}\right\rangle_{A}\}_{i=1}^{d_{A}},\left\{\left|f_{j}\right\rangle_{B}\right\}_{j=1}^{d_{B}}\) כך ש-\(\mathcal{H}=\mathcal{H}_{A}\times \mathcal{H}_{B}\). 
$$\rho_{A}=\sum_{j}\left\langle f_{j}\right|_{B}\rho\left|f_{j}\right\rangle_{B}\equiv\mathrm{Tr}_{B}\left[\rho\right]$$

מה ההגדרה של מצב טהור?
?
כאשר בצבר שלנו כל האובייקטים עם מצב קוונטי יחיד \(\ket{\psi}\). במקרה זה ניתן לכתוב \(\rho=\ket{\psi}\bra{\psi}\).

מה ההגדרה של מצב מכפלה?
?
יהיו \(\mathcal{H}_{A},\mathcal{H}_{B}\) מרחבים הילברט. מצב \(\ket{\Psi}\in \mathcal{H}_{A}\times \mathcal{H}_{B}\) נקרא מצב מכפלה אם קיימים \(\ket{\psi}_{A}\in A,\ket{\psi}_{B}\in B\) כך שמתקיים:
$$\ket{\Psi} =\ket{\psi} _{A}\otimes \ket{\psi} _{B}$$

מה ההגדרה של מצב שזור?
?
מצב שאינו מצב מכפלה. כלומר עבור \(\mathcal{H}_{A},\mathcal{H}_{B}\) מרחבי הילברט. מצב \(\ket{\Psi}\in \mathcal{H}_{A}\times \mathcal{H}_{B}\) נקרא מצב שזור אם לא קיימים \(\ket{\psi}_{A}\in A,\ket{\psi}_{B}\in B\) כך שמתקיים:
$$\ket{\Psi} =\ket{\psi} _{A}\otimes \ket{\psi} _{B}$$

מה ההגדרה של שזירה?
?
שזירה בין שתי מערכות אומרת לנו עד כמה אי ידיעת מצבה של מערכת אחת מונעת מאיתנו לדעת את מצבה של השנייה.

מה ההגדרה של אנטרופיית שאנון?
?
דרך לכמת את כמות האי ידיעה במערכת. אם הסתבריות נתונות ע"י \(\{ p_{i} \}\) אנטרופיית שאנון מוגדרת:
$$S=-\sum_{i}p_{i}\ln p_{i}$$

\section{יסודות הפיסיקה הקוונטית}

?

\begin{enumerate}
  \item מצב קוונטי הוא ווקטור מנורמל במרחב הילברט. 


  \item גדלים פיזיקלים מיוצגים ע"י אופרטורים לינארים הרמיטים. 


  \item לאחר מדידה, הגודל הפיזיקלי קורס לאחד המצבים העצמיים שלו בצורה לא דיטרמניסטית ולא הפיכה, כאשר ההסתברות להיות במצב עצמי כלשהו נקבע ע"י כלל בורן. 


  \item מערכת מתקדמת בזמן לפי משוואת שרדינגר. 


  \item מרחב הילברט של מערכת קוונטית המוכבת שתי מערכות \(A,B\) בעלות מרחב הילברט \(\mathcal{H}_{A},\mathcal{H}_{B}\) בהתאמה נתון ע"י מרחב המכפלה הטנזורי \(\mathcal{H}=\mathcal{H}_{A}\times \mathcal{H}_{B}\). 


\end{enumerate}
מהו אופרטור הקידום בזמן? איך מגיעים ממנו למשוואת שרדינגר?
?
זהו אופרטור אוניטרי אשר מבצע קידום בזמן. מוגדר על ידי:
$$U\left(t-t_{0}\right)=e^{-i H\left(t-t_{0}\right)/\hbar}$$
אם נגזור את שתי האגפים לפי זמן נקבל:
$$\frac{\partial}{\partial t}\left|\psi\left(t\right)\right\rangle=-\frac{i H}{\hbar}e^{-i H\left(t-t_{0}\right)/\hbar}\left|\psi\left(t_{0}\right)\right\rangle=-\frac{i H}{\hbar}\left|\psi\left(t\right)\right\rangle$$

איך מוגדר אי הוודואות של אופרטור? מה עקרון אי הוודאות ומשפט אהרנפסט אומרים על האי וודואות?
?
אי הוודאות יהיה למעשה הסטיית תקן, ומוגדר על ידי:
$$\triangle A = \sqrt{\left\langle\psi\right|\left(A-\left\langle A\right\rangle\right)^{2}\left|\psi\right\rangle}=\sqrt{ \langle A^{2} \rangle-\langle A \rangle ^{2}  }$$
עקרון אי הוודאות מתייחס למכפלה של אי וודואיות של שתי אופרטורים ואומר כי:
$$\triangle A \triangle B \geq \frac{1}{2}\left\lvert  \langle [A,B] \rangle   \right\rvert$$
כלומר יש חסם תחתון על מכפלה של האי הוודואות אשר תלוי ביחסי חילוף ביניהם.
משפט אהרנפסט מתייחס לאיך שהאי וודואות משתנה בזמן ואומר כי:
$$\frac{d\left<Q\right>}{d t}=\frac{i}{\hbar}\left<[H,Q]\right>+\left<\frac{\partial Q}{\partial t}\right>$$

איך ניתן להציג מצב תנע בבסיס המיקום? איך זה קשור להתמרת פורייה?
?
מתקיים:
$$\langle x|p\rangle=\frac{1}{\sqrt{2\pi\hbar}}e^{i p x/\hbar}$$
כאשר אם נרצה לקבל את הפונקציית גל בבסיס המיקום נדרש לבצע אינטגרל על ערך זה:
$$\psi(x)=\langle x|\psi\rangle=\int\!\!d p\;\langle x|p\rangle\langle p|\psi\rangle=\int\!\!d p\;{\frac{e^{i x p/\hbar}\tilde{\psi}(p)}{\sqrt{2\pi\hbar}}}$$
כאשר זה בדיוק התמרת פורייה עבור המשתנה \(k=\frac{p}{\hbar}\).

מהם החוקים לביצוע קוונטיזציה של מערכת קלאסית למערכת קוונטית בעזרת הקוונטיזציה הקנונית הראשונה
?

\begin{enumerate}
  \item מיקום \(x\) יהפוך לאופרטור \(X\). 


  \item התנע \(p\) יהפוך לאופרטור \(P\). 


  \item האופרטורים אינם חילופיים, ומקיימים את יחס החילוף \([X,P]=i\hbar\). 


  \item כל פונקציה שלהם תהפוך לפונקציה של האופרטור, כולל ההמילטוניאן \(H\), כאשר אם מקבלים אופרטור לא הרמיטי, מבצעים סימטריזציה באמצעות האנטי קומוטטור. 


\end{enumerate}
\section{מבוא}

?
האנרגיה של פוטון קשור ע"י:
$$E = h \nu = \hbar \omega$$
כאשר \(h\) זה קבוע פלאנק, \(\hbar=\frac{h}{2\pi}\) קבוע פלאנק מצומצם, \(\nu\) זה תדירות ו-\(\omega\) זה תדירות זוויתית.

מה אומר משפט דה ברויי?
?
גם חלקיקים חסרי מסה יכולים להתנהג כגלים, ומקיימים \(p=h\lambda=\hbar k\)

מה ההגדרה של קומוטטור של חוגים?
?
$$[A,B]=AB-BA$$

מה ההגדרה של אנטי קומוטטור?
?
$$\{A,B\}=AB+BA$$

מה ההגדרה של מרחב הילברט?
?
מרחב הילברט זה מרחב מכפלה פנימית שלם(כלומר כל סדרת קושי של איברים במרחב מתכנסת).

מה ההגדרה של בסיס של מרחב הילברט?
?
אוסף של ווקטורים אשר צירוף לינארי שלהם(לאו דווקא סופי) פורש את המרחב, וכל איבר יהיה בלתי תלוי לינארי ביתר האיברים.

מה ההגדרה של מרחב הילברט רציף?
?
מרחב הילברט כאשר הקבוצה של האיברי בסיס שלו בעוצמת הרצף.

מה ההגדרה של פונקציית הגל?
?
בהנתן בסיס \(\left\{  \ket{x}  \right\}\) ניתן להציג כל \(\ket{\psi}\) במרחב הילברט ע"י צירוף לינארי אינסופי של איברים מ-\(\left\{  \ket{x}  \right\}\). נסמן את מקדמים הנרמול של האיבר \(\ket{x}\) ב-\(\psi(x)\) ולכן מתקיים:
$$\left|\psi\right\rangle=\int d x\psi\left(x\right)\left|x\right\rangle$$

מה ההגדרה של ווקטור עצמי?
?
עבור אופרטור \(X\), ווקטור עצמי \(\ket{x}\) של \(X\) יהיה ווקטור המקיים:
$$X\left|x\right\rangle=x\left|x\right\rangle$$
עבור \(x \in \mathbb{C}\) כלשהו.

מה ההגדרה של אורתוגונאליות?
?
ווקטורים \(x,x'\) נקראים אורתוגונאלים אם:
$$\left\langle  x|x^{\prime} \right\rangle=\begin{cases}0 & \ket{x} \neq \ket{x'} \\1 & \ket{x} =\ket{x'}  
\end{cases}$$
כאשר עבור ווקטורים עצמיים אורתוגונאלים נקבל כי:
$$\braket{ x | x' } =\delta(x-x')  $$

מה ההגדרה של פונקציית גל מנורמלת?
?
$$\int d x\left|\psi\left(x\right)\right|^{2}=1$$

מה ההגדרה של ספין?
?
תכונה קוונטית חומר. ניתן למדוד ב-3 כיוונים, אשר נסמן אותם \(\sigma_{z},\sigma_{y},\sigma_{x}\)

מה ההגדרה של חופש פאזה?
?
כפל של המקדמים עם מספר על המעל היחידה \(z = e^{i\theta}\) לא משנה את ההסתברות.

מה ההגדרה של צמוד מרוכב?
?
הצמוד של מספר מרוכב \(z=a+bi\) יהיה המספר \(\overline{z}=a-bi\).

מה ההגדרה של צמוד הרמיטי של מטריצה?
?
הצמוד ההרימטי של מטריצה \(U\) יהיה השיחלוף של הצמוד המרוכב של כל האיברים. מסומן \(U^{\dagger}\)

מה ההגדרה של מטריצה אוניטרי?
?
מטריצה המקיימת \(U U^{\dagger}=U^{\dagger} U=Id\).

מה ההגדרה של מטריצה הרמיטית?
?
מטריצה שמקיימת \(U^{\dagger}=U\).

מה ההגדרה של מרחב הילברט?
?
מרחב מכפלה פנימית שלם. לעיתים מסמנים ב-\(\mathcal{H}\).

מה ההגדרה של מכפלה פנימית?
?
תבנית ססקילינארית \(\left\langle  \cdot \mid\cdot  \right\rangle:\mathbb{C}\times \mathbb{C}\to \mathbb{R}\) מוגדרת על ידי:
$$\left\langle  \vec{v},\vec{u}  \right\rangle ={\vec{v}}^{\dagger} \vec{u}$$

מה ההגדרה של קאט?
?
נקרא לאיבר(ווקטור) במרחב הילברט קאט. נסמן אותו \(\ket{\psi}\)

מה ההגדרה של ברא?
?
עבור קאט \(\ket{\psi}\) נגדיר את הברא המתאים בתור \(\bra{\psi}\) אשר מקיים \(\bra{\psi}=\ket{\psi}^{\dagger}\).

מה ההגדרה של אופרטור לינארי?
?
עבור מרחב הילברט \(\mathcal{H}\) אופרטור \(\hat{X}\) נקרא אופרטור לינארי אם לכל \(v,u \in \mathcal{H}\) ו-\(c \in \mathbb{R}\) מתקיים:
$$\hat{X}(v+cu)=\hat{X}v+c\hat{X}u$$

מה ההגדרה של אופרטור הצמוד?
?
יהי \(\mathcal{H}\) מרחב הילברט. עבור אופרטור \(\hat{X}\) נגדיר את \(\hat{X}^{\dagger}\) ע"י האופרטור היחיד שמקיים לכל \(\ket{v},\ket{w}\):
$$\hat{X}\bra{v} \ket{w} =\bra{v} \hat{X}^{\dagger}\ket{w} $$

מה ההגדרה של אופרטור נורמלי?
?
אופרטור \(\hat{X}\) המקיים \(\hat{X}^{\dagger}\hat{X}=\hat{X}\hat{X}^{\dagger}\).

מה ההגדרה של אופרטור הרמיטי?
?
אופרטור \(\hat{X}\) המקיים \(\hat{X}^{\dagger}=\hat{X}\).

מה ההגדרה של אופרטור אנטי-הרמיטי?
?
אופרטור \(\hat{X}\) המקיים \(\hat{X}=-\hat{X}^{\dagger}\)

מה ההגדרה של אופרטור אוניטרי?
?
יהי \(\mathcal{H}\) מרחב הילברט. האופרטור \(\hat{X}\) יקרא אניטרי אם לכל \(v,w \in \mathcal{H}\) מתקיים:
$$\bra{T(v)}\ket{T(w)} =\bra{v} \ket{w}  $$

מה ההגדרה של תת מרחב ווקטורי?
?
עבור מרחב הילברט \(\mathcal{H}\) כל מרחב ווקטורי שמוכל ב-\(\mathcal{H}\) יהיה תת מרחב ווקטורי.

מה ההגדרה של סכום ישר?
?
שתי מרחבים נמצאים בסכום ישר אם החיתוך שלהם הוא אפס והסכום שלהם הוא המרחב כולו.

מה ההגדרה של אופרטור הטלה?
?
אם שתי מרחבים נמצאים בסכום ישר, ניתן להגדיר אופרטור הטלה על מרחב, שלוקח את הרכיב של הווקטור ששייך למרחב.

מה ההגדרה של אופרטור הטלה אורתוגונאלי?
?
יהי \(V\) מרחב ווקטורי ו- \(W\leq V\) תת מרחב וקטורי עם בסיס \(\ket{1},\dots \ket{n}{}\). 
נגדיר אופרטור \(\Pi_{n}= \ket{n}\bra{n}\) אופרטור מעל המרחב \(W\).

מה ההגדרה של מרחב הניצב?
?
עבור תת מרחב ווקטור \(V\) קיים מרחב מאונך \(V^\perp\) אשר כל ווקטור ב-\(V\) באונך לכל ווקטור ב-\(V^\perp\). מרחבים אלו נמצאים בסכום ישר.

מה ההגדרה של אקספוננט של מטריצה?
?
$$e^{X}=\sum_{k=0}^{\infty}{\frac{1}{k!}}X^{k}$$

מה ההגדרה של טור טיילור של אופרטור?
?
ניתן לכתוב פונקציה \(F\left( \hat{A} \right)\) על ידי טור טיילור:
$$F({\hat{A}})=\sum_{n=0}^{\infty}{\frac{F^{(n)}(0)}{n!}}{\hat{A}}^{n}$$

\section{סימטרייה}

?
$$\delta S=S\left( x+\varepsilon \eta \right)-S(x)$$

מה אומר משפט נטר?
?
קיום של סימטרייה רציפה גורר חוק שימור.

מה ההגדרה של חבורה?
?
קבוצה \(A\) עם פעולה \(\cdot:A\times A\to A\)  אשר מקיימת:

\begin{enumerate}
  \item קיום איבר זהות \(1\) אשר מקיים: 
$$\forall a \in A\quad a \cdot 1 = 1 \cdot a = 1$$


  \item לכל איבר \(a \in A\) קיים \(a^{-1} \in A\) כך ש: 
$$a \cdot a^{-1} = a ^{-1} \cdot a = 1$$


  \item אסוצייטיביות. 


\end{enumerate}
מה ההגדרה של חבורה אבלית?
?
חבורה אשר קומוטטיבית.

מה ההגדרה של הצגה של חוברה?
?
יהי \(G\) חבורה, ו-\(V\) מרחב ווקטורי. הומומורפיזם \(\rho:G\to GL(V)\) כאשר \(GL(V)\) מייצג את החבורת אוטומורפיזמים של \(V\)(כלומר כל העתקות הלינארית ההפיכות על \(V\)) נקראת הצגה של החבורה \(G\).

מה ההגדרה של מימד של הצגה?
?
עבור הצגה \(R:G\to GL_{n}\left( \mathbb{F}  \right)\) המימד של ההצגה יהיה \(n\). כלומר זה יהיה גודל המטריצות שבתמונה.

מה ההגדרה של תת יצוג?
?
יהי \(R\) ייצוג על מרחב ווקטורי \(V\). תת מרחב \(U\subset V\) אשר אינווריאנטי תחת כל אופרטור \(R(g)\) כאשר \(g\in G\) יקרא תת ייצוג.

מה ההגדרה של הצגה בלתי פריקה?
?
זוהי תהיה הצגה שהתת ייצוג היחיד שלה יהיה התת יצוג הטריוויאלי והיצוג עצמו.

מה ההגדרה של הצגה אוניטרית?
?
הצגה \(T\) שעבורה לכל \(g \in G\) נקבל \(T(g)\) אוניטרית.

מה אומר משפט האורתוגונאליות?
?
יהי \(T:G\to GL(V)\) ו-\(S:G\to GL(V)\) הצגות בלתי פריקות של חבורה סופית \(G\). נסמן ב-\(T(g)_{ij}\) וב-\(S(g)_{ij}\) את אלמנטי המטריצה של המטריצות המתאימות. אזי:

\begin{enumerate}
  \item אם \(S\) ו-\(G\) הם לא שקולות: 
$$\sum_{g\in G}T^{\dagger}(g)_{i j}S(g)_{k l}=0.$$


  \item אם \(S\) ו-\(G\) שקולות נסמן את מימד ההצגה ב-\(d\) ונקבל: 
$$\sum_{g\in G}T^{\dagger}(g)_{i j}T(g)_{k l}=\frac{|G|}{d}\delta_{i l}\delta_{j k},$$


\end{enumerate}
מה ההגדרה של סימטריה?
?
אוטומורפיזם(טרנספורמציה המשמרת מבנה של המרחב ווקטורי) על המרחב אשר אינו משנה את חוקי הפיזיקה.

מה אומר משפט ווגנר?
?
אם טרנספורמציה \(U\) היא טרנספורמציה סימטרית אז מתקיים אחד התנאים הבאים:

\begin{enumerate}
  \item האופרטור \(U\) יהיה אופרטור לינארי ואוניטרי: 
$$\ket{\psi} \mapsto \ket{\psi'} =U\ket{\psi} \qquad U^{\dagger} U = Id $$


  \item האופרטור \(U\) יהיה אופרטור אנטי אוניטרי ואנטי לינארי: 
$$\ket{\psi} \mapsto \ket{\psi'} =U\ket{\psi} \qquad \braket{ U\phi | U\psi } =\braket{ \psi | \phi } $$


\end{enumerate}
מה ההגדרה של סימטרייה רציפה?
?
סימטריה הניתנת להצגה על ידי אופרטור אוניטרי וקשורה ליחידה על ידי משתנה רציף. כלומר פונקציה רציפה \(U_{\varepsilon}\) אשר מקיימת \(U_{\varepsilon=0}=\mathrm{Id}\).

מה ההגדרה של יוצר של טרנספורמציה רציפה?
?
אופרטור \(T\) כך ש:
$$U_{\varepsilon}=e^{ i\varepsilon T }$$
כלומר זהו אופרטור אשר יוצר את החבורה של הסימטרייה.

\section{שאלות - תורת ההפרעות}


    const tsvFileName = 'Quantum_12.tsv';

    let cards = [];
    let quizOrder = [];
    let currentIndex = 0;

    // 1) Fetch and parse TSV
fetch(tsvFileName)
  .then(r => {
    if (!r.ok) throw new Error('Failed to load ' + tsvFileName);
    return r.text();
  })
  .then(text => {
    cards = parseTSV(text);
    renderList(cards);

    // ⬇ After you’ve rendered the full Q&A list, re-typeset MathJax
    if (window.MathJax) {
      MathJax.typesetPromise()
        .catch(err => console.error('MathJax typeset failed: ' + err.message));
    }
  })
  .catch(err => {
    console.error(err);
    //document.body.innerHTML =  '<p style="color:red">Error loading questions: '+err.message+'</p>';
  });

    // 2) TSV → [{front,back,tags},…]
    function parseTSV(text) {
  const rows = [];
  let row = [];
  let field = '';
  let inQuotes = false;

  for (let i = 0; i < text.length; i++) {
    const ch = text[i];
    const next = text[i + 1];

    if (inQuotes) {
      if (ch === '"' && next === '"') {
        // "" inside quoted field -> literal "
        field += '"';
        i++;
      } else if (ch === '"') {
        // closing quote
        inQuotes = false;
      } else {
        field += ch;
      }
    } else {
      if (ch === '"') {
        inQuotes = true;                // opening quote
      } else if (ch === '\t') {
        row.push(field); field = '';    // field separator
      } else if (ch === '\r') {
        // ignore; handle \n as newline
      } else if (ch === '\n') {
        row.push(field); rows.push(row);
        row = []; field = '';           // record separator
      } else {
        field += ch;
      }
    }
  }
  // flush last field/row if file didn't end with newline
  row.push(field);
  if (row.length > 1 || row[0] !== '') rows.push(row);

  // map to objects (Front, Back, Tags)
  return rows.map(cols => ({
    front: cols[0] || '',
    back:  cols[1] || '',
    tags:  cols[2] || ''
  }));
}
// 3) Render full Q&A list + “Start Quiz” button
function renderList(cards) {
  const listDiv = document.getElementById('qa-list');
  const container = listDiv.querySelector('.cards');
  container.innerHTML = '';

  cards.forEach(c => {
    const card = document.createElement('div');
    card.className = 'card';

    const q = document.createElement('div');
    q.className = 'question';
    q.textContent = c.front;   // <- no replace, real \n already in the string

    const a = document.createElement('div');
    a.className = 'answer';
    a.textContent = c.back;    // <- no replace

    card.appendChild(q);
    card.appendChild(a);
    container.appendChild(card);
  });

  // Re-typeset MathJax for the newly added nodes
  if (window.MathJax) {
    MathJax.typesetPromise([container]).catch(err => console.error(err));
  }
}
\end{document}