\documentclass{tstextbook}

\usepackage{amsmath}
\usepackage{amssymb}
\usepackage{graphicx}
\usepackage{hyperref}
\usepackage{xcolor}

\begin{document}

\title{Example Document}
\author{HTML2LaTeX Converter}
\maketitle

\chapter{קטגוריות}

\section{בסיס קבוצתי - יקום}

\begin{definition}[יקום]
קבוצה \(U\) אשר מקיימת את התכונות הבאות:

  \begin{enumerate}
    \item טרנזטיביות - אם \(X \in U\) וגם \(Y \in X\) אזי \(U \in U\). זה מבטיח כי האיברים של איברים הם גם ביקום. 


    \item סגירות לזוגות - אם \(X,Y\in U\) אזי גם \(\{ X,Y \}\in U\). 


    \item סגירות לאיחודים - אם \(X \in U\) אזי גם \(\bigcup_{Y \in X}Y \in U\).  


    \item קבוצת חזקה - אם \(X \in U\) אז הקבוצת חזקה \(\mathcal{P}(X)\in U\). 


    \item מכיל את הטבעיים. כלומר \(\mathbb{N} \in U\). 


  \end{enumerate}
\end{definition}
\begin{remark}
התכונות האלה הופכות את היקום להיות מודל טוב של תורת הקבוצות, אשר סגור להרכבות סטנדרטיות של קבוצות. זה מאפשר לנו להמנע מפרדוקסים כמו פרדוקס ראסל. זה למעשה המטרה של היקום - מאפשר לנו להסתכל על אוספים גדולים במספיק אך לא גדולים מידי אשר אינם קבוצה(וגורמים לפרדוקסים).

\end{remark}
\begin{definition}[אקסיומת היקום של גרותנדיק]
לכל קבוצה \(X\) יש יקום \(U\) עם \(X \in U\). כלומר לא משנה איזה קבוצה יש לנו, יש יקום שמכיל אותה.

\end{definition}
\begin{corollary}
בפרט אם \(U\) יקום אינו יכול להכיל את עצמו, אך לכל יקום קיים יקום גדול יותר \(V\) אשר מכיל אותו(\(U \in V\)).

\end{corollary}
\begin{definition}[קבוצה קטנה]
נקבע יקום \(U\) ונקרא לקבוצה קטנה אם הוא איבר ב-\(U\)(כלומר \(X \in U\)).

\end{definition}
\begin{definition}[קבוצה גדולה]
נקבע יקום גדול יותר \(V\) אשר מכיל את \(U\). נקרא לקבוצה \(X\) גדולה אם \(X \in V\).

\end{definition}
\begin{remark}
נשים לב כי בפרט כל קבוצה קטנה היא גדולה. אנחנו נתייחס לכל הקבוצות כאילו הם גדולות. 

\end{remark}
\begin{definition}[קטגוריה קטנה מקומית]
נקרה לקטגוריה \(C\) קטנה מקומית אם לכל זוג \(X,Y \in C\) קבוצת ההומומוריזמים \(\mathrm{Hom}_{C}(X,Y)\) היא קטנה.

\end{definition}
\begin{definition}[קטגוריה קטנה]
קטגוריה נקראת קטנה אם קטנה מקומית ובנוסף אוסף האובייקטים \(\mathrm{Ob}(C)\) היא קבוצה קטנה.

\end{definition}
\section{הגדרת הקטגוריה}

\begin{definition}[קטגוריה]
קטגוריה \(\mathcal{C}\) היא מורכבת מ:

  \begin{enumerate}
    \item אוסף של אובייקטים \(\mathrm{Ob}(\mathcal{C})\). 


    \item אוסף של מורפיזמים - לכל שתי אובייקטים \(A,B\in \mathrm{Ob}(\mathcal{C})\) אוסף מורפיזמים \(\mathrm{Hom}_{\mathcal{C}}(A,B)\)(לעיתים מסומן \(\mathcal{C}(A,B)\)). 


    \item פונקציית הרכבה - לכל שלושה אובייקטים \(A,B,C \in \mathrm{Ob}(\mathcal{C})\) פונקציית הרכבה: 
$$\circ :\mathrm{Hom}(B,C)\times \mathrm{Hom}(A,B)\to \mathrm{Hom}(A,C)$$


    \item פונקציית זהות - לכל אובייקט \(A \in \mathrm{Ob}(\mathcal{C})\) קיים \(\mathrm{Id}_{\mathcal{C}}(A)\). 


  \end{enumerate}
\end{definition}
אשר מקיים את התכונות:

\begin{enumerate}
  \item הפונקציית הרכבה היא אסוצייטיבית - אם \(f:A\to B,g:B\to C\)  ו-\(h:C\to D\) נקבל: 
$$h\circ (g\circ  f)=(h\circ  g)\circ  f$$


  \item זהות - אם \(f:A\to B\) אזי: 
$$\mathrm{Id}_{B}\circ  f=f \circ  \mathrm{Id}_{A}=f$$


\end{enumerate}
\begin{definition}[אובייקט תחילי]
אובייקט הוא תחילי אם יש בדיוק מורפיזם אחד ממנו לכל אובייקט אחר.

\end{definition}
\begin{definition}[אובייקט סופי]
אובייקט הוא סופי אם יש בדיוק מורפיזם אחד עליו מכל אובייקט אחר.

\end{definition}
\begin{example}
  \begin{table}[htbp]
    \centering
    \begin{tabular}{|ccc|}
      \hline
      קטגוריה & תחילי & סופי \\ \hline
      \(\text{Sets, Top}\) & \(\varnothing\) & \(\{ e \}\) \\ \hline
      \(\text{Grp,Ab}\) & \(\{ e \}\) & \(\{ e \}\) \\ \hline
      \(\text{Vect}_{\mathbb{F}}\) & \(\{ 0 \}\) & \(\{ 0 \}\) \\ \hline
      \(\text{Rings, ComRings}\) & \(\mathbb{Z}\) & \(\{ 0 \}\) \\ \hline
      \(\text{Rings,ComRings}\) בלי יחידה & אין & \(\{ 0 \}\) \\ \hline
      \(\text{Fld}\) & אין & אין \\ \hline
      \(\text{Fld}_{p}\) & \(\mathbb{F}_{p}\) & אין \\ \hline
      \(\text{Fld}_{0}\) & \(\mathbb{Q}\) & אין \\ \hline
    \end{tabular}
  \end{table}
\end{example}
\begin{definition}[קטגוריית מכפלה]
יהו \(\mathcal{C}\) ו-\(\mathcal{D}\) קטגוריות. הקטגוריית מכפלה שלהם תהיה הקטגוריה \(\mathcal{C}\times \mathcal{D}\) כאשר האובייקטים יהיו:
$$\mathrm{Ob}\left( \mathcal{C} \times \mathcal{D}  \right)=\mathrm{Ob}\left( \mathcal{C}  \right)\times \mathrm{Ob}\left( \mathcal{D}  \right)$$
והמורפיזמים יהיו:
$$\mathrm{Hom}_{\mathcal{C} \times \mathcal{D} }((C,D),(C',D')) =\mathrm{Hom}_{\mathcal{C} }(C,C')\times \mathrm{Hom}_{\mathcal{D} }(D,D')$$
כלומר האובייקטים יהיו זוגות \((C,D)\) והמורפיזמים יהיו זוג \((f,g)\) כאשר \(f:C\to C'\) ו-\(g:D\to D'\).

\end{definition}
\begin{definition}[תת קטגוריה]
יהי \(A\) קטגוריה. תת קטגוריה \(B\) היא קטגוריה אשר האובייקטים הם תת מחלקה \(\mathrm{Ob}(B)\) של \(\mathrm{Ob}(A)\) וכך שלכל \(C,C' \in B\) המורפימים הם תת מחלקה של \(\mathrm{Hom}_{A}(C,C')\) כך שסגורות להרכבה או זהות. התת קטגוריה היא מלאה אם:
$$\mathrm{Hom}_{A}(C,C')=\mathrm{Hom}_{B}(C,C')$$
לכל \(C,C'\in \mathrm{Ob}(B)\).

\end{definition}
\begin{definition}[הקטגוריה ההפוכה]
תהי \(\mathcal{C}\) קטגוריה. אזי הקטגוריה ההפוכה תהיה הקטגוריה \(\mathcal{C}^{\text{op}}\) אשר מקיימת:
- אובייקטיים זהים -\(\text{Ob}(\mathcal{C})=\text{Ob}(\mathcal{C}^{\text{op}})\).
- לכל \(X,Y\in \mathcal{C}\) נגדיר:
$$\hom_{{\mathcal{C}}^{o p}}\left(X,Y\right)=\hom_{{\mathcal{C}}}\left(Y,X\right)$$
- לכל מורפיזם \(f:X\to Y\) ב-\(\mathcal{C}\) נסמן \(f^{\text{op}}:Y\to X\) את המורפיזם המתאים ב-\(\mathcal{C}^{\text{op}}\) כך שמקיים את כלל ההרכבה:
$$.f^{o p}\circ g^{o p}=(g\circ f)^{o p}$$
- נגדיר את הזהות של \(X \in \mathrm{Ob}(\mathcal{C}^{\text{op}})=\mathrm{Ob}(\mathcal{C})\) להיות \((\mathrm{Id}_{X})^{\text{op}}\).

\end{definition}
\section{דוגמאות לקטגוריות}

\begin{definition}[קטגוריה Sets - קבוצות]
  \begin{itemize}
    \item \textbf{האובייקטים:} קבוצות קטנות
    \item \textbf{המורפיזמים:} לכל שתי קבוצות \(X,Y\) המורפיזמים המוגדרים על ידי:
$$\operatorname{Hom}_{\operatorname{Sets}}(X,Y)=Y^{X}=\{f:X\to Y\}$$
כלומר אוסף כל הפונקציות מ-\(X\) ל-\(Y\).
  \end{itemize}
\end{definition}
\begin{remark}
נשים לב כי \(Y^{X}\) זו גם קבוצה קטנה.

\end{remark}
\begin{definition}[קטגוריה Grp - חבורות]
  \begin{itemize}
    \item \textbf{האובייקטים:} חבורות אשר מוגדרות על קבוצה שהיא קטנה.
    \item \textbf{המורפיזמים:} הומומורפיזמים של חבורות.
  \end{itemize}
\end{definition}
\begin{definition}[קטגוריה Ring - חוגים]
  \begin{itemize}
    \item \textbf{האובייקטים:} חוגים קטנים(כלומר מוגדרות על קבוצות קטנות) עם יחידה.
    \item \textbf{המורפיזמים:} והומומורפיזם של חוגים.
  \end{itemize}
\end{definition}
\begin{definition}[קטגוריה Ab - חבורות אבליות]
  \begin{itemize}
    \item \textbf{האובייקטים:} חבורות אבליות אשר מוגדרות על קבוצה שהיא קטנה.
    \item \textbf{המורפיזמים:} הומומורפיזמים של חבורות(אשר בפרט מכבדות קומוטטיביות).
זו תהיה תת קטגוריה של הקטגוריה \(\mathrm{Grp}\).
  \end{itemize}
\end{definition}
\begin{definition}[קטגוריה Com Ring - חוגים קומוטטיבים]
  \begin{itemize}
    \item \textbf{האובייקטים:} חוגים קומוטטיבים קטנים עם יחידה.
    \item \textbf{המורפיזמים:} והומומורפיזם של חוגים.
  \end{itemize}
\end{definition}
\begin{definition}[קטגוריה Mon - מונואידים]
  \begin{itemize}
    \item \textbf{האובייקטים:} מונואידים(קבוצה עם פעולה אסוצייטיבית ויחידה) מוגדרים על קבוצה קטנה
    \item \textbf{המורפיזמים:} הומומורפיזמים של מונואידים - פונקציות אשר משמרות את הפעולה של המונואיד ואת היחידה
    \item \textbf{דוגמאות:}\((\mathbb{N},+),(\mathbb{F},\cdot)\) או \((\mathbb{Z},\cdot)\).
  \end{itemize}
\end{definition}
\begin{definition}[קטגוריה POS - קבוצה סדורה חלקית]
  \begin{itemize}
    \item \textbf{אובייקטים:} קבוצות קטנות סדורות חלקית.
    \item \textbf{המורפיזמים:} פונקציות מונוטוניות(משמרות סדר). כלומר אם \(a\leq b\) בקבוצה סדורה חלקית אחת, אז \(f(a)\leq f(b)\).
  \end{itemize}
\end{definition}
\begin{definition}[קטגוריה Grph - גרפים]
  \begin{itemize}
    \item \textbf{אובייקטים:} גרפים קטנים. כלומר קבוצה קטנה המכילה קודקודים וקצוות.
    \item \textbf{המורפיזמים:} הומומורפיזמים של גרפים - פונקציה אשר ממפה כל קודקוד של גרף אחד לקודקוד של גרף אחר כך שאם יש קצה מקודקוד \(\alpha\) לקודקוד \(\beta\) במקור אז התמונה תקיים \(f(\alpha)\leq f(\beta)\). 
  \end{itemize}
\end{definition}
\begin{definition}[קטגוריה Met - מרחבים מטרים]
  \begin{itemize}
    \item אובייקטים: מרחבים מטרים קטנים אשר מצויידים במטריגה \(d_{X}\).
    \item מורפיזמים: פונקציות אשר אינם מרחבות, זאת אומרת פונקציה \(f:X\to Y\) אשר מקיימת:
$$d_{Y}(f(\alpha),f(\beta))\leq d_{X}(\alpha,\beta)$$
כלומר לכל \(\alpha,\beta \in X\) פונקציה כך שהמרחק בין שתי נקודות אשר לא גדל תחת \(f\).
  \end{itemize}
\end{definition}
\begin{definition}[הקטגוריה \(\mathrm{Set^{fin}}\) - קבוצות סופיות]
  \begin{itemize}
    \item \textbf{אובייקטים:} קבוצות אשר סופיות(בפרט יהיו קטנות).
    \item \textbf{מורפיזמים:} כל פונקציה בין הקבוצות האלה.
  \end{itemize}
\end{definition}
\begin{definition}[הקטגוריה \(\mathrm{Set^{inj}}\) - קבוצות עם מורפיזמים חח"ע]
  \begin{itemize}
    \item \textbf{אובייקטים:} קבוצות קטנות
    \item \textbf{מורפיזמים:} פונקציות חד חד ערכיות בין הקבוצות האלה.
  \end{itemize}
\end{definition}
\begin{definition}[הקטגוריה \(\mathrm{Grp^{fg}}\) - קבוצות נוצרות סופית]
  \begin{itemize}
    \item \textbf{אובייקטים:} חבורות קטנות שנוצרות סופית.
    \item \textbf{מורפיזמים:} הומומורפיזמים בין החבורות האלה.
  \end{itemize}
\end{definition}
\begin{definition}[הקטגוריה Top - מרחבים טופולוגיים]
  \begin{itemize}
    \item \textbf{אובייקטים:} מרחבים טופולוגים קטנים.
    \item \textbf{מורפיזמים:} פונקציות רציפות בין המרחבים הטופולוגיים.
  \end{itemize}
\end{definition}
\begin{definition}[הקטגוריה Rel - יחסים]
  \begin{itemize}
    \item \textbf{אובייקטים:} קבוצות קטנות.
    \item \textbf{מורפיזמים:} לכל שתי קבוצות קטנות \(X,Y\) מורפיזם יהיה יחס \(R\subseteq X\times Y\). נשים לב כי הקבוצה של כל הפונקציות \(Y^{X}\) היא תת קבוצה של היחסים האלו.
    \item \textbf{הרכבה:} בהנתן \(R\subseteq X \times Y\) וגם \(S\subseteq Y \times Z\) אז ההרכבה:
$$S\circ R=\{(x,z)\in X\times Z\mid\exists y\in Y,\;(x,y)\in R\;{\mathrm{and}}\;(y,z)\in S\}$$
  \end{itemize}
\end{definition}
\begin{definition}[הקטגוריה \(\mathrm{Sub_{X}}\)]
  \begin{itemize}
    \item \textbf{אובייקטים:} עבור קבוצה \(X \in U\) האובייקטים הם האיברים של קבוצת החזקה \(\mathcal{P}(X)\). כלומר כל התתי קבוצות של \(X\).
    \item \textbf{מורפיזמים:} לכל שתי תתי קבוצות \(Y,Z \subseteq X\) נגדיר:
$$\operatorname{Hom}_{\operatorname{Sub}_{x}}(Y,Z)={\left\{\begin{array}{l l}{\{I_{Y,Z}\}}&{{\mathrm{if~}}Y\subseteq Z,}\\ {\varnothing }&{{\mathrm{otherwise}}}\end{array}\right.}$$
  \end{itemize}
\end{definition}
\begin{definition}[הקטגוריה BG]
  \begin{itemize}
    \item \textbf{אובייקטים:} לחבורה קטנה \(G\) לקטגוריה BG יש בדיוק איבר יחיד המסומן ב-\(*\).
    \item \textbf{מורפיזמים:}\({\mathrm{Hom}}_{B G}(*,*)=G\), כלומר המורפיזמים הם איברי החבורה. איבר הזהות של החבורה מתאים למורפיזם הזהות.
למעשה כל מורפיזם אמנים אותו דבר מבחינת איך שהם מתנהגים על אובייקטים, אך המורפיזמים יהיו שונים בגלל שמתנהגים בצורה שונה עם הרכבה.
  \end{itemize}
\end{definition}
\begin{definition}[הקטגוריה \(\text{Mat}_{R}\)]
אלו הם המטריצות מעל חוג. מוגדר באופן הבא:
- \textbf{אובייקטים:} המספרים הטבעיים שאינם אפס(מייצגים מיימד)
- \textbf{מורפיזמים:} המורפיזמים בין \(m\) ל-\(n\) הם המטריצות \(m\times n\) עם איברים ב-\(R\).

\end{definition}
\section{דואליות}

\begin{definition}[הקטגוריה ההפוכה]
תהי \(\mathcal{C}\) קטגוריה. אזי הקטגוריה ההפוכה תהיה הקטגוריה \(\mathcal{C}^{\text{op}}\) אשר מקיימת:
- אובייקטיים זהים -\(\text{Ob}(\mathcal{C})=\text{Ob}(\mathcal{C}^{\text{op}})\).
- לכל \(X,Y\in \mathcal{C}\) נגדיר:
$$\hom_{{\mathcal{C}}^{o p}}\left(X,Y\right)=\hom_{{\mathcal{C}}}\left(Y,X\right)$$
- לכל מורפיזם \(f:X\to Y\) ב-\(\mathcal{C}\) נסמן \(f^{\text{op}}:Y\to X\) את המורפיזם המתאים ב-\(\mathcal{C}^{\text{op}}\) כך שמקיים את כלל ההרכבה:
$$.f^{o p}\circ g^{o p}=(g\circ f)^{o p}$$
- נגדיר את הזהות של \(X \in \mathrm{Ob}(\mathcal{C}^{\text{op}})=\mathrm{Ob}(\mathcal{C})\) להיות \((\mathrm{Id}_{X})^{\text{op}}\).

\end{definition}
\begin{example}[הקטגוריה \(\text{Mat}_{R}^{\text{op}}\)]
  \begin{itemize}
    \item עבור \(\text{Mat}_{R}\) האובייקטים היו המספרים הטבעיים, ולכן האובייקטים של \(\text{Mat}_{R}^{\text{op}}\) הם גם כן המספרים הטבעיים.
    \item עבור \(\text{Mat}_{R}\) המורפיזמים בין \(m\) ל-\(n\) היו המטריצות ה-\(m\times n\). כעת ב-\(\text{Mat}_{R}^{\text{op}}\) המורפיזמים בין \(m\) ל-\(n\) יהיו המורפיזמים בין \(n\) ל-\(m\) ב-\(\mathrm{Mat}_{R}\), כלומר המטריצות ה-\(n \times m\). ההרכבה הופכת את הסדר.
  \end{itemize}
\end{example}
\begin{definition}[פונקטור קונטרא וריאנטי]
יהיו \(\mathcal{C},\mathcal{D}\) קטגוריות. פונטקטור קונטרא וריאנטי הוא פונקטור \(\mathcal{C}^{\text{op}}\to \mathcal{D}\).

\end{definition}
\begin{remark}
לכל פונקטור \(\mathcal{C}\to \mathcal{D}\) קיים פונקטור \(\mathcal{C}^{\text{op}}\to \mathcal{D}^{op}\). וכל קטגוריה מקיימת \((\mathcal{C}^{\text{op}})^{\text{op}}=\mathcal{C}\). לכן ניתן לתאר פונקטור קונטרא וריאנטי על ידי פונקטור \(\mathcal{C}\to \mathcal{D}^{\text{op}}\).

\end{remark}
\begin{remark}
פונקטור \(\mathcal{C}\to \mathcal{D}\) לעיתים נקרא פונקטור קו וריאנטי כדי להדגיש.

\end{remark}
\begin{example}
יהי שדה \(k\). לכל שתי מרחביים וקטורים \(V,W\) מעל \(k\) קיים מרחב ווקטורי:
$$\mathrm{Hom}(V,W)=\{ \text{linear maps }V \to W\}$$
כעת נקבע ווקטור \(W\), כל העתקה לינארית \(f:V\to V'\) משרה העתקה לינארית:
$$f^{*}:\mathrm{Hom}(V',W)\to \mathrm{Hom}(V,W)$$
המוגדר ב-\(q \in \mathrm{Hom}\)

\end{example}
\section{איזומורפיזם של אובייקטים}

\begin{definition}[איזומורפיזם של אובייקטים]
יהיו \(A,B \in \mathrm{Ob}(\mathcal{C})\). מורפיזם \(f:A\to B\) נקרא איזומורפיזם אם קיים מורפיזם \(g:B\to A\) כך ש:
$$f\circ  g = \mathrm{Id}_{B}\qquad \text{and}\qquad g \circ  f = \mathrm{Id}_{A}$$

\end{definition}
\begin{example}
הזהות \(\mathrm{Id}\) הוא איזומורפיזם. אנו יודעים כי:
$$\mathrm{Id_{X}\circ Id_{X}=Id_{X}}$$
לכן כל אובייקט \(X\) איזומורפי לעצמו.

\end{example}
\begin{example}
  \begin{itemize}
    \item עבור הקטגוריה \(\text{Sets}\) האיזומורפיזמים הם בדיוק הפונקציות אשר חח"ע ועל. 
    \item עבור \(\text{Grp, Ring, Field}\) האיזומורפיזמים הם בדיוק הומומורפיזמים החח"ע ועל.
    \item עבור \(\text{Top}\) האיזומורפיזמים הם ההומאומורפיזמים, כלומר הפונקציות הרציפות עם הופכי רציף.
    \item עבור קבוצות סדורות חלקית \(\left( P,\leq \right)\) מהאקסיומה של האנטי סימטריה נקבל כי \(x\leq y\) ו-\(y\leq x\) לכן האיזומופיזמים היחידים הם הזהות.
  \end{itemize}
\end{example}
\begin{proposition}
עבור \(f:X\to Y\) אם קיימים \(g,h:Y\to X\) כך שמקיימים:
$$g\circ f=\operatorname{Id}_{X}\quad{\mathrm{and}}\quad f\circ h=\operatorname{Id}_{Y}$$
אז \(g=h\) וגם \(f\) איזומורפיזם. כלומר עבור איזומורפיזם קיים הופכי.

\end{proposition}
\begin{proof}
נתחיל מ-\(g\). נשים לב כי \(g=g\circ\mathrm{Id}_{Y}\). כיוון ש-\(\mathrm{Id}_{Y}=f\circ h\) ניתן להציב ולקבל:
$$g=g\circ(f\circ h)$$
מאסוצייטביות ההרכבה נקבל \(g\circ(f\circ h)=(g\circ f)\circ h.\) אבל \(g \circ f = \mathrm{Id}_{X}\) מההנחה ולכן:
$$(g\circ f)\circ h=\operatorname{Id}_{X}\circ h=h$$
כלומר קיבלנו \(g=h\). העובדה שאיזומורפיזם נובע ישירות כיוון ששווים, כיוון שכעת יש פונקציה \(g\) אשר מקיימת \(g\circ f = \mathrm{Id}_{X}\) וגם \(f\circ g = \mathrm{Id}_{Y}\).

\end{proof}
\begin{proposition}
אם \(f:X\to Y\) הוא איזומורפיזם אז ההופכי \(g\)(אשר מקיים \(g\circ f = \mathrm{Id}_{X}\) וגם \(f\circ g=\mathrm{Id}_{Y}\)) הוא יחיד, וניתן לסמן אותו ב-\(f^{-1}:Y\to X\).

\end{proposition}
\begin{proof}
נניח כי \(g\) ו-\(h\) שתיהם הופכיים של \(f\), כלומר:
\begin{gather*}g\circ f=\operatorname{Id}_{X}\quad{\mathrm{and}}\quad f\circ g=\operatorname{Id}_{Y} \\h\circ f=\operatorname{Id}_{X}\quad{\mathrm{and}}\quad f\circ h=\operatorname{Id}_{Y}
\end{gather*}
כעת נקבל כמו בטענה הקודמת:
$$g=g\circ{\mathrm{Id}}_{Y}=g\circ(f\circ h)=(g\circ f)\circ h={\mathrm{Id}}_{X}\circ h=h$$
ולכן ההופכי יחיד.

\end{proof}
\begin{definition}[אנדומורפיזם]
מורפיזם שהתחום זהם לטווח.

\end{definition}
\begin{definition}[אוטומורפיזם]
אנדומורפיזם שהוא איזומורפיזם.

\end{definition}
\begin{definition}[גרופויד]
קטגוריה שכל מורפיזם הוא איזומורפיזם.

\end{definition}
\begin{example}
חבורה היא גרופאיד עם אובייקט יחיד(אפשר אפילו להגדיר ככה חבורה).

\end{example}
\begin{example}[הגרופיד היסודי]
יהי \(X\) מרחב טופולוגי נגדיר קטגוריה חדשה בצורה הבאה:
- אובייקטיים: איברים במרחב הטופולוגי \(x \in X\).
- מורפיזימיים: מחלקות הומוטפייה של מסילות - \(\left[ \gamma \right]:x\to y\).
הרכבה - שרשור של מסילות.
זהות - המסילה הקבועה.
הופכי - המסילה ההפוכה \(\left[ \gamma ^{-1} \right]\).
נקבל לבסוף את הגרופאיד היסודי \(\Pi_{1}(X)\) אשר מכליל את החבורה היסודית \(\pi_{1}(X,x)\).

\end{example}
\chapter{פונקטורים}

\section{שיכונים}

פונקטור \(F:\mathcal{C}\to \mathcal{D}\) נקרא נאמן(faithful) אם לכל \(A,A' \in \mathrm{Ob}(\mathcal{C})\) ההעתקה:
$$F:\mathrm{Hom}_{\mathcal{C}}(A,A')\to \mathrm{Hom}_{\mathcal{D} }(F(A),F(A'))$$
הוא חח"ע.

\begin{definition}[פונקטור מלא]
פונקטור \(F:\mathcal{C}\to \mathcal{D}\) נקרא מלא(full) אם לכל \(A,A' \in \mathrm{Ob}(\mathcal{C})\) ההעתקה:
$$F:\mathrm{Hom}_{\mathcal{C}}(A,A')\to \mathrm{Hom}_{\mathcal{D} }(F(A),F(A'))$$
היא על.

\end{definition}
\begin{example}
\end{example}
עבור פונקטור המקיים \(F(f_{1})=h,F(f_{2})=h\) נקבל כי לא נאמן כי \(F(f_{1})=F(f_{2})\) לא נותן \(f_{1}=f_{2}\). עבור פונקטור מהצורה \(G(h)=f_{1}\) נקבל כי לא מלא כי לא על.

\begin{definition}[שיכון]
פונקטור שהוא נאמן ומלא. כלומר לכל \(A,A' \in \mathrm{Ob}(\mathcal{C})\) ההעתקה:
$$F:\mathrm{Hom}_{\mathcal{C}}(A,A')\to \mathrm{Hom}_{\mathcal{D} }(F(A),F(A'))$$
היא חח"ע ועל.

\end{definition}
\begin{proposition}
יהי \(F\) שיכון, ו-\(f\) מורפיזם אזי \(F(f)\) איזומורפיזם גורר ש-\(f\) איזומורפיזם. כלומר יש גם גרירה בכיוון ההפוך.

\end{proposition}
\begin{reminder}[תת קטגוריה]
יהי \(\mathcal{C}\) קטגוריה. \(\mathcal{S}\) נקרא תת קטגוריה של \(\mathcal{C}\) אם \(\mathrm{Ob}(\mathcal{S})\subseteq \mathrm{Ob}(\mathcal{C})\) וגם לכל אובייקטים \(X,Y \in \mathrm{Ob}(\mathcal{D})\) המורפיזמים מוכלים במורפיזמים של \(\mathrm{Hom}_{\mathcal{D}}(X,Y)\subseteq \mathrm{Hom}_{\mathcal{C}}(X,Y)\) כך שמהווה כקטגוריה בפני עצמה, כלומר מכיל את המורפיזמי יחידה וסגור להרכבות. תת קטגוריה תהיה מלאה אם כל המורפיזמים "מורשים" ל-\(\mathcal{D}\). כלומר \(\mathrm{Hom}_{\mathcal{D}}(X,Y)= \mathrm{Hom}_{\mathcal{C}}(X,Y)\).

\end{reminder}
\begin{proposition}[פונקטור ההכלה]
יהי \(\mathcal{D}\subseteq \mathcal{C}\) תת קטגוריה. אזי הפונקטור \(F:\mathcal{D}\to \mathcal{C}\) במוגדר על ידי הזהות יהיה נאמן.

\end{proposition}
\begin{proof}
פונקטור \(F:\mathcal{C}\to \mathcal{D}\) נקרא נאמן אם לכל \(X,Y \in \mathcal{C}\) המורפיזמים \(\mathrm{Hom}_C(X, Y) \to \mathrm{Hom}_D(F(X), F(Y))\) הוא חח"ע. 
יהיו \(f\)

\end{proof}
\begin{proposition}
יהי \(\mathcal{D}\subseteq \mathcal{C}\) תת קטגוריה. אזי פונקטור ההכלה יהיה מלא אם"ם התת קטגוריה מלאה.

\end{proposition}
\section{הפונקטור}

\begin{definition}[פונקטור]
יהיו \(\mathcal{C,D}\) קטגוריות. פונקטור \(F:\mathcal{C}\to \mathcal{D}\) מוגדר על ידי:

  \begin{enumerate}
    \item העתקה על אובייקטים - \(\mathrm{Ob}(\mathcal{C})\to \mathrm{Ob}(\mathcal{D})\). 


    \item העתקה על מורפיזמים - לכל \(A,A' \in \mathrm{Ob}(\mathcal{C})\) העתקה: 
$$\mathrm{Hom}_{\mathcal{C} }(A,A')\to \mathrm{Hom}_{\mathcal{D} }(F(A),F(B))$$


  \end{enumerate}
\end{definition}
כך שמתקיים:

\begin{enumerate}
  \item משמרת הרכבה - עבור \(A\xrightarrow{f}A'\xrightarrow{f'}A''\) נקבל: 
$$F(f\circ f')=F(f)\circ F(f')$$


  \item משמרות זהות - לכל \(A \in \mathrm{Ob}(\mathcal{C})\) מתקיים \(\mathrm{F}(\mathrm{Id}_{A})=\mathrm{Id}_{F(A)}\). 


\end{enumerate}
\begin{remark}
ההגדרה של פונקטור היא כך שלכל מחרוזת:
$$A_{0}\;{\xrightarrow{\ f_{1}\ }}\;\cdot\cdot\cdot\;{\xrightarrow{\ f_{n}\ }}\;A_{n}$$
של אובייקטים ומורפיזמים בקטגוריה קיים בדיוק העתקה אחת \(F(A_{0})\to F(A_{n})\).

\end{remark}
\begin{example}
נראה ש-\(\mathcal{P}:\mathrm{Sets}\to \mathrm{Sets}\) אשר שולח כל קבוצה לקבוצה החזקה שלו הוא פונקטור. נדרש לשם כך להגדיר במפורש איך יפעול על האובייקטים ואיך יפעול על המורפיזמים. על האובייקטים:
$${\mathcal{P}}(X):=\{A\subseteq X\}$$
על מורפיזם \(f:X\to Y\) נגדיר \({\mathcal{P}}(f):{\mathcal{P}}(X)\to{\mathcal{P}}(Y)\) על ידי:
$${\mathcal{P}}(f)(A):=f[A]:=\{f(a)\mid a\in A\}\subseteq Y$$
כעת נראה שמקיים את דרישות.

  \begin{enumerate}
    \item משמר יחידה - יהי \(\mathrm{Id}_{X}:X\to X\) העתקת הזהות, אזי לכל \(A\subseteq X\) מתקיים: 
$${\mathcal{P}}({\mathrm{id}}_{X})(A)={\mathrm{id}}_{X}[A]=\{{\mathrm{id}}_{X}(a)\mid a\in A\}=A$$


    \item משמר הרכבה - יהי \(f:X\to Y\) ו-\(g:Y\to Z\) אזי עבור \(A\subseteq X\) מתקיים מצד אחד: 
$${\mathcal{P}}(f)(A)=f[A]\subseteq Y\implies {\mathcal{P}}(g)(f[A])=g[f[A]]=\{g(f(a))\mid a\in A\}$$
כאשר מצד שני:
$${\mathcal{P}}(g\circ f)(A)=(g\circ f)[A]=\{g(f(a))\mid a\in A\}$$
ולכן:
$${\mathcal{P}}\left( g\circ f \right)(A)={\mathcal{P}}(g)\left( {\mathcal{P}}(f)(A) \right)\implies {\mathcal{P}}(g\circ f)={\mathcal{P}}(g)\circ{\mathcal{P}}(f)$$
ולכן קיבלנו כי אכן פונקטור.


  \end{enumerate}
\end{example}
\begin{proposition}
הרכבה של פונקטורים נותן פונקטור. כלומר בהנתן שתי פונקטורים:
$$F:C\to D\quad{\mathrm{and}}\quad G:D\to E,$$
נגדיר את ההרכבה שלהם \(G\circ F:C\to E\) על ידי:

  \begin{enumerate}
    \item עבור אובייקט \(X \in \mathrm{Ob}(C)\) נגדיר: 
$$(G\circ F)(X)=G(F(X))$$


    \item עבור מורפיזם \(f:X\to Y\) ב-\(C\) נגדיר: 
$$(G\circ F)(f)=G(F(f))$$
וזה נותן פונקטור.


  \end{enumerate}
\end{proposition}
\begin{proposition}
לכל קטגוריה \(\mathcal{C}\) קיים פונקטור \(\mathrm{Id}_{\mathcal{C}}:\mathcal{C}\to \mathcal{C}\) אשר פועל בצורה הבאה:

  \begin{enumerate}
    \item על אובייקטים מקיים \(\mathrm{Id}_{\mathcal{C}}(X)=X\) לכל \(X \in \mathrm{Ob}(\mathcal{C})\). 


    \item על מורפיזמים מוגדר על ידי \(\mathrm{Id}_{\mathcal{C}}(f)=f\) לכל \(f \in \mathrm{Hom}_{\mathcal{C}}(X,Y)\). 


  \end{enumerate}
\end{proposition}
\begin{remark}
פונקטור הזהות משמש כאיבר ניטרלי בהרכבה של פונקטורים. כלומר עבור כל פונקטורים \(F:C\to D\) ו-\(H:B\to C\) מתקיים:
$$F\circ{\mathrm{Id}}_{C}=F\qquad{\mathrm{Id}}_{C}\circ H=H$$

\end{remark}
\begin{proposition}
פונקטור משמר איזומורפיזם של אובייקטים.

\end{proposition}
\begin{proof}
נזכור כי מורפיזם \(f:X\to Y\) נקרא איזומורפיזם אם קיים מורפיזם \(g:Y\to X\) כך שמתקיים:
$$g\circ f=\operatorname{Id}_{X}\quad{\mathrm{and}}\quad f\circ g=\operatorname{Id}_{Y}$$
כיוון שפונקטורים משמרים הרכבה ויחידה נקבל כי תכונה זו נשמרת ולכן \(Ff\) יהיה איזומורפיזם ב-\(D\).

\end{proof}
\section{סיווג פונקטורים}

\begin{definition}[פונקטורים שוכחות מבנה]
פונקטורים אשר ממפות מבנה "עשיר" יותר למבנה פחות עשיר, כך שמאבד את הבנה שלו.

\end{definition}
\begin{example}[פונקטור שוכח מבנה]
נסתכל על הפונקטור \(F:\text{Top}\to\text{Sets}\).
- על האובייקטים נגדיר \(F(X,\tau)=X\) כאשר \(X\) זה המרחב הטופולוגיים ב-\(\tau\) זה הטופולוגיה. הפונקטור "שוכח" את המבנה של הטופולוגיה.
- עבור מורפיזמים נגדיר אם \(f:(X,\tau)\to (Y,\sigma)\) רציפה אזי \(F(f)=f\) זו הפונקציה בין קבוצות.

\end{example}
\begin{example}[הרכבה של פונקטורים שוכחים מבנה]
נגדיר פונקטור \(F:\text{Grp}\to\text{Sets}_{*}\) על ידי \(F(G,e,\cdot)=(G,e)\) אשר שוכחת את המבנה של החבורה. 
כעת נגדיר פונקטור נוסף \(H:\text{Sets}_{*}\to\text{Sets}\) על ידי \(H(X,x)=X\). נקבל כי ההכבה \(H\circ F:\text{Grp}\to\text{Sets}\) שוכחת את כל המבנה פרט למבנה של החבורה.

\end{example}
\begin{definition}[פונקטורים מכלילים מבנה]
פונקטורים אשר מכלליות את המבנה, כלומר מוסיף חופש נוסף למבנה

\end{definition}
\begin{example}[פונקטורים מכללים מבנה]
פונקטורים מהצורה \(F:\mathrm{Ab}\to \mathrm{Grp}\) או מהצורה \(F:\mathrm{Sets ^{fin}}\to \mathrm{Sets}\) יהיו מכלילות מבנה.

\end{example}
\begin{example}
החבורה היסודית מגדירה פונקטור \(\pi_{1}:\mathsf{Top}_{*}\to \mathsf{Group}\). פונקציה רציפה \(f:(X,x)\to (Y,y)\) משרה הומומורפיזם של חבורות \(f_{*}:\pi_{1}(X,x)\to\pi_{1}(Y,y)\) באופן פונקטוריאלי. ניתן להגיד שהחבורה היסודית אינווריאנטית להומוטפיה על ידי זה שכל פונקטור מתפרק ל-\(\mathsf{Top}_{*}\to \mathsf{Htpy}_{*}\) ול-\(\pi_{1}:\mathsf{Htpy}_{*}\to \mathsf{Group}\).

\end{example}
\begin{example}
הגרופויד היסודי המשרה פונקטור \(\Pi_{1}:\mathsf{Top}\to\mathsf{Groupoid}\) על ידי זה שכל פונקציה רציפה \(f:X\to Y\) משרה פונקטור \(f_{*}:\Pi_{1}(X)\to \Pi_{1}(Y)\) אשר לוקח כל נקודה \(x \in X\) לנקודה \(f(x)\in Y\). 

\end{example}
\begin{example}
הפונקטור \(F:\mathsf{Sets}\to\mathsf{Group}\) שלוקח כל קבוצה \(X\) לחבורה החופשית של \(X\). זוהי החבורה שהאיברים שלהם הם מילים סופיות אשר האותיות שלה הם \(x\) או \(x ^{-1}\), עד כדי היחס שקילות ש-\(x x ^{-1}\) ו-\(x ^{-1} x\) הם מזדהים עם הקבוצה הריקה.

\end{example}
\section{העתקות טבעיות}

אם יש לנו שתי פונקטורים \(F,G:\mathcal{C}\to \mathcal{D}\) אנחנו רוצים להסתכל על הקשר ביניהם. לשם נגדיר העתקה בין פונקטורים.

\includegraphics[width=0.8\textwidth]{diagrams/svg_1.svg}
וכדי שיתנהג יפה נרצה שהדיאגרמה תתחלף. ניתן לצמצם את זה לכך שיתחלף רק בחלק הצבוע.

\includegraphics[width=0.8\textwidth]{diagrams/svg_2.svg}
כלומר:
$$\alpha_{A}\circ  G(f)=F(f)\circ  \alpha_{B}$$
האיברים \(\alpha_{X}\) נקראים רכיבים.

\begin{definition}[קטגוריית פונקטורים]
בהנתן שתי קטגוריות \(\mathcal{A}\) ו-\(\mathcal{B}\) ניתן להגדיר קטגוריית פונקטורים על ידי \([\mathcal{A},\mathcal{B}]\) כך שהאובייקטים הם הפונקטורים \(F:\mathcal{A}\to \mathcal{B}\) והמורפיזמים הם ההעתקות הטבעיות מ-\(F\) ל-\(G\).

\end{definition}
\begin{definition}[איזומורפיזם טבעי]
איזומורפיזם טבעי בין פונקטורים מ-\(\mathcal{A}\) ל-\(\mathcal{B}\) יהיה איזומורפיזם ב-\([\mathcal{A},\mathcal{B}]\). כלומר העתקה \(\alpha:F\to G\) כך ש-\(\beta:G\to F\) טבעית, עם \(\beta \circ \alpha = \mathrm{Id}_{F}\) ו-\(\alpha \circ \beta = \mathrm{Id}_{G}\).

\end{definition}
\begin{proposition}[תנאי שקול לאיזומורפיזם טבעי]
העתקה \(\alpha:F\to G\) טבעית היא איזומורפיזם טבעי אם"ם \(\alpha_{X}:FX\to GX\) היא איזומורפיזם לכל \(A \in \mathcal{A}\).

\end{proposition}
\begin{proof}
נניח כי \(\alpha:F\to G\) היא העתקה טבעית כך שלכל \(X \in \mathcal{A}\) ההעתקה \(\alpha_X:FX\to GX\) היא איזומורפיזם. נגדיר \(\beta:G\to F\) כך שלכל \(X\) ניקח \(\beta_X = \alpha_X^{-1}\). נבדוק כי \(\beta\) טבעית:
לכל מורפיזם \(f:A\to B\) מתקיים:
$$\beta_B \circ G(f) = \alpha_B^{-1} \circ G(f)
$$
מצד שני,
$$F(f) \circ \beta_A = F(f) \circ \alpha_A^{-1}
$$
הטבעיות של \(\alpha\) נותנת:
$$\alpha_A \circ G(f) = F(f) \circ \alpha_B
$$
נכפיל ב-\(\alpha_B^{-1}\) מימין וב-\(\alpha_A^{-1}\) משמאל:
$$\alpha_A^{-1} \circ \alpha_A \circ G(f) \circ \alpha_B^{-1} = \alpha_A^{-1} \circ F(f) \circ \alpha_B \circ \alpha_B^{-1}
$$
כלומר:
$$G(f) \circ \alpha_B^{-1} = \alpha_A^{-1} \circ F(f)
$$
ולכן:
$$\beta_B \circ G(f) = F(f) \circ \beta_A
$$
כלומר \(\beta\) טבעית. בנוסף, \(\beta \circ \alpha = \mathrm{Id}_F\) ו-\(\alpha \circ \beta = \mathrm{Id}_G\) לפי הגדרת \(\beta_X = \alpha_X^{-1}\). לכן \(\alpha\) איזומורפיזם טבעי.

\end{proof}
\begin{definition}[פונקטורים איזומורפיים]
פונקטורים שקיים איזומורפיזם טבעי ביניהם.

\end{definition}
\begin{remark}
  \begin{itemize}
    \item הדרך הנכונה לחשוב על שתי איברים של קבוצה כאותו דבר זה שיוויון.
    \item הדרך הנכונה לחשוב על שתי אובייקטים בקטגוריה כאותו דבר הוא איזומורפיזם.
    \item לכן אם נסתכל על קטגוריית הפונקטורים \([\mathcal{A},\mathcal{B}]\) נקבל כי הדרך הנכונה לחשוב על שיוויון של שתי פונקטורים היא איזומורפיזם טבעי.
  \end{itemize}
\end{remark}
\begin{definition}[שקילות בין קטגוריות]
קטגוריות \(\mathcal{C},\mathcal{D}\) נקראות שקולות אם קיים זוג של של פונקטורים \(F:\mathcal{C}\to \mathcal{D},G:\mathcal{D}\to \mathcal{C}\) כך ש:
$$\eta:\mathrm{Id}_{\mathcal{C} }\to G\circ  F\qquad \varepsilon:F\circ G\to \mathrm{Id}_{\mathcal{D} }$$
איזומורפיזמים טבעיים.

\end{definition}
\begin{definition}[פונקטור מהותית על]
פונקטור \(F:\mathcal{C}\to \mathcal{D}\) נקרא מהותית על אם לכל \(B \in \mathcal{D}\) קיים \(A \in \mathcal{C}\) כך ש:
$$F(A)\cong  B$$

\end{definition}
\begin{proposition}
אם קיים פונקטור \(F:\mathcal{C}\to \mathcal{D}\) כך ש-\(F\) נאמן, מלא ומהותית על אזי \(\mathcal{C,D}\) קטגוריות שקולות.

\end{proposition}
\begin{corollary}
אם \(F:\mathcal{C}\to \mathcal{D}\) מלא ונאמן אזי \(\mathcal{C}\) שקול לתת קטגוריה מלאה \(\mathcal{D'}\) של \(\mathcal{D}\) אשר האובייקטים שלה הם מהצורה \(F(C)\) עבור \(C \in \mathcal{C}\).

\end{corollary}
\begin{proof}
הפונקטור \(F':\mathcal{C}\to \mathcal{D'}\) הוא מלא ונאמן(כי \(F\) כזה), ומהותית על לפי ההגדרה של \(\mathcal{D'}\).

\end{proof}
\chapter{הלמה של יונדה}

\section{ייצוג וקו יצוג}

\begin{definition}[פוקטור הום הקו וריאנטי]
יהי \(A \in \mathcal{C}\). אזי ניתן להגדיר את הפונקטור הקו וריאנטי על ידי:
$$h_{A}=\operatorname{Hom}(A,-)\colon{\mathcal{C}}\to\mathbf{Set}$$
זה שולח כל אוביקט \(X\) לקבוצת המורפיזמים \(\mathcal{C}(A,X)\) וכל מורפיזם \(f:X\to Y\) לפונקציה:
$$\mathrm{Hom}(A,f)\colon{\mathcal{C}}(A,X)\to{\mathcal{C}}(A,Y),\quad g\mapsto f\circ g$$

\end{definition}
\begin{definition}[פונקטור הום קונטרה וריאנטי]
יהי \(B \in \mathcal{C}\). אזי הפונקטור Hom הקונטרא וריאנטי(לעיתים נקרא presheaf) הוא:
$$h^{B}=\mathrm{Hom}(-,B)\colon C^{\mathrm{op}}\to\mathbf{Set}$$
כאשר על אובייקטים פעול על ידי \(X\mapsto \mathcal{C}(X,B)\) ועל מורפיזמים עבור מורפיזם \(h:X\to Y\) מפועל על ידי:
$$\mathrm{Hom}(h,B)\colon{\mathcal{C}}(Y,B)\to{\mathcal{C}}(X,B),\quad g\mapsto g\circ h$$

\end{definition}
\begin{definition}[פונקטור מיוצג]
כל פונקטור אשר קיים איזומומרפיזם טבעי ל-\(h_{A}\) נקרא מיוצג על ידי האובייקט \(A\).

\end{definition}
\begin{definition}[פונקטור קו מיוצג]
כל פונקטור אשר קיים איזומורפיזם טבעי ל-\(h^{A}\) נקרא קו מיוצג על ידי אובייקט \(A\).

\end{definition}
\begin{example}
נסתכל על הפנקטור הקונטרה וריאנטי \(P:\mathsf{Sets}^{\mathrm{op}}\to \mathsf{Sets}\) כך שעל אובייקטים פועל על ידי קבוצת החזקה ועל מורפיזמים שולח את \(f:X\to Y\) ל:
$$P(f):{\mathcal{P}}(Y)\to{\mathcal{P}}(X),\quad S\mapsto f^{-1}(S)$$
נטען כי זה פונקטור מיוצג, כלומר קיים \(A\) ואיזומורפיזם טבעי \(\mathcal{P}(-)\cong \mathrm{Hom}(-,A)\). נבחר את \(A=\{ 0,1 \}\). נגדיר את האיזומורפיזם:
$$\Phi_{X}\colon\operatorname{Hom}(X,A)\longrightarrow{\mathcal{P}}(X)\qquad \Phi_{X}(f)=f^{-1}(\{1\})$$
כלומר כל פונקציה \(f:X\to \{ 0,1 \}\) בוחרת תת קבוצה של \(X\). מתת קבוצה \(S\subseteq X\) ניתן לשחזר את \(f\) על ידי:
$$f=\begin{cases}1 & x \in S \\0 & x \not \in S
\end{cases}$$
זה משרה איזומורפיזם טבעי ב-\(X\):
$${\mathrm{Hom}}(X,A)\cong{\mathcal{P}}(X)$$
ולכן \(P\) מיוצג על ידי \(\left( \{ 0,1 \},\{ 1 \} \right)\)

\end{example}
\begin{proposition}
אם יש שתי יצוגים לאותו פונקטור אז הם איזומורפיים.

\end{proposition}
\section{הלמה של יונדה}

\begin{reminder}
הסימון \(\mathsf{Set}^{\mathcal{C}}\) מייצג את הקטגוריות הפונקטורים מ-\(\mathcal{C}\) ל-\(\mathsf{Set}\).

\end{reminder}
\begin{definition}[פונקטור ההעתקות הטבעיות]
$$\mathrm{{Nat}}(h_{A},F)\equiv\mathrm{{Hom}}(\mathrm{{Hom}}(A,-),F)$$
כאשר זהו פונקטור מ-\(\mathcal{C}\times \mathsf{Set}^{\mathcal{C}}\to\mathsf{Set}\).

\end{definition}
\begin{proposition}[הלמה של יונדה]
יהי \(F\) פונקטור מקטוריה קטנה מקומית \(\mathcal{C}\) ל-\(\mathsf{Sets}\). אזי קיים האיזומורפיזם הטבעי:
$$\mathrm{Nat}(h_{A},F)\cong F(A)$$

\end{proposition}
אזי לכל אובייקט \(A \in \mathrm{Ob}\left( \mathcal{C} \right)\) ההעקתקות הטבעיות  בהתאמה חח"ע ועל עם האיברים של \(F(A)\).

\begin{proposition}[הלמה של יונדה - גרסה קונטרא וריאנטית]
יהי \(\mathcal{C}\) קטגוריה קטנה מקומית. נסתכל על הקו יצוג \(h^{A}:=\mathrm{Hom}_{C}(-,A):{\mathcal C}^{\mathrm{op}}\to\mathbf{Set}\). יהי \(F: \mathcal{C}^{\mathrm{op}} \to \mathbf{Set}\) פונקטור. אזי יש איזומורפיזם טבעי:
$$\mathrm{Nat}(h^{A},F)\cong F(A)$$

\end{proposition}
\section{שיכון יונדה}

\begin{definition}[שיכון יונדה]
יהי \(\mathcal{C}\) קטגוריה קטנה מקומית. שיכון יונדה הקו וריאנטי ומגדר על ידי:
$$y:C\longrightarrow\left[ C^{\mathrm{op}},\mathbf{Set} \right]\qquad y(A):=h^{A}:=\operatorname{Hom}_{C}(-,A)$$
כאשר שיכון יונדה הקונטרא וריאטנטי מוגדר על ידי:
$$y:{\mathcal{C}}^{\mathrm{op}}\longrightarrow\left[ {\mathcal{C}},\mathbf{Set} \right]\qquad y(A):=h_{A}:=\operatorname{Hom}_{C}(A,-)$$

\end{definition}
\begin{proposition}
שיכון יונדה הוא שיכון, כלומר נאמן ומלא. 

\end{proposition}
\begin{proof}
נשים לב כי מהלמה של יונדה
$$\forall A,B\in{\mathcal{C}},\quad\mathrm{{Nat}}(h^{A},h^{B})\cong\mathrm{Hom}_{{\mathcal{C}}}(A,B)$$
באופן טבעי וזה בדיוק הדרישה שיהיה שיכון - נאמן ומלא.

\end{proof}
\chapter{פונקטורים צמודים}

\section{הגדרה של פונקטורים צמודים}

\begin{definition}[צמוד]
פונקטור \(F: \mathcal{C} \to \mathcal{D}\) נקרא צמדו שמאלי לפונקטור \(G: \mathcal{D} \to \mathcal{C}\) (מסומן \(F \dashv G\)) אם לכל אובייקט \(A \in \mathrm{Ob}(\mathcal{C})\) ולכל אובייקט \(B \in \mathrm{Ob}(\mathcal{D})\) קיים איזומורפיזם טבעי:
$$\mathrm{Hom}_{\mathcal{D} }(F(A),B)\cong  \mathrm{Hom}_{\mathcal{C} }(A,G(B))$$

\end{definition}
\begin{symbolize}
נסמן את האיזומורפיזם בין \(A\in \mathrm{Ob}_{\mathcal{C}}\) ל-\(B\in \mathrm{Ob}_{\mathcal{D}}\) ב-\(\Phi_{AB}\). כלומר:
$$(F(A)\xrightarrow{f}B)\mapsto(A\xrightarrow{\Phi_{A,B}(f)}G(B))$$
כאשר באופן דומה:
$$(A\xrightarrow{g} G(B))\mapsto(F(A)\xrightarrow{\Phi_{A,B}(g)}B)$$
לעיתים כותבים את זה בעזרת קו עליון וקוראים לזה לשחלוף - כלומר אם \(f:F(A)\to B\) אז \(\overline{f}=\Phi_{AB}(f):A\to G(B)\) ההעתקה המתאימה.

\end{symbolize}
\includegraphics[width=0.8\textwidth]{diagrams/svg_3.svg}
\section{יחידה וקו יחידה}

\begin{definition}[יחידה וקו יחידה]
היחידה \(\eta:\mathrm{Id}_{\mathcal{C}}\to G \circ F\) היא איזומורפיזם טבעי המוגדרת על ידי:
$$\eta_{A}:=\Phi_{A,F(A)}(\mathrm{id}_{F(A)})\in\mathrm{Hom}_{C}(A,G(F(A)))$$
כאשר הקו יחידה מוגדרת על ידי:
$$\varepsilon_{B}:=\Phi_{G(B),B}^{-1}(\mathrm{id}_{G(B)})\in\mathrm{Hom}_{\mathcal{D}}(F(G(B)),B)$$

\end{definition}
\begin{remark}
אנו יודעים כי \(\mathrm{Id}_{F(A)}\in\mathrm{Hom}_{\mathcal{D}}(F(A),F(A))\) ולכן \(\Phi_{A,F(A)}\left( \mathrm{Id}_{F(A)} \right)\in \mathrm{Hom}_{\mathcal{C}}\left( A,G\circ F(A) \right)\). כלומר זהו העתקה \(\eta:\mathrm{Id}_{A}\to GF\)

\end{remark}
\begin{proposition}[זהויות זיג זג]
מתקיים זהות המשולש השמאלית:
$$\varepsilon_{F(A)}\circ  F(\eta_{A})=\mathrm{Id}_{F(A)}\qquad G(\varepsilon_{B})\circ \eta_{G(B)}=\mathrm{Id}_{G(B)}$$

\end{proposition}
\begin{proof}
  \begin{enumerate}
    \item זהות המשולש השמאלית - נתחיל מההגדרה של \(\eta\): 
$$\eta_{A}=\Phi_{A,F(A)}(\mathrm{id}_{F(A)})$$
נפעיל כעת את \(\Phi ^{-1}_{A,F(A)}\) על שתי האגפים:
$$\Phi_{A,F(A)}^{-1}(\eta_{A})=\mathrm{id}_{F(A)}$$
ומההגדרה של \(\Phi ^{-1}_{A,F(A)}\) נקבל:
$$\Phi_{A,F(A)}^{-1}(\eta_{A})=\varepsilon_{F(A)}\circ F(\eta_{A})$$
ולכן:
$$\varepsilon_{F(A)}\circ F(\eta_{A})=\operatorname{id}_{F(A)}$$


    \item זהות המשולש הימינית - ההוכחה די זהה לההוכחה של הזהות השמאלית. נתחיל מההגדרה של \(\varepsilon_{B}\): 
$$\varepsilon_{B}=\Phi_{G(B),B}^{-1}(\mathrm{id}_{G(B)})$$
נפעיל את \(\Phi_{G(B),B}\) על שתי האגפים:
$$\Phi_{G(B),B}(\varepsilon_{B})=\mathrm{id}_{G(B)}$$
מההגדרה של \(\Phi\) נקבל:
$$\Phi_{G(B),B}(\varepsilon_{B})=G(\varepsilon_{B})\circ\eta_{G(B)}$$
ולכן:
$$G(\varepsilon_{B})\circ\eta_{G(B)}=\mathrm{id}_{G(B)}$$


  \end{enumerate}
\end{proof}
\includegraphics[width=0.8\textwidth]{diagrams/svg_4.svg}
\begin{example}
אם \(F:\mathcal{C}\to \mathcal{D}\) פונקטור מלא ונאמן, וצמוד לאיזשהו \(G:\mathcal{D}\to \mathcal{C}\), אזי קיים קו יחידה:
$$\varepsilon:F\circ  G\to\mathrm{Id}_{\mathcal{C} }$$
כאשר נזכור כי זהו איזומורפיזם טבעי. לכן כל הרכיבים שלו איזומורפיזמים(מורפיזמים הפיכים). בפרט \(\varepsilon_{B}\) איזומורפיזם. לכן מקיים:
$$\left( F\circ G(B) \right)\xrightarrow{\varepsilon_{B}} B$$
באופן איזומורפי ולכן \(F(G(B))\cong B\) ומהותית על ולכן \(\mathcal{C}\) ו-\(\mathcal{D}\) שקולים.

\end{example}
\end{document}