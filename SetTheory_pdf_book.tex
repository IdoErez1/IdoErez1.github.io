\documentclass{tstextbook}

\usepackage{amsmath}
\usepackage{amssymb}
\usepackage{graphicx}
\usepackage{hyperref}
\usepackage{xcolor}

\begin{document}

\title{Example Document}
\author{HTML2LaTeX Converter}
\maketitle

\section{תורת הקבוצות הנאייבית}

\subsection{מבוא}

בחלק הזה עוד לא נגדיר במפורש מה זה קבוצה, אבל נסתכל על הדברים שניתן לעשות כשיש לנו קבוצה. 

\begin{definition}[קבוצת חזקה]
אם \(A\) קבוצה אז \(\mathcal{P}(A)\) זה אוסף כל התתי הקבוצות של \(A\).

\end{definition}
\begin{remark}
כמו שנראה בהמשך, העובדה שקבוצת החזקה של קבוצה היא אכן קבוצה, זה אקסיומה אשר נקראת אקסיומת קבוצת החזקה.

\end{remark}
\begin{proposition}
אם \(A\) קבוצה סופית כך ש-\(|A|=n\) אז \(\mathcal{P}(A)=2^{n}\).

\end{proposition}
\begin{remark}
כשנתעסק עם עוצמות, נראה כי טענה זו תהיה נכונה גם במקרה הלא סופי.

\end{remark}
\begin{definition}[הקבוצה הריקה]
הקבוצה אשר אינה מכילה איברים. מסומנת ב-\(\varnothing\).

\end{definition}
\begin{remark}
כמו קבוצת החזקה, נראה בהמשך שהקיום של הקבוצה הריקה היא גם כן אקסיומה, אשר נקראת אקסיומת הקבוצה הריקה.

\end{remark}
\begin{definition}[זוג סדור]
זוג של שתי מספרים כך שיש חשיבות לסדר. מסומן \(\langle x,y \rangle\) כאשר בגלל החלק הסדור מתקיים \(\langle x, y \rangle\neq \langle y, x \rangle\).

\end{definition}
\begin{definition}[המכפלה הקרטזית]
מכפלה קרטזית של קבוצה \(A,B\) היא הקבוצה:
$$A\times B = \left\{  \langle a, b \rangle \mid a\in A\quad b \in B  \right\}$$

\end{definition}
\begin{definition}[יחס]
יחס בין \(A\) ל-\(B\) הוא תת קבוצה \(R\subseteq A\times B\).

\end{definition}
\begin{symbolize}
בהנתן יחס \(R\) על \(A\) עבור \(x,y \in A\) מתקיים \(\langle x, y \rangle\in R\) נסמן \(xRy\), כאשר נסמן \(\langle x,y \rangle\notin R\) ע"י \(x\not Ry\).

\end{symbolize}
\begin{definition}[יחס רפלקסיבי]
יחס \(R\) על \(A\) יקרא רפלקסיבי אם לכל \(x \in A\) מתקיים \(xRx\)

\end{definition}
\begin{definition}[יחס סימטרי]
אם לכל \(x,y \in A\) מתקיים אם \(xRy\) אז \(yRx\).

\end{definition}
\begin{definition}[יחס טרנזיטיבי]
אם לכל \(x,y,z \in A\) אם \(xRy\) ו-\(yRz\) אז מתקיים \(xRz\).

\end{definition}
\begin{definition}[יחס שקילות]
יחס אשר רפלקסיבי, סימטרי וטרנזיטיבי נקרא יחס שקילות

\end{definition}
\begin{definition}[מחלקת שקילות]
בהנתן יחס שקילות \(E\) על קבוצה \(A\), ניתן להגדיר את אוסף כל האיברים אשר נמצאים ביחס שקילות אחד עם השני בקבוצה:
$$[a]_{E}=\{b\in A\mid a E b\}$$

\end{definition}
\begin{definition}[אוסף מחלקות השקילות]
בהנתן קבוצה \(A\) ויחס \(E\) ניתן להגדיר קבוצה \(A / E\) של כל המחלקות השקילות של האיברים ב-\(A\) תחת היחס \(E\). כלומר:
$$A/E=\{[a]_{E}\mid a\in A\}$$

\end{definition}
\begin{remark}
סימון זה מאוד מזכיר קוסטים(מחלקות) מתורת החבורות, ואכן קוסטים זה מקרה פרטי של אוסף מחלקות השקילות, אך במקרה זה הגדלים של כל אחד מהמחלקות שקילות הם לא בהכרח שווים.

\end{remark}
\begin{definition}[חלוקה]
אם \(A\) קבוצה לא ריקה, \(P\subseteq \mathcal{P}(A)\) תקרא חלוקה אם \(\varnothing \not\in P\), איברי \(P\) זרים בזוגות(כלומר אם \(x,y \in P\)) אז \(x \cap y = \varnothing\). ו-\(\cup_{X \in P} X=A\). כלומר אוסף של תתי קבוצות זרות אשר האיחוד שלהם זה הקבוצה כולה.

\end{definition}
\begin{corollary}
אוסף מחלקות השקילות היא למעשה חלוקה של הקבוצה \(A\).

\end{corollary}
\begin{remark}
כמו שיחס שקילות יוצר חלוקה, חלוקה מגדירה יחס שקילות.

\end{remark}
\begin{definition}[פעולה על מחלקת שקילות]
בהנתן פעולה(יחס) \(*\) על \(X\) אז \(*\) משרה פעולה מוגדרת היטב על \(X / E\) אם \(*\) מקיימת את הטענה הבאה:
$$\forall (x_{1},x_{2})\in E\quad \forall(y_{1},y_{2})\in E\quad (x_{1}*y_{1},x_{2}*y_{2})\in E$$
כלומר קיים אי תלות של \(E\) ביחס לפעולת הכפל.

\end{definition}
\subsection{פונקציות}

\begin{definition}[פונקציה]
פונקציה \(F:A\to B\) הוא יחס \(F\subseteq A\times B\) המקיים כי:
$$\forall a \in A\quad \exists! b\in B\quad \langle a, b \rangle \in F$$
כאשר נזכור כי \(\exists!\) אומר קיים ויחיד.

\end{definition}
\begin{definition}[ערך של פונקציה]
בהנתן \(F:A\to B\), נסמן לכל \(a \in A\) את \(F(a)\) להיות ה-\(b \in B\) היחיד עבורו \(\langle a, b \rangle\in F\)

\end{definition}
\begin{example}
נגדיר:
$$A=\{ 0,1 \}\quad B=\left\{  3,\pi  \right\}$$
ונגדיר את היחס \(R_{1}=\left\{  \langle 0, 3 \rangle   \right\}\)
זהו אינו פונקציה מ-\(A,B\) כיוון שלא קיים \(b \in B\) כך ש-\(\langle 1, b \rangle\in R_{1}\).
כעת נגדיר את היחס:\(R_{2}=\left\{  \left\langle  0, \pi  \right\rangle ,\left\langle  1,\pi   \right\rangle   \right\}\). הפעם \(R_{2}\) מקיימת את תנאי ההגדרה של הפונקציה מ-\(A\) ל-\(B\) ולכן נוכל לרשום \(R_{2}:A\to B\).

\end{example}
\begin{definition}[פונקציית הזהות]
לכל קבוצה \(X\) נסמן:
$$id_{X}=\left\{  \langle a,a \rangle \mid a\in X  \right\}$$
מתקיים:
$$id_{X}:X\to X$$

\end{definition}
\begin{definition}[תחום]
יהי \(R\subseteq A\times B\) יחס. נגדיר את התחום של \(R\):
$$\operatorname{dom}\left(R\right)=\left\{a\in A\mid\exists b\in B\quad\left\langle a,b\right\rangle\in R\right\}$$

\end{definition}
\begin{definition}[טווח]
יהי \(R\subseteq A\times B\) יחס. נגדיר את הטווח של \(R\):
$$\operatorname{rng}\left(R\right)=\left\{b\in B\mid\exists a\in A\quad\langle a,b\rangle\in R\right\}$$

\end{definition}
\begin{remark}
אם \(R\subseteq A\times B\) היא פונקציה מ-\(A\) ל-\(B\), אז \(\mathrm{dom}(R)=A\) ו-\(\mathrm{rng}(R)\subseteq B\).

\end{remark}
\begin{definition}[חח"ע]
פונקציה \(F:A\to B\) היא חח"ע אם לכל \(a_{1}\neq a_{2}\) אברי \(A\) מתקיים \(F(a_{1})=F(a_{2})\).

\end{definition}
\begin{corollary}
פונקציה \(F:A\to B\) תהיה חח"ע אם \(F(x)=F(y)\) גורר \(x=y\).

\end{corollary}
\begin{definition}[על]
פונקציה \(F:A\to B\) היא על \(B\) אם \(\mathrm{rng}(F)=B\).

\end{definition}
\begin{remark}
נניח כי המספרים הטבעיים מתחילים מ-0.

\end{remark}
\begin{definition}[היחס ההופכי]
בהנתן יחס \(R\subseteq A\times B\) נגדיר את היחס ההופכי \(R^{-1}\subseteq B\times A\) להיות:
$$R^{-1}=\{\langle b,a\rangle\mid\langle a,b\rangle\in R\}$$

\end{definition}
\begin{definition}[פונקציה הופכית]
פונקציה \(F:A\to B\) מ-\(A\) ל-\(B\) נקראת הפיכה אם היחס ההופכי \(F^{-1}\) הוא פונקציה מ-\(B\) ל-\(A\). ובמקרה זה אפשר לרשום \(F^{-1}:B\to A\).

\end{definition}
\begin{proposition}
פונקציה \(F:A\to B\) היא הפיכה אם"ם היא חח"ע ועל \(B\).

\end{proposition}
\begin{corollary}
אם \(F:A\to B\) היא חח"ע ועל אז \(F^{-1}:B\to A\) גם חח"ע ועל.

\end{corollary}
\begin{proof}
נתון \(F:A\to B\) חח"ע ועל \(B\). לכן \(F\) הפיכה. כלומר \(F^{-1}:B\to A\) פונקציה. נשים לב כי \((F^{-1})^{-1}=F\) היא פונקציה, ולכן \(F^{-1}\) היא הפיכה. 

\end{proof}
\begin{definition}[הרכבת יחסים]
נניח \(A,B,C\) קבוצות, \(R\subseteq A\times B\) ו-\(S\subseteq B\times C\). נגדיר יחס \(S\circ R\subseteq A\times C\) ע"י:
$$S\circ R = \left\{  \langle a, c \rangle \mid a\in A\quad c\in C \quad \exists b \in B\quad \langle a, b \rangle \in R\quad \langle b, c \rangle \in S \right\}$$

\end{definition}
\begin{proposition}
אם \(F:A\to B, G:B\to C\) פונקציות אז היחס \(G\circ F\subseteq A\times C\) הוא גם כן פונקציה.

\end{proposition}
\begin{proposition}
בנינתן \(F:A\to B\) ו-\(G:B\to C\) מתקיים:

  \begin{enumerate}
    \item אם \(F,G\) חח"ע אז ההרכבה \(G\circ F\) חח"ע 


    \item אם \(F\) על \(B\) ו-\(G\) על \(C\) אז \(G\circ F\) על C 


    \item אם \(F,G\) הפיכות, אז \(G\circ F\) הפיכה. 


    \item אם \(F\) הפיכה, אז \(id_{A}=F^{-1}\circ F\quad id_{B}=F\circ F^{-1}\)


    \item אם \(G \circ F\) חח"ע אז \(F\) חח"ע. 


  \end{enumerate}
\end{proposition}
\begin{proof}
  \begin{enumerate}
    \item יהיו \(x,y \in A\) כך ש-\(G \circ F(x)=G \circ F(y)\). לכן \(G(F(x))=G(F(y))\) מכל ש-\(G\) חח"ע נובע \(F(x)=F(y)\) וכן מכך ש-\(F\) חח"ע נקבל \(x=y\). 


    \item אם \(F(x)=F(y)\) נקבל \(G(F(x))=G(F(y))\) כיוון ש-\(G\) פונקציה, ולכן מחח"ע נקבל \(x=y\) ולכן \(F\) חח"ע. 


  \end{enumerate}
\end{proof}
\begin{proposition}
פונקציה \(F:A\to B\) היא הפיכה אם"ם \(\exists G:B\to A\) כך ש:
$$F\circ G = id_{B}\quad G\circ  F = id_{A}$$

\end{proposition}
\subsection{יחסי סדר}

\begin{definition}[יחס סדר חלקי]
יחס \(\leq\) על קבוצה \(S\) יקרא יחס סדר חלקי אם מקיים:

  \begin{enumerate}
    \item רפלקיביות - לכל \(a \in S\) מתקיים \(a \leq a\). 


    \item אנטי סימטרייה - לכל \(a,b \in S\) אם \(a\leq b\) וגם \(b\leq a\) אז \(a=b\).  


    \item טרנזטיביות - לכל \(a,b,c \in S\) אם \(a\leq b\) וגם \(b\leq c\) אז \(a\leq c\). 


  \end{enumerate}
\end{definition}
\begin{definition}[יחס סדר חלקי חד]
יחס > על קבוצה \(S\) יקרא סדר חלקי חד אם מקיים:

  \begin{enumerate}
    \item אירפלקסיבי - לכל \(a \in S\) לא מתקיים \(a< a\). 


    \item אסימטרי - לכל \(a,b \in S\) אם \(a< b\) אז לא מתקיים \(b < a\). 


    \item טרנזטיבי - לכל \(a,b,c\) אם \(a<b\) וגם \(b<c\) אז \(a< c\). 


  \end{enumerate}
\end{definition}
\begin{example}
  \begin{enumerate}
    \item אם \(\left\langle  \mathbb{N},\leq  \right\rangle\) כאשר \(\leq\) זה היחס סדר הרגיל(קס"ח). כאשר עם \(<\)  זה סדר חד. 


    \item לכל קבוצה \(A\), \(\left\langle  \mathcal{P}(A),\subseteq  \right\rangle\) זה סדר חלקי. 


  \end{enumerate}
\end{example}
\begin{remark}
  \begin{enumerate}
    \item אם \(R\) טרנזיטיבי אז \(R\) הוא אסימטרי אם"ם הוא אירפלקסיבי. 


    \item אם \(\langle A,R \rangle\) סדר חלקי, נגדיר את יחס \(R'\) ע"י: 
$$a\neq b\land aRb \iff a R' b$$
אז \(\langle A,R' \rangle\) זה סדר חלקי חד, ובאופן דומה, אם \(\langle A,Q \rangle\) סדר חד אז עבור היחס \(Q^*\) המוגדר:
$$aQ^*b \iff a=b\lor aQb$$


  \end{enumerate}
\end{remark}
\begin{definition}[יחס מלא/לינארי/קווי]
עבור קבוצה סדורה חלקית \(\left\langle   A,\leq  \right\rangle\) היחס \(\leq\) יקרא סדר קווי/מלא/לינארי אם לכל \(a,b \in A\) מתקיים:
$$a\leq b \lor b\leq a$$

\end{definition}
כלומר כל יחס סדר מלא הוא יחס סדר חלקי, אבל לא כל יחס סדר חלקי הוא יחס סדר מלא.

\begin{definition}
תהי \(\left\langle  A,\leq  \right\rangle\) קפ"ח, \(a\in A\) יקרא:

  \begin{enumerate}
    \item מזערי/מינימלי אם: 
$$\forall b \in A\qquad  b\leq a\implies b=a$$


    \item מירבי/מקסימלי אם: 
$$\forall b \in A\qquad b\geq a\implies b=a$$


    \item מינימום אם: 
$$\forall b \in A\qquad a\leq b$$


    \item מקסימום אם: 
$$\forall b \in A\qquad  b \leq a$$


  \end{enumerate}
\end{definition}
\begin{proposition}[יחידות המינימום]
אם \(a \in A\) מינימום, אז הוא מינימלי והוא המינימלי היחיד.

\end{proposition}
כאשר נשים לב כי הכיוון השני לא נכון בהכרח. כלומר ייתכן סדר חלקי עם מינימלי יחיד אך ללא מינימום. נראה דוגמא:

\begin{example}
נסתכל על \(A=\mathbb{Z}\cup \{ a \}\) כאשר על השלמים נשתמש ביחס הידוע \(\leq\), ועל \(a\) לא נגדיר יחס. נשים לב כי \(\{ a \}\) המינימלי היחיד כיוון שבאופן ריק מתקיים:
$$\forall c \in A\quad a\leq c\implies a=c$$
כאשר עבור השלמים, כידוע אין איבר מינימלי. וכן אין גם מקסימום כי לא מתקיים:
$$\forall b \in A\qquad a\leq b$$
עבור אף איבר.

\end{example}
\begin{proposition}
עבור יחס סדר מלא, נקבל כי איבר הוא מינימום אם"ם הוא מינימלי.

\end{proposition}
\begin{definition}[קבוצה מבוססת היטב]
תהי \(\left\langle  A,\leq  \right\rangle\) קבוצה סדורה חלקית. נאמר ש-\(A\) מבוססת היטב עם היחס \(\leq\) אם לכל תת קבוצה לא ריקה \(\varnothing \neq B \subseteq A\) יש איבר מינימלי ב-\(B\).

\end{definition}
\begin{definition}[קבוצה סדורה היטב/מקיימת סדר טוב]
קבוצה שהיא גם יחס מלא וגם מבוססת היטב.

\end{definition}
\begin{definition}[שיכון של סדרים]
אם \(\left\langle  A,\leq_{A}  \right\rangle\) ו-\(\left\langle  B,\leq_{B}  \right\rangle\) הם קבוצות סדרות חלקית, אז שיכון הוא פונקציה \(F:A\to B\) כך ש:
$$F(a)\leq_{A}F(a')\iff a \leq_{A}a'$$

\end{definition}
\begin{proposition}
שיכון של סדרים הוא חח"ע.

\end{proposition}
\begin{definition}[איזומורפיזם של סדרים]
שיכון של סדרים שהוא גם על.

\end{definition}
\begin{proposition}
בהנתן שתי קבוצות סדרות היטב \((X,<_{X})\) ו-\((Y,<_{Y})\) ניתן לשכן את \(X\) בתחילית של \(Y\) או לשכן את \(Y\) בתחילית של \(X\), כאשר שיכון כזה הוא יחיד.

\end{proposition}
\subsection{בנייה של קבוצות ידועות}

\begin{definition}[המספרים השלמים]
נגדיר את המספרים השלמים כך שזוג \(\langle n,m \rangle\) מייצג את המספר \(n-m\) תחת יחס השקילות:
$$\langle n,m\rangle\sim_{\mathbb{Z}}\langle k,l\rangle\iff n+l=m+k$$
כלומר השלמים יהיו הקבוצה \(.(\mathbb{N}\times\mathbb{N})\,/\sim_{\mathbb{Z}}\)

\end{definition}
\begin{definition}[המספר אלגברי]
מספר \(a \in \mathbb{R}\) יקרא ממשי אלגבי אם קיים פולינום במקדמים רציונאליים \(P \in \mathbb{Q}[x] \setminus \{  0 \}\) כך ש-\(P(a)=0\)

\end{definition}
\begin{proposition}
קבוצת הממשיים האלגבריים היא בת מנייה.

\end{proposition}
\begin{proof}
נסמן ב-\(A\) את קבוצת הממשיים אלגבריים. נזהה את \(Q[x]\) עם קבוצת הסדרות הרציונאליות הסופיות, אשר תת קבוצה סופית של הרציונאליים, ולכן בת מנייה. ולכן קיים \(f:\mathbb{Q}[x]\to \mathbb{N}\) חח"ע ועל.
כעת נגדיר \(g:\mathrm{Alg}_{\mathbb{R}}\to \mathbb{Q}[x] \times \mathbb{N}\) באופן הבא:
$$g(a)=\langle P,n \rangle \iff P\neq 0\land f(P)=\min\left\{  f(Q)\mid Q(a)=0  \right\}$$
ו-\(a\) הוא השורש ה-\(n\) של \(P\) לפי הסדר הרגיל של \(\mathbb{R}\). נשים לב כי \(g\) חח"ע, כי אם \(g(a)=g(b)=\langle P,n \rangle\) נקבל כי גם \(a\) וגם \(b\) הם השורש ה-\(n\) של \(P\), ולכן שווים - כלומר \(a=b\). לכן נקבל כי \(|A|\leq |\mathbb{Q}[x]\times \mathbb{N}|=|\mathbb{N}|\). כאשר מטענה שראינו קבוצה כזו היא או סופית או בת מנייה, כיוון שלא סופית(מכך ש-\(x^{n}-2\) נותן לנו אינסוף שורשים למשל), נקבל כי בת מנייה. 

\end{proof}
\begin{definition}[מספרים רציונאליים]
נגדיר יחס שקילות:
$$(a,b)\sim (c,d)\iff a\cdot d=b\cdot c$$
כעת נגדיר את המספרים הרציונאלים ע"י אוסף מחלקות השקילות:
$$\mathbb{Q} =\mathbb{Z} \times \left( \mathbb{Z} \setminus  \{  0 \} \right) / \sim$$

\end{definition}
כעת נרצה להגדיר את המספרים הממשים, לשם כך נגדיר ראשית מספר מושגים:

\begin{definition}[חתכי דדקינד]
תת קבוצה \(\alpha \subset \mathbb{Q}\) עם התכונות הבאות:

  \begin{enumerate}
    \item לא ריקה ותת קבוצה ממש - \(\alpha \neq \varnothing,\alpha \neq \mathbb{Q}\). 


    \item סגורה כלפי מטה - אם \(a \in \alpha, y \in \mathbb{Q}\) ו-\(b\leq a\) אז \(b \in \alpha\). 


    \item אין איבר מקסימלי - אם \(a \in \alpha\) אז קיים \(b \in \alpha\) כך ש- \(a < b\). 
כלומר חתך דדקינד זה סוג של קרן פתוחה שמאלית ב-\(\mathbb{Q}\).


  \end{enumerate}
\end{definition}
\begin{definition}[המספרים הממשיים]
קבוצת כל חתכי הדדקינד

\end{definition}
\subsection{עוצמות}

כאשר נתונות לנו שתי קבוצות נרצה להשוואות את ה-"גדלים" שלהם.

\begin{definition}[קבוצות שוות עוצמה]
זוג קבוצות \(A,B\) נקראות שוות עוצמה אם קיימת פונקציה חח"ע ועל(כלומר הפיכה) \(f:A\to B\). במקרה זה נסמן \(|A|=|B|\).

\end{definition}
\begin{proposition}
שיוויון עוצמות הוא יחס שקילות.

\end{proposition}
\begin{proof}
בין קבוצה לעצמה קיימת פונקציה חח"ע ועל - פונקציית הזהות - ולכן רפלקסיבי. אם \(|A|=|B|\) אז קיימת פונקציה הפיכה \(f:A\to B\) ולכן \(f^{-1}:B\to A\) גם כן פונקציה הפיכה ומתקיים \(|B|=|A|\). בנוסף הרכבה של פונקציות הפיכות היא הפיכה ולכן טרנזיטיבי.

\end{proof}
\begin{definition}[יחס סדר של שיוויון עוצמות]
נאמר כי עוצמת קבוצה \(B\) גדולה או שווה לעוצמה של קבוצה \(A\) אם קיימת פונקציה חח"ע \(f:A\to B\). במקרה זה נסמן \(|A|\leq |B|\). אם בנוסף מתקיים \(|A| \neq |B|\)(כלומר לא שוות עוצמה), נאמר כי העוצמה של \(B\) גדולה ממש מהעוצמה של \(A\) ונסמן \(|A|<|B|\).

\end{definition}
\begin{remark}
הגדרה חלופית עבור \(|A|<|B|\) זה אם קיים פונקציה חח"ע \(f:A\to B\) אך לא קיימת פונקצייה על.

\end{remark}
\begin{symbolize}
עבור \(n\) טבעי נסמן את קבוצת כל האיברים הטבעיים עד המספר ע"י:
$$[n]=\left\{  0,\dots, n-1  \right\}$$

\end{symbolize}
\begin{definition}[קבוצה סופית]
קבוצה אשר שוות עוצמה לקבוצה \([n]\) עבור \(n\) טבעי כלשהו. כלומר קבוצה \(A\) המקיימת \(|A|=|[n]|\).  במקרה זה נאמר כי העוצמה של הקבוצה תהיה \(n\).

\end{definition}
\begin{proposition}[קבוצה בת מנייה]
קבוצה \(A\) תקרא בת מנייה אם שוות עוצמה לטבעיים, כלומר \(|A|=|\mathbb{N}|\), וקיים פונקצייה חח"ע בין הטבעיים לקבוצה.

\end{proposition}
\begin{proposition}
אם \(|A|\leq |\mathbb{N}|\) אז \(A\) סופית או בת מנייה. כלומר למעשה קבוצות בנות מנייה הם הקבוצות האינסופיות "הקטנות ביותר".

\end{proposition}
\begin{proof}
מספיק להראות כי \(A\subseteq \mathbb{N}\) אז \(A\) סופית או בת מנייה. נגדיר פונקציה \(g:A\to \mathbb{N}\) ע"י \(g(x)=|A\cap [x]|\). 
ראשית נשים לב כי \(g\) מונוטונית עולה ממש ולכן חח"ע. אם \(\mathrm{rng}(g)=\mathbb{N}\) סיימנו, והקבוצה בת מנייה. אחרת נניח \(\mathrm{rng}(g)\neq \mathbb{N}\). נסמן \(N \in \mathbb{N}\) בתור האיבר הטבעי המינימלי כך ש-\(N \notin \mathrm{rng}(g)\). יהיו \(a_{0},\dots,a_{N-1}\) איברים כך ש-\(g(a_{i})=i\). אם \(A=\left\{  a_{0},\dots,a_{N-1}  \right\}\) נקבל \(|A|=N\) ולכן סופית. אחרת נניח \(A \setminus \left\{   a_{0},\dots, a_{N - 1}  \right\}\) אינה ריקה, ונראה כי מצב זה לא אפשרי. יהי \(b \in A \setminus \left\{   a_{0},\dots,a_{N-1}  \right\}\) האיבר המינימלי בה(קיים בתור תת קבוצה של הטבעיים). 
נראה כעת \(\left\{a_{0},\ldots,a_{N-1}\right\}\subseteq A\cap[b]\) בעזרת הכלה דו כיוונית. מתקיים \(A\cap[b]\subseteq\ \{a_{0},\ldots,a_{N-1}\}\) כי אם קיים איבר \(c\in\left(A\cap[b]\right)\smallsetminus\ \left\{a_{0},\cdot\cdot\cdot,a_{N-1}\right\}\) נקבל \(c< b\) בסתירה למינימליות של \(b\). נראה כעת \(\left\{a_{0},\ldots,a_{N-1}\right\}\subseteq A\cap[b]\), כלומר שלכל \(i<N\) מתקיים \(a_{i}< b\). אחרת יש \(i< N\) כך ש-\(b< a_{i}\). כיוון ש-\(g\) מונוטונית עולה נקבל \(g(b)< g(a_{i})=i\) אבל אז \(g(b)=j=g(a_{j})\) עבור \(j< i\) כלשהו בסתירה לכך ש-\(g\) היא חח"ע. כלומר קיבלנו סה"כ כי:
$$A\cap[b]=\ \left\{ a_{0},\ldots,a_{N-1} \right\} \implies |A\cap[b]|=|\ \left\{ a_{0},\dots,a_{N-1} \right\}|=N$$
ולכן \(g(b)=N\) בסתירה לבחירה של \(N\).

\end{proof}
\begin{definition}[קבוצה בעוצמת הרצף]
קבוצה \(A\) תרא בעוצמת הרצף אם שוות עוצמה לממשיים. 

\end{definition}
\begin{example}
נקבל כי \(|\left( -\frac{\pi}{2},\frac{\pi}{2} \right)|=|\mathbb{R}|\) כיוון שהפונקציה \(F(x):\left( -\frac{\pi}{2},\frac{\pi}{2} \right)\to \mathbb{R}\) המוגדרת ע"י \(F(x)=\tan(x)\) היא הפיכה. וכעת ע"י הזזה וכפל בקבוע נקבל כי כל קטע פתוח \((a,b)\) יהיה בעוצמת הרצף. בנוסף כל קטע החצי סגור או סגור גם כן יהיה בעוצמת הרצף.

\end{example}
\begin{proposition}
  \begin{itemize}
    \item לכל \(n \in \mathbb{N}\setminus \{ 0 \}\) מתקיים \(\left\lvert  \mathbb{R}^n  \right\rvert=\left\lvert  \mathbb{R}  \right\rvert=\left\lvert  \mathbb{R}^\mathbb{N}  \right\rvert\). 
    \item מתקיים \(\left\lvert  \mathbb{R}^\mathbb{R}  \right\rvert=\left\lvert  \mathcal{P}\left( \mathbb{R} \right)  \right\rvert>\left\lvert  \mathbb{R}  \right\rvert\)
  \end{itemize}
\end{proposition}
\begin{theorem}[קנטור שרדר ברנשטיין]
יהיו \(A, B\) קבוצות.  אם \(|A|\leq |B|\) וגם \(|B|\leq |A|\) אז \(|A|=|B|\).

\end{theorem}
נניח בה"כ כי \(A\subseteq B\). כמו כן נניח כי \(f:A\to B\) חח"ע. נגדיר:
$$ A^*=\{x\in A\mid\exists n\in\mathbb{N}\quad\exists a\in A\backslash B\quad x=f^n\left(a\right)\}$$
כאשר הפונקציה ההפיכה המבוקשת תהיה מוגדרת ע"י:
$$h(x)=\begin{cases}f(x) & x \in A^{*} \\x & x \in A \setminus  A^{*}
\end{cases}$$
ראשית נשים לב כי \(A / B \subseteq A^{*}\), כיוון ש-\(f^{0}(a)=a\) לכל \(a\). לכן \(h\) מוגדרת היטב.
כעת נראה ש-\(h\) חח"ע:
יהיו \(x_{1}\neq x_{2} \in A\). מספיק להראות כי \(h(x_{1})\neq h(x_{2})\). נסתכל בנפרד מתי \(x_{1},x_{2} \in A^{*}\), \(x_{1},x_{2} \in A \setminus A^{*}\), וכאשר \(x_{1},\in A^{*},x_{2} \in A\setminus A^{*}\).
אם \(x_{1},x_{2} \in A^{*}\) מתקיים \(x_{1},x_{2}\in A^{*}\) אז מתקיים:
$$h(x_{1})=f(x_{1})\neq f(x_{2})=h(x_{2})$$
כיוון ש-\(f\) חח"ע. אם \(x_{1},x_{2} \in A \setminus A^{*}\) נקבל:
$$h(x_{1})=x_{1}\neq x_{2}=h(x_{2})$$
אם \(x_{1} \in A^{*}\) ו-\(x_{2} \in A\setminus A^{*}\) נשים לב ראשית כי \(f(x_{1})\in A^{*}\) לפי ההגדרה של \(A^{*}\), ולכן:
$$h(x_{1})=f(x_{1})\neq x_{2}=h(x_{2})$$
ולכן מקיים חח"ע. נדרש רק להראות כי על. נבצע כעת חלוקה למקרים \(b \in A^{*}\) ו-\(b \in A\setminus A^{*}\):
אם \(b \in A \setminus A^{*}\) נקבל כי \(b=h(b)\). אם \(b \in A^{*}\) אז מהגדרת \(A^{*}\) נקבל כי קיים \(n\geq 0\) כך ש-\(b=f^{n}(a)\). נשים לב כי בהכרח \(n> 0\), כי אם כי אם \(n=0\) נקבל:
$$b=f^{0}(a)=a \in A\setminus  B$$
אבל לא ייתכן ש-\(b \in A\setminus B\). לכן אפשר להסתכל על \(n-1\geq 0\). נסמן \(a'=f^{n-1}(a)\) ולכן גם \(a' \in A^{*}\subseteq A\) ולכן:
$$h(a')=f(a')=f(f^{n-1}(a))=f^{n}(a)=b$$
והראנו כי על, ולכן קיבלנו פונקציה הפיכה בין שתי הקבוצות.

\begin{proposition}
נסמן ב-\(C\left( \mathbb{R} \right)\) את קבוצת הפונקציות הרציפות אז מתקיים:
$$\left\lvert  C\left( \mathbb{R} \right)  \right\rvert =\left\lvert  \mathbb{R}  \right\rvert $$

\end{proposition}
\begin{proof}
נשתמש בקש"ב(קנטור שרדר ברנשטיין). לכל ממשי הפונקציה הקבועה עם ערך \(a\) היא רציפה ולכן יש לכל הפחות פונקציות רציפות כלומר הממשיים - \(\left\lvert  \mathbb{R}  \right\rvert\leq \left\lvert  C\left( \mathbb{R} \right)  \right\rvert\).
בכיוון השני נגדיר \(F:C\left( \mathbb{R} \right)\to \mathbb{R}^\mathbb{Q}\) על ידי \(F(1)=f\upharpoonright \mathbb{Q}\).
עובדה(מאינפי): אם \(f,g:\mathbb{R}\to \mathbb{R}\) רציפות ומסכימות על הרציונאלים, אז \(f=g\). 
מעובדה, \(F\) חח"ע. ומתקיים:
$$\left\lvert  C\left( \mathbb{R} \right)  \right\rvert \leq \left\lvert  \mathbb{R}^\mathbb{Q}   \right\rvert =\left\lvert  \mathbb{R}^\mathbb{N}  \right\rvert =\lvert \mathbb{R} \rvert $$

\end{proof}
\subsection{חשבון עוצמות}

\begin{definition}[עוצמה - זמנית]
נסתכל על היחס:
$$E=\left\{  (A,B)\mid \exists f:A\to B \text{ invertable}  \right\}$$
ראינו ש-\(E\) יחס שקילות. נגגיר עוצמה כאוסף מחלקות השקולות של \(E\) על \(A\).

\end{definition}
\begin{symbolize}
  \begin{enumerate}
    \item העוצמה \(|\mathbb{N}|\) מסומן ב-\(\aleph_{0}\). 


    \item העוצמה \(|\mathbb{R}|\) מסומן ב-\(\aleph\) או \(\mathfrak{c}\)


    \item באופן כללי משתמשים באותויות גוטיות - לדוגמא \(\mathfrak{a,b,c}\) כדי לסמן עוצמות(מחלקות עוצמה) 


  \end{enumerate}
\end{symbolize}
\begin{definition}[חיבור עוצמות]
בהנתן קבוצות \(A,B\) המקיימות \(|A|=\mathfrak{a},|B|=\mathfrak{b}\) כך ש-\(A\cap B = \varnothing\) נגדיר את חיבור העוצמות ע"י:
$${\mathfrak{a}}+{\mathfrak{b}}=|A\cup B|$$

\end{definition}
\begin{definition}[כפל עוצמות]
בהנתן עוצמות \(\mathfrak{a,b}\) נגדיר \(\mathfrak{a\cdot b}\) באופן הבא:
יהי \(A,B\) קבוצות. \(|A|=\mathfrak{a}\) ו-\(|B|=\mathfrak{b}\). נגדיר \(\mathfrak{a\cdot b}=|A\times B|\).

\end{definition}
\begin{proposition}
יהיו \(\mathfrak{a,b,c}\) עוצמות. אזי:
$$\begin{gather}1.\;\mathfrak{a+(b+c)=(a+b)+c} \qquad 2.\;\mathfrak{a+b=b+a} \\3.\;\mathfrak{a\cdot \left( b\cdot c \right)=\;\left( a\cdot b \right)\cdot c} \qquad 4.\;\mathfrak{a\cdot b=b\cdot c} \\5.\;\mathfrak{a+a}=2\cdot\mathfrak{a}    
\end{gather} $$

\end{proposition}
\begin{proof}
יוהו \(A,B,C\) נציגים זרים. מתקיים:
$$A\cup\left( B\cup C \right)=\left( A\cup B \right)\cup C\qquad A\cup B=B\cup A$$
לכן שתי הטענות הראשונות מתקיימות. נראה את הטענה השלישית. נגדיר:
$$f:A\times\left( B\times C \right)\to \left( A\times B \right)\times C$$
ע"י:
$$f\left( \left\langle  a,\langle b,c \rangle   \right\rangle  \right)=\left\langle  \langle a,b \rangle ,c  \right\rangle $$
ו-\(f\) הפיכה. לכן:
$$\left\lvert  A\times\left( B\times C \right)  \right\rvert =\left\lvert  \left( A\times B \right)\times C  \right\rvert $$
מראים את הטענה הרביעית באופן דומה. נראה את הטענה החמישית:
יהי \(A'\) נציג נוסף של \(\mathfrak{a}\) זר ל-\(A\). תהי \(g:A'\to A\) הפיכה. נגדיר:
$$h:A\cup A' \to A \times[2]\quad h(x)=\begin{cases}\langle x,u \rangle  & x \in A \\\langle g(x),1 \rangle  & x \in A'
\end{cases}$$
אפשר לבדוק ש-\(h\) חח"ע ועל(הפיכה) ולכן סיימנו

\end{proof}
\begin{definition}[חזקת קבוצות]
בהנתן קבוצות \(A,B\) נגדיר את החזקה \(A^{ B}\) בתור אוסף כל הפונקציות \(f:B\to A\). כלומר:
$$A^{B}=\left\{ f\subseteq B\times A\mid f:B\to A\;\;\mathrm{is\;a\;function} \right\}$$

\end{definition}
\begin{definition}[חזקת עוצמות]
בהנתן עוצמות \(\mathfrak{a,b}\) נגדיר \(\mathfrak{a^b}\) באופן הבא:
יהיו \(A,B\) קבוצות כך ש- \(|A|=\mathfrak{a},|B|=\mathfrak{b}\). נגדיר:
$$\mathfrak{a} ^{\mathfrak{b} }=|A^{B}|$$

\end{definition}
\begin{proposition}
העוצמה של חזקות מקיימת את הגדרת החזקה הרגילה, כלומר:
$$|A^{B}|=|A|^{|B|}$$

\end{proposition}
\begin{proposition}
יהי \(A\) קבוצה. העוצמה של \(\mathcal{P}(A)\) תהיה \(|2^{A}|=2^{|A|}\).

\end{proposition}
\begin{proposition}
$$2^{\aleph_{0}}=\aleph_{0}^{\aleph_{0}}$$

\end{proposition}
\begin{proof}
ניקח את \(\mathbb{N}\) להיות הנציג של \(\aleph_{0}\) ואת \(\{ 0,1 \}=[2]\) להיות הנציג של 2.
נשתמש במשפט קנטור שרדר ברנשטיין. ראשית \([2]^\mathbb{N}\subseteq \mathbb{N}^\mathbb{N}\) ולכן \(2^{\aleph_{0}}\leq \aleph_{0}^{\aleph_{0}}\). צריך להראות  \(2^{\aleph_{0}}\geq \aleph_{0}^{\aleph_{0}}\). בהינתן פונקציה 
$$f:\mathbb{N}\to \mathbb{N}\quad G(f):\mathbb{N}\to \mathbb{N}\quad G(f)(k)=\left( \sum_{i=0}^k f(i) \right)+k$$
כעת נגדיר:
$$H:\mathbb{N}^\mathbb{N} \to [2]^\mathbb{N}\qquad (H(f))(n)=\begin{cases}1 &  \exists k\quad n=G(f)(k) \\0 & else
\end{cases}$$
אינטואיטבית, זהי סדרה של מספר בסדרה מייצג את כמות האפסים שתהיה בסדרה שנעבור עלייה, למשל:
$$\left\langle  3,1,5,\dots  \right\rangle \mapsto \left\langle  0,0,0,1,0,1,0,0,0,0,0,1,\dots  \right\rangle$$
תרגיל: לבדוק ש-\(H\) אכן חח"ע. היא לא על, ואין אף \(f:\mathbb{N}^\mathbb{N}\) כך ש:
$$H(f)=\left\langle  0,0,0,0,0,\dots  \right\rangle $$
אבל לא אכפת לנו כי קנטור שרדר ברנשטיין. 

\end{proof}
\begin{remark}
דרך נוספת להראות את האי שיוויון בכיוון השני היא לשים לב כי מתקיים \(\mathbb{N}^\mathbb{N}\subseteq \mathcal{P}\left( \mathbb{N}\times \mathbb{N} \right)\) ולכן:
$$\aleph_{0}^{\aleph_{0}}\leq 2^{\mathbb{N}\times \mathbb{N}}=2^{\aleph_{0}\cdot \aleph_{0}}=2^{\aleph_{0}}$$

\end{remark}
\begin{proposition}
יהיו \(\mathfrak{a,b,c}\) עוצמות, ו-\(\mathfrak{a\leq b}\) אז \(\mathfrak{a^c\leq b^c}\).

\end{proposition}
\begin{proof}
יהיו \(A,B,C\) נציגים בהתאמה. תהי \(i:A\to B\) חח"ע. נגדיר:
$$F:A^C\to B^C\quad F(f)=i\circ f$$
נראה ש-\(F\) חח"ע. יהיו \(f,g\) כך ש-\(F(f)=F(g)\). 
לכן לכל \(x \in C\) מתקיים \(i(f(x))=i(g(x))\) גורר לכל \(x \in C\) מתקיים \(f(x)=g(x)\) כי \(i\) חח"ע. ולכן \(g=f\).

\end{proof}
\begin{proposition}
אם \(\mathfrak{a,b,c}\) עוצמות, \(\mathfrak{a}\neq 0\) ו-\(\mathfrak{b\leq c}\) אזי \(\mathfrak{a^b\leq a^c}\)

\end{proposition}
\begin{proof}
יהיו \(A,B,C\) נציגים. תהי \(i: B\to C\) חח"ע. אז \(\mathfrak{b}=|B|=|rng(i)|\) ואפשר להשתמש ב-\(rng(i)\) במקום ב-\(B\). לכן בלי הגבלת הכלליות \(B\subseteq C\). נראה \(\lvert A^B \rvert\leq \lvert A^C \rvert\). יהי \(a_{0} \in A\) כלשהו. נגדיר 
$$H:A^B\to A^C\quad f\mapsto H(f)\qquad (H(f))(c)=\begin{cases}f(c) & c\in B \\a_{0} &  c \in C \setminus  B
\end{cases}$$
נראה ש-\(H\) חח"ע. (סימון: אם \(\varphi:X\to Y\) פונקציה \(X'\subseteq X\) אז \(\varphi\upharpoonright X'\) זה הצמצום של \(\varphi\) ל-\(X'\) כלומר \(\varphi \upharpoonright X'= \varphi \cap (X' \times Y)\)). נניח \(H(f)=H(g)\) בפרט:
$$f=H(f)\upharpoonright B=H(g)\upharpoonright B = g$$
כלומר \(f=g\)

\end{proof}
יהיו \(\mathfrak{a,b,c}\) עוצמות. מתקיים:

\begin{enumerate}
  \item \({\mathfrak{a}}\cdot({\mathfrak{b}}+{\mathfrak{c}})=({\mathfrak{a}}\cdot{\mathfrak{b}})+({\mathfrak{a}}\cdot{\mathfrak{c}})\)


  \item \({\mathfrak{a}}^{\mathfrak{c}}\cdot{\mathfrak{b}}^{\mathfrak{c}}=({\mathfrak{a}}\cdot{\mathfrak{b}})^{\mathfrak{c}}\)


  \item \({\mathfrak{a}}^{\mathfrak{b}}\cdot{\mathfrak{a}}^{\mathfrak{c}}={\mathfrak{a}}^{\mathfrak{b}+{\mathfrak{c}}}\)


  \item \(\left(a^{\mathfrak{b}}\right)^{\mathfrak{c}}={\mathfrak{a}}^{{\mathfrak{b}}\cdot{\mathfrak{c}}}\)


\end{enumerate}
\begin{proposition}
$$2^{\aleph_{0}}=\mathfrak{c} = |(0,1)|$$

\end{proposition}
\begin{proof}
נשתמש בקנטור שרדר ברנשיין. נראה ראשית \(2^{\aleph_{0}}\leq \mathfrak{C}\). ניקח נציגים:
$$\left\{  f:\mathbb{N}\to \{ 0,1 \}  \right\}=\{ 0,1 \}^\mathbb{N}$$
עבור \(2^x\) ו-\((0,1)\)

\end{proof}
תזכורת:
לכל מספר ממשי \(x \in (0,1)\) יש הצגה סטנדרטית יחידה של פיתוח עשרוני.
$$x=0.x_{0}x_{1}x_{2}\dots x_{n}\dots$$
כאשר הסטנדרטי אומרים שלא מתקיים \(x_{n}=9\) החל מ-\(m \in \mathbb{N}\).
נגדיר פונקציה:
$$G:\{ 0,1 \}^\mathbb{N}\to (0,1)$$
באופן הבא:
לכל \(f:\mathbb{N}\to \{ 0,1 \}\) נגדיר:
$$G(f)=0.f(0)f(1)f(2)\dots f(n)\dots$$
יחידות הפיתוחים של מספירים כנל מבטיח כי \(G\) חח"ע. מסקנה:
$$2^{\aleph_{0}}=|\{ 0,1 \}^\mathbb{N}|\leq|[0,1)|=\mathfrak{c}  $$

\section{תורת הקבוצות האקסיומטית}

\subsection{מחלקות וקבוצות}

נרצה בעזרת תורת הקבוצות להציג כל אובייקט מתמטי. לשם כך נצטרך להגדר במדויק מה זה קבוצה. נעשה זאת בעזרת "השפה הפורמלית":

\begin{definition}[השפה הפורמלית]
השפה היסודית של \(\in, =\) וכמובן הסמלים הלוגיים הרגלים - סוגריים, כמתים, קשרים.

\end{definition}
נראה מספר דוגמאות של דברים שניתן להציג בעזרת קבוצות:
\textbf{דוגמא}
המספרים טבעיים
$$0 = \varnothing \quad 1 = \{ 0 \}\quad n+1=n\cup \{ n \}$$
ולכן ניתן לכתוב למשל:
$$2=1 \cup \{ 1 \} = \{ 1,0 \}=\left\{  \left\{  \varnothing   \right\},\varnothing   
\right\}$$

\begin{example}
זוג סדור ניתן להציג בעזרת קבוצות:
$$\langle x,y \rangle =\left\{  \{ x \},\{ x,y \}  \right\}$$

\end{example}
ניתן גם לתאר קבוצה בעזרת תכונה:

\begin{definition}[תכונה של קבוצה]
תכונה \(p(x)\) של קבוצות היא כזו שניתנת לתיאור ע"י נוסחה בשפה הפורמלית.

\end{definition}
\begin{example}
ניתן לתאר את הקבוצה הריקה \(\varnothing\) בעזרת התכונה
$$\varphi_{0}(x)= \forall y\quad y \not \in x$$
(התכונה התואמת עבור \(x\): לכל קבוצה \(y\) מתקיים \(x\not \in y\)).

\end{example}
\begin{example}
הכלה בין קבוצות \(x\subseteq y\). בשפה הפורמלית:
$$\forall z\left( z \in x \implies z \in y \right)$$

\end{example}
\begin{example}
כיצד נבטא \(x=1\)?
$$X=\left\{  \varnothing   \right\}$$
בשפה הפורמלית:
$$\varphi_{1}(x)= \exists y\quad \left[ \left( \varphi_{0}(y)\land y \in x \right) \land  \forall z \left( z \in x\implies z = y \right)\right]$$
עבור \(x=2\) נקבל:
$$2=\left\{  \varnothing ,\left\{  \varnothing   \right\}  \right\}$$$$\varphi_{2}\left(x\right):=\forall y\quad\left(y\in x\leftrightarrow\varphi_{0}\left(y\right)\vee\varphi_{1}\left(y\right)\right)$$

\end{example}
הניסיון נאיבי להגדיר קבוצה - לכל תכונה \(P(x)\) יש קבוצה \(\left\{  x \mid P(x)  \right\}\).
בעיה - עבור התכונה \(P(x)=x \not \in x\) לא ייתכן קבוצה:
$$\left\{  x\mid x \not  \in x  \right\}$$
הגישה שלנו: נרשום רשימת אקסיומת:
$$ZFC = ZF+AC$$
שתכליתן לתת לנו תיאר של הקבוצות הקיימות.
\textbf{הגדרה} מחלקה
באופן כללי, לאוסף מהצורה:
$$\left\{  x\mid P(x)  \right\}$$
כאשר \(P(x)\) תכונה כלשהי יקרא מחלקה.

\begin{definition}[מחלקה נאותה]
מחלקה שאינה קבוצה.

\end{definition}
\subsection{המערכת ZF}

נגדיר כעת את האקסימות של ZF

\begin{definition}[אקסיומת ההיקפיות]
שתי קבוצות הם שוות אם"ם יש בהם את אותם איברים. או פורמלית:
$$\forall x\forall y\left( x = y \iff \left( \forall z\quad z \in x \iff z \in y \right) \right)$$

\end{definition}
\begin{corollary}
  \begin{enumerate}
    \item \(\forall x\quad x = x\)


    \item \(\forall x \forall y\quad x=y\implies y = x\)


    \item \(\forall x\forall y\forall z \left[ \left( x=y\land y=z \right)\implies x=z \right]\)


  \end{enumerate}
\end{corollary}
\begin{definition}[אקסיומת הקבוצה הריקה]
קיימת קבוצה ריקה:
$$\exists x\quad \varphi_{0}(x)$$

\end{definition}
\begin{remark}
יש מקומות בהם מגדירים את האקסימה הזו בצורה אחרת - ומשלבים את זה ביחד עם הגדרת המחלקה - וקוראים זה אקסימת בניית המחלקה. האקסיומה אומרת כי אוסף כל האובייקטים המתאים תכונה הם מחלקה, ואז בפרט הקבוצה הריקה היא מחלקה.

\end{remark}
\begin{definition}[אקסיומת הזוג הלא סדור]
לכל קבוצות \(x,y\) קיימות קבוצה \(\{ x,y \}\). כלומר:
$$\forall x\forall y\exists z\quad[\forall w\quad w\in z\leftrightarrow(w=x\lor w=y)]$$

\end{definition}
\begin{remark}
נסיק שקיים יחידונים לכל \(x\).
$$\{ x,x \}=\{ x \}$$

\end{remark}
\begin{definition}[אקסיומת האיחוד]
לכל קבוצה \(x\) קיימת הקבוצה \(\displaystyle{\cup_{a \in x}a}\).
$$\forall x\exists y\quad\forall w\,[w\in y\leftrightarrow\exists z\,(z\in x\land w\in z)]$$

\end{definition}
\begin{definition}[אקסימות קבוצות החזקה]
לכל קבוצה קיימת קבוצת חזקה \(\mathcal{P}(x)\). כלומר:
$$\forall x\exists y\forall z\quad(z\in y\leftrightarrow z\subseteq x)$$

\end{definition}
\begin{definition}[אקסיומת הסדירות]
אם \(x\) קבוצה לא ריקה, אז קיים \(y \in x\) שהוא מזערי ביחס \(\in\) מאיברי \(x\).
(כלומר אין \(z \in x\)\(z \in y\))
$$\forall x\left[\underbrace{\exists z\qquad z\in x}_{x\neq\varnothing}\rightarrow\exists y\,(y\in x\land\forall z\quad z\in x\rightarrow z\notin y)\right]$$

\end{definition}
\begin{definition}[אקסיומת האינסוף]
קיים קבוצה אינסופית, בפרט קיים קבוצה \(Z\) שמכילה את \(\varnothing\) ומתקיימת אם \(A \in Z\) אז \(\cup \left\{  A,\{ A \}  \right\}\in Z\). כלומר:
$$\exists y\quad[\varnothing\in y\land\forall x\,(x\in y)\rightarrow S\,(x)\in y]$$

\end{definition}
\begin{remark}
אקסיומת האינסוף נותת לנו שקבוצה אינדוקטיבית היא קבוצה. כלומר קבוצה המקיימת את התנאי:
$$X \in A\implies X\cup \{ X \} \in A$$

\end{remark}
לכל תכונה \(p(x)\) וקבוצה \(A\) קיימת קבוצה \(\left\{  x \in A \mid p(x)  \right\}\). כלומר:
$$\forall x\forall u_{1}\ldots\forall u_{k}\exists y\quad\left[\forall z\quad z\in y\leftrightarrow z\in x\land\varphi\left(z;u_{1},\ldots,u_{k}\right)\right]$$
או לחלופין קיים איבר ב-\(A\) שהחיתוך עם \(A\) הוא ריק.

\begin{definition}[אקסיומת(סכמת) ההחלפה]
לכל מחלקה המקיימת את תנאי הפונקציה(כלומר אוסף של זוגות סדורים ביחד עם תכונה המקיימת את תנאי הפונקציה) נקבל כי עבור קבוצה \(X\), התמונה:
$$F[X]=\left\{  \langle x,y \rangle \mid p\left( \langle x,y \rangle  \right)  \right\}$$
תהיה קבוצה.

\end{definition}
\begin{remark}
שתי האקסיומות האחרונות נקראות סכמה כיוון שלמעשה זהו לא אקסיומה אחת, אלה אקסיומה נפרדת עבור כל תכונה \(p(x)\).

\end{remark}
\begin{definition}[אקסיומת היסוד]
לכל קבוצה לא ריקה \(A\) קיים איבר מינימלי תחת היחס \(\in\). 

\end{definition}
\begin{corollary}
  \begin{enumerate}
    \item \(\forall x\forall y \exists z\quad z = x\cup y\)


    \item \(\forall x\forall y \exists z\quad z=x\cap y\)


    \item \(\forall x \forall y x \in y \implies y \not \in x\)


    \item \(\forall x \quad x \not \in x\)


  \end{enumerate}
\end{corollary}
\begin{remark}
אוסף האקסיומות ללא אקסיומת הבחירה נראה ZF כאשר ביחד עם אקסימת הבחירה נקרא ZFC

\end{remark}
\subsection{טענות ובניות בסיסיות}

\begin{proposition}
לא ייתכן קבוצה שהיא איבר של עצמה. כלומר קבוצה \(x\) המקיימת \(x \in x\).

\end{proposition}
\begin{proof}
נקבל במקרה כזה כי קיים השרשת הכלות:
$$\dots\in x \in x \in x \in x \in \dots $$
בסתירה לאקסיומת היסוד האומרת כי קיים איבר מינימלי ביחס ל-\(\in\).

\end{proof}
\begin{proposition}
תהי \(r\) קבוצה של זוגות סדורים. אז \(\left\{  x\mid \exists y\quad\left( \langle x,y \rangle \right)\in r  \right\}\) ו-\(\left\{  y\mid \exists x\quad\langle x,y \rangle\in r  \right\}\) הם קבוצות.

\end{proposition}
\begin{proposition}
קיימת קבוצה אינדוקטיבית מינימלית ביחס ל-\(\subseteq\).

\end{proposition}
\begin{proof}
ראשית נשים לב כי חיתוך של קבוצות אינדוקטיביות תהיה קבוצה אינדוקטיבית. זאת כיוון שהקבוצה הריקה תהיה בחיתוך, וגם העוקב שלה בפרט בחיתוך.
תהי \(I\) קבוצה אינדוקטיבית מאקסיומת האינסוף. בפרט קבוצת החזקה שלה \(\mathcal{P}(I)\) קיימת מאקסיומת האינסוף. נסתכל על הקבוצה:
$$z=\left\{  J \in \mathcal{P}(I) \mid J\text{ is inductive}   \right\}$$
כאשר נקבל כי \(\cap z\) היא קבוצה אינדוקטיבית מהטענה לעיל, כאשר החיתוך יהיה מינימלי ביחס להכלה.

\end{proof}
\begin{definition}[המספרים הטבעיים]
הקבוצה האינדוקטיבית המינימלית תוגדר בתור המספרים הטבעיים.

\end{definition}
\begin{proposition}[אינדוקציה על הטבעיים]
תהי \(p(x)\) תכונה. נניח כי \(p(0)\) ולכן \(n \in \mathbb{N}\) מתקיים \(p(n)\implies p(n+1)\) אזי \(p(n)\) לכל \(n \in \mathbb{N}\).

\end{proposition}
\begin{proof}
נסמן \(J'=\left\{   n \in \mathbb{N}\mid p(n)  \right\}\) אשר קיימת מאקסיומת ההפרדה. תני המשפט מבטיחים כי \(J' \subseteq \mathbb{N}\) קבוצה אינדוקטיבית, ולכן כיוון ש-\(\mathbb{N}\) הקבוצה האינדוקטיבית המינימלית נקבל כי \(J'=\mathbb{N}\).

\end{proof}
\begin{definition}[רקורסיה על הטבעיים]
תהי \(A\) מחלקה לא ריקה. אם \(F:A\to A\) מחלקת פונקציה(אוסף זוגות סדורים עם תכונה המקיימת את תנאי הפונקציה) אז קיימת ויחידה פונקציה \(f:\mathbb{N}\to A\) כך ש-\(g(0)=a\) ולכל \(n \in \mathbb{N}\) מתקיים \(g(n+1)=F(g(n))\).

\end{definition}
\subsection{סודרים ומונים}

\begin{definition}[מחלקה טרנזטיבית]
מחלקה \(A\) נקראת טרנזטיבית אם לכל \(b \in A\) מתקיים \(b \subseteq A\).

\end{definition}
\begin{definition}[קבוצה טרנזטיבית]
קבוצה אשר כל התתי איברים שלה גם מוכלים בקבוצה נקראת טרנזטיבית. כלומר:
$$x\in y\in A\implies x\in A$$

\end{definition}
\begin{proposition}[הגדרות שקולות לקבוצה טרנזטיבית]
  \begin{itemize}
    \item מתקיים \(x \in S\implies x\subseteq S\)
    \item האיחוד של האיברים מוכל בקבוצה - \(\bigcup S\subseteq S\)
    \item הקבוצה מוכלת בקבוצת החזקה שלה - \(A\subseteq \mathcal{P}(A)\).
  \end{itemize}
\end{proposition}
\begin{proposition}
אם \(A\) קבוצה כך שלכל \(a \in A\) מתקיים כי \(a\) טרנזטיבית אז גם \(\cup A\) ו-\(\cap A\) הם מחלקות טרנזטיביות.

\end{proposition}
\begin{proof}
נראה ראשית \(\cup A\) היא טרנזטיבית. ניקח \(x \in \cup A\). מהגדרת האיחוד קיים קבוצה \(A_{i}\) כך ש-\(x \in A_{i}\). כיוון ש-\(A_{i}\) קבוצה טרנזטיבית מתקיים \(x \subseteq A_{i}\). כאשר מתקיים לכל קבוצה \(A_{i}\subseteq \cup A\) ולכן \(x\subseteq \cup A\) וטרנזטיבית.
כעת נסתכל על החיתוך \(\cap A\). לפי ההגדרה:
$$\bigcap A=\{z\mid\forall B\in A,z\in B\}$$
נניח \(x \in \cap A\). לכן לפי ההגדרה \(x \in B\)  לכל \(B \in A\). כיוון ש-\(B\) טרנזיטיבי נקבל כי אם \(x \in B\) וגם \(y \in x\) אזי \(y \in B\). כאשר כיוון ש-\(y \in B\) לכל \(B \in A\) נקבל כי \(y \in \cap A\).

\end{proof}
\begin{definition}[סודר]
קבוצה \(x\) נקרא סודר אם הוא טרנזטיבי כך שביחד עם היחס \(\in\) הוא מבוסס ביטב ויחס לינארי(כלומר סדר טוב). 

\end{definition}
\begin{remark}
ניתן לתאר את התכונה של להיות טרנזטיבית, וכן את התכונה של להיות מבוסס היטב בשפה הפורמלית. ולכן ניתן להגדיר את המחלקה של כל הסדורים.

\end{remark}
\begin{symbolize}
מסמנים תמיד סודרים באותיות יווניות, פרט לסודרים סופיים, אשר מסמנים באותיות לטיניות.

\end{symbolize}
\begin{proposition}
כל שתי סודרים \(\alpha,\beta\) מתקיים \(\alpha \leq \beta\) או \(\beta \leq \alpha\)(כאשר \(\alpha \leq \beta\) אומר \(\alpha \in \beta\) או \(\alpha = \beta\)). כלומר מחלקת הסודרים יוצרת יחס סדר טוב ביחד עם היחס \(\in\), כאשר יחס סדר זה יהיה יחס סדר חד(מתקיים \(\alpha \not \in \alpha\) לכל איבר כיוון שסודרים הם קבוצות)

\end{proposition}
\begin{symbolize}
עבור סודרים \(\alpha,\beta\) נהוג לסימן \(\alpha<\beta\) עבור \(\alpha \in \beta \iff \alpha \subset \beta\), וכן \(\alpha \leq \beta\) עבור \(\alpha \subseteq \beta\). 

\end{symbolize}
\begin{proposition}
קבוצה \(\alpha\) הוא סודר אם"ם \(\alpha\) קבוצה טרנזטיבית וכל האיברים שלה הם קבוצות טרנזטיביות.

\end{proposition}
\begin{corollary}
אם \(\alpha\) סודר גם \(\cup \alpha\) סודר וגם \(\cap \alpha\) סודר. וכן כל איבר \(\beta \in \alpha\) יהיה סודר.

\end{corollary}
\begin{proposition}
עבור קבוצה של סודרים \(A\) נקבל כי \(\cup A,\cap A\) סודרים וכן:

  \begin{enumerate}
    \item \(\cup A=\sup A\)


    \item \(\cap A=\min A\)


  \end{enumerate}
\end{proposition}
\begin{proof}
כיוון ש-\(A\) היא קבוצה של סוגרים וסודרים הם טרנזיטיבים אז \(\cup A,\cap A\) הוא טרנזיטיבי.

  \begin{enumerate}
    \item נסתכל על \(\cup A\). נראה ראשית כי סודר. כל איבר \(x \in \cup A\) שייך לאיזשהו סודר \(\alpha \in A\). כל \(\alpha\) הוא סדר טוב. אם נסתכל על קבוצה לא ריקה \(S\subseteq \cup A\) נקבל כי כל איבר ב-\(S\) נמצא באיזשהו \(\alpha \in A\). וכיוון שסודרים הם מובססים היטב, קיים איבר מינימלי ב-\(S\), ולכן \(\cup A\) הוא סדר טוב וטרנזטיבי ולכן סודר. 
נראה כי \(\sup A = \cup A\). לכל \(\alpha \in A\) מתקיים \(\alpha \subseteq \cup A\). כיוון ש-\(\cup A\) הוא סודר וסודרים הם מסודרים היטב, אז \(\alpha \leq \cup A\) לכל \(\alpha \in A\). לכן \(\cup A\) הוא החסם העליון של \(A\). כיוון שכל איבר ב-\(A\) הוא קטן או שווה ל-\(\cup A\) נקבל כי זהו החסם העליון המינימלי, ולכן שווה לסופרמום.


    \item נסתכל על \(\cap A\). נראה כי סודר. כיוון שהראינו כבר כי טרנזיטיבי מספיק להראות כי מסודר היטב. אנו יודעים כי \(\cap A\) זה תת קבוצה של כל \(\alpha \in A\). ולכן כיוון שכל \(\alpha \in A\) מסודר היטב, אז גם \(\alpha\) מסודר היטב, ולכן החיתוך מסודר היטב ונקבל כי גם התת קבוצה שלו \(\cap A\) מסודר היטב. 
נראה מינימליות. אנו יודעים כי לכל \(\alpha \in A\) מתקיים \(\cap A\subseteq \alpha\). ולכן כיוון ש-\(\cap A\) הוא סודר וסודרים הם מסודרים היטב אז \(\cap A\leq \alpha\) לכל \(\alpha \in A\). לכן \(\cap A\) הוא האיבר המינימלי.


  \end{enumerate}
\end{proof}
\begin{definition}[סודר עוקב]
אם \(\alpha\) סודר אז \(\alpha+1\equiv\alpha \cup \left\{  \alpha  \right\}\) נקרא הסודר העוקב.

\end{definition}
\begin{definition}[סודר גבולי]
סודר שלא עוקב של אף סודר אחר. במקרה של-\(A\) אין מקסימום נקבל כי \(\cup A\) הוא סודר גבולי.

\end{definition}
\begin{proposition}
המחלקה של הסודרים, אשר נסמן:
$$\mathrm{Ord} = \left\{  \alpha \mid \alpha \text{ is an ordinal}  \right\}$$
היא טרנזיטיביות וסדורה היטב, ולכן מחלקה נאותה.

\end{proposition}
\begin{proof}
נניח בשלילה כי \(\mathrm{O rd}\) קבוצה. מאקסיומת האיחוד נקבל כי האיחוד של \(\mathrm{Ord}\) - כלומר האיחוד של כל הסודרים יהיה סודר, לכן \(\mathrm{Ord}\) יהיה סודר, ובפרט איבר בקבוצה. בסתירה לאקסיומת היסוד.

\end{proof}
\begin{definition}[אומגה]
יהי \(\omega\) הקבוצה הקטן ביותר אשר מכיל את 0 וסגור ע"י הפעולה של לקחת עוקב(כלומר הוספת אחד). בצורה פורמלית:
$$\omega=\cap \left\{  u \mid 0 \in u \land \forall v \in u\quad v+1 \in u  \right\}$$
כאשר \(\omega\) היא קבוצה לפי אקסיומת האינסוף. 

\end{definition}
\begin{proposition}
הקבוצה \(\omega\) היא סודר.  

\end{proposition}
\begin{proposition}[אינדוקציה על סודרים]
תהי \(p\) תכונה. אם לכל \(\alpha \in \mathrm{Ord}\) מתקיים:
$$\forall \beta \in \alpha \quad p\left( \beta \right)\implies p\left( \alpha \right)$$
אז \(p\left( \alpha \right)\) לכל \(\alpha \in \mathrm{Or d}\). ניתן לחלופין לחלק את כל הסודרים לסודרים גבוליים ועוקבים, ובמקרה זה אם מתקיים:

  \begin{enumerate}
    \item אם \(\alpha = \alpha'+1\) וגם \(p\left( \alpha' \right)\) אז \(p\left( \alpha \right)\). 


    \item אם \(\alpha\) סודר גבולי ו-\(p\left( \beta \right)\) מתקיים לכל \(\beta \in \alpha\) אז \(p\left( \alpha \right)\). 
אז התכונה \(p\left( \alpha \right)\) מתקיימת לכל \(\alpha \in \mathrm{ Or d}\).


  \end{enumerate}
\end{proposition}
השימוש העיקרי של סודרים זה לתאר קבוצות סדורות היטב.
\textbf{טענה}
לכל יחס סדר \(\langle A,< \rangle\) כך ש-\(<\) יחס סדר טוב קיים איזומורפיזם יחיד ל-\(\left\langle  \alpha,\in  \right\rangle\) עבור סודר כלשהו \(\alpha\).

נוכיח קיום.

\begin{proof}
נגדיר את האיזומורפיזם \(f:A\to\alpha\) בצורה אינדוקטיבית:
$$f(y)=\left\{  f(z)\mid z<y  \right\}$$
כאשר \(\alpha=\mathrm{Im}(f)\) זה יהיה הסודר המתאים.

\end{proof}
\begin{definition}[מונה]
סודר \(\kappa\) כך שלכל \(\alpha< \kappa\) מתקיים \(|\alpha|<|\kappa|\).

\end{definition}
\begin{proposition}
יהי \(\alpha\) סודר. אז קיים מונים \(\kappa,\lambda\) כך ש-\(|\kappa|=|\alpha|<|\lambda|\) ולכל \(\beta<\lambda\) מתקיים \(|\beta|\leq \kappa\). כלומר, \(\lambda\) הוא העוקב של \(\kappa\) בסדר על המונים. 

\end{proposition}
\subsection{אקסיומת הבחירה}

\begin{definition}[פונקציית בחירה]
בהינתן קבוצה \(X\) של קבוצות לא ריקות, פונקציית בחירה היא פונקציה אשר לוקחת איבר מכל קבוצה ב-\(X\), וממפה אותה לאיבר ב-\(X\). כלומר עבור הקבוצה \(\left\{  B_{i}\mid i \in I  \right\}\) קיימת פונקציה בחירה \(F:I\to \cup \left\{  B_{i}\mid i \in I  \right\}\) כך ש-\(\forall i \in I\quad f(i) \in B_{i}\). בכתיבה פורמלית:
$$\forall X\left[ \varnothing \not \in X\implies \exists f:X\to \bigcup X \quad \forall A \in X\left( f(A)\in A \right)  \right]$$

\end{definition}
\begin{definition}[אקסיומת הבחירה]
לכל קבוצה לא ריקה \(A\) קיים פונקציית בחירה \(f\) אשר מקשר כל קבוצה \(a \in A\) לאיבר \(x \in A\). 

\end{definition}
\begin{proposition}
אקסיומת הבחירה שקול לכך שלכל פונקציה \(f:A\to B\) על קיים פונקציה \(g:B\to A\) אשר חח"ע.

\end{proposition}
נוכיח רק את הכיוון \(\impliedby\).
\textbf{הוכחה}
נניח את אקסיומת הבחירה:

\begin{enumerate}
  \item יהי \(f:A\to B\) פונקציה אשר על. נגדיר את הקבוצה: 
$$C_{b}=\left\{  a \in A\mid f(a)=b  \right\}$$
כאשר קבוצה זו אינה ריקה כיוון שהפונקציה \(f\) היא על. 
הערה: קבוצה זו לפעמים נקראת התמונה ההפוכה(preimage) ומסמן ב-\(f^{-1}(b)\).


  \item כעת נגדיר את הקבוצה: 
$$X=\left\{  C_{b}\mid b \in B  \right\}$$
נשים לב כי זו קבוצה של קבוצות לא ריקות, כאשר האיחוד \(\cup X=A\) כיוון שהפונקציה \(f\) היא על.


  \item מאקסיומת הבחירה קיימת פונקציה \(g:B\to \cup X=A\) המקיימת \(g(b)\in C_{b}\). 


  \item אם \(g(b_{1})=g(b_{2})\) נקבל כי \(b_{1},b_{2}\in C_{b_{1}}\cap C_{b_{2}}\) וכיוון שהקבוצות זרות, נקבל כי לכל שתי איברים מתקיים \(b_{1}=b_{2}\)


\end{enumerate}
\subsubsection{הלמה של צורן}

\begin{definition}[שרשרת]
יהי \(\left( Z,<_{Z} \right)\) יחס סדר חלקי. תת קבוצה \(C\subseteq Z\) תיקרא שרשרת אם ב-\(<_{Z}\) הוא היחס סדר מלא על \(C\).

\end{definition}
\begin{definition}[חסם למעיל ואיבר מקסימלי]
תהי \((Z,<_{Z})\) יחס סדר חלקי ו-\(C\) שרשרת של \(Z\). איבר \(x \in Z\) יקרא חסם מלעיל של שרשרת \(C\) אם:
$$\forall \in C\quad y\leq_{z} x$$
כאשר איבר \(m \in Z\) יקרא איבר מקסימלי אם לא קיים \(y \in Z\) כך ש-\(m<_{Z}y\).

\end{definition}
\begin{definition}[הלמה של צורן]
לכל יחס סדר חלקי \((Z,<_{Z})\) אם לכל שרשרת \(C\subseteq Z\) יש חסם מלעיל אז יש איבר מקסימלי בסדר \((Z,<_{Z})\).

\end{definition}
\begin{proposition}
הלמה של צורן גורר כי לכל מרחב ווקטורי קיים בסיס.

\end{proposition}
\begin{proof}
  \begin{enumerate}
    \item יהי \(V\) מרחב ווקטורי. נגדיר את הקבוצה: 
$$Z=\left\{  B\subseteq V\mid B \text{ is linearly independent}  \right\}$$
כאשר נשים לב כי \(Z\) לא ריקה כי למשל \(\varnothing \in Z\).


    \item ניקח את היחס \(<_{Z}\) לביות \(\subset_{Z}\). כלומר \(B_{1}<_{Z}B_{2}\) אם \(B_{1}\subset_{Z}B_{2}\). ביחד עם היחס הזה, אם תנאי הלמה של צורן מתקיימים זה אומר שקיים \(B\subseteq V\) מקסימלי, כך שהוספה של כל איבר אחר יגרום לכך שלא יהיה מקסימלי. ולכן יהיה בסיס, ונדרש רק להראות את תנאי הלמה. 


    \item תהי \(C\subseteq Z\) שרשרת. כלומר \(C\) מורכבת מבסיסים כך שכל אחת מוכלת בשנייה. 


    \item נתבונן בקבוצה: 
$$B=\bigcup C=\bigcup \left\{  B'\mid B' \in C  \right\}$$
נדרש להראות כי חסם מלעיל וכי נמצא בקבוצה Z(אחרת היחס סדר בכלל לא מוגדר). כלומר כי בתל. העובדה שחסם מלעיל ברורה כיוון שכל איבר בקבוצה מוכל בה.


    \item נראה כי \(B\) בתל. יהיו \(v_{1},\dots,v_{n}\in B\) כך שקיימים \(a_{1},\dots,a_{n}\) לא כולם אפס כך ש: 
$$\sum_{i=1}^{\infty}a_{i}v_{i}=0$$


    \item מהגדרת ההכלה נקבל כי \(v_{1},\dots,v_{n}\in B_{i}\) עבור \(B_{i}\in C\). כאשר נזכור כי מהגדרת השרשרת מתקיים: 
$$B_{1}\subseteq \dots \subseteq B_{n}$$
ולכן נקבל כי \(v_{1},..,v_{n}\in B_{n}\) ולכן בתל כיוון ש-\(B_{n}\) בתל.


    \item קיבלנו כי \(B\) בתל, לכן לכל שרשרת יש חסם עליון וקיים מקסימום לקבוצה \(Z\), אשר יהיה בסיס של המרחב הווקטורי. 


  \end{enumerate}
\end{proof}
\begin{proposition}
הלמה של צורן שקולה לכך שניתן להרחיב כל סדר חלקי לסדר קווי.

\end{proposition}
\begin{proof}
  \begin{enumerate}
    \item 
  \end{enumerate}
\end{proof}
\subsection{עקרון הסדר הטוב}

\begin{definition}[עיקרון הסדר הטוב]
לכל קבוצה \(A\) קיים יחס סדר טוב \(<_{A}\subseteq A\times A\).

\end{definition}
\begin{proposition}
עקרון הסדר הטוב, ובפרט אקסיומת הבחירה שקולה לטענה כי לכל זוג קבוצות \(A,B\) מתקיים כי \(|A|\leq|B|\) או \(|B|\leq |A|\).

\end{proposition}
\begin{proposition}
עקרון הסדר הטוב גורר כי לכל קבוצה \(X\) קיים מונה \(\theta\) כך ש-\(|X|=|\theta|\).

\end{proposition}
\begin{theorem}[השקילות]
הביטויים הבאים שקולים:

  \begin{enumerate}
    \item אקסיומת הבחירה. 


    \item הלמה של צורן. 


    \item עקרון הסדר הטוב. 


  \end{enumerate}
\end{theorem}
\end{document}