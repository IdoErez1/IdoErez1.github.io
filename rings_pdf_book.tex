\documentclass{tstextbook}

\usepackage{amsmath}
\usepackage{amssymb}
\usepackage{graphicx}
\usepackage{hyperref}
\usepackage{xcolor}

\begin{document}

\title{Example Document}
\author{HTML2LaTeX Converter}
\maketitle

\section{הגדרת החוג}

\subsection{חוגים}

\begin{definition}[חוג]
אוסף \(\left( R,+,\cdot \right)\)  כאשר:

  \begin{enumerate}
    \item הקבוצה \(R\) ביחד עם חיבור יוצר חבורה חיבורית, עם איבר ניטרלי אפס. 


    \item הקבוצה \(R\) ביחד עם כפל היא אסוצייטיבית  


    \item מכיל יחידה כפלית 1(לא תנאי הכרחי בחלק מההגדרות) 


    \item מקיים שתי חוקי פילוג: 
$$r(a+b)=r\cdot a+r\cdot b\quad (a+b)\cdot r=a\cdot r+b\cdot r$$


  \end{enumerate}
\end{definition}
נשים לב כי אין בהכרח הופכי כפלי. חוג עם הופכי לכפל נקרא חוג חילוק. יש דבר כזה חוגים ללא יחידה כפלית. אך אנחנו לא מתעניין בהם כעת.

\begin{example}
  \begin{enumerate}
    \item המספרים השלמים \(\mathbb{Z}\) עם החיבור והכפל הרגילים. 


    \item החיבור והכפל המודולרי \(\mathbb{Z} _n\). 


    \item שדה כללי \(\mathbb{F}\). 


    \item פולינומים \(\mathbb{F}[x]\), \(\mathbb{F}[x,y]\). 


    \item פולינום עם חוג אחר \(R[x]\) כש-\(R\) חוג. 


    \item מטריצות \(M_{n\times n}\left( \mathbb{F} \right)\), \(M_{n\times n}(R)\)(כש-\(R\) חוג) 


  \end{enumerate}
\end{example}
\begin{remark}
עבור \(M_{n\times n}(R)\) לא בהכרח אי-הפיכות גורר דטרמיננטה 0.

\end{remark}
\begin{definition}[החוג הטריוויאלי]
החוג המוגדר על ידי:
$$R=\{ 0 \}$$
כאשר \(0+0=0, 0\cdot 0 =0\).

\end{definition}
\begin{definition}
אם \(R\) חוג ו-\(X\) קבוצה אז \(R^X\) (פונקציות \(X\to R\)) יהיה חוג עם הפעולות
$$\begin{gathered}(f+f')(x)=f(x)+f'(x) \\\left( f\cdot f' \right)(x)=f(x)\cdot f'(x)
\end{gathered}$$

\end{definition}
אם \((A,+)\) חבורה אבלית. \(\mathrm{End}(A)\) הוא חוג ביחס ל- 
$$\begin{gathered}(f+f')(a)=f(a)+f(a') \\\left( f\cdot f' \right)(a)= f(f'(a)) \\0=\left( a\mapsto 0_{A} \right)\qquad 1=\mathrm{Id}_{A}
\end{gathered}$$
הפילוג פה יותר מעניין. הפילוג מצריך את האבליות! מושאר כתרגיל להראות פילוג.

\begin{example}
עבור \(A=\mathbb{Z} _p^n\) כאשר \(\mathrm{End}(A)=M_{n\times n}\left( \mathbb{F}_{p} \right)\).

\end{example}
\begin{proposition}[משפט קיילי]
כל חוג איזומורפי לתת חוג של \(\mathrm{End}(A)\) לחבורה אבלית \(A\) כלשהי.

\end{proposition}
\begin{proposition}[תכונות של חוגים]
יהיו \(a,b,c \in R\). אזי מתקיים:

  \begin{enumerate}
    \item \(a 0=0a=0\)


    \item \(a(-b)=(-a)b=-(ab)\)


    \item \((-a)(-b)=ab\)


    \item \(a(b-c)=ab-ac\) וגם \((b-c)a=ba-ca\)


    \item \((-1)a=-a\)


    \item \((-1)(-1)=1\)


  \end{enumerate}
\end{proposition}
\begin{definition}
איבר \(r\in R\) נקרא הפיך אם קיים \(s \in R\) כך ש-\(rs=sr=1\). 

\end{definition}
\begin{proposition}
אם קיים ל-\(r\) הופכי הוא יחיד, ומסומן \(r^{-1}\).

\end{proposition}
\begin{example}
נסתכל על \(End\left( \mathbb{Z} ^{\mathbb{N} } \right)\). סדרות של שלמים זה חבורה אבלית. נסתכל על האנדומורפיזם שלה. עבור \(T\in End\left( \mathbb{Z} ^{\mathbb{N} } \right)\) המוגדר:
$$\begin{gathered}T\left( \left( a_{1},a_{2},\dots \right) \right)=\left( a_{2},a_{3},\dots \right) \\S\left( \left( a_{1},a_{2},\dots \right) \right)=\left( 0,a_{1},a_{2},\dots \right)
\end{gathered}$$
כאשר נשים לב כי \(S\circ T = id\) אך \(T\circ S \neq id\). זה הדוגמא הסטנדרטית ללמה הופכי מימין לא דווקא ההופכי משמאל.

\end{example}
\begin{proposition}
אוסף ההפיכים ב-\(R\) היא חבורה, מסומנת \(R^\times\).

\end{proposition}
\begin{example}
$$M_{n\times n}\left( \mathbb{F} \right)^\times=GL_{n}\left( \mathbb{F} \right)\qquad \mathbb{Z} ^\times=\left\{  \pm 1  \right\}\qquad \mathbb{F}[x]^\times=\mathbb{F}^\times$$

\end{example}
\begin{definition}[הומומורפיזם של חוגים]
פונקציה \(f:R\to S\) שעבורה משמר חיבור, כפל ויחידה: 
$$\begin{gathered}f(x+y)=f(x)+f(y) \\f\left( x\cdot y \right)=f(x)\cdot f(y) \\f(1)=1
\end{gathered}$$

\end{definition}
\begin{example}
קיים הומומורפיזם יחיד \(f:\mathbb{Z} \to R\) עבור כל חוג! זה יהיה
$$f(m)=\begin{cases}\underbrace{ 1+\dots+1 }_{ m }  & m\geq 0\\-\underbrace{ \left( 1+\dots+1 \right) }_{ m  } & m<0
\end{cases}$$

\end{example}
\begin{example}
עבור \(f:\mathbb{R} \to M_{2\times 2}\left( \mathbb{R}  \right)\) המוגדר
$$f\left( \alpha \right)=\begin{pmatrix}\alpha & 0 \\0 & 0
\end{pmatrix}$$
הפונקציה \(f\) משמרת חיבור וכפל אבל לא יחידה!

\end{example}
\begin{example}
הפונקציה \(0:R\to \{ 0 \}\) היא הומומורפיזם ונקראת הומומורפיזם האפס. אם \(S\neq \{ 0 \}\) אזי \(0:R\to S\) היא לא הומומורפיזם(כי לא משמרת יחידה)

\end{example}
\begin{remark}
ברגע שיש הומומורפיזמים, יש גם איזומורפיזים, מונומורפיזם, אפימורפיזם וכו.

\end{remark}
\begin{proposition}
לחוג יש איבר יחידה כפלי יחיד.

\end{proposition}
\begin{definition}[חוג קומוטטיבי]
חוג אשר גם הפעולה של הכפל היא קומוטטיבית. כלומר \(\forall a,b \in R\quad a\cdot b=b\cdot a\)

\end{definition}
\begin{definition}[חוק חילוק]
חוג אשר גם לכל איבר פרט לאיבר הניטרלי החיבורי יש הוכפי כפלי. כלומר \(\forall a \in R\quad a^{-1}\in R\).

\end{definition}
\begin{definition}[שדה]
חוג חילוק קומוטטיבי.

\end{definition}
\subsection{תת חוג}

\begin{definition}[תת חוג]
קבוצה לא ריקה \(S\) של \(R\) נקראת תת חוג אם \(S\) מהווה תת חבורה חיבורית תחת חיבור(כלומר סגורה לחיבור ולהופכי חיבורי) וכן סגורה תחת כפל, כלומר אם \(a,b\in R\) אז \(a\cdot b \in R\). מסומן \(S\leq R\).

\end{definition}
\begin{example}
$$\{ 0 \}\not\leq \mathbb{Z} \qquad \mathbb{F},\mathbb{F}[x^2]\leq \mathbb{F}[x]$$

\end{example}
\begin{reminder}
כשהיינו בחבורות, דיברנו על קבוצה \(X\) ו-\(S_{X}\) חבורה. פעולה של \(G\) על \(X\) שקול ל-\(\rho:G\to S_{X}\) הומומורפיזם(המבנה).

\end{reminder}
\begin{definition}[פעולה על חוג]
עבור \(A\) חבורה אבלית, \(\mathrm{End}(A)\) חוג. פעולה של חוג \(R\) על \(A\)(\(R \circlearrowright A\)): $$\rho:R\to \mathrm{End}(A)$$ כאשר \(\rho\) זה הומומורפיזם של חוגים. נתרגם לפעולה \(\cdot:R\times A\to A\)$$\begin{gathered}(r+r')a=ra+r'a\qquad (rr')a=r(r'a) \\r(a+a')=ra+ra'\qquad 1a=a
\end{gathered}$$

\end{definition}
עבור שדה \(\mathbb{F}\), מרחב וקטורי מעל \(\mathbb{F}\) שקול לחבורה אבלית עם פעולה של \(\mathbb{F}\). אמנם במרחבים וקטורים היו יותר מ-\(4\) אקסיומות. זאת כיוון שחלק מהאקסימות נדרשו בשביל האבליות.

\begin{remark}
לא קוראים ל-\(A\) עם פעולה של חוג \(R\) מרחב וקטורי מעל \(R\). קוראים להם מודול.

\end{remark}
כעת ניתן להוכיח את משפט קיילי לחוגים

\begin{reminder}[משפט קיילי לחוגים]
כל חוג איזומורפי לתת חוג של \(\mathrm{End}(A)\) לחבורה אבלית \(A\) כלשהי.

\end{reminder}
\begin{proof}
ניקח \(A=R^+\). אזי \(R\circlearrowright A\) ע\"י \(r.a=ra\). נקבל:
$$\rho:R\to \mathrm{End}(R^+)$$
נקבל כי \(\rho\) חח"ע אם"ם:
$$\left\{  r \mid \rho(r)=0  \right\}=\ker\left( \rho \right)=\{ 0 \}$$
כעת:
$$r\in \ker\left( \rho \right):\rho(r)=0\quad r=r\cdot 1=\rho(r)(1)=0(1)=0$$

\end{proof}
תת חוג \(S\leq R\) יקיים \(1\in S\), \(S\pm S, S\cdot S\subseteq S\) אידיאל \(I\trianglelefteq R\) יקיים \(RIR\subseteq I\), \(I+I\).(\(I=R\iff 1\in I\))

\begin{example}
עבור \(m\mathbb{Z} \trianglelefteq \mathbb{Z}\). \(M_{n}\left( \mathbb{F} \right)\trianglelefteq M_{n}\left( \mathbb{F} \right)\), \(\{ 0 \}\) ורק הם.

\end{example}
\subsection{הומומורפיזם של חוגים}

\begin{definition}[הומומורפיזם של חוגים]
יהיו \(S,R\) חוגים. הומומורפיזם של חוגים זו העתקה \(\varphi:S\to R\) כך שמתקיים:

  \begin{enumerate}
    \item זהו הומומורפיזם של החבורה החיבוריות 


    \item משמר כפל, כלומר \(\varphi\left( s_{1}\cdot s_{2} \right)=\varphi(s_{1})\cdot \varphi(s_{2})\). 


    \item מעביר יחידה ליחידה - \(\varphi(1)=1\). 


  \end{enumerate}
\end{definition}
\begin{definition}[גרעין של הומומורפיזם]
אוסף האברים בחוג \(R\) שההומומורפיזם \(\varphi\) שולח ל-0. כלומר:
$${\mathrm{ker}}\,\left( \varphi \right)=\left\{r\in R\,|\,\varphi\,(r)=0\right\}$$

\end{definition}
\begin{definition}[תמונה של הומומורפיזם]
אוסף כל האיברים שההומומורפיזם מגיע עליהם:
$${\mathrm{Im}}\left( \varphi \right)=\left\{\varphi\left(r\right)\left|\,r\in R\right\}\right.$$

\end{definition}
\begin{proposition}
התמונה של הומומורפיזם היא תת חוג, והגרעין של הומומורפיזם היא אידיאל.

\end{proposition}
\begin{proposition}[משפט האיזומורפיזם הראשון של חוגים ]
נניח \(\phi:R\to S\) הומומורפיזם של חוגים \(R,S\). אז ההעתקה:
$$R / \ker \phi \to \mathrm{Im}\left( \phi \right)$$
היא איזומורפיזם בין חוגים

\end{proposition}
\subsection{אידיאלים}

\begin{definition}[אידיאל]
תת חוג \(I\) של \(R\) נקרא אידיאל(דו צדדי) אם לכל \(r\in R\) ולכן \(i \in I\) מתקיים \(ri,ir\in I\)

\end{definition}
\begin{proposition}[מבחן האידיאל]
קבוצה לא ריקה \(I\) הוא אידיאל של החוג \(R\) אם:

  \begin{enumerate}
    \item אם \(a,b \in I\) גורר \(a-b\in I\)


    \item כאשר \(i \in I\) ו-\(r \in R\) גורר \(ir,ri \in I\)


  \end{enumerate}
\end{proposition}
נשים לב כי תנאי אחד מראה כי תת חבורה חיבורית, ותנאי 2 מראה שזה אידיאל

\begin{remark}
  \begin{enumerate}
    \item החוג כולו \(R\) והאידיאל הטריוויאלי \(\{ 0 \}\) הם תמיד אידיאלים 


    \item מסומן \(I\trianglelefteq R\). 


    \item האידיאל יכיל את היחידה רק אם הוא החוג כולו. 


  \end{enumerate}
\end{remark}
כעת ננסה להגדיר חוגי מנה כמו בחבורות. 

\begin{proposition}[חוג מנה]
יהי \(R\) חוג, ו-\(A\) תת חוג של \(R\). הקבוצה:
$$\left\{  r+A\mid r\in R  \right\}$$
יוצרת תת חוג ביחד עם הפעולה \((s+A)+(t+A)=s+t+A\) ו-\((s+A)\cdot(t+A)=st+A\) אם"ם החוג \(A\) הוא אידיאל.

\end{proposition}
\begin{definition}[חוג פשוט]
חוג נקרא פשוט אם האידיאלים היחידים בו הם \((0)=0\) ו-\((1)=R\).

\end{definition}
\begin{proposition}
כל חוג חילוק פרט לחוג הטריוויאלי הוא פשוט

\end{proposition}
\begin{proposition}
עבור חוג קומוטטיבי \(R\), \(R\) הוא פשוט אם"ם הוא שדה.

\end{proposition}
\begin{proposition}
לכל שדה \(F\) ו-\(n\) טבעי, חוג המטריצות \(M_{n}(F)\) פשוט.

\end{proposition}
\subsection{תחום שלמות}

\begin{definition}[מחלק אפס]
איבר \(a\neq 0\) נקרא מחלק אפס של חוג \(R\) אם קיים \(0\neq b \in R\) ומקיים \(a\cdot b =0\)

\end{definition}
\begin{definition}[תחום שלמות]
חוג קומוטטיבי עם יחידה וללא מחלק אפס נקרא חוק שלמות.

\end{definition}
\begin{definition}[פריקות]
יהא \(D\) תחום שלמות. אז \(r \in D\) יקרא פריק אם קיים \(a,b \in \mathbb{R} \setminus \mathbb{R}^\times\) כך ש-\(r=a\cdot b\).

\end{definition}
\begin{example}
החוגים הבאים הם חוגי שלמות:

  \begin{enumerate}
    \item חוג המספרים שלמים 


    \item חוג המספרים הגאוסים \(Z[i]=\{a+b i\mid a,b\in Z\}\). 


    \item חוג הפולינומים עם מקדמים שלמים \(Z[x]\). 


    \item החוג \(Z[{\sqrt{2}}]=\{a+b{\sqrt{2}}\mid a,b\in Z\}\). 


    \item החוג המודולארי \(\mathbb{Z}_{p}\) כאשר \(p\) ראשוני. 


  \end{enumerate}
\end{example}
\begin{proposition}[חוק הביטול]
יהיו \(a,b,c\) בחוג שלמות. אם \(a\neq 0\) ו-\(ab=ac\) אז \(b=c\).

\end{proposition}
\begin{proposition}
תחום שלמות סופי הוא שדה.

\end{proposition}
\section{חוג הפולינומים}

\subsection{פולינומים ופריקות}

\begin{definition}[חוג פולינומים]
חוג המסומן ב-\(R[x]\) כאשר 
$$R[x]=\{a_{n}x^{n}+a_{n-1}x^{n-1}+\cdot\cdot\cdot+a_{1}x+a_{0}\mid a_{i}\in R, x \in \mathbb{F} \} $$

\end{definition}
\begin{definition}[פריקות של פולינום]
יהי \(D\) תחום שלמות. פולינום \(f(x)\in D\) שהוא לא פולינום האפס ולכן פולינום היחידה נקרא פריק אם קיימים \(g(x),h(x)\in D\) כך ש-\(deg(g(x)),deg(h(x))>1\) ומקיימים \(f(x)=g(x)h(x)\).

\end{definition}
\begin{proposition}
אם יש לפולינום \(f(x)\in \mathbb{F}[x]\) שורש, אז הוא פריק. 

\end{proposition}
כאשר נשים לב כי הכיוון השני בדרך כלל לא נכון.

\begin{proposition}
לפולינום \(f(x)\in \mathbb{F}[x]\) מסדר 2 או 3 יש שורש אם"ם הוא פריק.

\end{proposition}
\subsection{מבחני פריקות}

\begin{proposition}[הלמה של גאוס]
יהא \(p(x)\in \mathbb{Z}[x]\) פולינום ממעלה חיובית. אז \(p(x)\) אי פריק מעל \(\mathbb{Z}[x]\) אם"ם הוא אי פריק מעל \(\mathbb{Q}\)

\end{proposition}
\begin{proposition}[קריטריון אייזנשטיין]
יהא \(\mathbb{Z}[x]\ni q(x)=a_{n}x^n+\dots+a_{0}\), \(n>0\).
אם קיים \(p\) ראשוני כך ש:

  \begin{enumerate}
    \item מתקיים \(p\nmid a_n\)


    \item מתקיים \(p\mid a_{i}\) לכל \(0\leq i<n\). 


    \item מתקיים \(p^2 \nmid a_{0}\) 
אז \(q(x)\) אי פריק


  \end{enumerate}
\end{proposition}
\begin{proposition}[קרטריון השורש הרציונאלי]
נתון פולינום מהצורה $$p(x)=a_{n}x^{n}+\ldots+a_{0}\in\mathbb{Z}[x]$$ אם \(p\left( \frac{a}{b} \right)=0\) ו-\(\frac{a}{b}\in \mathbb{ Q}\) מצומצם אזי \(a\mid a_{0}\) ו-\(b\mid a_{n}\).

\end{proposition}
\begin{proposition}[טרנספורמציה לינארית]
לכל \(a,b \in \mathbb{F}\) כך ש-\(a\neq 0\) אז \(p(x) \in \mathbb{F} [x]\) פריק אם"ם \(f(ax+b)\) פריק.

\end{proposition}
\subsection{תחום ראשי}

\begin{definition}[תחום ראשי]
תחום שבו כל אידיאל נוצר ע"י איבר יחיד.

\end{definition}
\begin{definition}[איבר ראשוני]
יהא \(R\) תחום שלמות. איבר \(r\in R\) לא הפיך יקרא ראשוני אם \(r \mid a\cdot b\) גורר \(r \mid a\) או \(r \mid b\).

\end{definition}
\begin{proposition}
בתחום ראשי \(R\). איבר הוא ראשוני אם"ם הוא אי פריק. 

\end{proposition}
\subsection{תחום שלמות}

\begin{definition}[מחלק אפס]
איבר \(a\neq 0\) נקרא מחלק אפס של חוג \(R\) אם קיים \(0\neq b \in R\) ומקיים \(a\cdot b =0\)

\end{definition}
\begin{definition}[תחום שלמות]
חוג קומוטטיבי עם יחידה וללא מחלק אפס נקרא חוק שלמות.

\end{definition}
\begin{example}
החוגים הבאים הם תחום שלמות:

  \begin{enumerate}
    \item חוג המספרים שלמים \(\mathbb{Z}\)


    \item חוג המספרים הגאוסים \(\mathbb{Z}[i]=\left\{ a+b i\mid a,b\in \mathbb{Z} \right\}\)


    \item חוג הפולינומים עם מקדמים שלמים \(\mathbb{Z}[x]\)


    \item החוג \(\mathbb{Z}\left[ {\sqrt{2}} \right]=\left\{ a+b{\sqrt{2}}\mid a,b\in \mathbb{Z} \right\}\)


    \item החוג המודולארי \(\mathbb{Z}_{p}\) כאשר \(p\) ראשוני 


  \end{enumerate}
\end{example}
\begin{example}
החוגים הבאים הם לא חוגי שלמות:

  \begin{enumerate}
    \item החוג המודולארי \(\mathbb{Z}_{n}\) כאשר \(n\) אינו ראשוני 


    \item חוג המטריצות \(M_{2}\left( \mathbb{Z} \right)\)


  \end{enumerate}
\end{example}
\begin{proposition}[חוג הביטול]
יהיו \(a,b,c\) בחוג שלמות. אם \(a\neq 0\) ו-\(ab=ac\) אז \(b=c\)

\end{proposition}
\begin{proof}
מ-\(ab=ac\) נקבל \(a(b-c)=0\). כיוון ש-\(a\neq 0\) בהכרח מתקיים \(b-c=0\) ולכן \(b=c\).

\end{proof}
\begin{proposition}
אם \(R\) תת חוג של שדה \(F\). אזי ב-\(R\) אין מחלקי אפס.

\end{proposition}
\begin{proof}
נניח \(xy=0\) ו-\(x\neq 0\). נראה כי \(y=0\):
$$y=\left(x^{-1}x\right)y=x^{-1}\left(x y\right)=x^{-1}\cdot0=0$$

\end{proof}
\begin{corollary}
כל תת חוג של שדה תחום שלמות.

\end{corollary}
\begin{proposition}
תחום שלמות סופי הוא שדה.

\end{proposition}
\section{מודולים}

\subsection{הגדרת ותכונות}

\begin{definition}[מודול]
יהי \(R\) חוג עם יחידה. מודול מעל R(לעיתים גם נקרא \(R\) מודול) יהיה חבורה אבלית \((M,+)\) ביחד עם כפל סקלארי \(R\times M\to M\) אשר מקיים לכל \(r,s \in R\) ו-\(m,n \in M\):

  \begin{enumerate}
    \item פילוגי מעל \(M\): 
$$r\cdot\left(m+n\right)=r\cdot m+r\cdot n$$


    \item פילוגי מעל \(R\): 
$$(r+s)\cdot m=r\cdot m+s\cdot m$$


    \item אסוצייטיבי: 
$$(rs)\cdot m=r\cdot(s\cdot m)$$


    \item יחידה: 
$$1\cdot m= m$$
מסומן \(R^{M}\).


  \end{enumerate}
\end{definition}
\begin{remark}
זה למעשה הכללה של מרחב ווקטורי, כאשר הווקטורי זה האיבר בחבורה האבלית והסקלר זה איבר בחוג. בפרט אם המודול הוא מעל שדה נקבל מרחב ווקטורי.

\end{remark}
\begin{remark}
זה למעשה נקרא מודול שמאלי. ניתן להגדיר מודול ימיני מוגדר באותו אופן עם סקלר מימין.

\end{remark}
כדי לקבל אינטאיציה על ההגדרה נזכר במשפט קיילי לחוגים.

\begin{proposition}
אם \(A\) חבורה אבלית אזי \(\text{End}(A)\) הוא חוג, כאשר חיבור מוגדר על ידי:
$$(f+g)(m)=f(m)+g(m)$$
וכן כפל מוגדר על ידי הרכבה:
$$(f\cdot g)(m)=f(g(m))$$

\end{proposition}
\begin{reminder}[משפט קיילי לחוגים]
כל חוג איזומורפי לתת חוג של \(\mathrm{End}(A)\) לחבורה אבלית \(A\) כלשהי.

\end{reminder}
\begin{proposition}[תנאי שקול למודול]
יהי \(M\) חוברה אבלית ו-\(\rho:R\to \mathrm{End}(M)\) הומומורפיזם של חוגים כאשר  \(\mathrm{End}(M)\) הוא החוג של כל האנדומורפיזמים של חבורה \(M\).
הפעולה של \(r \in R\) על \(m \in M\) נתונה על ידי \(\rho(r)(m)\).

\end{proposition}
\begin{symbolize}
מסמנים \(R^{M}\) כדי לסמן ש-\(M\) הוא \(R\) מודול.

\end{symbolize}
\begin{example}
המודול \(R^{R}\) זה המודול שהחבורה האבלית היא \(R\) בעצמה, וכפל בסקלר זה מכפלת החוג.

\end{example}
\begin{definition}[סכום ישר של מודולים]
אם \(M\) ו-\(N\) הם \(R\) מודולים, אז הסכום הישר שלהם מוגדר על ידי:
$$M\oplus N=\{(m,n)\mid m\in M,\,n\in N\}$$
כאשר כפל בסקלר מוגדר רכיב רכיב:
$$r\cdot(m,n)=(r\cdot m,\,r\cdot n)$$

\end{definition}
\begin{remark}
הגדרה זו משמרת את ההגדרה הידועה עבור שדות ווקטורים.

\end{remark}
\begin{definition}[תת מודול]
יהי \(R^{M}\) מודול. תת קבוצה \(N\subseteq M\) נקרא תת מודול אם:
- לא ריקה.
- סגורה תחת חיבור. כלומר לכל \(n_{1},n_{2} \in N\) נקבל \(n_{1}+n_{2} \in N\).
- סגורה תחת כפל בסקלר - כלומר לכל \(r \in R\) ו-\(n \in N\) נקבל \(r\cdot n \in N\).

\end{definition}
\begin{remark}
זה בדיוק ההגדרה שהייתה לנו לתת מרחב ווקטורי. 

\end{remark}
\begin{remark}
בפרט מקיים את הנגדי כיוון שמתקיים:
$$(-1)\cdot n+n=(-1)n+1\cdot n=(-1+1)\cdot n=0\cdot n=0$$

\end{remark}
\begin{proposition}
אם \(I\) אידיאלי שמאלי אז בפרט תת חוג ולכן \(R^{I}\) היא מודול.

\end{proposition}
\begin{example}
יהי \(R\) חוג. ניתן לקחת עבור החוג \(M_{3}(R)\) והאידיאל השמאלי \(I=\begin{pmatrix}0& \star &\star\\0& \star &\star\\0& \star &\star\\\end{pmatrix}\) ולקבל את המודול\(M_{3}(R)^{\begin{pmatrix}0& \star &\star\\0& \star &\star\\0& \star &\star\\\end{pmatrix}}\). 

\end{example}
\begin{definition}[\(\mathbb{Z}\) מודול]
מודול של חבורה אבלית \(M\) מעל החוג \(\mathbb{Z}\) אשר כפל בסקלר מוגדר על ידי:
$$n\cdot m=(\underbrace{1+1+\cdot\cdot\cdot+1}_{n\;\mathrm{times}})\cdot m=\underbrace{m+m+\cdot\cdot\cdot+m}_{n\;\mathrm{times}}$$
עבור \(n> 0\) כאשר נגדיר \(0\cdot m = 0\) נקרא \(\mathbb{Z}\) מודול.

\end{definition}
\begin{proposition}
עבור \(\mathbb{Z}\) מודול הכפל מקיים:
$$(-n)\cdot m=-(m+\dots+m)$$

\end{proposition}
\begin{corollary}
לכל חבורה אבלית \(M\) יש מבנה יחיד כ-\(\mathbb{Z}\) מודול, כאשר הכפל מוגדר על ידי:
$$n\cdot m={\left\{\begin{array}{l l}{m+m+\cdot\cdot\cdot+m}&{n>0}\\ {0}&{n=0}\\ {-{\big(}m+m+\cdot\cdot\cdot+m{\big)}}&{n<0}\end{array}\right.}$$

\end{corollary}
\begin{remark}
זה מתקשר לכך שקיים חוג יחיד(כד כדי איזומורפיזם) של הקבוצה \(\mathbb{Z}\).

\end{remark}
\begin{definition}[מודול מנה]
אם \(N\) הוא תת מודול של \(M\) אז החוג מנה מותר בתור קבוצת הקוסטים:
$$M/N=\{m+N\mid m\in M\}$$
כאשר חיבור מוגדר על ידי:
$$(m+N)+(m^{\prime}+N)=(m+m^{\prime})+N$$
וכפל בסקלר מוגדר על ידי:
$$r\cdot(m+N)=(r\cdot m)+N$$

\end{definition}
נוכיח כי מוגדר היטב.

\begin{proof}
אם \(m+N=m'+N\) אז:
$$r m - r m' = r(\underbrace{ m - m' }_{ N })$$
ולכן:
$$r m +N= r m' +N$$
עבור \(m\mathbb{Z}\leq \mathbb{Z}\) מודול מנה מקבלים \(\mathbb{Z} / m\mathbb{Z}\).

\end{proof}
\begin{definition}[הומומורפיזם של מודולים]
פונקציה \(f:M\to N\) בין \(R\) מודולים אשר:
- משמרת חיבור - כלומר לכל \(m,m' \in M\) מתקיים:
$$f(m+m^{\prime})=f(m)+f(m^{\prime})$$
- משמרת כפל בסקלר, כלומר לכל \(r \in R\) ו-\(m \in M\) מתקיים:
$$f(r\cdot m)=r\cdot f(m)$$

\end{definition}
\begin{remark}
זה למעשה מה שקראנו לו טרנספורמצייה לינארית במקרה של מרחבים ווקטורים.

\end{remark}
\begin{proposition}[משפט האיזומורפיזם הראשון]
אם \(f:M\to N\) הוא הומומורפיזם של \(R\) מודולים אזי:
- הגרעין \(\ker f = \{ m \in M\mid f(m)=0 \}\) הוא תת מודול של \(M\).
- התמונה \(\mathrm{Im}f\) היא תת מודול של \(M\).
- קיים איזומורפיזם:
$$M/\ker f\cong\operatorname{Im}f$$
אשר נתון על ידי ההעתקה \(m+\ker f\mapsto f(m)\).

\end{proposition}
\begin{proposition}[משפט האיזומורפיזם השני]
אם \(N\) ו-\(M\) הם תתי מודולים של מודול \(L\) אזי:
$$\frac{N}{N\cap M}\cong\frac{N+M}{M}$$

\end{proposition}
\begin{proposition}[משפט האיזומורפיזם השלישי]
אם \(K\subseteq M \subseteq N\) הם תתי מודולים של מודול \(R\) אזי:
$${\frac{N/K}{M/K}}\cong{\frac{N}{M}}.$$

\end{proposition}
\begin{proposition}[משפט האיזומורפיזם הרביעי / משפט ההתאמה]
קיים התאמה חח"ע בין תתי מודולים של \(M / N\) לבין תתי מודולים של \(M\) אשר מכילות את \(N\).

\end{proposition}
\subsection{מודולים פשוטים}

\begin{definition}[מודול פשוט]
מגול \(M\) מעל \(R\) הוא פשוט אם התתי מודולים היחידים שלו הם 0 ו\(M\).

\end{definition}
\begin{proposition}
מודול \(R^{M}\) הוא פשוט אם \(M\neq 0\) ולכל תת מודול \(N\leq M\) מתקיים \(N=0\) או \(N=M\)

\end{proposition}
\begin{definition}[מודול ציקלי]
מודול \(M\) נקרא ציקלי אם קיים \(m \in M\) כך ש:
$$M=R\,m=\{\,r\,m:r\in R\}.$$
כאשר במקרה זה כותבים \(M=\langle m \rangle\).

\end{definition}
\begin{example}
החבורה הציקלית \(C_{8}= \mathbb{Z} / 8\mathbb{Z}\) היא \(\mathbb{Z}\)-מודול ציקלי הנוצרת על ידי 1.

\end{example}
\begin{proposition}
כל מודול ציקלי הוא מנה של בחוג. כלומר אם \(M=\langle m \rangle\) ניתן להגדיר:
$$f:R\,\longrightarrow\,M,\quad f(r)=r\,m$$

\end{proposition}
\begin{proof}
מהמשפט האיזומורפיזם הראשון מתקיים:
$$M\,\cong\,R/\ker f.$$
כאשר \(\ker f=\{\,r\in R:r\,m=0\}\) הוא אידיאל שמאלי של \(R\) ולכת תת מודול. לחלופין לכל אידיאל.

\end{proof}
\begin{remark}
זה מקביל לפעולה נאמנה בחבורות.

\end{remark}
\subsection{מודולים ומרחבים ווקטורים}

\begin{proposition}
אם \(M\) מודול נוצר סופית מעל שדה \(\mathbb{F}\) אז \(M\cong \mathbb{F}^{n}\).

\end{proposition}
\begin{proof}
מודול נוצר סופית מעל שדה הוא מרחב ווקטורי, וניתן תמיד למצוא בסיס \(\mathcal{B}\). ההעתקה \(v\mapsto[v]_{\mathcal{B}}\) היא איזומורפיזם ולכן נקבל כי איזומורפי ל-\(\mathbb{F}^{n}\).

\end{proof}
\begin{remark}
זה לא נכון עבור חוגים כלליים. למשל עבור \(R=\mathbb{Z}\) ו-\(M=\mathbb{Z} / 3\mathbb{Z}\) נקבל כי \(\mathbb{Z}\) לא איזומורפי ל-\(\mathbb{Z}^{m}\) כיוון ש-\(\mathbb{Z}^{n}\) הוא אינסופי אבל \(\mathbb{Z} / 3\mathbb{Z}\) הוא סופי.

\end{remark}
\begin{reminder}
במרחבים ווקטורים אם \(W\leq V\) הוא תת מרחב תמיד קיים משלים \(U\leq V\) כך ש-\(V=W\oplus U\).

\end{reminder}
\begin{remark}
זה לא נכון תמיד לגבי מודולים. למשל אם \(R=\mathbb{Z}\) ו-\(M=C_{4}\) אז עבור תת מודול \(N=\langle 2 \rangle=\{ 0,2 \}\) לא נקבל משלים. זאת כי התתי מודולים של \(M\) יהיו \(0,N,M\) ולא קיים משלים כיוון שאף אחד מהם לא מקיים \(M=N\oplus U\) עבור כל תת מודול \(U\).

\end{remark}
\begin{definition}[חוג פשוט למחצה]
חוג שלכל תת מודול שלו יש משלים.

\end{definition}
\begin{reminder}
במרחבים ווקטורים כל קבוצה פורשת מינימלית הוא בסיס, וכן כל קבוצה בת"ל מקסימלית תהיה בסיס.

\end{reminder}
\begin{remark}
לא תמיד נכון לגבי מודולים. למשל עבור \(R=\mathbb{Z},M=\mathbb{Z}\) הקבוצה \(\{ 2,3 \}\) פורשת את \(\mathbb{Z}\)(כיוון ש-\(3-2=1\)) אבל לא בת"ל:
$$3\cdot 2+(-2)\cdot 3 = 0$$
לעומת זאת הקבוצה \(\{ 2 \}\) היא בת"ל אבל לא פורשת את \(\mathbb{Z}\)(לא ניתן לקבל מספרים אי זוגיים).

\end{remark}
\begin{proposition}
מודול מעל \(\mathbb{F}[x]\) שקול למרחב ווקטורי \(V\) מעל \(\mathbb{F}\) עם אופרטור לינארי \(T:V\to V\) כך שהפעולה על \(x\) מוגדרת על ידי \(x \cdot v = T(v)\) והפעולה של פולינום תהיה מהצורה \(p(x)=a_{0}+a_{1}x+\dots + a_{n}x^{n}\) תהיה:
$$p(x)\cdot v=a_{0}v+a_{1}T(v)+\cdot\cdot\cdot+a_{n}T^{n}(v)$$
כלומר קיים הומומואפיזם \(\rho:\mathbb{F}[x]\to \mathrm{End}(V)\) אשר פועל כנאמר לעיל.

\end{proposition}
\begin{example}
עבור \(V=\mathbb{F}^{2}\) ו-\(T=\begin{pmatrix}0&1\\1& 0\end{pmatrix}\) נקבל:
$$x\cdot\left(v_{1},v_{2}\right)=(v_{2},v_{1}).$$
ועבור פילינום:
$$\left(x^{2}+1\right)\cdot\left(v_{1},v_{2}\right)=T^{2}(v)+v=\left(v_{1},v_{2}\right)+\left(v_{1},v_{2}\right)=\left(2v_{1},2v_{2}\right)$$

\end{example}
\subsection{מודולים חופשיים}

\begin{definition}[בסיס]
תת קבוצה \(S\subseteq M\) תקרא בסיס אם מקיימת:

  \begin{enumerate}
    \item פורשת - לכל \(m \in M\) ניתן לכתוב בצורה: 
$$m=\sum r_{i}s_{i}$$
כאשר \(s_{i}\in S,r_{i}\in R\).


    \item בת"ל - הצגה זו היא יחודית 


  \end{enumerate}
\end{definition}
\begin{definition}[מודול חופשי]
מודול שיש לו בסיס.

\end{definition}
\begin{definition}[סכום ישר]
עבור קבוצה \(S\) וחוג \(R\) נגדיר:
$$R^{\oplus S}=\{f\in R^{S}\mid f(s)=0\;{\mathrm{for~almost~all}}\;s\in S\}.$$
במילים אחרות כל איבר של \(R^{\oplus S}\) היא פונקציה מ-\(S\) ל-\(R\) שהיא לא אפס רק במספר סופי של ערכים.

\end{definition}
\begin{remark}
ההבדל בין הגדרה זו של סכום ישר ל-\(R^{S}\)(אוסף הפונקציות מ-\(f:S\to R\)) היא ש-\(R^{\oplus S}\) זה אוסף כל הפונקציות עם תומך סופי - כלומר מכיל מספר סופי של ערכים שינם אפס.

\end{remark}
\begin{example}[המודול \(\mathbb{R}^{\oplus \mathbb{N}}\)]
הסכום \(\mathbb{R}^{\oplus \mathbb{N}}\) מכיל את כל הסדרות מהצורה:
$$(a_{0},a_{1},\dots,a_{m}, 0,0,\dots)$$
כאשר הבסיס יהיה \(\{ e_{1} \}\) כאשר \(e_{i}(j)=\delta_{ij}\). כיוון שיש בסיס זה בפרט מודול חופשי.

\end{example}
\begin{example}[המודול \(\mathbb{Z}^{\mathbb{N}}\)]
עבור סופי אנו יודעים כי \(C_{3}=\mathbb{Z} / 3\mathbb{Z}\) - לא \(\mathbb{Z}\) מודול חופשי.
עבור המקרה הכללי הבעיה קצת יותר מסובכת. לפי Baer התשובה היא לא באופן כללי, אך לפי Specker כל תחבורה בת מנייה היא חופשית. בפרט ממשפט Specker מתקיים שאם \(\mathbb{Z}^{\oplus S}\cong \mathbb{Z}^{\mathbb{N}}\) אזי \(\lvert S \rvert> \aleph_{0}\).

\end{example}
\begin{definition}[אנדומורפיזמים של מודולים]
עבור \(R\) מודול \(M\) נגדיר:
$$\operatorname{End}_{R}(M)=\{\operatorname{All}\,R{\mathrm{-linear\;maps}}\;f:M\to M\}.$$
כאשר זה תת חוג של \(\mathrm{End}(M)\), האדומורפיזמים של \(M\) כחבורה אבלית

\end{definition}
\begin{example}[\(\mathrm{End}_{\mathbb{Z}}(\mathbb{Z})\)]
עבור \(\mathrm{End}_{\mathbb{Z}}(\mathbb{Z})\cong \mathbb{Z}\) לכל \(m\) ב-\(\mathbb{Z}\) נגדיר:
$$f_{m}(n)=mn$$
זה אנדומורפיזם:
$$\begin{gather}f_{m}(n+n')=m(n+n')=f_{m}(n)+f_{m}(n')\\ f_{m}(m'n)=m(m'n)=(mm')n=m'mn=m'f_{m}(n)
\end{gather}$$
מצד שני אם \(f \in \mathrm{End}_{\mathbb{Z}}(\mathbb{Z})\) אזי \(\varphi=f_{\varphi(1)}\) כי:
$$\varphi(n)=\varphi(n\cdot 1)=n\cdot \varphi( 1)=f_{\varphi(1)}(n)$$

\end{example}
\begin{corollary}
עבור חוג כללי \(R\) נקבל:
$$\mathrm{End}_{R}(R)\cong  R$$
לכל חוג קומוטטיבי \(R\).

\end{corollary}
\begin{proof}
זהה למה שעשינו בדוגמא עבור \(R=\mathbb{Z}\). יהי \(f \in \mathrm{End}_{R}(R)\). נגדיר \(f(1)=a\) ואז לכל \(r \in R\) נקבל:
$$f(r)=f(r\cdot1)=r\cdot f(1)=r\cdot a.$$
ולכן \(f\) מתאים לכפל ב-\(a\) ונקבל \(\mathrm{End}_{R}(R)\cong R\).

\end{proof}
\begin{proposition}
$$M_{n}(\mathbb{F} )=\mathrm{End}_{\mathbb{F} }(\mathbb{F} ^{n})$$
וכל \(\mathbb{F}\) מודולים נוצרים סופיים איזומורפיים ל-\(\mathbb{F}^{n}\) כלשהו.

\end{proposition}
\begin{remark}
זה עדין נכון אם מחלפים את \(\mathbb{F}\) עם \(M_{n}(\mathbb{F})\):
$$\mathbb{F} = \mathrm{End}_{M_{n}}(\mathbb{F}^{n} )$$

\end{remark}
\begin{definition}[אנטי הומומרפיזם של חוגים]
הומומורפיזם שבמקום שמשמר כפל הופך את הכפל, כלומר:
$$\Phi(rr')=\Phi(r')\Phi(r)$$

\end{definition}
\begin{reminder}
אם \(f \in \mathrm{End}_{R}(R)\) אז \(f\) נקבע על ידי \(f(1)\). כי:
$$\forall x \in R\quad f(x)=f(x\cdot 1)=xf(1)$$
בנוסף, לכל \(r \in R\) אפשר להגדיר \(f_{r}(x)=xr\). נשים לב כי \(f_{r}\in \mathrm{End}_{R}(R)\). זה כיוון שמתקיים:
$$f_{r}(r'x)=(r'x)r=r'(xr)=r'f_{r}(x)$$
סה"כ:
$$\Phi:R\to \mathrm{End}_{R}(R)\qquad \Phi(r)=f_{r}$$
וכן \(\Phi\) חח"ע ועל.

\end{reminder}
\begin{proposition}
ההעתקה \(\Phi:R\to \mathrm{End}_{R}(R)\) המוגדרת לעיל היא אנטי הומומואפיזם של חוגים.

\end{proposition}
\begin{proof}
$$\Phi(r+r')(x)=f_{r+r'}(x)=xr+xr'=(\Phi(r)+\Phi(r'))(x)$$
עבור הכפל:
$$\Phi(rr')(x)=f_{rr'}(x)=xrr'=(\Phi(r')\circ  \Phi(r))(x)$$
כלומר קיבלנו:
$$\Phi(rr')=\Phi(r')\Phi(r)$$

\end{proof}
\begin{definition}[חוג הפוך]
לחוג \(R\) החוג ההפוך \(R^{\text{op}}\) הוא \(R\) כקבוצה וכחבורה חבורה אבל עם הכפל:
$$x\cdot_{R^{op}}y := y\cdot_{R} x$$

\end{definition}
\begin{corollary}
קיבלנו:
$$\Phi:R^{\text{op}}\xrightarrow{\cong }\mathrm{End}_{R}(R)$$

\end{corollary}
\begin{definition}[מודול נאמן]
מודול \(R^{M}\) נקרא האמן אם הומומורפיזם המבנה \(\rho:R\to \mathrm{End_{Ab}}(B)\) הוא חח"ע. באופן שקול אם לכל \(0\neq r \in R\) קיים \(m \in M\) כך שמתקיים \(r m \neq 0\).

\end{definition}
\begin{lemma}
לחוג חילוק אין אידיאלים שמאליים אם"ם חוג חילוק.

\end{lemma}
\begin{proof}
לכל \(0\neq r \in R\) יש הופכי משמאל. כי \(\langle r \rangle=Rr=R\) וזה אומר שיש הופכי משמאל. נסמן ב-\(\bar{r}\) להופכי משמאל. נדרש רק להראות כי יש הופכי מימין כדי להראות שחוג חילוק. נראה ש-\(r\bar{r}=1\). נסתכל עליו בריבוע:
$$(r\bar{r})^{2}=r\cancel{ \bar{r}r }\bar{r}=r\bar{r}$$
בנוסף \(r\bar{r}\neq 0\) כי \(\bar{r}r\bar{r}r=1\) ולכן ייתכן כי \(r\bar{r}\) מתאפס.
לכן:
$$\overline{r\bar{r}} (r\bar{r})^{2}=\overline{r\bar{r}} r\bar{r}\implies r\bar{r}=1$$

\end{proof}
\begin{lemma}
אפשר להרחיב קבוצה בת"ל שאינה פורשת לבסיס.

\end{lemma}
\begin{proof}
  \begin{itemize}
    \item אם \(m\neq \langle m_{1},\dots,m_{n} \rangle\) אזי \(m_{1},\dots,m_{n},m\) בת"ל כי אם:
$$a_{1}m_{1}+\dots+a_{n}m_{n}+am=0$$
    \item כעת אם \(a=0\) נקבל כי לא אפשרי כי \(a_{1},\dots,a_{n}\) בת"ל.
    \item אם \(a\neq 0\) נקבל:
$$m=-\frac{a_{1}}{a}m_{1}+\dots-\frac{a_{n}}{a}m_{n}$$
סתירה.
  \end{itemize}
\end{proof}
\begin{proposition}
התנאים הבאים שקולים:

  \begin{enumerate}
    \item החוג \(R\) הוא חוג חילוק. 


    \item אין ל-\(R\) אידיאלים שמאליים. 


    \item אין ל-\(R\) אידיאלים ימניים. 


    \item החוג \(R\) פשוט כ-\(R\) מודול. 


    \item כל \(R\) מודול הוא חופשי(כלומר לכל \(R\) מודול יש בסיס) 


  \end{enumerate}
\end{proposition}
\begin{proof}
את 1-4 הראנו. נותר להראות רק ש-5 שקול לקודמים.
נראה חוג חילוק גורר כי בסיס כמו מעל שדה(קבוצה בת"ל שאינה פורשת אפשר להרחיב לקבוצה בת"ל.
אחרי שחוזרים על זה מספיק פעמיים(אינסוף כלשהו - ייתכן וידרש אינדוקציה טרנספיניטית), מגיעים לבסיס).
אם כל \(R\) מודול הוא חופשי, ו-\(0 \neq r\in R\). 
נניח בשלילה שיש \(0\lneq I\lneq R\)(אידיאל שמאלי) ניקח \(M= R / I\). אזי \(1+I\) יוצר אם \(M\) ולכן בסיס. נשלים אולי בהפסקה.
נראה כעת כי אם \(R\) מודול חופשי אז \(R\) חוג חילוק. ניקח \(M\neq 0\) מודול פשוט. (למשל \(M = R / I\) כש-\(I\leq R\) אידיאל שמאלי מקסימלי - קיים מהלמה של צורן).
ניקח בסיס \(B\) ל-\(M\). ניקח \(b\in B\) ונסתכל על:
$$f:R\to M; r\mapsto rb$$
הומומורפיזם של מודולים. \(f\) על כי \(M\) פשוט ו-\(0\neq b \in \mathrm{Im} f\) ולכן \(B=\{ b \}\). בנוסף כיוון ש-\(B\) בסיס אז חופשי ולכן \(f\) חח"ע. לכן \(R\cong M\) כמודול. לכן אין ל-\(R\) תת מודולים - כלומר אין אידיאלים שמאליים ולכן חוג חילוק.

\end{proof}
\subsection{סופיות}

\begin{definition}[מודול נוצר סופית]
מודל שנוצר על ידי איזשהי קבוצה סופית כלשהי:
$$M=\langle m_{1},\dots,m_{n} \rangle $$
עבור \(m_{1},\dots ,m_{n}\) כלשהם.

\end{definition}
\begin{definition}[מודול נתרי - Noetherian]
מודול \(M\) נקרא נתרי אם כל תת מודול של \(M\) נוצר סופית.

\end{definition}
\begin{example}[מודול לא נתרי]
המודול \(R=M=\mathbb{F}[x_{1},x_{2},\dots]\) הוא לא נתרי. זאת כיוון שלמרות שהחוג כולו מיוצר על ידי 1 האידיאל:
$$I=(x_{1},x_{2},\dots)$$
הוא לא נוצר סופית, כיוון שאם היה אז מספר סופי של פולינומים היה מייצר את כול המונומים, אשר לא אפשרי כי תמיד נתין להוסיף עוד משתנה.

\end{example}
\begin{proposition}
הפסוקים הבאים שקולים:

  \begin{enumerate}
    \item המודול \(M\) נתרי. 


    \item כל סדרה עולה של תתי מודולים של \(M\) מתייצבת, כלומר לא ייתכן: 
$$M_{1} \lneq M_{2}\lneq \dots \lneq \dots \lneq M$$


    \item לכל אוסף של תתי מודולים יש איבר מקסימלי. 


  \end{enumerate}
\end{proposition}
\begin{proof}
מ-2 ל-3 ניתן לעבור מהלמה של צורן.
אם יש סדרה עולה ממש:
$$M_{1} \subsetneq M_{2} \subsetneq \dots$$
ניקח \(N=\bigcup_{i=1}^{\infty}M_{i}\) ונראה שלא נוצר סופית - אם הוא היה ב-\(M_{j}\) כלשהו כל היוצרים כבר מופיעים ולכן הסדרה כבר לא עולה. ואם \(N\leq M\) לא נוצר סופית ניקח \(n_{1},n_{2},n_{3},\dots\) כש-\(n_{j}\not\in \langle n_{1},\dots,n_{j-1} \rangle\) ואז:
$$N_{j}=\langle n_{1},\dots n_{j} \rangle $$
עולה ממש.

\end{proof}
\begin{example}
המודול \(\mathbb{Q}\) הוא לא \(\mathbb{Z}\)-מודול נתרי:
$$\mathbb{Z}\subsetneq \frac{1}{2}\mathbb{Z} \subsetneq \frac{1}{4}\mathbb{Z} \subsetneq \dots$$
כאשר \(\mathbb{Z}\) כן \(\mathbb{Z}\) מודול נתרי:
$$m_{1}\mathbb{Z} \subsetneq m_{2}\mathbb{Z} \subseteq \dots $$
כאשר נדרש \(m_{2}\) כך שמחלק את \(m_{1}\). בסופו של דבר נתייצב כי כל \(m_{i}\) צריך לחלק את כל ה-m-ים הקודמים.

\end{example}
\begin{definition}[מודול ארטיני]
מודול נקרא ארטיני אם אין סדרה אינסופית ירדת ממש של תתי מודולים. כלומר כל שרשרת יורדת מתייצבת.

\end{definition}
\begin{remark}
ארטיני זה תנאי חזק. יותר חזק מנתרי, מאוד מגביל (אומנם זה לא פרמלית חזק יותר כי ייתכן נתרי ולא ארטני).

\end{remark}
\begin{example}
למשל \(\mathbb{Z}^{\mathbb{Z}}\) לא ארטיני:
$$\mathbb{Z}\supset 2\mathbb{Z} \supset 4\mathbb{Z}\supset \dots$$

\end{example}
\begin{example}
נסתכל על \(\mathbb{Z}\left[ \frac{1}{2} \right] / \mathbb{Z}\) כ-\(\mathbb{Z}\) מודול. זה ארטיני ולא נתרי:
$$\left\langle  \frac{1}{2}  \right\rangle \subset \left\langle  \frac{1}{4}  \right\rangle \subset \left\langle  \frac{1}{8}  \right\rangle \subset \dots$$
כל תת מודול הוא \(\left\langle  \frac{1}{2^{n}}  \right\rangle\) ל-\(n\) כלשהו. ואז:
$$\left\langle  \frac{1}{2^{n_{1}}}  \right\rangle \supset \left\langle  \frac{1}{2^{n_{2}}}  \right\rangle\supset $$
כאשר \(n_{2}< n_{1}\) וזה חייב להעצר. ניתן לבנות דוגמא דומה על נתרי ולא ארטני.

\end{example}
\begin{definition}[מאורך סופי]
אם \(M\) גם נתרי וגם ארטיני הוא נקרא מאורך סופי.

\end{definition}
\begin{proposition}
אם \(M\) מאורך סופי אזי יש סדרה:
$$0=M_{n}\leq  \dots \lneq M_{2} \lneq  M_{1}=M$$
כך ש-\(M_{i} / M_{i+1}\) פשוט לכל \(i\).  זה למעשה סדרת הרכב.

\end{proposition}
\begin{remark}
השם מאורך סופי בא מכך שניתן להגדיר אורך:
$$\mathrm{{length}}\left(M\right)=\operatorname*{sup}\left\{\ell\,|\,0=M_{0}\lneq M_{1}\lneq\ldots\lneq M_{\ell}=M\right\}$$
ולקבל מהטענה כי סופי אם"ם ארטני וגם נתרי.

\end{remark}
\begin{theorem}[ג'ורדן הולדר]
אם יש ל-\(M\) סדרת הרכב אזי לכל 2 סדרות הרכב יש את אותן מנות(עד כדי סדר ואיזומורפיזם). 

\end{theorem}
\begin{corollary}
אם \(M\) מאורך סופי אז יש סדרת הרכב.

\end{corollary}
\begin{example}
$$(0)\overset{C_{2}}{<}(6)\overset{C_{3}}{<}  (2)\overset{C_{2}}{<}   C_{12}$$
או לחלופין ניתן לבנות:
$$(0)\overset{C_{2}}{<} (6) \overset{C_{2}}{<} (3)\overset{C_{3}}{<} C_{12}$$

\end{example}
\begin{proposition}
כל מודול נוצר סופית הוא מנה של מודול חופשי עם דרגה סופית.

\end{proposition}
\begin{proof}
אם מודול \(M=\langle m_{1},\dots,m_{n} \rangle\) נוצר סופית ניתן להגדיר הומומואפיזם:
$$f:R^{n}\rightarrow{M},\quad f(r_{1},\ldots,r_{n})=r_{1}m_{1}+\cdot\cdot\cdot+r_{n}m_{n}$$
כאשר ממשפט האיזומורפיזם הראשון כיוון ש-\(f\) היא על נקבל:
$$M\cong R^{n}/\ker f$$

\end{proof}
\begin{lemma}
אם מודול \(M\) הוא נתרי ו-\(f:M\to M\) הומומורפיזם אז קיים \(n \in \mathbb{N}\) כך ש:
$$\ker (f^{n})\cap \mathrm{\mathrm{Im}}(f^{n})=\{ 0 \}$$

\end{lemma}
\begin{corollary}
אם \(M\) נתרי ו-\(f:M\to M\) הומומורפיזם על אז \(f\) חח"ע, ובפרט איזומורפיזם.

\end{corollary}
\begin{proof}
נובע מכך שאם \(M=\mathrm{Im}(f^{n})\) נקבל \(\ker(f^{n})\cap M=\{ 0 \}\) ולכן \(\ker(f^{n})=\{ 0 \}\).

\end{proof}
\begin{lemma}
אם מודול \(M\) הוא ארטני ו-\(f:M\to M\) הומומורפיזם אז קיים \(n \in \mathbb{N}\) כך ש:
$$\ker (f^{n})+\mathrm{Im}(f^{n})=M$$

\end{lemma}
\begin{corollary}
אם \(M\) נתרי ו-\(f:M\to M\) הומומורפיזם חח"ע אז \(f\) על, ובפרט איזומורפיזם.

\end{corollary}
\begin{proposition}[פירוק פיטינג]
אם \(M\) מאורך סופי ו-\(f:M\to M\) הומומורפיזם, אז קיים \(N,I\) כך ש-\(f|_{N}\) יהיה נילפוטנטי(קיים \(n\) כך ש-\(f|_{N}^{n}\) מתאפס) ו-\(f|_{I}\) אוטומורפיזם כך שמתקיים:
$$N\oplus I = M$$

\end{proposition}
\begin{proof}
אם \(M\) מאורך סופי אז נתרי וגם ארטני ובפרט מתקיים:
$$\ker (f^{n})\cap \mathrm{\mathrm{Im}}(f^{n})=\{ 0 \}\qquad \ker (f^{n})+\mathrm{Im}(f^{n})=M$$
כלומר \(\ker (f^{n})\oplus \mathrm{\mathrm{Im}}(f^{n})=M\). 
אנו יודעים כי \(f|_{\ker f^{n}}\) הוא נילפוטנטי כיוון ש-\(f|_{\ker f^{k}}^{k}=0\) וכן כדי שאנדומרופיזם \(f|_{\mathrm{Im}f^{k}}\) יהיה אוטומורפיזם נדרש שיהיה הפיך. זה אכן חח"ע כי בסכום ישר עם הגרעין ולכן הגרעין אפס, וכיוון שאנחנו מסתכלים על הצמצום לתמונה נקבל כי על ולכן סה"כ אוטומורפיזם כנדרש.

\end{proof}
\begin{definition}[חוג נתרי/ארטני]
חוג נקרא נתרי / ארטיני אם הוא כזה כמודול מעל עצמו.

\end{definition}
\begin{theorem}[הופקינס לויצקי]
חוג ארטיני הוא נתרי.

\end{theorem}
\begin{proposition}
יהי \(N\leq M\) תת מודול. אזי \(M\) נטרי אם"ם \(N\) ו-\(M / N\) נטרי

\end{proposition}
\begin{proof}
אם \(M\) נטרי ו-\(N\leq M\) אז \(N\) נטרי(כי \(L\leq N \leq M\)). כמו כן, \(M / N\) נטרי:
אם \(L\leq M / N\) אזי ממשפט ההתאמה יש \(\overline{L}\leq M\)(ספציבית \(\overline{L}=\bigcup_{\ell \in L}\ell\)) כך ש-\(L = L / N\). אזי \(\overline{L}\) נוצר סופית על ידי \(\ell_{1},\dots,\ell_{n}\) ואז \(L=\langle \ell_{1}+N, \dots,\ell_{n}+N \rangle\).
אם \(N\) ו-\(M / N\) נטריים:
עבור \(L\leq M\) נקבל מאיזו 2:
$$\frac{L+N}{N}\overset{II}{\cong } \frac{L}{N\cap  L}$$
(כאשר \(\frac{L+N}{N}\) תת מודול של \(\frac{M}{ N}\) לכן נוצר סופית) לכן \(\frac{L}{N \cap L}\) נוצר סופית. כמו כן \(N\cap L\) נוצר סופית(כי \(N\) נטרית) ולכן \(L\) נוצר סופית(על ידי יוצרים של \(N\cap L\) + יוצרים של \(\frac{L}{N\cap L}\)).

\end{proof}
\begin{remark}
נוצרות סופית לא מתנהגת כזה יפה, זה הופך את נתרי לסוג של תיקון של נוצר סופית.

\end{remark}
\begin{corollary}
אם \(R\) נתרי(חוג נתרי = חוג כמודול מעל עצמו = כל אידיאל שמאלי נוצר סופית) אז ל-\(R\) מודולים נ"ס = נתרי

\end{corollary}
\begin{proof}
תמיד נתרי גורר נוצר סופית. אם \(R\) נתרי גם \(R^{n}\) נתרי לכל \(n\).
$$\frac{R\oplus R}{R\oplus 0}\cong R\cong R\oplus 0 \leq  R\oplus R$$
ואז גם \(R^{n} / N\) נתרי לכל \(N\leq R^{n}\) וכל \(R\) מודול נוצר סופית איזומורפי ל-\(R^{n} / N\) לאיזשהו \(n \in \mathbb{N}\) ו-\(N\leq R^{n}\) כלשהם.

\end{proof}
\begin{proposition}[משפט הבסיס של הילברט]
אם \(R\) חוג נתרי אזי \(R[x]\) נתרי. בפרט \(R[x_{1},\dots,x_{n}]\) נתרי.

\end{proposition}
\begin{remark}
היום מה שנקרה בסיסי Grobner זו הדרך הנכונה לעשות את זה.

\end{remark}
\subsection{אנדומורפיזמים}

\begin{proposition}
עבור מודול \(R^{M}\) מתקיים האנדומורפיזמים של המודול יהיה תת חוג של האנדומורפיזמים של החוג. כלומר:
$$\operatorname{End}_{R}(M)=\left\{f:M\to M\mid f(r m)=r f(m)\,{\mathrm{for~all}}\,r\in R,m\in M\right\}\leq \mathrm{End}(M)$$

\end{proposition}
\begin{example}[מרחבים ווקטורים]
$${\mathrm{End}}_{\mathbb{F}}{\bigl(}\mathbb{F}^{n}{\bigr)}=M_{n}{\bigl(}\mathbb{F}{\bigr)}$$

\end{example}
\begin{example}[מודולים מעל חוג מטריצות]
$$\operatorname{End}_{M_{n}(\mathbb{F})}(\mathbb{F}^{n})=\mathbb{F}$$

\end{example}
\begin{example}
$$\operatorname{End}_{R}(R)=R^{\mathrm{op}}$$
כש-\(R^{\text{op}}\) היא \(R\) עם כפל הפוך. לכל \(r \in R\) הגדרנו \(m_{r}(x)=xr\), וראינו \(m_{r} \in \mathrm{End}_{R}(R)\) כאשר \(r \mapsto m_{r}:R\to \mathrm{End}_{R}(R)\) חח"ע ועל, מכבדת חיבור:
$$m_{r+r'}=m_{r}+m_{r'}$$
והופכת כפל:
$$m_{r r'}= m_{r'}\circ m_{r}$$

\end{example}
\begin{proposition}[הלמה של שור]
אם \(R^{M}\) מודול פשוט אזי \(\mathrm{End}_{R}(M)\) חוג חילוק.

\end{proposition}
\begin{proof}
אם \(0\neq f \in \mathrm{End}_{R}(M)\) אזי \(\mathrm{\mathrm{Im}}f\neq \{ 0 \}\) ולכן \(\mathrm{Im} f = M\) ולכן \(f\) על.
אם \(\ker f \neq M\) אז \(\ker f = \{ 0 \}\) ואז \(f\) חח"ע.
זה גורר \(f\) הפיך כפונקציה מ-\(M\) ל-\(M\). ואז \(f ^{-1} \in \mathrm{End}_{R}(M)\).
$$f^{-1} (m+n)=f^{-1} (f(f^{-1} (m))+f(f^{-1} (n)))=f^{-1} (f(f^{-1} (m)+f^{-1} (n)))$$

\end{proof}
\subsection{חוג הפוך}

\begin{definition}[אנטי הומומרפיזם של חוגים]
הומומורפיזם שבמקום שמשמר כפל הופך את הכפל, כלומר:
$$\Phi(rr')=\Phi(r')\Phi(r)$$

\end{definition}
\begin{reminder}
אם \(f \in \mathrm{End}_{R}(R)\) אז \(f\) נקבע על ידי \(f(1)\). כי:
$$\forall x \in R\quad f(x)=f(x\cdot 1)=xf(1)$$
בנוסף, לכל \(r \in R\) אפשר להגדיר \(f_{r}(x)=xr\). נשים לב כי \(f_{r}\in \mathrm{End}_{R}(R)\). זה כיוון שמתקיים:
$$f_{r}(r'x)=(r'x)r=r'(xr)=r'f_{r}(x)$$
סה"כ:
$$\Phi:R\to \mathrm{End}_{R}(R)\qquad \Phi(r)=f_{r}$$
וכן \(\Phi\) חח"ע ועל.

\end{reminder}
\begin{proposition}
ההעתקה \(\Phi:R\to \mathrm{End}_{R}(R)\) המוגדרת לעיל היא אנטי הומומואפיזם של חוגים.

\end{proposition}
\begin{proof}
$$\Phi(r+r')(x)=f_{r+r'}(x)=xr+xr'=(\Phi(r)+\Phi(r'))(x)$$
עבור הכפל:
$$\Phi(rr')(x)=f_{rr'}(x)=xrr'=(\Phi(r')\circ  \Phi(r))(x)$$
כלומר קיבלנו:
$$\Phi(rr')=\Phi(r')\Phi(r)$$

\end{proof}
\begin{definition}[חוג הפוך]
לחוג \(R\) החוג ההפוך \(R^{\text{op}}\) הוא \(R\) כקבוצה וכחבורה חבורה אבל עם הכפל:
$$x\cdot_{R^{\mathrm{op}}}y := y\cdot_{R} x$$

\end{definition}
\begin{corollary}
קיבלנו:
$$\Phi:R^{\text{op}}\xrightarrow{\cong }\mathrm{End}_{R}(R)$$\textbf{דוגמא}
נקבל כי \(f(A)=A^{T}\) הוא איזומורפיזם \(f:M_{R}(\mathbb{F})\to M_{n}(\mathbb{F})^{\mathrm{op}}\).
$$f(I)=I\qquad f(A+B)=fA+fB\qquad f(AB)=f(A)\cdot f(B)=f(B)f(A)$$

\end{corollary}
\begin{remark}
אם \(R\) קומוטטיבי אז \(R\cong R^{\text{op}}\). כמו כן בחבורות תמיד מתקיים \(R\cong R^{\text{op}}\) כאשר באותו אופן \(M_{n}(R)^{\text{op}}\cong M_{n}(R^{\text{op}})\) בעזרת האיזומורפיזם \(A\mapsto A^{T}\). אבל לא לא תמיד מתקיים עבור מודולים.

\end{remark}
\begin{proposition}[אנדומורפיזמים של מודולים חופשיים]
נתון \(R\) מודול חופשי \(M=R^{n}\) מדרגה \(n\). אזי:
$$\operatorname{End}_{R}(R^{n})\cong M_{n}(R^{\mathrm{op}})$$

\end{proposition}
\begin{proof}
כפל ב-\(R\) וב-\(M_{n}(R)\) נסמן ב- \(\cdot\) כאשר כאפל ב-\(R^{\text{op}}\) וב-\(M_{n}(R^{\text{op}})\) יסומן על ידי \(\times\).
המודול \(M_{n}(R)^{\text{op}}\) פועל על \(R^{n}\) כאנדומורפיזם \(\Psi:M_{n}(R^{\text{op}})\to \mathrm{End}_{R}(R^{n})\) על ידי: 
$$A\times v = \begin{pmatrix}v_{1}a_{11}+\dots+v_{1}a_{1n} \\\vdots \\v_{n}a_{n_{1}}+\dots+v_{n}a_{nn}
\end{pmatrix}=(v^{T}A^{T})^{T}$$
כאשר אם החוג היה קומוטטיבי זה היה פשוט \(Av\). כעת נראה כי \(R\)-לינארי:

  \begin{enumerate}
    \item הומוגניות: 
$$A\times(r\,v)=\left((r\,v)^{T}A^{T}\right)^{T}=\left(r\,v^{T}A^{T}\right)^{T}=r\left(v^{T}A^{T}\right)^{T}=r\,(A\times v)$$


    \item אדיטיביות: 
$$A\times(v+w)=\left((v+w)^{T}A^{T}\right)^{T}=\left((v^{T}+w^{T})A^{T}\right)^{T}=\left(v^{T}A^{T}+w^{T}A^{T}\right)^{T}=(A\times v)+(A\times w)$$
ולכן \(\Psi(A)\in \mathrm{End}_{R}(R^{n})\). נראה כי הומומואפיזם:


    \item משמר חיבור: 
$$\Psi(A+B)(v)=(A+B)\times v=A\times v+B\times v=(\Psi(A)+\Psi(B))(v)$$


    \item משמר כפל(עם היפוך): 
$$(A\star B)_{i j}=\sum_{k=1}^{n}a_{i k}\star b_{k j}=\sum_{k=1}^{n}b_{k j}\,a_{i k}$$
כאשר:
$$\Psi(A\star B)(v)=(A\star B)\times v=\left(v^{T}(A\star B)^{T}\right)^{T}=\left(v^{T}(B^{T}A^{T})\right)^{T}=A\times(B\times v)=(\Psi(A)\circ\Psi(B))(v)=(\alpha\star B)$$
ולכן \(\Psi(A\star B)=\Psi(A)\circ\Psi(B).\) כלומר \(\Psi\) הומומורפיזם של חוגים. נראה כי על. יהי \(f\in \mathrm{End}_{R}(R^{n})\). נמצא \(A \in M_{n}(R^{\text{op}})\) עם \(\Psi(A)=f\). יהי \(\{ e_{1},\dots,e_{n}  \}\) הבסיס הסטנדרטי של \(R^{n}\). עבור כל \(j\) נכתוב:
$$f(e_{j})\;=\;\sum_{i=1}^{n}a_{i j}\,e_{i},\quad a_{i j}\in R.$$
נגדיר את המטריצה \(A=(a_{ij})\) וכעת עבור \(v=\sum_{j}v_{j}e_{j}\) נקבל:
$$f(v)=\sum_{j}v_{j}f(e_{j})=\sum_{j}v_{j}\sum_{i}a_{i j}e_{i}=\sum_{i}\Bigl(\sum_{j}v_{j}a_{i j}\Bigr)e_{i}=A\times v$$
ולכן \(f=\Psi(A)\). הייחודיות של \(a_{ij}\) מראה ש-A ייחודי. נראה כי חח"ע. אם \(\Psi(A)=0\) אזי \(A\times e_{j}=0\) לכל \(j\) אבל:
$$A\times e_{j}=\begin{pmatrix}a_{1j}\\ a_{2j}\\ \vdots\\a_{nj}
\end{pmatrix}$$
לכן כל עמודה מתאפסת ו-\(A=0\). כלומר קיבלנו כי הומומואפיזם חח"ע ועל ולכן איזומומורפיזם של חוגים.


  \end{enumerate}
\end{proof}
\end{document}