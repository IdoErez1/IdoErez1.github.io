\documentclass{tstextbook}

\usepackage{amsmath}
\usepackage{amssymb}
\usepackage{graphicx}
\usepackage{hyperref}
\usepackage{xcolor}

\begin{document}

\title{Example Document}
\author{HTML2LaTeX Converter}
\maketitle

\section{מרחבים טופולוגיים}

\subsection{מרחבים טופולוגיים}

\begin{definition}[טופולוגיה]
טופולוגיה על קבוצה \(X\) היא אוסף \(\tau\) של תתי קבוצות של \(X\)(אשר נקראות קבוצות פתוחות) אשר מקיימות:

  \begin{enumerate}
    \item מכילות את הקבוצה כולה והקבוצה הריקה, כלומר \(X,\varnothing  \in \tau\). 


    \item סגירות תחת חיתוכים סופיים, כלומר: 
$$\begin{array}{r l}{A,B\in\tau}&{{}\Rightarrow\quad A\cap B\in\tau.}\end{array}$$


    \item סגירות תחת איחודים כלשהן: 
$$\{U_{i}\}_{i\in I}\subseteq\tau\implies\bigcup_{i\in I}U_{i}\in\tau$$


  \end{enumerate}
\end{definition}
\begin{example}[הטופולוגיה הטריוויאלית]
לכל קבוצה \(X\) ניתן להגדיר את הטופולוגיה הטריווילאית על ידי \(\tau=\{ \varnothing,X \}\).

\end{example}
\begin{example}[הטופולוגיה הדיסקרטית]
לכל קבוצה \(X\) ניתן להגדיר את הטופולוגיה הדיסקרטית על ידי \(\tau=\mathcal{P}(X)\).

\end{example}
\begin{example}[הטופולוגיה הסטנדרטית על \(\mathbb{R}\)]
קבוצה \(U\subseteq \mathbb{R}\) תהיה פתוחה אם לכל \(x \in U\) קיים \(\epsilon>0\) כך שהכדור הפתוח מקיים:
$$B_{\varepsilon}(x)=(x-\varepsilon,x+\varepsilon)\subseteq U$$

\end{example}
\begin{example}[טופולוגיית המשלים הסופית]
לכל קבוצה \(X\) ניתן להגדיר את הטופולוגיה:
$$\tau=\{A\subseteq X\mid X\setminus A\,\mathrm{is\,finite}\}\cup\{\varnothing \}$$

\end{example}
\begin{example}[טופולוגית המשלים הבן מנייה]
לכל קבוצה \(X\) ניתן להגדיר את הטופולוגיה:
$$\tau=\{A\subseteq X\mid X\setminus A\,\mathrm{is\;countable}\}\cup\{\varnothing \}.$$

\end{example}
\subsection{בסיס של טופולוגיה}

\begin{definition}[בסיס של טופולוגיה]
בסיס \(\mathcal{B}\) של טופולוגיה על \(X\) זו אוסף של תתי קבוצות על \(X\) אשר מקיימות:

  \begin{enumerate}
    \item כיסוי של \(X\). כלומר מתקיים: 
$$\bigcup_{B\in{\mathcal{B}}}B=X$$


    \item אם \(x \in B_{1}\cap B_{2}\) עבור \(B_{1},B_{2}\in \mathcal{B}\) אזי קיים \(B_{3} \in \mathcal{B}\) כך שמתקיים: 
$$x\in B_{3}\subseteq B_{1}\cap B_{2}$$


  \end{enumerate}
\end{definition}
\begin{definition}[טפולוגיה הנוצרת על ידי בסיס]
הטופולוגיה הנוצרת על ידי בסיס \(\mathcal{B}\) תהיה הטופולוגיה שמתקבלת על ידי איחוד של כל איבר בבסיס:
$$\tau=\left\{\bigcup S\mid S\subseteq\mathcal{B}\right\}$$

\end{definition}
\begin{example}[הטופולוגיה הדיסקרטית]
הטופולוגיה שבה כל יחידון הוא איבר בבסיס היא יוצרת את הטופולגיה הדיסקרטית, כיוון שניתן להרכיב כל קבוצה על ידי איחוד של יחידונים.

\end{example}
\begin{example}[הבסיס האוקלידי]
עבור \(\mathbb{R}^{n}\) האוסף של כל הכדורים הפתוחים מהווה בסיס לטופולוגיה הסטנדרטית

\end{example}
\begin{example}[בסיס בן מנייה עבור \(\mathbb{R}^{n}\)]
עבור \(\mathbb{R}^{n}\) אוסף הכדורים הפתוחים אשר מרכזם בנקודות רציונאליות ורדיוסים רציונאלי.

\end{example}
\begin{definition}[תת בסיס]
תת בסיס \(\mathcal{S}\) על טופולוגיה \(X\) הוא אוסף של תתי קבוצות כך שהחיתוך הסופי של האיברים יוצר בסיס:
$${\mathcal{B}}=\{B_{1}\cap B_{2}\cap\cdots\cap B_{n}\mid n\in\mathbb{N},B_{i}\in{\mathcal{S}}\}.$$
כך שהטופולוגיה הנוצרת על ידי \(\mathcal{S}\) תהיה:
$$\tau=\left\{\bigcup S\mid S\subseteq\mathcal{B}\right\}$$

\end{definition}
\begin{definition}[טופולוגייה סדר]
יהי \(P\) קבוצה סדורה קווית(כלומר ניתן להשוואות כל שתי איברים). אזי טופולוגיית הסדר על \(P\) תהיה הטופולוגיה שנוצרת על ידי הבסיס:
- הקטעים הפתוחים:
$$(a,b)=\{x\in P\mid a<x<b\}.$$
- אם ל- \(P\) יש מינימום \(m\) או מקסימום \(M\) אזי גם:
$$[m,b)=\{x\in P\mid m\leq x<b\},\quad(a,M]=\{x\in P\mid a<x\leq M\}$$

\end{definition}
\end{document}