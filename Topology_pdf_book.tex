\documentclass{tstextbook}

\usepackage{amsmath}
\usepackage{amssymb}
\usepackage{graphicx}
\usepackage{hyperref}
\usepackage{xcolor}

\begin{document}

\title{Example Document}
\author{HTML2LaTeX Converter}
\maketitle

\chapter{מרחבים טופולוגיים}

\section{מרחבים טופולוגיים}

\begin{definition}[טופולוגיה]
טופולוגיה על קבוצה \(X\) היא אוסף \(\tau\) של תתי קבוצות של \(X\)(אשר נקראות קבוצות פתוחות) אשר מקיימות:

  \begin{enumerate}
    \item מכילות את הקבוצה כולה והקבוצה הריקה, כלומר \(X,\varnothing  \in \tau\). 


    \item סגירות תחת חיתוכים סופיים, כלומר: 
$$\begin{array}{r l}{A,B\in\tau}&{{}\Rightarrow\quad A\cap B\in\tau.}\end{array}$$


    \item סגירות תחת איחודים כלשהן: 
$$\{U_{i}\}_{i\in I}\subseteq\tau\implies\bigcup_{i\in I}U_{i}\in\tau$$


  \end{enumerate}
\end{definition}
\begin{definition}[קבוצה פתוחה]
קבוצה \(U\subseteq X\) פתוחה אם \(U\in \tau\).

\end{definition}
\begin{example}[טופולוגיה של שלושה איברים]
נמפה את כל הטופולגיות על שלושה איברים:

 Created with Inkscape (http://www.inkscape.org/) \includegraphics[width=0.8\textwidth]{diagrams/svg_1.svg}
\end{example}
כאשר נקודה מייצגת איבר ועיגולים מייצגים קבוצות פתוחות. 

\begin{definition}[טופולגיה עדינה]
טופולוגיה \(\tau_{1}\) עדינה מ-\(\tau_{2}\) אם כל קבוצה פתוחה ב-\(\tau_{2}\) פתוחה גם ב-\(\tau_{1}\)(כלומר \(\tau_{2}\subseteq \tau_{1}\)).

\end{definition}
\begin{definition}[טופולוגיה גסה]
טופולוגיה \(\tau_{1}\) גסה מ-\(\tau_{2}\) אם כל קבוצה פתוחה ב-\(\tau_{1}\) פתוחה ב-\(\tau_{2}\)(כלומר \(\tau_{1}\subseteq \tau_{2}\)).

\end{definition}
\begin{remark}
לא כל שתי טופולוגיות על מרחב ניתנות להשוואה.

\end{remark}
\begin{example}[הטופולוגיה הטריוויאלית]
לכל קבוצה \(X\) ניתן להגדיר את הטופולוגיה הטריווילאית על ידי \(\tau=\{ \varnothing,X \}\).

\end{example}
\begin{example}[הטופולוגיה הדיסקרטית]
לכל קבוצה \(X\) ניתן להגדיר את הטופולוגיה הדיסקרטית על ידי \(\tau=\mathcal{P}(X)\).

\end{example}
\begin{definition}[טפולוגיה מושרת ממטריקה]
יהי \((X,d)\) מרחב מטרי. קבוצה \(A\) היא פתוחה אם \(A\) פתוחה במרחב מטרי, כלומר כל נקודה מוכלת בכדור פתוח אשר מוכל ב-\(A\).

\end{definition}
\begin{example}[הטופולוגיה הסטנדרטית על \(\mathbb{R}\)]
קבוצה \(U\subseteq \mathbb{R}\) תהיה פתוחה אם לכל \(x \in U\) קיים \(\epsilon>0\) כך שהכדור הפתוח מקיים:
$$B_{\varepsilon}(x)=(x-\varepsilon,x+\varepsilon)\subseteq U$$

\end{example}
\begin{example}[טופולוגיית המשלים הסופית]
לכל קבוצה \(X\) ניתן להגדיר את הטופולוגיה:
$$\tau=\{A\subseteq X\mid X\setminus A\,\mathrm{is\,finite}\}\cup\{\varnothing \}$$
נשים לב כי כל קבוצה סופית היא סגורה.

\end{example}
\begin{example}[טופולוגית המשלים הבן מנייה]
לכל קבוצה \(X\) ניתן להגדיר את הטופולוגיה:
$$\tau=\{A\subseteq X\mid X\setminus A\,\mathrm{is\;countable}\}\cup\{\varnothing \}$$
נשים לב כי כל קבוצה בת מנייה היא סגורה.

\end{example}
\begin{proposition}
הטופולוגיה המשלים הסופי עדינה מטופולוגיה המשלים הבן מנייה.

\end{proposition}
\begin{example}[הטופולוגיה החלשה]
יהיו \(\left\{  \tau_{i}  \right\}_{i\in I}\) טופולוגיות על \(X\), אזי גם \(\cap_{i \in I}\tau_{i}\) טופולוגיה על \(X\). בפרט אם \(P\) אוסף של תתי קבוצות של \(X\) קיימת טופולוגיה מינימלית(גסה ביותר) \(\tau\) כך ש-\(P\subseteq \tau\).

\end{example}
\begin{definition}[קבוצה סגורה]
קבוצה \(F\subseteq X\) היא סגורה אם \(X\setminus F\) היא פתוחה.

\end{definition}
\begin{proposition}
יהי \((X,\tau)\) מרחב טופולוגי. מתקיים:

  \begin{enumerate}
    \item הקבוצות \(\varnothing,X\) הם סגורות. 


    \item איחודים כלשהם של קבוצות סגורות הם סגורות. 


    \item איחודים סופיים של קבוצות סגורות הם סגורות. 


  \end{enumerate}
\end{proposition}
\begin{proposition}
יהי \(Y\) תת מרחב טופולוגי של \(X\). אזי קבוצה A סגורה ב-\(Y\) אם"ם היא חיתוך של קבוצה סגורה כלשהי ב-\(X\) עם \(Y\).

\end{proposition}
\section{בסיס של טופולוגיה}

\begin{definition}[בסיס של טופולוגיה]
בסיס \(\mathcal{B}\) של טופולוגיה על \(X\) זו אוסף של תתי קבוצות על \(X\) אשר מקיימות:

  \begin{enumerate}
    \item כיסוי של \(X\). כלומר מתקיים: 
$$\bigcup_{B\in{\mathcal{B}}}B=X$$


    \item אם \(x \in B_{1}\cap B_{2}\) עבור \(B_{1},B_{2}\in \mathcal{B}\) אזי קיים \(B_{3} \in \mathcal{B}\) כך שמתקיים: 
$$x\in B_{3}\subseteq B_{1}\cap B_{2}$$


  \end{enumerate}
\end{definition}
\begin{definition}[טופולוגיה הנוצרת על ידי בסיס]
הטופולוגיה הנוצרת על ידי בסיס \(\mathcal{B}\) תהיה הטופולוגיה שמתקבלת על ידי איחוד של כל איבר בבסיס:
$$\tau=\left\{\bigcup S\mid S\subseteq\mathcal{B}\right\}$$

\end{definition}
\begin{proposition}
אם \(\left( X,\tau \right)\) טופולוגיה הנוצרת ממטריקה אז אוסף הכדורים הפתוחים.

\end{proposition}
\begin{proof}
נובע מהטענה המוכרת של מרחבים מטרים שכל קבוצה פתוחה היא איחוד של כדורים פתוחים.

\end{proof}
\begin{example}[הטופולוגיה הדיסקרטית]
הטופולוגיה שבה כל יחידון הוא איבר בבסיס היא יוצרת את הטופולוגיה הדיסקרטית, כיוון שניתן להרכיב כל קבוצה על ידי איחוד של יחידונים.

\end{example}
\begin{example}[הבסיס האוקלידי]
עבור \(\mathbb{R}^{n}\) האוסף של כל הכדורים הפתוחים מהווה בסיס לטופולוגיה הסטנדרטית.

\end{example}
\begin{example}[בסיס בן מנייה עבור \(\mathbb{R}^{n}\)]
עבור \(\mathbb{R}^{n}\) אוסף הכדורים הפתוחים אשר מרכזם בנקודות רציונאליות ורדיוסים רציונאלי.

\end{example}
\begin{proposition}
יהי \(\left( X,\tau \right)\) מרחב טופולוגי. אם \(\mathcal{B}\) בסיס של טופולוגיה \(\tau\) אזי הטופולוגיה הנוצרת מ-\(\beta\) היא \(\tau\).

\end{proposition}
\begin{definition}[תת בסיס]
תת בסיס \(\mathcal{S}\) על טופולוגיה \(X\) הוא אוסף של תתי קבוצות כך שהחיתוך הסופי של האיברים יוצר בסיס:
$${\mathcal{B}}=\{B_{1}\cap B_{2}\cap\cdots\cap B_{n}\mid n\in\mathbb{N},B_{i}\in{\mathcal{S}}\}.$$
כך שהטופולוגיה הנוצרת על ידי \(\mathcal{S}\) תהיה:
$$\tau=\left\{\bigcup S\mid S\subseteq\mathcal{B}\right\}$$

\end{definition}
\begin{definition}[טופולוגייה סדר]
יהי \(P\) קבוצה סדורה קווית(כלומר ניתן להשוואות כל שתי איברים). אזי טופולוגיית הסדר על \(P\) תהיה הטופולוגיה שנוצרת על ידי הבסיס:
- הקטעים הפתוחים:
$$(a,b)=\{x\in P\mid a<x<b\}.$$
- אם ל- \(P\) יש מינימום \(m\) או מקסימום \(M\) אזי גם:
$$[m,b)=\{x\in P\mid m\leq x<b\},\quad(a,M]=\{x\in P\mid a<x\leq M\}$$

\end{definition}
\section{טפולוגיית מכפלה}

\begin{definition}[טופולוגיית מכפלה]
יהיו \(X,Y\) מרחבים טופולוגיים. אזי הטופולוגיית המכפלה על \(X,Y\) תהיה הטופולוגיה הנוצרת מהבסיס אשר מורכב מכל הקבוצות מהצורה \(U\times V\) כאשר \(U\) קבוצה פתוחה ב-\(X\) ו-\(V\) קבוצה פתוחה ב-\(Y\).

\end{definition}
\begin{lemma}
טופולוגיית המכפלה אכן טופולוגייה.

\end{lemma}
\begin{proof}
מספיק להראות כי הבסיס \(\mathcal{B}\) אשר נוצר מהמהכפלה של הקבוצות הפתוחות הוא אכן בסיס. התנאי הראשון של בסיס טריוויאלי כי \(X\times Y\) בפרט איבר בבסיס. התנאי השני מתקיים כיוון שעבור שתי איברי בסיס \(U_{1}\times V_{1}\) ו-\(U_{2}\times V_{2}\) נקבל:
$$(U_{1}\times V_{1})\cap(U_{2}\times V_{2})=(U_{1}\cap U_{2})\times(V_{1}\cap V_{2}),$$
אשר גם איבר בבסיס כי \(U_{1}\cap U_{2}\) ו-\(V_{1}\cap V_{2}\) הם קבוצות פתוחות.

\end{proof}
\begin{proposition}
אם \(\mathcal{B}\) בסיס של הטופולוגיה \(X\) ו-\(\mathcal{C}\) בסיס של הטופולוגיה \(Y\) אזי:
$${\mathcal{D}}=\{B\times C\mid B\in{\mathcal{B}}\;a n d\,C\in{\mathcal{C}}\}$$
הוא בסיס של הטופולוגיה \(X\times Y\).

\end{proposition}
\begin{definition}[הטלות על מרחבים]
יהי \(X,Y\) טופולוגיית מכפלה. העתקות \(\pi_{1}:X\times Y\to X\) ו-\(\pi_{2}:X\times Y\to Y\) אשר מקיימת:
$$\pi_{1}(x,y)=x\qquad \pi_{2}(x,y)=y$$
נקראות ההטלות על המרחבים \(X\) ו-\(Y\) בהתאמה.

\end{definition}
\begin{proposition}[תכונות של הטלות]
  \begin{itemize}
    \item הטלות יהיו על(כל עוד אחת הקבוצות לא ריקות).
    \item אם \(U\) קבוצה פתוחה ב-\(X\) אז \(\pi_{1}^{-1}(U)\) תהיה הקבוצה \(U\times Y\) אשר פתוחה ב-\(X\times Y\). באופן דומה עבור \(V\) פתוחה ב-\(Y\) נקבל \(\pi_{2}^{-1}(V)\) תהיה פתוחה בטופולוגיית המכפלה.
    \item ההטלות יהיו פונקציות רציפות.
  \end{itemize}
\end{proposition}
\begin{proposition}
האוסף:
$${\mathcal{S}}=\{{\pi}_{1}^{-1}(U)\mid U\ \text{open in } X\}\cup\{{\pi}_{2}^{-1}(V)\mid V\ \text{open in } Y\}$$
הוא תת בסיס של טופולוגית המכפלה \(X\times Y\)

\end{proposition}
\section{טופולוגיית התת מרחב}

\begin{definition}[טופולוגיית תת מרחב]
יהי \((X,\tau)\) מרחב טופולוגי ו-\(Y\subseteq X\). אזי האוסף:
$$\tau_{Y}=\{ Y\cap  U\mid U \in \tau \}$$
תקרא טופולוגיית התת מרחב.

\end{definition}
\begin{lemma}
טופולוגיית התת מרחב היא אכן טופולוגיה.

\end{lemma}
\begin{proof}
נדרש להראות את שלושת התכונות:
- נשים לב כי \(\varnothing=Y\cap \varnothing\) ו-\(Y=Y\cap X\).
- סגירות לחיתוכים סופיים נובעת מכך שמתקיים:
$$(U_{1}\cap Y)\cap\cdot\cdot\cdot\cap(U_{n}\cap Y)=(U_{1}\cap\cdot\cdot\cdot\cap U_{n})\cap Y$$
- סגירות לאיחודים כלשהם נובעת מכך שמתקיים:
$$\bigcup_{\alpha\in J}(U_{\alpha}\cap Y)=(\bigcup_{\alpha\in J}U_{\alpha})\cap Y$$

\end{proof}
\begin{proposition}
אם \(\mathcal{B}\) הוא בסיס של הטופולוגיה \(X\) אזי האוסף:
$$\mathcal{B} _{Y}=\{ B\cap  Y \mid B \in \mathcal{B}  \}$$
הוא בסיס של הטופולוגיית התת מרחב.

\end{proposition}
\begin{proposition}
יהי \(Y\) תת מרחב של \(X\). אם \(U\) פתוח ב-\(Y\) ו-\(Y\) פתוח ב-\(X\) אזי \(U\) פתוח ב-\(X\).

\end{proposition}
\begin{proof}
כיוון ש-\(U\) פתוח ב-\(Y\) מתקיים \(Y=U\cap V\) עבור איזשהו \(V\) פתוח ב-\(X\). כיוון ש-\(Y\) ו-\(V\) שניהם פתוחים ב-\(X\) גם \(Y \cap V\).

\end{proof}
\section{טופולוגיית מכפלה אינסופית}

עבור מרחבים טופולוגיים \(X_{1},X_{2},\dots\) נרצה להגדיר טופולוגיה עבור \(X_{1}\times X_{2}\times\dots\)

\begin{definition}[טופולוגיית הקופסא]
יהיו \(X_{1},X_{2},X_{3},\dots\) מרחבים טופולוגיים. הטופולוגיה על \(X_{1}\times X_{2}\times X_{3}\times\dots\) הנוצרת על ידי הבסיס:
$$U_{1}\times U_{2}\times U_{3}\times\dots$$
כאשר \(U_{i}\) היא קבוצה פתוחה ב-\(X_{i}\) נקראת טופולוגיית הקופסא.

\end{definition}
\begin{definition}[טופולוגיית המכפלה]
יהיו \(X_{1},X_{2},X_{3},\dots\) מרחבים טופולוגיים. עבור כל קבוצה פתוחה \(U_{i}\) של \(X_{i}\) הטופולוגייה הנוצרת על ידי התת בסיס \(\pi_{i}^{-1}(U_{i})\) נקראת טופולוגיית המכפלה.

\end{definition}
\begin{remark}
במקרה של מכפלה סופית של מרחבים טופולוגיים הטופולוגיית מכפלה וטופולוגיית הקופסא יהיו זהות.

\end{remark}
\begin{summary}
  \begin{itemize}
    \item \textbf{טופולוגיית מכפלה (סופית):} מוגדרת על \(X \times Y\) באמצעות בסיס של קבוצות מהצורה \(U \times V\) (\(U\) פתוחה ב-\(X\), \(V\) פתוחה ב-\(Y\)).
    \item אם \(\mathcal{B}\) בסיס ל-\(X\) ו-\(\mathcal{C}\) בסיס ל-\(Y\), אז \(\{B \times C \mid B \in \mathcal{B}, C \in \mathcal{C}\}\) הוא בסיס ל-\(X \times Y\).
    \item \textbf{הטלות (\(\pi_1, \pi_2\)):} העתקות מ-\(X \times Y\) לרכיבים \(X\) ו-\(Y\).
    \item הטלות הן רציפות, ומקורות של קבוצות פתוחות תחת הטלות הן קבוצות פתוחות במכפלה.
    \item אוסף המקורות של קבוצות פתוחות תחת ההטלות מהווה תת בסיס לטופולוגיית המכפלה.
    \item \textbf{טופולוגיית תת מרחב:} על תת קבוצה \(Y \subseteq X\), קבוצה פתוחה היא מהצורה \(Y \cap U\) כאשר \(U\) פתוחה ב-\(X\).
    \item אם \(\mathcal{B}\) בסיס לטופולוגיה על \(X\), אז \(\{B \cap Y \mid B \in \mathcal{B}\}\) הוא בסיס לטופולוגיית התת מרחב על \(Y\).
    \item אם קבוצה פתוחה בתת מרחב \(Y\), ו-\(Y\) עצמו פתוח במרחב המקורי \(X\), אז הקבוצה פתוחה גם ב-\(X\).
    \item \textbf{טופולוגיית קופסא (מכפלה אינסופית):} נוצרת ע"י בסיס של מכפלות אינסופיות \(U_1 \times U_2 \times \dots\) שכל \(U_i\) פתוחה ברכיב המתאים.
    \item \textbf{טופולוגיית מכפלה (מכפלה אינסופית):} נוצרת ע"י תת בסיס של מקורות קבוצות פתוחות תחת כל אחת מההטלות לרכיבים \(\pi_i^{-1}(U_i)\).
    \item עבור מכפלה סופית, טופולוגיית הקופסא וטופולוגיית המכפלה זהות.
  \end{itemize}
\end{summary}
\section{מרחבי מנה}

\begin{reminder}[יחס שקילות]
תהי X קבוצה. אומרים שקבוצה \(R\subset X\times X\) היא יחס שקילות על X, אם מתקיימים שלושת התנאים הבאים:

  \begin{enumerate}
    \item לכל \(x\in X\) מתקיים \((x,x)\in R\) . 


    \item לכל \(x,y\in X\) מתקיים כי אם \((x,y)\in R\) אז גם \((y,x)\in R\) . 


    \item לכל \(x,y,z\in X\) מתקיים כי אם \((x,y)\in R\) וגם \((y,z)\in R\) אז גם \((x,z)\in R\) . 
באופן כללי, אם \((x,y)\in R\) מסמנים \(x\sim y\) .
בהינתן קבוצה X עם יחס שקילות R עליה, מחלקת שקילות של איבר \(x\in X\) מוגדרת להיות \([x]=:\{y\in X|x\sim y\}\) .
לכל \(x\in X\) מתקיים \(x\in[x]\) ולכן \([x]\ne\emptyset\) .
בהינתן קבוצה X עם יחס שקילות R עליה, מסמנים את אוסף כל מחלקות השקילות של X ב- \(X/R\) . קל לראות שהאוסף \(X/R\) מגדיר חלוקה של X (כלומר אוסף תתי קבוצות זרות בזוגות שאיחודן מכסה את X) .


  \end{enumerate}
\end{reminder}
\begin{definition}[טופולוגיית המנה]
יהי X מרחב טופולוגי ויהי R יחס שקילות עליו.
נגדיר העתקה \(P:X\rightarrow X/R\) על ידי \(x\mapsto[x]\) .
נרצה להפוך את \(X/R\) למרחב טופולוגי. נעשה זאת על ידי בחירת הטופולוגיה החזקה ביותר שתחתיה \(P\) רציפה.
לשם כך נגדיר קבוצה \(U\subset X/R\) להיות פתוחה אם ורק אם \(P^{-1}(U)\subset X\) פתוחה.

\end{definition}
\begin{remark}
ההעתקה P אינה בהכרח העתקה פתוחה.

\end{remark}
\begin{example}
ניקח למשל את המרחב \([0,1]\) ונגדיר יחס שקילות \(R=\{(x,x) \mid x\in[0,1]\}\cup\{(0,1),(1,0)\}\) (כלומר מזהים את הנקודות 0,1 כנקודה אחת במרחב הטופולוגי של מחלקות השקילות).
הקבוצה \((\frac{1}{2},1]\) פתוחה ב- \([0,1]\), אבל \(P^{-1}(P((\frac{1}{2},1]))=\{0\}\cup(\frac{1}{2},1]\), וזו קבוצה שאינה פתוחה ב-\([0,1]\) . לכן \(P((\frac{1}{2},1])\) אינה פתוחה ב-\([0,1]/R\) .

\end{example}
\begin{lemma}[תכונות הנשמרות במרחב המנה]
יהי X מרחב טופולוגי ויהי R יחס שקילות עליו. אזי:

  \begin{enumerate}
    \item אם X קומפקטי אז גם \(X/R\) קומפקטי. 


    \item אם X קשיר או קשיר מסילתית, אז גם \(X/R\) קשיר או קשיר מסילתית, בהתאמה. 


  \end{enumerate}
\end{lemma}
\begin{proof}
תכונות אלו הן תכונות טופולוגיות, כלומר נשמרות תחת העתקה רציפה, ומתקיים \(P(X)=X/R\). תחת הטופולוגיה שהגדרנו על \(X/R\) זו העתקה רציפה, ולכן התמונה משמרת את התכונות הללו.

\end{proof}
\begin{lemma}[התכונה האוניברסלית של מרחב המנה]
יהיו X מרחב טופולוגי, יהי R יחס שקילות עליו ותהי \(P:X\rightarrow X/R=:\overline{X}\) ההעתקה שהגדרנו .
יהי Y מרחב טופולוגי נוסף ותהי \(f:X\rightarrow Y\) העתקה רציפה, שהיא קבועה על מחלקות השקילות של X . אזי:

  \begin{enumerate}
    \item קיימת העתקה רציפה \(\overline{f}:\overline{X}\rightarrow Y\) כך ש-\(\overline{f}\circ P=f\) . 


    \item אם \(\overline{f}\) הנ"ל חח"ע ועל וגם f העתקה פתוחה (או סגורה), אזי \(\overline{f}\) היא הומאומורפיזם . 


    \item אם \(\overline{f}\) הנ"ל חח"ע ועל וגם X קומפקטי ו-Y האוסדורף, אזי \(\overline{f}\) היא הומאומורפיזם . 


  \end{enumerate}
\end{lemma}
\begin{proof}
  \begin{enumerate}
    \item נגדיר את הפונקציה \(\overline{f}:\overline{X}\rightarrow Y\) על־ידי \([x]\mapsto f(x)\) . מההנחה כי f קבועה על מחלקות השקילות נובע כי \(\overline{f}\) מוגדרת היטב . נוודא שזו העתקה רציפה . תהי \(A\) פתוחה ב-Y. צריך להראות כי \(\overline{f}^{-1}(A)\) פתוחה ב-\(\overline{X}\), כלומר, לפי הגדרת הטופולוגיה על \(\overline{X}\) צריך להראות כי \(P^{-1}(\overline{f}^{-1}(A))\) פתוחה ב-X. 
נשים לב שמתקיים: \(\overline{f}^{-1}(A)=\{[x]|f(x)\in A\}=P(f^{-1}(A))\) .
מההנחה ש-f קבועה על מחלקות השקילות נובע כי: \(f^{-1}(A)=P^{-1}(P(f^{-1}(A)))=P^{-1}(\overline{f}^{-1}(A))\) .
ומההנחה ש-f רציפה נובע כי קבוצה זו פתוחה ב-X .


    \item צריך להראות כי \(\overline{f}^{-1}\) רציפה, לשם כך נראה כי \(\overline{f}\) העתקה פתוחה . תהי \(A\) פתוחה ב-\(\overline{X}\), נראה כי \(\overline{f}(A)\) פתוחה ב-Y . מההנחה ש-f קבועה על מחלקות השקילות נובע כי \(\overline{f}(A)=f(P^{-1}(A))\) . אבל מהיות A פתוחה נובע כי \(P^{-1}(A)\) פתוחה, ומההנחה כי f העתקה פתוחה נובע ש- \(f(P^{-1}(A))\) פתוחה . 


    \item הראינו לעיל שכל העתקה רציפה \(f:X\rightarrow Y\) שהיא חח"ע ועל, ל-X קומפקטי ו-Y האוסדורף, היא הומאומורפיזם . 


  \end{enumerate}
\end{proof}
\textbf{דוגמה כללית} הדבקה
יהי X מרחב טופולוגי ויהיו \(A,B\subset X\) עם העתקה \(f:A\rightarrow B\) כלשהי. נגדיר יחס שקילות על X, על־ידי \(a\sim a'\) אם ורק אם \(f(a)=f(a')\) . אם f היא העתקה על, אז המרחב \(X/\sim f\) הוא "הדבקה" של A ל-B במרחב X .

\textbf{דוגמאות}

\begin{enumerate}
  \item \(X=[a,b]\cup[c,d]\) קטעים ממשיים, \(f:\{b\}\rightarrow\{c\}\). מתקיים \(X/\sim f\cong [a,d-c+b]\) (כלומר זה הומאומורפי לקטע שאורכו כסכום אורכי שני הקטעים) . 


  \item \(X=[0,1]\), \(f:\{0,1\}\rightarrow\{0\}\). מתקיים \(X/\sim f\cong S^{1}\) . 


  \item \(X=[0,1]\times[0,1]\), \(f:\{0\}\times[0,1]\rightarrow\{1\}\times[0,1]\) על ידי \((0,t)\mapsto(1,t)\). מתקיים \(X/\sim f\) הומאומורפי לגליל בגובה 1 ובהיקף 1 . 


  \item נסמן את דיסק היחידה \(\mathbb{D}=\{x\in\mathbb{R}^{2} \mid ||x||\le1\}\subset\mathbb{R}^{2}\). נגדיר עליו יחס שקילות \(x\sim y\) אם ורק אם \(x,y\in S^{1}\) (כלומר כל נקודות השפה \(S^{1}\) שקולות)[. 


\end{enumerate}
\begin{proposition}
\(\mathbb{D}/\sim\cong S^{2}\) .

\end{proposition}
\begin{proof}
זה הגליל בגובה 2, \(X=S^{1}\times[-1,1]\) . נגדיר העתקה \(g:X\rightarrow S^{2}\) על־ידי \((x_{1},x_{2},t)\mapsto(\sqrt{1-t^{2}}x_{1},\sqrt{1-t^{2}}x_{2},t)\) . העתקה זו מכווצת כל מעגל בגובה t על הגליל, למעגל בגובה t על הספירה .
g מגדירה שתי מחלקות שקילות לא טריוויאליות על הגליל: \(\{(x_{1},x_{2},1)|x_{1},x_{2}\in S^{1}\}\) ו- \(\{(x_{1},x_{2},-1)|x_{1},x_{2}\in S^{1}\}\) . כל שאר האיברים בגליל מועתקים לאיבר יחיד ב-\(S^{2}\) .
g רציפה וקבועה על מחלקות השקילות, ולכן מהלמה נובע שקיימת \(\overline{g}: X/\sim g\rightarrow S^{2}\) רציפה, חח"ע ועל . X קומפקטי ו-\(S^2\) האוסדורף, ולכן מהלמה נובע כי היא הומאומורפיזם .
נגדיר העתקה \(f:X\rightarrow\mathbb{D}\) על־ידי \((x_{1},x_{2},t)\mapsto(\frac{t+1}{2}x_{1},\frac{t+1}{2}x_{2})\) . העתקה זו מעבירה את הגליל למעגלים על D .
f חח"ע בכל מקום, למעט בנקודה \(t=-1\). מחלקת השקילות הלא טריוויאלית היחידה היא \(\{(x_{1},x_{2},-1)|x_{1},x_{2}\in S^{1}\}\) .
לכן כמקודם קיימת \(\overline{f}:X/\sim f\rightarrow\mathbb{D}\) רציפה, חח"ע ועל, ומהיות X קומפקטי ו-D האוסדורף, נובע כי \(\overline{f}\) הומאומורפיזם.
נסיק מכאן: \(S^{2}\cong X/\sim g\cong(X/\sim f)/(\sim g/\sim f)\cong\mathbb{D}/\sim\) .
כאשר המעבר השני נובע מכך שהיחס \(\sim f\) מוכל ביחס \(\sim g\) והסימון \(\mathbb{D}/\sim\) הוא ליחס השקילות שהגדרנו בראשית הדוגמה .

\end{proof}
\begin{remark}
בדרך דומה ניתן להראות שכל \(\mathbb{D}_{n}=\{x\in\mathbb{R}^{n} \mid ||x||\le1\}\subset\mathbb{R}^{n}\) ניתן להדביק באופן שיהיה הומאומורפי ל-\(S^n\) .

\end{remark}
\begin{definition}[יריעה n-ממדית]
יהי X מרחב טופולוגי. אומרים כי X הוא יריעה n-ממדית, אם לכל \(x\in X\) קיימת סביבה הומאומורפית ל-\(\mathbb{R}^n\).

\end{definition}
\textbf{דוגמאות}

\begin{enumerate}
  \item הספירה \(S^{n}\) היא יריעה n-ממדית. (נוכיח בהמשך). 


  \item תהי G חבורה ותהי \(H\le G\) תת חבורה. מגדירים את \(G/H\) כאוסף מחלקות השקילות המתקבלות מהיחס \(x\sim y \iff xH=yH \iff x^{-1}y\in H\) . 


\end{enumerate}
\begin{proposition}
\(\mathbb{R}/\mathbb{Z}\cong S^{1}\) (מחלקות השקילות בחבורת המנה הן \(\{r+\mathbb{Z} \mid r\in\mathbb{R}\}\) .

\end{proposition}
\begin{proof}
נגדיר העתקה \(f:\mathbb{R}\rightarrow S^{1}\) על ידי \(t\mapsto(\cos(2\pi t),\sin(2\pi t))\) .
מתקיים כי f רציפה וקבועה על מחלקות השקילות, ולכן קיימת העתקה רציפה \(\overline{f}:\mathbb{R}/\mathbb{Z}\rightarrow S^{1}\) שהיא חח"ע ועל. ההעתקה \(f\) פתוחה ולכן \(\overline{f}\) הומאומורפיזם .

\end{proof}
\begin{remark}
באופן דומה, הטורוס (בייגל) הומאומורפי ל- \(\mathbb{R}^{2}/\mathbb{Z}^{2}\) .

\end{remark}
\chapter{פונקציות בין מרחבים טופולוגים}

\section{פונקציות רציפות}

\begin{definition}[פונקציה רציפה]
יהי \(X,Y\) מרחבים טופולוגיים. \(f:X\to Y\) פונקציה רציפה אם לכל \(U \subseteq X\) פתוחה ב-\(Y\) נקבל \(f^{-1}(U)\subseteq X\) פתוחה ב-\(X\).

\end{definition}
\begin{example}
כל פונקציה מהצורה \(f:X\to Y\) כאשר \(X\) היא טופולוגיה דיסקרטית תהיה רציפה. זה כיוון שכל תת קבוצה של המרחב הדיסקרטי היא פתוחה ולכן \(f^{-1}(U)\) תהיה פתוחה לכל \(U\).

\end{example}
\begin{proposition}
התכונות הבאות של ההעתקה \(f:X\to Y\) הן שקולות:

  \begin{enumerate}
    \item הפונקציה \(f\) רציפה. 


    \item לכל \(F\subseteq Y\) סגורה \(f^{-1}(F)\subseteq X\) סגורה. 


    \item אם \(B\) בסיס לטופולוגיה של \(Y\) אזי לכל \(B \in \mathcal{B}\) נקבל \(f^{-1}(B)\) פתוחה. 


    \item לכל \(x \in X\) ולכל סביבה \(f(x) \in W\subseteq Y\) מתקיים ש-\(f^{-1}(W)\) סביבה של \(x\). 


    \item קיים כיסוי פתוח \(\{ U_{\alpha} \}_{\alpha \in \Omega}\) כך ש- \(\bigcup_{\alpha \in \Omega}U_{\alpha}=X\) וגם \(U_{\alpha}\) פתוחה לכל \(\alpha \in \Omega\) כך ש-\(f|_{U_{\alpha}}\) רציפה לכל \(\alpha \in \Omega\). 


    \item קיים כיסוי סופי \(X=\bigcup_{i=1}^{n}F_{i}\) על ידי קבוצות סגורות כך ש-\(f|_{F_{i}}:F_{i}\to Y\) פונקציה רציפה. 


    \item לכל \(A\subseteq X\) מתקיים \(f(\overline{A})\subseteq \overline{f(A)}\). 


  \end{enumerate}
\end{proposition}
\begin{proof}
השלב \(1\iff 2\) ברור - פשוט עוברים למשלימים.
נראה \(1\iff 3\). הכיוון \(\impliedby\) ברור כי דורשים פחות. הכיוון \(\implies\) את \(U\) ב-\(Y\) פתוחה אזי \(U\) אוסף קבוצות בסיס \(B_{\alpha}\) כאשר \(\alpha \in I\) כך ש-\(U=\bigcup B_{\alpha}\)$$f^{-1} (U)=f^{-1} \left( \bigcup B_{\alpha} \right)=\bigcup f^{-1} (B_{\alpha})$$
כעת נקרא \(1\iff 4\). עבור \(\impliedby\) נקבל סביבה 
$$x \in X \quad  f(x) \in W$$
ולכן קיימת \(f(x) \in U \subseteq W\) כך ש-\(U\) פתוחה וכן:
$$x \in f^{-1} (U)\subseteq f^{-1} (W)$$
כאשר עבור \(\implies\) ניקח \(U \subseteq Y\) פתוחה וצריך להראות ש-\(f^{-1}(U)\) פתוחה. תהי \(x \in f^{-1}(U)\). אזי \(U\) סביבה ל-\(f(x)\) לכן לפי 4 \(f^{-1}(U)\) היא סביבה של \(x\) כלומר קיימת \(x \in V_{X}\subseteq f^{-1}(U)\) פתוחה גורר \(f^{-1}(U)\) פתוחה.
נראה \(5 \iff 1\). עבור \(5\implies 1\) נקבל \(\{ U_{\alpha} \}\) כיסוי פתוח כך שלכל \(\alpha\) נקבל \(f|_{U_{\alpha}}\to Y\) רציפה וש-\(V\subseteq Y\) פתוחה.
$$f^{-1} (V)=\bigcup_{\alpha} f^{-1} |_{U_{\alpha}}(v)=\bigcup_{\alpha}(U_{\alpha}\cap  W_{\alpha})$$
כאשר \(W_{\alpha}\) פתוחה.
לסיים את ההוכחה

\end{proof}
\begin{proposition}[הרכבה של פונקציות רציפות]
יהיו \(X,Y\) ו-\(Z\) מרחבים טופולוגיים. אזי:

  \begin{enumerate}
    \item פונקציה קבועה \(f:X\to Y\) אשר ממפה כל \(X\) לנקודה \(y_{0}\) של \(Y\) תהיה רציפה. 


    \item אם \(A\) תת מרחב של \(X\) אז הפונקציה \(j:A\to X\) היא רציפה. 


    \item הרכבה - אם \(f:X\to Y\) ו-\(g:Y\to Z\) רציפות אזי \(g\circ f:X\to Z\) רציפה. 


    \item אם \(f:X\to Y\) רציף ו-\(A\) תת מרחב של \(X\) אז גם הצמצום \(f|_{A}:A\to Y\) יהיה רציף. 


    \item יהי \(f:X\to Y\) רציפה. אם \(Z\) הוא תת מרחב של \(Y\) המכיל את \(f(X)\) אזי הפונקציה \(g:X\to Z\) אשר מצמצמת את הטווח של \(f\) תהיה רציפה. באופן דומה אם \(Y\) תת מרחב של \(Z\) אז פונקציה \(h:X\to Z\) אשר מרחיבה את הטווח של \(f\) תהיה רציפה. 


    \item העתקה \(f:X\to Y\) תהיה רציפה אם \(X\) הוא איחוד של קבוצות פתוחות \(U_{\alpha}\) כך ש-\(f|_{U_{\alpha}}\) רציפה לכל \(\alpha\). 


  \end{enumerate}
\end{proposition}
\begin{proposition}[למת ההדבקה]
יהי \(X=A\cup B\) כאשר \(A,B\) סגורות ב-\(X\). נגדיר \(f:A\to Y\) ו-\(g:B\to Y\) פונקציות רציפות. אם \(f(x)=g(x)\) לכל \(x \in A \cap B\) אזי ניתן לאחד את \(f,g\) כדי ליצור פונקציה רציפה \(f:X\to Y\) המוגדרת על ידי:
$$h(x)=\begin{cases}f(x) & x \in A \\g(x) & x \in B
\end{cases}$$

\end{proposition}
\begin{proof}
יהי \(C\) תת קבוצה סגורה של \(Y\). כעת מתכונות של תורת קבוצות:
$$h^{-1}(C)=f^{-1}(C)\cup g^{-1}(C)$$
כיוון ש-\(f\) היא רציפה נקבל \(f^{-1}(C)\) סגורה ב-A ולכן סגורה ב-\(X\). באופן דומה \(g^{-1}(C)\) סגורה ב-\(B\) ולכן סגורה ב-\(X\). האיחוד שלהם \(h^{-1}(C)\) יהיה כעת סגור ב-\(C\).

\end{proof}
\begin{proposition}
יהי \(f:A\to X\times Y\) נתונה על ידי:
$$f(a)=(f_{1}(a),\,f_{2}(a))$$
אזי \(f\) רציף אם"ם ההעתקות \(f_{1}:A\to X\) ו-\(f_{2}:A\to Y\) רציפות.

\end{proposition}
\begin{proof}
נניח ראשית כי \(f\) רציפה. נסתכל על ההטלות \(\pi_{1}:X\times Y\to X\) ו-\(\pi_{2}:X\times Y\to Y\) כאשר אנו יודעים כי העתקות אלו רציפות. נשים לב כי לכל \(a \in A\) מתקיים:
$$f_{1}(a)=\pi_{1}(f(a))\ \ \ \ \ \mathrm{and}\ \ \ \ \ f_{2}(a)=\pi_{2}(f(a))$$
כאשר כיוון שההטלות ו-\(f\) רציפות נקבל כי ההרכבה שלהם רציף ולכן \(f_{1},f_{2}\) רציפות.
עבור הכיוון השני נניח כי \(f_{1},f_{2}\) רציפות. נראה כי עבור כל איבר בסיס \(U\times V\) של הטופולוגיית מכפלה \(X \times Y\) מקיים \(f^{-1}(U\times V)\) היא קבוצה פתוחה. נקובה a היא ב-\(f^{-1}(U\times V)\) אם"ם \(f(a)\in U \times V\). כלומר אם"ם \(f_{1}(a) \in U\) ו-\(f_{2}(a)\in V\). לכן:
$$f^{-1}(U\times V)=f_{1}^{-1}(U)\cap f_{2}^{-1}(V).$$
כאשר כיוון שהקבוצות \(f_{1}^{-1}(U)\) ו-\(f_{2}^{-1}(U)\) פתוחות, גם החיתוך שלהם יהיה.

\end{proof}
\begin{proposition}[הרכבת פונקציות רציפות]
הרכבת פונקציות רציפות היא רציפה. כלומר, אם \(f: X \to Y\) ו-\(g: Y \to Z\) רציפות, אז \(g \circ f: X \to Z\) רציפה.

\end{proposition}
\begin{proof}
תהי \(W \subset Z\) קבוצה פתוחה. היות ש-\(g\) רציפה, הקבוצה \(g^{-1}(W) \subset Y\) פתוחה ב-\(Y\). היות ש-\(f\) רציפה, הקבוצה \(f^{-1}(g^{-1}(W)) \subset X\) פתוחה ב-\(X\). אבל \(f^{-1}(g^{-1}(W)) = (g \circ f)^{-1}(W)\). לכן, לכל קבוצה פתוחה \(W \subset Z\), הקבוצה \((g \circ f)^{-1}(W)\) פתוחה ב-\(X\). לפיכך, \(g \circ f\) רציפה.

\end{proof}
\begin{proposition}
פונקציית הזהות רציפה.

\end{proposition}
\begin{proof}
מידית מהגדרה.

\end{proof}
\begin{proposition}
תהי \(A \subset X\). אזי פונקציית ההכלה \(i: A \to X\) המוגדרת על ידי \(i(a) = a\) היא רציפה ביחס לטופולוגיה המושרית על \(A\) מ-\(X\).

\end{proposition}
\begin{proof}
מידית מהגדרה.

\end{proof}
\begin{proposition}
תהי \(f: (X, \tau) \to (Y, \sigma)\) רציפה. יהיו \(\tau^* \supset \tau\) ו-\(\sigma^* \subset \sigma\) טופולוגיות חדשות על \(X\) ו-\(Y\) בהתאמה. אזי \(f: (X, \tau^*) \to (Y, \sigma^*)\) גם רציפה.

\end{proposition}
\begin{proof}
תהי \(V \in \sigma^*\). היות ש-\(\sigma^* \subset \sigma\), מתקיים ש-\(V \in \sigma\). היות ש-\(f: (X, \tau) \to (Y, \sigma)\) רציפה, מתקיים ש-\(f^{-1}(V) \in \tau\). היות ש-\(\tau \subset \tau^*\), מתקיים ש-\(f^{-1}(V) \in \tau^*\). לכן, לכל \(V \in \sigma^*\), \(f^{-1}(V) \in \tau^*\). לפיכך, \(f: (X, \tau^*) \to (Y, \sigma^*)\) רציפה.

\end{proof}
\begin{proposition}[קריטריון לרציפות למרחבים מטרים]
תהי \(f: X \to Y\) פונקציה בין מרחבים מטריים \((X, d_X)\) ו-\((Y, d_Y)\). \(f\) רציפה בנקודה \(x \in X\) אם ורק אם לכל \(\epsilon > 0\) קיים \(\delta > 0\) כך שאם \(d_X(x, x') < \delta\) אז \(d_Y(f(x), f(x')) < \epsilon\). כמו כן, \(f\) רציפה על \(X\) אם ורק אם היא רציפה בכל נקודה \(x \in X\).

\end{proposition}
\begin{proof}
נניח ש-\(f\) רציפה ויהיו \(\epsilon > 0\) ו-\(x_0 \in X\). היות ש-\(B_\epsilon(f(x_0))\) כדור פתוח ב-\(Y\), ו-\(f\) רציפה, הקבוצה \(f^{-1}(B_\epsilon(f(x_0)))\) פתוחה ב-\(X\). היות ש-\(x_0 \in f^{-1}(B_\epsilon(f(x_0)))\), קיימת קבוצה פתוחה \(U \subset f^{-1}(B_\epsilon(f(x_0)))\) כך ש-\(x_0 \in U\). היות ש-\(U\) פתוחה ב-\(X\) ו-\(X\) מרחב מטרי, קיים \(\delta > 0\) כך ש-\(B_\delta(x_0) \subset U\). אם \(x' \in B_\delta(x_0)\), אז \(x' \in U\), ולכן \(x' \in f^{-1}(B_\epsilon(f(x_0)))\), מה שאומר ש-\(f(x') \in B_\epsilon(f(x_0))\), כלומר \(d_Y(f(x_0), f(x')) < \epsilon\).
לכיוון השני, נניח ש-\(f\) מקיימת את תנאי \(\epsilon\)-\(\delta\) בכל נקודה \(x_0 \in X\). תהי \(V \subset Y\) קבוצה פתוחה. אנו רוצים להראות ש-\(f^{-1}(V)\) פתוחה ב-\(X\). יהי \(x_0 \in f^{-1}(V)\). אז \(f(x_0) \in V\). היות ש-\(V\) פתוחה ב-\(Y\), קיים \(\epsilon > 0\) כך ש-\(B_\epsilon(f(x_0)) \subset V\). לפי ההנחה, קיים \(\delta > 0\) כך שאם \(d_X(x_0, x') < \delta\) אז \(d_Y(f(x_0), f(x')) < \epsilon\). כלומר, אם \(x' \in B_\delta(x_0)\), אז \(f(x') \in B_\epsilon(f(x_0))\). היות ש-\(B_\epsilon(f(x_0)) \subset V\), מתקיים ש-\(f(x') \in V\), ולכן \(x' \in f^{-1}(V)\). לפיכך, \(B_\delta(x_0) \subset f^{-1}(V)\). הראינו שלכל נקודה \(x_0 \in f^{-1}(V)\) קיים כדור פתוח סביבה המוכל כולו ב-\(f^{-1}(V)\). לכן \(f^{-1}(V)\) פתוחה ב-\(X\).
החלק השני של הטענה, ש-\(f\) רציפה על \(X\) אם ורק אם היא רציפה בכל נקודה, נובע ישירות מהקריטריון הראשון של רציפות באמצעות קבוצות פתוחות. רציפות על \(X\) מוגדרת כהפיכת כל קבוצה פתוחה לקבוצה פתוחה. רציפות בנקודה באמצעות קבוצות פתוחות (שהיא שקולה להגדרת האפסילון-דלתא במרחבים מטריים) אומרת שלכל נקודה \(x\) וכל סביבה פתוחה \(V\) של \(f(x)\), קיימת סביבה פתוחה \(U\) של \(x\) כך ש-\(f(U) \subset V\), או באופן שקול, \(U \subset f^{-1}(V)\). הקריטריון הראשון (הטופולוגי) שקול לחלוטין לרציפות בכל נקודה כאשר משתמשים בהגדרת סביבות פתוחות.

\end{proof}
\begin{summary}
  \begin{itemize}
    \item העתקה \(f:X\to Y\) נקראת רציפה אם לכל \(U \subseteq Y\) פתוחה נקבל כי \(f^{-1}(U)\subseteq X\) פתוחה.
    \item התנאים הבאים שקולים לרציפות:


    \item עבור כל קבוצה סגורה \(F \subseteq Y\) נקבל \(f^{-1}(F)\subseteq X\) סגורה. 


    \item לכל איבר בסיס \(B\) של \(Y\) נקבל \(f^{-1}(B)\) פתוחה. 


    \item לכל \(x\) ולכל סביבה \(W\in f(x)\) נקבל \(f^{-1}(W)\) הוא סביבה של \(x\). 


    \item קיים כיסוי פתוח \(X=\bigcup U_{\alpha}\) כך שכל צמצום \(f|_{U_{\alpha}}\) הוא רציף 


    \item קיים כיסוי סופי סגור \(X=\bigcup_{i=1}^{n}F_{i}\) כאשר כל \(f|_{F_{i}}\) רציף. 


    \item לכל \(A\subseteq X\) מתקיים \(f\left( \overline{A} \right)=\overline{f(A)}\). 


    \item תכונות של פונקציות רציפות:
  \end{itemize}
\end{summary}
\section{הומאומורפיזם}

\begin{definition}[הומאומורפיזם]
יהיו \(X,Y\) מרחבים טופולוגיים. העתקה \(f:X\to Y\) תקרא הומיאומורפיזם אם:

  \begin{enumerate}
    \item העתקה \(f\) חח"ע ועל. 


    \item העתקה \(f:X\to Y\) רציפה 


    \item ההעתקה \(f^{-1}:Y\to X\) רציפה. 


  \end{enumerate}
\end{definition}
\begin{definition}[מרחבים הומאומורפיים]
מרחבים אשר קיים הומיאומורפיזם ביניהם.

\end{definition}
\begin{example}
הקטע \((0,1)\) ו-\(\mathbb{R}\). הומאומורפיזם אפשרי יהיה:
$$\varphi:x\mapsto \frac{e^{ x }}{e^{ x }+1}$$
כאשר זהו הומאומרפיזם כי:
$$\lim_{ x \to -\infty } \varphi(x)=0\qquad \lim_{ n \to \infty } \varphi(x)=1$$
וכן \(\varphi'(x)> 0\) לכל \(x \in \mathbb{R}\).

\end{example}
\begin{example}
הקבוצה:
$$\eta=\{ x+iy \mid  x,y \in \mathbb{R}\quad y> 0\}$$
תהיה הומאומורפית לקבוצה:
$$D=\{ z \in \mathbb{C}\mid  \lvert z \rvert < 1 \}$$
בעזרת ההעתקה:
$$z\mapsto \frac{z-i}{z+i}$$

\end{example}
\begin{example}
נראה דוגמא לא שתי מרחבים שלא הומאומורפיים. נסתכל למשל על:
$$S'=\{ z \in \mathbb{C}\mid \lvert z \rvert =1 \}\qquad J=[0,2\pi]$$
אם ניקח את \(J\setminus \{ 1 \}\) נקבל איחוד זר של שתי קבוצות פתוחות כאשר אם ניקח כל נקודה במעגל היחידה לא נקבל כי איחוד זר של שתי קבוצות פתוחות לא ריקות.

\end{example}
\begin{remark}
כל תכונה שנובעת מתכונות של הטופולוגיה נשמרת תחת הומאומורפיזם.

\end{remark}
\begin{proposition}[שימור תתי מרחבים תחת הומאומורפיזם]
יהי \(f:(X,\tau)\to (Y,\sigma)\) הומאומורפיזם. אזי לכל \(A\subseteq X\) הפונקציה \(g:(A,\tau|_{A})\to(f(A),\sigma_{f(A)})\) היא הומאומורפיזם.

\end{proposition}
\begin{definition}[העתקה פתוחה]
העתקה שמעבירה קבוצות פתוחות לקבוצות פתוחות. כלומר העתקה \(f:X\to Y\) כך שאם \(A\subseteq X\) פתוחה אז \(f(X)\subseteq Y\) תהיה פתוחה.

\end{definition}
\begin{proposition}
העתקה \(f\) היא פתוחה אם \(f^{-1}\) רציפה.

\end{proposition}
\chapter{הפרדה ומנייה}

\section{אקסיומות ההפרדה}

נניח כי \((X,\mathcal{T})\) מרחב טופולוגי.

\begin{definition}[ניתנות להפרדה]
  \begin{enumerate}
    \item \textbf{תתי קבוצות}\(A,B \subset X\). אזי \(A,B\) ניתנות להפרדה אם קיימות קבוצות פתוחות \(U,V\) כך ש-\(A\subset U\) ו-\(B \subset V\) ומתקיים \(U \cap V = \varnothing\). 


    \item \textbf{נקודות}\(x,y \in X\) נקראות ניתנות להפרדה אם \(\{ x \}\) ו-\(\{ y \}\) ניתנות להפרדה(בתור תתי קבוצות) 


    \item תת קבוצה \(A \subset X\) ונקודה \(x \in X\) נקראות ניתנות להפרדה אם הקבוצות \(\{ x \}\) ו-\(A\) ניתנות להפרדה(בתור תתי קבוצות) 


  \end{enumerate}
\end{definition}
\begin{definition}[אקסיומת ההפרדה \(T_{0}\)]
אם לכל זוג נקודות שונות קיימת קבוצה פתוחה שמכילה אחת מהן ולא את השנייה. כלומר:
$$\forall x\neq y\in X,\;\exists\,U\in{\mathcal{T}}:{\big(}x\in U,\;y\notin U{\big)}{\mathrm{~or~}}{\big(}y\in U,\;x\notin U{\big)}$$

\end{definition}
\begin{definition}[אקסיומת ההפרדה \(T_{1}\)]
לכל זוג נקודות שונות קיימת קבוצה פתוחה שמכילה אחת מהן ולא את השנייה, וקיימת קבוצה פתוחה שמכילה את השנייה ולא את הראשונה. כלומר:
$$\forall x\neq y\in X,\;\exists U\in{\mathcal{T}}\;(x\in U,\;y\notin U)\quad{\mathrm{and}}\quad\exists V\in{\mathcal{T}}\;(y\in V,\;x\notin V)$$

\end{definition}
\begin{proposition}[תנאי שקול לאקסומת ההפרדה \(T_{1}\)]
כל יחידון הוא קבוצה סגורה. כלומר לכל \(x \in X\) נקבל \(\{ x \}\) סגורה.

\end{proposition}
\begin{proof}
  \begin{enumerate}
    \item נניח כי \(X\) הוא \(T_{1}\). לכל \(x \in X\) נסתכל על \(X \setminus \{  x \}\). לכל \(y \neq x\) נקבל מ-\(T_{1}\) קבוצה פתוחה \(V_{y}\) אשר מקיימת \(y \in V_{y}\) ו-\(x \not\in V_{y}\). כעת \(X\setminus \{ x \}=\bigcup_{y\neq x}B_{y}\) פתוחה בתור איחוד של פתוחה ולכן \(\{ x \}\) סגורה ובפרט כל יחידון סגור. 


    \item נניח כי כל יחידון סגור. לכן לכל \(x,y \in X\) נקבל \(\{ y \}\) סגור ו-\(U=X\setminus \{  y \}\) פתוח, עם \(x \in U,y\not\in U\). באופן דומה \(\{ x \}\) סגור ולכן \(V=X\setminus \{ x \}\) פתוח עם \(y\in V,x\not\in V\) ולכן \(X\) היא \(T_{1}\). 


  \end{enumerate}
\end{proof}
\begin{definition}[אקסוימת ההפרדה \(T_{2}\) - האוסדורף]
כל זוג נקודות שונות ניתנות להפרדה. כלומר:
$$\forall x\neq y\in X,\;\exists U,V\in{\mathcal{T}}:\left(x\in U,\;y\in V,\,U\cap V=\varnothing \right)$$

\end{definition}
\begin{definition}[מרחב רגולרי]
מרחב \(X\) נקרא רגולרי אם לכל \(x \in X\) ו-\(A \subseteq X\) סגורה ואינה מכילה את \(x\) קיימות סביבות שמפרידות ביניהם. כלומר:
$$\forall x\in X,\,\forall A\subseteq X{\mathrm{~closed}},\;x\notin A,\quad\exists\,U,V\in{\mathcal{T}}:\,x\in U,\;A\subseteq V,\;U\cap V=\varnothing$$

\end{definition}
\begin{definition}[אקסוימת ההפרדה \(T_{3}\)]
מרחב מקיים את אקסיומת ההפרדה \(T_{3}\) אם רגולרי ומקיים את \(T_{1}\).

\end{definition}
\begin{definition}[מרחב נורמלי]
מרחב \(X\) נקרא נורמלי אם שתי תתי קבוצות זרות של \(X\) ניתנות להפרדה. כלומר מתקיים:
$$\forall A,B\subseteq X{\mathrm{~closed}},\;A\cap B=\varnothing,\quad\exists\,U,V\in{\mathcal{T}}:\,A\subseteq U,\,B\subseteq V,\,U\cap V=\varnothing$$

\end{definition}
\begin{definition}[אקסיומת ההפרדה \(T_{4}\)]
אם המרחב \(X\) מקיימת את \(T_{1}\) וגם נורמלי.

\end{definition}
\begin{proposition}
$$T_{4}\implies T_{3}\implies T_{2}\implies T_{1}\implies T_{0}$$

\end{proposition}
\begin{proof}
  \begin{enumerate}
    \item הגרירה \(T_{1}\implies T_{0}\) היא מיידית - מופיע בהגדרה. 


    \item הגדירה \(T_{2}\implies T_{1}\) נובעת מכך שאם \(U\cap V=\varnothing\) עם \(x \in U\) ו-\(u \in V\) אז \(U\) מפרידה \(x\) מ-\(y\) ו-\(V\) מפרידה \(y\) מ-\(x\). ולכן היחידונים הם סגורים. 


    \item הגרירה \(T_{3}\implies T_{2}\) מתקבל כמקרה פרטי. 


    \item הגרירה \(T_{4}\implies T_{3}\) נובע מכך שאם ניתן להפריד כל זוג קבוצות זרות וסגורות, אז ניתן להפריד גם קבוצה סגורה ונקודה. כי מההנחת \(T_{1}\) כל יחידון היא קבוצה סגורה. 


  \end{enumerate}
\end{proof}
\begin{proposition}
מרחב \(X\) המקיים את \(T_{1}\) הוא נורמלי אם"ם לכל קבוצה סגורה A ולכל קבוצה פתוחה \(U\) עם \(A\subseteq U\) קיימת קבוצה פתוחה \(V\) כך שמתקיים:
$$A\subseteq V\subseteq \overline{V} \subseteq U$$

\end{proposition}
\begin{proof}
  \begin{enumerate}
    \item נניח כי \(X\) נורמלי, תהי \(A\) קבוצה סגורה ו-\(U\) קבוצה פתוחה המכילה את A. לכן \(X\setminus U\) היא קבוצה סגורה הזרה ל-\(A\). 


    \item מנורמליות קיימות קבוצות פתוחות זרות \(W\) ו-\(V\) כך ש-\(A \subset V\) ו-\(X \setminus U \subseteq W\). 


    \item מ-\(X\setminus U \subseteq W\) נקבל \(X\setminus W\subseteq U\) כאשר כיוון ש-\(W\) היא פתוחה נקבל \(X\setminus W\) היא סגורה. 


    \item מ-\(V\cap W=\varnothing\) נקבל \(V\subseteq X \setminus W\), ולכן \(\overline{V}\subseteq X \setminus W\subseteq U\). 


    \item כעת עבור הכיוון השני נניח כי הקיום של \(V\) גורר נורמליות. יהיו \(A,B\) קבוצות סגורות זרות, אזי \(X\setminus B\) היא קבוצה פתוחה המכילה את A. מההנחה, קיים \(V\) כך ש-\(A\subseteq V\subseteq \overline{ V}\subseteq X\setminus B\).  


    \item יהי \(U=X \setminus \overline{V}\) אשר קבוצה פתוחה כיוון ש-\(\overline{ V}\) סגורה. מ-\(\overline{V}\subseteq X \setminus B\) נקבל כי \(B\subseteq X\setminus \overline{ V}=U\). בנוסף \(V\cap U = V\cap(X \setminus \overline{ V})=\varnothing\) כיוון ש-\(V\subseteq \overline{ V}\). לכן \(V,U\) הם קבוצות זרות פתוחות המפרידות את \(A\) ו-\(B\) ולכן \(X\) נורמלי. 


  \end{enumerate}
\end{proof}
\begin{proposition}
המרחב \(X\) הוא האוסדורף(\(T_{2}\)) אם ורק אם מתקיים:
$$X\times X\supseteq \Delta_{X}=\{ (x,x)\mid x \in X \}$$
תת קבוצה סגורה(כמובן בטופולוגיית המכפלה)

\end{proposition}
\begin{proof}
  \begin{enumerate}
    \item נניח כי \(X\) הוא האוסדורף ונקרא כי \(\Delta_{X}\) סגורה. מספיק להראות כי \((X\times X)\setminus \Delta_{X}\) פתוחה.  


    \item לכל \((x,y)\in (X\times X)\setminus \Delta_{X}\) מכך שאוסדורף קיימות קבוצות פתוחות \(U,V \in X\) עם \(x \in U\) ו-\(y \in V\). המכפלה \(U\times V\) פתוחה ב=\(X \times X\) ומכילה את \((x,y)\) וזרה מ-\(\Delta X\)(כיוון שאם \((a,a)\in U\times V\) אז \(a \in U \cap V\) בסתירה) ולכן \(X \times X \setminus \Delta _X\) היא פתוחה. 


    \item עבור הכיוון השני נניח כי \(\Delta X\) סגורה ונראה כי האוסדורף, אנו יודעים כי \((X\times X)\setminus \Delta_{X}\) פתוחה ולכן לכל \((x,y )\in X\times X \setminus \Delta_{X}\) קיים קבוצה פתוחה \(U \times V \subseteq X \times X\) כך ש-\((x,y)\in U \times V\subseteq(X \times X)\setminus \Delta_{X}\) ו-\(U,V\) פתוחות ב-\(X\). 


    \item כעת \(x \in U,y \in V\). מתקיים \(U\cap V=\varnothing\), זאת כיוון שאם לא ניתן לסמן \(z = U \cap V\) כאשר מתקיים \((z,z )\in U\times V\) אבל \((z,z)\in \Delta _X\) בסתירה לכך ש-\(U\times V\subseteq (X\times X)\setminus \Delta_{X}\). כלומר \(U\cap V=\varnothing\) והמרחב \(X\) הוא האוסדורף. 


  \end{enumerate}
\end{proof}
\begin{proposition}
עבור \(i \in \{ 1,2,3 \}\) אם \(X\) הוא \(T_{i}\) ו-\(Y\subseteq X\) תת מרחב אזי גם \(Y\) הוא \(T_{i}\).

\end{proposition}
\begin{proof}
  \begin{enumerate}
    \item נניח כי \(X\) הוא \(T_1\). לכל \(x \in Y\), הקבוצה היחידה \(\{x\}\) סגורה ב-\(X\). הטופולוגיה המושרית על \(Y\) קובעת שקבוצה סגורה ב-\(Y\) אם ורק אם היא חיתוך של קבוצה סגורה ב-\(X\) עם \(Y\). לכן, \(\{x\} \cap Y = \{x\}\) סגורה ב-\(Y\). מכיוון שכל סינגלטון ב-\(Y\) סגור, \(Y\) הוא \(T_1\). 


    \item נניח כי \(X\) הוא האוסדורף(\(T_{2}\)). יהיו \(x, y \in Y\) כך ש-\(x \neq y\). מכיוון ש-\(X\) האוסדורף, קיימות קבוצות פתוחות זרות \(U, V\) ב-\(X\) כך ש-\(x \in U\) ו-\(y \in V\). נתבונן בקבוצות הפתוחות ב-\(Y\): \(U_Y = U \cap Y\) ו-\(V_Y = V \cap Y\). מתקיים \(x \in U_Y\), \(y \in V_Y\). כמו כן, \(U_Y \cap V_Y = (U \cap Y) \cap (V \cap Y) = (U \cap V) \cap Y = \emptyset \cap Y = \emptyset\). לכן, \(Y\) הוא האוסדורף. 


    \item נניח כי \(X\) הוא \(T_3\), ולכן \(X\) הוא \(T_1\) ורגולרי. כבר הראינו ש-\(Y\) הוא \(T_1\). עלינו להראות ש-\(Y\) רגולרי. יהי \(x \in Y\) ו-\(A \subseteq Y\) קבוצה סגורה ב-\(Y\) כך ש-\(x \notin A\). מכיוון ש-\(A\) סגורה ב-\(Y\), \(A = C \cap Y\) עבור קבוצה סגורה \(C\) ב-\(X\). מכיוון ש-\(x \notin A\) ו-\(x \in Y\), נובע ש-\(x \notin C\). מכיוון ש-\(X\) רגולרי, קיימות קבוצות פתוחות זרות \(U, V\) ב-\(X\) כך ש-\(x \in U\) ו-\(C \subseteq V\). נתבונן בקבוצות הפתוחות ב-\(Y\): \(U_Y = U \cap Y\) ו-\(V_Y = V \cap Y\). מתקיים \(x \in U_Y\) ו-\(A = C \cap Y \subseteq V \cap Y = V_Y\). כמו כן, \(U_Y \cap V_Y = (U \cap Y) \cap (V \cap Y) = (U \cap V) \cap Y = \emptyset \cap Y = \emptyset\). לכן, \(Y\) רגולרי. מכיוון ש-\(Y\) הוא \(T_1\) ורגולרי, \(Y\) הוא \(T_3\). 


  \end{enumerate}
\end{proof}
\begin{proposition}[הלמה של אוריסון]
יהי \(X\) מרחב \(T_{4}\). אזי לכל זוג קבוצות סגורות זרות \(C,D\subseteq X\) יש פונקציה רציפה \(f:X\to[0,1]\) כך ש-\(f|_{C}=0\) ו-\(f|_{B}=1\).

\end{proposition}
\begin{summary}
  \begin{itemize}
    \item מרחב נקרא \(T_{0}\) אם עבור כל שתי נקודות לפחות אחת מהם מוכלת בקבוצה אשר לא מכילה את השנייה
    \item מרחב יהיה \(T_{1}\) אם לכל שתי נקודות כל אחת מהם מוכלת בקבוצה פתוחה שאינה מכילה את השנייה.
    \item מרחב יהיה \(T_{2}\) אם לכל שתי נקודות קיימות קבוצות פתוחות אשר אינם מכילות את השנייה, והחיתוך שלהם ריק.
    \item מרחב יהיה \(T_{3}\) אם הוא \(T_{1}\) ולכל נקודה וקבוצה סגורה קיימות קבוצות פתוחות זרות שמפרידות ביניהם(רגולרי).
    \item מרחב יהיה \(T_{4}\) אם הוא \(T_{1}\) ולכל זוג שתתי קבוצות סגורות זרות קיימות זוג של קבוצות פתוחות זרות המכילות אותם.
    \item מתקיים ההכלה:
$$T_{0}\subsetneq T_{1}\subsetneq T_{2}\subsetneq T_{3}\subsetneq T_{4}$$
    \item מרחב \(T_1\) הוא נורמלי אם ורק אם לכל קבוצה סגורה \(A\) וקבוצה פתוחה \(U\) המכילה את \(A\), קיימת קבוצה פתוחה \(V\) כך ש-\(A \subseteq V \subseteq \overline{V} \subseteq U\).
    \item מרחב \(X\) הוא האוסדורף (\(T_2\)) אם ורק אם האלכסון \(\Delta_X = \{(x,x) \mid x \in X\}\) סגור במרחב המכפלה \(X \times X\).
    \item אם מרחב הוא \(T_i\) עבור \(i \in \{1, 2, 3\}\), אז כל תת מרחב שלו הוא גם \(T_i\).
    \item \textbf{הלמה של אוריסון:} במרחב \(T_4\), לכל שתי קבוצות סגורות זרות \(C, D\), קיימת פונקציה רציפה \(f: X \to [0,1]\) המפרידה ביניהן (\(f|_C = 0\), \(f|_D = 1\)).
  \end{itemize}
\end{summary}
\section{אקסיומות המנייה}

\begin{definition}[בסיס למערכת סביבות]
אוסף של קבוצות פתוחות \(\{U_i\}_{i \in I}\) המכילות את \(x\) הוא בסיס למערכת סביבות של \(x\), אם לכל קבוצה פתוחה \(V\) המכילה את \(x\), קיים \(i \in I\) כך ש- \(x \in U_i \subset V\).

\end{definition}
\begin{definition}[מרחב המקיים את אקסיומת המנייה הראשונה]
מרחב טופולוגי \(X\) מקיים את אקסיומת המנייה הראשונה אם לכל \(x \in X\) קיים בסיס בן מנייה למערכת סביבות שלו.

\end{definition}
\begin{definition}[מרחב המקיים את אקסיומת המנייה השנייה]
מרחב טופולוגי \(X\) מקיים את אקסיומת המנייה השנייה אם לטופולוגיה שלו קיים בסיס בן מנייה.

\end{definition}
\begin{definition}[מרחב לינדלוף]
מרחב טופולוגי \(X\) נקרא לינדלוף (Lindelöf) אם לכל כיסוי פתוח של המרחב קיים תת-כיסוי בן מנייה.

\end{definition}
\begin{remark}
לינדלוף זה כמו מקרה כללי יותר של קומפקטיות - קומפקטיות דורש שלכל כיסוי קיים תת כיסוי סופי כאשר לינדלוף דורש שלכל כיסוי קיים תת כיסוי בן מנייה.

\end{remark}
\begin{definition}[מרחב ספרבילי]
מרחב טופולוגי \(X\) נקרא ספרבילי אם קיימת בו קבוצה צפופה שהיא בת מנייה.

\end{definition}
\begin{proposition}
מרחב רגולרי המקיים את אקסיומת המניה השנייה, הוא נורמלי.

\end{proposition}
\begin{proof}
  \begin{enumerate}
    \item יהי \(X\) מרחב רגולרי המקיים את אקסיומת המנייה השנייה. נניח כי \(B\) בסיס בן מנייה לטופולוגיה על \(X\), ויהיו \(A, C \subset X\) קבוצות סגורות וזרות. 


    \item עבור כל \(a \in A\), \(a \notin C\). היות ש-\(C\) סגורה, \(X \setminus C\) פתוחה ומכילה את \(a\). מהיות המרחב רגולרי, קיימת סביבה פתוחה \(U_a\) של \(a\) כך ש-\(\overline{U_a} \subset X \setminus C\), כלומר \(\overline{U_a} \cap C = \emptyset\). אוסף הקבוצות הפתוחות \(\{U_a\}_{a \in A}\) מהווה כיסוי פתוח של \(A\). היות ש-\(X\) מקיים את אקסיומת המנייה השנייה, קיים תת-כיסוי בן מנייה, נסמנו \(\{U_n\}_{n=1}^\infty\), כך ש-\(A \subset \bigcup_{n=1}^\infty U_n\). מתקיים כי לכל \(n\), \(\overline{U_n} \cap C = \emptyset\). 


    \item באופן דומה, עבור כל \(c \in C\), \(c \notin A\). היות ש-\(A\) סגורה, \(X \setminus A\) פתוחה ומכילה את \(c\). מהיות המרחב רגולרי, קיימת סביבה פתוחה \(V_c\) של \(c\) כך ש-\(\overline{V_c} \subset X \setminus A\), כלומר \(\overline{V_c} \cap A = \emptyset\). אוסף הקבוצות הפתוחות \(\{V_c\}_{c \in C}\) מהווה כיסוי פתוח של \(C\). היות ש-\(X\) מקיים את אקסיומת המנייה השנייה, קיים תת-כיסוי בן מנייה, נסמנו \(\{V_m\}_{m=1}^\infty\), כך ש-\(C \subset \bigcup_{m=1}^\infty V_m\). מתקיים כי לכל \(m\), \(\overline{V_m} \cap A = \emptyset\). 


    \item נגדיר סדרות של קבוצות פתוחות: לכל \(k\) טבעי נגדיר \(S_k = U_k \setminus \bigcup_{i=1}^k \overline{V_i}\) ו-\(T_k = V_k \setminus \bigcup_{j=1}^k \overline{U_j}\). נגדיר שתי קבוצות פתוחות: \(O = \bigcup_{k=1}^\infty S_k\) ו-\(P = \bigcup_{k=1}^\infty T_k\). 


    \item ניתן לראות כי \(A \subset O\) וגם \(C \subset P\). (הוכחה מפורטת לחלק זה מושמטת בטקסט המקורי). 


    \item הקבוצות הפתוחות \(O\) ו-\(P\) זרות, כלומר \(O \cap P = \emptyset\). (הוכחה לחלק זה היא מוסרת כתרגיל). 


  \end{enumerate}
\end{proof}
\section{מטריזביליות}

\begin{definition}[מרחב מטריזבילי]
מרחב טופולוגי \(X\) נקרא מטריזבילי אם יש מטריקה \(d\) על \(X\) אשר משרה את הטופולוגיה.

\end{definition}
\begin{proposition}
מטרזביליות נשמר תחת הומאומורפיזם.

\end{proposition}
\begin{proposition}
תת מרחב של מרחב מטריזבילי הוא מטריזבילי.

\end{proposition}
\begin{proof}
אם \((X,d)\) מרחב מטרי ו-\(A\subseteq X\) אזי \((Y,d|_{Y})\) מרחב מטרי.

\end{proof}
\begin{theorem}[מטריזביליות של אוריסון]
אם \(X\) מרחב טופולוגי \(T_{3}\) שמקיים את אקסיומת המנייה השנייה אזי \(X\) מטריזבילי.

\end{theorem}
\begin{proof}
הרעיון הוא לשכן במרחב מטרי שאנחנו מכירים. נשכן ב-\([0,1]^{\mathbb{N}}\) עם טופולוגיית המכפלה.

\end{proof}
\begin{proposition}
במרחב מטריזבילי לידלאוף שקול למנייה שנייה שקול לספרבילי.

\end{proposition}
\begin{proposition}
כל מרחב מטרי הוא \(T_{2},T_{3}\) ו-\(T_{4}\).

\end{proposition}
\begin{proposition}
כל מרחב מטרי הוא מנייה ראשונה.

\end{proposition}
\chapter{קשירות}

\section{מרחבים קשירים}

\begin{definition}[מרחב קשיר]
מרחב טופולוגי \(X\) הוא קשיר אם לא ניתן להציג אותו כאיחוד זר של שתי קבוצות פתוחות(או באופן שקול סגורות), זרות לא ריקות.

\end{definition}
\begin{remark}
נשים לב כי קשירות זו תכונה טופולוגית כיוון שמנוסחת רק על ידי קבוצות פתוחות, לכן נשמרת תחת הומאומורפיזם.

\end{remark}
\begin{proposition}
מרחב הוא קשיר אם"ם התתי קבוצות היחידות של \(X\) שהם גם סגורות וגם פתוחות הם \(X\) עצמו והקבוצה הריקה \(\varnothing\).

\end{proposition}
\begin{proof}
אם \(\varnothing\neq A\subsetneq X\) שגם פתוחה וגם סגורה אז כיוון שסגורה \(X \setminus A\) פתוחה ולכן \(A\) ו-\(X\setminus A\) הם שתי קבוצות פתוחות זרות אשר האיחוד שלהם יהיה \(X\) ולכן המרחב לא קשיר. בכיוון השני אם המרחב קשיר אז קיימות \(U,V\) זרות לא ריקות שהאיחוד שלהם \(X\) ולכן \(V=X\setminus U\) ונקבל ש-\(U\) תת קבוצה ממש סגורה לא ריקה של \(X\).

\end{proof}
\begin{example}
הקבוצה \(\mathbb{R}\) קשירה כאשר \(\mathbb{Q}\) לא קשיר. כאשר עבור \(\mathbb{R}\) התתי קבוצות הקשירות יהיו:
$$[a,b]\qquad (a,b)\qquad [a,b)\qquad (b,a]$$

\end{example}
\begin{lemma}
אם הקבוצות \(C,D\) לא ריקות הם הפרדה של \(X\), כלומר \(C\sqcup D = X\) ו-\(Y\) תת מרחב קשיר של \(X\), אז \(Y\) מוכל לחלוטין או ב-\(C\) או ב-\(D\).

\end{lemma}
\begin{proof}
אחרת הקבוצות \(C\cap Y\) ו-\(D\cap Y\) יהיו הפרדה של \(Y\).

\end{proof}
\begin{proposition}
אם \(f:X\to Y\) רציפה ו-\(X\) קשיר אז \(f(X)\) קשירה.

\end{proposition}
\begin{proof}
יהי \(f:X\to Y\) העתקה רציפה ו-\(X\) קשירה. נגדיר \(Z=f(X)\) וכעת גם הצמצום של \(f\) המוגדר על ידי \(g:X\to Z\) יהיה העתקה רציפה. נניח \(Z=A\sqcup B\) הפרדה של \(Z\) לשתי קבוצות זרות לא ריקות. אזי \(g^{-1}(A)\) ו-\(g^{-1}(B)\) הם הפרדות לא ריקות של \(X\) שהאיחוד שלהם הוא \(X\), בסתירה לכך ש-\(X\) קשיר.

\end{proof}
\begin{proposition}[למת הכוכב]
איחוד של מרחבים קשירים שיש להם נקודה משותפת הם קשירים. כלומר, אם \(\{ A_{\alpha} \}_{\alpha \in I}\) תתי קבוצות קשירות וקיים \(\beta \in I\) כך ש-\(A_{\alpha} \cap A_{\beta}\) ב-\(X\) קשירה לכל \(\alpha \in I\) אז \(\bigcup_{\alpha \in I}U_{\alpha}\) קשיר.

\end{proposition}
\begin{proof}
יהיו \(\left\{  A_{\alpha}  \right\}\) קבוצה של תתי מרחבים קשירים של \(X\), ותהי \(p \in \bigcap_{\alpha} A_{\alpha}\). נניח בשלילה ש-\(Y=\bigcup_{\alpha}A_{\alpha}\) לא קשיר. לכן קיימות קבוצות \(C,D\) לא ריקות כך ש-\(Y=C\sqcup D\). נדרש כי \(p \in C\) או \(p \in D\). נניח בלי הגבלת הכלליות כי \(p \in C\). מהלמה כיוון ש-\(A_{\alpha}\) קשירה ולכן \(A_{\alpha}\subseteq C\). כיוון שנכון לכל \(\alpha\) נקבל \(\bigcup A_{\alpha}\subseteq C\) בסתירה לכך ש-\(D\) לא ריקה.

\end{proof}
\begin{proposition}
אם \(A\) קשירה כך ש-\(A\subseteq B\subseteq \overline{A}\) אזי \(B\) גם קשירה. כלומר הוספה של נקודות גבול לא משנה קשירות. 

\end{proposition}
\begin{proof}
נניח בשלילה כי \(B\) לא קשירה. אזי קיימים \(C,D\) לא ריקות כך ש-\(B=C\sqcup D\). לכן מהלמה נדרש כי \(A\subseteq C\) או \(A\subseteq D\). נניח בלי הגבלת הכלליות \(A\subseteq C\). אזי \(\overline{A}\subseteq \overline{C}\). כיוון ש-\(C\) סגור ב-\(B\) אז \(\overline{C},D\) הם זרות ולכן גם \(B\subseteq \overline{C}\) ונקבל \(B\cap D=\varnothing\) בסתירה לכך ש-\(D\) לא ריקה. 

\end{proof}
\begin{proposition}
מכפלה סופית של מרחבים קשירים היא קשירה.

\end{proposition}
\begin{proposition}
מרחב טופולוגי \(X\) הוא קשיר אם"ם כל פונקציה רציפה \(f:X\to \{ 0,1 \}\) היא קבועה.

\end{proposition}
\begin{proof}
נניח כי \(X\) קשירה ו-\(f:X\to \{ 0,1 \}\) פונקציה רציפה. נניח בשלילה כי \(f\) לא קבוע. נגדיר \(U=f^{-1}(\{ 0 \})\) ו-\(V=f^{-1}(\{ 1 \})\). כיוון ש-\(f\) לא קבועה נקבל \(U,V\neq \varnothing\). כיוון ש-\(\{ 0 \},\{ 1 \}\) הם קבוצות פתוחות ו-\(f\) רציפה אזי גם \(U,V\) קבוצות פתוחות. הקבוצות \(U,V\) זרות כי אם \(x \in U\cap V\) אז \(f(x)=0\) וגם \(f(x)=1\) בסתירה. האיחוד של \(U,V\) יהיה \(X\) כי:
$$U\cup  V=f^{-1} (\{ 0 \})\cup  f^{-1} (\{ 1 \})=f^{-1} (\{ 0 \}\cup \{ 1 \})=f^{-1} (\{ 0,1 \})=X$$
ולכן קיבלנו כי \(X\) לא קשירה בסתירה.
עבור הכיוון השני נניח כי לכל פונקציה  \(f:X\to \{ 0,1 \}\) רציפה היא קבועה. נניח בשלילה כי \(f\) לא קשירה. לכן קיימות \(\varnothing\neq U,V\subseteq X\) פתוחות כך ש-\(U\cup V=X\) וגם \(U\cap V=\varnothing\). נגדיר פונקציה \(f:X\to \{ 0,1 \}\) באופן הבא:
$$f(x)=\begin{cases}0 & x \in U \\1 & x \in V
\end{cases}$$
כיוון ש-\(V,U\) לא ריקות זוהי לא פונקציה קבועה. נראה כי רציפה. נדרש להראות כי המקור של כל קבוצה פתוחה ב-\(\{ 0,1 \}\) היא פתוחה ב-\(X\). כל הקבוצות הפתוחות ב-\(\{ 0,1 \}\) הם \(\varnothing,\{ 0 \},\{ 1 \},\{ 0,1 \}\) ולכן מספיק לעבור על כולם. מתקיים \(f^{-1}(\varnothing)=\varnothing\), אשר פתוחה ב-\(X\). כמו כן \(f^{-1}(\{ 0 \})=U\) ו-\(f^{-1}(\{ 1 \})=V\) אשר קבוצות פתוחות לפי ההנחה. בנוסף \(f^{-1}(\{ 0,1 \})=X\) אשר פתוחה ב-\(X\). לכן קיבלנו כי הפונקציה רציפה בסתירה, ולכן התחום קשיר.

\end{proof}
\begin{definition}[קשירה מקומית]
נאמר שמרחב טופולוגי \(X\) קשיר מקומית בנקודה \(x \in X\) אם לכל סביבה \(W\) של \(x\) יש קבוצה פתוחה וקשירה \(x \in U \subseteq W\). נאמר ש-\(X\) קשיר מקומית אם \(x\) קשיר מקומית לכל \(x \in X\).

\end{definition}
\begin{definition}[רכיב קשירות]
רכיב הקשירות של \(x\) במרחב טופולוגי \(X\) הוא תת הקבוצה הקשירה מקסימלית אשר מכילה את \(x\).

\end{definition}
\begin{remark}
אכן קיימת קבוצה כזו, כיוון שהטופולוגיה סגורה לאיחוד, ניתן לבחור את:
$$\bigcup_{x \in Z\subseteq X} Z$$

\end{remark}
\begin{example}
הרכיב קשירות של \(\frac{1}{3}\) ב-\(\mathbb{Q}\) יהיה \(\left\{  \frac{1}{3}  \right\}\).

\end{example}
\begin{proposition}
להיות באותו תת מרחב קשיר זה יחס שקילות, כאשר מחלקות השקילות יהיו רכיבי הקשירות.

\end{proposition}
\begin{corollary}
האיחוד של רכיבי הקשירות הוא המרחב כולו, ורכיבי הקשירות הם זרים.

\end{corollary}
\begin{proposition}
מרחב \(X\) קשיר מקומית אם"ם כל לכל קבוצה פתוחה \(U\subseteq X\) רכיבי הקשירות של \(U\) הם פתוחים ב-\(X\).

\end{proposition}
\begin{proof}
נניח ש-\(X\) קשיר מקומית. תהי \(U\) קבוצה פתוחה של \(X\) ו-\(C\) רכיב קשירות של \(U\). יהי \(x \in C\). כיוון ש-\(X\) קשיר מקומית קיימת סביבה פתוחה \(V\) של \(c\) כך ש-\(V\subseteq U\) קשיר. כיוון ש-\(V\) קשיר צריך להיות מוכל ברכיב הקשירות \(C\). ולכן \(C\) פתוח ב-\(X\) כי לכל נקודה יש סביבה פתוחה.
עבור הכיוון השני נניח רכיבי הקשירות של הקבוצות הפתוחות של \(X\) הם פתוחות. תהי \(x \in X\) ו-\(U\) סביבה פתוחה של \(x\). יהי \(C\) הרכיב הקשירות של \(U\) המכיל את \(x\). \(C\) קשיר, וכיוון שפתוח ב-\(X\) לפי ההנחה, נקבל כי \(X\) קשיר מקומית ב-\(x\).

\end{proof}
\section{קשירות מסילתית}

\begin{definition}[מסילה]
יהיו \(x,y\in X\). מסילה ב-\(X\) היא העתקה רציפה \(f:[0,1]\to X\) כך ש-\(f(0)=x\) ו-\(f(1)=y\).

\end{definition}
\begin{remark}
לפעמים מגדירים על ידי פונקציה \(f:[a,b]\to X\).

\end{remark}
\begin{definition}[קשירה מסילתית]
מרחב טופולוגי \(X\) נקרא קשיר מסילתית אם לכל שתי נקודות קיימת מסילה המחברת ביניהם, כלומר לכל \(x,y \in X\) קיימת מסילה \(\alpha:[0,1]\to X\) כך ש-\(\alpha(1)=y\) ו-\(\alpha(0)=x\).

\end{definition}
\begin{example}
כדור היחידה \(B^{n}\) ב-\(\mathbb{R}^{n}\) היא קשירה מסילתית כי לכל שתי נקודות \(x,y \in B^{n}\) ניתן לחבר על ידי הפונקציה הרציפה:
$$f:[0,1]\to X\qquad f(t)=(1-t)\mathbf{x}+t\mathbf{y}$$

\end{example}
\begin{definition}[קשירה מסילתית מקומית]
המרחב \(X\) קשיר מסילתית מקומית ב-\(x \in X\) אם לכל סביבה \(W\) של \(x\) יש קבוצה פתוחה \(x \in U\subseteq W\) כך ש-\(U\) קשירה מסיליתית. המרחב \(X\) קשיר מסיליתית מקומית אם קשיר מסיליתית מקומית בכל \(x \in X\).

\end{definition}
\begin{proposition}
אם \(X\) קשיר מסילתית ו-\(f:X\to Y\) רציפה אז \(f(X)\) קשירה מסילתית.

\end{proposition}
\begin{proposition}
אם \(X\) קשיר מסיליתית אז \(X\) קשיר.

\end{proposition}
\begin{proof}
אם \(X\) לא קשיר אז יש פונקציה רציפה \(f:X\to \{ 0,1 \}\) עם הטופולוגיה הדיסקרטית כך ש-\(f(X)=\{ 0,1 \}\) אבל \(\{ 0,1 \}\) לא קשיר מסילתית כי \([0,1]\) קשיר.

\end{proof}
\begin{remark}
הכיוון השני לא נכון(קשירות לא גורר קשירות מסילתית). לדוגמא אם נתבונן בגרף של \(\sin\left( \frac{1}{x} \right)\) עבור \(0<x\leq 1\) נקבל תת קבוצה \(G\subseteq \mathbb{R}^{2}\). ונניח ש-\(X\) הסגור של הגרך הזה ב-\(\mathbb{R}^{2}\). וכן:
$$X=\{ 0 \}\times[-1,1]\cup  G$$
הסגור של קבוצה קשירה הוא קשיר ולכן סגור זה אכן קשיר, אך לא קשיר מסילתית, כיוון שלא קיימת מסילה \(\alpha:[0,1]\to X\) כך ש:
$$\alpha(0)=(0,0)\quad \alpha(1)=(1,\sin 1)$$

\end{remark}
\begin{proposition}
אם \(X\) קשיר וקישיר מסילתית מקומית אזי \(X\) קשיר מסילתית.

\end{proposition}
\begin{proof}
תהי \(x_{0} \in X\). נתבוהן במרכיבי קשירות המסלתית של \(x_{0}\) שנסמנו ב-\(A\). מתקיים \(x_{0} \in A\neq \varnothing\). \(A\) פתוחה כי לכל \(a \in A\) ו-\(X\) קשיר מסילתית מקומית בפרט יש סביבה פתוחה קשירה מסילתית של \(a\) וב-\(X\).
ולכן \(U\subseteq A\) ולכל מריכבי הקשירות מסיליתית יתאים המשלים של שאר מרכיבי הקשירות המסילתית ולכן מסגירות של איחודת פתוחות גורר הוא סגור.

\end{proof}
\begin{proposition}
כמו כל תכונה שקשורה לטופולוגיה, טופולוגיה משמרת קשירות מסילתית.

\end{proposition}
\begin{example}
נראה שמרחב \(\mathbb{R}^{1}\) לא הומאומורפי למרחב \(\mathbb{R}^{2}\). נניח בשלילה שהומאומורפי, כלומר קיים \(\varphi:\mathbb{R}^{1}\to \mathbb{R}^{2}\) רציף עם הופכי רציף.
ניקח \(s \in \mathbb{R}\). מתקיים עבור הטופולוגיות התת מרחב כי \(\mathbb{R}\setminus \{ s \}\) הומאומורפי ל-\(\mathbb{R}^{2}\setminus \varphi(s)\). נשים לב כי \(\mathbb{R}\setminus \{ s \}\) לא קשיר מסיליתית, כאשר \(\mathbb{R}^{2}\setminus \varphi(s)\) קשיר מסילתית כי אם ניתן לחבר בקו ישר אז זו המסילה ואם לא ניתן לחבר בקו ישר ניתן להקיף את הנקודת אי רציפות(ניתן לבנות מפורשות). ולכן סתירה, ונקבל כי לא הומאומורפיים.

\end{example}
\chapter{קומפקטיות}

\section{קומפקטיות}

\begin{definition}[כיסוי]
תהי \(A\) תת קבוצה של מרחב טופולוגי \(X\), ותהי \(\mathcal{O}\) אוסף של תתי קבוצה של \(X\). האוסף \(\mathcal{O}\) הוא כיסוי של \(A\) אם \(A\subseteq \bigcup_{ B \in \mathcal{O}}B\).

\end{definition}
\begin{definition}[כיסוי פתוח]
כיסוי אשר מורכב מקבוצות פתוחות.

\end{definition}
\begin{definition}[תת כיסוי]
אם \(\mathcal{O}\) מכסה את \(A\) ו-\(\mathcal{O}'\) תת אוסף של \(\mathcal{O}\) אשר גם כן מכסה את \(A\) אז \(\mathcal{O'}\) נקרא תת כיסוי של \(\mathcal{O}\).

\end{definition}
\begin{definition}[מרחב קומפקטי]
מרחב טופולוגי \(X\) יקרא קומפקטי אזי לכל כיסוי פתוח של \(X\) יש תת כיסוי סופי.
כלומר לכל אוסף קבוצות פתוחות \(\{ U_{\alpha} \}_{\alpha \in I}\) כך ש-\(X=\bigcup_{\alpha \in I}U_{\alpha}\) יש תת אוסף סופי, כלומר \(I_{0} \subseteq I\) כך ש-\(\lvert I_{0} \rvert<\infty\) ומתקיים:
$$X=\bigcup_{\alpha \in I_{0}}U_{\alpha}$$

\end{definition}
\begin{example}
הישר הממשי \(\mathbb{R}\) אינו קומפקטי כי:
$$\mathcal{O}=\{...\;,(-1,\,1),\,(0,2),\,(1,3),\,...\}$$
הוא כיסוי פתוח אך אין תת כיסוי סופי אשר מכסה את \(\mathbb{R}\)

\end{example}
\begin{example}
כל מרחב טופולוגי סופי הוא קומפקטי כי קיים עבורו כמות סופית של קבוצות פתוחות, ולכן כל כיסוי פתוח יהיה סופי בכל מקרה.

\end{example}
\begin{definition}[תת מרחב קומפקטי]
תת קבוצה \(K\subseteq X\) במרחב טופולוגי \(X\) תקרא קומפקטית ב-\(X\) אם היא מרחב טופולוגי קומפקטי עם הטופולוגיה המורשת מ-\(X\).

\end{definition}
\begin{proposition}
תהי \(f:X\to Y\) פונקציה רציפה ותהי \(A\) קומפקטית ב-\(X\). אזי \(f(A)\) קומפקטית ב-\(Y\).

\end{proposition}
\begin{proof}
נניח \(f(x)\subseteq \bigcup_{\alpha \in I}U_{\alpha}\) אזי:
$$X\subseteq f^{-1} \left( \bigcup_{\alpha \in I}U_{\alpha} \right)=\bigcup_{\alpha \in I}f^{-1} (U_{\alpha})$$
וזה גורר כי יש \(I_{0} \in I\) סופית כך ש-\(X\subseteq \bigcup_{{\alpha \in I_{0}}}f^{-1}(U_{\alpha})\).

\end{proof}
\begin{corollary}
מרחב המנה של מרחב קומפקטי \(X\) הוא קומפקטי כיוון שהתמונה של \(X\) תחת העתקת מנה.

\end{corollary}
\begin{proposition}
איחוד סופי של מרחבים קומפקטים הוא קומפקטי, כלומר אם \(X\) מרחב טופולוגי, ו-\(C_{1},\dots,C_{n}\) הם קומפקטיים ב-\(X\) אזי \(\bigcup_{j=1}^{n}C_{j}\) קומפקטי ב-\(X\). 

\end{proposition}
\begin{proposition}
עבור מרחבי האוסדורף, חיתוך של מרחבים קומפקטים הם קומפקטים, כלומר אם \(X\) האוסדורף, ו-\(\left\{  C_{\alpha}  \right\}_{\alpha \in A}\) אוסף קבוצות קומפקטיות ב-\(X\), אזי \(\bigcap_{\alpha \in A}C_{\alpha}\) קומפקטי ב-\(X\).

\end{proposition}
\begin{proposition}
אם \(X\) קומפקטית ו-\(A\subseteq X\) סגורה אזי \(A\) קומפקטית.

\end{proposition}
\begin{proof}
נניח כי \(\mathcal{O}\) כיסוי פתוח של \(A\). הקבוצה \(X\setminus A\) היא פתוחה, לכן ניתן לבנות כיסוי פתוח נוסף על ידי:
$$\mathcal{O} '=\mathcal{O}\cup \left\{  X\setminus A  \right\} $$
האוסף הזה כיסוי פתוח של \(X\) ולכן כיוון ש-\(X\) קומפקטי קיים עבור תת כיסוי סופי. וכיסוי הזה הוא בפרט כיסוי של \(A\) כי \(A\subseteq X\) ולכן \(A\) קומפקטי.

\end{proof}
\begin{proposition}
אם \(X\) מרחב האוסדורף ו-\(A\subseteq X\) קומפקטית אז \(A\) סגורה ב-\(X\).

\end{proposition}
תהי \(A\) קומפקטית ב-\(X\). כדי להראות ש-\(A\) סגורה נראה כי \(X\setminus A\) פתוחה. יהי \(x \in X\setminus A\). נראה כי קיים \(U\) פתוח כך ש-\(x \in U\subseteq X \setminus A\). כיוון ש-\(X\) האוסדורף, לכל \(a \in A\) קיים קבוצות זרות \(U_{a}\) ו-\(V_{a}\) פתוחות כך ש-\(x \in U_{a}\) ו-\(a \in V_{a}\). ניקח את הסביבות האלה עבור כל הנקודות ונקבל כיסוי פתוח של \(A\). כלומר:
$$\mathcal{O} =\{ V_{a} \}_{a \in A}$$
הוא כיסוי פתוח של \(A\). כיוון ש-\(A\) קומפקטית יש תת כיסוי סופי \(\left\{  V_{a_{1}},\dots,V_{a_{n}}  \right\}\) של \(\mathcal{O}\). נסמן \(V=\bigcup_{i=1}^{n}V_{a_{i}}\) ו-\(U=\bigcap_{i=1}^{n}U_{a_{i}}\). נשים לב כי \(V,U\) הם קבוצות פתוחות כך ש-\(A\subseteq V\) ו-\(x \in U\) וכן כיוון שלכל \(i\)\(U_{a_{i}},V_{a_{i}}\) זרות בפרט גם \(U,V\) זרות, ולכן \(x \in U \subseteq X \setminus A\) כמו שרצינו להראות, ולכן \(X\setminus A\) פתוחה ו-\(A\) סגורה.

\begin{proposition}
מרחב טופלוגי \(X\) הוא קומפקטי אם"ם לכל אוסף \(\{ F_{\alpha} \}_{\alpha \in I}\) של תתי קבוצות סגורות של \(X\) כך שיש להן תכונות החיתוך הסופי החיתוך \(\bigcap_{\alpha \in I}F_{\alpha}\neq \varnothing\). נאמר של-\(\{ F_{\alpha} \}_{\alpha \in I}\) יש תכונת החיתוך הסופי אם לכל \(I_{0} \subseteq I\) כך ש-\(\lvert I_{0} \rvert<\infty\) מתקיים \(\bigcap_{\alpha \in I_{0}}F_{\alpha }\neq \varnothing\).
זה שקול לכך שאם \(F_{\alpha}\) סגורה לכל \(\alpha \in I\) ו-\(\phi = \bigcap_{\alpha \in I} F_{\alpha}\) אזי יש \(I_{0}\subseteq I\) סופית כך ש-\(\phi=\bigcap_{\alpha \in I_{0}}F_{\alpha}\).

\end{proposition}
\begin{remark}
ראתים שקבוצה \(A\subseteq \mathbb{R}\) היא קומפקטית אם"ם סגורה וחסומה.

\end{remark}
\begin{proposition}
תת קבוצה קומפקטית \(A\) במרחב טופולוגי האוסדורף \(X\) היא סגורה.

\end{proposition}
\begin{remark}
קיימים מרחבים טופולוגיים עם תת קבוצה קומפקטית שאינה סגורה. לדוגמא, עבור \(X=\{ a,b \}\) עם הטופולוגיה הטריוויאלית. הקבוצה \(A=\{ a \}\) קומפקטית אבל לא סגורה.

\end{remark}
\begin{proposition}
אם \(X\) קומפקטית ו-\(A\subseteq X\) סגורה אזי \(A\) קומפקטית. זה באמת נכון באופן כללי.

\end{proposition}
\begin{proposition}
תמונה רציפה של מרחב טופולוגי היא קומפקטית. כלומר אם \(X\) מרחב טופולוגי קומפקטי ו-\(f:X\to Y\) פונקציה רציפה מ-\(X\) למרחב טופולוגי \(Y\) אזי \(f(X)\subseteq Y\) קומפקטית.

\end{proposition}
\begin{proof}
נניח \(f(x)\subseteq \bigcup_{\alpha \in I}U_{\alpha}\) אזי:
$$X\subseteq f^{-1} \left( \bigcup_{\alpha \in I}U_{\alpha} \right)=\bigcup_{\alpha \in I}f^{-1} (U_{\alpha})$$
וזה גורר כי יש \(I_{0} \in I\) סופית כך ש-\(X\subseteq \bigcup_{{\alpha \in I_{0}}}f^{-1}(U_{\alpha})\).

\end{proof}
\begin{proposition}
אם \(X\) מרחב האוסדורף קומפקטי אזי \(X\) מרחב רגולרי.

\end{proposition}
\begin{proof}
נזכור כי רגולרית אם"ם אפשר להפריד בין כל קבוצה סגורה \(A\) ו-\(b \not\in A\). קבוצה \(A\subseteq X\) סגורה קומפקטית אז \(A\) קומפקטית. לכל \(a \in A\) ו-\(a\neq b\) נקבל 
להשלים

\end{proof}
\begin{corollary}
יהי \(X\) מרחב טופולוגי קומפקטי, \(Y\) מרחב טופלוגי האוסדרוף ו-\(f:X\to Y\) רציפה, חח"ע ועל. אזי \(f\) הומאומורפיזם.

\end{corollary}
\begin{proof}
כל מה שנ
להשלים

\end{proof}
\begin{proposition}
אם \(X\) מרחב האוסדורף קומפקטי אזי \(X\) מרחב רגולרי.

\end{proposition}
\begin{proposition}
אם \(X\) מרחב האוסדורף קומפקטי אזי \(X\) מרחב נורמלי.

\end{proposition}
\begin{proposition}
יהי \(X\) מרחב טופולוגי קומפקטי ו-\(f:X\to \mathbb{R}\) רציפה אזי:

  \begin{enumerate}
    \item הפונקציה \(f(x)\) חסומה וסגירה 


    \item יש ל-\(f\) מקסימום ומינימום. 


    \item נניח \(X\) מטריזבילי ותהי \(d\) מטריקה אזי \(f\) רציפה במ"ש. (תרגיל - רמז - מספר לבג) 


  \end{enumerate}
\end{proposition}
\begin{proof}
  \begin{enumerate}
    \item כי ראינו \(f(x)\subseteq \mathbb{R}\) קומפקטית ותת קבוצה קומפקטית של \(\mathbb{R}\) היא סגורה וחסומה. 


    \item נסמן \(M=\sup_{x \in X} f(x)<\infty\). נגדיר \(A=f(X)\). נאמר ש-\(M\) היא סופרמום של A זה לאמר שלכל \(x \in X\) נקבל \(f(x)\leq M\) וגם לכל \(0<\varepsilon\) יש \(x \in X,a \in A\) כך ש-\(M-\varepsilon \leq a=f(x)\). 


  \end{enumerate}
\end{proof}
\begin{definition}[מספר לבג]
יהי \(X\) מרחב ופלוגי קומפקטי מטרי ו-\(\{ U_{\alpha} \}\) כיסוי פתוח של \(X\). אזי \(\lambda> 0\) יקרא מספר לבג של הכיסוי אם לכל \(x \in X\) קיימים \(\alpha \in I\) כך ש-\(B_{\lambda}(x)\subseteq U_{\alpha}\).

\end{definition}
\begin{reminder}
עבור מרחבים מטריים יש לנו:
- סדרת קושי
- שלמות
- מרחב מטרי \(X\) נקרא חסום לחלוטין אם לכל \(\varepsilon> 0\) יש כיסוי של \(X\) על ידי מספר סופי של כדורים פתוחים ברדיוס \(\varepsilon\).

\end{reminder}
\begin{theorem}
יהי \(X\) מרחב אזי התנאים הבאים שקולים:

  \begin{enumerate}
    \item הקבוצה \(X\) קומפקטית. 


    \item הקבוצה \(X\) סדרתית. 


    \item הקבוצה \(X\) שלמה וחסום לחלוטין. 


  \end{enumerate}
\end{theorem}
\begin{proof}
לא עשינו.

\end{proof}
\begin{definition}[סדרה מתכנסת]
סדרה \((x_{n})\) במרחב טופלוגי \(X\) מתכנסת ל-\(x \in X\) אם לכל סביבה פתוחה \(x \in U\) מתקיים \(x_{n}\in U\) לכמעט כל \(n\). כלומר יש \(N \in \mathbb{N}\) כך ש-\(N\leq n\) ו-\(x_{n}\in U\)

\end{definition}
\begin{remark}
אין יחידות הגבול באופן כללי. יחידות הגבול זה עבור מרחבי האוסדורף.

\end{remark}
\begin{definition}[קומפקטיות סדרת]
אם כל סדרה של איברים בסדרה מתכנסת לאיבר בסדרה.

\end{definition}
\begin{remark}
באופן כללי קומפקטיות לא גורר קומפקטיות סדרתית, וקומפקטיות סדרתית לא גורר קומפקטיות.

\end{remark}
\begin{example}
נראה קומפקטיות סדרתית לא גורר קומפקטיות. תהי \(I=[0,1],X=\{ 0,1 \}\) טופלוגיה מכפלה וסופית. לסיים.

\end{example}
\section{קופקטיפיקציה}

\section{קומפקטיפיקציית סטון צ'ך}

\begin{theorem}
יהי \(X\) מרחב טופולוגי האוסדורף קומפקטי מקומית, אזי קיים מרחב טופולוגי קומפקטי האוסדורף \(Y\) כך שיש שיכון \(\iota:X\hookrightarrow Y\) כך ש-\(\overline{\iota(X)}=Y\) וכל פונקציה \(f:X\to \mathbb{R}\) רציפה וחסומה של \(X\) ניתנת להרחב לפונקציה רציפה של \(Y\), כאשר ההרחבה היא יחידה.

\end{theorem}
\begin{proof}
נגדיר \(F=C(X,[0,1])\) אוסף כל הפונקציהות הרציפות מ-\(X\) ל-\([0,1]\subseteq \mathbb{R}\). נתבונן בחרב המכפלה \([0,1]^{F}\) כאשר לפי משפט טיכונוף \([0,1]^{F}\) מרחב קופקטי. כמו כן הוא אוסדורף. נגדיר העתקה:
$$\iota :X\to[0,1]^{F}\qquad \forall x \in X, \forall f \in F\quad \iota(x)(f)=f(x)$$
ונגדיר \(Y=\overline{\iota}\). נשים לב כי \(Y\) קומפקטי מקומית כי היא תת קבוצה סגורה של מרחב קומפקטי,  וכן \(Y\) האוסדורף. כעת \(X\hookrightarrow Y\) שיכון אם"ם ההעתקה חח"ע כך ש-\(X\to \iota(X)\) הומאומורפיזם.

\end{proof}
להשלים להראות שחח"ע בעזרת הלמה של אוריסון.

נותן להראות ש-\(\iota:X\to \iota(X)\) היא הומאומורפיזם. כלומר צירך להראות שכל קבוצה פתוחה \(W\subseteq X\) נקבל כי \(\iota(W)\subseteq \iota(X)\) פתוחה. (וגם באופן כללי צריך להראות ש-\(\iota\) פונקציה רציפה אך זה מתקיים כי \(f\) רציפה?)
תהי \(W\subseteq X\) פתוחה לא ריקה. רציפם להראות ש-\(\iota(W)\) פתוחה

....

\begin{proposition}
יהי \(X\) מרחב טופלוגי הואסדורף קומפקטי מקומית, כאשר \(C\) קומפקטי + האוסדורף, אזי כל פונקציה רציפה \(\varphi:X\to C\) נותנת הרחבה רציפה \(\hat{\varphi}:\beta(X)\to C\)

\end{proposition}
\begin{proof}
קיימת קבוצת אינדקסים \(J\) כך שיש שיכון \(C\hookrightarrow[0,1]^{J}\). 
...

\end{proof}
\begin{proposition}
התכונה שניתן להרחיב כל פונקציה רציפה וחסומה \(X\) ל-\(\beta(X)\) באופן יחיד מגדירה את \(\beta(X)\) כד כדי הומאומורפיזם.

\end{proposition}
\begin{definition}[קבוצה דלילה]
קבוצה \(A\subseteq X\) תקרא דלילה אם \((\overline{A})^{\circ}=\varnothing\). 

\end{definition}
\begin{example}
למשל \(\mathbb{Z}\subseteq \mathbb{R}\) או קבוצת קנטור הן דלילות.

\end{example}
\begin{definition}[קבוצה מקטגוריה ראשונה/שנייה]
קבוצה נקראת מקטגוריה ראשונה היא איחוד בין מנייה של קבוצות דלילות. אחרת נאמר שהיא מקטגוריה שנייה.

\end{definition}
\begin{theorem}[בייר]
יהי \(X\) מרחב קומפקטי האוסדורף או מרחב מטרי שלם
אזי לכל אוסף בין מנייה \(\{ A_{n} \}_{n=1}^{\infty}\) מתקיים שלאיחוד \(\bigcup_{n=1}^{\infty}A_{n}\) יש פנים ריק.

\end{theorem}
יש כאן למעשה שתי מקרים שצריך להוכיח - את המקרי של מרחב קומפקטי האוסדורף ואת המקרה של מרחב מטרי שלם. 

\begin{proposition}
משפט בייר שקול לכך שאם \(\{ u_{n} \}_{n=1}^{\infty}\) קבוצות פתוחות וצפופות אזי \(\bigcap_{n=1}^{\infty} U_{n}\) צפופה.

\end{proposition}
נוכיח את משפט בייר עבור המקרה של קומפקטי מקומית, המקרה של מרחב מטרי שלם הוא דומה עם כמה הבדלים טכניים.

\begin{proof}
נניח \(X\) קומפקטי והסאודורף. נניח ש-\(A_{n}\) דלילה. נניח בה"כ כי \(A_{n}\) סגורות, ונוכיח של \(\bigcup_{n=1}^{\infty}A_{n}\) יש פנים ריק.
תהי \(U\subseteq X\) פתוחה לא ריקה ונראה ש-\(U\not\subseteq \bigcup_{n=1}^{\infty}A_{n}\). נבנה סדרה של קבוצות פתוחות \(V_{n}\) באופן הבא - \(X\) קומפקטי האוסדורף - נורמלי. \(A_{1}\) דלילה וסגורה - \(U_{1}=U\cap (X \setminus A_{1})\) פתוחה לא ריקה. 
...

\end{proof}
\begin{definition}[מרחב בייר]
נרמא שמרחב \(X\) הוא מרחב בייר אם מתקיים שכל איחוד בין מנייה של קבוצות דלילות אין לו פנים.

\end{definition}
\begin{proposition}
נניח \(X\) מרחב בייר, ו-\(Y\) מרחב מטרי. נניח \(f_{n}:X\to Y\) היא סדרת פונקציות רציפות על \(X\) כך שלכל \(x \in X\) מתקיים \((f_{n}(x_{0}))_{n=1}^{\infty}\) מתכנסת ל-\(f(x_{0})\). אזי \(f\) רציפה כקבוצה צפופה של נקודות.

\end{proposition}
\begin{proof}
תרגיל - ניתן להוכיח עם משפט הקטגוריה של בייר.

\end{proof}
\section{מסננים}

\begin{definition}[תכונת החיתוך הסופית]
תהי \(X\) קבוצה. אוסף \(\mathcal{A}\) של תתי קבוצות של \(X\) מקיימות את תכונת החיתוך הסופי(FIP) אם לכל אוסף סופי \(\left\{  A_{1},\dots,A_{n}  \right\}\subseteq \mathcal{A}\) החיתוך לא ריק:
$$\bigcap_{k=1}^{n}A_{k}\neq \varnothing $$

\end{definition}
\begin{proposition}
מרחב טופולוגי \(\left( X,\tau \right)\) הוא קומפקטי אם"ם כל אוסף \(\mathcal{A}\) של קבוצות סגורות של \(X\) עם תכונת החיתוך הסופי מתקיים \(\bigcap \mathcal{A}\neq \varnothing\).

\end{proposition}
\begin{definition}[מסנן]
מסנן \(F\) על קבוצה \(S\neq \varnothing\) היא אוסף של תתי קבוצות של \(S\)(\(F\subseteq \mathcal{P}(S)\)) המקיים:

  \begin{enumerate}
    \item לא ריק - \(F \neq \varnothing\). 


    \item לא מכיל את הקבוצה הריקה - \(\varnothing \not\in F\). 


    \item סגור כלפי מעלה - אם \(A \in F\) וגם \(A\subseteq B \subseteq S\) אז \(B\in F\). 


    \item סגור תחת חיתוכים סופיים - אם \(A,B \in F\) אז \(A\cap B \in F\). 


  \end{enumerate}
\end{definition}
\begin{proposition}[מסנן הנוצר מאוסף]
תהי \(S\subseteq \mathcal{P}(S)\) קבוצה כך שלכל \(D_{0} \subseteq D\) סופית מתקיים \(\bigcap D_{0} \neq \varnothing\)(תנאי החיתוך הסופי). אזי קיים מסנן מינמלי שכולל את \(D\), שהוא:
$$\left\langle D\right\rangle=\left\{A\subseteq S\mid\exists B_{1},...,B_{n}\in D:\bigcap_{i=1}^{n}B_{i}\subseteq A\right\}$$
אם \(D\) לא ריקה ו-\(\{ S \}\) אם היא ריקה.

\end{proposition}
\begin{definition}[מסנן הסביבות]
יהי X מ"ט \(x\in X\). מסנן הסביבות של \(x,\) המסומן אצלנו \(F_{x}\) הוא אוסף כל הסביבות של \(x\) כלומר
$$F_{x}=\{N\subseteq X|\exists U\in\tau:x\in U\subseteq N\}$$

\end{definition}
\begin{definition}[התכנסות מסנן]
נאמר שמסנן \(F\) מתכנס ל-\(x\) ונסמן \(F\rightarrow x\) אם כל סביבה של \(x\) שייכת ל-\(F\), כלומר אם
$$F_{x}\subset F$$

\end{definition}
\begin{proposition}[התכנסות מסנן ופנים קבוצה]
יהי X מרחב טופולוגי \(A\subset X\) אז \(x\in A^{\circ}\) אם"ם לכל מסנן \(F\rightarrow x\) מתקיים \(A\in F\). בפרט, A פתוחה אם"ם לכל \(x\in A\) וכל מסנן \(F\rightarrow x\) מתקיים \(A\in F\).

\end{proposition}
\begin{proposition}[התכנסות מסנן וסגור קבוצה]
יהי X מרחב טופולוגי ו- \(A\subset X\). אז \(x\in \overline{A}\) אם"ם קיים מסנן \(F\rightarrow x\) כן ש- \(A\in F\).

\end{proposition}
\begin{definition}[דחיפת מסנן על ידי פונקציה]
יהי F מסנן על X, פונקציה \(f:X\rightarrow Y\). אז
$$f_{*}F=\{A\subset Y|f^{-1}(A)\in F\}$$
הוא מסנן על \(Y\).

\end{definition}
\begin{proposition}[רציפות באמצעות מסננים]
יהיו \(X,Y\) מרחבים טופולוגיים, \(f:X\rightarrow Y\) פונקציה. אז f רציפה אם"ם לכל מסנן מתכנס \(F\rightarrow x\) מתקיים \(f_{*}F\rightarrow f(x)\).

\end{proposition}
\begin{proposition}[מרחב האוסדורף ומסננים]
יהי X מרחב טופולוגי, אז X הוא האוסדורף אם"ם לכל מסנן יש לכל היותר גבול אחד.

\end{proposition}
\begin{definition}[על-מסנן]
מסנן U מעל X יקרא על-מסנן אם הוא מסנן ואין אף מסנן אחר מעל X שמרחיב אותו (כלומר U מקסימלי בהכלה).

\end{definition}
\begin{proposition}[אפיון על-מסננים]
מסנן F מעל X הוא על מסנן אם"ם לכל \(A\subset X\) מתקיים
$$A\in F \quad \text{או} \quad X\backslash A\in F$$

\end{proposition}
\begin{proposition}[על-מסננים ואיחודים סופיים]
אם F על-מסנן מעל X , אז לכל \(A_{0},...,A_{n}\) כך ש- \(X=A_{0}\cup...\cup A_{n}\) קיים i כך ש- \(A_{i}\in F\).

\end{proposition}
\begin{proposition}[הלמה של טרסקי / למת העל מסנן]
לכל מסנן \(F\) מעל \(X\) קיים מסנן \(U\supseteq F\) שהוא מקסימלי ביחס להכלה מבין המסננים שמרחיבים את F. מסנן מקסימלי נקרא על מסנן.

\end{proposition}
\begin{proposition}[קומפקטיות והתכנסות על-מסננים]
מרחב X הוא קומפקטי אם"ם כל על מסנן מתכנס.

\end{proposition}
\begin{proposition}[מרחב האוסדורף וגבול יחיד לעל-מסנן]
מרחב X הוא האוסדורף אם"ם לכל על מסנן יש לכל היותר גבול אחד.

\end{proposition}
\chapter{קבוצות סגורות וצפיפות}

\section{סגור ופנים}

\begin{definition}[סגור של קבוצה]
תהי \((X,\tau)\) מרחב טופולוגי. תהי \(A\subseteq X\) תת קבוצה. \(\overline{A}\) הסגור של A היא הקבוצה הסגורה הקטנה ביותר המכילה את A. כלומר:
$$\overline{A} := \bigcap_{A\subseteq F\subseteq X} F=\bigcap\left\{ C\supseteq A\mid C{\mathrm{~is~closed}} \right\}$$

\end{definition}
\begin{proposition}
יהי \((X,\tau)\) מרחב טופולוגי. תהי \(A\subseteq X\) תת קבוצה. אזי:
$$x\in \overline{A} \iff \forall U \in \tau \quad x \in U \implies U\cap  A\neq \varnothing $$
כלומר איבר נמצא בסגור אם"ם החיתוך של כל קבוצה פתוחה שמוכל בה עם הקבוצה היא ריקה.

\end{proposition}
\begin{proof}
נניח \(x \in X\) וקיימת \(U\) פתוחה כך ש-\(x \in U\) וגם:
$$A\cap  U = \varnothing \implies A\subseteq \underbrace{ X \setminus  U }_{ \text{Closed} } \not\ni x$$
לכן \(\overline{A} \subseteq X \setminus U\) ולכן \(x \not\in \overline{ A}\).
כעת נניח \(x \not\in \overline{A}\). לכן קיימת. אזי ניקח \(U= X \setminus \overline{A}\). זו פתוחה כך ש-\(x \in U\) ו-\(U\cap A = \varnothing\).

\end{proof}
\begin{corollary}
אם \(\mathcal{ B}\) בסיס אז \(x \in \overline{A}\) אם"ם לכל \(B \in \mathcal{B}\) כך ש-\(x \in B\) נקבל \(B\cap A \neq \varnothing\).

\end{corollary}
\begin{definition}[פנים של קבוצה]
הפנים של קבוצה A מסומן על ידי:
$$A^{\circ }=\bigcup_{U\subseteq A \text{ open}}U$$

\end{definition}
\begin{remark}
ניתן לחשוב על זה גם על ידי אוסף כל הנקודות הפנימיות, כאשר נקודה פנימית זה נקודה שמוכלת בקבוצה פתוחה שמוכלת בקטע.

\end{remark}
\begin{proposition}
אם \(\mathcal{B}\) בסיס של המרחב אז \(x \in A^{\circ}\) אם"ם קיים \(B \in \mathcal{B}\) כך ש-\(x \in B\subseteq A\).

\end{proposition}
\begin{proposition}
$$A^{\circ }\subseteq A\subseteq \overline{A} $$

\end{proposition}
\begin{definition}[שפה של קבוצה]
$$\partial A=\overline{A} \setminus  A^{\circ }$$

\end{definition}
\begin{proposition}[תכונות של פנים ושפה]
יהי \((X,\tau)\) מרחב טופולוגי. אזי:

  \begin{enumerate}
    \item לכל \(A,B \subseteq X\) מתקיים: 
$$(A\cap  B)^{\circ }=A^{\circ }\cap  B^{\circ }$$


    \item לכל \(A\subseteq X\) מתקיים: 
$$A^{\circ }\cup  (A^{c})^{\circ }=(\partial A)^{\circ }$$


    \item לכל \(A\subseteq X\) מתקיים \(\partial A \supseteq \overline{\partial A}\). 


  \end{enumerate}
\end{proposition}
\begin{definition}[סביבה של נקודה]
נאמר ש-\(L\subseteq X\) היא סביבה של \(x\) אם קיימת קבוצה פתוחה \(x \in U \subseteq L\).

\end{definition}
\begin{definition}[נקודת הצטברות/נקודת גבול]
יהי \(A \subseteq B\) תת קבוצה. נאמר ש-\(x \in X\) היא נקודת הצטברות של \(A\) אם כל סביבה של \(x\) מכילה נקודה מ-\(A\) שונה מ-\(x\). נסמן ב-\(A'\) את קבוצת נקודות ההצטברות של \(A\).

\end{definition}
\begin{remark}
ניתן לקחת במקום סביבה כללית, סביבה פתוחה, זה שקול.

\end{remark}
\begin{proposition}
הסגור של קבוצה תהיה האיחוד של הקבוצה עם נקודות ההצטברות. כלומר:
$$\overline{A} =A\cup  A'$$

\end{proposition}
\begin{proof}
זו הוכחה ישירה של לפתוח הגדרות.
ראשית נניח \(x \in A\cup A'\). נראה ש-\(x \in \overline{A}\) שקול ללהראות שלכל קבוצה \(x \in U\) פתוחה \(U\cap A\neq \varnothing\).
אם \(x \in A\) אזי \(U\cap A = \varnothing\) מתקיים.
אם \(x \in A'\) אזי כל \(U\) פתוחה המכילה את \(x\) מקיימת:
$$U\cap  A \supseteq U\cap  (A\setminus  \{ x \})$$
וקיבלנו \(A\cap A'\subseteq \overline{A}\).
כעת נראה את הכיוון השני. אם \(x \in \overline{A}\) אזי לכל \(U\) פתוחה כך ש-\(x \in U\), \(U\cap A \neq \varnothing\).
אם \(x \in A\) אזי \(x \in A\cup A'\).
אם \(x \not\in A\) נקבל \(U\cap A \neq \varnothing\) גורר \(\varnothing \neq U \cap (A \setminus \{  x \})\) ולכן \(x \in A'\). כלומר \(\overline{A}\subseteq A\cup A'\) ונקבל סה"כ \(\overline{A}=A\cup A'\)

\end{proof}
\begin{corollary}
קבוצה היא סגורה אם"ם מכילה את כל הנקודות הצטברות שלה.

\end{corollary}
\begin{proposition}
יהי \(Y\) תת מרחב טופולוגיה של \(X\). וכן יהי \(A\subseteq Y\) תת קבוצה. אזי אם \(\overline{A}\) הוא הסגור של A ב-\(X\) הסגור של \(A\) ב-\(Y\) יהיה:
$$\overline{A} \cap  Y$$

\end{proposition}
\section{דלילות ומשפט הקטגוריה}

\begin{definition}[קבוצה דלילה - nowhere dense]
יהי X מרחב טופולוגי. קבוצה \(A\subset X\) נקראת דלילה אם \(X\setminus\overline{A}\) צפופה ב-\(X\).
במילים אחרות, לכל קבוצה פתוחה ולא ריקה \(U\subset X\) מתקיים כי \(\left( X\setminus\overline{A} \right)\cap U\ne\emptyset\). כלומר לכל \(A\subseteq X\) פתוחה לא ריקה מתקיים \(\overline{A}\cap U\neq U\).

\end{definition}
\begin{remark}
האינטואיציה זה שזו קבוצה שפרושה באופן "דליל" במרחב, כך שאין לו תת קבוצה פתוחה שצפופה במרחב.

\end{remark}
\begin{remark}
עבור מרחב מידה, להיות דליל זה חזק יותר מלהיות ממידה אפס. אפשר להיות ממידה אפס אבל עדיין צפוף.

\end{remark}
\begin{remark}
נשים לב שאם קבוצה \(A\subset X\) לא צפופה באף קבוצה פתוחה \(U\subset X\) (בטופולוגיה המושרית), אז היא דלילה.

\end{remark}
\begin{example}[קבוצות דלילות]
הקבוצה \(\mathbb{Z}\subseteq \mathbb{R}\) דלילה, ובפרט לא מכילה נקודות הצטברות. כמו כן הקבוצה \(\left\{  \frac{1}{n}  \right\}_{n \in \mathbb{N}}\) דלילה למרות שמכיל נקודת הצטברות ב-\(0\) כי אינו מכיל קטע פתוח(בפרט הפנים יהיה ריק ולכן דלילה).

\end{example}
\begin{proposition}
קבוצה \(A\subset X\) דלילה אם ורק אם \(\text{int}\left( \overline{A} \right)=\varnothing\).

\end{proposition}
\begin{corollary}
אם \(A\) דלילה אז \(\overline{A}\) דלילה.

\end{corollary}
\begin{definition}[קבוצה מקטגוריה ראשונה]
קבוצה \(A\subset X\) היא מקטגוריה ראשונה, אם היא איחוד בן-מניה לכל היותר של קבוצות דלילות. כלומר אם קיים \(\{ A_{n} \}_{}\) דלילות כך ש:
$$A=\bigcup_{n=1}^{\infty}A_{n}$$

\end{definition}
\begin{example}[הרציונאלים]
הקבוצה \(\mathbb{Q}\) אינה דלילה ב-\(\mathbb{R}\), כי \(\overline{\mathbb{Q}}=\mathbb{R}\) ולכן \(\mathbb{R}\setminus\overline{\mathbb{Q}}=\varnothing\). אולם היא בכל זאת מקטגוריה ראשונה, כי \(\mathbb{Q}=\bigcup_{q\in\mathbb{Q}}\{q\}\) וזה איחוד בן-מניה של קבוצות דלילות.

\end{example}
\begin{definition}[קבוצה מקטגוריה שנייה]
קבוצה שאיננה מקטגוריה ראשונה נקראת מקטגוריה שנייה.

\end{definition}
\begin{theorem}[משפט הקטגוריה של בייר]
יהי X מרחב טופולוגי קומפקטי האוסדורף או מרחב מטרי שלם.
אזי לכל אוסף בן-מניה של קבוצות דלילות \(\{A_{n}\}_{n\in\mathbb{N}}\) מתקיים כי \(X\setminus\bigcup_{n\in\mathbb{N}}A_{n}\) צפופה ב-\(X\).

\end{theorem}
\begin{corollary}
כל מרחב טופולוגי קומפקטי האוסדורף וכל מרחב מטרי שלם הם מקטגוריה שנייה.

\end{corollary}
\begin{proof}
כי אם הם היו מקטגוריה ראשונה אז הם היו איחוד בן-מניה של דלילות \(\{A_{n}\}_{n\in\mathbb{N}}\) כלשהן, והיינו מקבלים \(X\setminus\bigcup_{n\in\mathbb{N}}A_{n}=\varnothing\).

\end{proof}
\begin{proof}
(למשפט הקטגוריה של בייר - נוסח למרחבים טופולוגיים)\\

יהי \(X\) מרחב טופולוגי קומפקטי האוסדורף, ויהי \(\{A_{n}\}_{n\in\mathbb{N}}\) אוסף כלשהו של קבוצות דלילות במרחב. נניח ללא הגבלת הכלליות שכולן סגורות (כי נוכל לקחת את אוסף הסגורים שלהן).
כדי להראות ש-\(X-\bigcup_{n\in\mathbb{N}}A_{n}\) צפופה ב-X די להראות שלכל קבוצה פתוחה ולא ריקה \(U\subset X\) מתקיים \((X-\bigcup_{n\in\mathbb{N}}A_{n})\cap U\ne\emptyset\).
נשתמש בכך ש-X קומפקטי האוסדורף, ולכן כפי שהראינו לעיל נובע שהוא נורמלי.
נתון כי \(A_{1}\) דלילה וסגורה, ולכן מההגדרה נובע כי \(U_{1}=:U\cap(X-A_{1})\ne\emptyset\) וזו קבוצה פתוחה. תהי \(a_{1}\in U_{1}\).
X נורמלי ולכן קיימת \(V_{1}\) פתוחה המקיימת \(a_{1}\in V_{1}\subset\overline{V_{1}}\subset U_{1}\subset U\).
נשים לב כי \(U_{1}\cap A_{1}=\emptyset\) וכי \(\overline{V_{1}}\subset U_{1}\) ולכן \(\overline{V_{1}}\cap A_{1}=\emptyset\).\\

נתון כי \(A_{2}\) דלילה וסגורה, ולכן מההגדרה נובע כי \(U_{2}=:(X-A_{2})\cap V_{1}\ne\emptyset\) וזו קבוצה פתוחה.
תהי \(a_{2}\in U_{2}\).\\

X נורמלי ולכן קיימת \(V_{2}\) פתוחה המקיימת \(a_{2}\in V_{2}\subset \overline{V_{2}}\subset U_{2}\subset V_{1}\subset U\).
ובאופן כללי, לאחר n צעדים נקבל אוסף פתוחות \(\{V_{j}\}_{j=1}^{n}\) המקיימות \(\overline{V_{j}}\subset V_{j-1}\).\\

נשים לב כי \(U_{2}\cap(A_{1}\cup A_{2})=\emptyset\) וכי \(\overline{V_{2}}\subset U_{2}\) ולכן \(\overline{V_{2}}\cap(A_{1}\cup A_{2})=\emptyset\).
וכן \(\overline{V_{n}}\cap(\bigcup_{j=1}^{n}A_{j})=\emptyset\).
לכן \(\{\overline{V_{n}}\}_{n\in\mathbb{N}}\) הוא אוסף של סגורות המקיים את תכונת החיתוך הסופי, ומקומפקטיות X נובע שהחיתוך כולו אינו ריק.
תהי \(a\in\bigcap_{n\in\mathbb{N}}\overline{V_{n}}\subset U\).\\

נשים לב כי \(a\notin\bigcup_{n\in\mathbb{N}}A_{n}\) ולכן \(a\in(X-\bigcup_{n\in\mathbb{N}}A_{n})\cap U\), כנדרש.

\end{proof}
\begin{remark}
בשלב זה בהוכחת הנוסח למרחבים מטריים משתמשים בכך שכל מרחב מטרי הוא נורמלי.  

\end{remark}
\begin{remark}
בשלב זה בהוכחת הנוסח למרחבים מטריים משתמשים בהנחת השלמות.  

\end{remark}
\begin{definition}[מרחב טופולוגי מושלם]
מרחב טופולוגי נקרא מושלם, אם כל נקודה בו היא נקודת הצטברות.\\

כלומר מרחב טופולוגי X הוא מושלם אם לכל \(x\in X\) ולכל \(U\subset X\) פתוחה המכילה את x, קיים \(y\in U\) כך ש-\(y\ne x\).  

\end{definition}
\begin{corollary}
אם X מרחב טופולוגי קומפקטי האוסדורף וגם מושלם, אז הוא לא בן-מניה.

\end{corollary}
\begin{proof}
(למסקנה הקודמת)
X מושלם והאוסדורף ולכן כל יחידון \(\{x\}\) הוא קבוצה דלילה, כי מהיות המרחב האוסדורף \(\overline{\{x\}}^{o}=\{x\}^{o}=\emptyset\).\\

נניח בשלילה כי X קומפקטי ובן מניה, אזי מתקיים \(X=\bigcup_{x\in X}\{x\}\) כאיחוד בן־מניה של קבוצות דלילות, בסתירה למסקנה ממשפט בייר.  

\end{proof}
\begin{summary}
  \begin{itemize}
    \item דלילה אומר שהפנים של הסגור הוא ריק. בפרט אם \(A\) דלילה אין קבוצה פתוחה אשר מוכלת לחלוטין ב-\(\overline{A}\).
    \item קבוצה היא מקטגוריה ראשונה אם היא איחוד בן מנייה של קבוצות דלילות, אחרת נקראת מקטגוריה שנייה.
    \item משפט הקטגוריה של בייר אומר שאם מרחב מטרי הוא שלם או קומפקטי האוסדורף אז הוא לא יכול להיות מקטגוריה ראשונה.
    \item 
  \end{itemize}
\end{summary}
\chapter{החבורה היסודית}

\section{הומוטופיות}

\begin{definition}[הומוטופיה]
אם \(f,f':X\to Y\) הם העתקות רציפות אז \(f\) הומוטפי ל-\(f'\) אם קיימת העתקה רציפה \(F:X\times[0,1]\to Y\) כך שמתקיים:
$$F(x,0)=f(x)\qquad F(x,1)=f'(x)$$

\end{definition}
\begin{definition}[מסילה]
העתקה מהצורה \(f:[0,1]\to X\) נקראת מסילה.

\end{definition}
\begin{definition}[מסילות הומוטופיות]
זוג מסילות \(f,f':[0,1]\to X\) נקראות הומוטופיות אם \(f(0)=f'(0)=x_{0},f(1)=f'(1)=x_{1}\) וגם קיימת פונקציה רציפה \(F:[0,1]\times[0,1]\to X\) כך ש:
$$F(s,0)=f(s)\quad F(s,1)=f'(s)\qquad F(0,t)=x_{0}\quad F(1,t)=x_{1}$$

\end{definition}
\begin{proposition}
להיות הומוטפיות זה יחס שקילות, כאשר נסמן את המחלקת שקילות של \(f\) ב-\([f]\).

\end{proposition}
\begin{proposition}
כל מסילות המחברות בין שתי נקודות בקבוצה קמורה הם הומוטופיות.

\end{proposition}
\begin{proof}
הפונקציה המתאימה תהיה:
$$F(x,t)=(1-t)f(x)+t g(x)$$

\end{proof}
\begin{definition}[שרשרור]
תהי \(f\) מסילה מ-\(x_{0}\) ל-\(x_{1}\) ו-\(g\) מסילה ב-\(X\) מ-\(x_{1}\) ל-\(x_{2}\). נגדיר את השרשור \(h=f*g\) על ידי:
$$h(s)=\begin{cases}f(2s) & s \in \left[ 0,\frac{1}{2} \right] \\g(2s-1) & s \in \left[ \frac{1}{2},1 \right]
\end{cases}$$
וזה מסילה ב-\(X\) מ-\(x_{0}\) ל-\(x_{2}\).

\end{definition}
\begin{remark}
שרשור נותן מסילה רציפה מלמת ההדבקה.

\end{remark}
\begin{proposition}
שרשור מוגדר היטב על מחלקות שקילות של הומוטופיות, כלומר הפועולה:
$$[f]\ast[g]=[f\ast g].$$
מוגדרת היטב.

\end{proposition}
\begin{proof}
תהי \(F\) פונקציה רציפה בין שתי מסילות \(f,f'\) ותהי \(G\) פונקציה רציפה בין שתי מסילות \(g,g'\). נגדיר:
$$H(s,t)=\begin{cases}F(2s,t) & s \in \left[ 0,\frac{1}{2} \right] \\G(2s-1,t) & s \in \left[ \frac{1}{2},1 \right]
\end{cases}$$
כיוון ש-\(F(1,t)=x_{1}=G(0,t)\) ההעתקה \(H\) מוגדרת היטב ורציפה מלמת ההדבקה, ולכן \(f*g\) ו-\(f'*g'\) הומוטופיים.

\end{proof}
\begin{proposition}
לפועלה \(*\) על מחלקות שקילות יש את התכונות של חבורה. כלומר:

  \begin{enumerate}
    \item אסוצייטיביות: כאשר \([f]*([g]*[h])\) מוגדר גם \(([f]*[g])*[h]\) מוגדר ומתקיים: 
$$[f]*([g]*[h])=([f]*[g])*[h]$$


    \item קיום הופכיים שמאליים וימניים. 


    \item קיום הופכי. 


  \end{enumerate}
\end{proposition}
\section{החבורה היסודית}

\begin{definition}[לולאה סגורה]
יהי \(X\) מרחב טופולוגי, ו-\(x_{0} \in X\). לולאה סגורה זו מסילה \(\gamma:[0,1]\to X\) אשר מקיימת \(\gamma(0)=\gamma(1)\).

\end{definition}
\begin{definition}[החבורה היסודית]
הקבוצה של מחלקות הומוטופיה של הלולאות הסגורות מנקודה \(x_{0}\) ביחד עם פעולת השרשור. מסומן \(\pi_{1}(X,x_{0})\).

\end{definition}
\begin{proposition}
תהי \(\alpha\) מסילה מ-\(x_{0}\) ל-\(x_{1}\). ההעתקה:
$$\hat{\alpha}:\pi_{1}(X,x_{0})\longrightarrow\pi_{1}(X,x_{1})\qquad \hat{\alpha}([f])=[\bar{\alpha}]*[f]*[\alpha]$$
היא איזומורפיזם של חבורות, כאשר \(\overline{\alpha}\) זו המסילה ההופכית של \(\alpha\).

\end{proposition}
\begin{proof}
ראשית הומומורפיזם כי:
\begin{gather*}\hat{\alpha}([f])*\hat{\alpha}([g])=\left( \left[ \bar{\alpha} \right]*[f]*\left[ \alpha \right] \right)*\left( \left[ \bar{\alpha} \right]*[g]*\left[ \alpha \right] \right)  \\=[\bar{\alpha}]\ast[f]\ast[g]\ast[\alpha]={\hat{\alpha}}([f]\ast[g])
\end{gather*}
וכעת איזומורפיזם כי:

\end{proof}
\begin{corollary}
אם \(x_{0}\) ו-\(x_{1}\) נמצאים באותו תחום קשירות מסילתית אז  \(\pi_{1}(X,x_{0})\cong \pi_{1}(X,x_{1})\).

\end{corollary}
\begin{definition}[פשוט קשר]
מרחב \(X\) נקרא פשוט קשר אם קשיר מסילתית ואם \(\pi_{1}(X,x_{0})\cong\{ e \}\) לאיזשהו \(x_{0} \in X\)(ולכן גם לכל \(x \in X\)).

\end{definition}
\begin{proposition}
במרחב פשוט קשר כל שתי מסילות עם אותם נקודות התחלה וסיום הם הומוטופיות.

\end{proposition}
\begin{example}
כל מרחב קמור של \(\mathbb{R}^{n}\) יהיה עם חבורה יסודית טריוויאלית.

\end{example}
\begin{proposition}
עבור \(n\geq 2\) נקבל \(\pi_{1}(S^{n})\cong \{ e \}\) כלומר החבורה הטריוויאלית היא טריוויאלית. 

\end{proposition}
\begin{proof}
תהי \(\gamma:[0,1]\to S^{n}\) לולאה על \(S^{n}\). ותהי \(x \in S^{n}\) נקודה לא על הלולאה. נזכור את ההומאומורפיזם:
$$S^{n}\setminus \{ x \}\cong \mathbb{R}^{n}$$
על ידי למשל הטלה סטראוגרפית. אנו יודעים כי כיוון ש-\(\mathbb{R}^{n}\) קבוצה קמורה החבורה היסודית שלה טריוויאלית, ולכן \(\left[ \gamma \right]\) הוא טרייויאלי ב-\(\pi_{1}\left( \mathbb{R}^{2} \right)\), ולכן טריוויאלי ב-\(\pi_{1}\left( S^{n}\setminus \{ x \} \right)\) כי המרחבים הומאומורפיים. וכיוון שניתן לכווץ כל לולאה ב-\(S^{n}\setminus \{ x \}\) ניתן לכווץ אותה ב-\(S^{n}\) והחבורה היסודית טריוויאלית.

\end{proof}
\begin{proposition}
עבור \(n\geq 3\) נקבל כי \(\mathbb{R}^{2}\) לא הומאומורפי ל-\(\mathbb{R}^{n}\)

\end{proposition}
\begin{proof}
נניח בשלילה כי \(\mathbb{R}^{2}\cong \mathbb{R}^{n}\) עבור \(n\geq 3\). תהי \(x\in \mathbb{R}^{2}\). נסתכל על \(\mathbb{R}^{2}\setminus \{x \}\). מתקיים:
$$\mathbb{R}^{2}\setminus \{ x \}\cong S^{1}\times \mathbb{R}$$

\end{proof}
\begin{definition}[לולאה סגורה]
יהי \(X\) מרחב טופולוגי, ו-\(x_{0} \in X\). לולאה סגורה זו מסילה \(\gamma:[0,1]\to X\) אשר מקיימת \(\gamma(0)=\gamma(1)\).

\end{definition}
\begin{definition}[החבורה היסודית]
הקבוצה של מחלקות הומוטופיה של הלולאות הסגורות מנקודה \(x_{0}\) ביחד עם פעולת השרשור. מסומן \(\pi_{1}(X,x_{0})\).

\end{definition}
\begin{proposition}
תהי \(\alpha\) מסילה מ-\(x_{0}\) ל-\(x_{1}\). ההעתקה:
$$\hat{\alpha}:\pi_{1}(X,x_{0})\longrightarrow\pi_{1}(X,x_{1})\qquad \hat{\alpha}([f])=[\bar{\alpha}]*[f]*[\alpha]$$
היא איזומורפיזם של חבורות, כאשר \(\overline{\alpha}\) זו המסילה ההופכית של \(\alpha\).

\end{proposition}
\begin{proof}
ראשית הומומורפיזם כי:
\begin{gather*}\hat{\alpha}([f])*\hat{\alpha}([g])=\left( \left[ \bar{\alpha} \right]*[f]*\left[ \alpha \right] \right)*\left( \left[ \bar{\alpha} \right]*[g]*\left[ \alpha \right] \right)  \\=[\bar{\alpha}]\ast[f]\ast[g]\ast[\alpha]={\hat{\alpha}}([f]\ast[g])
\end{gather*}
וכעת איזומורפיזם כי:

\end{proof}
\begin{corollary}
אם \(x_{0}\) ו-\(x_{1}\) נמצאים באותו תחום קשירות מסילתית אז  \(\pi_{1}(X,x_{0})\cong \pi_{1}(X,x_{1})\).

\end{corollary}
\begin{definition}[פשוט קשר]
מרחב \(X\) נקרא פשוט קשר אם קשיר מסילתית ואם \(\pi_{1}(X,x_{0})\cong\{ e \}\) לאיזשהו \(x_{0} \in X\)(ולכן גם לכל \(x \in X\)).

\end{definition}
\begin{proposition}
במרחב פשוט קשר כל שתי מסילות עם אותם נקודות התחלה וסיום הם הומוטופיות.

\end{proposition}
\begin{example}
כל מרחב קמור של \(\mathbb{R}^{n}\) יהיה עם חבורה יסודית טריוויאלית.

\end{example}
\begin{example}
עבור \(n\geq 2\) נקבל \(\pi_{1}(S^{n})\cong \{ e \}\) כלומר החבורה הטריוויאלית היא טריוויאלית. נוכיח זאת:
תהי \(\gamma:[0,1]\to S^{n}\) לולאה על \(S^{n}\). ותהי \(x \in S^{n}\) נקודה לא על הלולאה. נזכור את ההומאומורפיזם:
$$S^{n}\setminus \{ x \}\cong \mathbb{R}^{n}$$
על ידי למשל הטלה סטראוגרפית. אנו יודעים כי כיוון ש-\(\mathbb{R}^{n}\) קבוצה קמורה החבורה היסודית שלה טריוויאלית, ולכן \(\left[ \gamma \right]\) הוא טרייויאלי ב-\(\pi_{1}\left( \mathbb{R}^{2} \right)\), ולכן טריוויאלי ב-\(\pi_{1}\left( S^{n}\setminus \{ x \} \right)\) כי המרחבים הומאומורפיים. וכיוון שניתן לכווץ כל לולאה ב-\(S^{n}\setminus \{ x \}\) ניתן לכווץ אותה ב-\(S^{n}\) והחבורה היסודית טריוויאלית.

\end{example}
\begin{proposition}
החבורה היסודית של מכפלה של מרחבים תהיה איזומורפית מכפלה של החבורות היסודיות. כלומר אם \(X,Y\) מרחבים טופולוגיים אזי:
$$\pi_{1}\left( X\times Y \right)\cong \pi_{1}(X)\times \pi_{1}(Y)$$

\end{proposition}
\begin{example}
נזכור כי טורוס איזומורפי ל-\(S^{1}\times S^{1}\) וכן:
$$\pi_{1}\left( S^{1}\times S^{1}\right)\cong \pi_{1}(S^{1})\times \pi_{1}(S^{1})\cong  \mathbb{Z}\times \mathbb{Z}$$

\end{example}
\begin{example}
עבור \(n\geq 3\) נקבל כי \(\mathbb{R}^{2}\) לא הומאומורפי ל-\(\mathbb{R}^{n}\). נוכיח זאת:
נניח בשלילה כי \(\mathbb{R}^{2}\cong \mathbb{R}^{n}\) עבור \(n\geq 3\). תהי \(x\in \mathbb{R}^{2}\). באופן כללי עבור \(m \geq 1\) עבור \(\mathbb{R}^{m}\setminus \{x \}\) נקבל:
$$\mathbb{R}^{m}\setminus \{ x \}\cong S^{m-1}\times \mathbb{R}$$
כי קורדינטות קוטביות. ולכן נקבל:
$$S^{1}\times \mathbb{R}\cong S^{n-1}\times \mathbb{R}\implies \pi_{1}\left( S^{1}\times \mathbb{R} \right)\cong \pi_{1}\left( S^{n-1}\times \mathbb{R} \right)\implies \pi_{1}(S^{1})\cong \pi_{1}(S^{n-1})$$
כאשר נזכור כי \(\pi_{1}(S^{1})\cong\mathbb{Z}\) וכן עבור \(n\geq 3\) נקבל \(n-1\geq 2\) ולכן \(\pi_{1}(S^{n-1})\cong\{ e \}\) ונקבל כי \(\{ e \}\cong \mathbb{Z}\) בסתירה ולכן לא הומאומורפיים.

\end{example}
\begin{proof}
נניח בשלילה כי \(\mathbb{R}^{2}\cong \mathbb{R}^{n}\) עבור \(n\geq 3\). תהי \(x\in \mathbb{R}^{2}\). נסתכל על \(\mathbb{R}^{2}\setminus \{x \}\). מתקיים:
$$\mathbb{R}^{2}\setminus \{ x \}\cong S^{1}\times \mathbb{R}$$

\end{proof}
\section{הרמות}

\section{מרחבי כיסוי}

\begin{definition}[כיסוי אחיד]
יהי \(p:E\to B\) העתקה על רציפה. העתקה פתוחה \(U\subseteq B\) נקראת מכוסה באחידות על ידי \(p\) אם המקור \(p ^{-1}(U)\) יכולה להכתב כאיחוד של קבוצות זרות \(\left\{  V_{\alpha}  \right\}\) של \(E\). כך שלכל \(\alpha\) הצמצום של \(p|_{V_{\alpha}}\) הוא הומאומורפיזם.

\end{definition}
\begin{definition}[מרחב כיסוי]
תהי \(p:E\to B\) רציף ועל. אם כל נקודה \(b \in B\) מכילה סביבה \(U\) שמכוסה באיחידות על ידי \(p\) אז \(p\) נקראת העתקת כיסוי, ו-\(E\) נקרא מרחב כיסוי של \(B\).

\end{definition}
\begin{remark}
אם \(p:E\to B\) מרחב כיסוי אז לכל \(b \in B\) התת מרחב \(p ^{-1}(b)\) יש את הטופולוגיה הדיסקרטי. 

\end{remark}
\begin{definition}[מרחב כיסוי]
נגיד כי \(\tilde{X}\) הומא מרחב כיסוי של \(X\) אם יש העתקה \(p:\tilde{X}\to X\) המכונה העתקת כיסוי כך שמתקיים התנאי הבא:
לכל \(x_{0} \in X\) קיימת סביבה פתוחה \(x_{0} \in U\subseteq X\) שעבורה \(p ^{-1}(U)=\sqcup_{\alpha}\tilde{U}_{\alpha}\) עבור \(\tilde{U}_{\alpha}\) פתוחות וזרות, כך שלכל \(\alpha\) העתקה מצומצמת \(p|_{\tilde{U}_{\alpha}}:\tilde{U}_{\alpha}\to U\)  היא הומומואפיזם.

\end{definition}
\chapter{נספחים}

\section{טופולוגיות נוספות}

\section{טופולוגיית התת מרחב}

\begin{definition}[טופולוגיית התת מרחב]
יהי \(\left( X,\tau \right)\) מרחב טופולוגי ויהי \(Y\subseteq X\). הטופולוגיה המושרת(תת הטופולוגיה) על \(Y\) היא:
$$\tau|_Y = \tau_Y = \{ U \cap Y \mid U \in \tau \}$$
כלומר, קבוצה פתוחה ב-\(Y\) היא חיתוך של קבוצה פתוחה ב-\(X\) עם \(Y\).

\end{definition}
\begin{proposition}
אם \(\mathcal{B}\) הוא בסיס של \(\tau\) אזי זוהי הטופולוגיה הנוצרת על ידי הבסיס:
$$\mathcal{B}|_{Y}=\{ B\cap Y\mid B \in \mathcal{B} \}$$

\end{proposition}
\begin{proposition}
אם \(\tau\) מושרה ממטריקה \(d\) אזי \(\tau|_{Y}\) מושרה ממטריקה \(d|_{Y}\).

\end{proposition}
\section{טופולוגיית סורג'נפרי}

\begin{definition}[טופולוגיית סורג'נפרי - Sorgenfrey]
טופולוגיית סורג'נפרי על \(\mathbb{R}\) היא הטופולוגיה שמתקבלת מהבסיס של כל הקבוצות מהצורה \([a, b)\) כאשר \(a < b\) ו-\(a, b \in \mathbb{R}\).

\end{definition}
\begin{proposition}
טופולוגיית סורג'נפרי עדינה מהטופולוגיה הרגילה של \(\mathbb{R}\) (כל קבוצה פתוחה רגילה פתוחה גם בסורג'נפרי, אך לא להיפך).

\end{proposition}
\begin{proposition}
אינה מטריזבילית.

\end{proposition}
\section{המרחב הפרוייקטיבי}
\end{document}