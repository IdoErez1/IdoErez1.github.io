\documentclass{tstextbook}

\usepackage{amsmath}
\usepackage{amssymb}
\usepackage{graphicx}
\usepackage{hyperref}
\usepackage{xcolor}

\begin{document}

\title{Example Document}
\author{HTML2LaTeX Converter}
\maketitle

\section{מבוא לחבורות}

\subsection{הגדרה ותכונות}

\subsection{הגדרת החבורה}

\begin{definition}[חבורה]
חבורה היא קבוצה \(G\) עם פעולה בינארית \(\circ:G\times G\to G\)  המקיימת:

  \begin{enumerate}
    \item אסוציאטיביות: \(\forall x,y,z\in G\quad(x\circ y)\circ z=x\circ(y\circ z)\)


    \item קיים איבר ניטרלי/יחידה \(e\in G\)


    \item קיום הופכי - לכל \(x \in G\) קיים \(y\in G\) כך ש- \(x\circ y=y\circ x=e\)


  \end{enumerate}
\end{definition}
\begin{lemma}
איבר ניטרלי הוא יחיד

\end{lemma}
\begin{proof}
אם \(e_{1},e_{2}\in G\) שניהם ניטרלים אזי
$$e_{1}=e_{1}\circ e_{2}=e_{2}$$

\end{proof}
\begin{remark}
לא דרשנו קומוטטיביות

\end{remark}
\begin{example}
הצירופים הבאים של קבוצות ופעולה יהיו חבורה:
- השלמים עם חיבור - \((\mathbb{Z},+)\)
- רציונלים ללא 0 עם פעולת הכפל
- רציונאלים חיוביים עם פעולת הכפל
- חיבור מודולרי
- קבוצה טבעיים ללא 0 עם כפל מודולו \(n\) כאשר \(n\) ראשוני
- הטבעיים עם חיסור - \((\mathbb{Z},-)\). יש ניטרלי מצד אחד(0) והופכי(\(x\) הופכי ל-\(x\))

\end{example}
\begin{example}
הצירופים הבאים של קבוצות ופעולה הם לא חבורה:
- שלמים בלי אפס עם כפל - \((\mathbb{Z}\setminus \{ 0 \},\times)\) - לא חבורה כיוון שאין איבר הופכי לכל איבר
- קבוצה טבעיים ללא 0 עם כפל מודולו \(n\) כאשר \(n\) לא ראשוני

\end{example}
\begin{remark}
בכל שדה יש עבור כל פעולה חבורות. ולכן מכל שדה ניתן ליצור 2 חבורות. כנל עם מרחב וקטורי. זאת כיוון שהאקסיומות של המרחב וקטורי והשדה מכילות בתוכם את האקסיומות של החבורה.

\end{remark}
\textbf{דוגמה}
חבורת הסימטריות - \((Sym_{x},\circ)\) כאשר \(\circ\) היא פעולת ההרכבה ו- \(Sym_{x}\) זה הקבוצת הפונקציות החח"ע ועל. פונקציות הזהות היא איבר היחידה, כיוון שהפונקציות הם חח"ע ועל אז הם הפיכות ולכל איבר קיים הופכי.

\begin{example}
סוגרי לי(Lie): \((M_{n}(\mathbb{R}),[\cdot,\cdot])\) - לא אסוצייטבית:

$$[A,B]=A\cdot B-B\cdot A$$

\end{example}
\begin{example}
מטריצות הפיכות \(n\times n\) עם כפל מטריצות. מסומן ב-\(GL_{n}(\mathbb{R})\)

\end{example}
\begin{remark}
בדרכ נקרא לפעולה כפל, ונסמן אותה ב-\(\times\) ונשמיט אותה. 

\end{remark}
\subsection{תכונות של חבורות}

במשפטים הבאים, \((G,\cdot)\) חבורה, ו-\(e\) איבר ניטרלי.

\begin{proposition}
ההופכי הוא יחיד.

\end{proposition}
\begin{proof}
אם \(x \in G\) ו-\(y,z\) הופכיים של \(x\) אזי:
$$xy=yx=xz=zx=e$$
לכן:
$$y=ye=y(xz)=(yx)z=ez=z$$

\end{proof}
\begin{proposition}[תכונות של חבורות]
  \begin{enumerate}
    \item מתקיים \((x^{-1})^{-1}=x\)


    \item הופכי של כפל מקיים \((xy)^{-1}=y^{-1}x ^{-1}\)


    \item אם \(xy=e\) אז \(y=x ^{-1}\)


    \item צמצום:\\
$$xy=xz\implies y=z \impliedby yx=zx$$


    \item העברת אגף:\\
$$xy=z\iff y=x ^{-1}z$$


    \item פתרון משוואה: לכל \(a,b\in G\), למשוואה \(ax=b\) יש פתרון יחיד ב-\(G\). 


    \item אם \(a\in G\), אזי \(f:G\to G\) המוגדרת \(f(g)=ag\) היא תמורה על \(G\). 


  \end{enumerate}
\end{proposition}
\begin{theorem}[פרמה]
אם \(p\) ראשוני ו-\(0<a <p\) אזי \(a^{p-1}\equiv 1\mod p\)

\end{theorem}
נוכיח טענה כללית יותר:

\begin{theorem}
אם \((G,\cdot)\) חבורה והפעולה קומוטטיבית אזי לכל \(a\in G\) מתקיים \(a^{|G|}=e\) (כאשר פרמה זה במקרה \(G=(\mathbb{F}_{p}\setminus \{ 0 \},\otimes)\))

\end{theorem}
\begin{proof}
נמספר את איברי \(G\): \(g_{1},g_{2},\dots,g_{|G|}\). 
ראינו שכפל ב-\(a\) זה תמורה, כלומר:
$$\{ ag_{1},ag_{2},\dots,ag_{|G|} \}=\{ g_{1},\dots,g_{|G|} \}$$
ולכן:
$$a^{|G|}\prod_{i=1}^{|G|}(g_{i})=\prod_{i=1}^{|G|}(ag_{i})=\prod_{i=1}^{|G|}(g_{i})$$
וניתן לצמצם ולקבל:
$$a^{|G|}=e$$

\end{proof}
\begin{proposition}[אלגוריתם של פרמה לבדיקת ראשוניות]
אלגוריתם הסתבוריתי להכרעה אם מספר הוא ראשוני הפועל בצורה הבאה:

  \begin{enumerate}
    \item בהנתן \(n\in \mathbb{N}\), מגרילים \(1<a<n\). ובודקים האם: 
$$a^{n-1} \equiv 1\mod n$$


    \item אם כן - לא ידוע אם ראשוני. נצפה שראשוני, וננסה לחזור עם עוד \(a\). 


    \item אם לא - בוודאות לא ראשוני. 


    \item אם חוזרים על זה הרבה פעמים, הסבירות שהמספר ראשוני יעלה.  


  \end{enumerate}
\end{proposition}
\subsection{לוח כפל}

עבור חבורה ניתן לבנות לוח כפל עם האיברים של החבורה. נזכיר כי בגלל חוק הביטול אם \(a\cdot b=a\cdot c\) אז \(b=c\). לכן בכל שורה ועמודה כל ערך יכול להופיע רק פעם אחת. נבנה לוח כפל עבור החבורה \(D_{3}\):

\begin{table}[htbp]
  \centering
  \begin{tabular}{|ccccccc|}
    \hline
    \(\circ\) & \textbf{Id} & \(\mathbf{R_{120}}\) & \(\mathbf{R_{240}}\) & \(\mathbf{F_{0}}\) & \(\mathbf{F_{120}}\) & \(\mathbf{F_{240}}\) \\ \hline
    \textbf{Id} & Id & \(R_{120}\) & \(R_{240}\) & \(F_{0}\) & \(F_{120}\) & \(F_{240}\) \\ \hline
    \(\mathbf{R_{120}}\) & \(R_{120}\) & \(R_{240}\) & Id & \(F_{240}\) & \(F_{120}\) & \(F_{0}\) \\ \hline
    \(\mathbf{R_{240}}\) & \(R_{240}\) & Id & \(R_{120}\) & \(F_{120}\) & \(F_{240}\) & \(F_{0}\) \\ \hline
    \(\mathbf{F_{0}}\) & \(F_{0}\) & \(F_{120}\) & \(F_{240}\) & Id & \(R_{120}\) & \(R_{240}\) \\ \hline
    \(\mathbf{F_{120}}\) & \(F_{120}\) & \(F_{240}\) & \(F_{0}\) & \(R_{240}\) & Id & \(R_{120}\) \\ \hline
    \(\mathbf{F_{240}}\) & \(F_{240}\) & \(F_{0}\) & \(F_{120}\) & \(R_{123}\) & \(R_{240}\) & Id \\ \hline
  \end{tabular}
\end{table}
כאשר \(R_{\theta}\) מסמנת סיבוב ב-\(\theta\) מעלות, ו-\(F_{\theta}\) מסמן להפוך את הצורה לאורך הציר שנמצא ב-\(\theta\) מעלות.
ניתן לראות כי חבורה זו אינה קומוטטיבית

\subsection{איזומורפיזם}

ננסה לבנות חבורה כללית עם שלושה איברים, נסמן \(G=\{ e,a,b \}\). ננסה לבנות טבלת כפל:

\begin{table}[htbp]
  \centering
  \begin{tabular}{|cccc|}
    \hline
    \(\cdot\) & e & a & b \\ \hline
    \textbf{e} & e & a & b \\ \hline
    \textbf{a} & a &  &  \\ \hline
    \textbf{b} & b &  &  \\ \hline
  \end{tabular}
\end{table}
כאן נראה כאילו יש לנו בחירה, אך למעשה לא ייתכן ש-\(a\cdot a=a\) ויהיה את אותו ערך פעמיים בשורה. לכן הטבלת כפל שלנו תראה:

\begin{table}[htbp]
  \centering
  \begin{tabular}{|cccc|}
    \hline
    \(\cdot\) & e & a & b \\ \hline
    \textbf{e} & e & a & b \\ \hline
    \textbf{a} & a & b & e \\ \hline
    \textbf{b} & b & e & a \\ \hline
  \end{tabular}
\end{table}
ועבור מספרים כלליים קיבלנו טבלה יחידה. זה אומר שלכל החבורות עם שלושה איברים תהיה את הטבלה הזו, כלומר אפשר לחשוב על זה כאילו יש סוג של שיוויון בין בחבורות בגודל 3

\begin{definition}
אם \((G,\circ)\) ו-\((H,\times)\) שתי חבורות, אזי פונקציה \(f:G\to H\) נקראת איזומורפיזם(של חבורות) אם היא חח"ע ועל ושומרת כפל:
$$\forall g,g' \in G,\quad f(g\circ g')=f(g)\times f(g')$$

\end{definition}
\begin{symbolize}
אם קיים איזומורפיזם מחבורה \(G\) לחבורה \(H\), הן נקראות איזומורפיות. מסומן \(G\cong H\)

\end{symbolize}
\begin{proposition}
איזומורפיות היא יחס שקילות. אם \(G,H,K\) חבורות, אזי:

  \begin{enumerate}
    \item חבורה איזומורפית לעצמה - \(G\cong G\)


    \item אם \(G \cong H\) אזי \(H\cong G\)


    \item אם \(G\cong H\cong K\) אזי \(G\cong K\)


  \end{enumerate}
\end{proposition}
\begin{example}
$$(\{ \pm 1 \},x)\cong (\mathbb{Z}_{2},\oplus_{2}) \cong (\mathbb{F}_{3}\setminus \{ 0 \}, \otimes_{3} )$$$$(\mathbb{R},+)\cong (R_{>0},\times)$$

\end{example}
\begin{definition}[מכפלה ישרה]
אם \(G\),\(H\) חבורות, המכפלה הישרה שלהם (מסומנת \(G\times H\)) היא החבורה שאיבריה הם הזוגות\((g,h)\) כש-\(g\in G\) ו-\(h\in H\), והפעולה:
$$(g,h)\cdot (g',h')=(g\cdot g', h\cdot h')$$

\end{definition}
\begin{remark}
שאלה מתבקשתי היא מתי חבורה מסויימת איזומורפית למכפלה?
למשל, החבורה.
$$\mathbb{C}^{\times}\cong\mathbb{R}_{>0}^{\times}\times \underbrace{ S^1 }_{ \{ z\in \mathbb{C}\mid |z|=1 \} }$$
זו שאלה שנענה עליה בהמשך.

\end{remark}
\subsection{תתי חבורות}

\subsection{תתי חבורות}

\begin{definition}[תת חבורה]
אם \(G\) חבורה, ו-\(H\subseteq G\), אזי \(H\) נקראת תת חבורה של \(G\)

\end{definition}
\begin{definition}[תת חבורה נוצרת על ידי איבר]
אם \(G\) חבורה ו-\(S\subseteq G\), התת חבורה שנוצרת ע"י \(S\) היא התת חבורה המינימלית שמכילה את \(S\), מסומנת \(\langle S\rangle\).

\end{definition}
\begin{remark}
יש כזו! נקח את \(\bigcap_{{H\leq G,S\subseteq H}}H\)

\end{remark}
\begin{proposition}
$$\langle s\rangle=\left\{  \prod_{i=1}^{l}s_{i}\bigg| l\in\mathbb{N}\setminus \{ 0 \},s_{i}\in S,e_{i} \in \{ 1,-1 \}  \right\}$$

\end{proposition}
\begin{example}
  \begin{enumerate}
    \item עבור \(G=\mathbb{Z}\) מתקבל \(\mathbb{Z}=\langle 1\rangle\)


    \item עבור \(G=2\mathbb{Z}\) מתקבל \(2\mathbb{Z}=\langle 2\rangle\)


    \item עבור החבורה \(G=\mathbb{Z}_{9}^x\) מתקבל \(G=\langle 2\rangle\) 
תזכורת: \(Z_{n}^x=\{ 1\leq t\leq n \;\big|\;gcd(t,n)=1 \}\)


    \item עבור \(G=D_{n}\) אם \(\sigma\) סיבוב -\(\frac{2\pi}{n}\) ימינה ו-\(\tau\) שיקוף כלשהו אז: 
$$D_{n}=\langle \sigma,\tau\rangle $$


  \end{enumerate}
\end{example}
\begin{definition}[חבורה ציקלית]
חבורה נקראת ציקלית אם יש איבר שיוצר אותה, כלומר \(G=\langle g\rangle\). 

\end{definition}
לדוגמא \(\mathbb{Z},\mathbb{Z}_{9}^x\).
לא דוגמא: \(\mathbb{Z}_{8}\).

\begin{definition}[חבורה אבלית]
חבורה נקראת אבלית אם הכפל בה קומוטטיבי

\end{definition}
\begin{proposition}
חבורה ציקלית היא אבלית

\end{proposition}
\begin{proof}
קיים \(g\in G\) כך ש-\(G=\langle g\rangle\). הובע שכל איבר \(x \in G\) היא מהצירה \(g^m\) עבור \(m\in \mathbb{Z}\) כלשהו. ואז אם \(x,y \in G\) קיימים \(m,n \in \mathbb{Z}\) כך ש-\(x=g^m\), \(y=g^n\). ולכן:
$$x\cdot y=g^m \cdot g^n=g^{m+n}=g^n g^m=y\cdot x$$

\end{proof}
\begin{corollary}
החבורה \(D_{3}\) אינה ציקלית. 

\end{corollary}
\begin{definition}[מרכז של חבורה]
המרכז(center) של חבורה הוא:
$$Z(G) =\{ x \in G \big| \forall h\in G\quad gh=hg \}$$

\end{definition}
\begin{example}
  \begin{enumerate}
    \item אם \(G\) אבלית אז \(\mathbb{Z}(G)=G\)(ורק אם!) 


    \item \(\mathbb{Z}(D_{3})=\{ e \}\)


    \item \(Z(D_{4})=\{ e, R_{180} \}\)


  \end{enumerate}
\end{example}
\begin{proposition}
המרכז של חבורה היא תת חבורה שלה(\(\mathbb{Z}(G)\leq G\))

\end{proposition}
\begin{proof}
מכיל את איבר היחידה - \(eg=ge\). וכן:
$$\forall k \in G, gh \in \mathbb{Z}(G) \quad (gh)k=g(hk)=g(kh)=(gk)h=(kg)h=k(gh)$$
ואז:
$$g^{-1}k=\dots=kg^{-1}$$
וסגור להופכי

\end{proof}
\begin{definition}[גרף קיילי]
של חבורה ביחס לאיברים \(S\subseteq G\) הוא גרף עם קודקודים \(G\). וקשתות מכוונות מקודקוד \(g\), לקודקוד \(sg\) ו-\(s \in S\).

\end{definition}
\begin{proposition}
מתקיים \(\langle s\rangle=G\) אם"ם יש מסילה מכל קודקוד לכל קודקוד(גרף קשיר), כשמתעלמים מכיווני החצים.

\end{proposition}
\begin{definition}[קוסט]
אם \(G\) חבורה ו-\(H\leq G\), אז לכל \(g \in G\) הקוסט השמאלי של \(H\) המתקבל מ-\(g\) הוא:
$$gH=\{ gh\;|\; h\in H\}$$
כאשר הקוסט הימיני יהיה:
$$Hg=\{ hg \;|\; h\in H \}$$

\end{definition}
\begin{example}
  \begin{enumerate}
    \item ישריות במרחב וקטורי 


    \item עבור \(G=\mathbb{Z}\), \(H=5\mathbb{Z}\) נקבל: 
$$17+5\mathbb{Z}=\{ \dots,7,12,17,22,27,32,\dots \}$$


    \item עבור \(G=D_{3}\) ו-\(H=\{ e,F_{0} \}\) נקבל: 
$$\begin{gather}R_{120}\cdot H=\{ R_{120}, F_{120} \} \\H\cdot R_{120}=\{ R_{120}, F_{240} \}
\end{gather}$$


  \end{enumerate}
\end{example}
\begin{corollary}
אם \(C\subseteq G\) קוסט שמאלי של \(H\), ו-\(g \in C\) אזי \(C=gH\).

\end{corollary}
\begin{proof}
יודעים שיש \(x \in G\) כך ש-\(C=xH\). בפרט \(g=xh\) ל-\(h \in H\) כלשהו. ואז:
$$gH=(xh)H=x(hH)=xH$$

\end{proof}
\begin{corollary}
שני קוסטים שמאליים(ימניים) הם זהים או זרים. לכן \(G\) הוא איחוד זר של הקוסטים השמאליים של \(H\), וגם של הקוסטים הימניים של \(H\). כלומר: \(G=\bigsqcup gH\)

\end{corollary}
\begin{example}
$$G=3\mathbb{Z} \bigsqcup 1+3\mathbb{Z}\bigsqcup 5+3\mathbb{Z}$$

\end{example}
\begin{remark}
נשים לב אם \(G\) חבורה ו-\(H\subseteq G\), אז "להיות באותו קוסט שמאלי של \(H\)" הוא יחס שקילות על \(G\).

\end{remark}
\begin{proposition}
איברים \(x,y \in G\) באותו קוסט שמאלי אם"ם:
$$xH=yH \iff y^{-1}x \in H ,y^{-1}xH=H \iff y\in xH$$

\end{proposition}
\begin{example}
כאשר \(G=\mathbb{Z}\), \(H=mz\). מתי \(a\),\(b\) האותו קוסט שמאלי של \(H\)? זה יהיה כאשר \(a-b \in m\mathbb Z\), כלומר \(a\equiv b\mod(m)\)

\end{example}
\begin{definition}[אינדקס]
האנדקס של \(H\) ב-\(G\) (כש-\(H\leq G\)), הוא מספר הקוסטים השמאליים של \(H\) ב-\(G\). מסומן \([G:H]\)

\end{definition}
\begin{remark}
זה גם מספר הקוסטים הימניים - ההופכי של קוסט שמאלי הוא קוסט ימני:
$$(gH)^{-1}=\{ (gh)^{-1} \big| h\in H \}=\{ h^{-1}g^{-1} \big|h\in H \}=Hg^{-1}$$
מקבלים העתקה חח"ע ועל בין הקוסטים הימניים לשמאליים.

\end{remark}
\begin{proposition}
לכל הקוסטים של \(H\) אותו הגודל. 

\end{proposition}
\begin{proof}
ההעתקה \(T:G\to G\quad T(x)=gx\) היא חח"ע ועל. בפרט:
$$|gH|=|T(H)|=|H|$$

\end{proof}
\begin{corollary}
הגודל של \(G\) הוא הגדול(הזהה) של כל קוסט כפול מספר הקוסטים הזרים. כלומר:
$$\boxed{|G|=[G:H]|H|}
$$

\end{corollary}
\begin{remark}
זה כולל את המקרה האינסופי

\end{remark}
\begin{theorem}[לגרנג']
הגודל של תת חבורה מחלק את הגודל של החבורה. כלומר:
$$|H| \;\big|\;|G|$$

\end{theorem}
\begin{corollary}
אם \(G\) סופית ו-\(g^{|G|}=e\)

\end{corollary}
\begin{proof}
תהי \(H=\langle g\rangle\), אזי \(|H| \;\big|\;|G|\), נניח \(|G|=|H|\cdot m\). ואז:
$$g^{|G|}=(g^{|H|})^m=e^m=e$$

\end{proof}
\begin{corollary}
אם \(|G|\) ראשוני, אז אין ל-\(G\) תת חבורה לא טריוויאלית, ובפרט היא ציקלית(לכל \(e\neq g\) בהכרח \(\langle g\rangle=G\)).

\end{corollary}
\begin{theorem}[אוילר]
אם \(gcd(a,n)=1\), אזי \(a^{\varphi(n)}=1\mod n\). כאשר:
$$\varphi(n)=|Z_{n}^\times|=|\left\{  1\leq t\leq n \;\big|\;\mathrm{gcd}(t,n)=1  \right\}|$$

\end{theorem}
\begin{remark}
משומש באלגוריתם RSA

\end{remark}
\begin{definition}[תת חבורה נורמלית]
תת חבורה \(H\) של \(G\) נקראת נורמלית אם \(gH=Hg\) לכל \(g \in G\). כלומר המחלק הימיני שווה למחלק השמאלי. 
סימון: \(H \trianglelefteq G\).

\end{definition}
\begin{example}
  \begin{enumerate}
    \item תת חבורה אבלית היא נורמלית 


    \item המרכז של חבורה היא נורמלית גם אם לא אבלית(\(\mathbb{Z}(G)\trianglelefteq G\)). 


    \item סיבוב ב-\(D_{3}\) - \(\{ e,R_{120},R_{240} \}\trianglelefteq D_{3}\). 


  \end{enumerate}
\end{example}
\begin{remark}
תת חבורה של שיקוף הוא לא תת חבורה נורמלית ב-\(D_{3}\).

\end{remark}
\begin{proposition}
אם \([G:H]=2\) אזי \(H\trianglelefteq G\). 

\end{proposition}
\begin{proof}
$$\begin{gather}\forall g \in H\quad  gH=H=Hg\\\forall g\notin H\quad  gH=G\setminus H=Hg \\H \bigsqcup gH=G=H  \bigsqcup G \setminus H
\end{gather}$$

\end{proof}
\begin{proposition}
הביטויים הבאים שקולים:

  \begin{enumerate}
    \item \(H   \trianglelefteq G\)


    \item \(gHg^{-1}=H\)


    \item הקוסטים הימניים של \(G\) הם שמאליים ולהיפך. 


  \end{enumerate}
\end{proposition}
\begin{symbolize}
הקוסטים השמאליים של \(H\) ב-\(G\) מסומנים \(G/H\) והימניים מסומנים \(H/G\).

\end{symbolize}
\begin{example}
$$\mathbb{Z}/3\mathbb{Z}=\{ 3\mathbb{Z},3\mathbb{Z}+1,3\mathbb{Z}+2 \}$$
לכן ניתן להגיד שתת חבורה \(H\leq G\) היא נורמלית אם"ם \(G/H=H/G\).

\end{example}
\begin{definition}[הצמדה]
עבור איבר \(g\), הפונקציה מ-\(G\) ל-\(G\) ששולחת \(h\) ל-\(ghg^{-1}\)  נקראת הצמדה ב-\(g\).
לפעמים מסומנת \(\phi_{g}\):
$$\phi_{g}:G\to G \quad \phi_{g}(h)=ghg^{-1}$$
עוד סימון: \(^{g}h=ghg^{-1}\)

\end{definition}
\begin{corollary}
תת חבורה \(H\) תהיה תת חבורה נורמלית אם"ם \(\phi_{g}(H)=H\)  לכל \(g\in G\)

\end{corollary}
\begin{definition}[אוטומורפיזם]
איזומורפיזם ממנה לעצמה

\end{definition}
\begin{example}
$$f:\mathbb{Z}\to \mathbb{Z}\quad f(n)=-n$$

\end{example}
\textbf{שאלה}
מתי \(f:g\to g^{-1}\) היא אוטומורפיזם של חבורה \(G\).

\begin{proposition}
עבור כל \(g\in G\) , מתקיים ש-\(\varphi_{g}\) הוא אוטומורפיזם של \(G\)(נקראות אוטומורפיזמים פנימיים)

\end{proposition}
\begin{proof}
ראשית \(\phi_{g}\) חח"ע ועל. זאת כיוון ש:
$$\phi_{g}=\overbrace{\underbrace{ gh }_{ הרומת } g^{-1}}^{הרומת}$$
 או \(\phi_{g^{-1}}\) ההופכית שלה. 

\end{proof}
\subsection{תמורות}

\begin{reminder}
החבורה \(S_{n}\) היא החבורה של הפונקציות של איברים מהקבוצה \(\{ 1,\dots,n \}\) לעצמה שהם חח"ע ועל:

\end{reminder}
\begin{definition}[מחזור]
עבור \(1\leq k_{0},\dots,k_{m}\leq n\) שונים זה מזה
 מחזור הוא איבר \((k_{0},..,k_{m-1})\) של חבורת התמורות \(S_{n}\) כאשר:
 $$(k_{0},..,k_{m-1})(x)=\begin{cases}k_{i+1} & x=k_{i} \\k_{0}  & x=k_{m-1} \\x & \text{else}
\end{cases}$$

\end{definition}
\begin{example}
  \begin{enumerate}
    \item המחזור הטריוויאלי - \(()=e=(k_{0})\). מחזור הוא לא טריוויאלי אם"ם \(m\geq 2\). 


    \item המחזור המתואר בצורה הבאה: 
$$S_{3}\ni\begin{pmatrix}1  & 2 & 3 \\2 & 3 & 1\end{pmatrix} = \begin{pmatrix}1 & 2 & 3\end{pmatrix}=\begin{pmatrix}3 & 1 & 2\end{pmatrix}=\begin{pmatrix}2 & 3 & 1
\end{pmatrix}$$


  \end{enumerate}
\end{example}
\begin{remark}
הסדר של \(\begin{pmatrix} k_{0}& \dots & k_{m-1}\end{pmatrix}\in S_{n}\)  הוא \(m\) כש-\(m>2\)

\end{remark}
\begin{definition}[מחזורים זרים]
2 מחזורים \(\begin{pmatrix} k_{0}& \dots & k_{m-1}\end{pmatrix},\begin{pmatrix} l_{0}& \dots & l_{m-1}\end{pmatrix}\in S_{n}\) יקראו זרים אם \(\begin{pmatrix} k_{0}& \dots & k_{m-1}\end{pmatrix}\cap\begin{pmatrix} l_{0}& \dots & l_{m-1}\end{pmatrix}=\emptyset\)

\end{definition}
\begin{proposition}
כל תמורה שווה להרכבה של מחזורים זרים יחידים עד כדי סדר

\end{proposition}
\begin{definition}[תת מחזור]
תהי \(\sigma \in S_{n}\). המחזור 
$$\begin{pmatrix} k_{0}& \dots & k_{m-1}\end{pmatrix},\begin{pmatrix} l_{0}& \dots & l_{m-1}\end{pmatrix}\in S_{n}$$
 יקרא תת מחזור של \(\sigma\) אם:
$$\sigma\big|_{(k_{0},\dots,k_{m-1})} = (k_{0},\dots,k_{m})\big|_{(k_{0},\dots,k_{m-1})}$$

\end{definition}
\begin{lemma}
תהא \(\sigma \in S_{n}\), כל  \(1\leq x\leq n\) נמצא באיזשהו תת מחזור של \(\sigma\) לא טריוויאלי אם"ם \(x\neq \sigma x\).

\end{lemma}
"הוכחה"
$$(x,\sigma(x),\sigma^2(x),\dots,\sigma^{m-1}(x))$$

\begin{lemma}
תהא \(\sigma \in S_{n}\). ונניח כי \(c,c' \in S_{n}\) תתי מחזורים של \(\sigma\). אזי או ש-\(c,c'\) זרים או ש\(c=c'\).

\end{lemma}
\begin{proof}
אם \(c,c'\) זרים, סיימנו.
אחרת קיים \(1\leq k_{0}\leq n\) שנמצא גם ב-\(c\) וגם ב-\(c'\). ומתקיים עבורו.
$$\begin{gather}c=(x,\sigma(x),\sigma^2(x),\dots,\sigma^{m-1}(x)) \\c'=(x,\sigma(x),\sigma^2(x),\dots,\sigma^{m-1}(x))
\end{gather}$$
נניח כי \(c\) באורך \(m\) המינימלי מבין \(c,c'\), אז \(\sigma^M(k_{0})=k_{0}\). אבל \(\sigma^{m-1}(k_{0})\) נמצא ב-\(c'\), ומתקיים:
$$c'(\sigma^{m-1}(k))=\sigma^M(k_{0})=k_{0}$$
לכן יש c' גם נסגר באורך \(m\).

\end{proof}
כעת נוכל להוכיח את המשפט

\begin{proof}
תהא \(\sigma \in S_{n}\). נסתכל על הקבוצה
$$\{ c \in S_{n} \big|c\text{ לש ילאיווירט אל רוזחמ תת }\sigma \}=\{ c_{1},\dots,c_{s} \}$$

\end{proof}
\begin{proposition}
$$\sigma=c_{1}\circ\dots,c_{s}$$

\end{proposition}
\begin{proof}
יהא \(1\leq x\leq n\), נניח כי \(x\) נמצא ב-\(c_{i}\) עבור \(1\leq i\leq s\) כלשהו. 
בפרט, \(x\) לא ב-\(c_{i}\) עבור \(i\neq j\). אחרת \(c_j\), \(c_{i}\) לא זרים, ולכן \(c_{i}=c_{j}\).
$$c_{1}\circ \dots c_{i}\circ \dots c_{s}(x)=c_{1}\circ \dots c_{i}(x)$$
וכן, \(c_{i}\) תת מחזור של \(\sigma\), לכן \(c_{i}(x)=\sigma(x)\). בפרט, גם \(\sigma(x)\) נמצא

\end{proof}
הראנו כי עבור \(x\) שנמצא ב-\(c_i\)

\section{פעולות על חבורות}

\subsection{מסלולים ומייצבים}

\begin{definition}[פעולות\Action]
פעולה של חבורה \(G\) על קבוצה \(X\) היא אופרטור בינארי \(G\times X\to X\) שמקיימת:

\end{definition}
\begin{enumerate}
  \item \(\forall x \in X\quad ex=x\)


  \item \(\forall g,h \in G \quad x \in X \qquad (gh)x=g(hx)\)


\end{enumerate}
לרוב נסמן אותו בכפל, כאשר את הפעולה "\(G\) פועלת על X" נסמן ב-\(G\circlearrowright X\).

\begin{example}[פעולות]
  \begin{itemize}
    \item נגדיר \(S \circlearrowright \{ 1,\dots,n \}\) ע"י \(\sigma\cdot j=\sigma(j)\).
    \item נגדיר \(S_{x}\circlearrowright X\) כנ"ל
    \item נגדיר \(GL_{n}(\mathbb{F})\circlearrowright \mathbb{F}^n\) ע"י \(A\cdot v=Av\)(כפל מטריצה בווקטור)
    \item אם \(G\circlearrowright X\) ו-\(H\leq G\) אזי נגדיר \(H\circlearrowright X\) ע"י צמצום הפעולה.
    \item אם \(V\) מרחב וקטורי מעל \(\mathbb{F}\), \(\mathbb{F}^\times \circlearrowright V\) ע"י כפל בסקלר.
    \item החבורה \(D_{n}\)  פועלת על קודקודי המצולה המשוכלל ה-\(n\)
    \item החבורה \(D_n\) פועלת על צביעת קודקודי ה-\(n\) מצולע ב-\(k\) צבעים(כאשר \(|x|=k^n\)).
  \end{itemize}
\end{example}
\begin{remark}
אם נתון הפעולה \(G\circlearrowright X\) אז \(\langle s\rangle=G\), מספיק להבין איך איברי \(S\) פועלים כדי להבין את הפעולה. 
זה נעשה ע"י אסוציטיביות: \(s_{1}(s_{2}x)=(s_{1}s_{2})x\)  ובנוסף:
$$s(s ^{-1}x)=(ss ^{-1})x-ex=x$$
כלומר \(s ^{-1}x\) היה האיבר \(y\in X\) שנשלח ל-\(x\) ע"י \(s\).

\end{remark}
\begin{remark}
אם \(G\circlearrowright X\) אזי כל \(g\in G\) משרה, ע"י כפל, פרמוטציה של X:
$$\Pi_{g}(x)=gx$$
כאשר \(\Pi_{g}\) היא פרמוטציה של \(X\) כי \(\Pi_{g^{-1}}\) היא ההופכית שלה:
$$\Pi_{g^{-1}}(\Pi_{g}(x))=\Pi_{g^{-1}}(gx)=g^{-1}(gx)=(g^{-1}g)x=ex=x$$
בפרט זה אומר שאם \(s\) פועלת ע"י תמורה \(\Pi_{s}\), \(s ^{-1}\) פועלת ע"י התמורה ההופעית \(\Pi_{s}^{-1}\).

\end{remark}
\begin{example}[עוד פעולות]
  \begin{itemize}
    \item לכל חבורה \(G\) נגדיר \(G\circlearrowright G\) ע"י כפל:
 $$g,x \in G\qquad g\cdot x:=g\times x$$
 כאשר \(\cdot\) היא הפעולה ו-\(\times\) היא כפל ב-\(G\).
    \item ניתן להגדיר \(G\circlearrowright G\)  ע"י הצמדה: 
 $$\forall g,x \in  G \qquad g* x :=g\times g^{-1}$$
נבדוק פעולה:
$$\begin{aligned}1.&  &\forall x \in G &  e_{x}=e\times e^{-1}=x \\2.& &\forall g,h,x \in  G &  ^h(^yx)=hgx g^{-1}h^{-1}=(hg)\times (hg)^{-1}=x
\end{aligned}$$
  \end{itemize}
\end{example}
\textbf{הגדרה} מסלול
אם \(G\circlearrowright X\) ו-\(x \in G\), המסלול(orbit) של \(X\)(תחת G) הוא:
$$O_{G}(x)=O(x)=\{ gx \;\big|\; g\in G \} \subseteq X$$

\begin{proposition}
אם \(G\circlearrowright X\) אז היחס "להיות באותו מסלול" זה יחס שקילות

\end{proposition}
\begin{proof}
נראה שמקיים את שלושת הדרישות:
$$\begin{gather}1.\quad  ex=x \implies x \in O(g) \\2.\quad  y\in O(x)\implies \exists g\quad y=gx\implies x=g^{-1}y\implies x \in O(y) \\3.\quad z\in O(y),y\in O(x)\implies \dots \;?\;\dots\implies z\in O(x)
\end{gather}$$
כלומר פעולה של \(G\) על \(X\) משרה פירוק זר למסלולים.

\end{proof}
\begin{corollary}
אם \(x_{1},\dots,x_{r}\) נציגים למסלולים שונים אז הקבוצה היא האיחוד הזר של המסלולים:
$$X=\bigsqcup_{i=1}^r O(x_{i})$$

\end{corollary}
\begin{example}
עבור:
$$\left\{ \begin{pmatrix}\cos \theta & \sin \theta \\-\sin \theta  & \cos \theta
\end{pmatrix} \right\}=SO(2)\circlearrowright\mathbb{R}^2$$
המסלולים של הפעולה הם המעגלים ברדיוסים שונים. 

\end{example}
\begin{symbolize}
עבור \(G\circlearrowright X\) מסמנים ב-\(G\setminus X\) את אוסף המסלולים של הפעולה.

\end{symbolize}
\begin{remark}
הרבה בעיות במתמטיקה הם מהצורה \(G\setminus X=?\). למשל \(|D_{n}\setminus X |=\) כאשר \(X\) זה צבעות של n-מצולע ב-\(k\) צבעים

\end{remark}
\begin{remark}
הסימון של \(G\setminus H\) מסמן גם את הקוסטים הימניים של \(H\) ב-\(G\), וכן גם מסמן את המסלולים של \(H\circlearrowright G\) ע"י כפל אשר שווה ל-\(\{ hg|h\in H \}=Hg\) כאשר גדלים אלו שווים!
ניתן גם להגדיר פעולה ימינית של חבורה באופן מקביל לאיך שהגדרנו פעולה, ויתקבל שהמסלולים של הפעולה יהיה שווה לקוסטים השמאליים.

\end{remark}
\begin{definition}[פעולה טרנזטיבית]
פעולה נקראת טרנזטיבית אם יש מסלול יחיד

\end{definition}
\begin{definition}[פעולה נאמנה]
אם האיבר היחיד שמשרה את פרמוטציית הזהות על \(X\) הוא \(e\).

\end{definition}
\begin{definition}[פעולה חופשייה]
אם \(gx=x\) גורר \(g=e\)

\end{definition}
\begin{definition}[מייצב]
אם \(G\circlearrowright X\), המייצב של \(x \in X\) הוא:
$$\text{Stab}_{G}(x)=G_{x}=\left\{  g\in G \;\big|\;gx=x  \right\}$$

\end{definition}
\begin{proposition}
המייצב תמיד תת חבורה של G:
$$\begin{gather}ex=x\implies e\in  G_{x} \\gx=hx=x\implies ghx=x\implies g^{-1} \in  G_{x} \\
\end{gather}$$

\end{proposition}
\begin{theorem}[מסלול מייצב]
אם \(G\circlearrowright X\) ו-\(x \in X\) אזי:
$$|O(x)|=[G:G_x]$$
ואם \(|G|\) סופית נובע \(|O(x)|\cdot|G_{x}|=|G|\) מלגרנג'.

\end{theorem}
\begin{proof}
נשים לב שאם ורק אם \(g_{1},g_{2}\) באותו קוסט(שמאלי) של \(G_{x}\), אזי \(g_{1}x=g_{2}x\). זאת כי:
$$g_{1}^{-1} g_{2}\in G_{x}\rightleftarrows g_{1}^{-1} g_{2}x=x \rightleftarrows g_{2}x=g_{1}x$$
בנוסף לכל \(y\in O(x)\) יש \(g \in G\) עם \(gx=y\)(מהגדרת המסלול)
קיבלנו התאמה בין אוסף הקוסטים בשמאליים של \(G_{x}\) לבין המסלולים \(O(x)\).
פורמלית, בנינו פונקציה:
$$f:G/G_{x}\to O(x)\quad f(gG_{x})=gx$$
הוכחנו \(f\) מוגדרת היטב, על וחח"ע.
נסכם את ההוכחה. הפונקציה \(f\) מוגדרת היטב כי לא תלוייה באיזה איבר מתחילים
$$C=g_{1}G_{x}=g_{2}G_{x}$$
כאשר הגדרנו:
$$f(C)=g_{1}x\qquad f(C)=g_{2}x$$
על: מהגדרת המסלול, וחח"ע:
חח"ע:
$$f(g_{1}G_{x})=f(g_{2}G_{x})\to g_{1}x=g_{2}x\to g_{1}^{-1} g_{2}x\to g_{1}^{-1} g_{2}\in G_{x}\to g_{1}G_{x}=g_{2}G_{x}$$

\end{proof}
\begin{remark}
ניתן להוכיח ע"י בניית  \(f:O(x)\to G/G_{x}\).

\end{remark}
\begin{lemma}[שאינה של ברנסייד]
תהי \(G\) סופית ו-\(G\circlearrowright X\). מתקיים:
$$\overbracket{ |G\setminus X| }^{ orbits }=\frac{1}{|G|}\sum_{g\in  G}|\mathrm{fix}(g)|$$
כאשר \(\mathrm{fix}(g)=\{ x|gx=x \}\) נקודות השבת של \(g\).

\end{lemma}
\begin{proof}
$$\begin{gather}\frac{1}{|G|}\sum_{g\in G}|fix(g)|=\frac{1}{|G|} \sum_{g \in  G}\sum_{x \in  X}\delta_{x,gx}=\frac{1}{|G|}\sum_{x \in  X}\sum_{g \in  G}\delta_{x,gx}= \\\frac{1}{|G|}\sum_{x \in X}|G_{x}|=\sum_{x \in X} \frac{1}{|O(x)|}=|G\setminus X|
\end{gather}$$

\end{proof}
\subsection{מחלקות צמידות}

\begin{definition}[מחלקות צמידות]
המסלולים של הפעולה של חבורה \(G\) על עצמה על ידי הצמדה נקראים מחלקות צמידות של \(G\).
$$O(g)=\{ hgh^{-1} |h\in G \}=\{  ^hG | h \in  G \}=  ^Gg$$

\end{definition}
\begin{example}[מחלקות צמידות]
  \begin{enumerate}
    \item אם \(G\) אבלית \(G_{g}=\{ g \}\). 


    \item אם \(g\in Z(G)\), אז \(G_{g}=\{ g \}\) ורק אם: 
אם \(g\neq Z(G)\) יש \(h\) עם \(gh=hg\) ואז \(g\neq hgh^{-1}\).


  \end{enumerate}
\end{example}
\subsection{מחלקות צמידות}

הסתכלנו על הפעולה של חבורה על עצמה ע"י הצמדה. כלומר:
$$\forall g,x \in G\qquad g\cdot x=gxg^{-1} = \prescript{g}{}{x}$$
המסלול:
$$O(x)=\{ \prescript{g}{}{X}\;\bigg|\;g\in G \}$$
כאשר מתקיים \(O(x)=\{ x \}\) אם ורק אם \(x \in Z(G)\)

המייצב בפעולה הזו תקרא המרכז של \(x\):
$$C_{G}(x)=\{ g\in G\;\big|\; \prescript{g}{}{x}=x \}$$

\begin{definition}[מחלקות הצמידות]
המסלול של \(x\) נקרא מחלקת צמידות:
$$O(x)=\prescript{G}{}{x}=\{ \prescript{g}{}{x} \;\big|g \in G \}=[x]$$

\end{definition}
\begin{remark}
נשים לב כי אם \(\sigma,\tau \in S_{n}\), ו-\(\sigma=(a_{1},\dots,a_{r})\). אזי:
$$\prescript{\tau}{}{\sigma}=(\tau(a_{1}), \tau(a_{2}), \dots, \tau(a_{r}))$$

\end{remark}
\begin{remark}
מכיוון שהצמדה היא אוטומורפיזם, זה מספיק לחשב ל-\(\sigma\) כללית:
$$\prescript{\tau}{}{\sigma}=\prescript{\tau}{}{(\rho_{1},\rho_{2},\rho_{3})}=\tau \rho_{1}\rho_{2}\rho_{3}\tau ^{-1} =\tau \rho_{1}\tau ^{-1} \tau \rho_{2}\tau ^{-1} \tau \rho_{3}\tau ^{-1} =\prescript{\tau}{}{\rho_{1}}\prescript{\tau}{}{\rho_{2}}\prescript{\tau}{}{\rho_{3}}$$

\end{remark}
\begin{proof}
$$(\prescript{\tau}{}{\sigma})(\tau(a_{j}))=\tau \sigma \tau ^{-1} \tau(a_{j})=\tau(a_{j+1})$$

\end{proof}
\begin{corollary}
מחלקת הצמידות של תמורה ב-\(S_{n}\) היא כל התמורות עם אותו מבנים מחזורים, (כלומר כמה מחזורים יש מכל אורך בפירוק למחזורים זרים).

\end{corollary}
כש-\(G\circlearrowright X\), האיחוד זר של הנציגים למחלקת הצמידות   \(X=\bigsqcup_{}^{}O(x)\). במקרה של \(G\circlearrowright G\) בהצמדה, עבור האיחוד של \(g\) נציגים למחלקות הצמידות:
$$G=\bigsqcup_{\text{g }}^{} \prescript{G}{}{g}$$
ומתקיים:
$$\boxed{|G|=\sum|\prescript{G}{}{g}|=\sum \frac{|G|}{|C_{G}(g)|}=|Z(G)|+\sum[G:C_{G}(g)]}
$$
זה נקרא משוואת המחלקות.
דוגמא עבור \(S_{4}\):

\begin{table}[htbp]
  \centering
  \begin{tabular}{|cccccc|}
    \hline
    \(\sigma\) & \(()\) & \((1,2)\) & \((1,2,3)\) & \((1,2,3,4)\) & \((1,2)(3,4)\) \\ \hline
    \([\sigma]\) & 1 & 6 & 8 & 6 & 3 \\ \hline
  \end{tabular}
\end{table}
וקיבלנו שהסכום 24, ואכן, \(|S_{4}|=24\)

\begin{proposition}
אם \(G\neq \{ e \}\) חבורה בגודל חזקה ראשוני, אזי \(Z(G)\neq \{ e \}\).

\end{proposition}
\begin{proof}
יהיו \(g_{1},\dots,g_{r}\) נציגים למחלקות הצמידות שאינה במרכז.
אזי:
$$|G|=|Z(G)|+\sum_{i=1}^r \frac{|G|}{|C_{G}(g_{i})|}$$
מכיוון ש-\(g_{i}\not\in Z(G)\), \(C_{g}\lneq G\)(כלומר תת חבורה שאינה G).
לכן מלגרנג' \(|C_{G}(g_{i})|=p^m\) ל-\(m<k\), ואז:
$$p|p^{k-m}=\frac{|G|}{|C_{G}(g_{i})|}$$
לכן \(p\;\big|\;|Z(G)|\).
לכן \(p\;\big|\;|Z|\), ובפרט \(|Z(G)|\geq p\).

\end{proof}
\begin{theorem}[קושי]
אם \(p\;\big|\;|G|\) (ראשוני) אז יש ב-\(G\) איבר מסדר \(p\).

\end{theorem}
\begin{proof}
נגדיר:
$$X:=\{ (g_{0},g_{1},\dots,g_{p-1})\;\Big| \;g_{i}\in G\quad g_{0}\cdot \dots\cdot g_{p-1}=e \}$$
נשים לב ש-\(\mathbb{Z}_{p}\) פועלת על \(X\) ע"י סיבוב ציקלי:
$$\begin{align}1\cdot (g_{0},g_{1},\dots,g_{p-1})&=(g_{1},g_{2},\dots,g_{p-1},g_{0}) \\2\cdot (g_{0},g_{1},\dots,g_{p-1})&=(g_{2},g_{3},\dots,g_{0},g_{1}) \\
\end{align}$$
כעת אם נסכום על \(\vec{g}\) נציגים למסלולים של \(\mathbb{Z} _{p}\) על \(X\):
$$|X|=\sum|O(\vec{g})|$$
כאשר ממשפט מסולול מייצב:
$$|O(\vec{g})|=\frac{|\mathbb{Z} _{p}|}{Stab_{\mathbb{Z} _{p}}(\vec{g})}=\begin{cases}1  & Stab_{\mathbb{Z} _{p}}(\vec{g}) \\p & else
\end{cases}$$
מתי \(Stab_{\mathbb{Z} _{p}}(\vec{g})\) הוא כל \(\mathbb{Z}_p\)? אם (ורק אם) \(\vec{g}=(g,g..,g)\) ל-\(g\in G\) כלשהו. נשים לב שזה ב-\(X\) אם"ם \(g^p=e\), שזה מה שרצינו.
בנוסף \(|X|=G^{p-1}\),
$$X=\{ (g_{0},g_{1},g_{2},\dots,g_{p-2}, (g_{0},g_{1},\dots,g_{p-2})^{-1} )\;\big|\;g_{0},\dots,g_{p-2} \}$$
סה"כ \(p\) מחלק את \(|X|\) וגם את \(|\{ \vec{g}\in X\;\big|\; \vec{g}  \text{ קבוע} \}|\)
בנוסף, \((e,e, \dots,e)\) מבטיח שמספר ב-\(g\) הקבועים אינו 0.
זה מבטיח לפחות עוד \(p-1\) פתרונות.

\end{proof}
\begin{remark}
ניתן גם להוכיח ממשוואת המחלקה: 
$$p\;\big|\;G\to p\;\big|\;Z(g) \;\lor\;\exists g\in  G\quad p\;\big|\;C_{G(g)}$$
ומזה נובע קושי

\end{remark}
\begin{proposition}
יש 2 חבורות מסדר 6 עד כדי איזומורפיזם

\end{proposition}
\begin{proof}
תהי \(|G|=G\). מקושי, יש בה איבר \(x\)  מסדר 2, ואיבר \(y\) מסדר 3. נגדיר \(H:=\langle y\rangle\) מאינדקס 2. לכן \(H\trianglelefteq G\). ולכן \(xyx ^{-1}\in xHx ^{-1}=H\)
בפרט:
$$\begin{cases}xyx^{-1} =e\to y=e\to \text{סתירה} \\xyx^{-1} =y\to xy=yx \to |xy|=6 \to G \cong \mathbb{Z}_{6} \\xyx^{-1} =y^2 \to xy=y^2x \to  G\cong D_{3}
\end{cases}$$
כאשר את המקרה האחרון ניתן לראות ע"י מילואי טבלת הכפל.
והשורה השנייה מתקיימת כי:
$$(xy)^l=x^ly^l=e\to x^l=y^{-l}\in \langle x\rangle \cap \langle y\rangle =\{ e \}$$
כאשר:
$$\begin{cases}x^l=e\implies 2|l \\y^l=e\implies 3|l
\end{cases}\implies 6|l$$

\end{proof}
\section{הומומורפיזמים}

\subsection{הומומורפיזמים}

\begin{definition}[הומומורפיזם]
כאשר \(\left( G,\cdot  \right)\),\(\left( H,\times  \right)\) חבורות,
פונקציה \(f:G\to H\) אפשר מכבדת את הכפל, כלומר:
$$\forall x,y\in G\quad f\left(x\cdot y  \right)=f(x)\times f(y)$$

\end{definition}
\begin{definition}[מונומורפיזם אפימורפיזם איזומורפיזם ואנדומורפיזם]
אם \(f\) הומומופריזם אז אם \(f\):

  \begin{enumerate}
    \item חח"ע אז נקרא \underline{מונומורפיזם}. 


    \item על אז נקרא \underline{אפימורפיזם}. 


    \item חח"ע ועל אז נקרא \underline{איזומורפיזם}. 


    \item מקבוצה לעצמה אז נקרא \underline{אנדומורפיזם}. 


  \end{enumerate}
\end{definition}
\begin{example}
הפונקציות הבאות הם הומומורפיזמים:

  \begin{enumerate}
    \item כל \(G\),\(H\)\(f:G\to H\) המוגדרת \(f(g)=e_{H}\)


    \item אם \(G\leq H\), \(f(g)=g\) לכל \(g \in  G\). 


    \item עבור \(f:\mathbb{Z} \to \mathbb{Z} _n\), נגדיר \(f(j)=j\mod n\)


    \item אם \(\mathbb{F}\) שדה, אז: \(\det:GL_{n}\to \mathbb{F}^\times\) המוגדרת \(\det\left( A\cdot B \right)=\det(A)\cdot \det (B)\) זה אפימורפיזם כי: 
$$\forall \alpha \in \mathbb{F}  ^{\times}\qquad \det \left( \begin{matrix} \alpha  &  &  \\   & 1 & & \\  &  & \ddots &  \\  &  &  & 1
 \end{matrix} \right)=\alpha$$


    \item נגדיר \(\mathrm{Tr}:M_{n}(\mathbb{F)}\to \mathbb{F}^+\) כך ש-\(\mathrm{Tr}(A+B)=\mathrm{Tr}(A)+\mathrm{Tr}(B)\)


    \item נגדיר \(\mathrm{sgn}:S_{n}\to \left\{  \pm 1  \right\}\) כי \(\mathrm{sgn}\left( \sigma \circ \tau \right)=\mathrm{sgn}\left( \sigma \right)\cdot \mathrm{sgn}\left( \tau \right)\)


    \item נגדיר \(f:\mathbb{R} \to \mathbb{C} ^\times\) כך ש-\(f\left( \theta \right)=e^{i\theta}\)


    \item נניח \(A\in M_{n}\left( \mathbb{R}  \right)\). נגדיר \(f:\mathbb{R} \to GL_{n}\left( \mathbb{R}  \right)\) כך ש-\(f(t)=e^{tA}\)


  \end{enumerate}
\end{example}
\begin{proposition}[תכונות של הומומורפיזמים]
אם \(f:G\to H\) הומומורפיזם:

  \begin{enumerate}
    \item \(f(e)=e\)


    \item \(f(g^{-1} )=f(g)^{-1}\)


    \item אם \(f':H\to K\) עוד הומומורפיזם, \(f'\circ f:G\to K\) גם הומומורפיזם 


  \end{enumerate}
\end{proposition}
\begin{proposition}
התמונה של הומומורפיזם תהיה תת חבורה. כלומר \(im(f)\leq H\).

\end{proposition}
\begin{proof}
אם \(h,h'\in im(f)\). קיימים \(g,g'\in G\) עם \(f(g)=h,f(g')=h'\) ואז: $$f(gg')=f(g)f(g')=hh'$$ כמו כן, \(f(g)^{-1} =f(g^{-1} )\) וכן \(f(e)\in Im(f)\).

\end{proof}
\begin{definition}[גרעין של הומומורפזים]
הגרעין של הומומורפיזם \(f\) מוגדר:
 $$\ker(f):=\left\{  g\in G\;|\;f(g)=e  \right\}$$

\end{definition}
\begin{proposition}
הומומורפיזם \(f\) חח"ע אם"ם \(\ker(f)=\{ e \}\)

\end{proposition}
\begin{proposition}
הגרעין של הומומורפיזם יהיה תת חבורה נורמלית. כלומר מתקיים \(\ker(f) \trianglelefteq G\).

\end{proposition}
\begin{proof}
  \begin{enumerate}
    \item : אם \(g,g'\in \ker(f)\) אזי: 
$$\ker(f)\leq G \impliedby  \begin{cases}f\left( g\cdot g' \right)=f(g)f(g') = ee=e \\f(g^{-1} )=f(g)^{-1} =e^{-1} =e \\f(e)=e
\end{cases}$$
ואכן \(f\left( g\times g^{-1}  \right)=f(g)f(x)f(g^{-1} )=e\)
  \end{enumerate}
\end{proof}
\begin{proposition}
אם \(f(g)=h\) אזי \(f^{-1} (h)=g\ker(f)=\ker(f)g\)

\end{proposition}
\begin{proof}
אם \(k\in \ker(f)\), אז: $$f(gk)=f(g)f(k)=h$$ מצד שני, אם \(f(g')=h\), נרצה להראות \(g'\in \ker(f)\).
$$f(g')=h=f(g)\implies f(g')f(g)^{-1} = f(g'g^{-1} )=e\implies g'g\in \ker(f)\implies g'\in \ker(f)g=g\ker(f)$$

\end{proof}
\begin{definition}[הומומורפיזם המבנה]
אם \(G\circlearrowright X\) אזי ראינו שכל \(g\in G\) פועל על \(X\) ע"י פרמוטציה \(\Pi_{g}\in S_{x}\). נסתכל על \(\rho:G\to S_{X}\). \(\rho\) נקראת הומומורפיזם המבנה של הפעולה.

\end{definition}
\begin{proposition}
הומורפיזם המבנה \(\rho\) היא הומומורפיזם.

\end{proposition}
\begin{proof}
$$\forall x\quad  \rho\left( g\cdot g' \right)(x)=\Pi_{g,g'}=(gg')(x)=g(g'x)=\Pi_{g}\left( \Pi_{g'}(x) \right)=\left( \rho(g) \circ  \rho(g') \right)(x)$$
בכיוון ההפוך: אם \(G\) חבורה, \(X\) קבוצה, ונתון הומומורפיזם\(\rho:G\to S_{X}\). אפשר להגדיר פעולה של \(G\) על \(X\) ע"י \(g\cdot x=\rho(g)(x)\). זו אכן פעולה. לכל \(x \in X\) ולכל \(g,g'\in G\) מתקיים: 
$$\begin{gathered}e x=\rho(e)(x)=id(x)=x \\(g'g)x= \rho(gg')(x)=\left( \rho(g)\circ \rho(g') \right)(x)=\rho(g)\left( \rho(g')(x) \right)=g'\cdot \left( g\cdot x \right)
\end{gathered}$$ בסופו של דבר, למצוא הומורפיזם מ-\(G\) ל-\(S_{X}\) זה שקול ללהגדיר פעולה של \(G\) על \(X\).

\end{proof}
\begin{example}
למשל אם \(G\) פעולת על עצמה ע"י הצמדה: \(g.x=gxg^{-1}\). נגדיר \(\rho:G\to Aut(G)\leq S_{G}\) כך ש-\(\rho(g)(x)=gxg^{-1}\), או \(\rho(g)=\varphi_{g}\). מה הגרעין של \(\rho\)? \(Z(G)\) (\(g\in Z(G)\iff \varphi_{g}=id\)).
וכן \(Im\left( \rho \right)=Inn(G)\)(כאשר \(Inn(G)\) זה האיזומורפיזמים הפנימיים) של \(G\).

\end{example}
\begin{remark}
מתקיים \(Aut(S_{n})=Inn(S_{n})\) לכל \(n\neq 6\).

\end{remark}
\begin{remark}
אם \(G\circlearrowright X\) ו-\(\rho\) הומומורפיזם המבנה, \(G\) חח"ע אם"ם הפעולה נאמנה(faithful).

\end{remark}
\begin{remark}
אם \(f:G\to H\)(הומומורפיזם חח"ע). אזי \(G\cong Im(f)\) (כי \(f:G\to Im(f)\) על, חח"ע ומכבדת כפל). לכן אם \(G\circlearrowright X\) בצורה נאמנה מתקבל איזומורפיזם של \(G\) לתת חבורה של \(S_{X}\).

\end{remark}
\subsection{מנות}

תהי \(G\) חבורה ו-\(H\leq G\) תת חבורה. מתי הקוסטים \(G/H\) מקבלים מבנים של חבורה מתוך המבנה של \(G\)?

\begin{example}
עבור \(G=\mathbb{Z} ,H=7\mathbb{Z}\) נקבל:
$$G/H=\left\{  7\mathbb{Z} , 1+7\mathbb{Z} ,2+7\mathbb{Z} ,\dots,6+7\mathbb{Z}\right\}$$
יש התאמה בין \(\mathbb{Z} /m\mathbb{Z}\) ע"י:
$$(j+m)\mathbb{Z} \to j\mod m$$
הגדרה זו מכבדת את הפעולות במובן ש-\((j+k)+m\mathbb{Z}\) עובר ל-\(j\oplus k\)(\(\oplus\) זה חיבור ב-\(\mathbb{Z}_{m}\)). למעשה אפשר להגדיר חיבור על \(\mathbb{Z} /m\mathbb{Z}\) ע"י:
$$\left( j+m\mathbb{Z}  \right)+\left( k+m\mathbb{Z}  \right)=(j+k)+m\mathbb{Z}$$
ומתקבל חבורה (איזומורפית ל-\(\mathbb{Z} _m\)). לדוגמא:
$$m=7: \left( 3+7\mathbb{Z}  \right)+\left( 6+7\mathbb{Z}  \right)=9+7\mathbb{Z} =2+7\mathbb{Z}$$
נראה כי החיבור מוגדר היטב:
$$\left( j+xm+m\mathbb{Z}  \right)+\left( k+ym+m\mathbb{Z}  \right)=j+k+xm+ym+m\mathbb{Z} =x+y+m\mathbb{Z}$$

\end{example}
\begin{proposition}[משפט קיילי]
כל חבורה איזומורפית לתת חבורה של חבורת תמורות.

\end{proposition}
\begin{proof}
ראינו שאם \(G\) פועלת נאמנה על קבוצה \(X\) אזי הומומורפיזם המבנה \(\rho:G\to S_{X}\) הוא חח"ע = מונומורפיזם = שיכון. ואז \(G\cong Im(P)\leq S_{X}\). הפעולה של \(G\) על עצמה על ידי כפל היא נאמנה ולכן \(G\) איזומורפית לתת חבורה של \(S_{G}\).

\end{proof}
\begin{example}
למשל \(G=D_{4}\), מתקבל \(G\cong Im\left( \rho \right)\leq S_{8}\cong S_{(D_{4})}\). ו-\(|S_{8}|=40320\). אם \(H\leq G\), אפשר להגדיר פעולה של \(G\) על \(G/H\):
$$g.(g'H)=gg'H$$
ראינו שאפשר להגדיר מבנה של חבורה על\(\mathbb{Z} /n\mathbb{Z}\) על ידי:
$$\left( j+n\mathbb{Z} \right) \oplus\left( k+n\mathbb{Z}  \right)=j+k+n\mathbb{Z}$$
וחבורה הזו איזומורפית ל-\(\mathbb{Z} _n\) על ידי:
$$j+n\mathbb{Z} \to j \mod n$$
וניתן למשל בקלות עם הגדרה זו להראות אסוציטיביות: 
$$\begin{gathered}\left( \left( j+n\mathbb{Z}  \right)\oplus\left( k+n\mathbb{Z}  \right) \right)\oplus\left( l+n\mathbb{Z}  \right)=(j+k)+l+n\mathbb{Z} \\=j+(k+l)+n\mathbb{Z}  = \left( j+n\mathbb{Z}  \right)\oplus\left( \left( k+n\mathbb{Z}  \right)\oplus l+n\mathbb{Z}  \right)
\end{gathered}$$

\end{example}
נרצה להכליל רעיון זה באופן כללי. בהנתון \(H\leq G\),רוצים להגדיר מבנה של חבורה על הקוסטים \(G/H\): 
$$(gH)\cdot  (g'H)=gg'H$$
בעיה: לא מוגדר היטב! יכול להיות ש-\(g''H-g'H\), אבל \(gg''H\neq gg'H\)

\begin{example}
$$H=\langle (12)\rangle$$

\end{example}
\begin{proposition}
אם \(N\trianglelefteq G\) תת חבורה נורמלית, אזי \((gN)(g'N)=gg'N\) מוגדר היטב.

\end{proposition}
\begin{proof}
הנציגים של \(gN\) הם בדיוק \(gn\) כש-\(n\in N\). צריך להראות שלכל \(n,n' \in N\):
$$\overbrace{ (gnN) }^{ gN }(g'n'N)O=gng'n'N\stackrel{?}{=} gg'N$$
נראה: 
$$gng'n'N=gng'N=gnNg'=gNg'=gg'N$$

\end{proof}
\begin{corollary}
אם \(N\trianglelefteq G\), אפשר להגדיר חבורה(נקראת חבורת המנה של \(G\) ב-\(N\)) שאיבריה \(G/N\) והכפל בה הוא:
$$(gN)\times_{G/N} (g'N)=\left( g\times_{G} g' \right)N$$
הכפל שהגדרנו זהה לכפל הקבוצתי. כלומר: $$(gN)\times_{G} (g'N)=gg'N$$ כי:
$$(gN)\cdot g'N=g(Ng')N=g(g'N)N=gg'(NN)=gg'N$$
אם \(N\leq G\), הכפל שהגדרנו על הקוסטים \(G/N\) מוגדר היטב אזי \(N\trianglelefteq G\).

\end{corollary}
\begin{example}
  \begin{enumerate}
    \item אנו יודעים כי חבורת התמורות הזוגיות היא תת חבורה נורמלית של חבורת התמורות. מתקיים: 
$$S_{n} / A_{n} \cong \mathbb{Z} _{2}$$


    \item \(S^1 \cong \mathbb{C} ^{\times } / \mathbb{R} _{>0}\)


    \item \(\mathbb{C} /\mathbb{R} \cong i\mathbb{R} \leq \mathbb{C}\)


    \item אם \(N\trianglelefteq G\) מסמנים את חבורת המנה \(G/N\). יש אפימורפיזם ("ההטלה הקנונית") מ-\(G\) ל- \(G / N\).\\
$$\begin{gathered}\pi:G\to G / N \\\pi(g)=gN
\end{gathered}$$


  \end{enumerate}
\end{example}
\begin{proposition}
תת חבורה \(H\leq G\) היא נורמלית אם"ם היא גרעין של הומומורפיזם

\end{proposition}
\begin{proof}
ראינו שגרעין הוא נורמלי. מצד שני אם \(H\trianglelefteq G\) אזי
היא הגרעין של ההטלה הקנונית: $$\begin{gathered}\pi:G\to G / H \\g\in \ker\left( \pi \right) \leftrightarrow gH=\pi(g)=e_{G / H}=H \leftrightarrow g\in H
\end{gathered}$$ מכך נסיק כי: $$H=\ker\left( \pi \right)$$

\end{proof}
אם \(N\trianglelefteq G\), אם \(G\) נוצרת ע"י \(S\) אזי \(G / N\) נוצרת ע"י \(\left\{  sN\;|\; s \in S  \right\}\). בפרט אם \(G\) ציקלית אזי \(G / N\) ציקלית

\begin{proposition}
אם \(G / Z(G)\) ציקלית אז \(G\) אבלית.

\end{proposition}
\begin{proof}
נכתוב \(N=Z(G)\), ונניח ש-\(gN\) יוצרת את \(G / N\). אם \(x,y \in G\),
אזי \(y\in g^kN\), \(x \in g^jN\) ל-\(j,k\in \mathbb{N}\) כלשהם. ואז:
$$xy=g^j n g^k n'=g^h n'g^jn=yx$$

\end{proof}
\begin{proposition}
אם \(G/Z(G)\) ציקלית אז \(G\) אבלית(ולכן \(|G/Z(G)|=1\)).

\end{proposition}
\begin{proof}
נסמן \(N=Z(G)\). אם \(G/N\) ציקלית, אז יש לה יוצר מ-\(gN\):
$$G/N=\left\{  \dots,g^{-1} N,N,gN,g^2N, \dots  \right\}$$
לכן אם \(x,y \in G\) אז יש \(x \in g^iN\) ו-\(y\in g^kN\). ואז ישנם \(n,n' \in N\) כך ש-\(x=g^in\) ו-\(y=g^kn'\). כעת כולם מתחלפים זה עם זה: $$xy=g^ing^kn'=yx$$

\end{proof}
\subsection{משפטי האיזומורפיזם}

\begin{reminder}
הראנו שתמיד \(\ker(f) \trianglelefteq G\), \(im(f)\leq H\)

\end{reminder}
\begin{theorem}[האיזומורפיזם הראשון]
יהי \(f:G\to H\) הומומורפיזם. אזי: 
$$G/\ker(f) \cong im(f)$$

\end{theorem}
\begin{proof}
ראינו כבר ש-\(f\) משרה התאמה בין קוסטים של \(\ker(f)\) ב-\(G\) לאיברי \(im(f)\):
כי אם \(f(g)=h\) אזי לכל \(k\in \ker(f)\) גם \(f(gh)=f(g)f(k)=h\) כלומר כל \(g\cdot \ker(f)\) הולך ל-\(h\). ומצד שני אם \(f(g')=h\) אזי \(g'\in g\ker(f)\) כי 
$$f(g^{-1} g')=f(g)^{-1} \cdot f(g')=h^{-1} \cdot h=e\implies g^{-1} g\in \ker(f)\to g'\in g\cdot \ker(f)$$
כעת נסמן ב-\(\bar{f}\) את הפונקציה ש-\(f\) משרה על הקוסטים של \(\ker(f)\). כלומר:
$$\bar{f}:G/\ker(f)\to \mathrm{Im}(f) \qquad \bar{f}\left( g\cdot \ker(f)  \right)=f(g)$$
למעשה מה שעשינו עד עכשיו זה להראות ש-\(f\):

  \begin{enumerate}
    \item מוגדרת היטב 


    \item חחע - 
$$\bar{f}(g'K)=\bar{f}(gK)\implies f(g)=f(g')\implies g'\in gK\implies gK=g'K$$


    \item ועל - \(im(f)=im\left( \bar{f} \right)\) כי \(\bar{f}\left( g\ker(f)  \right)=f(g)\) ונותר להראות \(f\) הומומורפיזם, ולכן יהיה איזומורפיזם כיוון שחח"ע ועל: 
$$\bar{f}((gK)(g'K))= f(gg'K)=f(gg')=f(g)f(g')=  \bar{f}(gK)\cdot \bar{f}(g'K)$$
וקיבלנו כי \(f\) איזומורפיזם.


  \end{enumerate}
\end{proof}
\begin{example}
$$f:\mathbb{Z} \to \mathbb{Z} _{n} \qquad f(k)=k\mod n$$
ומתקיים:
$$im(f)=\mathbb{Z} _{n}\quad \ker(f)=n\mathbb{Z}$$
ולכן:
$$\mathbb{Z} / n\mathbb{Z}  \cong \mathbb{Z} _{n}$$

\end{example}
\begin{example}
$$\begin{gathered}f:\mathbb{R} \to \mathbb{C} ^\times \quad f\left( \alpha \right)=e^{i\alpha}
\end{gathered}$$
וכן: 
$$\mathrm{ im }(f)=S^1\qquad \ker(f)=\mathbb{Z}$$
ולכן:
$$\mathbb{R} / \mathbb{Z}  \cong S^1$$

\end{example}
\begin{example}
$$f:G\to Aut(G)\qquad f(g)=\varphi_{g}\;\;\left( \varphi_{g}(x)=gxg^{-1}  \right)$$
הראנו ש-\(f\) הוא הומומורפיזם.
$$\mathrm{Im}(f)=Inn(G)\qquad \ker(f)=Z(G)$$
ולכן:
$$\mathrm{Inn}(G) \cong G/ Z(G)$$

\end{example}
\begin{example}
אם \(V,W\) מרחבים וקטורים, ו-\(T:V\to W\) העתקה לינארית, אז אם נשכח את נושא הכפל בסקלר \(T:V\to W\) הומומורפיזם של חבורות. אז:
$$V/ \ker(T) \cong im(T)$$
וזהו משפט המימדים.

\end{example}
\begin{theorem}[האיזומורפיזם השני]
אם \(H\leq G\), \(N\trianglelefteq G\) אזי:
$$HN\leq G \quad H\cap N\trianglelefteq H$$
ומתקיים:
$$HN/ N \cong H / H\cap N$$

\end{theorem}
\begin{proof}
את הטענה \(HN\leq G \quad H\cap N\trianglelefteq H\) מוכיחים בתרגיל בית נסתכל על \(f:H\to HN/N\) המוגדר: $$f(h)=hN$$ מספיק להראות ש-\(f\) אפימורפיזם עם גרעין \(N\cap H\):

  \begin{enumerate}
    \item מוגרת היטב 


    \item על - לכל \(hnN\)(איבר ב-\(HN / N\)) מתקבל \(f(h)=hN=hnN\) 
נראה כעת כי \(f\) הומומורפיזם:
$$f(hh')=hh'N=hh'\mathbb{N} =hNh'N=f(h)\cdot f(h')$$
כאשר השתמשו בזה של \(N\trianglelefteq G\). נשאר הראות את הגרעין:
$$h \in \ker(f) \leftrightarrow f(h)=e\leftrightarrow hN=N\leftrightarrow h \in N$$
ולכן \(\ker(f)\) הוא כל איברי \(H\) ששייכים ל-\(N\), כלומר \(H\cap N\). סה"כ קיבלנו: 
$$H / H\cap N = H / \ker(f) \cong im(f) = HN/ N$$


  \end{enumerate}
\end{proof}
\begin{example}
$$\begin{gathered}G=\mathbb{C}^\times \qquad  N=S^1\qquad H=\mathbb{R}^\times \\ HN=\mathbb{C} ^\times \qquad H\cap N=\{ \pm 1 \}
\end{gathered}$$
ולכן:
$$C^\times / \mathbb{R} ^\times  \cong S^1/\{ \pm 1 \}$$

\end{example}
אם \(N\trianglelefteq G\), ו-\(\pi:G\to G/N\) ההטלה הקנונית(\(\pi(g)=gN\)). מה קורה לתת חבורה של \(G\) תחת \(\pi\)? \(\pi(H)\leq G/N\) (\(im\) של הומומורפיזם).
$$\begin{gather}H\leq G\quad \pi(H)=\left\{  \pi(h)|h\in H  \right\}=\left\{  hN | h\in H  \right\}=HN / N  \\= \left\{  gN |g\in HN  \right\}= \left\{  hnN | h \in H\quad n\in N  \right\}=\left\{  hN|h\in H  \right\}
\end{gather}$$
מה קורה ל-\(K\leq G/N\) תחת \(\pi ^{-1}\)? נניח \(\pi ^{-1}(K)\leq G\) (בתרגיל - לכל הומומורפיזם), \(K=\left\{  g_{1}N,g_{2}N, \dots,g_{r}N, \dots  \right\}\)$$\pi ^{-1} (K)=\bigcup_{gN\in K} \overbracket{ \pi ^{-1} \underbrace{ (gN) }_{ \in G/N } }^{ gN\subseteq G }=\bigcup_{gN\in K}gN$$
מה קורה כאשר מפעילים את שניהם? אם \(K\leq G / N\)$$\pi\left( \pi ^{-1} (H) \right)=K$$
אם \(f:X\to Y\) ו-\(f\) על, אזי:
$$\forall S\subseteq Y\quad f(f^{-1} (S))=S$$
מה לגבי \(\pi ^{-1} \circ \pi\)? נניח \(H\leq G\):
$$\pi ^{-1} \left( \pi(H) \right)=\pi ^{-1} (HN / N)=\bigcup_{gN\in HN / N} gN $$
וכיוון שמתקיים \(HN / N=\left\{  hN | h\in H  \right\}\) נקבל:
$$\bigcup_{h \in H} hN=HN$$

\begin{remark}
אפשר גם בדרך ישירות לחשוב על זה בתור איחוד של כל איברים של חבורת מנה - ולכן נקבל את כל החבורה.

\end{remark}
למעשה, \(\pi\) לוקחת תת חבורה של \(G\) ומחזירה תת חבורה של \(G / H\), כאשר \(\pi ^{-1}\) מקבלת תת חבורה של \(G / H\) ומחזיר תת חבורה של \(G\).

\begin{theorem}[ההתאמה(איזומורפיזם רביעי)]
אם \(N\trianglelefteq G\), \(\pi\) ו-\(\pi ^{-1}\) משראות התאמה חח"ע ועל בין תת חבורה של \(G / N\) לתת חבורה שמכילות את \(N\). וההתאמה הזו משמרת הכלה, נורמליות אינדקסים.

\end{theorem}
\begin{symbolize}
איברי \(G / N\) הם \(\left( g\in G \right)\quad gN\). יששמסמנים את \(gN\) ב-\(\bar{g}\) ואז:
$$\bar{g}\cdot \bar{h}=\overline{gh}\qquad g^{-1} h\in N\iff \bar{g}=\bar{h}$$
או לחלופין זוכרים/מציינים שאנחנו במנה על ידי המילים "מודולו \(N\)".

\end{symbolize}
ראינו שיש התאמה בין תתי חבורות של \(G/N\) לתתי חבורות של \(G\) שמכילות את \(N\).
$$\left\{  H\leq G \mid N\leq H  \right\}\xleftrightarrow[H\to H/N  ]{\bigcup_{gN\in K} gN\leftarrow K} \left\{  K\leq G / N \right\}$$
התאמה זו ממשמרת הכלה, אינדקס(סדר) ונורמליות. לגבי הנורמליות, אם \(N\leq K\trianglelefteq H\leq G\) וזה מתקיים אם"ם \(1=N / N\leq K / N \trianglelefteq H / N \leq G / N\) וזה מוביל למשפט האיזומורפיזם השלישי

\begin{theorem}[האיזומורפיזם השלישי]
אם \(N\trianglelefteq K\trianglelefteq H\), \(N\trianglelefteq H\) אזי:
$$^{H / N }/ _{ K  / N } \cong H/K$$
או באופן מילולי: "חבורת מנה של חבורת מנה היא איזומורפית לחבורת מנה".

\end{theorem}
ניתן להוכיח באמצעות משפט האיזומורפיזם הראשון.

\begin{example}
$$^{\mathbb{Z} / 12 \mathbb{Z} }
/ _{ 6\mathbb{Z}  / 12 \mathbb{Z} } \cong 3 \mathbb{Z}  / 6\mathbb{Z}$$

\end{example}
\section{פירוק של חבורות}

\subsection{מכפלה ישרה}

\begin{definition}[מכפלה ישרה חיצונית]
לחבורות \(H,K\leq G\) המכפלה הישרה שלהם תהיה:
$$ H\times K=\{(h,k)\mid h\in H,k\in K\}$$

\end{definition}
\begin{proposition}
אם \(H_{1}, H_{2}\) חבורות כך ש-\(G=H_{1}\times H_{2}\) אזי קיימת תת חבורה \(N\) כך ש-\(N\cong H_{1}\) כאשר \(N\trianglelefteq G\) ו-\(G / N \cong H_{2}\)

\end{proposition}
\begin{proof}
אם \(H_{1}, H_{2}\) חבורות, אפשר לקחת \(G=H_{1}\times H_{2}, N=H_{1}\times 1=\left\{  (h,e)\mid h\in H_{1}  \right\}\) ואכן \(N\cong H_{1}\). כמו כן \(N\trianglelefteq G\) כך שמתקיים:
$$(h_{1}h_{2})(h_{1}e)(h_{1}^{-1} h_{2}^{-1} )=h_{1}h_{2}he\in N$$
כעת \(G / N \cong H_{2}\), ממשפט האיזומורפיזם הראשון עבור: 
$$\begin{cases}f:G\to H \\f(h_{1},h_{2})=h_{2}\end{cases}\implies \begin{array}{c }\operatorname{im}(f)=H_{2} \\\ker(f)=N 
\end{array}$$

\end{proof}
\begin{corollary}
ניתן לקבל את אותה טענה עבור \(H_{2}\). כלומר אם \(G=H_{1}\times H_{2}\) אזי קיים תת חבורה \(N\cong H_{2}\) כך ש-\(N\trianglelefteq G\) וגם \(G / N \cong H_{1}\).

\end{corollary}
נרצה למצוא מתי אפשר לפרק חבורה למכפלה ישרה של שתי חבורות "קטנות" יותר. למשל אם \(H,K\leq G\), נסתכל על הפונקציה: 
$$\begin{cases}f:H\times K\to G \\f(h,k)=hk
\end{cases}$$
מתי \(f\) היא איזומורפיזם? למשל:
$$S_{1}\times R^{\times}_{>0}\cong C^{\times}$$
אבל:
$$\langle (1,2) \rangle \times A_{3}\not\cong  S_{3}$$
כיוון שאומנם חח"ע ועל אך לא הומומורפיזם.

\begin{proposition}
פונקציה \(f:H\times K \to G\) המוגדרת על ידי \(f(h,k)=hk\) היא איזומורפיזם אם"ם:

  \begin{enumerate}
    \item \(HK=G\)


    \item \(H\cap K=\{ e \}\)


    \item \(H,K\trianglelefteq G\)


  \end{enumerate}
\end{proposition}
\textbf{הוכחה}

\begin{enumerate}
  \item שקול לעל 


  \item שקול לכך ש-\(f\) חח"ע(הראנו כי \(\frac{|H||K|}{|H\cap K|}=|HK|\)) 


  \item אם \(f\) איזומורפיזם היא משמרת הכל, ובפרט משמרת נורמליות: 
אנו יודעים כי \(H\times G\) נורמלית ב-\(H\times K\). לכן:
$$H=f\left( H\times 1 \right)\trianglelefteq f\left( H\times K \right)=G$$
ו-\(K\) באותו אופן. 
מצד שני, אם \(H,K\trianglelefteq G\), נדרש להוכיח \(f\) הומומורפיזם(מ-1, 2נובע חח"ע ועל).
$$f\text{ הומומורפיזם}\iff f((h',k)(h,k'))\stackrel{?}{=} f((h',k))\cdot f(h,k')) \quad \forall h,h'\in H\quad \forall k,k'\in K$$
כאשר הצד הימיני שקול ל:
$$f((h'h,kk'))=h'hkk'\stackrel{?}{=} h'k \cdot hk'$$
כלומר \(f\) הומומורפיזם אם"ם:
$$\forall h\in H\quad k\in K\quad hk=kh \iff \forall h,k\quad \underbrace{ h\overbracket{ kh^{-1}k^{-1} }^{ \in H } }_{ \in H } =1$$
וקיבלנו: 
$$hkh^{-1} k^{-1}  \in H\cap K=1$$


\end{enumerate}
\begin{definition}[מכפלה ישרה פנימית]
אם \(H,K\) תתי חבורות של \(G\), \(G\) נקראת מכפלה ישרה פנימית של \(H\) ו-\(K\) אם מתקיים:

  \begin{enumerate}
    \item המכפלה זה הכל - \(HK=G\)


    \item החיתוך טריוויאלי - \(H\cap K=\{ e \}\)


    \item תתי חבורות נורמליות - \(H,K \trianglelefteq G\) 
ומסומן \(G=H\times K\). למשל \(\mathbb{C} ^\times =\mathbb{R} _{>0}\times S^1\).


  \end{enumerate}
\end{definition}
\begin{remark}
סימון זה יכול להראות מבלבל. \(G=H\times K\) יכול להתכוון או ל-\(H,K\) תתי חבורות של \(G\) ו-\(G\) המכפלה ישרה פנימית, או בתור \(G\) כמכפלה הישרה החיצונית של \(H\) ו-\(K\). אבל הראנו כי למעשה 2 אופציות אלו איזומורפיות.

\end{remark}
\begin{corollary}
אם \(G\) היא מכפלה ישרה פנימית(מי"פ) של \(G\geq H,K\) אז \(G\cong H\times K\). וכן מתקיים:
$$f^{-1} :G\xrightarrow{\cong}H\times K$$

\end{corollary}
\begin{example}
$$\begin{gathered}G=\mathbb{C} ^\times \quad H=\mathbb{R} _{>0}\quad K=S^1 \\f^{-1} \left( \alpha \right)=\left( |\alpha|, \frac{\alpha}{|\alpha|} \right)
\end{gathered}$$

\end{example}
\begin{proposition}
אם \(p\) ו-\(q\) הם גורמים ראשוניים זרים, אזי כל חבורה אבלית מסדר \(pq\) איזומורפיית לטבעיים מודולו \(pq\).

\end{proposition}
\subsection{מכפלה חצי ישרה}

ראינו כי אם יש שתי תתי חבורות נורמליות שהחיתוך שלהם טריוואילי והכפלה שלהם הכל נקבל כי החבורה איזומורפית למכפלה ישרה. לא כל החבורות כמובן מתקבלות בצורה הזו, ולכן נרצה למצוא דרכים לפרק חבורות נוספות. דרך אחד לעשות זאת זה עם מכפלה חצי ישרה.

\begin{definition}[מכפלה חצי ישרה פנימית]
אם \(H,K\leq G\) אזי \(G\) נקראת מכפלה חצי ישרה של \(H\) ו-\(K\) אם מתקיים:

  \begin{enumerate}
    \item \(HK=G\)


    \item \(H\cap K=\{ e \}\)


    \item \(K\trianglelefteq G\) 
מסומן: \(G=H\ltimes K\)


  \end{enumerate}
\end{definition}
\begin{proposition}
אם \(G=H\ltimes K\) אזי \(G / K \cong H\).

\end{proposition}
\begin{proof}
ממשפט האיזומורפיזם הראשון:
$$f:G\to H \quad \forall k\in K, h\in H\quad f(hk)=h$$
(כלומר לכל \(g\in G\) יש יצוג יחיד כ-\(hk\)) בברור, \(im(f)=H\), \(\ker(f)=K\). צריך להראות הומומורפיזם: 
$$\begin{gathered}f(hk)f(h'k')=hh'\\f(hkh'k')=f(hh'k''k')=hh'
\end{gathered}$$\textbf{דוגמא}$$\begin{gathered}D_{n}=\left\langle  \sigma \right\rangle\rtimes \left\langle  \tau \right\rangle \\S_{n}=A_{n} \rtimes \langle (1,2)\rangle\\\mathrm{Aff}\left( \mathbb{F}^n  \right)=GL_{n}\left( \mathbb{F} \right)\ltimes \mathbb{F}^n
\end{gathered}$$

\end{proof}
\begin{remark}
מכפלה חצי ישרה של שתי חבורות אבליות לאו דווקא תהיה אבלית.

\end{remark}
\subsection{חבורות פשוטות}

אם \(G\) חבורה, ו-\(N\trianglelefteq G\) אז רוצים לחשוב על \(G\) כ-"מורכבת" מ-\(N\) ו-\(G/N\). למשל:
$$\mathbb{Z} _{3} \cong A_{3}\trianglelefteq S_{3}\qquad S_{3} / A_{3} \cong \mathbb{Z} _{2}$$
הרעיון:
אם יש ל-\(G\) תת חבורה נורמלית \(N\neq G,1\), אפשר לעבור לחקור את \(N\) ו-\(G / N\). ולהמשיך כך עבורן: אם ל-\(N\) יש תת חבורה \(1\lneq N_{1}\triangleleft N\), ניתן "לפרק את \(N\)" ל-\(N_{1}\) ו-\(N / N_{1}\). אם ל-\(G / N\) יש תת חבורה \(1\lneq K \triangleleft G / N\). ועכשיו ניתן "לפרק את \(G / N\)".

\begin{definition}[חבורה פשוטה]
חבורה \(G\) נקראת פשוטה אם \(N\trianglelefteq G\) גורר \(N=G\) או \(N=1\) (ו-\(|G|\neq 1\))

\end{definition}
\begin{definition}[תוכנית הולדר/Holder]
ב-1892 לודויג אוטו הולדר הגה את המטרות הבאות בתורת החבורות:

  \begin{enumerate}
    \item למצוא את כל החבורות הסופיות הפשוטות 


    \item להבין את כל הדרכים להרכיב, בהנתון חבורות סופיות \(H_{1},H_{2}\). חבורה \(G\) עם תת חבורה \(N\trianglelefteq G\) כך ש-\(N\cong H_{1}\) 
 ו-\(G / N \cong H_{2}\).
 למשל עבור\(H_{1}=\mathbb{Z} _3\), \(H_{2}=\mathbb{Z} _3\):
$$\langle 2\rangle \trianglelefteq \mathbb{Z} _{6}\qquad A_{3}\trianglelefteq S_{3}\implies \langle 2\rangle \cong A_{3}\cong\mathbb{Z} _{3}\quad \mathbb{Z} _{6} / \langle 2\rangle \cong S_{3} / A_{3} \cong \mathbb{Z} _{2}$$
כיום סימנו את המטרה הראשונה של התוכנית, אשר לקחה כ-100 שנה.


  \end{enumerate}
\end{definition}
\begin{example}[חבורות פשוטות]
  \begin{enumerate}
    \item החבורה \(\mathbb{Z} _{p}\) ל-\(p\) ראשוני (אם \(G\) סופית פשוטה אזי \(G\cong\mathbb{Z} _p\) ל-\(p\) ראשוני) 


    \item התמורות הזוגיות \(A_{n}\) ל-\(n\geq 5\).(כאשר \(A_{4}\) היא חבורה נורמלית הנקראת חבורת קליין) 


    \item החבורה \(PSL_{n}\left( \mathbb{F}_{p} \right)\) ל-\(n\geq 2\) כאשר \(p\) ראשוני. (למעט \(PSL_{2}(2),PSL_{2}(3)\) ) כאשר \(PSL_{n}\left( \mathbb{F} \right)=SL_{n}\left( \mathbb{F} \right)/Z\) כאשר המרכז 
 $$Z=\{ \alpha I\in SL_{n}\left( \mathbb{F} \right)\mid \alpha^n=1 \}$$
 החבורה הכי קטנה תהיה: \(|PSL_{3}\left( \mathbb{F}_{2} \right)|=168\).


  \end{enumerate}
\end{example}
\subsection{פשטות התמורות הזוגיים}

נרצה להוכיח את הטענה הבאה:

\begin{proposition}
חבורת תמורות הזוגיות \(A_{5}\) פשוטה.

\end{proposition}
\begin{reminder}
מחלקת הצמידות של תמורה \(\sigma\) ב-\(S_{n}\) עם אותו מבנה מחזורים.
$$\tau\left( a_{1}, a_{2},\dots,a_{r} \right)\tau ^{-1} =\left( \tau (a_{1}), \tau(a_{2}),\dots,\tau(a_{r})\right)$$

\end{reminder}
\begin{proposition}
תת חבורה \(H\) של \(G\) היא נורמלית אם"ם איחוד של מחלקות צמידות.

\end{proposition}
\begin{proposition}
מחלקות הצמידות של \(\sigma\) ב-\(S_{n}\) היא או:

  \begin{enumerate}
    \item נשארת מחלקת צמידות ב-\(A_{n}\)


    \item מתפרקת לשתי מחלקות צמידות שווה לגודל ב-\(A_{n}\). 
בפרט עבור \(\sigma, \tau \in A_{n}\) עבור \(n\geq 2\) אם הם צמודות ב-\(A_{n}\)(זאת אומרת קיים \(\rho \in A_{n}\) כך ש-\(\rho \sigma \rho ^{-1} =\tau\) ) הן צמודות ב-\(S_{n}\)(כלומר \(\exists \rho \in S_{n}\quad \rho \sigma \rho ^{-1} =\tau\)). אבל לא בהכרח להיפך!


  \end{enumerate}
\end{proposition}
\begin{proof}
ממשפט מסלול מייצב:
$$|\prescript{A_{n}}{}{\sigma}|= \frac{|A_{n}|}{|C_{A_{n}}\left( \sigma \right)|}=\frac{|A_{n}|}{|A_{n}\cap C_{S_{n}}\left( \sigma \right)|}$$
כאשר ממשפט האיזומורפיזם השני:
$$\frac{|A_{n}|}{|A_{n}\cap C_{S_{n}}\left( \sigma \right)|}=\frac{|A_{n}\cdot C_{S_{n}}\left( \sigma \right)|}{|C_{S_{n}}|}$$
כאשר השתמשנו בזה ש:
$$H / H\cap N \cong HN / N \implies \frac{|H|}{|H\cap N|}= \frac{|HN|}{|N|}\implies \frac{|N|}{|H\cap N|}= \frac{|HN|}{|H|}$$
ניתן לפצל כעת למקרים - אם המרכז מורכב כולו מתמורות זוגיות וכאשר לא,
כלומר:
$$\frac{|A_{n}\cdot C_{S_{n}}\left( \sigma \right)|}{|C_{S_{n}}|}=\begin{cases}\frac{|A_{n}|}{|C_{S_{n}}\left( \sigma \right)|} & C_{S_{n}}\subseteq A_{n} \\\frac{|S_{n}|}{|C_{S_{n}}\left( \sigma \right)|} & else\end{cases}=\begin{cases}\frac{1}{2}|\prescript{S_{n}}{}{\sigma} & C_{S_{n}}\subseteq A_{n} \\\frac{|S_{n}|}{| \prescript{S_{n}}{}{\sigma} |} & else
\end{cases}$$

\end{proof}
כעת נחזור להוכיח את הטענה ש-\(A_{5}\) פשוטה.

\begin{proof}
רעיון: נמצא את מחלקות הצמידות, ונראה שאין איחוד של חלק מהן שמהווה תת חבורה למעט \(e,A_{5}\). נתחיל במחלקות הצמידות של \(S_{5}\)

  \begin{table}[htbp]
    \centering
    \begin{tabular}{|cccccccc|}
      \hline
      \(\sigma\) & \(()\) & \((12)\) & \((123)\) & \((1234)\) & \((12345)\) & \((12)(34)\) & \((12)(345)\) \\ \hline
      \(\sigma\) & 1 & 10 & 20 & 30 & 24 & 15 & 20 \\ \hline
      \(C_{S_5}\) & 120 & 12 & 6 & 4 & 5 & 8 & 6 \\ \hline
      \(C_{S_5}(\sigma)\) & \(S_5\) &  & \(\langle (123), (45) \rangle \cong \mathbb{Z}_3 \times \mathbb{Z}_2\) &  & \(\langle (12345) \rangle\) &  &  \\ \hline
      \(A_5\sigma\) & 1 &  & 20 &  & 12 &  &  \\ \hline
    \end{tabular}
  \end{table}
\end{proof}
כאשר ניתן לפסול את העמודות האי זוגיות של התמורות האי זוגיות. סך הכל
מחלקות הצמידות ב-\(A_{5}\) הם: $$\begin{matrix}()  & (123) & (12345) & (21345) & (12)(34) \\1 & 20 & 12 & 12 & 15
\end{matrix}$$ תהי \(1\nleq N \triangleleft A_{5}\). אזי הגודל של \(N\) שווה
לסכום של חלק מבין \(20,12,12,15\) ועוד אחד, ולכן \(|N|\geq 13\). מצד שני, \(|N|\big|60\) מלגרנג': $$N=\cancel{ 60 },30,20,15,\cancel{ 12,10,6 }$$
כאשר פסלנו את האיברים שקטנים מ-13 כאשר 30,20,15 נפסלות כיוון שלא ניתן להגיע עליהם בעזרת סכימה של צירוף של האיברים \(15,12,12,20\). ואין פתרון. כלומר הוכחנו כי \(A_{n}\) פשוטה(אם \(N\trianglelefteq A_{5}\) אז \(N=A_{5}\) או \(N=1\)).

\begin{proposition}
התמורות הזוגיים \(A_{n}\) נוצרת ע"י ה-3 מחזורים ב-\(S_{n}\) עבור \(n>3\).

\end{proposition}
\begin{proposition}
ב-\(A_{n}\) כל ה-3 מחזורים צמודים(עבור \(5\leq n\)).

\end{proposition}
\begin{proof}
כל ה-3 מחזורים ב-\(S_{n}\) הם צמודים, וראינו ששני מחלקות צמידות
של \(\sigma\) נשארת מחלקת צמידות ב-\(A_{n}\) אם"ם \(\sigma\) זוגית,
ו-\(C_{S_{n}}\left( \sigma \right)\not\subseteq A_{n}\). עבורנו:
$$|\prescript{S_{n}}{}{\left( 1\;2\;3 \right)}|= \frac{n(n-1)(n-2)}{3}$$
ולכן:
$$|C_{S_{n}}\left( \left( 1\;2\;3 \right) \right)|= \frac{n!}{\frac{n(n-1)(n-2)}{3}}=3(n-3)!$$
לכן:
$$C_{S_{n}}\left( \left( 1\;2\;3 \right) \right)=\left\langle  \left( 1\;2\;3 \right) \right\rangle \times S_{\left\{  4,5,\dots,n  \right\}}=S_{4\dots n}\bigsqcup \left( 1\;2\;3 \right)\cdot S_{4\dots n}\bigsqcup \left( 1\;2\;3 \right)^2 S_{4\dots n}$$
כי הם מתחלפות. וזה המספר שחיפשנו. נשאר להחליט האם כל אלו זוגיות. אם \(n\geq 5\). לא: \(\left( 4\;5 \right)\in S_{4\dots n}\). לכן סיימנו.

\end{proof}
כעת נוכיח את הטענה כללית.

\begin{proposition}
חבורת התמורות הזוגיות \(A_{n}\) פשוטה עבור \(n\geq 5\).

\end{proposition}
\begin{proof}
יהי \(1\neq N\trianglelefteq A_{n}\)(\(n\geq 6\)). נחפש \(i,j,k,l,r\) כך ש-\(\tau \in H=A_{\{ i,j,k,l,r \}}\)(התמורות הזוגיות ב-\(S_{\{ i,j,k,l,r \}}\)). ואז נסתכל על \(H\cap N\trianglelefteq H\cong A_{5}\). \(H\cong A_{5}\) פשוטה, לכן או \(N\cap H=1\) (לא ייתכן כי \(\tau \in H\cap N\)) או ש-\(N\cap H=H\), ולכן \(A_{i,j,k,l,r}=H\leq N\). בפרט מכילה 3-מחזור(למשל \((i,j,k)\)). מכיוון ש-\(N\) נורמלית ב-\(A_{n}\), היא מכילה את כל הצמודים (ב-\(A_{n}\)) של \((i,j,k)\) כלומר את כל ה-3 מחזורים, מהטענה השנייה. ואז מטענה 1 נובע \(N=A_{n}\). נבחר \(id\neq \sigma \in N\). יש \(i\neq j\) כך ש-\(\sigma(i)=j\). נסתכל על:
$$\tau=\underbracket{ \sigma\left( x\;y\;z \right)\sigma ^{-1} }_{ \left( \sigma(x)\;\sigma(y)\;\sigma(z) \right) } \left( x\;y\;z \right)$$
ניקח \((x\;y\;z)=(i\;j\;k)\) כש-\(k\notin \left\{  i,j,\sigma(j)  \right\}\) ואז:
$$\tau=\left( \sigma(i),\sigma(j),\sigma(k) \right)(i,k,j)=\left( j, \sigma(j), \sigma(k) \right)(i,k,j)$$
ולכן \(\tau\) מזיזה 5 איברים לכל היותר: \(i,j,k,\sigma(j),\sigma(k)\) ו-\(\tau \neq id\) ולכן \(\tau(k)=\sigma(j)\neq k\)

\end{proof}
\section{סדרות הרכב ופתירות}

\subsection{סדרות הרכב}

\begin{definition}[סדרה נורמלית]
סדרה נורמלית של חבורה \(G\) היא סדרה מהצורה:
$$1=H_{0}\trianglelefteq \dots \trianglelefteq H_{r-2}\trianglelefteq H_{r-1}\trianglelefteq H_{r}=G$$
כלומר סדרה של תתי חבורות נורמליות כך שבסוף מקבלים את התת חבורה הטריווילאית.

\end{definition}
\begin{remark}
נשים לב כי תמיד קיים סדרה נורמלית עבור חבורה \(G\) כי בפרט החבורה הטריוויאלית 1 היא תמיד תת חבורה נורמלית, אך לא כל סדרה של תתי סדרות נורמלית תהיה סדרה נורמלית כיוון שלא תמיד תגיע ל-1. לדוגמא:
$$\dots \trianglelefteq 16\mathbb{Z}\trianglelefteq 8\mathbb{Z} \trianglelefteq 4\mathbb{Z}  \trianglelefteq \mathbb{Z}$$
כאשר לא נגיע ל-1 כי כל תת חבורה של \(\mathbb{Z}\) איזומורפית ל-\(\mathbb{Z}\).

\end{remark}
\begin{example}
$$1=\langle 0\rangle \trianglelefteq \overbracket{ \langle 4\rangle }^{ \cong \mathbb{Z} _{3} } \trianglelefteq \overbracket{ \langle 2\rangle  }^{ \cong \mathbb{Z} _{6} } \trianglelefteq \mathbb{Z} _{12}$$
וכן גם:
$$\langle 0\rangle \trianglelefteq \langle 4\rangle \trianglelefteq  \langle 4\rangle \trianglelefteq  \mathbb{Z} _{4}$$

\end{example}
\begin{definition}[סדרת הרכב]
סדרה נורמלית של \(G\) נקראת סדרת הרכב של \(G\) אם \(H_{i}\triangleleft H_{i+1}\) לכל \(i\). ואי אפשר להוסיף תת חבורה לסדרה.
להוסיף: עבור \(H_{i}\triangleleft H_{i+1}\) כלשהו למצוא\(H_{i}\triangleleft K\triangleleft H_{i+1}\). כאשר נשים לב כי \(\triangleleft\) מסמל נורמלית ממש, כלומר תת חבורה נורמלית שאינה שווה לחבורה.

\end{definition}
\begin{corollary}
ל-\(\mathbb{Z}\) אין סדרות הרכב:
$$\{ 0 \}=H_{0}\triangleleft H_{1}\triangleleft \dots \triangleleft H_{r-1}\triangleleft \mathbb{Z}$$
לכל סדרה נורמלית בלי חזרות \(H_{1}\cong\mathbb{Z}\) (תת חבורה לא טריוויאלית של \(\mathbb{Z}\)), ולכן אפשר להוסיף תת חבורה בינה לבין \(\{ 0 \}=H_{0}\).

\end{corollary}
\begin{corollary}
עבור \(S_{n}\) כאשר \(n\geq 5\) נקבל כי:
$$1\triangleleft A_{n}\triangleleft S_{n}$$
כיוון ש-\(A_{n}\) פשוטה.

\end{corollary}
\begin{proposition}
לחבורה סופית יש סדרת הרכב

\end{proposition}
\begin{proof}
אם \(G\) פשוטה, \(1\triangleleft G\) סדרת הרכב. אם \(G\) לא פשוטה, יש לה תת חבורה נורמלית. ניקח תת חבורה נורמלית מקסימלית ב-\(G\)(ביחס להכלה), נקרא לה \(H_{r}\), ונמשיך ממנה. התהליך יעצר כי \(G\) הינה סופית.

\end{proof}
\begin{proposition}
אם \(1=H_{0}\trianglelefteq \dots \trianglelefteq H_{r}=G\) סדרה נורמלית של \(G\) אזי היא סדרת הרכב אם"ם \(H_{i+1}/ H_{i}\) פשוטה לכל \(i\).

\end{proposition}
\begin{proof}
$$H_{i}=H_{i+1}\iff 1=H_{i+1} / H_{i}$$
קיים \(H_{i}\triangleleft \;?\triangleleft H_{i+1}\) אם"ם (ממשפט האיזומורפיזם הרביעי) יש תת חבורה:
$$1=H_{i }/ H_{i} \triangleleft \;?\triangleleft H_{i+1} / H_{i}$$
כלומר\(H_{i+1} / H_{i}\) לא פשוטה.

\end{proof}
\begin{definition}[גורמי הרכב]
המנות \(H_{i+1} / H_{i}\) נקראים גורמי ההרכב(composition factors) שלה.

\end{definition}
\begin{example}
$$1 \overset{ \mathbb{Z} _{2} }{ \triangleleft }\left\langle  \left( 1\; 2\right)(3\;4) \right\rangle \overset{ \mathbb{Z} _{2} }{ \triangleleft }V\overset{ \mathbb{Z} _{3} }{ \triangleleft } A_{9}\overset{ \mathbb{Z} _{2} }{ \triangleleft } S_{4}$$
כאשר מעל הסימון של תת חבורה נורמלית ממש אנו כותבים את החבורה שעליה הגורמי הרכב \(G_{i+1} / G_{i}\) איזומורפיות עלייה.

\end{example}
\begin{theorem}[ג'ורדן הולדר]
לכל סדרות ההרכב יש את אותם גורמי הרכב(לאו דווקא באותו סדר!)

\end{theorem}
\begin{example}
$$\begin{gathered}0\stackrel{\mathbb{Z} _{2} }{\trianglelefteq} \langle 6\rangle \stackrel{\mathbb{Z} _{2} }{\trianglelefteq} \langle 3\rangle \stackrel{\mathbb{Z} _{3} }{\trianglelefteq} \mathbb{Z} _{12}\\0\stackrel{\mathbb{Z} _{3} }{\trianglelefteq} \langle 4\rangle \stackrel{\mathbb{Z} _{2} }{\trianglelefteq} \langle 2\rangle \stackrel{\mathbb{Z} _{2} }{\trianglelefteq} \mathbb{Z} _{12}\\0\stackrel{\mathbb{Z} _{2} }{\trianglelefteq} \langle 6\rangle \stackrel{\mathbb{Z} _{3} }{\trianglelefteq} \langle 2\rangle \stackrel{\mathbb{Z} _{2} }{\trianglelefteq} \mathbb{Z} _{12}
\end{gathered}$$

\end{example}
\begin{proposition}
אם \(G\) אבלית סופית, אז גורמי ההרכב שלה הם:
$$\mathbb{Z} _{p},\mathbb{Z} _{p_{2}},\dots,\mathbb{Z} _{p_{r}}$$
עבור ראשוניים \(p_{i}\) שמקיימים: 
$$\prod_{i=1}^r p_{i}= |G|$$

\end{proposition}
\begin{proof}
כל גורם הרכב של \(G\) הוא מנה של תת חבורות של \(G\) ולכן אבלי, ובנוסף פשוט. חבורה אבלית פשוטה אם"ם \(\mathbb{Z} _p\) ל-\(p\) ראשוני אם"ם חבורה בלי תתי חבורות אם"ם כל איבר הוא יוצר. לגבי הגודל:
$$1=G_{0}\triangleleft \dots \triangleleft G_{r}=G$$
גורמי ההרכב הם \(Z_{p_{i}}\cong G_{i+1} / G_{i}\) לכל \(i\), ואז: 
$$\begin{gathered}|G|=|G_{r}|=|G_{r}:G_{r-1}|=|G_{r-2}|[G_{r-1}:G_{r-2}]= \\=\dots=\prod_{i=0}^r [G_{i+1}:G_{i}]=\prod_{i=0}^r |G_{i-1} / G_{i}|=\prod_{i=0}^r p_{i}
\end{gathered}$$

\end{proof}
\begin{proposition}
אם \(G\) סופית ו-\(N\trianglelefteq G\) אזי גורמי ההרכב של \(G\) הם איחוד(עם חזרות) של גורמי ההרכב של \(N\) ושל \(G / N\).

\end{proposition}
\begin{proof}
נבנה סדרת הרכב שמכילה את \(N\): נתחיל מ-\(1\trianglelefteq N\trianglelefteq G\). אם זו סדרת הרכב, סיימנו. אם לא, אפשר להוסיף לה חבורת ביניים עד שמקבלים סדרת הרכב:
$$\star \quad 1=G_{0}\triangleleft G_{1} \triangleleft \dots \triangleleft G_{k}=N\triangleleft G_{k+1}\triangleleft \dots \triangleleft G_{r} = G$$
כאשר גורמי ההרכב של \(N\) הם \(G_{i+1} / G_{i}\) ל- \(i=0,\dots,k-1\). נשים לב שממשפט האיזומורפיזם השלישי עבור \(k\leq i\):
$$(G_{i+1} / N) / (G_{i} / N) \cong G_{i+1} / G_{i}$$
ולכן \(\star\) סדרת הרכב, וגורמי ההרכב שלה איזומורפים ל-\(G_{i+1} / G_{i}\).(\(k\leq i\leq r-1\)).

\end{proof}
\begin{example}
עבור \(V\trianglelefteq S_{4}\). גורמי ההרכב של \(V\). \(\mathbb{Z} _2,\mathbb{Z} _2\), ושל \(S_{4} / V \cong S_{3}\): \(\mathbb{Z} _3,\mathbb{Z} _2\).

\end{example}
\begin{remark}
נשים לב כי: $$\begin{cases}N\trianglelefteq G \\N\leq H\leq G
\end{cases}\implies N\trianglelefteq  H$$ כי \(\forall h'\quad h'N=Nh\).

\end{remark}
\begin{corollary}
אם \(|G|=p^r\) כאשר \(p\) ראשונית, אז גורמי ההרכב שלה הם \(\overbrace{ \mathbb{Z}_{p},\dots,\mathbb{Z} _p }^{ r }\).

\end{corollary}
\begin{proof}
באינדוקציה שלמה על \(|G|\), ניקח \(N=Z(G)\), יודעים ש-\(1=N\)(מרכז
של חבורת \(p\)) אם \(N=G\) אז \(G\) אבלית והוכחנו בעבר. אחרת, \(p^k=|N|,p^l=|G / N|\) חבורות מסדר חזקה \(p\) קטנה מ-\(|G|\). לכן עבור \(k,l <r\), ומהאינדוקציה נקבל גורמי ההרכב שלה הם \(\overbrace{ \mathbb{Z} _p, \dots, \mathbb{Z} _p }^{ r }\), והטענה אכן מתקיימת. ולכן:
$$\underbrace{ \overbrace{ \mathbb{Z} _{p},\dots , \mathbb{Z} _{p} }^{ l }\quad \overbrace{ \mathbb{Z} _{p},\dots,\mathbb{Z} _{p} }^{ k } }_{ \text{G לש בכרה ימרוג} }$$\textbf{דוגמא}
יש חמישה חבורות מגודל 8 עד עדי איזומורפיזם -
$$\mathbb{Z} _{2}\times \mathbb{Z} _{4},\mathbb{Z} _{2}^3, \mathbb{Z} _{8},Q, D_{4}$$
וגורמי ההרכב של כל אחת מהם יהיו \(\mathbb{Z} _2,\mathbb{Z} _2,\mathbb{Z} _2\).

\end{proof}
\subsection{פתירות}

\begin{definition}[פתירות]
חבורה \(G\) נקראת פתירה אם יש לה סדרה נורמלית עם מנות אבליות.

\end{definition}
\begin{proposition}
חבורה \(G\) סופית היא פתירה אם"ם גורמה ההרכב שלה הם חבורות ציקליות
ראשוניות(כלומר רק \(\mathbb{Z} _p\) ל-\(p\) ראשוני)

\end{proposition}
\begin{proof}
הכיוון הראשון ברור, אם יש \(G\) כך שחבורות ההרכב שלה הם חבורות
ציקליות ראשוניות, אז \(G\) פתירה. בכיוון השני, נניח:
$$1=G_{0}\triangleleft \dots \trianglelefteq G_{r}=G$$
סדרת נורמלית עם מנות אבליות, אפשר להמשיך להוסיף לה איברים עד שמגיעים לסדרת הרכב. אם \(G_{i}\triangleleft \dots \triangleleft H'\triangleleft \dots \triangleleft G_{i+1}\) אזי:
$$G_{i}\triangleleft \dots \triangleleft H\triangleleft H'\triangleleft \dots \triangleleft G_{i+1}$$
נלך למנה ב-\(G_{i}\):
$$1=G_{i} / G_{i} \triangleleft \dots \triangleleft H / G_{i}\triangleleft \dots G_{i+1} / G_{i}$$
ואז \(H' / H \cong (H' / G_{i}) / (H / G_{i})\cong H' / H\). כאשר מנה של תת חבורה של \(G_{i+1} / G_{i}\) אבלית. לכן גורמי ההרכב אבלים ולכן הם \(\mathbb{Z} _p\).

\end{proof}
\begin{proposition}
אם \(G\) פתירה ו-\(H\leq G\) אז \(H\) פתירה

\end{proposition}
\begin{proof}
נבחר \(1=G_{0}\triangleleft \dots \triangleleft G_{r} = G\). עם \(G_{i+1} / G_{i}\) אבלית. נחתוך עם \(H\):
$$1=H_{0}\triangleleft \dots \triangleleft H_{r}=H$$
נגדיר \(H_{i}=H\cap G_{i}\). צריך להראות ש-\(H_{i}\) נורמלית, \(H_{i}\trianglelefteq H_{i+1}\) והמנה אבלית.
$$H_{i}=H\cap G_{i}=H\cap G_{i+1}\cap G_{i}=H_{i+1}\cap G_{i}\trianglelefteq H_{i+1}$$
וניתן להשתמש במשפט האיזומורפיזם השני:
$$H_{i+1} / H_{i} = H_{i+1} / \left( H_{i+1}\cap G_{i} \right)\cong (G_{i} H_{i+1}) / G_{i} \leq G_{i+1} / G_{i}$$

\end{proof}
\begin{proposition}
אם \(N\trianglelefteq G\) אזי \(G\) פתירה אם"ם \(N\) ו-\(G / N\) פתירות.

\end{proposition}
\begin{proof}
אם \(G\) פתירה אז \(N\) פתירה(גם ל-\(N\leq G\), הוכחנו). נניח ש-\(N\) ו-\(G / N\) פתירות ונוכיח ש-\(G\) פתירה. ניקח סדרה נורמלית:
$$1=N / N = G_{0} / N \trianglelefteq  \dots \trianglelefteq G_{r-1} / N \trianglelefteq  G_{r} / N = G / N$$
עם מנות אבליות(אנחנו יודעים שתתי חבורות של \(G / N\) הן מהצורה \(H / N\) ל-\(N\leq H\leq G\)). ממשפט ההתאמה, \(G_{i}\triangleleft G_{i+1}\) מאיזו 3:
$$\overbracket{ (G_{i+1} / N) / (G_{i} / N) }^{ abelian! } \cong G_{i+1} / G_{i}$$
ניקח סדרה נורמלית \(\{ N_{i} \}_{i=0}^l\) ל-\(N\) עם מנות אבליות, ונקבל סדרה נורמלית עם מנות אבליות. כעת נראה \(G\) פתירה גורר \(G / N\) פתירה. נבחר סדרה
נורמלית עם מנות אבליות: 
$$\begin{aligned}1.\quad &1=G_{0}\triangleleft G_{1} \triangleleft \dots G_{r}= G \xRightarrow{\times N}\\2.\quad &1=N<G_{1}N<\dots <G_{r}N=G \xrightarrow{ \div N} \\3.\quad &1=N / N < G_{1} N / N<\dots < G_{r}N / N = G / N
\end{aligned}$$
נראה שהמעבר 2 הוא סדרה נורמלית עם מנות אבליות(סנעמ"ב). מתקיים \(G_{i}N\trianglelefteq G_{i+1}N\) כי: 
$$\begin{gathered}\forall g \in  G_{i+1}\quad  gG_{i}N\stackrel{G_{i} \triangleleft G_{i+1}}{=} G_{i}gN\stackrel{N\triangleleft G}{=} G_{i}Ng \\\forall n \in N\quad nG_{i}N \stackrel{N\triangleleft G}{=} = NG_{i}N = G_{i}Nn
\end{gathered}$$
ועכשיו:
$$G_{i+1}N / G_{i}N= G_{i+1 }G_{i}N / G_{i}N \cong G_{i+1} /\left(  G_{i+1}\cap G_{i}N \right) \cong (G_{i+1} / G_{i}) / \left( G_{i+1} \cap G_{i}N \right) / G_{i}$$
כעת נראה שמעבר 3 היא סנעמ"א. ממשפטי האיזו הרביעי(זו סדרה נורמלית) והשלישי (עם אותן מנות כמו 2).

\end{proof}
\subsection{נילפוטנטיות}

\section{משפטי סילו}

\subsection{חבורות p סילו}

\begin{definition}[חבורת \(p\) סילו]
תהי \(G\) חבורה סופית ו-\(p\) מספר ראשוני. נכתוב \(|G|=p^r\cdot m\) עם \(p\nmid m\). (כאשר נסמן \(p^r \mid \mid |G|\)). תת חבורה \(H\leq G\) נקראת חבורת \(p\)-סילו של \(G\) אם \(|H|=p^r\).

\end{definition}
\begin{example}
  \begin{enumerate}
    \item אם \(|G|=p^r\) אזי \(G\) עצמה(ורק היא) חבורה \(p\) סילו שלה. 


    \item עבור \(G=S_{4}\) נקבל \(D_{4}\) חבורת 2-סילו. \(\langle (123)\rangle\) חבורת 3-סילו, \(\langle id\rangle\) חבורת 5-סילו(\(24=1\cdot 24=5^0\cdot 24\)). 


  \end{enumerate}
\end{example}
\begin{symbolize}
נסמן \(Syl_{p}(G)\) - קבוצת תתי החבורת ב-\(p\) סילו של \(G\). כאשר נגדיר \(k_{p}:=|Syl_{p}(G)|\).

\end{symbolize}
\begin{theorem}[סילו 1]
יש חבורת \(p\) סילו.

\end{theorem}
\begin{proof}
נסתכל על \(X=\left\{  S\subseteq G  \mid |S|=p^r\right\}\). \(G\circlearrowright X\) ע"י כפל משמאל -\(g.S=\left\{  gs\mid s\in S  \right\}\). ראשית, נראה כי \(p\nmid |X|\).
מתקיים: 
$$\begin{pmatrix}p^r m\\p^r 
\end{pmatrix}=\frac{p^r m\cdot  (p^r m - 1) \cdot \dots\cdot (p^r m - p^r - 1)}{p^r (p^r-1)\cdot \dots\cdot (p^r-p^r+1)}$$
נראה ש-\(p\) מחלק את \(p^r m - j\) ואת \(p^r - j\) אותו מספר פעמים. לכל \(0\leq j<p^r\). נכתוב \(j=p^{r'}\cdot m'\) עם \(p\nmid m'\). נשים לב ש-\(r'<r\). ואז: 
$$\begin{gathered}p^r m -j = p^r m - p^{r'}m' = p^{r'}(p^{r-r'}m-m') \\p^r - j = p^r - p^{r'}m'=p^{r'}(p^{r-r'}-m')
\end{gathered}$$
מ-\(p \nmid|X|\) נובע שיש מסלול \(O(s)\) עם \(p \nmid |O(s)|\). ניקח \(P=Stab_{G}(S)\). ממשוואת מסלול מייצב:
$$|G|=|P|\cdot |O(s)|\implies p^r ||P|$$
מצד שני, \(S\) הוא איחוד של קוסטים של \(P\): 
$$S=PS=\bigsqcup_{i=1}^k P_{S_{i}}$$
לנציגים \(S_{1},\dots,S_{k}\) כלשהם, ולכן \(k\cdot |P|=p^r\). ולכן \(|P|\) חזקה של \(p\).

\end{proof}
\begin{theorem}[סילו 2]
כל חבורות \(p\) סילו צמודות זו לזו. כלומר \(Syl_{p}(G)\) מהווה מסלול בפעולת \(G\) על תתי החבורות שלה ע"י הצמדה.

\end{theorem}
\begin{proof}
נראה טענה קצת חזקה יותר: אם \(P \in Syl_{p}(G)\) ו-\(H\leq G\) אזי קיים \(g \in G\) כך ש-\(H\cap gPg^{-1}\in Syl_{p}(H)\). טענה זו תסיק את המשפט \(II\) של סילו כיוון ש: תהאנה \(P,P'\) חבורות \(p\) - סילו של \(G\). ניקח \(H=P'\). קיים \(g \in  G\) כך ש-\(P'\cap gPg'\) תת חבורה \(p\) סילו של \(P'\). וזה גורר:
$$P'\cap gP g^{-1} =P' \implies gPg^{-1}  \subseteq P' \implies gPg^{-1} =P'$$
כי גדליהן שווים. תובנה כללית: \(H,K\leq G\).\(K\leq G\circlearrowright G / H\). מעניין לחקור את \(K\circlearrowright G / H\). תרגיל - להראות ש- \(Stab_{k}(gH)=K\cap gHg^{-1}\) כאשר המסלולים נקראים קוסטים כפולים(\(\text{Double Cosets}\)). נחזור להוכיח את המשפט: נסתכל על \(H \circlearrowright G / P\). כאשר \(p\nmid m\) ולכן \(p\nmid |G / P|\) יש מסלול \(O(gP)\) כך ש:
$$p\nmid |O(gP)|\implies p\nmid [H:Stab_{H}(gP)]\implies p\nmid \frac{|H|}{|H\cap gPg^{-1} |}$$
כלומר אם \(|H|=p^{r'}m'\)(כאשר \(p\nmid m'\) ) אזי \(p^{r'}\mid|H\cap gPg^{-1} |\). מצד שני, \(|H\cap gPg^{-1} |\mid|gPg^{-1} |\) מלגרנג' כאשר \(|gPg^{-1} |\) חזקה של \(P\). ולכן \(|H\cap gPg^{-1} |\) חזקה של \(P\). ומשתי טענות אלו נובע המשפט.

\end{proof}
\begin{corollary}
חבורת \(p\) סילו היא נורמלית אם"ם היא החבורת \(p\) סילו היחידה.

\end{corollary}
\begin{definition}[נילפוטנטיות]
חבורת סופית שכל החבורות ה-\(p\) סילו שלה הן נורמליות נקראת נילפוטנטית. אם"ם \(k_{p}=1\) לכל ראשוני \(p\). כל תת חבורת \(p\) של \(G\) מוכלת בחבורת \(p\) סילו.

\end{definition}
\begin{remark}
משפט סילו זה סוג משפט הפוך של משפט לגרנג'.

\end{remark}
\begin{theorem}[סילו 3]
$$k_{p}=|Syl_{p}(G)|\implies \begin{cases}k_{p}\mid m \\k_{p} \equiv 1\mod p
\end{cases}$$

\end{theorem}
\begin{proof}
תהי \(P \in Syl_{p}(G)\). מתקיים:
$$k_{p}=|O(p)|=[G:N_{G}(P)]=\frac{|G|}{|N_{G}|}$$
כאשר המעבר הראשון נובע מסילו \(II\) ומעבר שני ממשפט מסלול מייצב במסלול של \(G\) ע"י הצמדה. כעת:
$$Stab(P)=\left\{  g\in G\mid gPg^{-1} =P  \right\}=N_{G}(P)\geq P\xrightarrow{lagrange} p^r \mid N_{G}(P)\implies \frac{|G|}{|N_{G}(P)|}\mid m$$
כעת נסתכל על הפעולה של \(P\) על \(Syl_{p}(G)\) ע"י הצמדה. המסלול של \(P\) הוא \(\{ P \}\). נראה שאם \(P\neq P'\in Syl_{p}(G)\) אזי \(1\neq|O(P')|\) ומזה נובע \(P\mid|O(P')|\) כי ממשפט מסלול מייצב. אבל המסלול מחלק את גודל החבורה הפועלת, \(p^r=|P|\) ומכאן נובע
$$k_{p}=|Syl_{p}(G)|=\sum_{orbits}|O(H)|\implies k_{p}\equiv 1\mod p$$
כעת נניח בשלילה ש-\(O(P')=\{ P' \}\). אזי \(gP'=P'g\) לכל \(g \in P\). לכן \(PP'=P'P\) וזה מתקיים אם"ם \(PP'\leq G\)(תרגיל). וכן:
$$|PP'|=\frac{|P|\cdot |P'|}{|P\cap P'}=\frac{p^{2r}}{|P\cap P'|}$$
נכתוב \(|P\cap P'|=p^{2r-k}\). ולכן \(0\leq k\leq r\). מכיוון ש:
$$\begin{cases}|PP'|=p^{2r-k} \\|PP'|\mid p^r m
\end{cases}\implies r\leq k\implies P=P'$$
וקיבלנו סתירה.

\end{proof}
\subsection{פתירות חבורות מסדר קטן מ-60}

\begin{reminder}
חבורה סופית היא פתירה אם"ם גורמי ההרכב שלה הם ציקלים.

\end{reminder}
\begin{proposition}
תהי \(G\) חבורה בגודל \(|G|<60\) אזי \(G\) פתירה.

\end{proposition}
מספיק להוכיח שכל חבורה מוגדל קטן מ-\(60\) היא ציקלית מסדר ראשוני.(כי גורמי ההרכב של חבורה חסומים בגודלה - לא יתכן גורם הרכב מסדר גדול יותר) ולכן מספיק להראות את הטענה הבאה

\begin{proposition}
אם \(|G|<60\) ו-\(|G|\) לא ראשוני אז יש לה ת"ח נורמלית לא טריוויאלית.

\end{proposition}
\begin{proof}
מקרה \(I\): \(|G|=p^n\) כאשר \(p\) ראשוני, \(n\geq 2\). זה אומר שחבורה \(p\) היא פתירה, ולכן יש:
$$1=G_{0}\triangleleft \dots \triangleleft G_{r-1}\triangleleft G_{r}=G$$
כאשר המנה איזומורפית ל-\(\mathbb{Z} _p\), ואיננה "פשוט לא ציקלית".

\end{proof}
מקרה \(II\): \(|G|=p^nq\) כאשר \(p\neq q\) ראשוניים. \(p^n<q\). למשל \(6,20,51\). נסתכל על \(k_{q}\). אנו יודעים כי: 
$$\begin{cases}k_{q}\mid p^n\implies k_{q}\in \left\{  1,p,p^2,\dots,p^n  \right\} \\k_{q}\equiv 1\mod q \implies k_{q}\in \{1, 1+q, 1+2q  \}
\end{cases}\implies k_{q}=1$$
ולכן חבורת \(q\) סילו היא נורמלית.
מקרה \(III\): כאשר \(|G|=p^mq^n \nmid(p^n)!\) למשל \(\left( 45\nmid 5!, 12\nmid 3! \right)\). נסתכל בפעולה של \(G\) בכפל על \(G / Q\) כש-\(Q\in Syl_{q}(G)\). נסתכל על:
$$\rho:G\to Sym_{G / Q}\cong S_{p^n}\qquad \ker\left( \rho \right)=Core_{G}(Q)=\bigcap_{g\in G}gQg^{-1}$$
נשים לב ש-\(Core_{G}(Q)\leq Q\), ואם \(Core_{G}(Q)=1\) אזי \(\rho\) שיכון, כלומר \(G\) איזומורפית שתת חבורה של \(S_{p^n}\) בסתירה ללגרנג'.
כעת נשאר להראות: \(30,40,42,56\). נראה בנפרד.
$$\begin{gather}|G|=40=8\cdot 5\implies \begin{cases}k_{5}\equiv 1 \mod 5\\k_{5}\mid 8\end{cases}\implies k_{5}=1\\|G|=42=2\cdot 3\cdot 7\implies \begin{cases}k_{7} \equiv 1 \mod 7 \\k_{7} \mid 6\end{cases}\implies k_{7}=1\\|G|=56=8\cdot 7\implies \begin{cases}k_{7}=1 \mod 7 \\k_{7} \mid 6\end{cases}\implies k_{7}=1 
\end{gather}$$
עבור \(k_{7}=1\) סיימנו. עבור \(k_{7}=8\) נשים לב ששתי חבורות \(7\) סילו נחתכות ב-\(e\) (לגרנג') לכן ב-\(G\) יש \(8\cdot 6=48\) איברים מסדר \(7\). ב-\(G\) יש \(56-48=8\) איברים שאינם מסדר \(7\) ולכן \(k_{2}=1\). נותן לבדוק רק את \(30\). נניח \(|G|=30=2\cdot 3\cdot 5\). נניח בשלילה ש-\(G\) פשוטה. ולכן \(1<k_{2},k_{3},k_{5}\). לכן: 
$$\begin{cases}k \equiv 1 \mod 5 \\k_{5} \mid 6
\end{cases}\implies k_{5}=6$$
ולכן יש \(24\) איברים מסדר \(5\) ב-\(G\).
$$\begin{cases}  k_{3} \equiv 1 \mod 5\\k_{3}\mid 10  
\end{cases}\implies k_{3}=10$$
ויש \(20\) שאיברים מסדר \(3\) ב-\(G\). וכיוון ש-\(24+20=44\) קיבלנו סתירה.

\section{חבורות חופשיות}

\subsection{מילים}

אנו יודעים כי \(D_{n}=\left\langle  \sigma,\tau \right\rangle\). לכן כל איבר ב-\(D_{n}\) ניתן לכתוב בעזרת צירוף שלהם. אפשר לחשוב על זה כמו מילה שניתן לכתוב בעזרת \(\sigma\) ו-\(\tau\).

\begin{definition}
מילה ב-\(S\) היא מחרוזת מהצורה \(s_{1}^{\varepsilon_{1}} s_{2}^{\varepsilon_{2}} \dots s_{r}^{\varepsilon_{r}}\) כש-\(r\in \mathbb{N} \cup \{ 0 \}\), \(s_{i}\in S\) ו-\(\varepsilon_{i}\in \left\{  \pm 1  \right\}\). למשל עבור \(S=\{ a,b,c \}\).

\end{definition}
\begin{example}
$$abacccba^{-1} b^{-1} \qquad a^5b^{-1} abc^7$$

\end{example}
\begin{symbolize}
המילה הריקה מסומנת \(\varnothing,e,1\).

\end{symbolize}
\begin{definition}[מילה מצומצמת]
מילה \(s_{1}^{\varepsilon_{1}} s_{2}^{\varepsilon_{2}} \dots s_{r}^{\varepsilon_{r}}\) נקראת מצומצמת אם אין בה מופעים של \(ss ^{-1}\) או \(s ^{-1} s\). פורמלית:
$$\forall i\quad s_{i+1}^{\varepsilon_{i+1}}\neq s_{i}^{-\varepsilon_{i}}$$\textbf{דוגמא}$$abaa ^{-1} cb$$
לא מצוצמת. הצמצום שלה הוא מחיקת האיבר \(aa^{-1}\). כלומר \(abcb\).

\end{definition}
\begin{definition}[קבוצת המילים המצומצמות]
קבוצת המילים המצומצמת ב-\(S\) מסומנת \(F(S)\). 

\end{definition}
\begin{proposition}
עבור קבוצה \(S\) קבוצת המילים המצומצמות \(F(S)\) יוצרת חבורה("החבורה החופשית על \(S\)") כשהכפל הוא שרשור וצמצום(צמוצום=מחיקת \(ss^{-1}\) או \(s ^{-1} s\) מהמילה). חבורה זו נקראת החבורה החופשית של \(S\) כאשר הקבוצה הריקה \(\varnothing\) ניטרלית. וההופכי מוגדר על ידי:
$$\left( \prod_{i=1}^r s_{i}^{\varepsilon_{i}} \right)^{-1} = s_{r}^{-\varepsilon_{r}}\dots s_{r}^{-\varepsilon _{1}}=\prod_{i=r}^1 s_{i}^{-\varepsilon_{i}}$$

\end{proposition}
\begin{proof}
איבר ניטרלי: המילה הריקה \(\varnothing\)
הופכי: כל שיקוף של איבר עם שינוי סימן. לדוגמא:
$$\left( s_{1}^{\varepsilon_{1}}\dots s_{r}^{\varepsilon_{r}} \right)^{-1} =s_{r}^{-\varepsilon_{r}}s_{r-1}^{-\varepsilon_{r-1}}\dots s^{-\varepsilon_{1}}$$
רעיון עבור אסוצייטיביות: שיכון של \(F(s)\) עם הכפל שהגדרנו בתוך \(Sym_{F(s)}\) ע"י \(s^{\varepsilon}\mapsto \sigma_{s^{\varepsilon}}\) שמוגדרת:
$$\sigma_{s^{\varepsilon}}\left( s_{1}^{\varepsilon_{1}}\dots s_{r}^{\varepsilon_{r}} \right)=\begin{cases}s^{\varepsilon}s_{1}^{\varepsilon_{1}}\dots s_{r}^{\varepsilon_{r}} & s^{\varepsilon}\neq s_{1}^{-\varepsilon_{1}} \\s_{2}^{\varepsilon_{2}}\dots s_{r}^{\varepsilon_{r}} & s^{\varepsilon}=s_{1}^{-\varepsilon_{1}}
\end{cases}$$

\end{proof}
\begin{remark}
ב-\(F(S)\), ל-\(s \in S\) מתקיים \((s)^{-1} =s ^{-1}\).

\end{remark}
\begin{example}
$$(abbc^{-1} a)(a^{-1} cbba)=abb\cancel{ c^{-1} aa^{-1} c }bba=abbbba$$

\end{example}
\begin{corollary}
  \begin{itemize}
    \item אם \(|S|=|T|\) אז \(F(S)\cong F(T)\). לכן מסמנים \(F_{n}\) עבור \(F_{S}\) עם  \(|S|=n\).
    \item לכן \(F_{1}\cong\mathbb{Z}\cong F_{\{ a \}}=\left\{  a^n\mid n\in \mathbb{Z}\right\}\).
  \end{itemize}
\end{corollary}
\begin{proposition}
אם \(S\) קבוצה, ל-\(F(S)\) יש את התכונה הבאה: לכל חבורה \(T\) ולכל פונקציה \(f:S\to T\) ישנו הומומורפיזם יחיד \(\tilde{f}:F(s)\to T\) שמרחיב את \(f\) (\(\tilde{f}|_{S}=f\)).

\end{proposition}
\begin{proof}
ניקח \(\tilde{f}\left( s_{1}^{\varepsilon_{1}}\dots s_{r}^{\varepsilon r} \right)=f(s_{1})^{\varepsilon_{1}}\dots f(s_{r})^{\varepsilon_{r}}\) כדי לקבל הומומורפיזם שמרחיב את \(f\) מכאן יחידות. מצד שני, \(\tilde{f}\) הזו
אכן מוגדרת היטב כי מילים מצוצמות שונות הם איברים שונים ב-\(F(s)\). נשאר להראות ש-\(\tilde{f}\) היא הומומורפיזם: נסמן:
$$w=s_{1}^{\varepsilon_{1}}\dots s_{r}^{\varepsilon_{r}} \qquad u={s'}_{1}^{\varepsilon_{1}'}\dots {s'}_{1}^{\varepsilon_{1}'}$$
נניח ש-\(wu=s_{1}^{\varepsilon_{1}}\dots s_{r-j}^{\varepsilon_j}{s'}_{{j+1}}^{\varepsilon'_{j+1}}\dots {s'}_{l}^{\varepsilon'_{l}}\) וכעת: 
$$\begin{gathered}\tilde{f}(wu)=f(s_{1})^{\varepsilon_{1}}\dots f(s_{r-j})^{\varepsilon_{j}} f({s'}_{j+1})^{\varepsilon'_{j+1}}\dots f({s'}_{l})^{\varepsilon'_{l}}= \\=f(s_{1})^{\varepsilon_{1}}\dots f(s_{r})^{\varepsilon_{r} }f({s'}_{1})^{\varepsilon'_{1}}\dots f({s'}_{l})^{\varepsilon'_{l}}=f(w)f(u)
\end{gathered}$$

\end{proof}
\begin{proposition}
כל חבורה היא מנה של חבורה חופשית.

\end{proposition}
\begin{proof}
תהי \(G\) חבורה. נבחר קבוצה נוצרת \(S\), וניקח \(f:S\to G\). לפי התכונה האוניברסלית \(f\) מתרחבת \(f(s)\to s\) להומומורפיזם \(\iota:F(s)\to G\). כאשר \(\iota\) על כיוון ש-\(S\subseteq Im(f)\). לכן \(G\cong F(s) / \ker\left( \iota \right)\). באופן יותר כללי, עבור \(S\subseteq G\) ניקח \(\iota:S\to G\) ההכלה אז \(\tilde{\iota}:F(S)\to G\) על אם"ם \(\langle S\rangle=G\). זאת כי:
$$\tilde{\iota}\left( \prod s_{i}^{\varepsilon_{i}} \right)=\prod s_{i}^{\varepsilon_{i}}$$
כאשר \(\iota\) חח"ע אם"ם \(\ker\left( \tilde{\iota} \right)=1\). כלומר המילה המצומצמת היחידה ב-\(S\), שמתרגמת ב-\(G\) ל-\(e\) היא המילה הריקה. \(\tilde{\iota}\) חח"ע אם לכל איבר ב-\(G\) יש יצוג יחיד(לכל היותר) כמילה מצומצמת ב-\(S\).

\end{proof}
\begin{proposition}
עבור חבורה \(G\) ו-\(S\subseteq G\) הבאים שקולים.

  \begin{enumerate}
    \item לכל \(g\in G\) יש יצוג יחיד כמילה מצומצמת ב-\(S\). 


    \item הפונקציה \(\iota:F(S)\to G\) היא חח"ע ועל(ובפרט \(F(s)\cong G\)). 


    \item לכל חבורה \(T\) ופונקציה \(f:S\to T\) יש הרחבה יחידה להומומורפיזם  \(\tilde{f}:G\to T\). 
כשהטענה מתקיימת, אומרים ש-\(G\) היא חבורה חופשית עם בסיס \(S\).


  \end{enumerate}
\end{proposition}
\begin{remark}
הטענות האלו מאוד מזכירים לנו טענות מקבילות לאלגברה לינארית. \(1\) מזכיר שלכל וקטור יש יצוג יחיד ע"י איברי הבסיס. \(2\) מזכיר שאיברי הבסיס הוא בת"ל ועל. \(3\) מזכיר שניתן להשלים אוסף וקטורים לבסיס.

\end{remark}
\begin{example}
  \begin{enumerate}
    \item החבורה \(D_{4}=\left\langle  \sigma,\tau  \right\rangle\) לא חופשית עם בסיס \(\left\{  \sigma,\tau  \right\}\) (כאשר אין יצוג יחיד \(\sigma^5=\sigma\)). 


    \item עבור \(S=\{ a,b \}\subseteq F(a,b,c)\) כאשר \(\langle S\rangle\) הם כל במילים שאין בהן \(c\). \(\langle S\rangle\) היא חופשית עם בסיס \(\{ a,b \}\). 


    \item עבור \(S=\{ a,bab^{-1} ,b^2ab^{-2} \}\). ניקח \(G=\langle S\rangle \leq\langle F(a,b)\rangle\). \(G\) חופשית עם בסיס \(S\). בפרט \(G\cong F_{3}\). 


  \end{enumerate}
\end{example}
\subsection{הצגות של חבורות}

נסתכל על:
$$D_{n}\cong \prescript{F\left( \sigma,\tau \right)}{}{/\ker\left( \tilde{\iota}:F\left( \sigma,\tau \right)\to D_{n} \right) }$$
כדי להבין את \(F\) נדרש להבין את הגרעין. הגרעין פה מכיל איברים כמו \(\sigma^n,\tau^2,\left( \sigma \tau \right)^2,\tau^8\).

\begin{definition}
עבור \(S\subseteq G\), התת חבורה הנורמלית שנוצרת ע"י \(S\) היא:
$$\langle S\rangle = \bigcap_{S\subseteq N\trianglelefteq G} N=\left\{  \prod_{i=1}^r gs_{i}^{\varepsilon_{i}}g^{-1} \mid s_{i}\in S\;\;\varepsilon_{i}\in \left\{  \pm 1  \right\}\;\;r\in \mathbb{N} \;\;g\in G  \right\}$$
נחפש \(S\) סופית כך ש:
$$\ker\left( \tilde{\iota} \right) = \left\langle  \langle S\rangle  \right\rangle$$
כאשר עבור \(D_{n}\):
$$\ker\left( \tilde{\iota} \right)=\left\langle  \left\langle  \tau^n, \sigma^n , \left( \tau \sigma \right)^2\right\rangle  \right\rangle$$

\end{definition}
\begin{example}
$$G=D_{n}=\left\langle  \sigma,\tau \right\rangle$$$$\tilde{\iota}:F\left( \sigma,\tau \right)\to D_{n}$$$$\ker\left( \tilde{\iota} \right)=\left\{  \sigma^n,\tau^2, \sigma^{2n}, \left( \sigma \tau \right)^2,\dots  \right\}$$

\end{example}
\begin{definition}[יוצרים ויחסים]
אם \(S\) יוצרת את \(G\), איברי \(S\) נקראים יוצרים (\(generators\)) ואיברי \(\ker\left( \tilde{\iota}:F(s)\to G \right)\) נקראים יחסים(\(relators / relations\)). למשל "\(\sigma \tau \sigma \tau\) הוא יחס ב-\(D_{n}\)", "\(\tau \sigma \tau=\sigma ^{-1}\)" הוא יחס ב-\(D_{n}\).

\end{definition}
\begin{example}
המילה "\(\sigma \tau \sigma \tau\) הוא יחס ב-\(D_{n}\)", ו-"\(\tau \sigma \tau=\sigma ^{-1}\)" הוא יחס ב-\(D_{n}\).

\end{example}
אנחנו הגדרנו עבור \(S\subseteq G\) את:
$$\left\langle  \langle S\rangle  \right\rangle =\bigcap_{S\subseteq N\trianglelefteq G}N=\left\langle  \left\{  \prescript{g}{}{s}\mid g\in G\quad s \in S  \right\} \right\rangle$$
כאשר אנחנו מחפשים \(R\subseteq F(s)\) כךש-\(\ker\left( \tilde{\iota} \right)= \left\langle \langle R\rangle\right\rangle\).

\begin{symbolize}
\(G=\left\langle  S\mid R  \right\rangle\) הצגה של \(G\) ע"י היוצרים של \(S\) ויחסים \(R\). 

$$\begin{gathered}S\subseteq R\quad S\subseteq G \\\left\langle  \langle R\rangle  \right\rangle =\ker\left( \left( \tilde{\iota}:F(s)\to G \right) \right)\quad R\subseteq F(s)
\end{gathered}$$
נובע:
$$G\cong F(s) / \left\langle  \langle R\rangle  \right\rangle$$

\end{symbolize}
\begin{example}
$$\qquad Z_{n}=\left\langle  a\mid a^n \right\rangle \qquad \ker\left( \tilde{\iota}:\overbracket{ F(a) }^{ \cong\mathbb{Z}  }\to \mathbb{Z} _{n} \right) = \langle a^n\rangle =\left\langle  \langle a^n\rangle  \right\rangle$$$$D_{n}=\left\langle  \sigma,\tau \mid \sigma^n,\tau^2,\left( \sigma \tau \right)^2 \right\rangle$$
כאשר תיכף נוכיח את הדוגמא השנייה.

\end{example}
\begin{proposition}
אם \(f:G\to H\) הומומורפיזם ו-\(N\trianglelefteq  G\) אזי ישנו הומומורפיזם \(\bar{f}:G / N\to H\) כך ש-\(f=\bar{f}\circ \pi\)

\end{proposition}
\begin{proof}
הטרנה מתקיימת אם ורק אם \(N\subseteq \ker(f)\) אם"ם \(f\) קבועה על כל קוסט של \(N\) אם"ם \(\bar{f}(gN)=f(g)\) מוגדרת היטב. נחזור לדוגמא על דיהדרלים: נוכיח ש-\(D_{n}=\left\langle  \underbrace{ \sigma,\tau }_{ S } \mid \underbrace{ \sigma^n,\tau^2,\left( \sigma \tau \right)^2 }_{ R }  \right\rangle\). נסמן \(K=\left\langle  \langle R\rangle\right\rangle \trianglelefteq F(s)\)(רוצים: \(K=\ker\left( \tilde{\iota} \right)\)). נסתכל על הדיאגרמה הבאה:

\end{proof}
כיוון ש-\(R\subseteq \ker\left( \tilde{\iota} \right)\) נקבל \(\left\langle  \langle R\rangle\right\rangle\subseteq \ker\left( \tilde{\iota} \right)\) ולכן \(\bar{\tilde{\iota}}\) קיים. וכן זה יהיה איזומורפיזם כיוון שעל(כי \(\left\{  \sigma,\tau  \right\}=im\left( \iota \right)\subseteq im\left( \tilde{\iota} \right)=im\left( \bar{\tilde{\iota}} \right)\)) נוכיח כי
$$2n\geq |F\left( \sigma,\tau \right) / \left\langle  \langle R\rangle  \right\rangle |$$
ולכן \(\tau\) איזומורפיזם. נוכיח שע"י מכפלה באיברי \(\left\langle  \langle R\rangle\right\rangle\) אפשר להגיע ממילה כללית ב-\(F\left( \sigma,\tau \right)\) לאחת מהמילים \(\sigma^i\tau^k\) כאשר \(0\leq j<n, 0\leq k\leq 1\). נשים לב שב \(F\left( \sigma,\tau \right) / \left\langle \langle R\rangle\right\rangle\) מתקיים \(\sigma^n=e\)(כלומר \(\sigma^n\left\langle \langle R\rangle\right\rangle=\left\langle  \langle R\rangle  \right\rangle\)) כאשר \(\tau^2=e\) וכן:
$$\sigma \tau=\sigma \tau\left( \sigma \tau \right)^{-2}=\tau ^{-1} \sigma ^{-1} =\tau ^{-1} \left( \tau^2 \right)\sigma ^{-1} =\tau ^{-1}$$
מ-\(e=\tau^2=\sigma^4\) , \(\sigma \tau=\tau \sigma ^{-1}\) אתם יודעים לעבור ממילה כללית ב-\(\sigma ,\tau,\sigma ^{-1} ,\tau ^{-1}\) לאחת המילים \(\sigma^i\tau^k\) כאשר \(0\leq j<n,0\leq k<2\) (כי אפשר להחליף \(\tau^k \sigma^j\) ב-\(\sigma^m\tau^\ell\) כלשהו).
נקבל כי \(\bar{f}\) מונומורפיזם(שיכון/חח"ע) אם"ם \(N=\ker(f)\): אם \(k\in \ker(f)\) אזי $$\bar{f}(kN)=\bar{f}(k)=f(k)=e$$$$\bar{f} \text{ ע"חח } \implies kN=N\implies k\in N$$

\begin{proposition}
$$\left\langle  \langle R\rangle  \right\rangle =\ker\left( \left( F(s)\to G \right) \right) \quad \langle s\rangle =G\implies G\cong F(s) / \left\langle  \langle R\rangle  \right\rangle$$
ו-\(H\) חבורה כלשהי אז אם \(f:S\to H\) פונקציה אז קיים הומומורפיזם \(\tilde{f}:G\to H\) שמרחיב את \(f\) (\(\tilde{f}|_{S}=f\)) אם"ם לכל \(R\ni r=\prod_{i=1}^m s_{i}^{\varepsilon_{i}}\) מתקיים \(\prod_{i=1}^mf(s_{i})^{\varepsilon_{i}}=1\) ב-\(H\).

\end{proposition}
\begin{example}
תהי \(H\) חבורה אזי יש הומומורפיזם \(f:D_{n}\to H\) שמקיים \(f\left( \pi \right)=h', f\left( \sigma \right)=h\) אם"ם \((hh')^2=h^n=h'^2=e\)

\end{example}
\begin{proof}
אם קיימת \(f\) כזו אז בפרט מתקיים:
$$\forall r=\prod_{i=1}^m s_{i}^{\varepsilon_{i}}\qquad \prod_{i=1}^{m}f(s_{i})^{\varepsilon_{i}}=\prod_{i=1}^{m}\tilde{f}(s_{i})^{\varepsilon_{i}}=\tilde{f}\left( \prod_{i=1}^{m}s_{i}^{\varepsilon_{i}}O \right)=\tilde{f}(e)=e$$
מצד שני, נסתכל על:
$$G= f(s) / \ker\left( \tilde{\iota} \right)=F(s) / \left\langle  \langle R\rangle  \right\rangle$$
לפי הקרטריון של התפרקות הומומורפיזם דרך חבורת מנה, \(\tilde{f}\) מוגדרת אם"ם \(\left\langle  \langle R\rangle  \right\rangle\subseteq \ker\left( \tilde{f}^{F(s)} \right)\)
וזה שקול ל-\(R\subseteq \ker\left( \tilde{f}^{F(s)} \right)\)

\end{proof}
\begin{example}
$$\begin{gathered}\mathbb{Z} =\left\langle  a\mid \varnothing\right\rangle \qquad \mathbb{Z} _{n}=\left\langle  a\mid a^n \right\rangle \qquad F_{n}=\left\langle  a_{1},\dots,a_{n}\mid \varnothing\right\rangle \\ \mathbb{Z} ^2=\left\langle  a,b\mid aba^{-1} b^{-1}  \right\rangle  \qquad \mathbb{Z}^3=\left\langle  a,b,c\mid[a,b],[b,c],[a,c] \right\rangle   \\S_{n}=\left\langle  a_{1},\dots,a_{n-1}\mid a_{i}^2 \;\;\forall i\leq n-1\quad (a_{i}a_{i+1})^3\;\;\forall i\leq n-2\qquad a_{i}a_{j}a_{i}^{-1} a_{j}^{-1}\;\; \forall j\geq i+2 \right\rangle
\end{gathered}$$

\end{example}
\begin{remark}
אין אלגוריתם שמכריע בהנתן \(S,R\) סופית האם \(\left\langle  S\mid R  \right\rangle\) טריווירליות סופיות או אין סופיות.

\end{remark}
\end{document}