\documentclass{tstextbook}

\usepackage{amsmath}
\usepackage{amssymb}
\usepackage{graphicx}
\usepackage{hyperref}
\usepackage{xcolor}

\begin{document}

\title{Example Document}
\author{HTML2LaTeX Converter}
\maketitle

\chapter{התכנסות של סדרת פונקציות}

\section{חסימות טוטלית}

\begin{proposition}[בולצנו ווירשטראס]
תהי \((X,\rho)\) מרחב מטרי ויהי \(A\subseteq X\) חסומה. אזי לכל סדרה \((a_{n})\) תהיה תת סדרה קושי.

\end{proposition}
\begin{proof}
  \begin{enumerate}
    \item כיוון ש-A חסומה, קיים \(R> 0\) כך ש-\(A\subset [-R,R]\). נסמן את התחום הזה ב-\(\Delta_{0}=[-R,R]\). 


    \item נחלק את \(\Delta_{0}\) לשתי חציים: 
$$\Delta_{0}^{(1)}=[0,R]\qquad \Delta_{0}^{(2)}=[-R,0]$$
כאשר כיוון ש-\((a_{n})\) סדרה אינסופית ב-\(A\subseteq \Delta_{0}\) נבחר את אחד מהחציים האלה(למשל את הראשון) ונסמן אותו ב-\(\Delta_{1}\).


    \item נחזור על התהליך - בהנתן \(\Delta_{k}\) נחלק אותו לשתי תתי אינטרוולים שווים, כך שלפחות אחד מהם(אשר יסומן \(\Delta_{k+1}\)) מכיל אינסוף נקודות של \((a_{n})\). נשים לב כי האורך של \(\Delta_{k}\) הוא: 
$$\lvert \Delta_{k} \rvert =\frac{2R}{2^{k}}$$


    \item נבחר נקודה אחד \(a_{n_{k}}\) מ-\(\Delta_{k}\). כעת לכל \(\ell \geq k\) גם \(a_{n_{k}}\) וגם \(a_{n_{\ell}}\) נמצאים בתוך \(\Delta_{k}\) ולכן: 
$$\rho\bigl(a_{n_{k}},a_{n_{\ell}}\bigr)\leq\frac{2R}{2^{k}}.$$
כיוון ש-\(\frac{2R}{2^{k}}\) יכול להיות קטן כרצונינו נקבל כי \((a_{n_{k}})\) קושי.


  \end{enumerate}
\end{proof}
\begin{definition}[מרחב נורמי]
יהי \(V\) מרחב ווקטורי מעל \(\mathbb{F}=\mathbb{R},\mathbb{C}\).  פונקציה \(No\)

\end{definition}
\begin{definition}[המרחב \(\ell_{2}\)]
$$\ell_{2}={\Bigg\{}x=\left(x_{1},x_{2},\dots\right)\mid x_{i}\in\mathbb{R}{\mathrm{~and~}}\sum_{i=1}^{\infty}x_{i}^{2}<\infty{\Bigg\}}$$
כאשר נגדיר את הנורמה על המרחב הזה על ידי:
$$\|x\|=\left(\sum_{i=1}^{\infty}x_{i}^{2}\right)^{1/2}$$

\end{definition}
\begin{definition}[חסימות טוטלית]
ניתן לכיסוי על ידי מספר סופי של כדורים.

\end{definition}
\begin{proposition}
מרחב מטרי חסום טוטלית הוא ספרבילי

\end{proposition}
\begin{proof}
יהי \(X\) מרחב מטרי חסום טוטלית. אזי לכל \(\varepsilon> 0\) קיים קבוצה סופית \(D_{\varepsilon}\subseteq X\) כך ש:
$$X\subseteq\bigcup_{x \in D_{\varepsilon}}B_{\varepsilon}(x)$$
יהי \(n \in \mathbb{N}\) כך שעבור \(\varepsilon=\frac{1}{n}\) נקבל:
$${ X}\subseteq\bigcup_{x\in D_{n}}B_{1/n}(x) $$
נגדיר כעת:
$$A=\bigcup_{n=1}^{\infty}D_{n}$$
כיוון ש-\(D_{n}\) סופי ואנחנו לוקחים איחוד בין מנייה נקבל כי A תת קבוצה בת מנייה של \(X\).
נזכור כי כדי להראות ש-A היא צפופה ב-\(X\) מספיק להראות שלכל \(x \in X\) יש נקודה \(a \in A\) כך שהמרחק ביניהם קטן כרצונינו.
יהי \(x \in X\) ויהי \(\varepsilon> 0\) שרירותי. אזי קיים \(n\in \mathbb{N}\) כך ש-\(\frac{1}{n}< \varepsilon\). כעת כיוון שמתקיים:
$$X\subseteq\bigcup_{x\in D_{n}}B_{1/n}(x)$$
ולכן \(x \in B_{1 / n}(a)\) כלומר:
$$d(x,a)<{\frac{1}{n}}<\varepsilon$$
ולכן A צפופה ב-\(X\).

\end{proof}
\begin{theorem}[האוסדורף - Hausdorff]
נניח \((x,\rho)\) מרחב מטרי כלשהו, ויש לנו קבוצה \(A\subseteq X\) אזי התנאים הבאים שקולים:

  \begin{enumerate}
    \item הקבוצה \(A\) חסומה טוטלית(עבור כל \(\varepsilon> 0\) קיים מספר סופי של כדורים \(B_{\varepsilon}(x_{1}),\dots,B_{\varepsilon}(x_{N})\) שמכסים, כלומר \(A\subseteq \bigcup_{j=1}^{N}B_{\varepsilon}(x_{j})\)). 


    \item כל סדרת נקודות ב-\(A\) כוללת תת סדרת קושי. 


  \end{enumerate}
\end{theorem}
\begin{proof}
  \begin{enumerate}
    \item נניח חסום לחלוטין. תהי \((a_{n})\) סדרה כלשהי. נבנה סדרה של קבוצות \(\{ V^{(n)} \}_{n=1}^{\infty}\) באופן רקורסיבי. 


    \item נגדיר את \(V^{(1)}\) באופן הבא. כיוון ש-\(A\) חסומה לחלוטין עבור \(\varepsilon=1\) קיים סדרה סופית של נקודות \(\{ x_{n} \}_{n=1}^{N}\) כך ש: 
$$A\subseteq \bigcup_{i=1}^{N}B_{1}(x_{i})$$


    \item כיוון ש-\((a_{n})\) סדרה אינסופית ו-\(\{ x_{i} \}_{i=1}^{N}\) קבוצה סופית קיים איזשהו \(x_{j}\) כך ש-\(B_{1}(x_{j})\) מכיל מספר אינסופי של נקודות. נגדיר: 
$$V^{(1)}=A\cap  B_{1}(x_{j})$$
כאשר נשים לב כי:
$$\mathrm{Diam}(V^{(1)})\leq \mathrm{Diam}(B_{1}(x_{j}))=2$$


    \item נניח כי נתון \(V^{(k)}\). נגדיר את \(V^{(k+1)}\). כיוון ש-\(V^{(k)}\) חסום לחלוטין עבור \(\varepsilon= \frac{1}{k}\) נקבל סדרה סופית של נקודות \(\{ x_{n} \}_{n=1}^{N}\) של נקודות כך ש-\(V^{(k)}\subseteq \bigcup_{i=1}^{N}B_{\frac{1}{k}}(x_{i})\). כמו מקודם קיימת איזשהו \(x_{j}\) כך ש-\(B_{\frac{1}{k}}(x_{j})\) מכיל מספר אינסופי של נקודות של \((a_{n})\). נגדיר: 
$$V^{(k+1)}=V^{(k)}\cap  B_{\frac{1}{k}}(x_{j})$$
כאשר כעת נשים לב כי:
$$\mathrm{Diam}(V^{(k+1)})\leq \mathrm{Diam}\left( B_{\frac{1}{k}}(x_{j}) \right)=\frac{2}{k}$$
כלומר כל שתי נקודות ב-\(V^{(k+1)}\) יהיו במרחק של לכל היותר \(\frac{2}{k}\). וכן מתקיים:
$$V^{(1)}\supseteq V^{(2)}\supseteq \dots$$


    \item נבנה תת סדרה באופן הבא. נגדיר \(n_{1}\) בתור האיבר הראשון כך ש-\(a_{n_{1}} \in V^{(1)}\)(קיים כזה כי להיות ב-\(V^{(1)}\) זו תכונה שכיחה). כמו כן נגדיר את \(n_{k}\) להיות בתור האיבר הראשון הגדול מ-\(n_{k-1}\) כך ש-\(a_{n_{k}}\in V^{(k)}\)(שוב קיים כזה כי להיות ב-\(V^{(k)}\) זו תכונה שכיחה).  


    \item נראה כי זוהי סדרת קושי. יהי \(\varepsilon> 0\). נבחר \(N>\frac{2}{\varepsilon}\). לכל \(k,m> N\) מתקיים: 
$$\rho(a_{n_{m}},a_{n_{k}})\leq  \mathrm{Diam}\left( V^{(N)} \right)\leq \frac{2}{N}\leq \varepsilon$$


    \item כעת נראה את הכיוון השני. נניח כי כל סדרת נקודות ב-\(A\) כוללת תת סדרת קושי ונראה כי חסום לחלוטין. זה שקול ללהוכיח כי אם לא חסום לחלוטין אז קיים סדרת נקודות אשר לא כוללת תת סדרת קושי. כיוון שלא חסום לחלוטין קיים איזשהו \(\varepsilon_{0}\) כך שלא ניתן לכסות את \(A\) עם מספר סופי של כדורים ברדיוס \(\varepsilon_{0}\). 


    \item נבנה סדרה \((a_{n})\) כזו באופן אינדוקטיבי. יהי \(a_{1} \in A\) נקודה כלשהי. תהי \(a_{2} \in A\) נקודה כך ש: 
$$\rho(a_{1},a_{2})>\varepsilon_{0}$$
כאשר קיים כזו אחרת:
$$A\subseteq B_{\varepsilon_{0}}(a_{1})$$
בסתירה לזה שלא חסום לחלוטין.


    \item כעת אם מוגדרות נקודות \((a_{1},a_{2},\dots,a_{N})\) נגדיר \(a_{N+1}\) על ידי נקודה ב-\(A\) כך שלכל \(1\leq k\leq N\) מתקיים: 
$$\rho(a_{N+1},a_{k})> \varepsilon_{0}$$
כאשר קיים כזו כי אחרת \(A\subseteq{ B_{\varepsilon_{0}}(a_{1}) }\cup \dots \cup {  B_{\varepsilon_{0}}(a_{N}) }\) בסתירה לכך שלא חסום במידה לחלוטין.


    \item לא קיימת תת סדרת קושי כי עבור \(\frac{\varepsilon_{0}}{2}\) ולכל \(m,n \in a_{N}\) מתקיים: 
$$\rho(a_{n},a_{m})>\varepsilon_{0}$$


  \end{enumerate}
\end{proof}
\section{רציפות וחסימות במידה אחידה}

\begin{definition}[קבוצת הפונקציות הרציפות]
נגדיר:
$$C([a,b])=\{ f\mid [a,b]\to \mathbb{R}, f \text{ is continous} \}$$
כאשר:
$$\lVert f \rVert _{\infty} \overset{\text{def}}{=} \max _{x \in [a,b]} \lvert f(x) \rvert $$
נשים לב כי \((C[a,b],\lVert \cdot \rVert_{a})\) הוא מרחיב נורמי.

\end{definition}
\begin{definition}[חסימות במידה אחידה - Uniformly Bound]
$$\Phi \subseteq C[a,b]$$
נניח שקיים איזשהו פרמטר \(\kappa > 0\) כך שמתקיים:
$$\forall x \in [a,b]\quad \forall \varphi \in \Phi \quad \lvert \varphi(x) \rvert \leq \kappa$$
(חשוב: \(\kappa\) אינו תלוי ב-\(x\) ואינו תלוי ב-\(\varphi\)) אזי \(\Phi\) נקראת חסומה במידה אחידה.

\end{definition}
\begin{example}
$$\Phi=(\sin(nx))_{n=1}^{\infty}$$
כאשר נשים לב כי \(\lvert \sin(nx) \rvert\leq 1\).

\end{example}
\begin{example}
$$f_{n}(x)= \frac{x^{2}}{x^{2}+(1-nx)^{2}}\qquad  n \in \mathbb{N}$$

\end{example}
\begin{definition}[רציפות במידה אחידה - equicontinous family of functions]
יהי \(\Phi \subseteq C[a,b]\). עבור כל \(\varepsilon > 0\) קיים \(\delta=\delta(\varepsilon)\) כך שמתקיים:
$$\forall x_{1},x_{2} \in [a,b]\quad \lvert x_{1}-x_{2} \rvert \leq  \delta(\varepsilon)\implies \lvert \varphi(x_{1})-\varphi(x_{2}) \rvert \leq  \varepsilon$$
לכל \(\varphi \in \Phi\). אזי \(\Phi\) נקראת רציפה במידה אחידה.

\end{definition}
\begin{example}[משפחה לא רציפה במידה אחידה]
$$\left\lvert  f_{n}\left( \frac{1}{n} \right)-f_{n}(0)  \right\rvert =1$$
גורר כי המשפחה אינה רציפה במידה אחידה.

\end{example}
\begin{example}
סדרה \((f_{n})_{n=1}^{\infty}\) גזירות ב-\((a,b)\) כך ש-\(f_{n}\in C[a,b]\) קיים \(\kappa > 0\) כך שמתקיים:

  \begin{enumerate}
    \item לכל \(x \in [a,b]\) ולכל \(n \in \mathbb{N}\) מתקיים \(\lvert f_{n}(x) \rvert\leq \kappa\). 


    \item מתקיים \(\lvert f'_{n}(x)\leq \kappa \rvert\). 
אזי משפחה \((f_{n})_{n=1}^{\infty}\) חסומה במידה אחידה, וגם רציפה במידה אחידה.


  \end{enumerate}
\end{example}
\begin{proof}
יהי \(\varepsilon> 0\). 
$$\lvert f_{n}(x_{1})-f_{n}(x_{2}) \rvert \leq \lvert f'(\xi) \rvert \lvert x_{1}-x_{2} \rvert \leq  \kappa \lvert x_{1}-x_{2} \rvert $$
ניתן לבחור \(\delta(\varepsilon)=\frac{\varepsilon}{M}\).

\end{proof}
\begin{proposition}
תהי \(\{ f_{n} \}\) משפחת פונקציות המתכנסות במידה אחידה על \([0,1]\), ונניח כי \(f:[0,1]\to \mathbb{R}\) הגבול הנקודתי של \(f_{n}(x)\) לכל \(x\). אזי הסדרה \(f_{n}\) מתכנסת במידה שווה.

\end{proposition}
\begin{proof}
  \begin{enumerate}
    \item נראה כי לכל \(\varepsilon> 0\) קיים \(N \in \mathbb{N}\) כך שלכל \(x \in [0,1]\) ולכל \(n> N\) מתקיים \(\lvert f_{n}(x)-f(x) \rvert<\varepsilon\). 


    \item יהי \(\varepsilon> 0\). מרציפות במידה אחידה קיים \(\delta> 0\) כך שלכל \(n \in \mathbb{N}\) מתקיים: 
$$\lvert x-y \rvert <\delta\implies \lvert f_{n}(x)-f_{n}(y) \rvert <\frac{\varepsilon}{3}\implies \lvert f(x)-f(y) \rvert \leq \frac{\varepsilon}{3}$$
כאשר השתמשנו בשימור של אי שיוויון חלש בין גבולות.


    \item יהי \(m> \frac{1}{\delta}\). נגדיר \(y_{i}=\frac{i}{m}\). כל נקודה בקטע \([0,1]\) היא לכל היותר במרחק \(\delta\) מאיזשהו \(y_{i}\). יהי \(N\) כך שלכל \(n>N\) ולכל \(1\leq i\leq m\) מתקיים \(\lvert f_{n}(y_{i})-f_{n}(y_{i}) \rvert<\frac{\varepsilon}{3}\). אזי לכל \(x \in [0,1]\) נבחר \(i\) כך ש-\(\lvert x-y_{i} \rvert<\delta\) ולכל \(n> N\) נקבל כי: 
$$|f_{n}(x)-f(x)|\leq|f_{n}(x)-f_{n}(y_{i})|+|f_{n}(y_{i})-f(y_{i})|+|f(y_{i})-f(x)|<\frac{\varepsilon}{3}\cdot 3=\varepsilon$$


  \end{enumerate}
\end{proof}
\begin{theorem}[ארצלה]
יהי \(\Phi \subseteq (C[a,b],\lVert \cdot \rVert_{\infty})\). התנאים הבאים שקולים:

  \begin{enumerate}
    \item לכל סדרה \((f_{n})_{n=1}^{\infty}\subset \Phi\) קיימת תת סדרה קושי \((f_{n})_{k}^{\infty}\) קיימת תת סדרת קושי \((f_{n_{k}})_{k=1}^{\infty}\) כך שמתקיים: 
$$\lVert f_{n_{k}}-f_{n_{k+\ell}} \rVert \xrightarrow{k\to \infty} 0 $$
עבור ל \(\ell \in \mathbb{N}\). 


    \item הקבוצה \(\Phi\) חסומה במידה אחידה וגם \(\Phi\) רציפה במידה אחידה. 


  \end{enumerate}
\end{theorem}
\begin{proof}
1 גורר 2

  \begin{enumerate}
    \item נניח כי לכל סדרה קיימת תת סדרת קושי. ממשפט האוסדורף \(\Phi\) חסומה לחלוטין. יהי \(\varepsilon>0\). מחסימות לחלוטין קיימות \(f_{1},\dots,f_{N}\in \Phi\) כך ש: 
$$\Phi \subseteq \bigcup_{i=1}^{N}B_{\frac{\varepsilon}{3}}(f_{i})$$


    \item כעת מתקיים: 
$$\lVert f \rVert _{\infty}=\lVert f-f_{i}+f_{i} \rVert_{\infty} \leq \lVert f-f_{i} \rVert_{\infty} +\lVert f_{i} \rVert _{\infty}\leq \frac{\varepsilon}{3}+\lVert f_{i} \rVert _{\infty}$$
כאשר ניתן להגדיר \(K=\max_{_{i}}\{ \lVert f_{i} \rVert_{\infty} \}\) וכן \(M=K+\frac{\varepsilon}{3}\) ולקבל \(\lVert f \rVert_{\infty}\leq M\) ולכן חסום במידה אחידה.


    \item כעת כיוון ש-\(f \in B_{\frac{\varepsilon}{3}}(f_{i})\) וכיוון ש-\(f_{i}\) רציפה ב-\([a,b]\) אזי רציפה במידה שווה ב-\([a,b]\). לכן קיים \(\delta_{i}\) כך שאם \(\lvert x-y \rvert<\delta\) מתקיים \(\lvert f_{i}(x)-f_{i}(y) \rvert<\frac{\varepsilon}{3}\). אם נגדיר \(\delta=\min_{i}\{ \delta_{i} \}\) נקבל שכאשר \(\lvert x-y \rvert<\delta\) ב-\([a,b]\) מתקיים: 
$$\lvert f(x)-f(y) \rvert =\lvert f(x)-f_{i}(x)-f(y)+f_{i}(y)+f_{i}(x)-f_{i}(y) \rvert \leq \lvert f(x)-f_{i}(x) \rvert +\lvert f(y)-f_{i}(y) \rvert +\lvert f_{i}(x)-f_{i}(y) \rvert \leq 3 \cdot \frac{\varepsilon}{3}=\varepsilon$$
ולכן \(\Phi\) רציפה במידה אחידה.


  \end{enumerate}
\end{proof}
\begin{proof}
2 גורר 1

  \begin{enumerate}
    \item נניח חסימות במידה אחידה ורציפות במידה אחידה. יהי \(\varepsilon> 0\). מרציפות במידה אחידה נקבל \(\delta>0\) כך שלכל \(f \in \Phi\) מתקיים: 
$$\forall x_{1},x_{2}\in[a,b]\qquad \lvert x_{1}-x_{2} \rvert <\delta\implies \lvert f(x_{1})-f(x_{2}) \rvert \leq  \frac{\varepsilon}{3} $$


    \item ניקח חלוקה של \([a,b]\) כאשר המרווחים בין כל איברים סמוכים בחלוקה קטנים מ-\(\delta\). כלומר: 
$$a = x_0 < x_1 < \dots < x_N = b \quad \text{with}\;\;x_{i+1} - x_i < \delta$$


    \item מחסימות במידה אחידה של קיים \(K>0\) כך ש-לכל \(f \in \Phi\) מתקיים \(\lVert f \rVert_{\infty}\leq K\) וכעת ניתן לחלק את התחום \([-K,K]\) למרווחים אשר קטנים מ-\(\varepsilon\). כלומר: 
$$-K=y_{0}<y_{1}<\cdot\cdot\cdot<y_{M}=K\quad \text{with }y_{i+1}-y_{i}<\frac{\varepsilon}{3}$$
בפרט לכל \(z \in [-K,K]\) קיים \(y_{j}\) כך ש-\(\lvert z-y_{j} \rvert<\frac{\varepsilon}{3}\).


    \item תהי \(f \in \Phi\). לכל \(x_{i}\) קיים \(y_{j_{i}}\) כך ש-\(\lvert f(x_{i})-y_{j_{i}} \rvert<\frac{\varepsilon}{3}\). כעת נגדיר פונקציה מקורבת \(\psi\) על ידי: 
$$\psi(x_{i})=y_{j_{i}}$$
וכן לכל נקודות בתחום \(x \in (x_{i},x_{i+1})\) נגדיר את הפונקציה להיות האינטרפולציה הלינארית:
$$\psi(x)=y_{j_{i}}+(y_{j_{i+1}}-y_{j_{i}})\cdot{\frac{x-x_{i}}{x_{i+1}-x_{i}}}$$
כך שרציף ולינארי למקוטעין.


    \item יהי \(x \in [x_{i},x_{i+1}]\). אזי: 
$$|f(x)-\psi(x)|\leq|f(x)-f(x_{i})|+|f(x_{i})-y_{j_{i}}|+|y_{j_{i}}-\psi(x)|<\frac{\varepsilon}{3}\cdot 3 = \varepsilon$$


    \item כיוון שיש כמות סופית של \(\psi\) שניתן לבנות וכן כל \(f\) מקיים \(f \in B_{\varepsilon}(\psi)\) עבור \(\psi\) כלשהו נקבל כי חסום לחלוטין. 


  \end{enumerate}
\end{proof}
\begin{corollary}
תת קבוצה של \(C[a,b]\) היא חסומה טוטלית אם"ם היא חסומה במידה אחידה ורציפה במידה אחידה.

\end{corollary}
\begin{proof}
נובע מיידית ממשפט האוסדורף שאומר ש-1 שקול לכך שחסומה טוטלית.

\end{proof}
\begin{remark}
הטענה הזו למעשה אומרת שהסגור של הקבוצה היא קומפקטית(לכל כיסוי פתוח יש תת כיסוי סופי בעזרת החסימות טוטלית).

\end{remark}
\begin{corollary}
תת קבוצה של \(C[a,b]\) היא קומפקטית אם"ם סגורה, חסומה במידה אחידה, ורציפה במידה אחידה.

\end{corollary}
\begin{corollary}
אם תת קבוצה של \(C[a,b]\) היא חסומה במידה אחידה ורציפה במידה אחידה אז לכל סדרה בה יש תת סדרה מתכנסת במידה שווה(כאשר הגבול לאו דווקא שייך לקבוצה).

\end{corollary}
\begin{theorem}[ארצלה אסקולי]
תהי \(\{ f_{n} \}_{n=1}^{\infty}\subseteq C^{1}[a,b]\) סדרה חסומה(בנורמה) של פונקציות גזירות ברציפות. כלומר הסדרה \(\{ f_{n} \}_{n=1}^{\infty}\) ו-\(\{ f'_{n} \}_{n=1}^{\infty}\) חסומות במידה אחידה, אזי הסדרה \(\{ f_{n} \}_{n=1}^{\infty}\) חסומה במידה אחידה וגזירה במידה אחידה(ולכן יש לה תת סדרה מתכנסת במידה שווה).

\end{theorem}
\begin{proof}
ממשפט ארצלה מספיק להראות כי החסימות במידה אחידה של הנגזרת גורר רציפות במידה אחידה.
ממשפט ערך הביניים אנו יודעים כי לכל שתי נקודות \(x,y \in [a,b]\) קיים נקודה \(a<c<b\) כך ש:
$$f_{n}(x)-f_{n}(y)=f_{n}^{\prime}(c)(x-y)\implies |f_{n}(x)-f_{n}(y)|=|f_{n}^{\prime}(c)|\cdot|x-y|$$
ואם נשתמש בחסם אחיד \(M\) על הנגזרת נקבל:
$$|f_{n}(x)-f_{n}(y)|\leq M\cdot|x-y|$$
אם \(M=0\) הטענה טריוויאלית(הפונקציות קבועות) ולכן נניח \(M> 0\). יהי \(\varepsilon> 0\). נבחר \(\delta=\frac{\varepsilon}{M}\). עבור \(x,y \in [a,b]\) כך ש-\(\lvert x-y \rvert<\delta\) נקבל:
$$|f_{n}(x)-f_{n}(y)|\leq M\cdot|x-y|<M\cdot\delta=M\cdot{\frac{\epsilon}{M}}=\epsilon$$
ולכן רציף במידה שווה ותנאי משפט ארצלה מתקיימות.

\end{proof}
\begin{theorem}[ארצלה עבור \(\ell_{2}\)]
$$\ell_{2}=\left\{  x = (x_{1},x_{2},\dots)\mid  \sum_{j=1}^{\infty} x_{j}^{2}<\infty  \right\}\qquad \lVert x \rVert^{2}=\sum_{j=1}^{\infty}x_{j}^{2}$$
כאשר \(K\subseteq \ell_{2}\). הסדרה \(K\) חסומה טוטלית אם"ם:

  \begin{enumerate}
    \item הסדרה \(K\) חסומה במרחב \((\ell_{2},\lVert \cdot \rVert)\) כלומר קיים \(R> 0\) כך שלכל \(x \in K\) מתקיים \(\lVert x \rVert\leq R\). 


    \item מתקיים: 
$$\lim_{ M \to \infty } \left\{  \sup _{x \in K}\left( \sum_{j=M}^{\infty}x_{j}^{2} \right)  \right\}=0$$


  \end{enumerate}
\end{theorem}
\section{נגזרות ואינטגרלים}

\begin{proposition}
תהי \(K(x,t)\) פונקציה רציפה על ריבוע היחידה \([0,1]\times[0,1]\). וכן נגדיר \(T_{k}:C[0,1]\to C[0,1]\) על ידי:
$$.T_{K}f(x)=\!\!\int\limits_{0}^{1}f(t)K(x,t)d t$$
אזי:

  \begin{enumerate}
    \item הפונקציה \(T_{k}\) מוגדרת היטב. 


    \item שולח קבוצות חסומות לקבוצות חסומות טוטלית. 


  \end{enumerate}
\end{proposition}
\begin{definition}[אופרטור קומפקטי]
יהי \(V,W\) מרחבים נורמלים, ו-\(T:V\to W\) העתקה לינארית. אזי אם לכל קבוצה חסומה \(B\subseteq V\) מתקיים \(\overline{T(B)}\) קומפקטי ב-\(W\) נקרא לאופרטור \(T\) קומפקטי.

\end{definition}
\section{משפט הקירוב של ווירשטראס וסטון וירשטראס}

\begin{lemma}
עבור \(\delta> 0\) מתקיים:
$$\int_{\delta}^{1} (1-u^{2})^{n} \, \mathrm{d}u\xrightarrow{n\to \infty} 0  $$

\end{lemma}
כיוון ש-\(1-u^{2}\) מונוטונית יורדת מתקיים:
$$\int_{\delta}^{1} (1-u^{2})^{n} \, \mathrm{d}u\leq  \int_{\delta}^{1} (1-\delta^{2})^{n} \, \mathrm{d}u=(1-\delta^{2})^{n}(1-\delta)\xrightarrow{n\to \infty} 0 $$

\begin{theorem}[הקירוב של ווירשטראס]
תהי \(f \in C[0,1]\). אזי קיימת סדרת פולינומים \(\{ p_{n}(x) \}_{n=1}^{\infty}\) אשר מתכנסת במידה שווה ל-\(f\) על \([0,1]\).

\end{theorem}
\begin{proof}
  \begin{enumerate}
    \item יהי \(f \in C[0,1]\). נגדיר: 
$$F(x)={\left\{\begin{array}{l l}{f(x),}&{x\in[0,1],}\\ {0,}&{{\mathrm{otherwise}},}\end{array}\right.}$$
כאשר נשים לב כי \(f\) רציפה ב-\([0,1]\) ולכן ממשפט קנטור רציפה במ"ש ולכן \(F\) רציפה במ"ש ב-\(\mathbb{R}\).


    \item נגדיר את הגרעין \(Q_{n}(u)=c_{n}{{\bigl(}1-u^{2}{\bigr)}^{n}},\) כאשר \(c_{n}\) הוא כך שמתקיים: 
$$\int_{-1}^{1}Q_{n}(u)\,\mathrm{d}u=1$$


    \item לכל \(n\) נגדיר: 
$$P_{n}(x)=\int_{-1}^{1}F(x+u)\,Q_{n}(u)\,\mathrm{d}u$$
ועל ידי ההחלפת משתנים \(t=x+u\) נקבל:
$$P_{n}(x)=\int_{x-1}^{x+1}F(t)\,Q_{n}(t-x)\,\mathrm{d}t=\int_{0}^{1}F(t)\,Q_{n}(t-x)\,\mathrm{d}t$$
כאשר זהו פולינום ב-\(x\). זהו סדרת הפולינומים שלנו. נוצר להראות שמתכנס במ"ש ל-\(F\).


    \item כיוון ש-\(\int_{-1}^{1} Q_{n}(x) \, \mathrm{d}x=1\) נקבל: 
$$F(x)=F(x)\,\int_{-1}^{1}Q_{n}(u)\,\mathrm{d}u=\int_{-1}^{1}F(x)\,Q_{n}(u)\,\mathrm{d}u$$
ולכן:
$$\left|P_{n}(x)-F(x)\right|=\left|\int_{-1}^{1}\bigl[F(x+u)-F(x)\bigr]Q_{n}(u)\,\mathrm{d}u\right|\leq\int_{-1}^{1}\bigl|F(x+u)-F(x)\bigr|\,Q_{n}(u)\,\mathrm{d}u$$


    \item יהי \(\varepsilon> 0\). מרציפות במ"ש של \(F\) נקבל \(\delta > 0\) כך לכל \(x\): 
$$|u|<\delta\implies|F(x+u)-F(x)|<\epsilon$$
כעת נפצל את האינטגרל:
$$\int_{-1}^{1}\!|F(x+u)-F(x)|\,Q_{n}(u)\,\mathrm{d}u=\int_{|u|<\delta}+\int_{|u|\geq\delta}$$


    \item עבור החלק \(\lvert u \rvert<\delta\) מתקיים \(\lvert F(x+u)-F(x) \rvert<\varepsilon\) ולכן: 
$$\int_{|u|<\delta}|F(x+u)-F(x)|\,Q_{n}(u)\,\mathrm{d}u<\epsilon\int_{-1}^{1}Q_{n}(u)\,\mathrm{d}u=\epsilon\cdot1=\epsilon.$$


    \item עבור החלק \(\lvert u \rvert\geq \delta\) אנו יודעים כי \(F\) חסום על ידי איזשהו \(M\), ולכן: 
$$|F(x+u)-F(x)|\leq|F(x+u)|+|F(x)|\leq2M.$$
נקבל:
$$\int_{|u|\geq\delta}|F(x+u)-F(x)|\,Q_{n}(u)\,\mathrm{d}u\leq2M\int_{|u|\geq\delta}Q_{n}(u)\,\mathrm{d}u.$$
ומהתכונה \(\operatorname*{lim}_{n\to\infty}\int_{|u|\geq\delta}Q_{n}(u)\,\mathrm{d}u=0\) של הגרעין נקבל \(N\) כך שלכל \(n\geq N\) מתקיים:
$$\int_{|u|\geq\delta}Q_{n}(u)\,\mathrm{d}u<{\frac{\epsilon}{2M}}$$
ולכן:
$$\int_{|u|\geq\delta}\mid\!F(x+u)-F(x)\!\mid Q_{n}(u)\,\mathrm{d}u\leq2M\cdot\frac{\epsilon}{2M}=\epsilon$$


    \item סה"כ נקבל: 
$$|P_{n}(x)-F(x)|\leq\underbrace{\int_{|u|<\delta}\bigl(\dots\bigr)}_{<\epsilon}+\underbrace{\int_{|u|\geq\delta}\bigl(\dots\bigr)}_{<\epsilon}<2\epsilon$$
ולכן \(\lVert P_{n}-F(x) \rVert_{\infty}\leq \varepsilon\) ומתכנס במ"ש.


  \end{enumerate}
\end{proof}
\begin{theorem}[סטון ווירטראס]
יהי \((X,\rho)\) מרחב מטרי, ותהי \(K\subseteq X\) קומפקטית. תהי \(A\subseteq(C(K),\lVert \cdot \rVert_{\infty})\) המקיימת:

  \begin{enumerate}
    \item אלגברה - כלומר סגורה לחיבור, כפל וכפל בסקלר. 


    \item מפרידה נקודות - לכל שתי \(x_{1}\neq x_{2}\) קיים \(f \in A\) כך ש-\(f(x)\neq f(y)\). 


    \item לא מתאפסת - לכל \(a \in K\) קיים \(f \in A\) כך ש-\(f(a)\neq 0\). 
אזי \(A\) צפוף ב-\(C(K)\)(כלומר \(\overline{A}=C(K)\)).


  \end{enumerate}
\end{theorem}
\begin{lemma}
אם \(f \in A\) אז \(\lvert f \rvert\in \overline{A}\).

\end{lemma}
\begin{proof}
יהי \(f \in A\). כיוון ש-\(K\) קומפקטית ו-\(f\) רציפה אז \(f(K)\) קומפקטית בתת תחום \([m,M]\subseteq \mathbb{R}\).
נגדיר \(\phi(t)=\lvert t \rvert\). יהי \(\varepsilon> 0\). ממשפט הקירוב של ווירשטראס קיים פולינום \(P(t)\) אשר מקיים:
$$\forall t \in [-M,M]\qquad \lVert \lvert t \rvert -P(t) \rVert <\varepsilon$$
נסתכל על \(P(f(x))\). נשים לב כי כיוון שזהו פולינום זהו צירוף לינארי של חזקות של \(f\) וכפל בסקלרים, וכיוון ש-\(f\) אלגברה נקבל \(P(f(x))\in A\). מתקיים כעת:
$$|||f(x)|-P(f(x))\rangle||_{\infty}=\operatorname*{sup}_{x\in K}||f(x)|-P(f(x))|\leq\operatorname*{sup}_{t\in[-M,M]}||t|-P(t)|<\epsilon$$
כיוון ש-\(P(f(x))\in A\) ו-\(P(f(x))\) מקרב את \(\lvert f(x) \rvert\) קרוב כרצונו נקבל כי \(\lvert f \rvert\in \overline{A}\).

\end{proof}
\begin{lemma}
אם \(f,g \in \overline{A}\) אזי \(\max\{ f,g \},\min\{ f,g \}\in \overline{A}\).

\end{lemma}
\begin{proof}
ניתן לכתוב:
$$\operatorname*{max}\{a,b\}={\frac{1}{2}}(a+b+|a-b|)\qquad \operatorname*{min}\{a,b\}={\frac{1}{2}}(a+b-|a-b|)$$
ולכן אם \(f,g \in \overline{A}\) אזי גם \(f+g \in \overline{A}\) כי הסגור של אלגברה היא אלגברה ומהלמה הקודמת נקבל \(\lvert f-g \rvert\in \overline{A}\) ולכן המקסימום והמינימום גם יהיו ב-\(\overline{A}\).

\end{proof}
\begin{lemma}
לכל שתי נקודות \(x_{1},x_{2} \in K\) ולכל שתי מספרים ממשיים \(c_{1},c_{2} \in \mathbb{R}\) קיים \(h \in A\) כך ש-\(h(x_{1})=c_{1}\) ו-\(h(x_{2})=c_{2}\).

\end{lemma}
כעת נוכיח את המשפט.

\begin{proof}
יהי \(f \in C(K)\) ו-\(\varepsilon> 0\). נרצה למצוא \(\varphi \in \overline{A}\) כך ש-\(\lVert f-\varphi \rVert_{\infty}<\varepsilon\).

  \begin{enumerate}
    \item יהי \(x \in K\). נבנה פונקציה \(g_{x}\in \overline{A}\) כך ש-\(g_{x}=f(x)\) ו-\(g(t)>f(t)-\varepsilon\) לכל \(t \in K\). לכל \(y\in K\) מהלמה של הפרדת נקודות קיימת פונקציה \(h_{y}\in A\) כך ש: 
$$h_{y}(x)=f(x)\qquad h_{y}(y)=f(y)$$


    \item נגדיר את הקבוצה: 
$$\tau_{y}=\{t\in K\mid h_{y}(t)>f(t)-\epsilon\}$$
כאשר זוהי קבוצה פתוחה כיוון ש-\(h_{y}-f\) היא פונקציה רציפה ו-\(\tau_{y}\) היא המקור של הסביבה הפתוחה \((-\varepsilon,\infty)\) תחת הפונקציה הרציפה. כמו כן מקיימת \(x ,y \in \tau_{y}\).


    \item האוסף הזה של \(\tau_{y}\) מהווה כיסוי פתוח של \(K\). ולכן מקומפקטיות קיים עבורו תת כיסוי סופי \(\{ \tau_{y_{1}},\tau_{y_{2}},\dots,\tau_{y_{n}} \}\) אשר מכסים את \(K\). זה מתאים לאוסף פונקציות סופיות \(\{ h_{y_{1}},\dots,h_{y_{n}} \}\) מ-\(A\). 


    \item כעת נגדיר: 
$$g_{x}(t)=\operatorname*{max}\{h_{y_{1}}(t),h_{y_{2}}(t),\ldots,h_{y_{n}}(t)\}$$
כאשר זה מקיים \(g_{x}(t)\in \overline{A}\) משימוש חוזר בלמה אשר אומר שהמקסימום של שתי פונקציות נמצא בסגור של האלגברה. כמו כן כיוון שלכל \(h_{y_{i}}\) מתקיים \(h_{y_{i}}(x)=f(x)\) אזי גם \(g_{x}(x)=f(x)\). ובאותו אופן כיוון שמתקיים לכל \(h_{i}\) נקבל כי גם עבור המקסימום מתקיים:
$$g_{x}(t)\geq h_{y_{i}}(t)>f(t)-\epsilon$$


    \item כעת נגדיר: 
$${\hat{\tau}}_{x}=\{t\in K\mid g_{x}(t)<f(t)+\epsilon\}.$$
כאשר זוהי קבוצה פתוחה כי \(g_{x}\) ו-\(f\) הם רציפות ו-\(g_{x}-f\) היא רציפה ולכן המקור של \((-\infty,\varepsilon)\) תהיה פתוחה. בנוסף מתקיים \(x \in \hat{\tau}_{x}\) כי \(g_{x}(x)=f(x)\) וגם \(f(x)<f(x)+\varepsilon\). 


    \item עבור כל \(x \in K\) נקבל כי \(\hat{\tau}_{x}\) הוא כיסוי פתוח של \(K\). ולכן מקומפקטיות של \(K\) קיים עבורו תת כיסוי סופי 
$$\{ \hat{\tau}_{x_{1}},\dots,\hat{\tau}_{x_{m}} \}$$
אשר מתאים אוסף סופי של פונקציות \(\{ g_{x_{1}},\dots,g_{x_{m}} \}\) מ-\(\overline{A}\).


    \item כעת נגדיר: 
$$\varphi(t)=\operatorname*{min}\{g_{x_{1}}(t),g_{x_{2}}(t),\ldots,g_{x_{m}}(t)\}$$
כאשר \(\varphi \in \overline{A}\) משימוש חוזר בלמה עבור המינימום.


    \item כיוון שכל \(g_{x}\) מקיים \(g_{x}(t)>f(t)-\varepsilon\) גם \(\varphi\) מקיים את זה וכיוון שכל \(g_{x_{j}}\) מקיים \(g_{x_{j}}(t)<f(t)+\varepsilon\). לכן נקבל לכל \(t \in K\) כי: 
$$f(t)-\epsilon<\varphi(t)<f(t)+\epsilon\implies \forall t \in K\quad |f(t)-\varphi(t)|<\varepsilon\implies \lVert f-\varphi \rVert _{\infty}<\varepsilon$$


  \end{enumerate}
\end{proof}
\chapter{טורי פורייה}

\section{מרחבי מכפלה פנימית}

\begin{definition}[מכפלה פנימית]
יהי \(V\) מרחב ווקטורי תחת \(\mathbb{F}=\mathbb{C}\) או \(\mathbb{F}=\mathbb{R}\). פונקציה \(\langle \cdot,\cdot \rangle:V\times V\to \mathbb{F}\) נקראת מכפלה פנימית אם מקיימת:

  \begin{enumerate}
    \item סימטרייה - לכל \(x,y \in V\) מקיימת: 
$$\big\langle x,y\big\rangle={\overline{{\big\langle y,x\big\rangle}}}$$


    \item בילנאריות - לכל \(x,y,z \in V\) ו-\(a \in \mathbb{F}\) מתקיים: 
$$\left\langle a x,y\right\rangle=a\left\langle x,y\right\rangle\quad \text{and}\quad\left\langle x+y,z\right\rangle=\left\langle x,z\right\rangle+\left\langle y,z\right\rangle$$


    \item חיוביות - לכל \(x \in V\) מתקיים \(\langle x,x \rangle\geq 0\) כאשר יש שיוויון אם"ם \(x=0\). 


  \end{enumerate}
\end{definition}
\begin{definition}[מרחב מכפלה פנימית]
מרחב ווקטורי המצוייד במכפלה פנימית.

\end{definition}
\begin{proposition}[אי שיוויון קושי שוורץ]
יהי \((V,\langle \cdot,\cdot \rangle)\) מרחב מכפלה פנימית. אזי לכל \(x,y \in V\) מתקיים:
$$\lvert \langle x,y \rangle  \rvert \leq  \lVert x \rVert \lVert y \rVert $$

\end{proposition}
\begin{proposition}
יהי \((V,\langle \cdot,\cdot \rangle)\) מרחב מכפלה פנימית. אזי המכפלה הפנימית היא רציפה ביחס למטריקה המושרת מהנורמה.

\end{proposition}
\begin{proof}
יהי \(x_{n}\to x\) ו-\(y_{n}\to y\). אזי:
$$\left\langle x_{n},y_{n}\right\rangle=\left\langle x,y\right\rangle+\left\langle x_{n}-x,y\right\rangle+\left\langle x_{n},y_{n}-y\right\rangle$$
כאשר מאי שיוויון קושי שוורץ נקבל:
$$\big|\left\langle x_{n},y_{n}\right\rangle-\left\langle x,y\right\rangle\big|\leq\big\|x_{n}-x\big\|\big\|y\big\|+\big\|x_{n}\big\|\big\|y_{n}-y\big\|.$$
כאשר כיוון שנורמה במרחב נורמי היא רציפה, אזי \(\lVert x_{n} \rVert\to \lVert x \rVert\) ולכן מאריתמטיקה של גבולות אגף ימין שואף ל-0.

\end{proof}
\begin{proposition}
נורמה \(\lVert \cdot \rVert\) על מרחב ווקטורי \(V\) היא מושרת ממכפלה פנימית אם"ם מקיימת את תנאי המקבילית:
$$\left\|x+y\right\|^{2}+\left\|x-y\right\|^{2}=2{\bigl(}\|x\|^{2}+\left\|y\right\|^{2}{\bigr)}$$
כאשר במקרה זה המכפלה הפנימית נתונה על ידי זהות הפולריזציה:
$$\left\langle x,y\right\rangle={\frac{1}{2}}\left(\left\|x+y\right\|^{2}-\left\|x\right\|^{2}-\left\|y\right\|^{2}\right)$$

\end{proposition}
\begin{definition}[מרחב מכפלה פנימי שלם]
מרחב מכפלה פנימי אשר שלם ביחס לנורמה המושרת. מרחב כזה נקרא מרחב הילברט.

\end{definition}
\section{מערכת אורתונורמלית וטור פורייה כללי}

\begin{definition}[מערכת אורתונורמלית]
יהי \(\left( V,\langle , \rangle \right)\) מרחב מכפלה פנימית. מערכת \(\{ v_{n} \}\subseteq V\) נקראת אורתונורמלית אם לכל \(v_{i},v_{j}\in \{ v_{n} \}\) מתקיים:
$$\langle v_{i},v_{j} \rangle =\delta_{ij}$$

\end{definition}
\begin{lemma}
עבור מערכת אורתונורמלית \(\{ e_{n} \}\) ו-\(v \in V\) מתקיים:
$$\left\lVert  v-\sum_{n=1}^{N} \alpha_{n}e_{n}  \right\rVert =\|v\|^{2}+\sum_{n=1}^{N}\left|x_{n}-\alpha_{n}\right|^{2}-\sum_{n=1}^{N}\left|x_{n}\right|^{2}$$
כאשר \(x_{n}=\langle e_{n},v \rangle\).

\end{lemma}
\begin{proof}
מתקיים:
\begin{gather*}\left\lVert  v-\sum_{n=1}^{N} \alpha_{n}e_{n}  \right\rVert =\left\langle v-\sum_{n=1}^{N}\alpha_{n}e_{n},v-\sum_{n=1}^{N}\alpha_{n}e_{n}\right\rangle= \\=\left\langle  v,v-\sum_{n=1}^{N} \alpha_{n}e_{n}  \right\rangle -\left\langle  \sum_{n=1}^{N} \alpha_{n}e_{n},v-\sum_{n=1}^{N} \alpha_{n}e_{n}  \right\rangle = \\= \lVert v \rVert^{2}-\left\langle  v, \sum_{n=1}^{N} \alpha_{n}e_{n}  \right\rangle  -\left\langle  \sum_{n=1}^{N} \alpha_{n}e_{n},v  \right\rangle+\left\langle  \sum_{n=1}^{N} \alpha_{n}e_{n},\sum_{n=1}^{N} \alpha_{n}e_{n}  \right\rangle  = \\=\lVert v \rVert ^{2}-\sum_{n=1}^{N}  \overline{\alpha} _{n}\langle v,e_{n} \rangle -\sum_{n=1}^{N} \alpha_{n}\langle e_{n},v \rangle +\sum_{n=1}^{N}\sum_{m=1}^{N}\alpha_{n}\overline{\alpha_{m}} \underbracket{ \langle e_{n},e_{m} \rangle }_{ \delta_{nm} } = \\=\lVert v \rVert ^{2}-\sum_{n=1}^{N} \left[  \overline{a_{n}} \overline{\langle e_{n},v \rangle }  +\alpha_{n}\langle e_{n},v \rangle -\alpha_{n}\overline{\alpha_{n}}   \right]
\end{gather*}
כעת נסמן \(x_{n}=\langle e_{n},v \rangle\) ונשים לב כי:
$$\left\lvert  x_{n}-\alpha_{n}  \right\rvert ^{2}=\left( x_{n}-\alpha_{n} \right)\left( \overline{x_{n}} -\overline{\alpha_{n}}  \right)=\overbrace{ x_{n}\overline{x_{n}} }^{ \lVert x_{n} \rVert  } -x_{n}\overline{\alpha_{n}} -\alpha_{n}\overline{x_{n}} +\overbrace{ \alpha_{n}\overline{\alpha_{n}} }^{ \left\lVert  \alpha_{n}  \right\rVert  } $$
ולכן נקבל:
$$\left\lVert  v-\sum_{n=1}^{N} c_{n}e_{n}  \right\rVert =
\|v\|^{2}+\sum_{n=1}^{N}\left|x_{n}-\alpha_{n}\right|^{2}-\sum_{n=1}^{N}\left|x_{n}\right|^{2}$$

\end{proof}
\begin{proposition}[אי שיוויון בסל]
יהי \(V\) מרחב הילברט(מרחב מכפלה פנימית שלם). נניח כי \(\{ e_{n} \}\) מערכת אורתונורמלית ב-\(H\). אזי לכל \(v \in V\) מתקיים:
$$\sum_{k=1}^{\infty}|\langle v,e_{k}\rangle|^{2}\leq\|v\|^{2}$$

\end{proposition}
\begin{proof}
\begin{gather*}{0\leq\left\lVert v-\sum_{k=1}^{n}\langle x,e_{k}\rangle e_{k}\right\rVert ^{2}=\|v\|^{2}-2\sum_{k=1}^{n}\operatorname{Re}\langle v,\langle v,e_{k}\rangle e_{k}\rangle+\sum_{k=1}^{n}|\langle v,e_{k}\rangle|^{2}}\\ {=\|v\|^{2}-2\sum_{k=1}^{n}|\langle v,e_{k}\rangle|^{2}+\sum_{k=1}^{n}|\langle v,e_{k}\rangle|^{2}}\\ {=\|v\|^{2}-\sum_{k=1}^{n}|\langle v,e_{k}\rangle|^{2}}\end{gather*}

\end{proof}
\begin{definition}[טור פורייה לפי מערכת אורתוגונלית במרחב מכפלה פנימית]
יהי \((V,\langle \cdot,\cdot \rangle)\) מרחב מרפלה פנימית \(V\) מעל שדה \(\mathbb{C}\). יהי \((v_{n})^{\infty}_{n=1}\) מערכת אורתונורמלית במרחב \((V,\langle \cdot,\cdot \rangle)\). אזי הטור:
$$\sum_{i=1}^{\infty} \frac{\langle v_{n},v \rangle}{\lVert v_{n} \rVert ^{2}}v_{n} $$
נקרא טור פורייה עבור \(v\). כאשר המקדמים של \(v_{n}\) נקראים מקדמי פורייה

\end{definition}
\begin{proposition}[תכונת הקירוב הטוב ביותר של מקדמי פורייה]
יהי \((V,\langle , \rangle)\) מרחב מכפלה פנימית. תהי \((v_{n})_{n=1}^{\infty}\) רשימה אורתוגונאלית, יהי \(N \in \mathbb{N}\) ויהי \(v \in V\) אזי:
$$\min _{d_{1},\dots,d_{n}\in \mathbb{C}}\left\lVert  v-\sum_{n=1}^{N} d_{n}v_{n}  \right\rVert = \left\lVert  v-\sum_{n=1}^{N} \frac{\langle v_{n},v \rangle}{\lVert v_{n}^{2} \rVert }v_{n}   \right\rVert $$
כלומר המקדמים \(d_{n}\) אשר ממזערים את הנורמה יהיו \(d_{n}=\frac{\langle v_{n},v \rangle}{\lVert v_{n}^{2} \rVert}\).

\end{proposition}
\begin{proof}
נסתכל על הנורמה הבאה:
$$\left\lVert v-\sum_{n=1}^{N}d_{n}v_{n}\right\rVert ^{2}=\left\langle v-\sum_{n=1}^{N}d_{n}v_{n},v-\sum_{m=1}^{N}d_{m}v_{m}\right\rangle$$
נפתח את המכפלה הפנימית:
$$=\langle v,v\rangle-\left\langle v,\sum_{m=1}^{N}d_{m}v_{m}\right\rangle-\left\langle\sum_{n=1}^{N}d_{n}v_{n},v\right\rangle+\left\langle\sum_{n=1}^{N}d_{n}v_{n},\sum_{m=1}^{N}d_{m}v_{m}\right\rangle$$
מהלינאריות של המכפלה הפנימית ניתן להוציא את הסכום, תוך כדי שימוש בתכונה \(\langle x,\alpha y \rangle=\overline{\alpha}\langle x,y \rangle\) נקבל:
$$=||v||^{2}-\sum_{m=1}^{N}\overline{{{d_{m}}}}\langle v,v_{m}\rangle-\sum_{n=1}^{N}d_{n}\langle v_{n},v\rangle+\sum_{n=1}^{N}\sum_{m=1}^{N}d_{n}\overline{{{d_{m}}}}\langle v_{n},v_{m}\rangle$$
מאורתוגונאליות נקבל \(\langle v_{n},v_{m} \rangle=\delta_{nm}\lVert v_{n} \rVert^{2}\) ולכן ניתן לפשט את הסכום אחרון:
$$=||v||^{2}-\sum_{n=1}^{N}\overline{{{d_{n}}}}\langle v,v_{n}\rangle-\sum_{n=1}^{N}d_{n}\langle v_{n},v\rangle+\sum_{n=1}^{N}d_{n}\overline{{{d_{n}}}}||v_{n}||^{2}$$
נגדיר את המקדם פורייה:
$$c_{n}=\frac{\langle v_{n},v \rangle}{\lVert v_{n} \rVert ^{2}}\implies \langle v_{n},v \rangle =c_{n}\lVert v_{n} \rVert ^{2}\implies \langle v,v_{n} \rangle =\overline{c_{n}} \lVert v_{n} \rVert ^{2}$$
ונוסיף ונחסיר מהביטוי שלנו \(\sum_{n=1}^{N}\lvert c_{n}\lVert^{2} v_{n} \rVert^{2} \rvert\) כך שנקבל:
$$=||v||^{2}+\sum_{n=1}^{N}|d_{n}|^{2}||v_{n}||^{2}-\sum_{n=1}^{N}d_{n}\big(c_{n}||v_{n}||^{2}\big)-\sum_{n=1}^{N}\overline{{{d_{n}}}}\big(\overline{{{c_{n}}}}||v_{n}||^{2}\big)+\sum_{n=1}^{N}|c_{n}|^{2}||v_{n}||^{2}-\sum_{n=1}^{N}|c_{n}|^{2}||v_{n}||^{2}\big)$$
נשים לב כי כל הגורמים בסכום מכילים \(\lVert v_{n} \rVert^{2}\) וניתן לקחת אותו כגורם משותף:
$$=||v||^{2}+\sum_{n=1}^{N}||v_{n}||^{2}\left(|d_{n}|^{2}-d_{n}c_{n}-\overline{{{d_{n}c_{n}}}}+|c_{n}|^{2}\right)-\sum_{n=1}^{N}|c_{n}|^{2}||v_{n}||^{2}$$
נבחין כי:
$$|d_{n}|^{2}-d_{n}c_{n}-{\overline{{d_{n}c_{n}}}}+|c_{n}|^{2}=(d_{n}-c_{n})({\overline{{d_{n}}}}-{\overline{{c_{n}}}})=|d_{n}-c_{n}|^{2}$$
ולכן:
$$\left\lVert v-\sum_{n=1}^{N}d_{n}v_{n}\right\rVert ^{2}=||v||^{2}+\sum_{n=1}^{N}|d_{n}-c_{n}|^{2}||v_{n}||^{2}-\sum_{n=1}^{N}|c_{n}|^{2}||v_{n}||^{2}$$
כדי למזער את הביטוי נדרש למזער את \(\textstyle\sum_{n=1}^{N}|d_{n}-c_{n}|^{2}||v_{n}||^{2}\). זה קורה כאשר \(c_{n}=d_{n}\) לכל \(n \in \{ 1,\dots ,N \}\). נקבל שהערך המינימלי יהיה:
$$\operatorname*{min}\left\lVert v-\sum_{n=1}^{N}d_{n}v_{n}\right\rVert ^{2}=||v||^{2}-\sum_{n=1}^{N}|c_{n}|^{2}||v_{n}||^{2}$$
כאשר נזכור כי \(c_{n}=\frac{\langle v_{n},v \rangle}{\lVert v_{n} \rVert}=d_{n}\) ולכן:
$$\operatorname*{min}_{d_{1},\ldots,d_{N}\in C}\left\lVert v-\sum_{n=1}^{N}d_{n}v_{n}\right\rVert =\left\lVert v-\sum_{n=1}^{N}{\frac{\langle v_{n},v\rangle}{\|v_{n}\|^{2}}}v_{n}\right\rVert $$

\end{proof}
\begin{definition}[מערכת אורתונורמלית שלמה]
הפרוש שלו צפוף במרחב. כלומר אם \(V\) מרחב אז מערכת \(\{ e_{n} \}\) תהיה שלמה אם:
$$\overline{{{\mathrm{Span}\{e_{n}\}}}}\;=\;V\;$$
או לחלופין לכל \(v \in V\) ולכל \(\varepsilon>0\) קיים \(N \in \mathbb{N}\) ומקדמים \(\alpha_{1},\dots,\alpha_{N}\in \mathbb{C}\) כך ש:
$$\Bigl\|\,v-\sum_{n=1}^{N}\alpha_{n}e_{n}\Bigr\|<\epsilon$$

\end{definition}
\begin{proposition}
נניח כי \(\{ e_{n} \}_{n=1}^{\infty}\) מערכת אורונורמלית במרחב מכפלה פנימי \(V\). התנאים הבאים שקולים:

  \begin{enumerate}
    \item המערכת \(\{ e_{n} \}_{n=1}^{\infty}\) שלמה ב-\(V\). 


    \item לכל \(v \in V\) מתקיים שיוויון פרסבל: 
$$\lVert v \rVert ^{2}=\sum_{n=1}^{\infty}\lvert \langle e_{n},v \rangle  \rvert ^{2}$$


    \item לכל \(v \in V\) מתקיים: 
$$v=\sum_{n=1}^{\infty} \langle e_{n},v \rangle e_{n}$$


  \end{enumerate}
\end{proposition}
\begin{proof}
\(2\impliedby 1\)

  \begin{enumerate}
    \item יהי \(\varepsilon> 0\). נניח \(\{ e_{n} \}\) מערכת שלמה ו-\(v \in V\). לכן: 
$$\left\lVert v-\sum_{n=1}^{N}\alpha_{n}e_{n}\right\rVert ^{2}<\varepsilon^{2}$$


    \item נסמן \(x_{n}=\langle e_{n},v \rangle\) ונקבל: 
$$\left\lVert v-\sum_{n=1}^{N}\alpha_{n}e_{n}\right\rVert ^{2}=\|v\|^{2}+\sum_{n=1}^{N}|x_{n}-\alpha_{n}|^{2}-\sum_{n=1}^{N}|x_{n}|^{2}$$
כך שמתקיים:
$$\|v\|^{2}-\sum_{n=1}^{N}\left|x_{n}\right|^{2}\leq\left\lVert v-\sum_{n=1}^{N}\alpha_{n}e_{n}\right\rVert ^{2}<\varepsilon^{2}$$
כאשר את הכיוון השני נקבל מאי שיוויון בסל:
$$\lVert v \rVert^{2}-\sum_{n=1}^{N}\left|x_{n}\right|^{2}\geq\lVert v \rVert^{2}-\sum_{n=1}^{\infty}\left|x_{n}\right|^{2}\geq0$$
כלומר סה"כ:
$$0\leq\|v\|^{2}-\sum_{n=1}^{N}\left|x_{n}\right|^{2}\leq\varepsilon^{2}\implies \lVert v \rVert =\sum_{n=1}^{\infty}\left|x_{n}\right|^{2}=\sum_{n=1}^{\infty}\left|\langle e_{n},v\rangle\right|^{2}$$


  \end{enumerate}
\end{proof}
\begin{proof}
\(3\impliedby 2\)
נניח כי מתקיים שיוויון פרסבל. מתקיים:
$$\left\lVert  v-\sum_{n=1}^{N} \alpha_{n}e_{n}  \right\rVert =\|v\|^{2}+\sum_{n=1}^{N}\left|x_{n}-\alpha_{n}\right|^{2}-\sum_{n=1}^{N}\left|x_{n}\right|^{2}$$
כאשר סימנו \(x_{n}=\langle e_{n},v \rangle\). בפרט עבור \(x_{n}=\alpha_{n}\) נקבל:
$$\left\lVert v-\sum_{n=1}^{N}\left\langle e_{n},v\right\rangle e_{n}\right\rVert ^{2}=\left\langle v-\sum_{n=1}^{N}x_{n}e_{n},v-\sum_{n=1}^{N}x_{n}e_{n}\right\rangle=\lVert v \rVert^{2}-\sum_{n=1}^{N}\left|x_{n}\right|^{2}$$
כעת משיוויון פרסבל אנו יודעים כי כאשר \(N\to \infty\) נקבל \(\lVert v \rVert^{2}-\sum_{n=1}^{N}\left|x_{n}\right|^{2}\to 0\). לכן כיוון שהנורמה הולכת לאפס מרציפות הנורמה נקבל:
$$v=\sum_{n=1}^{N}\left\langle e_{n},v\right\rangle e_{n}$$

\end{proof}
\begin{proof}
\(1\impliedby 3\)
כיוון שלכל \(v\) מתקיים \(v=\sum_{n=1}^{\infty}\left\langle e_{n},v\right\rangle e_{n}\) נקבל כי \(v \in\text{Span}\left( \{ e_{n} \} \right)\) ולכן \(V=\overline{\text{Span}\left( \{ e_{n} \} \right)}\).

\end{proof}
\begin{definition}[מרחק]
תהי \((V,\langle , \rangle)\) מרחב מכפלה פנימית ו-\(U\subseteq V\). המרחק מוגדר על ידי:
$$\text{Distance}(f,U)=\inf_{\psi \in U} \lVert f-\psi \rVert $$

\end{definition}
\begin{proposition}
יהי \((V,\langle , \rangle)\) מרחב מכפלה פנימי שלם(מרחב הילברט). תהי \(\varnothing\neq U\subseteq V\) סגורה וקמורה. לכל \(f \in V\) נגדיר. אזי קיים \(g \in U\) יחיד כך ש:
$$\text{Distance}(f,U)=\lVert f-g \rVert $$

\end{proposition}
\begin{proof}
\textbf{קיום:} נגדיר:
$$d= \text{Distance}(f,U)=\operatorname*{inf}_{\psi\in U}||f-\psi||$$
מהגדרת האינפימום קיימת סדרת נקודות \(\{ g_{n} \}_{n=1}^{\infty}\) ב-\(U\) כך ש:
$$\lVert f-g_{n} \rVert \xrightarrow{n\to \infty} d$$
נרצה להראות כי הסדרה \(\{ g_{n} \}\) היא קושי. מכלל המקבלית אנו יודעים כי לכל \(x,y \in V\) מתקיים:
$$||x+y||^{2}+||x-y||^{2}=2||x||^{2}+2||y||^{2}$$
ולכן עבור \(x=f-g_{n}\) ו-\(y=f-g_{m}\) נקבל:
$$||2f-(g_{n}+g_{m})||^{2}+||g_{m}-g_{n}||^{2}=2||f-g_{n}||^{2}+2||f-g_{m}||^{2}$$
כלומר:
$$||g_{m}-g_{n}||^{2}=2||f-g_{n}||^{2}+2||f-g_{m}||^{2}-||2f-\left(g_{n}+g_{m}\right)||^{2}$$
נסתכל על הגורם \(||2f-\left(g_{n}+g_{m}\right)||^{2}\). מתקיים:
$$||2f-(g_{n}+g_{m})||^{2}=\left\lVert 2\left(f-{\frac{g_{n}+g_{m}}{2}}\right)\right\lVert^{2}=4\lVert f-{\frac{g_{n}+g_{m}}{2}} \rVert ^{2}$$
כאשר כיוון ש-\(U\) קמור נקבל \(\frac{g_{n}+g_{m}}{2}\in U\). ולכן כיוון ש-\(d\) הוא המרחק המינימלי נקבל:
$$\left\lVert f-{\frac{g_{n}+g_{m}}{2}}\right\rVert \geq d\implies4\left\lVert f-{\frac{g_{n}+g_{m}}{2}}\right\rVert ^{2}\geq4d^{2}\implies||2f-(g_{n}+g_{m})||^{2}\geq4d^{2}$$
ולכן אם נציב חזרה בשיוויון נקבל:
$$||g_{m}-g_{n}||^{2}\leq2||f-g_{n}||^{2}+2||f-g_{m}||^{2}-4d^{2}$$
אנו יודעים כי עבור \(n,m\to \infty\) מתקיים \(\lVert f-g_{n} \rVert\to d\) ו-\(\lVert f-g_{m} \rVert\to d\) כלומר:
$$2||f-g_{n}||^{2}+2||f-g_{m}||^{2}-4d^{2}\to2d^{2}+2d^{2}-4d^{2}=0$$
ולכן \(\lVert g_{m}-g_{n} \rVert\to 0\) והסדרה קושי. לכן הסדרה \(\{ g_{n} \}\) מתכנסת לאיזשהו גבול \(g\). כיוון שהקבוצה סגורה נקבל \(g \in U\) ומרציפות הנורמה נקבל:
$$||f-g||=\operatorname*{lim}_{n\to\infty}||f-g_{n}||=d$$\textbf{יחידות:}
נניח בשלילה שיש \(g_{1},g_{2} \in U\) שמשיגות את המינימום, כלומר:
$$\lVert f-g_{1} \rVert=d\qquad \lVert f-g_{2} \rVert=d$$
מכלל המקבילית עבור \(x=f-g_{1}\) ו-\(y=f-g_{2}\) נקבל:
\begin{gather*}||(f-g_{1})+(f-g_{2})||^{2}+||(f-g_{1})-(f-g_{2})||^{2}=2||f-g_{1}||^{2}+2||f-g_{2}||^{2}\implies\\\implies||2f-{\big(}g_{1}+g_{2}{\big)}||^{2}+||g_{2}-g_{1}||^{2}=2d^{2}+2d^{2}=4d^{2}\implies \\ \implies ||g_{2}-g_{1}||^{2}=4d^{2}-||2f-(g_{1}+g_{2})||^{2}
\end{gather*}
כאשר נשים לב כי:
$$||2f-(g_{1}+g_{2})||^{2}=4\,||f-{\textstyle\frac{g_{1}+g_{2}}{2}}||^{2}$$
וכיוון ש-\(U\) קמור נקבל ש-\(\frac{g_{1}+g_{2}}{2}\in U\) ולכן מהגדרת האינפימום \(\left\lVert f-{\frac{g_{1}+g_{2}}{2}}\right\rVert \geq d\). נציב חזרה במשוואה:
$$||g_{2}-g_{1}||^{2}=4d^{2}-||2f-(g_{1}+g_{2})||^{2}\leq4d^{2}-4d^{2}=0$$
וקיבלנו שהמרחק הוא 0, ולכן \(g_{1}=g_{2}\).

\end{proof}
\section{טור פורייה טריגונומטרי - ממשי ומרוכב}

\begin{proposition}
המערכת \(\{1,\cos(n x),\sin(n x)\}_{n\geq1}\) היא מערכת אורתונואמלית שלמה

\end{proposition}
\section{עקרון הלוקאליות והתכנסות נקודתית של טור פורייה}

\begin{proposition}[עקרון הלוקאליות של טור פוריה ביחס למערכת הטריגונומטרית האורתונורמלית]
תהי \(f:\mathbb{R}\to \mathbb{R}\) אינטגרבילית רימן ב-\([-\pi,\pi]\) ומחזורית עם מחזור \(2\pi\). נגדיר:
$$S_{N}(x)=\frac{a_{0}}{2}+\sum_{n=1}^{N} a_{n}\cos(nx)+\sum_{n=1}^{N} b_{n}\sin(nx)$$
כאשר:
$$a_{n}=\frac{1}{\pi}\int_{-\pi}^{\pi} f(t)\cos(nt) \, \mathrm{d}t\qquad b_{n}=\frac{1}{\pi}\int_{-\pi}^{\pi} f(t)\sin(nt) \, \mathrm{d}t  $$
אזי לכל \(0<\delta \leq \pi\) מתקיים:
$$\lim_{ N \to \infty } \left[ S_{N}(x_{0})-\frac{1}{\pi}\int_{0}^{\delta} (f(x_{0}+u)-f(x_{0}-u))D_{N}(u) \, \mathrm{d}u  \right]=0$$

\end{proposition}
\begin{lemma}[הלמה של רימן]
עבור \(f\) כנ"ל מתקיים:
$$\lim_{ \omega \to \infty } \int_{-\pi}^{\pi} f(t)\sin(\omega t) \, \mathrm{d}t=0 $$

\end{lemma}
יש דרכים יותר פשוטות להוכיח את זה אבל לשם החוויה נוכיח עם משפט הקירוב של ווירשטראס.

\begin{proof}
ראשית נראה את הטענה עבור פולינומים. יהי \(P(t)\) פולינום מסדר \(n\). נראה כי:
$$\operatorname*{lim}_{x\to\infty}\int_{a}^{b}P(t)\sin(x t)d t=0$$
בעזרת אינדוקציה על דרגת הפולינום. עבור הבסיס של האינטרוקציה נניח כי \(n=0\) ולכן הפולינום יהיה מהצורה \(P(x)=c\). כעת האינטגרל הנתון יהיה:
$$\int_{a}^{b}c\sin(x t)d t=c\left[-{\frac{\cos(x t)}{x}}\right]_{a}^{b}=c\left(-{\frac{\cos(x b)}{x}}+{\frac{\cos(x a)}{x}}\right)={\frac{c}{x}}(\cos(x a)-\cos(x b))$$
כאשר אם ניקח את הגבול \(x\to \infty\) נקבל פונקציה חסומה כפול אפסה ולכן מתאפס והטענה מתקיימת.
כעת נניח כי מתקיים עבור כל הפולינומים עד דרגה \(k\). כלומר לכל פולינום \(Q(t)\) מדרגה לכל היותר \(k\) מתקיים:
$$\operatorname*{lim}_{x\to\infty}\int_{a}^{b}Q(t)\sin(x t)d t=0$$
יהי \(P(t)\) פולינום מדרגה \(k+1\). ניתן לכתוב \(P(t)=a\cdot t^{k+1}+Q(t)\) כאשר \(a\) זה המקדם המוביל ו-\(Q(t)\) זה פולינום מדרגה של לכל היותר \(k\). מלינאריות האינטגרל האינטגרל על \(Q(t)\) מתאפס ולכן מספיק להראות כי:
$$\operatorname*{lim}_{x\to\infty}\int_{a}^{b} a \cdot t^{k+1}\sin(x t)d t=0 \iff\operatorname*{lim}_{x\to\infty}\int_{a}^{b}t^{k+1}\sin(x t)d t=0$$
מאינטגרציה בחלקים נקבל:
\begin{gather*}\int_{a}^{b}t^{k+1}\sin(x t)d t=\left[-t^{k+1}{\frac{\cos(x t)}{x}}\right]_{a}^{b}-\int_{a}^{b}\left(-{\frac{\cos(x t)}{x}}\right)(k+1)t^{k}d t \\=\left(-{\frac{b^{k+1}\cos(x b)}{x}}+{\frac{a^{k+1}\cos(x a)}{x}}\right)+{\frac{k+1}{x}}\int_{a}^{b}t^{k}\cos(x t)d t
\end{gather*}
נשים לב כי הביטוי האינטגרלי מתאפס מהנחה אינדוקציה, כאשר הגבול:
$$\operatorname*{lim}_{x\to\infty}\left(-{\frac{b^{k+1}\cos(x b)}{x}}+{\frac{a^{k+1}\cos(x a)}{x}}\right)$$
מתאפס כיוון שכל אחד מהנסכמים מתאפס מחסומה כפול אפסה. לכן הטענה מתקיימת עבור פולינום כללי.
כעת נראה נכונות עבור פונקציה כללי. יהי \(\varepsilon> 0\). ממשפט ווירשטראס ניתן לקרב על ידי פולינום \(P(t)\) במידה שווה, כלומר עבור \(\frac{\varepsilon}{2(b-a)}\) נקבל כי:
$$||f-P||_{\infty}<\frac{\epsilon}{2(b-a)} \iff \forall t \in [a,b]\quad |f(t)-P(t)|<{\frac{\epsilon}{2(b-a)}}$$
כעת מתקיים:
\begin{gather*}\lvert  \int_{a}^{b}{f(t)\sin(x t)d t} \rvert=\lvert \int_{a}^{b}(f(t)-P(t)+P(t))\sin(x t)d t \rvert \leq  \\\leq \lvert \int_{a}^{b}(f(t)-P(t))\sin(x t)d t \rvert +\lvert \int_{a}^{b}P(t)\sin(x t)d t \rvert
\end{gather*}
כאשר נשים לב כי:
$$\left|\int_{a}^{b}(f(t)-P(t))\sin(x t)d t\right|\leq\int_{a}^{b}|f(t)-P(t)||\sin(x t)|d t<\int_{a}^{b}{\frac{\epsilon}{2(b-a)}}\cdot1d t={\frac{\epsilon}{2(b-a)}}(b-a)={\frac{\epsilon}{2}}$$
כאשר עבור \(\lvert \int_{a}^{b}P(t)\sin(x t)d t \rvert\) נקבל מהגדרת הגבול כי קיים מספר ממשי \(X\) כך שלכל \(x> X\) מתקיים:
$$\left|\int_{a}^{b}P(t)\sin(x t)d t\right|<{\frac{\epsilon}{2}}$$
ולכן נקבל עבור \(x> X\) כי:
$$\left|\int_{a}^{b}f(t)\sin(x t)d t\right|\leq\left|\int_{a}^{b}(f(t)-P(t))\sin(x t)d t\right|+\left|\int_{a}^{b}P(t)\sin(x t)d t\right|<\frac{\varepsilon}{2}+\frac{\varepsilon}{2}=\varepsilon$$
כלומר מהגדרת הגבול קיבלנו כי:
$$\operatorname*{lim}_{x\to\infty}\int_{a}^{b}f(t)\sin(x t)d t=0$$

\end{proof}
\begin{lemma}[נוסחאת דירכלה]
$$S_{N}(x)=\frac{1}{\pi}\int_{0}^{\pi} [f(x+u)+f(x-u)]D_{N}(u) \, \mathrm{d}u $$

\end{lemma}
\begin{proof}
נזכור כי:
$$S_{N}(x_{0})=\frac{1}{\pi}\int_{-\pi}^{\pi}f(x_{0}+u)D_{N}(u)d u$$
נחלק את האינטגרל לשתי תחומים מ-0 ל-\(\pi\) ומ-\(-\pi\) ל-0:
$$S_{N}(x_{0})=\frac{1}{\pi}\int_{-\pi}^{0}f(x_{0}+u)D_{N}(u)d u+\frac{1}{\pi}\int_{0}^{\pi}f(x_{0}+u)D_{N}(u)d u$$
נבצע החלפת משתנים. נגדיר \(v=-u\) ולכן \(du=-dv\):
$$\int_{-\pi}^{0}f(x_{0}+u)D_{N}(u)d u=\int_{\pi}^{0}f(x_{0}-v)D_{N}(-v)(-d v)=\int_{0}^{\pi}f(x_{0}-v)D_{N}(v)d v$$
כאשר במעבר האחרון השתמשנו בזוגיות של גרעין דריכלה. כלומר סה"כ נקבל:
$$S_{N}(x_{0})=\frac{1}{\pi}\int_{0}^{\pi}[f(x_{0}+u)+f(x_{0}-u)]D_{N}(u)d u$$

\end{proof}
כעת נוכיח את המשפט עצמו.

\begin{proof}
מנוסחאת דירכלה מתקיים:
$$S_{N}(x)=\frac{1}{\pi}\int_{0}^{\pi} [f(x+u)+f(x-u)]D_{N}(u) \, \mathrm{d}u $$
ולכן לכל \(0<\delta \leq \pi\) מתקיים:
\begin{gather*}S_{N}(x)-\frac{1}{\pi}\int_{0}^{\delta} [f(x+u)+f(x-u)]D_{N}(u) \, \mathrm{d}u=\\=\frac{1}{\pi}\int_{\delta}^{\pi} [f(x+u)+f(x-u)]D_{N}(u) \, \mathrm{d}u=\\=\frac{1}{\pi}\int_{\delta}^{\pi} \frac{f(x+u)+f(x-u)}{2\sin\left( \frac{u}{2} \right)}\sin\left( \left( N+\frac{1}{2} \right)u \right) \, \mathrm{d}u   
\end{gather*}
נגדיר \(\varphi:[\delta,\pi]\to \mathbb{R}\) על ידי:
$$\varphi(u)=\frac{f(x+u)+f(x-u)}{2\sin\left( \frac{u}{2} \right)}$$
כאשר \(\varphi\) אינטגרבילית ב-\([\delta,\pi]\) ולכן מהלמה של רימן נקבל:
$$\lim_{ N \to \infty } \frac{1}{\pi}\int_{\delta}^{\pi} \varphi(u)\sin\left( \left( N+\frac{1}{2} \right)u \right) \, \mathrm{d}u =0 $$

\end{proof}
\begin{lemma}
$$\frac{1}{\pi} \int_{0}^{\pi} D_N(u) du = \frac{1}{2}$$

\end{lemma}
\begin{proof}
עבור \(f(x)=\frac{1}{2}\) נקבל כי המקדמים של הטור פוריה יהיו \(a_{0}=1,a_{n\geq 1}=b_{n}=0\). כלומר:
$$S_{N}(x)={\frac{a_{0}}{2}}+\sum_{n=1}^{N}(a_{n}\cos(n x)+b_{n}\sin(n x))={\frac{1}{2}}+\sum_{n=1}^{N}(0\cdot\cos(n x)+0\cdot\sin(n x))={\frac{1}{2}}+\sum_{n=1}^{N}(0\cdot\cos(n x)+0\cdot\sin(n x))$$
מהנוסחה של הסכום החלקי נקבל:
$$S_{N}(0)=\frac{1}{\pi}\int_{-\pi}^{\pi}f(0+u)D_{N}(u)d u\implies\frac{1}{2}=\frac{1}{\pi}\int_{-\pi}^{\pi}\frac{1}{2}D_{N}(u)d u$$
וכיוון ש-\(D_{N}(u)\) פונקציה זוגית נקבל:
$$\frac{1}{2}=\frac{1}{\pi}\int_{0}^{\pi}D_{N}(u)d u$$

\end{proof}
\begin{proposition}[משפט ההתכנסות הנקודתית של טור פורייה]
תהי \(f:\mathbb{R}\to \mathbb{R}\) אינטגרבילית רימן ב-\([-\pi,\pi]\) ומחזורית עם מחזור \(2\pi\). תהי \(x_{0} \in \mathbb{R}\) כך ש-\(f\) מקיימת את תנאי ליפשיץ ב-\(x_{0}\). כלומר קיימים הגבולות:
$$\lim_{ x \to x_{0}^{-} } f(x_{0}^{-})=f(x_{0}^{-})\qquad \lim_{ x \to x_{0}^{+} } f(x)=f(x_{0}^{+})$$
וגם \(C>0\) ו-\(0<\delta \leq \pi\) כך ש לכל \(u \in (0,\delta)\) מתקיים:
$$\lvert f(x_{0}+u)-f(x_{0}^{+}) \rvert \leq  Cu\qquad \lvert f(x_{0}-u)-f(x_{0}^{-}) \rvert \leq  Cu$$
אזי:
$$S_{N}(x_{0})\to \frac{1}{2}[f(x_{0}^{-})+f(x_{0}^{+})]$$
בפרט אם \(f\) גזירה מימין ומשמאל בנקודה \(x_{0}\) נקבל \(S_{N}(x_{0})\to f(x_{0})\).

\end{proposition}
\begin{proof}
נסתכל על:
$$S_{N}(x_{0})-\frac{1}{2}[f(x_{0}^{+})+f(x_{0}^{-})]=\frac{1}{\pi}\int_{0}^{\pi}\left([f(x_{0}+u)+f(x_{0}-u)]-[f(x_{0}^{+})+f(x_{0}^{-})]\right)D_{N}(u)\mathrm{d} u$$
כאשר החלפנו את \(\frac{1}{2}\) ב-\(\frac{1}{2}=\frac{1}{\pi}\int_{0}^{\pi}D_{N}(u)d u\). נגדיר:
$$\Phi(u) = \frac{f(x_0+u)+f(x_0-u)-f(x_0^{+})-f(x_0^{-})}{2\sin(u/2)}$$
כך ש:
$$S_{N}(x_{0})-\frac{1}{2}[f(x_{0}^{+})+f(x_{0}^{-})]=\frac{1}{\pi}\int_{0}^{\pi}\Phi(u)\sin\left(\left(N+\frac{1}{2}\right)u\right)d u$$
נרצה להראות את האינטגרביליות של \(\Phi(u)\) ולשם כך נשתמש בליפשיציות. מתקיים מאי שיוויון המשולש:
$$|f(x_{0}+u)-f(x_{0}^{+})+f(x_{0}-u)-f(x_{0}^{-})|\leq|f(x_{0}+u)-f(x_{0}^{+})|+|f(x_{0}-u)-f(x_{0}^{-})|\leq C u+C u=2C u$$
ולכן עבור \(u \in (0,\delta)\) נקבל:
$$|\Phi(u)|=\left|{\frac{f(x_{0}+u)+f(x_{0}-u)-f(x_{0}^{+})-f(x_{0}^{-})}{2\sin(u/2)}}\right|\leq{\frac{2C u}{2\sin(u/2)}}={\frac{C u}{\sin(u/2)}}$$
וכיוון ש-\(\frac{u}{\sin\left( \frac{u}{2} \right)}\xrightarrow{u\to 0} 2\) נקבל כי חסום ולכן מוגדר ואינטגרבילי. 
מהלמה של רימן נקבל:
$$\operatorname*{lim}_{N\to\infty}\int_{0}^{\pi}\Phi(u)\sin\left(\left(N+\frac{1}{2}\right)u\right)d u=0\implies \operatorname*{lim}_{N\to\infty}\left(S_{N}(x_{0})-\frac{1}{2}[f(x_{0}^{+})+f(x_{0}^{-})]\right)=\frac{1}{\pi}\cdot0=0$$
כלומר כאשר \(N\to \infty\) מתקיים:
$$S_{N}(x_{0})\to\frac{1}{2}[f(x_{0}^{-})+f(x_{0}^{+})]$$

\end{proof}
\section{משפט פייר}

\begin{theorem}[פייר]
תהי \(f:\mathbb{R}\to \mathbb{R}\) פונקציה אינטגרבילית עם מחזור \(2\pi\) על \([-\pi,\pi]\). נסמן ב-\(S_{n}(x)\) את הסכום החלקי ה-\(n\) של הטור פורייה. נגדיר:
$$\sigma_{N}(x)\ {\stackrel{\mathrm{def}}{=}}\ {\frac{1}{N+1}}\sum_{n=0}^{N}S_{n}(x)$$
אזי:

  \begin{enumerate}
    \item אם \(f\) רציפה על \(\mathbb{R}\) אז \(\sigma_{N}\to f\) במידה אחידה על \(\mathbb{R}\). 


    \item אם \(f\) רציפה ב-\(x_{0} \in \mathbb{R}\) אזי: 
$$\lim_{ N \to \infty } \sigma_{N}(x_{0})=f(x_{0})$$


  \end{enumerate}
\end{theorem}
תהי \(f\) מ-\(R\) ל-\(R\) פונקציה אינטגרבילית עם מחזור \(2\pi\) על הקטע הסגור ממינוס פאי ל-פאי. אזי אם \(f\) רציפה על \(R\) הטור פייר מתכנס במידה שווה על \(R\).

\begin{lemma}[גרעין פייר]
מתקיים:
$$\sigma_{N}(x)=\frac{1}{\pi}\int_{-\pi}^{\pi}f(x+u)K_{N}(u)d u$$
כאשר:
$$K_{N}(u)={\frac{1}{N+1}}{\frac{\sin^{2}\left({\frac{N+1}{2}}u\right)}{2\sin^{2}\left({\frac{u}{2}}\right)}}$$
ומקיים את התכונות הבאות:

  \begin{enumerate}
    \item אי שלילי. 


    \item אינטגרל בתחום \([-\pi,\pi]\) יהיה 1. 


    \item מרוכז - לכל \(0<\delta\leq \pi\) מקיים: 
$$\int_{|x|>\delta} K_{N}(x)  \, \mathrm{d}x \xrightarrow{N\to \infty} 0$$


  \end{enumerate}
\end{lemma}
נוכיח כעת את המשפט

\begin{proof}
נראה חלק 2 - התכנסות נקודתית. נסתכל על ההפרש:
$$\sigma_{N}(x_{0})-f(x_{0})=\frac{1}{\pi}\int_{-\pi}^{\pi}(f(x_{0}+u)-f(x_{0}))K_{N}(u)d u$$
ולכן מספיק להראות שהאינטגרל מתכנס ל-0. יהי \(\varepsilon> 0\). כיוון ש-\(f\) רציפה סביב \(x_{0}\)  קיים \(\delta> 0\) כך ש:
$$\lvert x_{0}-x \rvert <\delta\implies \lvert f(x_{0})-f(x) \rvert <\varepsilon$$
ולכן:
$$-\delta+x_{0}<x<\delta+x_{0}\implies \varepsilon -f(x_{0})<f(x)<\varepsilon+f(x_{0})$$
ולכן:
$$\int_{x_{0}-\delta}^{x_{0}+\delta}(f(x_{0}+u)-f(x_{0}))K_{N}(u)d u\leq \int_{x_{0}-\delta}^{x_{0}+\delta}(f(x_{0})+\varepsilon-f(x_{0}))K_{N}(u)d u<2\varepsilon \delta$$
ולכן קטן כרצונינו. כעת נסתכל על:
$$\int_{|x_{0}-x|>\delta}(f(x_{0}+u)-f(x_{0}))K_{N}(u)d u-$$

\end{proof}
נחלק את האינטגרל לשתי תחומים - בסביבה \([-\delta,\delta]\) ובסביבת \(x\geq|\delta|\). עבור \(x \in [-\delta,\delta]\) כיוון ש-\(f\) רציפה

\chapter{שאלות}

\section{הגדרות}

?
\textbf{הגדרה} חסימות במידה אחידה - Uniformly Bound
$$\Phi \subseteq C[a,b]$$
נניח שקיים איזשהו פרמטר \(\kappa > 0\) כך שמתקיים:
$$\forall x \in [a,b]\quad \forall \varphi \in \Phi \quad \lvert \varphi(x) \rvert \leq \kappa$$
(חשוב: \(\kappa\) אינו תלוי ב-\(x\) ואינו תלוי ב-\(\varphi\)) אזי \(\Phi\) נקראת חסומה במידה אחידה.

\section{משפטים שצריך להוכיח}

?

\begin{enumerate}
  \item נניח חסום לחלוטין. תהי \((a_{n})\) סדרה כלשהי. נבנה סדרה של קבוצות \(\{ V^{(n)} \}_{n=1}^{\infty}\) באופן רקורסיבי. 


  \item נגדיר את \(V^{(1)}\) באופן הבא. כיוון ש-\(A\) חסומה לחלוטין עבור \(\varepsilon=1\) קיים סדרה סופית של נקודות \(\{ x_{n} \}_{n=1}^{N}\) כך ש: 
$$A\subseteq \bigcup_{i=1}^{N}B_{1}(x_{i})$$


  \item כיוון ש-\((a_{n})\) סדרה אינסופית ו-\(\{ x_{i} \}_{i=1}^{N}\) קבוצה סופית קיים איזשהו \(x_{j}\) כך ש-\(B_{1}(x_{j})\) מכיל מספר אינסופי של נקודות. נגדיר: 
$$V^{(1)}=A\cap  B_{1}(x_{j})$$
כאשר נשים לב כי:
$$\mathrm{Diam}(V^{(1)})\leq \mathrm{Diam}(B_{1}(x_{j}))=2$$


  \item נניח כי נתון \(V^{(k)}\). נגדיר את \(V^{(k+1)}\). כיוון ש-\(V^{(k)}\) חסום לחלוטין עבור \(\varepsilon= \frac{1}{k}\) נקבל סדרה סופית של נקודות \(\{ x_{n} \}_{n=1}^{N}\) של נקודות כך ש-\(V^{(k)}\subseteq \bigcup_{i=1}^{N}B_{\frac{1}{k}}(x_{i})\). כמו מקודם קיימת איזשהו \(x_{j}\) כך ש-\(B_{\frac{1}{k}}(x_{j})\) מכיל מספר אינסופי של נקודות של \((a_{n})\). נגדיר: 
$$V^{(k+1)}=V^{(k)}\cap  B_{\frac{1}{k}}(x_{j})$$
כאשר כעת נשים לב כי:
$$\mathrm{Diam}(V^{(k+1)})\leq \mathrm{Diam}\left( B_{\frac{1}{k}}(x_{j}) \right)=\frac{2}{k}$$
כלומר כל שתי נקודות ב-\(V^{(k+1)}\) יהיו במרחק של לכל היותר \(\frac{2}{k}\). וכן מתקיים:
$$V^{(1)}\supseteq V^{(2)}\supseteq \dots$$


  \item נבנה תת סדרה באופן הבא. נגדיר \(n_{1}\) בתור האיבר הראשון כך ש-\(a_{n_{1}} \in V^{(1)}\)(קיים כזה כי להיות ב-\(V^{(1)}\) זו תכונה שכיחה). כמו כן נגדיר את \(n_{k}\) להיות בתור האיבר הראשון הגדול מ-\(n_{k-1}\) כך ש-\(a_{n_{k}}\in V^{(k)}\)(שוב קיים כזה כי להיות ב-\(V^{(k)}\) זו תכונה שכיחה).  


  \item נראה כי זוהי סדרת קושי. יהי \(\varepsilon> 0\). נבחר \(N>\frac{2}{\varepsilon}\). לכל \(k,m> N\) מתקיים: 
$$\rho(a_{n_{m}},a_{n_{k}})\leq  \mathrm{Diam}\left( V^{(N)} \right)\leq \frac{2}{N}\leq \varepsilon$$


  \item כעת נראה את הכיוון השני. נניח כי כל סדרת נקודות ב-\(A\) כוללת תת סדרת קושי ונראה כי חסום לחלוטין. זה שקול ללהוכיח כי אם לא חסום לחלוטין אז קיים סדרת נקודות אשר לא כוללת תת סדרת קושי. כיוון שלא חסום לחלוטין קיים איזשהו \(\varepsilon_{0}\) כך שלא ניתן לכסות את \(A\) עם מספר סופי של כדורים ברדיוס \(\varepsilon_{0}\). 


  \item נבנה סדרה \((a_{n})\) כזו באופן אינדוקטיבי. יהי \(a_{1} \in A\) נקודה כלשהי. תהי \(a_{2} \in A\) נקודה כך ש: 
$$\rho(a_{1},a_{2})>\varepsilon_{0}$$
כאשר קיים כזו אחרת:
$$A\subseteq B_{\varepsilon_{0}}(a_{1})$$
בסתירה לזה שלא חסום לחלוטין.


  \item כעת אם מוגדרות נקודות \((a_{1},a_{2},\dots,a_{N})\) נגדיר \(a_{N+1}\) על ידי נקודה ב-\(A\) כך שלכל \(1\leq k\leq N\) מתקיים: 
$$\rho(a_{N+1},a_{k})> \varepsilon_{0}$$
כאשר קיים כזו כי אחרת \(A\subseteq{ B_{\varepsilon_{0}}(a_{1}) }\cup \dots \cup {  B_{\varepsilon_{0}}(a_{N}) }\) בסתירה לכך שלא חסום במידה לחלוטין.


  \item לא קיימת תת סדרת קושי כי עבור \(\frac{\varepsilon_{0}}{2}\) ולכל \(m,n \in a_{N}\) מתקיים: 
$$\rho(a_{n},a_{m})>\varepsilon_{0}$$


\end{enumerate}
\end{document}