\documentclass{tstextbook}

\usepackage{amsmath}
\usepackage{amssymb}
\usepackage{graphicx}
\usepackage{hyperref}
\usepackage{xcolor}

\begin{document}

\title{Example Document}
\author{HTML2LaTeX Converter}
\maketitle

\chapter{שדות}

\section{שדות ומרחבים ווקטורים}

\begin{definition}[שדה]
חוג חילוק קומטטיבי(עם יחידה). כלומר חוג קומוטטיבי שבו לכל איבר שונה מ-0 יש הופכי.

\end{definition}
\begin{definition}[המציין של שדה]
הסדר של האיבר 1 בחבורה החיבורית של השדה. עבור שדה אינסופי נגדיר את המיין בתור 0. לעיתים מסומן \(\mathrm{char}(F)\).

\end{definition}
\begin{proposition}[חיתוך של שדות]
החיתוך של שדות יהיה שדה.

\end{proposition}
\begin{proposition}[איחוד של שדות]
איחוד של שתי שדות יהיה שדה אם"ם שדה אחד מכיל את השדה השני.

\end{proposition}
\begin{definition}[מרחב ווקטורי]
קבוצה \(V\) נקראת מרחב ווקטורי מעל שדה \(F\) אם \(V\) היא חבורה אבלית עם פעולת החיבור(המסומנת ב \(+\)) וגם לכל \(a\in F\) ו-\(v \in V\) קיים איבר \(av \in V\) כך שמתקיים לכל \(a,b \in F\) ו-\(u,v \in V\):

  \begin{enumerate}
    \item \(a(v+u)=av+a u\)


    \item \((a+b)v=av+bv\)


    \item \(a(bv)=(a b)v\)


    \item \(1v=v\)


  \end{enumerate}
\end{definition}
כאשר איבר במרחב ווקטורי נקרא ווקטור, ואיברים בשדה נקראים סקלארים.
\textbf{הגדרה} תת מרחב ווקטורי
קבוצה \(W\subseteq V\) נקראת תת מרחב ווקטורי אם \(W\) מרחב ווקטורי בפני עצמו

\begin{definition}[תלות לינארית]
קבוצה \(S\) של ווקטורים נקראת תלויה לינארית מעל שדה \(F\) אם קיימים וקטורים \(v_{1},v_{2},\dots,v_{n}\in S\) ואיברים \(a_{1},a_{2},\dots,a_{n}\in F\) לא כולם אפס כך ש-\(a_{1}v_{1}+\dots+a_{n}v_{n}=0\). 

\end{definition}
כאשר אם קבוצה היא לא תלוייה לינארית היא נקראת בלתי תלוייה לינארית.

\begin{definition}[בסיס]
יהי \(V\) מרחב ווקטורי מעל \(F\). תת קבוצה \(B\) של \(V\) נקרא בסיס של \(V\) אם \(B\) בלתי תלוי לינארי מעל \(F\) וכל איבר ב-\(V\) הוא צירוף לינארי של האיברים ב-\(B\).

\end{definition}
\begin{proposition}
כל הבסיסים של המרחב הם באותו הגודל.

\end{proposition}
\begin{definition}[המימד של מרחב ווקטורי.]
גודל הבסיס המרחב הם באותו הגודל.

\end{definition}
\section{הרחבת שדות}

\begin{definition}[הרחבת שדות]
שתי שדות \(F\subseteq E\) כך ש-\(F\) הוא תת שדה של \(E\). מסומן \(E / F\).

\end{definition}
\begin{proposition}
להרחבת שדות יש מבנה של מרחב ווקטורי.

\end{proposition}
\begin{definition}[מימד ההרחבה]
המימד של המרחב הווקטורי שנוצר ע"י הרחבת השדות. מסומן \([E:F]\).

\end{definition}
\begin{definition}[הרחבה אינסופית]
הרחבה שהמימד הרחבה שלה הוא אינסופי.

\end{definition}
\begin{definition}[הרחבת שדות סופית]
תת שדות אשר נוצרת על ידי כמות ספית של איברים

\end{definition}
\begin{definition}[תת שדה הנוצר על ידי איברים]
עבור הרחבת שדות \(E / F\) נסמן את התת שדה המנימלי של \(E\) שמכיל את \(F\) ואת \(\alpha_{1},\dots,\alpha_{n}\in E\) ע"י \(F\left( \alpha_{1},\dots,\alpha_{n} \right)\)

\end{definition}
\begin{remark}
יש הבדל משמעותי בין הקבוצת יוצרים לבין הבסיס של המרחב הווקטורי.

\end{remark}
\begin{definition}[הרחבה אלגברית]
יהי \(E / F\) הרחבת שדות. אם עבור \(\alpha \in E\) קיים פולינום \(0 \neq f \in F\) כך ש-\(f\left( \alpha \right)=0\), נקרא ל-\(\alpha\) אלגברי.

\end{definition}
\begin{definition}[הרחבה טרנזנדנטית]
הרחבה שאינה אלגברית

\end{definition}
\begin{proposition}
הרחבה היא סופית אם"ם נוצר סופי ואלגברי

\end{proposition}
\begin{proof}
נניח \(E / F\) הרחבה סופית. לכן קיים עבורו בסיס \(\alpha_{1},\dots,\alpha_{m} \in E\). מתקיים \(E=F\left( \alpha_{1},\dots,\alpha_{m} \right)\) ולכן \(E / F\) נוצר סופית.

\end{proof}
כעת נניח נוצר סופי ואלגברי. לכן קיימים \(\alpha_{1},\dots,\alpha_{m} \in E\) אלגברים כך ש-\(E=F\left( \alpha_{1},\dots,\alpha_{m} \right)\). אנו יודעים כי:
$$[E:F]=\prod_{i=1}^{n}\left[ F\left( \alpha_{1},\dots,\alpha_{i} \right):F\left( \alpha_{1},\dots \alpha_{i-1} \right) \right]$$
כאשר כל אחת ההרחבות היא סופית כיוון שזו הרחבה פשוטה של איבר אלגברי, ולכן בפרט המכפלה שלהם פשוטה.

\begin{proposition}
אם \(E / F\) הרחבת שדות, ו-\(\alpha,\beta \in E\) אלגבריים מעל \(F\), אזי \(\alpha\pm \beta,\alpha \beta,\frac{\alpha}{\beta}\in E\) אלגברים.

\end{proposition}
\begin{proof}
אנו יודעים כי \(F\left( \alpha,\beta \right)\) הרחבה שנוצרת מאיברים אלגברים ולכן בפרט נוצרת סופית ואלגברית

\end{proof}
\begin{proposition}
אם \(F\subseteq E \subseteq K\)  הרחברת שדות אם \(K / E\) ו-\(E / F\) אלגברית, אז \(K / F\) אלגבריות.

\end{proposition}
\begin{proof}
יהי \(\alpha \in K\) ויהי \(m_{\alpha}=\sum_{i=0}^{n} \beta_{i}x^{i}\) כאשר \(\beta_{i} \in E\) הפולינום המינימלי של \(\alpha\) מעל \(E\). נסמן \(S=F\left( \beta_{1},\dots,\beta_{n} \right)\). מהטענה הקודמת \(S / F\) הרחבה סופית. .נקבל כי
$$\left[ F\left( \alpha \right):F \right]$$

\end{proof}
\begin{definition}[פולינום מינימלי]
מוגדר עבור איבר אלגברי \(\alpha \in E\). זהו הפולינום המתוקן \(m_{\alpha}\in F[x]\) מהמעלה הנמוכה ביותר שמקיים \(m_{\alpha}\left( \alpha \right)=0\).

\end{definition}
\begin{proposition}
הפולינום המינימלי הוא אי פריק

\end{proposition}
\begin{proposition}
אם יש פולינום אי פריק שמאפס את האיבר, אז הוא יהיה הפולינום המינימלי.

\end{proposition}
\begin{proposition}
מתקיים \(F\left( \alpha \right)\cong F[x] / \left( m_{\alpha}(x) \right)\). כלומר השדה התת שדה שנוצר ע"י \(\alpha\) איזומורפי להרחבת שדות של האידיאל שנוצר ע"י \(m_{\alpha}(x)\) עם הפולינומים עם מקדמים ב-\(F\).

\end{proposition}
\begin{proposition}
התת שדה הנוצר מקיים:
$$\left[ F\left( \alpha \right):F \right]=\deg\left( \alpha \right)=\deg\left( m_{\alpha}(x) \right)$$

\end{proposition}
\section{אוטומורפיזם של שדות}

\begin{definition}[אוטומורפיזם של שדות]
יהי \(F\) שדה. פונקציה הפיכה \(\phi:F\to F\) תקרא אוטומורפיזם של שדות אם לכל \(a,b\in F\) מתקיים:
$$\phi(a+b)=\phi(a)+\phi(b)\qquad \phi(ab)=\phi(a)\phi(b)$$

\end{definition}
כלומר \(\phi\) משמר את המבנה של השדה. אוטומורפיזם של שדות שקול לאוטומורפיזם של חוגים.

\begin{example}
עבור השדה \(F=\mathbb{Q}\left( \sqrt{ 2 } \right)\) נגדיר:
$$\phi:\mathbb{Q} \left( \sqrt{ 2 } \right)\to\mathbb{Q} \left( \sqrt{ 2 } \right) \qquad \phi:a+b\sqrt{ 2 }\mapsto a-b\sqrt{ 2 }$$
כאשר מוגדר היטב כי תחת הרחבה זו ניתן להציג כל מספר ב-\(a+\sqrt{ 2 }b\) כאשר \(a,b \in \mathbb{Q}\). זהו אוטומורפיזם כיוון שמתקיים:
\begin{gather*}\phi\left( \left( a+b{\sqrt{2}} \right)+\left( c+d{\sqrt{2}} \right) \right)=\cdot\cdot\cdot=\phi\left( a+b{\sqrt{2}} \right)+\phi\left( c+d{\sqrt{2}} \right)  \\\phi((a+b{\sqrt{2}})(c+d{\sqrt{2}}))=\cdot\cdot\cdot=\phi(a+b{\sqrt{2}})\,\phi(c+d{\sqrt{2}})
\end{gather*}

\end{example}
\begin{proposition}
אם \(\phi\) אוטומורפיזם של \(F / \mathbb{Q}\) אז מקיים:
$$\forall q \in \mathbb{Q} \qquad \phi(q)=q$$

\end{proposition}
\begin{proof}
נניח \(\phi(1)=q\). נשים לב כי \(q\neq0\) כיוון שאחרת נקבל כי לא חח"ע:
$$\phi(2)=\phi(1+1)=\phi(1)+\phi(1)=0=\phi(1)$$
כעת מתקיים:
\begin{gather*}q=\phi(1)=\phi\left( 1\cdot 1 \right)=\phi(1)\phi(1)=q^2  \\q=\phi(1)=\phi\left( 1\cdot 1 \cdot 1 \right)=q^3 
\end{gather*}
ולכן בהכרח מתקיים \(q^n=q\) לכל \(n\geq 1\), ולכן \(q=1\)

\end{proof}
\section{שדות סגורים אלגברים}

\begin{definition}[שדה סגור אלגברית]
שדה \(F\) נקרא סגור אלגברית אם לכל פולינום ממעלה \(1\leq\) ב-\(F[x]\) יש שורש.

\end{definition}
\begin{proposition}
שלושת התנאים הבאים שקולים עבור שדה \(F\):

  \begin{enumerate}
    \item השדה \(F\) סגור אלגברית 


    \item כל פולינום אי פריק ב-\(F[x]\) הוא ממעלה 1. 


    \item כל פולינום \(f \in F[x]\) ניתן לפרק ל- 
$$f(x)=a\left( x-\lambda_{1} \right)\cdot\;\!\dots\;\!\cdot\left( x-\lambda_{n} \right)$$\textbf{משפט} היסודי של האלגברה
השדה המרוכבים \(\mathbb{C}\) הוא סגור אלגברית.


  \end{enumerate}
\end{proposition}
\begin{definition}[סגור אלגברי]
הרחבת שדות \(E / F\) נקראת סגור אלגברי של \(F\) אם \(E\) סגור אלגברית. מסומן \(\overline{F}\).

\end{definition}
\begin{theorem}[המשפט היסודי של תורת השדות]
יהי \(F\) שדה ו-\(p \in F[x]\) פולינום עם מקדמים ב-\(F\). אזי קיים הרחבת שדה \(E\) של \(F\) כך שלפולינום \(p\) יש שורש. כלומר לכל שדה יש סגור אלגברי \(L\) שמכיל את \(F\).

\end{theorem}
\begin{remark}
הוכחה של משפט זה מניחה את אקסיומת הבחירה, לכן אומנם אנחנו יכולים לדעת שקיים כזה, אין לנו בפועל דרך לדעת מה היא תהיה.

\end{remark}
\chapter{הרחבת גלואה}

\section{שדה פיצול}

\begin{definition}[שדה פיצול]
עבור פולינום \(p \in F[x]\) השדה פיצול של \(p\) יהיה ההרחבת שדות \(E / F\) מהדרגה המינימלית בו \(p\) מתפצל לחלוטין.

\end{definition}
\begin{definition}[הרחבה נורמלית]
הרחבת שדות אשר נוצר משדה פיצול נקרא הרחבה נורמלית.

\end{definition}
\begin{proposition}
ההרחבה שבו כל הפולינומים מעל \(F\) מתפצלים (כלומר ההרחבה שמכילה את כל שדות הפיצול) יהיה הסוגר האלגברי.

\end{proposition}
\begin{proposition}
מימד ההרחבה המירבי של שדה פיצול של פולינום מדרגה \(n\) יהיה \(n!\).

\end{proposition}
\begin{theorem}[יחידות שדה הפיצול]
שדה הפיצול של פולינום הוא יחיד על כדי איזומורפיזם(כאשר מתכוונים לאיזומורפיזמים בין השדות שמקבעות את התת שדה \(K\)).

\end{theorem}
\begin{proof}
יהיו \(E_{1},E_{2}\) שדות פיצול של \(P \in F[x]\) ויהי \(\Omega\) שדה סגור אלגברית שמכיל את \(E_{2}\).

  \begin{enumerate}
    \item מספר השיכונים \(\tau:E_{1}\to \Omega\) כך ש-\(\tau|_{F}=id_{F}\) הוא מוגדר \(\iota(E_{1} / F)\) שהוא יהיה לפחות 1 


    \item נסמן \(\alpha_{1}^{i},\dots,\alpha_{k}^{i} \in E_{i}\) את השורשים של \(P\) ב-\(E_{i}\). כלומר \(E_{i}=F\left( \alpha_{1}^{i},\dots,\alpha_{k}^{i} \right)\). 


    \item כיוון ש-\(\tau(P)=P\) נקבל כי: 
$$\tau\left( \left\{  \alpha_{1}^{1},\dots,\alpha_{k}^{1}  \right\} \right)=P= \left\{  \alpha_{1}^{2},\dots,\alpha_{k}^{2}  \right\}$$


  \end{enumerate}
\end{proof}
\begin{example}
ניקח \(p(x)=x-a_{0}\) ב-\(F\). שדה הפיצול יהיה \(F\)

\end{example}
\begin{example}
ניקח \(p(x)=x^2+a_{1}x+a_{0}=0\). נניח שלא פריק. ניתן לבנות שדה חדש:
$$E =  F / (p)$$
זה מכיל את השורשים של \(p\). וזה יהיה שדה הפיצול.

\end{example}
\begin{example}
נסתכל על \(p(x)=x^3 - 2\) מעל \(\mathbb{Q}\). כיוון שלא פריק תחת \(\mathbb{Q}\) נגדיר:
$$L=\mathbb{Q} \left( \sqrt[3]{ 2 } \right) \cong \mathbb{Q} (x) / (x^3 - 2)$$
נשים לב כי הפולינום \((x^3 - 2)\) מתפצל תחת \(L\):
$$p=x^3-2=\left( x-\sqrt[3]{ 2 } \right)\left( x^2+\sqrt[3]{ 2 } x+\sqrt[3]{ 2 }^2\right)$$
כאשר הגורם השני לא פריק תחת \(L\)! נדרש לבצע הרחבה נוספת:
$$M= L  /\left( y^2+\sqrt[3]{ 2 }y+\sqrt[3]{ 2 }^2 \right)$$
כעת מתקיים \(K\subseteq L \subseteq M\)
כאשר \([M:K]=[M:L][L:K]=6\)

\end{example}
\begin{example}
נסתכל על \(p(x)=x^4+1\) מעל \(\mathbb{Q}\). אם \(\alpha\) הוא שורש, גם \(\alpha^3,\alpha^5, \alpha^7\) שורש. השדה הפיצול הוא:
$$\mathbb{Q} \left( \alpha \right)\cong  \mathbb{Q} [x] / (x^4+1)$$
כאשר מסדר 4.

\end{example}
\begin{example}
יהי \(p(x)=x^4+2x^2-8\) פולינום ב-\(\mathbb{Q}[x]\). אזי ל-\(p(x)\) יש את הגורמים האי פריקים \(x^2-2\) ו-\(x^2+4\). לכן השדה \(\mathbb{Q}\left( \sqrt{ 2 },i \right)\) יהיה השדה פיצול.

\end{example}
\section{הרחבה ספרבילית}

\begin{definition}[פולינום ספרבילי]
יהי \(K\) שדה. פולינום \(f \in K[x]\) נקרא ספרבילי אם אין לו שורשים מרובים בשדה שההרחבה \(L\) שבו הוא מתפצל.

\end{definition}
\begin{definition}[איבר ספרבילי]
תהי \(L / K\) הרחבת שדות. איבר \(\alpha \in L\) יקרא ספרבילי אם הפולינום המינימלי שלו מעל \(K\) הוא ספרבילי.

\end{definition}
\begin{definition}[הרחבה ספרבילית]
הרחבה נקראת ספרבילית אם כל איבר בהרחבה הוא ספרבילי.

\end{definition}
\begin{proposition}[הגדרה שקולה לספרביליות]
הרחבה \(E / F\) ו- \(P \in F[x]\) שמתפצל ב-\(E\). כלומר
$$P=c \prod_{i=1}^k  \left( x - \alpha _{i} \right)^{n_{i}}$$
כך ש-\(\alpha_{i}=\alpha_{j}\) לכל \(i\neq j\). אז נאמר ש-\(P\) ספרבלית אם \(n_{1}=\dots=n_{k}=1\) כל שורש שלו פשוט.

\end{proposition}
\begin{proposition}
אם \(L / M\) ו-\(M / K\) ספרביולות של שדות(לאו דווקא סופיות), אז גם \(L / K\) ספרבילי.

\end{proposition}
\begin{remark}
אם \(P,Q\in F[x]\) נסמן ב-\((P,Q)\in F[x]\) את ה-\(gcd\) שלהם שהוא הפולינום המתוקן המקסימלי שמחלק גם את \(P\) וגם את \(Q\).
$$(P,Q)=gcd(P,Q)$$

\end{remark}
\begin{lemma}
יהיו \(0 \neq P,Q \in F[x]\) ו-\(E / F\) הרחבת שדות. נסמן \((P,Q)_{E}\) את ה-\(gcd\) של \(P\) ו-\(Q\) בשדות \(E\) ו-\(F\) בהתאמה. אזי:  $$(P,Q)_{E}=(P,Q)_{F}$$

\end{lemma}
\begin{definition}[נגזרת]
בהנתן הפולינום:
$$P=\textstyle\sum_{i=0}^{n}a_{i}x^{i}\in F[x]$$
נגדיר את הנגזרת להיות:
$$.P^{\prime}=\textstyle\sum_{i=1}^{d}i a_{i}x^{i-1}\in F[x]$$

\end{definition}
\begin{remark}
אם השדה אינסופי, אז \(\deg P'=\deg P-1\)

\end{remark}
\begin{lemma}
כלל ליבניץ. נגזרות מקיימות:
$$.\left( P\cdot Q \right)^{\prime}=P^{\prime}\cdot Q+P\cdot Q^{\prime}\qquad (P+Q)^{\prime}=P^{\prime}+Q^{\prime}$$

\end{lemma}
\begin{corollary}
$$P=(x-\alpha_{1})^{n_{1}}\cdot\cdot\cdot(x-\alpha_{k})^{n_{k}}\in F[x]$$

\end{corollary}
\begin{proposition}
פולינום \(P \in F[x]\) ספרבילי אם"ם \((P,P^{\prime})=1\)

\end{proposition}
\begin{proposition}
עבור הרחבה סופית \(L / K\) התנאים הבאים שקולים:

  \begin{enumerate}
    \item ההרחבה \(L / K\) ספרבילית. 


    \item יש קבוצת יוצרים של \(L\) מעל \(K\) שכל איבריה ספרבליים. 


    \item כל קבוצת יוצרים של \(L\) מעל \(K\) מורכבת מאיברים ספרביליים. 


  \end{enumerate}
\end{proposition}
\begin{corollary}
שדה פיצול של פולינום ספרבילי היא הרחבה ספרבילית.

\end{corollary}
\begin{proposition}
כל הרחבה אלגברית של שדות ממציין 0 הוא ספרבילי.

\end{proposition}
\section{שיכונים של שדות}

\begin{definition}[שיכון של שדות]
הומומורפיזם שהוא חח"ע.

\end{definition}
\begin{example}
עבור \(F=\mathbb{Q}(i)\) ו-\(\Omega=\mathbb{C}\) נשים לב שיש שתי שיכונים:
\begin{gather*}\iota_{1}:F\to \Omega \qquad \iota_{1}(a+bi)=a+bi  \\\iota_{2}:F\to \Omega \qquad \iota_{2}(a+bi)=a-bi
\end{gather*}

\end{example}
\begin{definition}[פונקציית מספר השיכונים]
אם \(\varphi :K\to \Omega\) כאשר \(\Omega\) סגור אלגברית, ויהי \(L / K\) הרחבת שדות, נסמן את מספר ההרחבות של השיכון \(\varphi\) ע"י:
$$\iota _{\varphi}( L / K) = \#\left\{  \widehat{\varphi}:L\to \Omega \mid \widehat{\varphi}|_{K}=\varphi  \right\}$$

\end{definition}
כלומר זה יהיה מספר השיכונים של השדה החדש \(L\) אשר משמרים את השיכון \(\varphi\) תחת \(K\).

\begin{proposition}
עבור הרחבה פשוטה \(L=K\left( \alpha \right)\) נקבל כי \(\iota_{\varphi}(L / K)\) שווה למספר השורשים השונים של הפולינום המינימלי \(m_{\alpha}\).

\end{proposition}
\begin{proposition}
עבור הרחבה סופית נקבל כי:

  \begin{enumerate}
    \item מספר השיכונים \(\iota(L / K)\) אינו תלוי ב-\(\Omega\) או \(\varphi\). לכן ניתן לסמן \(\iota(L / K)\) וניקרא לו דרגת הספרביליות. 


    \item אם \(K\subseteq M\subseteq L\) מתקיים: 
$$\iota(L / K)=\iota(L / M) \cdot \iota(M / K)$$


    \item מספר השיכונים מקיים: 
$$1\leq \iota(L / K)\leq [L:K]$$
כאשר מתקיים שיוויון \(\iota(L / L)=[L:K]\) אם"ם ספרבילי.


  \end{enumerate}
\end{proposition}
\section{הרחבה נורמלית}

\begin{definition}[הרחבה נורמלית]
הרחבה \(E / F\) תקרא נורמלית אם לכל איבר ב-\(E\) הפולינום המינימלי שלו מעל \(F\) מתפצל ב-\(E\).

\end{definition}
\begin{example}
ההרחבה \(\mathbb{Q}\left( \sqrt[3]{ 2 }  \right) / \mathbb{Q}\)  לא נורמלית. כאשר  \(\alpha = \sqrt[3]{ 2 }\) נקבל \(m_{\alpha}=x^3 - 2\)   ולכן \(\beta = \sqrt[3]{ 2 }e^{2\pi i/3} \not\in \mathbb{Q}\left( \sqrt[3]{ 2 } \right)\).

\end{example}
\begin{proposition}
עבור הרחבה \(E / F\) סופית, מתקיים:
$$\left\lvert  \mathrm{Gal}( E / F)  \right\rvert \leq \iota (E / F)$$
ושיוויון מתקיים אם"ם לכל שדה סגור אלגברית \(\Omega\) שמכיל את \(E\) ולכל שיכון \(\tau:E \to \Omega\) כך ש-\(I|_{F}=Id_{F}\) מתקיים \(\tau(E)=E\).

\end{proposition}
\begin{proposition}
עבור הרחבה סופית התנאים הבאים שקולים

  \begin{enumerate}
    \item ההרחבה היא נורמלית 


    \item השדה \(E\) הוא שדה הפיצול של פולינום כלשהו \(P \in F[x]\). 


    \item מתקיים \(\left\lvert  \mathrm{Gal}(E / F)  \right\rvert=\iota(E / F)\)


  \end{enumerate}
\end{proposition}
\begin{proof}
נניח ראשית כי ההרחבה היא נורמלית. כיוון שההרחבה היא סופית ניתן לכתוב \(E=F\left( \alpha_{1},\dots,\alpha_{k} \right)\) עבור \(\alpha_{i}\) אלגברים מעל \(F\). מההנחה נקבל כי \(m_{\alpha_{i}}\) מתפצל ב-\(E\) לכל \(i\). בפרט נקבל:
$$P=\prod_{i=1}^{k}m_{\alpha_{i}}$$
מתפצל ב-\(E\). כמו כן לכל שדה \(F\subseteq K \subseteq \Omega\) המכיל את שורשי \(P\) מכיל גם את \(\alpha_{1},\dots,\alpha_{k}\) ולכן \(E\subseteq K\). כלומר \(E\) הוא שדה הפיצול של \(P\).

\end{proof}
\begin{proposition}
הרחבה היא נורמלית אם"ם יש שורש של \(p(x)\) גורר שיש את כל השורשים בהרחבה.

\end{proposition}
\begin{proposition}
$$1 \leq Gal(L / F) \leq \iota( L / F) \leq [L : F]$$

\end{proposition}
\begin{example}
1) מעל שדה סגור אלגברי, כל הרחבה היא נורמלית. 

2) ההרחבה \(\mathbb{Q}\left( \sqrt{ 2 } \right) / \mathbb{Q}\) נורמלית 

3) ההרחבה \(\mathbb{Q}\left( \sqrt[3]{ 2 } \right) / \mathbb{Q}\) לא נורמלית 

\end{example}
\section{הרחבת גלואה}

\begin{definition}[הרחבת גלואה]
הרבת שדות \(E / F\) תקרא גלואה אם היא ספרבילית ונורמלית.

\end{definition}
\begin{proposition}
עבור הרחבת שדות סופית התנאים הבאים שקולים:

  \begin{enumerate}
    \item ההרחבה \(E / F\) גלואה 


    \item השדה \(E\) הוא שדה פיצול של פולינום ספרבילי כלשהו. 


    \item האוטומורפיזמים שהמקבעים את השדה מקיימים \(\left\lvert  \mathrm{Gal}(E / F)  \right\rvert=[E:F]\)


  \end{enumerate}
\end{proposition}
\begin{proposition}
אם \(F\) שדה ממציין שונה מ-2 ו-\(E / F\) הרחבה מדרגה \([E:F]=2\) אז \(E / F\) הרחבת גלואה.

\end{proposition}
\begin{proposition}
אם \(K\subseteq M\subseteq L\) מגדל של הרחבות אלגבריות של שדות, אז אם \(L / K\) גלואה, גם \(L / M\) גלואה.

\end{proposition}
\begin{proposition}
אם הרחבה היא ספרבילית, ניתן להרחב את הרחבה כך שהיא תהיה גלואה, כלומר, אם \(E / F\) הרחבה ספרבילית סופית אז קיימת הרחבה סופית \(L / E\) כך ש-\(L / F\) הרחבת גלואה. 

\end{proposition}
\section{דיסקרימיננטה}

\begin{definition}[דיסקרימיננטה]
דיסקרימיננטה של פולינום מהצורה \(f(x)\,=\,\prod_{i=1}^{n}(x-\alpha_{i})\) יהיה מהצורה:
$$.\Delta(f)\,=\,D_{n}(f)\,:=\,\prod_{i<j}(\alpha_{i}-\alpha_{j})^{2}$$

\end{definition}
\begin{proposition}
עבור \(f \in F[x]\) מדרגה \(n\), ו-\(E\) שדה הפיצול של \(f\) מעל \(F\). אזי \(\sqrt{ \Delta(f) } \in F\) אם"ם \(\mathrm{Gal}(E / F)\leq A_{n}\).

\end{proposition}
\begin{proposition}
עבור \(f \in \mathbb{R}[x]\), פולינום ספרבילי, אז \(\Delta f>0\) אם"ם מספר השורשים הלא ממשיים מתחלק ב-4.

\end{proposition}
\begin{proposition}
הדיסקרימיטטנה של פולינום ממעלה שלישית מהצורה הבאה תהיה \(\Delta\left(x^{3}+a x+b\right)=-4a^{3}-27b^{2}\).

\end{proposition}
\chapter{המשפט היסודי ושימושיו}

\section{המשפט היסודי של תורת גלואה}

\begin{definition}[שדה שבת]
יהי \(G\leq \mathrm{Gal}(E / F)\) תת חבורה של אוטורמורפיזם של הרחבת שדות. שדה השבת יהיה השדה של כל האיברים אשר כל האוטומורפיזמים ב-\(G\) לא משנה אותם. כלומר כל אוסף האיברים \(a \in E\) המקיימים \(\sigma(a)=a\).

\end{definition}
\begin{example}
עבור המרוכבים \(\mathbb{C}\), האוטומורפיזם הלא טריוויאלי היחיד יהיה אוטומורפיזם ההצמדה, ושדה השבת יהיה \(\mathbb{R}\).

\end{example}
\begin{definition}
נגדיר את ההעתקה \(\mathcal{F}\) שמקבלת תת חבורה של האוטומורפיזמים ומחזירה את שדה השבת שלהם.

\end{definition}
\begin{proposition}
אם \(L\) שדה ו-\(G\leq \mathrm{Aut}(L)\) עבור שדה השבת \(K=\mathcal{F}(G)\) מתקיים \([L:K]=\lvert G \rvert\).

\end{proposition}
\begin{definition}
נסמן ב-\(\mathcal{G}\) את ההעתקה שמקבלת שדה ביניים \(K\subseteq M\subseteq L\) של \(\mathrm{Aut}(L / K)\) ומחזירה את שדה השבת המתאים.

\end{definition}
\begin{corollary}
הרחבה היא גלואה אם"ם \(\mathcal{F}\left( \mathcal{G}(F) \right)=F\).

\end{corollary}
\begin{theorem}[היסודי של תורת גלואה]
עבור הרחבת גלואה סופית \(L / K\) ו-\(G=\mathrm{Gal}(L / K)\) ההתאמות \(\mathcal{G,F}\) מהוות התאמות חח"ע, על והפיכות זו לזה בין תתי החבורות של \(G\) ושדות הביניים של \(L / K\). כלומר:

  \begin{enumerate}
    \item לכל שדה ביניים \(K\subseteq M\subseteq L\) מתקיים \(\mathcal{F}\left( \mathcal{G}(M) \right)=M\). 


    \item לכל תת חבורה \(H\leq G\) מתקיים \(\mathcal{G}\left( \mathcal{F}(H) \right)=H\). 


  \end{enumerate}
\end{theorem}
\begin{proposition}
גם כאשר הרחבה היא לא גלואה ההעתקות הופכות סדר. כלומר אם \(M_{1}\subseteq M_{2}\) אז \(\mathcal{G}(M_{1})\supseteq \mathcal{G}(M_{2})\) ובאופן דומה אם \(H_{1}\leq H_{2}\) אז \(\mathcal{F}(H_{1})\supseteq \mathcal{F}(H_{2})\).

\end{proposition}
\begin{proposition}
הרחבה סופית \(E / F\) היא פשוטה אם"ם יש לה מספר סופי של שדות ביניים.

\end{proposition}
\begin{theorem}[האיבר הפרימיטיבי]
כל הרחבה ספרבילית סופית היא פשוטה.

\end{theorem}
\begin{proof}
ראשית נשים לב ש-\(L / F\) הרחבת גלואה ולכן \(Gal(L / F)\) בעלת מספר סופי של תת חבורות ולכן מספר סופי של שדות בינים \(L / K / F\) ולכן מספר סופי של שדות ביניים \(E / F\) וסיימנו.

\end{proof}
\begin{remark}
אומנם משפט האיבר הפרימיטיבי מאוד חזק, קשה מאוד למצוא את האיבר הפרימיטיבי באופן כללי, ולכן השימוש העיקרי זה כדי לפשט הוכחות - מספיק להוכיח עבור הרחבה ספרבילית פשוטה - ולמעשה הוכחנו על כל הרחבה ספרבילית סופית!

\end{remark}
\section{שדות ממציין ראשוני}

\begin{reminder}
המציין של כל שדה יהיה חזקה של מספר ראשוני.

\end{reminder}
\begin{proposition}
קיים ויחיד עד כדי איזומורפיזם שדה מגודל \(p^{n}\), ונסמנו \(\mathbb{F}_{p^{n}}\).

\end{proposition}
\begin{proposition}
שדה סופי ממציין \(p\) מקיים \((x+y)^{p}=x^{p}+y^{p}\).

\end{proposition}
\begin{proposition}[אוטומורפיזם פרוביניוס]
עבור שדה סופי \(E\) ממציין \(p\) ההעתקה \(\phi(x)=x^{p}\) היא אוטומורפיזם.

\end{proposition}
\begin{proof}
נשים לב כי:
$$\phi(0)=0^{0}\qquad \phi(1)=1^{p}=1$$
ולכן \(p\) שולחת את איבר היחידה לאיבר היחידה, ואת איבר ה-0 לאיבר ה-0. יהי \(x,y \in E\) מתקיים:
\begin{gather*}\phi\left( x\cdot y \right)=\left( x\cdot y \right)^{p}=x^{p}\cdot y^{p}=\phi(x)\cdot \phi(y)  \\\phi(x+y)=(x+y)^{p}=\sum_{i=1}^{p} {p \choose i}x^{i}y^{p-i}\overset{*}{=} x^{p}+y^{p} 
\end{gather*}
כאשר המעבר \(*\) זה כיוון שכל המקדמים יהיו כפולה של \(p\) פרט לראשון ולאחרון. ולכן הומומורפיזם. כיוון ששדה סופי מספיק להראות כי חח"ע. ובשביל זה מספיק להראות כי הגרעיון טריוויאלי. יהי \(a \in \ker\left( \phi \right)\). ולכן:
$$\phi(a)=0\implies a^{p}=0\implies a=0$$
כאשר המעבר האחרון נובע מזה ששדה הוא תחום שלמות.

\end{proof}
\begin{proposition}
ההרחבה \(E / F_{p}\) כאשר \(E\) שדה סופי ממציין \(p\) היא הרחבת גלואה המקיימת \(\mathrm{Gal}(E / F_{p})=\left\langle  \phi  \right\rangle\).

\end{proposition}
\begin{proof}
ראשית נזכור כי מתקיים \(\lvert E \rvert=p^n\) עבור \(n\) כלשהו. נראה כי ההרחבת שדות היא שדה פיצול של הפולינום \(p(x)=x^{p^{n}}-x\). נשים לב כי מתקיים:
$$p(0)=0^{p^n}=0$$
כעת נראה כי כל איבר בשדה הכפלי \(E^{\times}\) הוא שורש של הפולינום. זה יראה לנו כי בסה"כ כל האיברים ב-\(E\) הם שורש של הפולינום. מתקיים:
$$\left\lvert  E^\times  \right\rvert =p^{n}-1$$
ולכן ממשפט לגרנג' מתקיים עבור \(a \in E^\times\) כי \(a^{p^{n}-1}=1\) ולכן \(a^{p^{n}}=a\)$$p(a)=a^{p^{n}}-a=a-a=0$$
וכיוון שזה מתקיים לכל \(a\in E^{\times}\) ועבור 0 נקבל כי יש \(p^{n}\) שורשים ולכן כל איבר ב-\(E\) הוא שורש, ולכן פולינום ספרבילי. וכיוון ששדה פיצול של פולינום ספרבילי נקבל כי זוהי הרחבת גלואה. 
נראה כעת כי \(\mathrm{Gal}(E / F_{p})=\left\langle  \phi  \right\rangle\). כעת נשים לב כי \(\phi|_{F_{p}}=id\). זה נכון לכל אוטומורפיזם על השדה הראשי שלו. נסמן \(\left\lvert  \left\langle  \phi  \right\rangle  \right\rvert=r\). לכן מתקיים:
$$\phi^r = id \implies \forall a \in E\quad \phi^{r}(a)=a^{p^{r}}=a\implies \forall a \in E\quad a^{p^{r}}-a=0$$
כאשר לפולינום \(a^{p^{r}}-a\) יש לכל היותר \(p^{r}\) שורשים, ולכן \(p^{n}\leq p^{r}\) ולכן \(n\leq r\). בנוסף, הסדר של \(\left\langle  \phi  \right\rangle\) חייב לחלק את הסדר של \(\lvert Gal(E / F_{p}) \rvert\). ולכן כיוון ש:
$$\lvert E / F_{p} \rvert=\lvert Gal(E / F_{p}) \rvert=n$$
נקבל כי \(r\leq n\) ולכן נקבל סה"כ כי \(r=n\) ולכן הסדר של \(\left\lvert  \left\langle  \phi  \right\rangle  \right\rvert=n=\lvert G \rvert\) ולכן \(G\) חבורה ציקלית עם היוצר \(\left\langle  \phi  \right\rangle\).
\textbf{טענה}
האיחוד \(\bigcup_{n}\mathbb{F}_{p^{n}}\) היא הרחבת שדות סגורה אלגברית של \(\mathbb{F} _p\).

\end{proof}
\section{פולינומים ציקלוטומים}

\begin{definition}[שורש יחידה]
יהי \(F\) שדה, \(n \in \mathbb{N}\). איבר \(y \in \mathbb{F}\) יקרא שורש יחידה מסדר \(n\) אם \(y^n=1\).
בנוסף הוא יקרא פרמיטיבי אם:
$$\forall m < n\quad y^m\neq 1$$

\end{definition}
\begin{example}
$$\omega_{n}=e^{ 2\pi i/n }\in \mathbb{C}$$

\end{example}
\begin{lemma}
אם \(y\) שורש יחידה פרימטיבי מסדר \(n\) אז אוסף כל שורשי היחידה הסדר \(n\) יהיה:
$$\left\{  y^{k}\mid {1}\leq k\leq n  \right\}$$
כאשר אוסף כל שורשי היחידה הפרימטריבים מסדר \(n\) יהיו:
$$\left\{  y^k \mid 1\leq k\leq n\quad (k,n)=1 \right\}$$

\end{lemma}
\begin{proposition}
יהי \(\xi \in F\) שורש יחידה פרימיטיבי. אזי:

  \begin{enumerate}
    \item הקבוצה \(\left\{\xi^{k}\mid1\leq k\leq n\right\}\) זה קבוצת שורשי היחידה מסדר \(n\). 


    \item הקבוצה \(\left\{\xi^{k}\,|\,1\leq k\leq n,\,(k,n)=1\right\}\) זה הקבוצת שורשי היחידה הפרימיטיביים מסדר \(n\). 


  \end{enumerate}
\end{proposition}
\begin{definition}[פונקציית אויילר]
פונקציה \(\varphi:\mathbb{N}\to\mathbb{N}\) המוגדרת ע"י:
$$\varphi\left(n\right)=\left|\left\{1\leq k\leq n\left|\left(k,n\right)=1\right\}\right|=\deg\left(\Phi_{n}\right)\right.$$

\end{definition}
\begin{definition}[פולינום ציקלוטומי]
הפולינום הציקלוטומי מסדר \(n\) מעל \(\mathbb{Q}\) מוגדר להיות:
$$\Phi_{n}(x)=\prod_{\begin{array}{l}{0\leq k\leq n-1}\\ {\operatorname*{gcd}(k,n)=1}\end{array}}\left(x-\omega_{n}^{k}\right)$$

\end{definition}
הפולינומים הציקלוטומים מקיימים:
$$x^{n}-1=\prod_{d|n}\Phi_{d}\left(x\right)$$

\begin{proposition}
הפולינום הציקלוטומי הוא פולינום תחת השלמים, כלומר \(\Phi_{n}\in \mathbb{Z}[x]\).

\end{proposition}
\begin{proposition}
אם \(y \in F\) שורש יחידה פרמיטיבי מסדר \(\sigma \in Aut(F)\) אז \(\sigma(F)\) הוא גם שורש יחידה פרימיטיבי מסדר \(n\).

\end{proposition}
\begin{proof}
$$\sigma (y)^n=\sigma(y^n)=\sigma(1)=1\implies \sigma(y)$$
שורש מסדר \(n\). נדרש להראות פרימטיביות. אם \(m<n\) מתקיים:
$$\sigma(y)^m = 1\implies \sigma(y^m)=\sigma(1)\implies y^m = 1 $$
וזוהי סתירה לפרמטיביות. ולכן \(\sigma(y)\) גם פרמטיבי.

\end{proof}
\begin{proposition}
לכל \(n\)\(\Phi(x) \in \mathbb{Q} [x]\) מתוקן.

\end{proposition}
\begin{proposition}
לכל \(n \in \mathbb{N}\) הפולינום \(\Phi_{n}\) אי פריק.

\end{proposition}
\begin{proof}
נסמן \(m(x)\in \mathbb{Q}[x]\) הפולינום המינימלי של \(\omega_{n}\). מספיק להוכיח \(m(x)=\Phi_{n}(x)\). מספיק להוכיח \(\deg(m(x))=\varphi(n)\).
נשים לב של-\(\Phi_{n}(x)\)  אם \(\Phi_{n}\left( \alpha \right)=0\) אז לכל ראשוני \(n\geq p\) שזר ל-\(n\) מתקיים \(\Phi\left( \alpha^p \right)=0\).
כעת נותר להוכיח כי \(m\left( \alpha^p \right)=0\). נניח בשלילה שלא. נשים לב:
$$\Phi_{n}=m(x)g(x)$$
כאשר \(g(x) \in R[x]\). כעת מהטענה מתקיים \(\alpha^p\) שורש של \(g(x)\). במילים אחרות \(\alpha\) הוא שורש של \(g(x^p)\). ולכן:
$$g(x^p)=m(x)h(x)$$
ל-\(n(x) \in R[x]\) כלשהו. ולכן כל הפולינומים ב-\(\mathbb{Z}[x]\). 

\end{proof}
\begin{proposition}
לכל \(n\) נקרא ל-\(\mathbb{Q}\left( \omega_{n} \right)\) הרחבה ציקלוטונית ה-\(n\). אזי:

  \begin{enumerate}
    \item ההרחבה \(\mathbb{Q}\left( \omega_{n} \right) / \mathbb{Q}\) הרחבת גלואה מסדר \(\varphi(n)\). 


    \item \(Gal\left( \mathbb{Q} \left( \omega_{n} \right) / \mathbb{Q}  \right) \cong  \left( \mathbb{Z} / n \mathbb{Z} \right)^\times\)


  \end{enumerate}
\end{proposition}
\begin{corollary}[פונקציית אוילר]
  \begin{enumerate}
    \item אם \(n=mk\) ו-\((m,k)=1\) אז: 
$$\varphi(n)=\varphi(1)\varphi(n)$$


    \item אם \(n=p^k\) אז: 
$$\varphi(n)=(p-1)p^{k-1}$$


    \item אם \(n=\prod_{i}p_{i}^{k_{i}}\) אזי: 
$$\varphi (n)= \prod_{i}(p_{i}-1)p_{1}^{k_{1}-1}$$


  \end{enumerate}
\end{corollary}
\begin{theorem}[השארית הסיני]
החבורת גלואה המתאימה להרחבה \(\mathbb{Q}\left( \omega_{n} \right) / \mathbb{Q}\) איזומורפית ל-\(\left( \mathbb{Z} / n\mathbb{Z} \right)^\times\).

\end{theorem}
\section{בניות סרגל ומחוגה}

\section{הרחבות רדיקאליות}

\begin{definition}[הרחבה רדיקלית פשוטה]
הרחבה שדות \(E / F\) תקרא רדיקלית פשוטה אם \(E=F\left( \alpha \right),\alpha \in E\).

\end{definition}
\begin{definition}[הרחבה ציקלית/אבלית/פתירה]
הרחבת גלואה נקראת ציקלית/אבלית/פתירה אם החבורת גלואה שלה ציקלית/אבלית/פתירה.

\end{definition}
\begin{reminder}[חבורה פתירה]
חבורה \(G\) תקרא פתירה אם קיימת סדרת הרכב עם מנות אבליות/ציקליות/פתירות(ניתן להראות כי הם שקולים, ניתן לפרק סדרת הרכב של מנות אבליות לסדרת הרכב של מנות ציקליות ממשפט המיון של חבורות אבליות נוצרות סופית)

\end{reminder}
\begin{reminder}
חבורת התמורות \(S_{n}\) היא פתירה אם"ם \(n\leq 4\).

\end{reminder}
\begin{reminder}
אם \(N\trianglelefteq G\) כך ש-\(N\) ו-\(G / N\) פתירות אז \(G\) פתירה.

\end{reminder}
\begin{proposition}
אם \(G,H\) חבורות פתירות אז גם \(G\rtimes H\) פתירה.

\end{proposition}
\begin{proof}
נרצה להשתמש בעובדה שאם \(N \trianglelefteq K\) נ נורמלית ופתירה וגם \(K / N\) פתירה אז \(K\) פתירה.
נשים לב שתת החבורה:
$$\{(g,1)\mid g\in G\}\leq G\rtimes H$$
זו תת חבורה נורמלית ומתקיים:
$$H\cong{\frac{G\rtimes H}{G}}$$
ולכן פתיר.

\end{proof}
\begin{remark}
אם \(G\) פתירה אז כל תת חבורה שלה פתירה.

\end{remark}
\begin{theorem}
אם הפולינום \(x^{n}-b\) אי פריק מעל \(\mathbb{R}\), ו-\(E\) שדה הפיצול אז:
$$Gal(E / Q)\cong \mathbb{Z}_{n}\rtimes\mathbb{Z}_{n}^{\times}$$

\end{theorem}
\section{המשפט היסודי של האלגברה}

\begin{proposition}
השדה \(\mathbb{C}\) סגור אלגברית. כלומר כל פולינום עם מקדמים מרוכבים מתפצל.

\end{proposition}
\begin{proof}
נראה כי \(p \in \mathbb{C}\) מתפצל מעל \(\mathbb{C}\). מספיק להראות עבור פולינום ממשי כי עבור פולינום מרוכב נקבל כי \(p \in \mathbb{C}[x]\) מתפצל אם"ם \(p\cdot \overline{p}\) מתפצל כאשר מתקיים \(p\cdot \overline{p}=\overline{p\cdot \overline{p}}\) ולכן ממשי.
מאינפי 1 לכל פולינום מדרגה אי זוגית יש שורש ממשי, ולכן אין הרחבה אי זוגית מעל \(\mathbb{R}\). נניח בשלילה שיש הרחבה \(E\) אי זוגית מעל \(\mathbb{R}\). ממשפט האיבר הפרימיטיבי נקבל כי הרחבת שדות תהיה מהצורה \(E=\mathbb{R}\left( \alpha \right)\). כאשר מתקיים
$$\left[ \mathbb{R}\left( \alpha \right) :\mathbb{R}\right]=\deg\left( m_{\alpha} \right) $$
כעת נראה כי כל הרחבה גלואה מקייימת \(\left[ E:\mathbb{R} \right]=2^{n}\). נסמן \(\left[ E:\mathbb{R} \right]=2^{n}q\). כאשר \(2\not\mid q\). ניקח חבורת 2 סילו של \(G\left( E / \mathbb{R} \right)\). נקבל:
$$\left[ E:\mathcal{F} (H) \right]=\lvert H \rvert =2^{n}$$
כאשר כעת מתקיים
$$\underbrace{ \left[ E:\mathbb{R} \right] }_{ 2^{n}q }=\underbrace{ \left[ E:\mathcal{F} (H) \right] }_{ 2^{n} }\underbrace{ \left[ \mathcal{F} (H):\mathbb{R} \right] }_{ q }$$
כאשר אנו יודעים כי \(q\) לא זוגי(כיוון שכל הריכיבים בזוגיים נכנסו לחבורת \(p\) סילו) ולכן \(q=1\).

\end{proof}
אנו יודעים כי לכל פולינום מרוכב מדרגה 2 יש שורש. מנוסחאת שורשים עבור פולינום \(p=x^{2}+bx+c\) נקבל:
$$x_{1,2}=\frac{-b\pm \sqrt{ b^{2}-4c }}{2}$$
כאשר חיסור, חיבור, כפל וחילוק של מספרים מרוכבים יחזיר מספר מרוכב. ולכן מספיק להראות כי שורש של מספר מרוכב הוא מספר מרוכב. אכן:
$$\sqrt{ re^{ i\theta } }=\sqrt{ r }e^{ i\theta/2 }$$
ואכן מרוכב. לכן אין הרחבה מדרגה 2 מעל \(\mathbb{C}\)(באותו אופן שהראנו ל-\(\mathbb{R}\)). כעת נראה כי אין הרחבת גלואה המקיימת \(\left[ E:\mathbb{C} \right]=2^{n}\). נניח בשלילה שיש, לכן נילפוטנטי ולכן קיים תת חבורה מגודל \(H=2^{n-1}\). כעת מתקיים
$$\underbrace{ \left[ E:\mathbb{C} \right] }_{ 2^{n} }=\underbrace{ \left[ E:\mathcal{F} (H) \right] }_{ 2^{n} −1}\left[ \mathcal{F} (H):\mathbb{C} \right]\implies \left[ \mathcal{F} (H):\mathbb{C} \right] =2$$
ולכן הפולנינום  המינימלי מדרגה 2 בסתירה שהפולינום המינימלי אי פריק ולכן \(\left[ E:\mathbb{C} \right]\neq 2^{n}\)
ניקח פולינום \(p\). נסתכל על שדה הפיצול של \((x^{2}+1)p\) ונסמן אותו ב-\(E\). מתקיים:
$$\underbrace{ \left[ E:\mathbb{R} \right] }_{ 2^{n} }=\left[ E:\mathbb{C} \right]\underbrace{ \left[ \mathbb{C}:\mathbb{R} \right] }_{ 2 }\implies E:\mathbb{C}=1\implies E=C$$

\chapter{שאלות}

\section{שדות}

?
חוק חילוק(קיים הופכי כפלי לכל איבר פרט ל-0) קומוטטיבי.

מהו הרחבת שדות?
?
השדה \(E\) הוא הרחבת שדות של השדה \(F\) אם \(F\subseteq E\) והפעולה של \(E\) היא הפעולה של \(F\) כאשר מצטמצים ל-\(F\). מסומן \(E / F\).

מהו איבר אלגברי מעל ההרחבת שדות \(E / F\)?
?
איבר \(a \in K\) נקרא אלגברי מעל \(F\) אם קיים פולינום \(p \in F[x]\) אשר מקיים \(p\left( \alpha \right)=0\).

מהי הרחבה אלגברית?
?
הרחבת שדות \(E / F\) נקראת הרחבה אלגברית אם לכל \(\alpha \in E\) קיים \(f \in F[x]\) כך ש-\(f\left( \alpha \right)=0\).

מהו פולינום מינימלי?
?
יהי \(E / F\) הרחבת שדות. יהי \(\alpha \in E\setminus F\) אלגברי. הפולינום המינימלי יהיה הפולינום \(f \in F[x]\) אשר יקיים \(f\left( \alpha \right)=0\) ולא קיים פולינום ממעלה קטנה יותר המקיים את זה.

פולינום לא פריק אשר מאפס את \(\alpha\) נקרא \textbf{פולינום מינימלי}

מהו שדה פיצול?
?
זה תכונה של פולינום. זהו ההרחת שדות בו הפולינום עם מקדמים מ-\(F\) מתפצלים

איך נקרא הרחבת שדות אשר נוצרה משדה פיצול של פולינום?
?
הרחבת שדה נורמלית

מהו הסגור האלגברי? מה הקשר שלו לשדות פיצול?
?
זהו תכונה של שדה. זהי ההרחבה שדות שבה כל פולינום עם מקדמים מהשדה מתפרקים לגורמים לינארים. למעשה זה יהיה השדה שמכיל את כל השדות פיצול של הפולינום מהשדה.

מהי הטענה המרכזית של סגורים אלגברים?
?
לכל שדה קיים סגור אלגברי יחיד עד כדי איזומורפיזם.

מהו פולינום ספרבילי?
פולינום \(f \in F[x]\) יקרא ספרבילי אם אין שורשים מרובים בשדה שבו מתפצל.

מהו איבר ספרבילי? מהי הרחבת שדות ספרבילית?
?
איבר \(\alpha \in E / F\) יקרא ספרבילי אם הפולינום המינימלי \(m_{\alpha}\) יהיה ספרבילי.
הרחבה \(E / f\) יקרא הרחבה ספרבילית אם כל איבר בהרחבה הוא ספרבילי.

מהי הטענה על ספרביליות ונגזרות?
?
פולינום הוא סספרבילי אם"ם \(gcd(f,f')=1\).

מהו הסגור הספרבילי?
?
הרחבה \(F\) בפני עצמה היא ספרבילית. זה כיוון שכל איבר ב-\(F\) הוא פולינום עם מקדימים ב-\(F\), ולכן בפרט כל פולינום הוא ספרבילי.
לא כל הרחבה אבל \(E / F\) היא ספרבילית. קיים אבל הרחבה \(F^{sep}_{E} / F\) ספרבילית. הרחבה זו נקראת הסגור הספרבילי.

מהי הרחבה נורמלית?
?
הרחבה שדות שנוצרת ע"י שדה פיצול של פולינום מסויים. הגדרה אחרת יכולה להיות של פולינום אם
\end{document}