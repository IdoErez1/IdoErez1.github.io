\documentclass{tstextbook}

\usepackage{amsmath}
\usepackage{amssymb}
\usepackage{graphicx}
\usepackage{hyperref}
\usepackage{xcolor}

\begin{document}

\title{Example Document}
\author{HTML2LaTeX Converter}
\maketitle

\chapter{סדרות של מספרים ממשיים}

\section{הגדרת הגבול}

\begin{definition}[סדרה של מספרים ממשיים]
פונקציה \(f:\mathbb{N}\to \mathbb{R}\) נקראת סדרה ממשית. בדרך כלל נסמן את האיבר ה-\(n\) של הסדרה ב-\(f_{n}\), ואת הסדרה כולה נסמן ב-\((f_{n})_{n=1}^{\infty}\).

\end{definition}
\begin{example}
  \begin{enumerate}
    \item עבור \(\lambda \in \mathbb{R}\) הסדרה \(f_{n}=\lambda\) נקראת הסדרה הקבועה 


    \item הסדרה \(h_{n}=\frac{1}{n}=\left( 1,\frac{1}{2},\frac{1}{3},\dots \right)\) נקראת הסדרה ההרמונית 


    \item הסדרה המתחלפת \(q_{n}=(-1)^{n-1}\).  


    \item הסדרה של המספרים הראשוניים \((p)_{n=1}^{\infty}=\left( 2,3,5,7,\dots \right)\). 


  \end{enumerate}
\end{example}
\begin{definition}[גבול של סדרה ממשית]
תהי \((f_{n})_{n=1}^{\infty}\) סדרה ממשית ו-\(L \in \mathbb{R}\). אזי \(L\) יקרא הגבול של הסדרה אם:
$$\forall\varepsilon>0\quad \exists N \in \mathbb{N}\quad \forall n\in \mathbb{N}\quad n>N\implies |f_{n}-L|<\varepsilon$$
כלומר עבור כל מספר חיובי \(\varepsilon>0\) ניתן למצוא איזשהו \(N \in \mathbb{N}\) כך שהחל ממנו, כל האיברים \(n>N\) יקיימו \(|f_{n}-L|<\varepsilon\), כלומר \(L-\varepsilon<f_{n}<L+\varepsilon\).
אם \(L\) הוא גבול, נאמר כי \((a_n)_{n=1}^\infty\) מתכנס ל-\(L\), ונסמן \(f_{n}\xrightarrow{n\to \infty}L\), או לחלופין \(\lim_{ n \to \infty }f_{n}=L\).

\end{definition}
\begin{remark}
סדר הכמתים חשוב פה. אנחנו לוקחים \(\varepsilon\) כלשהו, אז עבורו מקבלים \(N\in \mathbb{N}\) אשר יכול להיות תלוי ב-\(\varepsilon\) כיוון שנבחר אחריו. כלומר \(N=N\left( \varepsilon \right)\).

\end{remark}
\begin{proposition}
הביטויים הבאים שקולים:

  \begin{enumerate}
    \item הסדרה \((a_n)_{n=1}^\infty\) מתכנסת ל-\(L\). 


    \item הסדרה \((a_n-L)_{n=1}^\infty\) מתכנסת ל-0 - כלומר הסדרה שמחסירים \(L\) מכל איבר של \(a_{n}\). 


    \item הסדרה \((|a_{n}-L|)_{n=1}^\infty\) מתכנסת ל-0 - כלומר סדרה הערכים המוחלטים. 


  \end{enumerate}
\end{proposition}
\begin{definition}[סדרה מתכנסת]
סדרה \((a_n)_{n=1}^\infty\) עבורה קיים \(L \in \mathbb{R}\) כלשהו אשר מתקיים \(a_{n}\xrightarrow{n\to \infty}L\) נקראת סדרה מתכנסת. בעזרת כמתים:
$$\exists L \in \mathbb{R}\quad \forall\varepsilon>0\quad \exists N \in \mathbb{N}\quad \forall n\in \mathbb{N}\quad n>N\implies |f_{n}-L|<\varepsilon$$

\end{definition}
\begin{proposition}[יחידות הגבול]
תהי \((a_n)_{n=1}^\infty\) סדרה ממשית. אם \((a_n)_{n=1}^\infty\) מתכנס לגבול \(L\) אז גבול זה הוא יחיד. כלומר אם \(a_{n}\) מתכנס ל-\(L_{1}\) וגם מתכנס ל-\(L_{2}\) אז \(L_{1}=L_{2}\).

\end{proposition}
\begin{proof}
  \begin{enumerate}
    \item נניח בשלילה כי קיימים \(L_{1},L_{2} \in \mathbb{R}\) עם \(L_{2}>L_{1}\) כך ש-\((a_n)_{n=1}^\infty\) מתכנס גם ל-\(L_{1}\) וגם ל-\(L_{2}\). 


    \item נגדיר \(\varepsilon= \frac{L_{2}-L_{1}}{3}\). מהגדרת הגבול עבור \(L_{1}\) נקבל \(N_{1} \in \mathbb{N}\) כך שלכל \(n>N_{1}\) מתקיים \(\lvert a_{n}-L_{1} \rvert<\varepsilon\). כאשר מהגדרת הגבול עבור \(L_{2}\) נקבל \(N_{2}\) כך שלכל \(n>N_{2}\) מתקיים \(|a_{n}-L_{2}|<\varepsilon\). 


    \item נסתכל על \(n_{0}>N_{1}\) וגם \(n_{0}>N_{2}\) (קיים כזה, למשל \(n_{0}>\max(N_{1},N_{2})\) או \(n_{0}>N_{1}+N_{2}\)) נקבל כי \(|a_{n_{0}}-L_{1}|<\varepsilon\) וגם \(|a_{n_{0}}-L_{2}|<\varepsilon\). כעת: 
$$3\varepsilon=\lvert L_{2}-L_{1} \rvert =|L_{2}-a_{n_{0}}+a_{n_{0}}-L_{1}|\leq |-(a_{n_{0}}-L_{2})|+|(a_{n_{0}}-L_{1})|\leq 2\varepsilon$$
וקיבלנו סתירה.


  \end{enumerate}
\end{proof}
\begin{definition}[סדרה מתבדרת]
סדרה שאינה מתכנסת נקראת סדרה מתבדרת. כלומר:
$$\forall L \in \mathbb{R}\quad \exists\varepsilon>0\quad \forall N \in \mathbb{N}\quad \exists n\in \mathbb{N}\quad n>N \land|f_{n}-L|\geq\varepsilon$$

\end{definition}
\begin{example}
נוכיח את התבדרות הסדרה:
$$(a_n)_{n=1}^\infty =\left( \frac{1+(-1)^{n+1}}{2} \right)_{n=1}^\infty =\left( 1,0,1,0,\dots \right)$$

  \begin{enumerate}
    \item יהי \(L \in \mathbb{R}\). נגדיר \(\varepsilon=\frac{1}{4}\), וכן יהי \(N \in \mathbb{N}\). ניקח \(n_{0} \in \mathbb{N}\) כך ש-\(n_{0}>N\). כעת \(\{ a_{n_{0}},a_{n_{0}+1} \}=\{ 0,1 \}\) ולכן \(|a_{n_{0}}-a_{n_{0}+1}|=1\). 


    \item נניח בשלילה שמתקיים \(|a_{n_{0}}-L|<\varepsilon \land|a_{n_{0}+1}-L|<\varepsilon\). כעת: 
$$1=|a_{n_{0}}-a_{n_{0}+1}|=|a_{n_{0}}-L+L-a_{n_{0}}|=|a_{n_{0}}-L|+|a_{n_{0}+1}-L|<\frac{1}{4}+\frac{1}{4}=\frac{1}{2}$$
בסתירה. לכן או \(|a_{n_{0}}-L|\geq\varepsilon\) ובמקרה זה ניקח \(n=n_{0}\) וסיימנו או \(|a_{n_{0}+1}-L|\geq \varepsilon\) וניקח \(n=n_{0}+1\) וסיימנו.


  \end{enumerate}
\end{example}
\begin{definition}[סדרה חסומה]
  \begin{enumerate}
    \item קבוצה נקראת חסומה מלעיל אם קבוצת האיברים שלה חסומה מלעיל. כלומר קיים \(M\in \mathbb{R}\) כך שלכל \(n \in \mathbb{N}\) מתקיים \(a_{n}<M\). 


    \item קבוצה נקראת חסומה מלרע אם קבוצה האיברים שלה חסומה מלרע. כלומר קיים \(m \in \mathbb{R}\) כך שלכל \(n \in \mathbb{N}\) מתקיים \(a_{n}>m\). 


    \item קבוצה נקראת חסומה אם חסומה גם מלעיל וגם מלרע. כלומר קיימים \(m,M\in \mathbb{R}\) כך שלכל \(n \in \mathbb{N}\) מתקיים \(m\leq a_{n}\leq M\). 


  \end{enumerate}
\end{definition}
\begin{remark}
אם סדרה חסומה מלעיל עבור \(M\) אז בפרק לכל \(\tilde{M}\geq M\). וכן אם חסומה מלרע עבור \(m\) אז בפרט חסומה מלרע לכל \(\tilde{m}\leq m\). וכן מתכונות הערך המחולט:
$$M\leq|\max (|m|,|M|)|\qquad \text{and}\qquad -|\max(|m|,|M|)|\leq m $$
ולכן להיות חסום שקול למעשה לזה שקיים \(C\) כך שלכל \(n \in \mathbb{N}\) מתקיים \(|a_{n}|<C\).

\end{remark}
\begin{example}
  \begin{enumerate}
    \item הסדרה \((a_n)_{n=1}^\infty=\left( 1,0,1,\dots \right)_{n=1}^\infty\) עבור \(m=0,M=1\). 


    \item הסדרה \((n)_{n=1}^\infty=(a_n)_{n=1}^\infty\) אינה חסומה מלעיל אך חסומה מלרע על ידי \(m=1\). 


    \item הסדרה \((a_n)_{n=1}^\infty=((-1)^{n}n)_{n=1}^\infty\) אינה חסומה לא מלעיל ולא מלרע. 


  \end{enumerate}
\end{example}
\begin{proposition}
כל סדרה מתכנסת היא חסומה.

\end{proposition}
\begin{proof}
  \begin{enumerate}
    \item נניח ש-\((a_n)_{n=1}^\infty\) מתכנסת. לכן יש לה גבול שנסמן ב-\(L\). 


    \item לפי הגדרת הגבול עבור \(\varepsilon=1\) נקבל \(N_{1}\in \mathbb{N}\) כך שלכל \(n>N_{1}\) מתקיים \(|a_{n}-L|<1\). כלומר \(L-1<a_{n}<L+1\). 


    \item כעת נגדיר את הקבועים הבאים: 
\begin{gather*}M=\max \left\{  L+1,a_{1}, a_{2},\dots,a_{n}  \right\} \\m=\min \left\{  L-1 ,a_{1},a_{2},\dots,a_{n} \right\}
\end{gather*}


    \item יהי \(n \in \mathbb{N}\). אם \(n\leq N_{1}\) אזי \(m\leq a_{n}\leq M\) ואם \(n>N_{1}\) מתקיים: 
$$m\leq L-1<a_{n}<L+1\leq M$$


  \end{enumerate}
\end{proof}
\begin{remark}
הטענה נותת קריטרון להתבדרות - אם סדרה היא לא חסומה - היא מתבדרת.

\end{remark}
\section{אריתמטיקה של גבולות}

\begin{proposition}[סכום של כפל בסקלר אי שלילי]
אם \((a_n)_{n=1}^\infty\) מתכנס אז גם \(\left( c\cdot a_n \right)_{n=1}^\infty\) מתכנס עבור \(c\geq 0\), ומתקיים:
$$\lim_{ n \to \infty } \left( c\cdot a_{n} \right)=c\cdot\lim_{ n \to \infty } a_{n}$$

\end{proposition}
\begin{proof}
נסמן את הגבול של \((a_n)_{n=1}^\infty\) ב-\(L\).
יהי \(\varepsilon>0\). מהגדרת הגבול עבור \(\frac{\varepsilon}{c}\) נקבל \(N \in \mathbb{N}\) כך שלכל \(n>N\) מתקיים:
$$|a_{n}-L|<\frac{\varepsilon}{c}\implies |c\cdot a_{n}-cL|<\varepsilon$$

\end{proof}
\begin{proposition}[גבול של נגדי]
אם \((a_n)_{n=1}^\infty\) מתכנס אז גם \((-a_n)_{n=1}^\infty\) מתכנס ומתקיים:
$$\lim_{ n \to \infty } (-a_{n})=-\lim_{ n \to \infty }a_{n} $$

\end{proposition}
\begin{proof}
יהי \(\varepsilon>0\). מהגדרת הגבול אם \(a_{n}\to L\) אז קיים \(N\in \mathbb{N}\) כך שלכל \(n>N\) מתקיים:
$$\lvert a_{n}-L \rvert <\varepsilon\implies \lvert -(-a_{n}+L) \rvert <\varepsilon\implies \lvert -a_{n}+L \rvert <\varepsilon$$
ולכן \(-a_{n}\to -L\).

\end{proof}
\begin{corollary}
אם \((a_n)_{n=1}^\infty\) מתכנס ל-\(L\) אז לכל מספר \(c \in \mathbb{R}\) מתקיים \(\left( c\cdot a_n \right)_{n=1}^\infty\) מתכנס ל-\(c\cdot L\).

\end{corollary}
\begin{theorem}[סכום של גבולות]
יהיו \((a_n)_{n=1}^\infty\) ו-\((b_n)_{n=1}^\infty\) סדרות מתכנסות. אזי הסדרה \((a_n+b_{n})_{n=1}^\infty\) מתכנסת, ומתקיים:
$$\lim_{ n \to \infty } a_{n}+b_{n}=\lim_{ n \to \infty } a_{n}+\lim_{ n \to \infty }b_{n} $$

\end{theorem}
\begin{proof}
  \begin{enumerate}
    \item נסמן \(A=\lim_{ n \to \infty }a_{n}\) ו-\(B=\lim_{ n \to \infty }b_{n}\) כאשר \((a_n)_{n=1}^\infty,(b_n)_{n=1}^\infty\) סדרות מתכנסות. 


    \item יהי \(\varepsilon>0\). מהגדרות הגבול עבור \(\frac{\varepsilon}{2}\) נקבל \(N_{A},N_{B}\in \mathbb{N}\) כך ש: 
$$\forall n>N_{A}\quad |a_{n}-A|<\frac{\varepsilon}{2}\qquad \forall n > N_{B}\quad |b_{n}-B|<\frac{\varepsilon}{2}$$
כעת נגדיר \(N=\max\{ N_{A},N_{B} \}\). לכל \(n>N\) נקבל:
$$\lvert (a_{n}+b_{n})-(A+B) \rvert =\lvert (a_{n}-a)+(b_{n}-B) \rvert \leq \lvert a_{n}-A \rvert +\lvert b_{n}-B \rvert \leq \frac{\varepsilon}{2}+\frac{\varepsilon}{2}=\varepsilon$$


  \end{enumerate}
\end{proof}
\begin{remark}
אם \((a_n)_{n=1}^\infty\) ו-\((b_{n})_{n=1}^\infty\) סדרות כך ש-\((a_n+b_{n})_{n=1}^\infty\) מתכנס, אז לא בהכרח \((a_n)_{n=1}^\infty\) ו-\((b_{n})_{n=1}^\infty\) מתכנסות. אבל אם בנוסף גם אחד הסדרות מתכנסות(למשל \((a_n)_{n=1}^\infty\)) אז מסכום גבולות ומהגבול של הנגדי נקבל כי \((a_n+b_{n}-a_{n})_{n=1}^\infty\) מתכנס כלומר \((b_n)_{n=1}^\infty\) מתכנס.

\end{remark}
\begin{theorem}[גבול של מכפלה]
אם \((a_n)_{n=1}^\infty\) ו-\((b_n)_{n=1}^\infty\) מתכנסות אזי \((a_nb_{n})_{n=1}^\infty\) מתכנס ומתקיים:
$$\lim_{ n \to \infty } a_{n}b_{n}=\lim_{ n \to \infty } a_{n}  \cdot\lim_{ n \to \infty } b_{n}$$

\end{theorem}
\begin{proof}
נסמן \(\lim_{ n \to \infty }a_{n}=A\) ו-\(\lim_{ n \to \infty }b_{n}=B\). יהי \(\varepsilon>0\).

  \begin{enumerate}
    \item נשים לב כי מתקיים: 
$$|a_{n}b_{n}-AB|=|a_{n}b_{n}-A\cdot b_{n}+A\cdot b_{n}-AB|=|(a_{n}-A)b_{n}+A(b_{n}-B)|\leq|b_{n}||a_{n}-A|+|A||b_{n}-B|$$


    \item הסדרה \((b_n)_{n=1}^\infty\) מתכנסת ולכן היא חסומה, וקיים \(M \in \mathbb{R}\) כך שלכל \(n \in \mathbb{N}\) מתקיים \(|b_{n}|\leq M\). 


    \item מהגדרת הגבול של \((a_n)_{n=1}^\infty\) עבור \(\frac{\varepsilon}{2M}\) נקבל כי החל מאיבר מסיים מתקיים \(|a_{n}-A|<\frac{\varepsilon}{2M}\) ולכן \(|b_{n}||a_{n}-A|<\frac{\varepsilon}{2}\). 


    \item מהגדרת הגבול של \((b_n)_{n=1}^\infty\) עבור \(\frac{\varepsilon}{2|A|}\) נקבל כי החל מאיבר מסיים מתקיים \(|b_{n}-B|<\frac{\varepsilon}{2|A|}\) ולכן \(|A||b_{n}-B|<\frac{\varepsilon}{2}\). 
נקבל סה"כ כי החל מאיבר מסיים מתקיים:
$$|a_{n}b_{n}-AB|\leq |b_{n}||a_{n}-A|+|A||b_{n}-B|< \frac{\varepsilon}{2}+\frac{\varepsilon}{2}=\varepsilon$$
ולכן נקבל כי \(a_{n}b_{n}\to AB\).


  \end{enumerate}
\end{proof}
\begin{remark}
בעקרון מגבול של מכפלה אפשר לקבל כמסקנה את הגבול של כפל בקבוע \(\left( c\cdot a_n \right)_{n=1}^\infty\) ע"י שימוש בסדרה הקבועה.

\end{remark}
\begin{theorem}[הגבול של הופכי]
תהי \((a_n)_{n=1}^\infty\) סדרה מתכנסת. אזי אם \(a_{n}\to L\) כאשר \(L\neq 0\) אזי \(\left( \frac{1}{a_{n}} \right)_{n=1}^\infty\) מתכנס ומתקיים:
$$\lim_{ n \to \infty } \frac{1}{a_{n}}=\frac{1}{L}$$

\end{theorem}
\begin{proof}
נוכיח כאשר \(L>0\). מתקיים:
$$\left\lvert  \frac{1}{a_{n}}-\frac{1}{L}  \right\rvert =\left\lvert  \frac{L - a_{n}}{a_{n}L}  \right\rvert =\frac{|a_{n}-L|}{|L||a_{n}|}=\frac{1}{|L|}\cdot \frac{1}{|a_{n}|}|L-a_{n}|=$$

\end{proof}
\section{גבלות ויחס סדר}

\begin{definition}
תהי \(P(n)\) טענה עבור המספר הטבעי \(n\). 

  \begin{enumerate}
    \item נאמר ש-\(P(n)\) מתקיימת תמיד אם הפסוק \(\forall n\quad P(n)\) הוא פסוק אמת 


    \item נאמר ש-\(P(n)\) מתקיימת כמעט תמיד(או החל ממקום מסויים, או לכל \(n\) מספיק גדול) אם קיים \(N \in \mathbb{N}\) כך שלכל \(n>N\) מתקיים \(P(n)\). 


    \item נאמר ש-\(P(n)\) תכונה שכיחה(או מתקיימת אינסוף פעמים) אם \(\forall N\in N \quad\exists n \in \mathbb{N}\quad n>N\land P(n)\). 


  \end{enumerate}
\end{definition}
\begin{example}
  \begin{enumerate}
    \item עבור \(\varepsilon>0\) התכונה \(P(n): \frac{1}{n}<\varepsilon\) היא תכונה שכיחה 


    \item התכונה "\(n\) ראשוני" היא תכונה שכיחה 


  \end{enumerate}
\end{example}
\begin{proposition}
אם \(P_{1}(n),P_{2}(n)\) מתקיימות כמעט תמיד אז \(Q(n)=P_{1}(n)\land P_{2}(n)\) מתקיים כמעט תמיד.

\end{proposition}
\begin{proof}
קיים \(N_{1} \in \mathbb{N}\) כך שלכל \(n>N_{1}\) מתקיים \(P_{1}(n)\). כמו כן קיים \(N_{2} \in \mathbb{N}\) כך שלכל \(n>N_{2}\) מתקיים \(P_{2}(n)\). לכן עבור כל \(N>\max(N_{1},N_{2})\) מתקיים \(P_{1}(n)\land P_{2}(n)\) ולכן מתקיים כמעט תמיד.

\end{proof}
\begin{remark}
מהגדרת הגבול נקבל כי עבור \(\varepsilon>0\) התכונה \(P(n):|a_{n}-L|<\varepsilon\) מתקיימת כמעט תמיד.

\end{remark}
\begin{proposition}
יהי \(P(n)\) טענה עבור מספר טבעי \(n\). אם \(\lnot P(n)\) אז 

\end{proposition}
\begin{theorem}[אי שיוויון חריף בין הגבולות של שתי סדרות מתכנסות]
יהיו \((a_n)_{n=1}^\infty,(b_{n})_{n=1}^\infty\) שתי סדרות מתכנסות. נסמן:
$$A=\lim_{ n \to \infty } a_{n}\qquad B=\lim_{ n \to \infty } b_{n}$$
אזי אם \(A<B\) אז החל ממקום מסויים \(a_{n}<b_{n}\)

\end{theorem}
\begin{proof}
  \begin{enumerate}
    \item נגדיר \(\varepsilon= \frac{B-A}{2}\) כאשר גדול מ-0 כי \(B<A\). 


    \item מהגדרת הגבול נקבל כי התכונות \(P_{A}(n):|a_{n}-A|<\varepsilon\) וכן \(P_{B}(n):|b_{n}-B|<\varepsilon\) מתקיימות כמעט תמיד. 


    \item מהטענה לעיל אנו יודעים כי התכונה: 
$$Q:|b_{n}-B|<\varepsilon \land|a_{n}-A|<\varepsilon$$
מתקיים כמעט תמיד. כלומר כמעט תמיד מתקיים:
$$a_{n}<A+\varepsilon=A+\frac{B-A}{2}=\frac{A+B}{2}=B-\varepsilon<b_{n}$$


  \end{enumerate}
\end{proof}
\begin{remark}
אם ידוע רק כי \(A\leq B\) לא ניתן להסיק כי \(a_{n}\leq b_{n}\). לדוגמא עבור \(a_{n}=1+\frac{1}{n}\) ו-\(b_{n}=1-\frac{1}{n}\). חייבים אי שיוויון חריף.

\end{remark}
\begin{theorem}[אי שיוויון שכיח בין איבר של 2 סדרות מתכנסות]
יהיו \((a_n)_{n=1}^\infty,(b_{n})_{n=1}^\infty\) שתי סדרות מתכנסות. אם \(a_{n}\leq b_{n}\) מתקיים אינסוף פעמים אז:
$$A=\lim_{ n \to \infty } a_{n}\leq \lim_{ n \to \infty } b_{n}=B$$

\end{theorem}
\begin{proof}
נניח בשלילה כי \(B<A\). אז מהמשפט הקודם החל ממקום מסויים \(b_{n}<a_{n}\). כלומר קיים \(N_{0} \in \mathbb{N}\) כך שלכל \(n>N_{0}\) מתקיים \(b_{n}<a_{n}\).  זאת אומרת שקיים  \(n_{0}>N_{0}\) כך שמתקיים \(a_{n_{0}}<b_{n_{0}}\), בסתירה. לכן \(A\leq B\).

\end{proof}
\begin{remark}
אם ידוע כי \(a_{n}<b_{n}\) איסנוף פעמים לא ניתן להסיק כי \(A<B\).  לדוגמא \(b_{n}=1+\frac{1}{n}>a_{n}=1-\frac{1}{n}\) לכל \(n\) אבל הגבולות שלהם שווים.

\end{remark}
\begin{definition}[סדרה כמעט קבוע]
סדרה שקבועה החל מאיבר מסיים. כלומר קיים \(\lambda \in \mathbb{R}\) ו-\(N \in \mathbb{N}\) כך שלכל \(n>N\) מתקיים \(a_{n}=\lambda\)

\end{definition}
\begin{proposition}
הגבול של סדרה כמעט קבוע עם קבוע \(\lambda \in \mathbb{R}\) יהיה \(\lambda\).

\end{proposition}
\begin{corollary}
אם \(\lambda \leq a_{n}\) אז \(\lambda \leq \lim_{ n \to \infty }a_{n}\).

\end{corollary}
\begin{theorem}[הכריך]
יהיו \((a_n)_{n=1}^\infty,(b_{n})_{n=1}^\infty\) ו-\((c_n)_{n=1}^\infty\) סדרות כך ש:

  \begin{enumerate}
    \item קיים \(N_{0} \in \mathbb{N}\) כך שלכל \(n>N_{0}\) מתקיים \(a_{n}\leq b_{n}\leq c_{n}\). 


    \item הסדרות \((a_n)_{n=1}^\infty,(c_n)_{n=1}^\infty\) מתכנסות 


    \item הגבול מקיים: 
$$L=\lim_{ n \to \infty } a_{n}=\lim_{ n \to \infty }c_{n} $$
אזי מתקיים כי \((b_{n})_{n=1}^\infty\) גם מתכנסת, ו-\(\lim_{ n \to \infty }b_{n}=L\).


  \end{enumerate}
\end{theorem}
\begin{proof}
יהי \(\varepsilon>0\).

  \begin{enumerate}
    \item הגבול \(a_{n}\to L\) נותן \(N_{1} \in \mathbb{N}\) כך שלכל \(n>N_{1}\) מתקיים \(|a_{n}-L|<\varepsilon\) ובפרט \(a_{n}>L-\varepsilon\). 


    \item הגבול \(c_{n}\to L\) נותן \(N_{2} \in \mathbb{N}\) כך שלכל \(n>N_{2}\) מתקיים \(|c_{n}-L|<\varepsilon\) ובפרט \(c_{n}<L+\varepsilon\). 


    \item נגדיר \(N=\max\{ N_{0},N_{1},N_{2} \}\). לכל \(n \in \mathbb{N}\) אם \(n\geq N\) מתקיים: 
$$L-\varepsilon \leq a_{n}\leq b_{n}\leq c_{n}<L+\varepsilon$$


    \item קיבלנו כי \(|b_{n}-L|<\varepsilon\) ולכן \(\lim_{ n \to \infty }b_{n}=L\). 


  \end{enumerate}
\end{proof}
\section{מונוטוניות}

\begin{proposition}[הלמה של קנטור]
יהיו \((a_n)_{n=1}^\infty,(b_n)_{n=1}^\infty\) סדרות המקיימות:
$$\forall n \in \mathbb{N}\quad a_{n}\leq a_{n+1}\leq b_{n+1}\leq b_{n}$$
אז קיימים \(c,d\) יחידים כך שמתקיים:
$$\bigcap_{n=1}^{\infty} [a_{n},b_{n}]=[c,d]$$

\end{proposition}
\begin{proof}
  \begin{enumerate}
    \item הסדרה \((x_{n})\) חסומה מלעיל ו-\((b_{n})\) חסומה מלרע כיוון ש: 
$$\forall n \in \mathbb{N}\quad  a_{n}\leq b_{1}\land b_{n}\leq a_{1}$$


    \item לכן כיוון שמונוטניות מתקיים: 
$$\lim_{ n \to \infty } a_{n}=\sup a_{n}\quad \lim_{ n \to \infty }b_{n}=\inf b_{n} $$


    \item נניח בשלילה כי: 
$$\lim_{ n \to \infty } a_{n}=\sup a_{n}>\inf b_{n}=\lim_{ n \to \infty }b_{n}$$
כלומר קיים \(n \in \mathbb{N}\) שעבורו \(a_{n}> b_{n}\). בגלל משפט אי שיוויון חריף בין גבולות נקבל סתירה לנתון.


    \item נסמן את הגבול של \((a_{n})\) ב-\(c\) ואת הגבול של \((b_{n})\) ב-\(d\). כעת קיבלנו כי \(c\leq d\) ולכן לכל \(n \in \mathbb{N}\) מתקיים: 
$$c\leq a_{n},b_{n}\leq d\implies \bigcap_{n=1}^{\infty} [a_{n},b_{n}]=[c,d]$$


  \end{enumerate}
\end{proof}
\section{שורשים וממוצעים}

\begin{theorem}[צ'סרו]
אם \((a_n)_{n=1}^\infty\) מתכנס ל-\(L\), אז גם הסדרה של הממוצעים החשבוניים של \((a_n)_{n=1}^\infty\) מתכנס ל-\(L\).

\end{theorem}
\begin{proof}
מקרה 1 - \(L=0\). יהי \(\varepsilon>0\). מתקיים:
$$\lim_{ n \to \infty } a_{n}=L=0$$
ולכן חסומה, וקיים \(M \in \mathbb{R}\) כך שמתקיים:
$$\forall n \in \mathbb{N}\quad a_{n}\leq M$$
נגדיר \((b_n)_{n=1}^\infty\) סדרת הממוצעים החשבוניים. מתקיים:
$$\lvert b_{n}-0 \rvert =\left\lvert   \frac{a_{1}+\dots+a_{n}}{n}-0  \right\rvert \leq \frac{\lvert a_{1} \rvert +\left\lvert  a_{2}  \right\rvert + \dots +\lvert a_{n} \rvert }{\lvert n \rvert }$$
לפי הגדרת הגבול עבור \(\frac{\varepsilon}{2}\) נקבל \(N_{1}\) כך שלכל \(n> N_{1}\) מתקיים \(\lvert a_{n} \rvert<\frac{\varepsilon}{2}\). כעת:
$$\frac{\lvert a_{1} \rvert +\dots+\lvert a_{N_{1}} \rvert +\lvert a_{N_{1}+1} \rvert +\dots+\lvert a_{n} \rvert}{n}< \frac{N_{1}\cdot M }{n}+\frac{\varepsilon}{2} $$
מארכימדיות קיים איבר \(n> \frac{2N_{1}M}{\varepsilon}\) ונסמן אותו ב-\(N_{2}\). נגדיר \(N=\max\{ N_{1},N_{2} \}\) ונקבל:
$$N_{1}\cdot \frac{M}{n}+\frac{\varepsilon}{2}<\frac{\varepsilon}{2}+\frac{\varepsilon}{2}=\varepsilon$$
מקרה 2 - \(L\neq 0\). נגדיר \(a'_{n}=a_{n}-L\) ונקבל כי \(\lim_{ n \to \infty }a'_{n}=0\) מאריתמטיקה של גבולות. כעת:
\begin{gather*}\left\lvert  \frac{a_{1}'+\dots+a_{n}'}{n}  \right\rvert =\left\lvert  \frac{(a_{1}-L)+\dots+(a_{n}-L) }{n} \right\rvert =\left\lvert  \frac{a_{1}+\dots+a_{n}-nL}{L}  \right\rvert = \\=\left\lvert  \frac{a_{1}+\dots+a_{n}}{L}-L   \right\rvert   <\varepsilon
\end{gather*}
ולכן מהגדרת הגבול גבול הסדרה \((a_{n})\) יהיה \(L\).

\end{proof}
\section{תתי סדרות}

\begin{definition}[תת סדרה]
תהי \((a_n)_{n=1}^\infty\) סדרה ממשית נתונה. סדרה \((b_k)_{k=1}^\infty\) נקראת תת סדרה של \((a_n)_{n=1}^\infty\) אם קיימת סדרה מונוטונית עולה \underline{ממש}
של אינדקסים \((n_{k})_{k=1}^\infty\) כך ש-\(n_{k}\in \mathbb{N}\) ושכל \(\mathbb{N}\in k\) מתקיים:
$$b_{k}=a_{n_{k}}$$

\end{definition}
\begin{proposition}
אם \((n_{k})\) סדרה של מספרים טבעיים מונוטנית עולה ממש אזי לכל \(k \in \mathbb{N}\) מתקיים:
$$k\leq n_{k}$$

\end{proposition}
\begin{proof}
באידוקציה על \(k\). בסיס האינדוקציה: עבור \(k=0\) נקבל כי \(1\leq n_{1}\in \mathbb{N}\).
עבור צעד האינדוקציה נניח כי \(k\leq n_{k}\). כיוון ש-\((n_{k})\) מונוטונית עולה ממש לכן \(k<n_{k}<n_{k+1}\) וכיוון ש-\(n_{k+1}\) מספר טבעי נקבל \(k+1\leq n_{k+1}\).

\end{proof}
\begin{theorem}[ירושה]
תהי \(a_{n_{k}}\) תת סדרה של \((a_{n})\). אז מתקיים:

\end{theorem}
\begin{enumerate}
  \item אם \((a_{n})\) חסומה אז גם \((a_{n_{k}})\) תהיה חסומה. 


  \item אם \((a_{n})\) מונוטונית אז גם \((a_{n_{k}})\) תהיה מונוטונית. 


  \item אם \((a_{n})\) מתכנסת אז גם \((a_{n_{k}})\) מתכנסת ולאותו גבול. 


\end{enumerate}
\begin{definition}[גבול חלקי]
נתונה סדרה \((a_n)_{n=1}^\infty\). מספר ממשי \(\lambda \in \mathbb{R}\) נקרא גבול חלקי של \((a_n)_{n=1}^\infty\) אם קיימת תת סדרה של \((a_n)_{n=1}^\infty\) אשר מתכנסת ל-\(\lambda\).

\end{definition}
\begin{proposition}
לכל סדרה קיימת תת סדרה מונוטונית

\end{proposition}
\begin{definition}[פסגה]
מספר טבעי יקרא פסגה של סדרה \((a_n)_{n=1}^\infty\) אם לכל \(n\geq m\) מתקיים \(a_{m}\geq a_{n}\). כאשר נשים לב כי אז:
$$a_{m}=\max \left\{  a_{n}\mid n\geq m  \right\}$$

\end{definition}
כדי להוכיח את המשפט נפצל ל-2 למות:

\begin{lemma}
אם יש אינסוף פסגות, קיימת תת סדרה מונוטנית יורדת.

\end{lemma}
\begin{proof}
  \begin{enumerate}
    \item נגדיר אוסף קבוצות באופן רקורסיבי. ראשית נגדיר: 
$$A_{1}=\left\{  m \in \mathbb{N} \mid \text{\(m\) is peak}  \right\}$$
בכאשר פרט \(A_{1}\neq \varnothing\). לכן מעקרון הסדר הטוב קיים \(m_{1} \in \mathbb{N}\) כך ש-\(m_{1}=\min A_{1}\). עכשיו נגדיר \(A_{2}=A_{1}\setminus A_{1}\) ועבור קיים מינימום \(m_{2}\). כאשר נשים לב כי מתקיים \(m_{1}<m_{2}\). כעת נניח כי הגדרנו \(m_{1},\dots,m_{k}\) כך שמתקיים:
$$m_{1}<m_{2}<\dots<m_{k}$$
וקבוצות \(A_{1},\dots,A_{k}\) כך ש:
$$A_{k}=\left\{  m \in \mathbb{N} \mid \text{\(m\) is peak }\land m_{k}<m \right\}$$
ו-\(m_{k+1}=\min A_{k+1}\) ולכן(כיוון ש-\(m_{k}\) פסגה) \(a_{m_{k}}\geq a_{m_{k+1}}\) ן-\((m_{k})\) היא סדרה מונוטונית עולה ממש, ולכן נותת תת סדרה \((a_{m_{k}})\) של \((a_{n})\) מזה שלכל \(k \in \mathbb{N}\) מתקיים \(a_{m_{k}}\geq a_{m_{k+1}}\) רואים כי זו תת סדרה מונוטונית יורדת.
  \end{enumerate}
\end{proof}
\begin{lemma}
אם יש כמות סופית של פסגות, קיימת תת סדרה מונוטונית עולה.

\end{lemma}
\begin{proof}
נניח כי קיימות כמות סופית של איברי פסגה. לכן קיים \(N\in\mathbb{N}\) כך שכל \(n>N\) הוא לא איבר פסגה. נבנה ברקורסיה סדרה מונוטונית עולה

\end{proof}
\begin{enumerate}
  \item נגדיר \(n_{1}=N+1\). נשים לב כי \(n_{1}>N\) ולכן לא איבר פסגה. כלומר קיים \(n_{2} \in \mathbb{N}\) עם \(n_{2} > n_{2}\) כך ש-\(a_{n_{1}}\leq a_{n_{2}}\). 


  \item נניח כי הגדרנו \(n_{1}<n_{2}<\dots<n_{k}\) כך ש: 
$$a_{n_{1}}\leq a_{n_{2}}\leq \dots \leq a_{n_{k}}$$
ואז \(n_{k}>n_{1}>N\) ולכן \(n_{k}\) לא איבר פסגה. כלומר קיים \(n_{k+1}\in \mathbb{N}\) כך ש-\(n_{k+1}>n_{k}\) ו-\(a_{n_{k}}\leq a_{n_{k+1}}\) ונקבל כי \((a_{n_{k}})\) מונוטונית עולה.


\end{enumerate}
משתי למות אלה, והעבודה שייתכן או שקיים כמות סופית של פסוגות או כמות אינסופית של פסגות הוכחת המשפט מיידי

\begin{theorem}[בולצנו ווירשטראס]
לכל סדרה חסומה יש תת סדרה מתכנסת

\end{theorem}
\begin{proof}
מהטענה הקודמת קיימת תת סדרה מונוטונית, כאשר מירושה היא תהיה חסומה, ולכן קיימת תת סדרה מונוטונית וחסומה, כאשר ראינו כי סדרה כזו מתכנסת.

\end{proof}
\begin{proposition}
יהיו \((b_{n}),(a_{n})\) סדרות חסומות. אז קיימת תת סדרה \(n_{k}\) כך ש-\((a_{n_{k}})\) ו-\((b_{n_{k}})\) שתיהם מתכנסות.

\end{proposition}
\begin{proof}
כיוון ש-\((a_{n})\) חסומה לפי בוצלנו ווירשטראס קיימת תת סדרה כך ש-\((a_{n_{k}})\) מתכנסת. נסתכל על \((b_{n_{k}})\). כיוון ש\((b_{n})\) חסומה גם \((b_{n_{k}})\) חסומה ולכן קיימת עבורה תת סדרה כך ש-\((b_{n_{k_{l}}})\) מתכנסת. כעת, \((a_{n_{k_{l}}})\) גם כן מתכנסת מירושה. ולכן קיימת תת סדרה שעבורה גם \((a_{n_{k_{l}}})\) וגם \((b_{n_{k_{l}}})\) מתכנסת.

\end{proof}
\begin{proposition}
יהי \(\lambda \in \mathbb{R}\). אזי \(\lambda\) גבול חלקי של \((a_n)_{n=1}^\infty\) אם"ם לכל \(\varepsilon>0\) התכונה \(a_{n}-\lambda<\varepsilon\) היא תכונה שכיחה.

\end{proposition}
\section{סדרות אפסות וסדרות השואפות לאינסוף}

\begin{definition}[סדרה שואפת לאינסוף]
תהי \((a_n)_{n=1}^\infty\) סדרה ב-\(\mathbb{R}\). נאמר שהסדרה \((a_n)_{n=1}^\infty\) שואפת אינסוף(או מתבדרת לאינסוף) אם:
$$\forall M \in \mathbb{R}\quad \exists N \in \mathbb{N}\quad \forall n \in \mathbb{N}\quad  n>\mathbb{N}\implies a_{n}>M$$
ונסמן \(\underset{ n \to \infty }{\lim }a_{n}=\infty\).

\end{definition}
\begin{definition}[סדרה שואפת למינוס אינסוף]
תהי \((a_n)_{n=1}^\infty\) סדרה ב-\(\mathbb{R}\). נאמר שהסדרה \((a_n)_{n=1}^\infty\) שואפת למינוס אינסוף(או מתבדרת למינוס אינסוף) אם:
$$\forall M \in \mathbb{R}\quad \exists N \in \mathbb{N}\quad \forall n \in \mathbb{N}\quad  n>\mathbb{N}\implies a_{n}<M$$
ונסמן \(\underset{ n \to \infty }{\lim }a_{n}=\infty\).

\end{definition}
\begin{remark}
בהוכחה של שאיפה לאינסוף ניתן לקחת \(M>0\) ובשאיפה למינוס אינסוף ניתן לקחת \(M< 0\). ככה לא צריך להסתבך עם הכפלה באי שיוויונות.

\end{remark}
\begin{definition}[סדרה אפסה]
סדרה אשר מתכנסת לאפס.

\end{definition}
\begin{proposition}
אם הסדרה \((a_n)_{n=1}^\infty\) שואפת לאינסוף או מינוס אינסוף אז \(\frac{1}{a_{n}}\) היא סדרה אפסה.

\end{proposition}
נסכם את המקרים השונים בטלה

\begin{table}[htbp]
  \centering
  \begin{tabular}{|cccccc|}
    \hline
    \(a_n\cdot b_n\) & \(L_2 > 0\) & \(L_2 < 0\) & \(0\) & \(\infty\) & \(-\infty\) \\ \hline
    \(L_1>0\) & \(L_1\cdot L_2\) & \(L_1\cdot L_2\) & \(0\) & \(\infty\) & \(-\infty\) \\ \hline
    \(L_1<0\) & \(L_1\cdot L_2\) & \(L_1\cdot L_2\) & \(0\) & \(-\infty\) & \(\infty\) \\ \hline
    \(L_{1}=0\) & \(0\) & \(0\) & \(0\) & \(?\) & \(?\) \\ \hline
    \(L_{1}=\infty\) & \(\infty\) & \(-\infty\) & \(?\) & \(\infty\) & \(-\infty\) \\ \hline
    \(L_1=-\infty\) & \(-\infty\) & \(\infty\) & \(?\) & \(-\infty\) & \(\infty\) \\ \hline
  \end{tabular}
\end{table}
כאשר \(?\) מסמן שיכול להיות כל אחת מהאפשרויות האחרות.

\section{סדרות קושי}

\begin{definition}[סדרת קושי]
סדרה \((a_n)_{n=1}^\infty\) המקיימת:
$$\forall \varepsilon > 0\quad \exists N \in \mathbb{N}\quad \forall n,m \in \mathbb{N}\quad n,m>N\implies \lvert a_{n}-a_{m} \rvert <\varepsilon$$
תנאי שזה שקול לתנאי 
$$\forall \varepsilon > 0\quad \exists N \in \mathbb{N}\quad \forall n,m \in \mathbb{N}\quad n,k>N\implies \lvert a_{n}-a_{n+k} \rvert <\varepsilon$$

  \begin{itemize}
    \item העקרון כאן הוא שזוהי סדרה שהאיברים שלה מתקרבים אחד לשני.
    \item נועד לעזור להראות שסדרה מתכנסת בלי למצוא גבול.
  \end{itemize}
\end{definition}
\begin{proposition}
סדרה מתכנסת היא קושי

\end{proposition}
\begin{proof}
נעזר באי שיוויון המשולש - המרחק מ-\(a_{n}\) לגבול קטן או שווה למרחק מ-\(a_{m}\) כאשר \(n<m\) ועוד המרחק מ-\(a_{m}\) לגבול.
נניח כי \(a_{n}\xrightarrow{n\to \infty}L \in \mathbb{R}\). יהי \(\varepsilon>0\). מהגדרת הגבול עבור \(\frac{\varepsilon}{2}\) קיים \(N_{2} \in \mathbb{N}\) כך שמתקיים לכל \(n>N\) כי \(\lvert a_{n}-L \rvert<\frac{\varepsilon}{2}\) וכן
עבור \(m>N\) מתקיים באופן זהה \(\lvert a_{m}-L \rvert<\frac{\varepsilon}{2}\). לכן:
$$\lvert a_{n}-a_{m} \rvert \leq \lvert a_{n}-L \rvert +\lvert L-a_{m} \rvert <\frac{\varepsilon}{2}+\frac{\varepsilon}{2}=\varepsilon$$

\end{proof}
\begin{proposition}
סדרת קושי היא סדרה מתכנסת

\end{proposition}
\begin{proof}
נחלק הוכחה זו לשלושה חלקים. 1 - נראה שחסומה. 2 - נמצא מועמד לגבול. 3 - נוכיח שמתכנס לגבול.

  \begin{enumerate}
    \item ניקח \(\varepsilon = 1\). תנאי קושי נותן לנו \(N \in \mathbb{N}\) כך שלכל \(n,m> N\) מתקיים \(\lvert a_{n}-a_{m} \rvert< 1\). בפרט, אם ניקח \(N<n_{0}=N+1\) ולכל \(m>n_{0}\) מתקיים \(\lvert a_{n}-a_{m} \rvert<1\), לכן: 
$$a_{n_{0}}-1<a_{m}<a_{n_{0}}+1$$
נגדיר:
$$M=\max \left\{  a_{1},\dots,a_{N},a_{N+1}+1  \right\}\quad m=\min \left\{  a_{1},\dots,a_{N+1}-1  \right\}$$
לכל \(n \in \mathbb{N}\) אם \(n\leq N\) נקבל \(m\leq a_{n}\leq M\). אם \(n>N\) נקבל:
$$m\leq a_{N+1}-1<a_{n}<a_{N+1}+1\leq M$$
ולכן הסדרה חסומה.


    \item נמצא מועמד לגבול בעזרת בולצנו ווירשטרס. כיוון שהסדרה חסומה, קיימת תת סדרה \((a_{n_{k}})_{k=1}^\infty\) אשר מתכנס לגבול \(L\). 


    \item כעת נראה שהסדרה כולה מתכנסת לגבול \(L \in \mathbb{R}\). יהי \(\varepsilon > 0\). מהתכנסות של \(\left\lvert  (a_{n_{k}})_{n=1}^\infty  \right\rvert\) נקבל \(K \in \mathbb{N}\) כך שלכל \(k>K\) מתקיים: 
$$\lvert a_{n}-L \rvert <\frac{\varepsilon}{2}$$
מתנאי קושי נקבל \(N \in \mathbb{N}\) כך שלכל \(n,m \in \mathbb{N}\) מתקיים \(\lvert a_{n}- a_{m} \rvert\leq \frac{\varepsilon}{2}\). נבחר \(k_{0} \in \mathbb{N}\) כך שגם \(k_{0}>K\) וגם \(n_{k_{0}}>N\)(למשל \(k_{0}=\max(K,N)+1\)). נקבל:
$$\lvert a_{n_{k_{0}}}-L \rvert <\frac{\varepsilon}{2} \land \lvert a_{n_{k_{0}}}-a_{n} \rvert <\frac{\varepsilon}{2}$$
ונקבל כי לכל \(n \in \mathbb{N}\) עם \(n>N\):
$$\lvert a_{n}-L \rvert =\lvert (a_{n}-a_{n_{k}})+(a_{n_{k}}-L) \rvert \leq \lvert a_{n}-a_{n_{k}} \rvert +\lvert a_{n_{k}}-L \rvert \leq \frac{\varepsilon}{2}+\frac{\varepsilon}{2}=\varepsilon$$


  \end{enumerate}
\end{proof}
\begin{corollary}[קריטריון קושי להתכנסות של סדרות ממשיות]
סדרה ממשית \((a_n)_{n=1}^\infty\) מתכנסת אם"ם סדרת קושי.

\end{corollary}
\begin{remark}
לא מספיק לאיברים מתקרבים אחד לשני. למשל אם נסתכל על הסדרה \(\sqrt{ n }\) נקבל כי:
$$\sqrt{ n+1 }-\sqrt{ n }\to 0$$
אך הסדרה אינה מתכנסת.

\end{remark}
\chapter{טורים ממשיים}

\section{הגדרות ותכונות של טורים}

\begin{definition}[סדרת סכומים חלקיים]
אם \((a_{n})\) סדרה אז סדרת הסכומים החלקיים תהיה:
$$S_{k}=\sum_{n=1}^{k} a_{n}$$

\end{definition}
\begin{definition}[טור]
יהי \((a_{n})\) סדרה. נגדיר את הטור \(\sum a_{n}\) להיות:

\end{definition}
$$\sum a_{n}=\lim_{ k \to \infty } S_{k}=\lim_{ k \to \infty } \sum_{n=1}^{k} a_{k}$$
כאשר אם הגבול הזה קיים במובן הצר נאמר כי הטור מתכנס, כאשר אם לא נאמר כי הטור מתבדר.

\begin{remark}
נשים לב כי אם הטור מתכנס הביטוי \(\sum a_{n}\) זה פשוט מספר.

\end{remark}
\begin{proposition}
אם טור מתכנס אז הסדרה \((a_{n})\) שואפת לאפס.

\end{proposition}
\begin{proposition}
אם טור אי שלילי \(\sum a_{n}\) מתכנס אזי הסדרה \(a_{n}\) שואפת ל-0.

\end{proposition}
\begin{proposition}[אריתמטיקה של התכנסות טורים]
  \begin{enumerate}
    \item אם \(\sum a_{n}\) ו-\(\sum b_{n}\) שניהם מתכנסים, אז גם \(\sum(a_{n}+b_{n})\) מתכנס ומתקיים: 
$$\sum (a_{n}+b_{n})=\sum a_{n}+\sum b_{n}$$


    \item אם \(\sum a_{n}\) מתכנס ו-\(c \in \mathbb{R}\) אז \(\sum(ca_{n})\) מתכנס ומתקיים: 
$$\sum(c a_{n})=c \sum a_{n}$$


  \end{enumerate}
\end{proposition}
\begin{definition}[זנב של טור]
ה-\(N\) זנב של טור היא סדרה שמתחלה ב-\(N+1\). מסמן \(\sum a_{n+N}\).

\end{definition}
\begin{proposition}
טור מתכנס אם"ם זנב שלו מתכנס.

\end{proposition}
\begin{proposition}
הטור \(\frac{1}{n^{p}}\) מתבדר עבור \(p<1\) ומתכנס עבור \(p>1\).

\end{proposition}
הוכחה של טענה זו נעשת באמצעות מבחן האינטגרל אשר נראה בהמשך.

\section{הכנסת סוגריים ושינוי סדר}

\begin{definition}[הכנסת סוגריים]
"לקבץ" איברים ולסכום אותם ביחד. אחרי הקיבוץ איברים לא מקבלים אותה סדרה. סדרה זו תהיה תת סדרה של סדרת הסכומים החלקיים

\end{definition}
\begin{example}
נסתכל על הטור \(\sum(-1)^{n+1}\). זהו טור שסדרת הסכומים החלקיים שלו משתנה בין 0 ל-1. כלומר:
$$S_{n}=\left( 1,0,1,0,1,0,\dots \right)$$
כאשר זהו טור שבפרוש מתבדר. אם נכניס סוגריים כל שתי איברים אבל נקבל את הטור 0, והטור הזה הוא טור מתכנס.

\end{example}
\begin{proposition}
אם הטור מתכנס אז ניתן להכניס סוגרים וערך הטור לא ישתנה.

\end{proposition}
\begin{proposition}
אם קיימים 2 הכנסות סוגריים עם תוצאות שונות אז הטור מתבדר.

\end{proposition}
\begin{proposition}
הכנסת סוגריים על איברים שווי סימן לא משנה את התכנסות הטור או את ערך הטור.

\end{proposition}
\begin{proof}
  \begin{enumerate}
    \item נניח כי \(\sum b_{n}\) הוא הטור המתקבל מהכנסת סוגריים של \(\sum a_{n}\) עבור איברי שווי סימן. נסמן ב-\((S_{n})\) את סדרת הסכומים החלקיים של \(\sum a_{n}\) וב-\((T_{n})\) את סדרת הסכומים החלקיים של \(\sum b_{n}\). 


    \item כיוון ש-\((b_{n})\) הכנסת סוגריים מתקיים \(T_{k}=S_{n_{k}}\) עבור כל \(k \in \mathbb{N}\). 


    \item לכל \(n \in \mathbb{N}\) יש \(k \in \mathbb{N}\) יחיד כך ש: 
$$n_{k-1}<n\leq n_{k}$$ 
ומתקיים:
$$S_{n_{k-1}}= S_{n}+ \sum_{j=n_{k}+1}^{n_{k+1}} a_{j}$$


    \item אם \(a_{n_{k+1}},\dots,a_{n_{k}+1}\) כלום אי חיוביים אז: 
$$S_{n_{k-1}}\leq S_{n}\leq S_{n_{k}}$$
בכל מקרה, מתקיים:
$$\min(S_{n_{k-1}},S_{n_{k}})\leq S_{n}\leq \max (S_{n_{k-1}},S_{n_{k}})$$


    \item אם \(\sum b_{n}\) מתכנס אז \(\underset{ n \to \infty }{\lim }S_{n_{k}}=L\) ולכן: 
$$\lim_{ n \to \infty } \max (S_{n_{k}},S_{n_{k-1}})=\lim_{ n \to \infty } \min(S_{n_{k}},S_{n_{k-1}})$$
וקיבלנו כי \(\sum a_{n}\) מתכנס ממשפט הסנדוויץ.


  \end{enumerate}
\end{proof}
\begin{proposition}
הוספת סוגריים באורך חסום לא משנה את התכנסות הטור או את ערך הטור כאשר הטור שואף ל-0.

\end{proposition}
\begin{proof}
יהי \(\sum b_{k}\) טור המתקבל מהכנסת סוגרים של \(\sum a_{n}\) במקומות ה-\(n_{k}\). נניח כי \(n_{k+1}-n_{k}\leq M\) לכל \(k \in \mathbb{N}\)(כלומר הכנסת סוגריים באורך חסום). נסמן ב-\((S_{n})\) את סדרת הסכומים החלקיים של \(\sum a_{n}\). נשים לב כי מתקיים:
$$\sum_{k=1}^{\infty}b_{k}=S_{n_{k}} $$
נניח \(\underset{ n \to \infty }{\lim }a_{n}=0\). יהי \(\varepsilon > 0\).  מהגדרת הגבול קיים \(N_{1}> 0\) כך שלכל \(n>N_{1}\) מתקיים:
$$\lvert a_{n} \rvert =\lvert a_{n}-0 \rvert < \frac{\varepsilon}{M}$$
אנו יודעים כי לכל \(n \in \mathbb{N}\) קיים \(k_{0} \in \mathbb{N}\) כך שמתקיים:
$$n_{k_{0}}<n\leq n_{k_{0}+1}$$
כאשר נשים לב להבדל בין הסדרה \(n_{k}\) למספר הטבעי \(n\). כעת עבור \(n>N\) מתקיים:
$$S_{n_{k_{0}}}-S_{n}=\sum_{j=n_{k_{0}}}^{n_{k_{0}+1}}a_{j}\leq \sum_{j=n_{k_{0}}}^{n_{k_{0}+1}}\lvert a_{j} \rvert \leq \frac{\varepsilon}{M}(n_{k_{0}+1}-n_{k_{0}})\leq \frac{\varepsilon}{M}\cdot M=\varepsilon $$
ולכן \(\underset{ n \to \infty }{\lim }S_{n_{k_{0}}}-S_{n}=0\) ולכן אם \(S_{n}\) מתכנס ל-\(L\) נקבל מאריתמטיקה של גבולות כי:
$$\lim_{ n \to \infty } S_{n_{k_{0}}}=\lim_{ n \to \infty } S_{n_{k_{0}}}-S_{n}+S_{n}=\lim_{ n \to \infty } (S_{n_{k_{0}}}-S_{n})+\lim_{ n \to \infty } S_{n}=0+L$$

\end{proof}
\begin{definition}[שינוי סדר איברים]
סדרה \(b_n\) היא שינוי סדר של \(a_n\) אם קיימת תמורה \(\sigma(n)\) (חח"ע ועל) כך ש-\(b_n = a_{\sigma(n)}\).

\end{definition}
\begin{proposition}
אם טור הוא מתכנס בהחלט ניתן לשנות את הסדר וערך הטור לא ישתנה

\end{proposition}
\begin{theorem}[רימן לטורים]
אם מתכנס בתנאי ניתן לבנות טור עם החלפת סדר שמתבדר, או אפילו שווה לכל מספר ממשי אפשרי(משפט רימן)

\end{theorem}
\begin{proposition}
אם קיימים 2 ערכים שונים של הטור שמתקבלים ע"י שינוי סדר, אז הטור לא מתכנס בהחלט

\end{proposition}
\begin{proposition}
ביצוע תמורה באורך חסום על האינדקסים של הטור(כלומר קיים \(M\) כך ש-\((\sigma(n) - n)<M\)) לא משנה את התכנסות הטור או את ערך הטור.

\end{proposition}
\section{טורים כלליים}

\begin{definition}[התכנסות בהחלט]
טור \(\sum a_{n}\) מתכנס בהחלט אם \(\sum\lvert a_{n} \rvert\) מתכנס

\end{definition}
\begin{proposition}
אם הטור המקורי \(\sum a_{n}\) מתכנס אזי הטור מתכנס בהחלט, כלומר \(\sum\lvert a_{n} \rvert\) מתכנס.

\end{proposition}
\begin{proposition}
אם טור מתכנס בהחלט אז ניתן לשנות את איבר הסכימה.

\end{proposition}
\begin{proposition}
מכפלה של 2 טורים מתכנסים בהחלט היא הקונבולוציה הבדידה שלהם, כלומר:
$$\sum_{n=0}^\infty a_n \cdot \sum_{n=0}^\infty b_n = \sum_{m=1}^\infty c_m \qquad c_m = \sum_{n=0}^m a_n b_{m-n}$$

\end{proposition}
\begin{example}
נמצא את הטור השקול של \(\sum \left(\frac{1}{2}\right)^n \cdot \sum \left(\frac{1}{2}\right)^n\). מתקיים:
$$c_n = \sum_{n=0}^m a_n b_{m-n} = \sum_{n=0}^m \left(\frac{1}{2}\right)^n \cdot \left(\frac{1}{2}\right)^{m-n}=$$$$= \sum_{n=0}^m \left(\frac{1}{2}\right)^m = \left(\frac{1}{2}\right)^m \sum_{n=0}^m 1 = (m+1)\left(\frac{1}{2}\right)^m$$
ונקבל:
$$\sum \left(\frac{1}{2}\right)^n \cdot \sum \left(\frac{1}{2}\right)^n = \sum_{m=0}^\infty (m+1) \left(\frac{1}{2}\right)^m$$

\end{example}
\begin{definition}[טור ליבניץ']
טור אי שלילית, מונוטונית יורדת ואפסה. לפי משפט לייבניץ טור מהצורה \(\sum(-1)^{n+1} a_n\) מתכנס.

\end{definition}
\begin{proposition}
טור ליבניץ מתכנס.

\end{proposition}
\begin{proof}
נסמן ב-\(S_{n}\) את סדרת הסכומים החלקיים. נראה ש-\((S_{2n})\) תת סדרה עולה:
$$\forall k \in \mathbb{N}\quad S_{2(k+1)}-S_{2k}=\sum_{n=1}^{2k+2}(-1)^{n+1}a_{n}-\sum_{n=1}^{2k}(-1)^{n+1} a_{n}=a_{2k+2}+a_{2k+1}\geq 0$$
ולכן מונוטונית עולה. כעת נראה ש-\((S_{2n-1})\) תת סדרה יורדת:
$$\forall k \in \mathbb{N}\quad S_{2(k-1)-1}-S_{2k-1}=\dots =a_{2k+1}-a_{2k}\leq 0$$
כעת נשים לב כי \(S_{2k+1}\leq S_{2k-1}\) כיוון ש:
$$\forall k \in \mathbb{N}\quad  S_{2(k+1)}-a_{2k}\leq S_{2k-1}$$
כיוון שגם:
$$\lim_{ k \to \infty } S_{2k-1}-S_{2k}=\lim_{ k \to \infty } a_{2k}=0 $$
אז הסדרות \((S_{2k}),(S_{2k-1})\) מקיימות את תנאי הלמה של קנטור ולכן:
$$\lim_{ n \to \infty } S_{n}=\lim_{ k \to \infty } S_{2k}=\lim_{ k \to \infty } S_{2k-1}=L$$

\end{proof}
\begin{proposition}[מבחן דיריכלה]
אם הטור \(\sum b_n\) חסום והסדרה \((a_n)\) מונוטונית ואפסה אז \(\sum (a_n b_n)\) מתכנס.

\end{proposition}
\begin{proposition}[מבחן אבל]
אם \(a_n\) מונוטונית וחסומה ו-\(\sum b_n\) מתכנס אז \(\sum (a_n b_n)\) מתכנס.

\end{proposition}
\section{מבחני התכנסות טורים אי שליליים}

\begin{theorem}[מבחן ההשוואה]
יהיו \((a_n)_{n=1}^\infty\) ו-\((b_n)_{n=1}^\infty\) סדרות כך ש-\(0\leq a_{n}\leq b_{n}\) לכל \(n \in \mathbb{N}\). אם \(\sum b_{n}\) מתכנס אז \(\sum a_{n}\) מתכנס. ואם \(\sum a_{n}\) מתבדר אז \(\sum b_{n}\) מתבדר.

\end{theorem}
\begin{proof}
יהיו \((T_{k}),(S_{k})\) סדרות הסכומים של \((b_{n}),(a_{n})\) בהתאמה. נניח כי \(\sum b_{n}\) מתכנס. מתקיים \((T_{k}),(S_{k})\) מונוטוניות עולות כיוון שהסדרות אי שליליות. כיוון ש-\(\sum b_{n}\) מתכנס נקבל כי \((T_{k})\) מתכנס. מזה נקבל:
$$\forall n \in \mathbb{N} \quad a_{n}\leq b_{n}$$
וכן מתקיים:
$$S_{k}=\sum_{i=1}^{n} a_{k}\leq \sum_{k=1}^{\infty}b_{k}=T_{k}$$
וכן \(T_{k}\) מתכנס ולכן חסומה ולכן \(S_{k}\) חסומה מלעיל ומונוטונית ולכן מתכנסת. ההוכחה עבור התבדרות כמעט זהה.

\end{proof}
\begin{proposition}[מבחן המנה]
אם \(a_n, b_n\) אי-שליליים ו-\(\frac{a_n}{b_n} = L >0\) אז \(\sum a_n , \sum b_n\) מתכנסים ומתבדרים יחד

\end{proposition}
\begin{proposition}[מבחן המנה הגבולי]
אם \(a_n, b_n\) אי-שליליים ו-\(\lim_{n\rightarrow \infty} \frac{a_n}{b_n} = L >0\) אז \(\sum a_n , \sum b_n\) מתכנסים ומתבדרים יחד

\end{proposition}
\begin{proposition}[מבחן המנה באמצעות עוקבים]
אם \(\sum a_{n},\sum b_{n}\) טורים אי שליליים, אזי אם \(\frac{a_{n+1}}{a_{n}}\leq \frac{b_{n+1}}{b_{n}}\) כמעט תמיד, ו-\((b_{n})\) מתכנס אז \((a_{n})\) מתכנס.

\end{proposition}
\begin{proof}
יהי \(n \in \mathbb{N}\) כך שלכל \(n > N\) מתקיים:
$$0 < \frac{a_{n+1}}{a_{n}}<\frac{b_{n+1}}{b_{n}}$$
עבור \(k \in \mathbb{N}\) מתקיים:
$$0< \frac{a_{N+2}}{a_{N+1}}\leq \frac{b_{N+2}}{b_{N+1}}\land \dots \land 0< \frac{a_{N+k}}{a_{N+k-1}}\leq \frac{b_{N+k}}{b_{N+k-1}}$$
כאשר נכפיל את כל האי שיוויונות ונקבל מכפלה שמתבטלת:
$$0< \frac{a_{N+k}}{a_{N+1}}\leq \frac{b_{N+k}}{b_{N+1}}\implies \frac{a_{N+k}}{b_{N+k}}\leq \frac{a_{N+1}}{b_{N+1}}=\mathrm{const}$$
כאשר נסמן את הקבוע ב-\(v\). מתקיים לכל \(n>N\) כי \(\frac{a_{n}}{b_{n}}\leq v\) ולכן \(a_{n}\leq v b_{n}\) ולכן כיוון ש-\(\sum b_{n}\) מתכנס אז מאריתמטיקה של טורים גם \(\sum v b_{n}\) מתכנס וכעת ממבחן ההשוואה גם \(\sum a_{n}\) מתכנס.

\end{proof}
\begin{proposition}[מבחן דלמבר]
יבי \(\sum a_{n}\) טור חיובי כך ש-\(a_{n}\neq 0\) כמעט תמיד. אזי:

  \begin{enumerate}
    \item אם קיים \(0<q<1\) כך ש-\(\frac{a_{n+1}}{a_{n}}<q\) אז \(\sum a_{n}\) מתכנס. 


    \item אם קיים \(q> 1\) כך ש-\(\frac{a_{n+1}}{a_{n}}>q\) אז \(\sum a_{n}\) מתבדר. 


  \end{enumerate}
\end{proposition}
\begin{proof}
נניח \(\frac{a_{n+1}}{a_{n}}<q=\frac{q^{n+1}}{q^{n}}\) כאשר \(0<q<1\). ולכן כיוון ש-\(\sum q^{n}\) מתכנס נקבל ממבחן השורש באמצעות מנות עוקבות כי \(\sum a_{n}\) מתכנס. עבור \(q\geq 1\) נקבל כי הסדרה \(a_{n}\) מונוטונית עולה, ולכן בפרט לא אפסה והטור \(\sum a_{n}\) לא מתכנס.

\end{proof}
\begin{proposition}[מבחן השורש של קושי]
יהי \(\sum a_{n}\) טור אי שלילי.

  \begin{enumerate}
    \item אם קיים \(q \in \mathbb{R}\) עם \(0<q<1\) כך ש-\(\sqrt[n]{ a_{n} }\leq q\) כמעט תמיד אז \(\sum a_{n}\) מתכנס. 


    \item אם \(\sqrt[n]{ a_{n} }\geq 1\) באופן שכיח אז \(\sum a_{n}\) מתבדר 


  \end{enumerate}
\end{proposition}
\begin{proof}
  \begin{enumerate}
    \item קיים \(N \in \mathbb{N}\) כך שלכל \(n > N\) מתקיים \(\sqrt[n]{ a_{n} }\leq q\). לכן \(0\leq a_{n}\leq q^{n}\). היות ו- \(0<q< 1\) הטור \(\sum q^{n}\) מתכנס. ולכן ממבחן ההשוואה \(\sum a_{n}\) מתכנס. 


    \item אם \(\sqrt[n]{ a_{n} }> 1\) באופן שכיח, אזי \(a_{n}\geq 1^{n}=1\) באופן שכיח, ולכן \((a_n)_{n=1}^\infty\) לא מתכנס ל-0, והטור \(\sum a_{n}\) לא מתכנס. 


  \end{enumerate}
\end{proof}
\begin{proposition}[מבחן השורש הגבולי]
יהי \(\sum a_{n}\) טור אי שלילי. אם \(\lim_{ n \to \infty } \sqrt[n]{ a_{n} }=L \in \mathbb{R}\), מתקיים:

  \begin{enumerate}
    \item אם \(0\leq L < 1\) אז \(\sum a_{n}\) מתכנס. 


    \item אם \(1< L\) אז \(\sum a_{n}\) מתבדר. 


    \item אם \(L=1\) אז לא ניתן להכריע. 


  \end{enumerate}
\end{proposition}
\begin{proof}
  \begin{enumerate}
    \item אם \(0\leq L< 1\) אז נגדיר \(L<q<1\). וכעת נבחר \(\varepsilon=q-L> 0\) ובהגדרת הגבול נקבל כי \(\sqrt[n]{ a_{n} }<L+\varepsilon>0\) כמעט תמיד. ולכן ממבחן השורש של קושי נקבל כי \(\sum a_{n}\) מתכנס. 


    \item אם \(1< L\)  אז נגדיר \(1<q<L\). נבחר \(\varepsilon=L-q>0\). מהגדרת הגבול נקבל כי \(q=L-\varepsilon<\sqrt[n]{ a_{n} }\) כמעט תמיד. לכן ממבחן השורש מתבדר. 


    \item נסתכל על \(a_{n}=\frac{1}{n^{2}}\). אנו יודעים כי \(\sum a_{n}\) מתכנס. וכן מתקיים\\
$$\frac{1}{\sqrt[n]{ n^{2} }} \xrightarrow{n\to \infty}1$$
באופן דומה עבור \(b_{n}=\frac{1}{n}\) אנו יודעים כי \(\sum b_{n}\) מתדבר, ומתקיים:
$$\frac{1}{\sqrt[n]{ n }}\xrightarrow{n\to \infty} 1$$


  \end{enumerate}
\end{proof}
\begin{proposition}[מבחן האינטגרל]
אם \(f:[1:\infty)\rightarrow [0:\infty)\)  פונקציה חיובית
הטור \(\sum f(n)\) מתכנס אם"ם האינטגרל \(\int_1^\infty f(x)\) מתכנס, ומתקיים $$\sum_{n=2}^\infty f(x) \leq \int_1^\infty f dx \leq \sum_{n=1}^\infty f(x) $$

\end{proposition}
\chapter{סדרות וטורי פונקציות}

\section{התכנסות במ"ש של טורי פונקציות}

\begin{proposition}[מבחן ה-\(M\) של ווירשטראס]
אם קיים טור מתכנס של מספרים חיוביים \(\sum M_n\) כך שלכל \(n\in \mathbb N\) מתקיים ולכל \(x\) בתחום מתקיים \(|f_n(x)|\leq M_n\) אז \(\sum f_n(x)\) מתכנס בהחלט במ"ש

\end{proposition}
\begin{proof}
יהי \(\varepsilon>0\). מתנאי קושי של התכנסות של טור מספרי קיים \(N \in \mathbb{N}\) כך שלכל \(n > N\) ולכן \(p \in \mathbb{N}\) מתקיים:
$$\sum_{i=n}^{n+p} M_{i}<\varepsilon$$
יהי \(n>N\), \(p \in \mathbb{N}\). מתקיים:
$$\sum_{k=n+1}^{n+p}f_{k}(x)\leq \sum_{k=n+1}^{n+p} M_{k}<\varepsilon$$
ולכן מתכנס במ"ש מתנאי קושי להתכנסות במש של טורי פונקציות.

\end{proof}
\begin{proposition}[שימור רציפות]
אם \((f_{n})\) סדרת פונקציות המתכנסת במ"ש ל-\(f\), ו-\(f_{n}\) רציפה לכל \(n\in \mathbb{N}\) אזי \(f\) רציפה

\end{proposition}
\begin{proof}
יהי \(\varepsilon>0\). כיוון ש-\((f_{n})\) מתכנס במ"ש ל-\(f\) קיים \(N \in \mathbb{N}\) כך שלכל \(x_{0} \in [a,b]\) מתקיים:
$$\forall n > N\quad \lvert f_{n}(x_{0})-f(x_{0}) \rvert <\frac{\varepsilon}{3}$$
יהי \(n_{0}>N\). מתקיים מהגדרת הרציפות:
$$\forall x \in [a,b]\quad \exists \delta> 0\quad \lvert x-x_{0} \rvert <\delta\implies \lvert f_{n_{0}}(x)-f_{n_{0}}(x_{0}) \rvert $$
כעת מתקיים:
\begin{gather*}\lvert f(x)-f(x_{0}) \rvert =\lvert f(x)-f_{n}(x_{0})+f_{n}(x_{0})-f_{n_{0}}(x)+f_{n_{0}}(x)-f(x_{0}) \rvert \leq \\\leq \lvert f(x)-f_{n_{0}}(x) \rvert +\lvert f_{n_{0}}(x)-f_{n_{0}}(x_{0}) \rvert +\lvert f_{n_{0}}(x_{0})-f(x_{0}) \rvert <\frac{3\varepsilon}{3}=\varepsilon 
\end{gather*}
ולכן עובר ה-\(\delta\) הזו מתקיים:
$$\lvert x-x_{0} \rvert <\delta\implies \lvert f(x)-f(x_{0}) \rvert <\varepsilon$$
ולכן \(f\) רציף.

\end{proof}
\begin{proposition}[שימור אינטגרביליות]
אם \((f_{n})\) סדרת פונקציות המתכנסת במ"ש ל-\(f\) בקטע \([a,b]\) אם \(f_{n}\) אינטגרבילית לכל \(n \in \mathbb{N}\) אז גם \(f\) אינטגרבילית.

\end{proposition}
\begin{proof}
מהתכנסות במ"ש נובע כי:
$$\forall\varepsilon>0\quad \exists N \in \mathbb{N} \quad \forall x \in \left[ a,b\right]\quad \forall n> N\quad \lvert f_{n}(x)-f(x) \rvert <\varepsilon$$
יהי \(\varepsilon>0\). עבור \(\frac{\varepsilon}{4(b-a)}\) נקבל \(N \in \mathbb{N}\) כך שמתקיים:
$$\forall x \in [a,b]\quad  \forall n \in N\quad \lvert f_{n}(x)-f(x) \rvert <\frac{\varepsilon}{4(b-a)}$$
כיוון ש-\(f_{n}\) אינטגרבילית מתנאי דרבו לינאגרביליות קיימת חלוקה \(P=\left\{  a=x_{0},\dots,x_{n}=b  \right\}\) כך ש:
$$U(f,P)-L(f,P)<\frac{\varepsilon}{2}\implies\sum_{i=1}^{n} \omega_{i,f}(x_{i}-x_{i-1})<\frac{\varepsilon}{2}$$
נשים לב כי עבור \(u,v \in [x_{i-1,x_{i}}]\) מתקיים:
\begin{gather*}\lvert f(u)-f(v) \rvert =\lvert f(u)-f_{n}(u)+f_{n}(u)-f_{n}(v)+f_{n}(v)-f(v) \rvert \leq \\\leq \lvert f(u)-f_{n}(u) \rvert +\lvert f_{n}(v)-f_{n}(u) \rvert +\lvert f_{n}(v)-f(v) \rvert  \\<\frac{\varepsilon}{4(b-a)}+\omega_{i,f_{n}}+\frac{\varepsilon}{4(b-a)}=\frac{\varepsilon}{2(b-a) }+\omega_{i,f_{n}}
\end{gather*}
ובפרט מתקיים עבור ה-\(u,v\) המינימלים והמקסימלים בכל תחום, ולכן:
$$\omega_{i,f}\leq \lvert f(u)-f(v) \rvert \leq \frac{\varepsilon}{2(b-a)}+\omega_{i,f_{n}}$$
וכעת:
\begin{gather*}U(f,P)-L(f,P)=\sum_{i=1}^{n}\omega_{i,f}(x_{i}-x_{i-1})\leq\sum_{i=1}^{n} \left( \frac{\varepsilon}{2(b-a)}+\omega_{i, f_{n}} \right)(x_{i}-x_{i-1})=  \\=\frac{\varepsilon}{2(b-a)}\sum_{i=1}^{n}(x_{i}-x_{i-1})+\sum_{i=1}^{n} \omega_{i,f_{n}}(x_{i}-x_{i-1})=\frac{\varepsilon}{2(b-a) }(b-a)+\frac{\varepsilon}{2}=\frac{\varepsilon}{2}+\frac{\varepsilon}{2}=\varepsilon 
\end{gather*}
ולכן \(f\) אינטגרבילית.

\end{proof}
\begin{proposition}[אינטגרציה איבר איבר]
אם \(\sum f_n\) מתכנסת במ"ש ל-\(f\). ואז מתקיים:
 $$\int_a^b \left(\sum_{n=1}^\infty f_n\right) dx=\sum_{n=1}^\infty \left(\int_a^b f_n dx\right)=\int_a^b f dx$$

\end{proposition}
\begin{proposition}
טור המתכנס במ"ש משמר חסימות.

\end{proposition}
\begin{proposition}
אם סדרה מתכנסת במ"ש ניתן להחליף גבולות:
$$\operatorname*{lim}_{x\to x_{0}}\left(\operatorname*{lim}_{n\to\infty}f_{n}(x)\right)=\operatorname*{lim}_{n\to\infty}\left(\operatorname*{lim}_{x\to x_{0}}f_{n}(x)\right)$$

\end{proposition}
\begin{proposition}[גזירה איבר איבר]
יהי \(\sum f\) טור פונקציות כך שמתקיים:

  \begin{enumerate}
    \item טור הנגזרות  \(\sum f_n'(x)\) מתכנס במ"ש על \([a,b]\)


    \item הטור \(\sum f_n(x_0)\) מתכנס נקודתית 
אזי מתקיים \(\left( \sum f_{n} \right)'=\sum(f'_{n})\).


  \end{enumerate}
\end{proposition}
\section{טורי חזקות}

\begin{definition}[טור חזקות]
טור מהצורה
$$S_{n}=\sum_{k=1}^{\infty}a_{k}(x-x_{0})^{k}$$

\end{definition}
\begin{definition}[תחום התכנסות]
הקבוצה של ה-\(x\)-ים עבורו הטור מתכנס.

\end{definition}
\begin{proposition}[הלמה של אבל]
אם קיים \(x_{1}\) עבורו הטור \(\sum a_{k}x_{1}^{k}\) מתכנס אז הטור \(\sum a_{k}x^{k}\) מתכנס בהחלט לכל \(x \in \mathbb{R}\) המקיים \(\lvert x \rvert<\lvert x_{1} \rvert\).

\end{proposition}
\begin{proof}
יהי \(x \in \mathbb{R}\) כך ש-\(\lvert x \rvert<\lvert x_{1} \rvert\). התכנסות הטור \(\sum a_{k}x^{k}\) גוררת שהסדרה \((a_{k}x^{k})\) אפסה, ובפרט חסומה. כלומר קיים \(M \in \mathbb{R}\) כך ש-\(\lvert a_{k}x_{1}^{k} \rvert<M\). מתקיים:
$$\lvert a_{k}x^{k} \rvert =\left\lvert  a_{k}x^{k}\cdot \frac{x_{1}^{k}}{x_{1}^{k}}  \right\rvert =\lvert a_{k}x_{1}^{k} \rvert \left\lvert  \frac{x^{k}}{x_{1}^{k}}  \right\rvert =\lvert a_{k}x_{1}^{k} \rvert \left\lvert  \frac{x}{x_{1}} \right\rvert^{k} $$
כיוון ש-\(x<x_{1}\) הטור \(\left\lvert  \frac{x}{x_{1}}  \right\rvert\) מתכנס מכאן \(\sum M\left\lvert  \frac{x}{x_{1}}  \right\rvert^{k}\) מתכנס וממבחן ההשוואה הטור \(\sum a_{k}x^{k}\) מתכנס בהחלט.

\end{proof}
\begin{proposition}
תחום התכנסות של טור חזקות הוא סימטרי

\end{proposition}
\begin{proof}
  \begin{enumerate}
    \item אם תחום ההתכנסות \(D\) אינו חסום אז לכל \(x \in \mathbb{R}\) קיים \(x_{1} \in D\) עבורו \(\lvert x \rvert<\lvert x_{1} \rvert\) ולכן עבור כל \(x \in \mathbb{R}\) מתכנס מהלמה של אבל, ולכן \(D=\mathbb{R}\). 


    \item אם התחום ההתכנסות \(D\) חסום נסמן \(R=\sup(D)\). נחלק למקרים. עבור \(R=0\) נרצה להראות כי \(D=\{ 0 \}\). נניח בשלילה כי \(D\neq \{ 0 \}\). לכן קיים \(0<\lvert x_{1} \rvert\) עבורו \(0<\left\lvert  \frac{x_{1}}{2}  \right\rvert\) ומתכנס מהלמה של אבל, בסתירה לכך ש-\(R=0\), ולכן \(D=\{ 0 \}\). 


    \item אם \(R> 0\) אם \(x< R\) אז קיים איבר \(x_{1} \in D\) כך ש-\(x<x_{1}\) ולכן \(x \in D\) מהלמה של אבל. אם \(\lvert x \rvert> R\) קיים \(\lvert x \rvert>x_{1}>R\) ולכן לו \(x \in D\) אז \(x_{1} \in D\) בסתירה לזה ש-\(x > R\). 


  \end{enumerate}
\end{proof}
\begin{definition}[רדיוס התכנסות]
המרחק מנקודה \(x_0\) עבורה הטור מתכנס. מרחק זה תמיד יהיה סימטרי, ולכן יהיה רק מהשלוש אפשריות הבאות: הנקודה עצמה בלבד, תחום סימטרי או כל הממשיים.

\end{definition}
\begin{reminder}
רדיוס בבתכנסות של טור חזקות בעזרת מבחן דלמבר יהיה:
$$\operatorname*{lim}_{n\to\infty}\left|{\frac{a_{n+1}}{a_{n}}}\right|=L$$
כאשר בעזרת קושי הדמר:
$$L=\sup \sqrt[n]{a_n}$$

\end{reminder}
\begin{proposition}
רדיוס בבתכנסות של טור חזקות בעזרת מבחן דלמבר יהיה:
$$\operatorname*{lim}_{n\to\infty}\left|{\frac{a_{n+1}}{a_{n}}}\right|=L\,,$$
כאשר רדיוס ההתכנסות יהיה \(\frac{1}{L}\) כאשר אם \(L=0\) תחום ההתכנסות יהיה \(R\) ואם \(L=\infty\) אז רדיוס ההתכנסות יהיה 0.

\end{proposition}
\begin{example}
  \begin{enumerate}
    \item עבור הטור \(\sum_{n=0}^{\infty}n!x^{n}\) נקבל: 
$$\operatorname*{lim}_{n\to\infty}\left|{\frac{a_{n+1}}{a_{n}}}\right|=\operatorname*{lim}_{n\to\infty}{\frac{(n+1)!}{n!}}=\operatorname*{lim}_{n\to\infty}(n+1)=\infty.$$
ולכן רדיוס ההתכנסות הוא \(R=0\)


    \item עבור $$\sum_{n=10}^{\infty}\left(-1\right)^{n}{\frac{x^{n}}{n!}}$$ 
נקבל כי:
$$\operatorname*{lim}_{n\to\infty}\left|{\frac{a_{n+1}}{a_{n}}}\right|=\operatorname*{lim}_{n\to\infty}{\frac{n!}{(n+1)!}}=\operatorname*{lim}_{n\to\infty}{\frac{1}{n+1}}=0.$$
ולכן רדיוס ההתכנסות יהיה \(R=\infty\).


    \item עבור $$\sum_{n=0}^{\infty}2^{n}n^{2}(x-1)^{n}.$$ 
נקבל:
$$\operatorname*{lim}_{n\to\infty}\left|{\frac{a_{n+1}}{a_{n}}}\right|=\operatorname*{lim}_{n\to\infty}{\frac{2^{n+1}(n+1)^{2}}{2^{n}n^{2}}}=2\operatorname*{lim}_{n\to\infty}\left(1+{\frac{1}{n}}\right)^{2}=2.$$
ולכן \(R=\frac{1}{2}\).


  \end{enumerate}
\end{example}
\begin{proposition}
טור חזקות מתכנס במ"ש בכל קטע סגור בתחום התכנסות. בפרט ניתן לגזור איבר איבר גזיר איבר איבר כאשר תחום ההתכנסות של הנגזרת עלול לא לכלול קצוות וכן ניתן לבצע אינטגרציה איבר איבר כאשר התחום התכנסות של הטור של האינטגרל עלול כן לכלול קצוות.

\end{proposition}
\begin{definition}[טור טיילור]
תהי \(f\) פונקציה גזירה מכל סדר בנוקדה \(x=x_{0}\). אזי הטור טיילור סביב \(x_{0}\) יהיה:
$$\sum_{n=0}^{\infty}\frac{f^{(n)}(x_{0})}{n!}(x-x_{0})^{n}.$$

\end{definition}
\begin{example}
$$\begin{array}{r c l}{{e^{x}}}&{{=}}&{{\sum_{n=0}^{\infty}\frac{x^{n}}{n!},\quad-\infty<x<\infty,}}\\ {{\sin x}}&{{=}}&{{\sum_{n=0}^{\infty}(-1)^{n}\frac{x^{2n+1}}{(2n+1)!},\quad-\infty<x<\infty,}}\\ {{\cos x}}&{{=}}&{{\sum_{n=0}^{\infty}(-1)^{n}\frac{x^{2n}}{(2n)!},\quad-\infty<x<\infty,}}\\ {{\frac{1}{1-x}}}&{{=}}&{{\sum_{n=0}^{\infty}x^{n},\quad-1<x<1.}}\end{array}$$

\end{example}
\begin{proposition}
טור הוא טור חזקות אם"ם הוא טור טיילור.

\end{proposition}
\begin{definition}
הכיוון שטור טיילור הוא טור חזקות הוא ברור. בכיוון השני נניח כי:
$$f(x)=\sum_{n=0}^{\infty}a_{n}(x-x_{0})^{n}$$
כעת ניתן לקבל את הנגזרות של \(f(x)\) על ידי גזירה של \(f\) (גזיר כי \((x-x_{0})^{n}\) גזיר). כלומר:
\begin{gather*}f^{\prime}(x)=\sum_{n=1}^{\infty}na_{n}(x-x_{0})^{n-1}\\f^{\prime\prime}(x)=\sum_{n=2}^{\infty}n(n-1)a_{n}(x-x_{0})^{n-2}\\ \vdots \\f^{(k)}(x)=\sum_{n=k}^{\infty}n(n-1)\cdots(n-k+1)a_{n}(x-x_{0})^{n-k} 
\end{gather*}
כאשר בפרט לכל טורים האלו יש את אותו רדיוס התכנסות כמו \(f(x)\). כעת ניתן לבצע הזזה של הטור על ידי ההצבה \(n-k=m\) ולקבל:
$$f^{(k)}(x)=\sum_{m=0}^{\infty}(m+k)(m+k-1)\cdots(m+1)a_{m+k}(x-x_{0})^{m}$$
אם נציב \(x=x_{0}\) נקבל כי כל הגורמים מתאפסים פרט הגורם עם החזקה \(m=0\). ונקבל:
$$f^{(k)}(x_{0})=k!a_{k}\implies a_{k}=\frac{f^{(k)}(x_{0})}{k!}$$
כאשר זהו לפי ההגדרה המקדם של טור טיילור.

\end{definition}
\begin{proposition}[יחידות הטור חזקות]
אם:
$$\sum_{n=0}^{\infty}a_{n}(x-x_{0})^{n}=\sum_{n=0}^{\infty}b_{n}(x-x_{0})^{n}$$
לכל \(x\) בסביבה שמכילה את \(x_{0}\), אזי \(a_{n}=b_{n}\) לכל \(n\).

\end{proposition}
\section{סיווג התכנסויות}

\begin{definition}[התכנסות נקודתית]
אם עבור \(x\in D\) מתקיים \(f_n \xrightarrow[n\rightarrow\infty]{} f\) אז \((f_n)\) מתכנסת נקודתית ל-\(f\) בתחום. כלומר:
$$\forall \epsilon > 0 \quad \exists N \in\mathbb{N}\quad \forall x\in A \quad\forall n\in\mathbb{N} \Rightarrow |f_n(x)-f(x)|<\epsilon$$

\end{definition}
\begin{definition}[התכנסות במידה שווה]
סדרת פונקציות \((f_{n})\) מתכנסת במידה שווה אם מתקיים:
$$\forall \epsilon > 0 \quad \exists N \in\mathbb{N}\quad \forall x\in A \quad\forall n\in\mathbb{N} \Rightarrow |f_n(x)-f(x)|<\epsilon$$

\end{definition}
\begin{proposition}[תנאי שקול להתכנסות במ"ש]
נגדיר \(d_\infty(f,g)=\sup_{x\in D}|f(x)-g(x)|\). סדרה מתכנסת במ"ש אם"ם \(d_\infty(f_n,f)\xrightarrow[]{n\rightarrow\infty} 0\).

\end{proposition}
\begin{proposition}[תנאי קושי להתכנסות במידה שווה]
עבור סדרת פונקציות \((f_{n})\) התנאים הבאים שקולים:

  \begin{enumerate}
    \item הסדרה \((f_{n})\) מתכנסת במ"ש. 


    \item לכל \(\varepsilon>0\) קיים \(N\in \mathbb N\) כך שלכל \(N<m,n\in\mathbb N\) מתקיים \(|f_n-f_m|<\varepsilon\). 


    \item לכל \(\varepsilon>0\) קיים \(N\in \mathbb N\) כך שלכל \(N<m,n\in\mathbb N\) מתקיים \(d_A(f_n, f_m)\). 


  \end{enumerate}
\end{proposition}
נראה 1 שקול ל-2.

\begin{proof}
נראה \(1\implies 2\). תהי \((f_{n})\) סדרת פונקציות המתכנסת במ"ש. יהי \(\varepsilon>0\). מהגדרת התכנסות במ"ש, קיים \(N \in \mathbb{N}\) כך שלכל \(n > N\) מתקיים \(\lvert f_{n}(x)-f(x) \rvert<\frac{\varepsilon}{2}\). מכאן עבור כל \(m,n>N\) מתקיים:
$$\lvert f_{n}(x)-f_{m}(x) \rvert \leq \lvert f_{n}(x)-f(x) \rvert +\lvert f(x)-f_{m}(x) \rvert <\frac{\varepsilon}{2}+\frac{\varepsilon}{2}$$
כעת נראה \(2\implies 1\). תהי \((f_{n})\) סדרת פונקציות המקיימת את תנאי קושי.  עבור כל \(x_{0}\) מקיימת את תנאי קושי של סדרת מספרים ולכן מתכנסת נקודתית. יהי \(\varepsilon>0\). מההנחה קיים \(N \in \mathbb{N}\) וקיימים \(n,m \in \mathbb{N}\) כך שלכל \(n,m> N\) מתקיים:
$$\lvert f_{n}(x)-f_{m}(x) \rvert <\frac{\varepsilon}{2}$$
נקבע את \(m\). עבור כל \(n>N\) מתקיים:
$$f_{m}(x)-\frac{\varepsilon}{2}<f_{n}(x)<f_{m}(x)+\frac{\varepsilon}{2}$$
וכאשר ניקח את הגבול נקבל:
$$f_{m}(x)-\frac{\varepsilon}{2}<f_{n}(x)<f_{m}(x)+\frac{\varepsilon}{2}\implies \lvert f(x)-f_{m}(x) \rvert <\frac{\varepsilon}{2}$$
כעת עבור כל \(n>N\) וכל \(x \in \mathrm{Dom}(f)\) מתקיים:
$$\lvert f(x)-f_{n}(x) \rvert \leq \lvert f(x)-f_{m}(x) \rvert +\lvert f_{m}(x)-f_{n}(x) \rvert \leq \frac{\varepsilon}{2}+\frac{\varepsilon}{2}=\varepsilon$$
ומתכנס במידה שווה.

\end{proof}
\begin{proposition}
אם מתכנס במידה שווה אז מתכנס נקודתית

\end{proposition}
\begin{proposition}
סדרת פונקציות מתכנסת במידה שווה בקטע סגור אם"ם מתכנס במידה שווה בקטע הפתוח.

\end{proposition}
\end{document}