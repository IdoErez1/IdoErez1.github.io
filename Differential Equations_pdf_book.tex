\documentclass{tstextbook}

\usepackage{amsmath}
\usepackage{amssymb}
\usepackage{graphicx}
\usepackage{hyperref}
\usepackage{xcolor}

\begin{document}

\title{Example Document}
\author{HTML2LaTeX Converter}
\maketitle

\chapter{משוואות דיפרנציאליות רגילות לינאריות}

\section{מושגים כללים}

\begin{definition}[משוואה דיפרנציאלית רגילה]
ביטוי מהצורה:
$$F\left( x,y,\frac{\mathrm{d} y}{\mathrm{d} x} ,\frac{\mathrm{d} ^2y}{\mathrm{d} x^{2}},\dots  \right)=0$$
אנחנו פותרים אותה עבור קבוצת פונקציות כאשר אנחנו מתייחסים רק לפונקציות גזירות. כאשר נשים לב כי \(F\) הוא פונקציה אשר מקבל פונקציה ומחזיר פונקציה, ולכן השיוויון הוא עם הפונקציה הקבועה 0.

\end{definition}
\begin{definition}[אופרטור המייצג משוואה]
אופרטור המקיים \(\mathcal{L}y=F\left( x,y,\frac{\mathrm{d} y}{\mathrm{d} x} ,\frac{\mathrm{d} ^2y}{\mathrm{d} x^{2}},\dots  \right)\). בעזרת האופרטור הזה משוואה תראה מהצורה הבאה:
$$\mathcal{L}y=0$$

\end{definition}
\begin{reminder}[אופרטור לינארי]
אופרטור \(\mathcal{L}:V\to V\) נקרא לינארי אם לכל \(v,u \in V\) וסקלר \(c \in \mathbb{F}\) מתקיים:

  \begin{enumerate}
    \item משמר חיבור - \(\mathcal{L}(u+v)=\mathcal{L}u+\mathcal{L}v\). 


    \item משמר כפל בסקלאר - \(\mathcal{L}(cu)=c\mathcal{L}u\). 
או לחלופין מספיק שמתקיים \(\mathcal{L}(cv+u)=c\mathcal{L}(v)+\mathcal{L}(u)\).


  \end{enumerate}
\end{reminder}
\begin{reminder}
מרחב הפונקציות הרציפות עם פעולת החיבור והכפל הידועים יוצר מרחב ווקטורי. וכן פעולת הגזירה והאינטגרל הם אופרטורים לינאריים.

\end{reminder}
\begin{definition}[משוואה דיפרנציאלית לינארית]
זוהי משוואה שהאופרטור שמייצג אותה הוא אופרטור לינארי. כלומר לכל שתי פונקציות \(f,g\) מתקיים:
$$\mathcal{L}(cf+g)=c\mathcal{L}(f)+\mathcal{L}(g)$$

\end{definition}
\begin{proposition}
פתרון של משוואה דיפרנציאלית לינארית יוצר תת מרחב ווקטורי של מרחב הפונקציות.

\end{proposition}
\begin{proof}
ראינו כי גרעין של אופרטור הוא תת מרחב ווקטור.

\end{proof}
\begin{corollary}
אם \(f,g\) שניהם פתרונות של משוואה דיפרנציאלית ו-\(a,b \in \mathbb{R}\) אז גם הסכום שלהם \(af+bg\) יהיה פתרון של המשוואה.

\end{corollary}
\begin{proposition}[הפיכת פתרונות מדומים לממשיים]
אם פונקציה \(f \in \mathbb{C}\) פתרון של משוואה עם מקדמים ממשיים אזי גם הצמוד \(\overline{f}\in \mathbb{C}\) יהיה פתרון. ולכן בפרט \(\frac{f+\overline{f}}{2}=\mathrm{Re}(f)\) יהיה פתרון וגם \(\frac{f-\overline{f}}{2}=\mathrm{Im}(f)\) יהיה פתרון.

\end{proposition}
\begin{remark}
נראה כי מה שאמרנו עד עכשיו יהיה נכון גם עבור משוואות דיפרנציאליות חלקיות.

\end{remark}
\begin{proposition}[צורה של משוואה דיפרנציאלית רגילה לינארית]
$$a_{0}(x)y+a_{1}(x)y^{\prime}+a_{2}(x)y^{\prime\prime}\cdots+a_{n}(x)y^{(n)}=b(x)$$

\end{proposition}
\begin{definition}[סדר של משוואה דיפרנציאלית]
מספר המתאר את הכמות הגבוהה ביותר של פעמים שגוזרים איבר כלשהו, כלומר הנגזרת הכי גבוהה שמופיעה במשוואה

\end{definition}
\begin{definition}[מעלה של משוואה דיפרנציאלית]
החזקה של האיבר שנגזר הכי הרבה פעמים.

\end{definition}
\begin{definition}[בעיית תנאי התחלה]
כאשר נתון לנו את הנגזרת \(y'=f(x,y)\) עם מצב בהתחלה \(y(x_{0})=y_{0}\).

\end{definition}
הגיוני שלבעיית תנאי התחלה תמיד יהיה פתרון יחיד - הרי אפשר לדמות את זה למהירות של חלקיק כאשר ידוע המיקום שלו בהתחלה - הוא מתחיל איפשהו ואנחנו יכולים לדעת לאן יתקדם.

\begin{theorem}[קיום ויחידות - סדר ראשון]
אם יש לנו פונקציה מהצורה \(y'=f(x,y)\) ונתון תנאי התחלה \(y(x_0)=y_0\) אז:
- \textbf{משפט קיום(Peano):}
אם \(f(x,y)\) רציפה בתחום אז קיים פתרון בתת תחום.
- \textbf{משפט קיום ויחידות(Picard):}
אם \(f(x,y)\) רציף, ו-\(\frac{\partial f}{\partial y}\) רציפה או ליפשיצית אז קיים פתרון יחיד בתת תחום.

\end{theorem}
\begin{reminder}[פונקציה לפישיצית]
פונקציה \(f:D\to \mathbb{R}\) עבורה קיים קבוע \(M\) כך שלכל \(x_{1},x_{2} \in D\) מתקיים:
$$\lvert f(x_{1})-f(x_{2}) \rvert \leq M|x_{1}-x_{2}|$$
כאשר \(M\) נקרא הקבוע ליפשיץ.

\end{reminder}
\begin{proposition}
ניתן להמיר משוואה מסדר גבוה מ-1 למערכת של משוואות מסדר ראשון.

\end{proposition}
\begin{proof}
נגדיר:
$${{Y_{1}=y}}\quad  {{Y_{2}=y^{\prime}}}\quad  {{Y_{3}=y^{\prime\prime}}}\quad  {{\dots}}\quad  {{Y_{n}=y^{(n-1)}}}$$
ועבור המשוואה האחרונה ניקח את המשוואה הדיפרנציאלית ונציב את \(Y_{1},\dots,Y_{n}\) כדי לקבל משוואה דיפרנציאלית מסדר ראשון עם הנגזרת \(y^{(n)}=Y_{n}'\).

\end{proof}
\begin{example}
נניח \(y'' + 2y' + 3y=0\). נסמן את הנגזרות בתור משתנים חדשים:
$$\begin{cases}x_1(t) = y(t) \\ x_2(t)=y'(t)\end{cases}$$
כעת מתקיים:
$$\begin{cases}x_2'(t) = 2x_2 + 3 x_1 \\ x_1'(t) = x_2(t)
\end{cases}$$

\end{example}
\begin{theorem}[קיום ויחידות - סדר גבוה]
\textbf{עבור משוואה לינארית:} תהיה מהצורה
$$P_n(x) y^{(n)} +...+P_1(x)y'+P_0 (x) =0$$
מספיק לבדוק רציפות בתחום של כל אחד מהמקדמים \(P_n(x),...,P_1(x),P_0(x)\).
 \textbf{עבור משוואה כללית:} תהיה מהצורה
$$f_n(x,y) y^{(n)} +...+f_1(x,y)y'+f_0(x,y) =0$$
ניתן להמיר במערכת משוואות מסדר ראשון (עם איברים \(y_0, ..., y_n\)) ולבדוק שהנגזרת החלקית של המשוואה עם 
כל אחד מהאיברים \(y_1,...,y_n\) היא רציפה בתחום.

\end{theorem}
\section{משוואת מסדר ראשון}

\begin{reminder}[משפט החלפת משתנה חד מימדי]
מתקיים:
$$\int f(g(x))\cdot g^{\prime}(x)\,d x=\int f(u)\,d u$$
כאשר \(u\) הוא משתנה חדש שלפניו ניתן לבצע אינטגרציה.

\end{reminder}
\begin{proposition}
פתרון משוואה דיפרנציאלית מהצורה:
$$y'=f(x)$$
תהיה:
$$y=\int f(x) \;\mathrm{d}x$$

\end{proposition}
\begin{proof}
נתחיל מהמשוואה ונבצע אינטגרל על שתי האגפים:
$$\int y'\; \mathrm{d}x=\int f(x)\;\mathrm{d}x$$
כאשר נשים לב כי אגף שמאל ממשפט החלפת משתנה תהיה \(\int dy\) ולכן:
$$y+C_{1}=\int f(x)\;\mathrm{d}x\implies y = \int f(x) \;\mathrm{d} x $$

\end{proof}
\begin{proposition}[פתרון בעזרת הפרדת משתנים]
אם נתון משוואה מהצורה:
$$y' = g(x)h(y)$$
כאשר \(h(y)\neq 0\) נקבל כי הפתרון מקיים:
$$\frac{y'}{h(y)}=g(x)\implies \int \frac{1}{h(y)} \;\mathrm{d} y =\int g(x)\;\mathrm{d}x$$
כאשר ייתכן כי יהיה בצורה סתומה.

\end{proposition}
\begin{proof}
ניתן להפעיל את האינטגרל לפי \(x\) על שתי האגפים ולקבל:
$$\frac{y'}{h(y)}=g(x)\implies \int \frac{y'}{h(y)} \;\mathrm{d} x =\int g(x) \;\mathrm{d} x $$
כאשר נקבל כי ממשפט החלפת משתנה כי:
$$\int \frac{1}{h(y)} \;\mathrm{d} y =\int g(x)\;\mathrm{d}x$$

\end{proof}
\begin{remark}
קיים טריק שפיזיקאיים משתמשים הרבה כדי לזכור את הביטוי הזה. נכתוב את הנגזרת בתור \(y'=\frac{\mathrm{d} y}{\mathrm{d} x}\) ונקבל:
$$\frac{\mathrm{d} y}{\mathrm{d} x} =g(x)\cdot h(y)$$
כאשר נתייחס לנגזרת כמו שבר, ונבודד כל משתנה:
$$\frac{1}{h(y)}\mathrm{d}y=g(x)\mathrm{d}x$$
כעת ניתן "לבצע אינטגרל" על כל אגף ולקבל:
$$\int \frac{1}{h(y)} \;\mathrm{d} y =\int  g(x)\;\mathrm{d} x $$
וקיבלנו את אותו הביטוי.

\end{remark}
\begin{proposition}[משוואה הומוגנית]
עבור משוואה מהצורה:
$$\frac{\mathrm{d} y}{\mathrm{d} x} =f\left( \frac{y}{x} \right)$$
נקבל כי הפתרון יהיה:
$$y=v x\implies\ln(x)=\int{\frac{d v}{f(v)-v}}+c$$
כאשר כעת ניתן לפתור עבור \(v\) ולהציב חזרה את \(y=\frac{v}{x}\).

\end{proposition}
\begin{proof}
נציב \(y=vx\) כאשר \(v=v(x)\) ונקבל:
$$\frac{\mathrm{d} }{\mathrm{d} x} (vx)=f\left( \frac{vx}{x} \right)=f(v)$$
כאשר ניתן לגזור את \(vx\) לפי נגזרת מכפלה:
$$\frac{\mathrm{d} v}{\mathrm{d} x} x+\frac{\mathrm{d} x}{\mathrm{d} x} v=f(v)\implies x\frac{\mathrm{d} v}{\mathrm{d} x}+v=f(v)\implies \frac{\mathrm{d} v}{\mathrm{d} x} =\frac{f(v)-v}{x} $$
כאשר נשים לב כי זו משוואה פרידה, כאשר:
$$h(v)=f(v)-v\qquad g(x)=\frac{1}{x}$$
ולכן מהטענה הקודמת הפתרון יהיה:
$$\int \frac{1}{h(v)} \;\mathrm{d} v=\int g(x)\;\mathrm{d}x \iff \int \frac{1}{f(v)-v} \;\mathrm{d} v = \int \frac{1}{x} \;\mathrm{d} x $$
כאשר אנחנו יודעים כי האינטגרל של האגף הימיני יהיה \(\ln(x)\) ולכן נקבל:
$$\ln(x)=\int{\frac{d v}{f(v)-v}} +C$$

\end{proof}
כלל השרשרת של פונקציה סקלארית רב מימדית
אם \(x=x(t),y=y(y)\) ו-\(z=z(x,y)\) אז הנגזרת של \(z\) לפי \(t\) תהיה:
$$\frac{\mathrm{d} z}{\mathrm{d} t} =\frac{\partial z}{\partial x} \frac{\mathrm{d} x}{\mathrm{d} t} +\frac{\partial z}{\partial y} \frac{\mathrm{d} y}{\mathrm{d} t} $$
כאשר אם למשל \(y=y(x)\) במקום כאשר כעת \(z=z(x,y(x))\)נקבל כי:
$$\frac{\mathrm{d} z}{\mathrm{d} x} =\frac{\partial z}{\partial x} +\frac{\partial z}{\partial y} \frac{\mathrm{d} y}{\mathrm{d} x} $$

\begin{definition}[משוואה מדוייקת]
משוואה מהצורה:
$$M\left(x,y\right)+N\left(x,y\right){\frac{d y}{d x}}=0$$
כאשר פונקציות \(M,N\) מקיימות:
$$\frac{\partial M(x,y)}{\partial y} =\frac{\partial N(x,y)}{\partial x} $$

\end{definition}
\begin{proposition}[פתרון משוואה מדוייקת]
אם \(M\left(x,y\right)+N\left(x,y\right){\frac{d y}{d x}}=0\) משוואה מדוייקת, אז ניתן להגדיר פונקציה \(\psi=\psi(x,y(x))\) כך שמתקיים:
$$\frac{\partial \psi}{\partial x} =M(x,y)\qquad \frac{\partial \psi}{\partial y} =N(x,y)$$
וכעת פתרון המשוואה יהיה:
$$\psi(x,y)=c$$

\end{proposition}
\begin{proof}
בעזרת ההגדרה של \(\psi\) נקבל כי המשוואה הדיפרנציאלית תהיה:
$$\frac{\partial \psi}{\partial x} +\frac{\partial \psi}{\partial y} \frac{dy}{dx}=0$$
כאשר נשים לב כי לפי כלל השרשרת זה יהיה שווה:
$$\frac{\mathrm{d} \psi}{\mathrm{d} x} =0$$
כאשר כיוון שזוהי פונקציה של \(x\) בלבד נקבל כי הפתרון הכולל יהיה:
$$\psi(x,y)=c$$

\end{proof}
\begin{proposition}
ניתן להפוך משוואה מהצורה:
$$M\left(x,y\right)+N\left(x,y\right){\frac{d y}{d x}}=0$$
למשוואה מדוייקת על ידי כפל בפונקצייה \(\mu=\mu(x,y)\) ולקבל את המשוואה:
$$M\left(x,y\right)\mu(x,y)+N\left(x,y\right)\mu(x,y){\frac{d y}{d x}}=0$$
כך שמתקיים:
$$\frac{\partial(M\mu)}{\partial y}\,=\,\frac{\partial(N\mu)}{\partial x}$$
כאשר לפונקציה \(\mu\) קוראים גורם אינטגרציה.

\end{proposition}
\begin{remark}
אומנם תמיד קיים פונקציה \(\mu\) כזו אך לא תמיד כזה פשוט למצוא אותה.

\end{remark}
\begin{proposition}
עבור משוואה מהצורה:
$$M\left(x,y\right)+N\left(x,y\right){\frac{d y}{d x}}=0$$

  \begin{enumerate}
    \item אם מקיימת: 
$${\frac{1}{N}}\left({\frac{\partial M}{\partial y}}\,-\,{\frac{\partial N}{\partial x}}\right)=f(x)$$
כלומר הביטוי באגף שמאל יהיה פונקציה של \(x\) בלבד, אז גורם האינטגרציה יהיה:
$$\mu=e^{\int f(x)\;\mathrm{d}x}$$


    \item אם מקיימת: 
$$-{\frac{1}{M}}\left({\frac{\partial M}{\partial y}}\,-\,{\frac{\partial N}{\partial x}}\right)=f(y)$$
כלומר הבטוי האגף שמאל יהיה פונקציה של \(y\) בלבד, אז גורם האינטגרציה יהיה:
$$\mu=e^{ \int f(y)\;\mathrm{d}y }$$


  \end{enumerate}
\end{proposition}
\begin{proposition}[גורם אינטגרציה של משוואה לינארית]
עבור משוואה לינארית מהצורה:
$$y^{\prime}+P(x)y=Q(x)$$
הגורם אינטגרציה יהיה \(\mu=e^{ \int P\;\mathrm{dx} }\) ולכן הפתרון יהיה:
$$y e^{\int P d x}=\int Q e^{\int P d x}d x+c$$

\end{proposition}
\begin{proposition}[טכניקות של הצבות]
  \begin{enumerate}
    \item אם מהצורה: 
$$\frac{\mathrm{d} y}{\mathrm{d} x} ={\frac{a x+b y+c}{a^{\prime}x+b^{\prime}y+c^{\prime}}}$$
אז ניתן להציב \(y=v+y_{0}\) ו-\(x=u+x_{0}\) ולקבל משוואה הומוגנית.


    \item אם מהצורה: 
$$\frac{\mathrm{d} y}{\mathrm{d} x}=f(ax+b)$$
אז ניתן להציב 
$$z=y+ax\implies \frac{\mathrm{d} z}{\mathrm{d} x} =\frac{\partial z}{\partial y} \frac{\mathrm{d} y}{\mathrm{d}x }+a=f(z)+a $$
ולקבל משוואה פרידה.


    \item אם נתון משוואה מהצורה: 
$$\frac{\mathrm{d} x}{\mathrm{d} y} =f(x,y)$$
אז ניתן להשתמש בנוסחא של נגזרת הופכית ולקבל:
$$\frac{\mathrm{d} y}{\mathrm{d} x} = \frac{1}{f(x,y)}$$


  \end{enumerate}
\end{proposition}
\begin{definition}[משוואות אוטונומיות]
משוואה דיפרנציאלית מהצורה:
$$y'=f(y)$$

\end{definition}
\begin{proposition}[תכונות של משוואה אוטונומית]
  \begin{itemize}
    \item אם \(g(y_k)=0\) אז \(y(t)\equiv y_k\) פתרון.
    \item אם \(y(t)\) פתרון פרטי אז \(y(t+k)\) פתרון פרטי.
    \item מתקיים \(g(a)=0\) אםם \(y(t)\equiv 0\) פתרון פרטי.
  \end{itemize}
\end{proposition}
\section{מערכת של משוואות לינאריות מסדר ראשון}

הרעיון זה להפוך את המערכת משוואות אשר תלויות אחת בשנייה לבלתי תלויות, כלומר אוסף של \(n\) משוואות מסדר ראשון כאשר כל משוואה תהיה עם משתנה משלה. ניתן לעשות זאת במספר דרכים:

\begin{enumerate}
  \item מניפולציה אלגברית ישירה. 


  \item המרה למטריצה ודירוג/מציאת הופכי. 


  \item ליכסון המטריצה בעזרת ערכים עצמיים. המטריצה לא תמיד תהיה לכסינה אך תמיד ניתן לפרק למטריצות בלוקים, ולפתור באחת בדרכים האחרות עבור הבלוק הקטן יותר. 


\end{enumerate}
\begin{proposition}[פתרון בעזרת לכסון]
כאשר האופרטור לכסין נפעל לפי השלבים הבאים:

  \begin{enumerate}
    \item ממירים את המערכת למטריצה 


    \item מוצאים פולינום אופייני, מרחבים עצמיים. 


    \item כותבים את המטריצה תחת הבסיס המלוכסן. 


    \item מקבלים משוואה חד מימדית עבור כל משתנה, פותרים אותה. 


    \item ווקטור הפתרון יהיה הווקטור של כל המשתנים.  


  \end{enumerate}
\end{proposition}
\begin{definition}[אקספוננט מטריצות]
יהי \(A\in M_{n,n}\left( \mathbb{F}  \right)\). אזי נגדיר את האקספוננט \(e^{At}\) בצורה הבאה:
$$e^{ At }=\sum_{n=0}^{\infty}\frac{1}{n!}(At)^{n}= I+At+ \frac{1}{2}(At)^{2}+\dots$$

\end{definition}
\begin{proposition}
הנגזרת של אקספוננט מטריצה תהיה נגזרת איבר איבר:
$$\frac{\mathrm{d} }{\mathrm{d} t} (e^{ At })=A+A^{2}t+ \frac{1}{2}A^{3}t\dots=A\left( I+At+\frac{1}{2}A^{2}t+\dots \right)=Ae^{ At }$$

\end{proposition}
\begin{proposition}
נניח כי יש לנו \(A\) לכסין. אזי קיים \(A=P D P ^{-1}\) כך ש-\(D\) אלכסונית. נקבל:
$$e^{ At }=I+PDP ^{-1} + \frac{1}{2} P D^{2} P ^{-1}+\dots=Pe^{ Dt }P ^{-1}=P \begin{pmatrix}e ^{ \lambda_{1}t} & 0 & 0 \\0 & \ddots & 0 \\0 & 0 & e^{ \lambda_{n} t}
\end{pmatrix}$$

\end{proposition}
\begin{proposition}
עבור משוואה דיפרנציאלית מטריציונית(כלומר מערכת משוואות) מהצורה:
$$y'=Ay$$
נקבל כי הפתרון יהיה:
$$y=e^{At}$$

\end{proposition}
\section{משוואות מסדר גבוה}

\begin{definition}[קבוצה פורשת של מרחב הפתרונות]
קבוצה של פונקציות אשר פורשת את מרחב הפתרונות. כלומר המרחב הפתרונות יהיה הפרוש של כל הפונקציות בקבוצה. כלומר זוהי קבוצה \(A\) כך שאם פונקציה \(y\) היא פתרון אז \(y \in \mathrm{Span}(A)\).

\end{definition}
\begin{definition}[בסיס של מרחב הפתרונות]
הקבוצה הפורשת המינימלית. נזכור כי זו תהיה קבוצה בלתי תלייה לינארית.

\end{definition}
\begin{definition}[וורונסקיאן]
$$W(f_{1},f_{2},\ldots,f_{n})=\left|\begin{array}{cccc}f_{1}&f_{2}&\cdot&\cdot&\cdot&f_{n}\\ f_{1}^{\prime}&f_{2}^{\prime}&\cdot&\cdot&\cdot&f_{n}^{\prime}\\ \cdot&\cdot&\cdot&\cdot&\cdot\\ \cdot&\cdot&\cdot&\cdot&\cdot\\ f_{1}^{(n-1)}&f_{2}^{(n-1)}&\cdot&\cdot&\cdot&f_{n}^{(n-1)}\end{array}\right|$$

\end{definition}
\begin{definition}[משוואה לינארית הומוגנית עם מקדמים קבועים]
זוהי משוואה דיפרנציאלית מהצורה:
$$a_n y^{(n)} + ...+ a_1 y' + a_0 y = 0$$

\end{definition}
\begin{proposition}
ניתן למצוא את הבסיס של מרחב הפתרונות של משוואה מהצורה \(a_n y^{(n)} + ...+ a_1 y' + a_0 y = 0\) בצורה הבאה:

  \begin{enumerate}
    \item נפתור את המשוואה: 
$$a_n \lambda^n + ... + a_1\lambda + a_0=0$$


    \item לכל \(\lambda \in \mathbb{C}\) פתרון, \(e^{\lambda x}\) נמצא בקבוצה הפורשת.  


    \item אם קיים ל-\(\lambda\) ריבוי אלגברי \(k\) אזי גם \(xe^{ \lambda x },\dots,x^{k}e^{ \lambda x }\) יהיו פתרונות. 


  \end{enumerate}
\end{proposition}
\begin{proof}
כיוון שאנו יודעים לפתור את המשוואה מסדר ראשון נמיר את המשוואה למערכת משוואות מסדר ראשון. נגדיר:
$${{Y_{1}=y}}\quad  {{Y_{2}=y^{\prime}}}\quad  {{Y_{3}=y^{\prime\prime}}}\quad  {{\dots}}\quad  {{Y_{n}=y^{(n-1)}}}$$
כעת נבטא את המשתנים בעזרת הנגזרת:
$${{Y_{1}^{\prime}=Y_{2}}}\quad  {{Y_{2}^{\prime}=Y_{3}}}\quad  \dots\quad  {{Y_{n-1}^{\prime}=Y_{n}}}$$
כרגע יש לנו \(n-1\) משוואות. ניתן לקבל את המשוואה האחרונה - ביטוי עבור \(Y_{n}\) בעזרת המשוואה הדיפרנציאלית:
$$Y_{n}^{\prime}=y^{(n)}=-{\frac{a_{n-1}}{a_{n}}}y^{(n-1)}-\cdots-{\frac{a_{0}}{a_{n}}}y=-{\frac{a_{n-1}}{a_{n}}}Y_{n}-\cdots-{\frac{a_{0}}{a_{n}}}Y_{1}$$
כעת ניתן לכתוב את המערכת משוואות בעזרת מטריצה בבסיס הציקלי תחת אופרטור הגזירה. כלומר נגדיר \(e_{i}=Y_{i}\) בסיס כאשר \(1\leq i\leq n\). נקבל:
$$A=\begin{pmatrix}0&0&0&\dots&-{\frac{a_{0}}{a_{n}}}\\ 1&0&0&\dots&-{\frac{a_{1}}{a_{n}}}\\ 0&1&0&\dots&-{\frac{a_{2}}{a_{n}}}\\ \vdots&\vdots&\vdots&\ddots&\vdots\\ \vdots&\vdots&\vdots&\ddots&\vdots\\ 0&0&0&\dots&-{\frac{a_{n-1}}{a_{n}}}\end{pmatrix}$$
כאשר המשוואה שלנו תהיה \(A\vec{Y}=Y'\). כיוון שאנחנו תחת המרוכבים, קיימת צורת ג'ורד \(J\):
$$J=\begin{pmatrix}J_{1}&0&\cdot\cdot\cdot&0\\ 0&J_{2}&\cdot\cdot\cdot&0\\ \vdots&\vdots&\cdot\cdot&\vdots\\ 0&0&\cdot\cdot\cdot&J_{k}\end{pmatrix}\qquad J_{\lambda}=\begin{pmatrix}\lambda&0&0&\cdot\cdot\cdot&0\\ 1&\lambda&0&\cdot\cdot\cdot&0\\ 0&1&\lambda&\cdot\cdot\cdot&0\\ \vdots&\vdots&\vdots&\cdot\cdot\cdot&0\\ \vdots&\vdots&\vdots&\ddots&0\\ 0&0&0&1&\lambda\end{pmatrix}$$
כאשר \(J_{\lambda}\) זה הבלוק ג'ורדן המתאים לערך עצמי \(\lambda\). ולכן המשוואה שקולה למשוואה \(J\vec{x}=0\). מספיק לפתור עבור כל בלוק.
נקבל:

  \begin{enumerate}
    \item כאשר הבלוק מגדול אחד המשוואה תהיה \(y'=\lambda y\) ונקבל \(y=e^{ \lambda x }\). 


    \item כאשר יש בלוק בגודל גדול \(k>1\) נקבל מערכת משוואות פשוטה שנותנת: 
$$y_{1}=e^{\lambda_{1}x}\quad y_{2}=xe^{ \lambda x }\quad \dots \quad y_{k}=x^{k}e^{\lambda x}$$


  \end{enumerate}
\end{proof}
\begin{definition}[משוואת אויילר]
משוואה דיפרנציאלית מהצורה:
$$a_n x^n y^{(n)}+...+a_1 x y' + a_0 y = 0$$

\end{definition}
\begin{proposition}
ניתן למצוא את הבסיס של מרחב הפתרונות של משוואת אויילר מהצורה \(a_n x^n y^{(n)}+...+a_1 x y' + a_0 y = 0\) בצורה הבאה:

  \begin{enumerate}
    \item נפתור את המשוואה: 
$$a_n \lambda^n + ... + a_1\lambda + a_0$$


    \item לכל \(\lambda \in \mathbb{C}\) פתרון, נקבל כי \(x^{\lambda}\) נמצא בבסיס של מרחב הפתרונות. 


    \item אם יש ל-\(\left( x-\lambda \right)\) ריבוי אלגברי \(k\), אז גם \(\ln(x)x^{\lambda},\dots,\ln(x)^{k}x^{\lambda}\) איברים בבסיס של מרחב הפתרון. 


  \end{enumerate}
\end{proposition}
ההוכחה דומה להוכחה עם מקדמים קבועים.

\begin{example}[מערכת של משוואות מסדר גבוה]
נסתכל על המשוואה:
$$My''+Ky=0$$
כאשר \(M,K\) מטריצות(ניתן לחשוב על זה כמערכת של קפיצים מצומדים כאשר \(M\) זה המסות ו-\(K\) זה קבוע קפיץ).

\end{example}
\chapter{פתרון באמצעות טור ושטרום ליוביל}

\section{טורי חזקות ופונקציות אנליטיות}

\begin{definition}[טור חזקה]
טור של פונקציות מהצורה:
$$\sum_{n=0}^{\infty} a_{n}(x-x_{0})^{n}$$
כאשר \(x_{0},a_{n}\in \mathbb{R}\).

\end{definition}
\begin{example}
  \begin{enumerate}
    \item הטור של האקספוננט: 
$$\lim_{ N \to \infty } \sum_{n=0}^{N} \frac{1}{n!}x^{n}=e^{ x }$$
מתכנס לכל \(x \in \mathbb{R}\).


    \item טור ההנדסי: 
$$\lim_{ N \to \infty } \sum_{n=0}^{N} (-1)^{n}x^{n}=\frac{1}{1+x}$$
מתכנס לכל \(x \in (-1,1)\).


  \end{enumerate}
\end{example}
\begin{definition}[רדיוס התכנסות]
עבור טור חזקות הרדיוס ההתכנסות יהיה \(0\leq R\leq \infty\) כך שלכל \(\lvert x-x_{0} \rvert< R\) הטור מתכנס ולכל \(|x-x_{0}|> R\) הטור מתבדר.

\end{definition}
\begin{proposition}
לכל טור חזקות קיים רדיוס התכנסות יחיד. 

\end{proposition}
\begin{remark}
נשים לב כי רדיוס התכנסות לא אומר כלום על השפה של התחום.

\end{remark}
\begin{proposition}
  \begin{enumerate}
    \item כפל בסקלאר לא משנה רדיוס התכנסות. 


    \item הרדיוס התכנסות של שסכום של שתי טורים עם רדיוס התכנסות \(R_{1},R_{2}\) יקיים \(R\geq \min(R_{1},R_{2})\). וערך ההתכנסות יהיה שווה לסכום הערכים 


    \item הרדיוס ההתכנסות של מכפלה של טורי חזקות יהיה גדול או שווה לרדיוס המינימלי וערך ההתכנסות מתקבל בעזרת קונבולוציה בדידה/מכפלת קושי: 
$$\left(\sum_{n=0}^{\infty}a_{n}\left(x-x_{0}\right)^{n}\right)\left(\sum_{n=0}^{\infty}b_{n}\left(x-x_{0}\right)^{n}\right)=\sum_{n=0}^{\infty}\left(\sum_{k=0}^{n}a_{k}b_{n-k}\right)(x-x_{0})^{n}$$


  \end{enumerate}
\end{proposition}
\begin{proposition}
ניתן לגזור כל טור חזקות איבר איבר, כלומר:
$${\frac{d}{d x}}\left(\sum_{n=0}^{\infty}a_{n}\left(x-x_{0}\right)^{n}\right)=\sum_{n=0}^{\infty}{\frac{d}{d x}}a_{n}\left(x-x_{0}\right)^{n}=\sum_{n=0}^{\infty}n a_{n}\left(x-x_{0}\right)^{n-1}$$

\end{proposition}
\begin{definition}[טור טיילור]
יהי \(f:\mathbb{R}\to \mathbb{R}\) פונקציה גזירה אינסוף פעמים. אזי הטור טיילור יהיה:
$$\sum_{n=0}^{\infty}{\frac{f^{(n)}\left(x_{0}\right)}{n!}}\left(x-x_{0}\right)^{n}$$

\end{definition}
\begin{proposition}
טור טיילור הוא בפרט טור חזקות עבור:
$$a_{n}={\frac{f^{(n)}\left(x_{0}\right)}{n!}}$$
מכאן נובע יחידות הטור טיילור.

\end{proposition}
\begin{definition}[פונקציה אנליטית]
פונקציה \(f:\mathbb{R}\to \mathbb{R}\) נקראת אנליטית בנקודה \(x_{0}\) אם גזירה אינסוף פעמים והפיתוח טיילור שלה שווה לערך הפונקציה ברדיוס ההתכנסות. כלומר מתקיים:
$$f\left(x\right)=\sum_{n=0}^{\infty}{\frac{f^{\left(n\right)}\left(x_{0}\right)}{n!}}\left(x-x_{0}\right)^{n}$$

\end{definition}
\begin{proposition}
אם פונקציה אנילטית ב-\(x_{0}\) ולפיתוח טיילור יש רדיוס התכנסות \(R\) אז בפרט אנליטית בכל נקודה המקיימת \(\lvert x-x_{0} \rvert<R\).

\end{proposition}
\begin{proposition}
סכום ומכפלה של פונקציות אנליטיות יהיה אנליטי. כאשר חילוק של פונקציות אנליטיות תהיה אנילטית כאשר הפונקציה במכנה שונה מ-0.

\end{proposition}
\section{פתרון בעזרת טור חזקות}

\begin{definition}[נקודה רגולרית וסינגולארית]
עבור משוואה מהצורה:
$$P\left(x\right)y^{\prime\prime}+Q\left(x\right)y^{\prime}+R\left(x\right)y=0$$
כאשר \(P(x),Q(x),R(x)\) אנילטיות, נקרא לנקודה \(x_{0} \in \mathbb{R}\) נקודה רגולרית אם \(P(x_{0})\neq 0\) ונקודה סינגולארית אם \(P(x_{0})=0\).

\end{definition}
\begin{proposition}[קיום פתרון בעזרת טור חזקות]
תהי \(x_{0}\) נקודה רגולרית של המשוואה \(P\left(x\right)y^{\prime\prime}+Q\left(x\right)y^{\prime}+R\left(x\right)y=0\). אזי קיים פתרון כללי מהצורה של טור חזקות:
$$y\left(x\right)=\sum_{n=0}^{\infty}a_{n}\left(x-x_{0}\right)^{n}$$
או לחלופין ניתן להציג כצירוף לינארי \(y=ay_{1}+by_{2}\) של שתי פונקציות אנליטיות בלתי תלויות לינארית \(y_{1},y_{2}\) כאשר רדיוס ההתכנסות יהיה לפחות המינימום בין הרדיוסים של \(y_{1},y_{2}\). 

\end{proposition}
\begin{proposition}[שיטת פתרון בעזרת טור חזקות]
  \begin{enumerate}
    \item נציב טור חזקות מהצורה \(y=\sum_{n=0}^\infty a_{n}x^n\)


    \item כיוון שטור חזקות גזיר איבר איבר מתקיים: 
\begin{gather*}y'=\sum_{n=0}^\infty na_{n}x^{n-1}=\sum_{n=1}^\infty n a_{n}x^{n-1}=\sum_{n=0}^\infty (n+1)a_{n}x^{n} \\y''=\sum_{n=0}^\infty n(n-1)a_{n}x^{n-2}=\sum_{n=2}n(n-1)a_{n}x^{n-2}=\sum_{n=0}(n+2)(n+1)a_{n}x^n
\end{gather*}


    \item נאחד את כל הטורים ואנו יודעים כי טור חזקות שווה ל-0 אם"ם כל אחד מהמקדמים שלו שווים ל-0. ולכן נשוואה מקדם כללי של טור ל-0 


    \item נקבל נוסחת נסיגה. נעביר אותה לצורה \(a_{n}=f(n)a_{n-1}\)(יתכן וצריך לפרק לזוגיים ואי זוגיים) כעת נזכור את הנוסחה: 
$$a_{n}=f(n)a_{n-1}\implies a_{n}= \prod_{k=0}^n f(n)$$


  \end{enumerate}
\end{proposition}
\begin{reminder}[אנליטית בנקודה]
פונקציה \(f(x)\) נקראת אנליטית בנקודה \(x_{0}\) אם הטור חזקות שלה מתכנס בנקודה \(x_{0}\).

\end{reminder}
\begin{definition}[נקודה רגולרית/רגילה]
יהי  \(P\left(x\right)y^{\prime\prime}+Q\left(x\right)y^{\prime}+R\left(x\right)y=0\) משוואה דיפרנציאלית. נקודה \(x_{0}\) נקראת נקודה רגולרית אם \(\frac{Q(x_{0})}{P(x_{0})}\) ו-\(\frac{R(x_{0})}{P(x_{0})}\) פונקציות אנליטיות.

\end{definition}
\begin{definition}[סינגולאריות רגילה]
עבור משוואה מהצורה \(P\left(x\right)y^{\prime\prime}+Q\left(x\right)y^{\prime}+R\left(x\right)y=0\) נקודה \(x_{0}\in \mathbb{R}\) תקרא סינגולראית אם הפונקציות:
$$(x-x_{0})\,\frac{Q(x)}{P(x)},\,\,\left(x-x_{0}\right)^{2}\frac{R(x)}{P(x)}$$
אנליטיות ב-\(x_{0}\)

\end{definition}
\begin{corollary}
נקודה \(x_{0}\) היא סינגולארית רגילה אם"ם קיימים שתי הגבולות:
$$\operatorname*{lim}_{x\to x_{0}}\,\left(x-x_{0}\right){\frac{Q\left(x\right)}{P\left(x\right)}}\;,\quad\operatorname*{lim}_{x\to x_{0}}\,\left(x-x_{0}\right)^{2}{\frac{R\left(x\right)}{P\left(x\right)}}$$

\end{corollary}
\begin{remark}
הרעיון הוא להעביר את המשוואה לצורה:
$$y''+P(x)y'+Q(x)=0$$
ואז מספיק לדרוש כי \(P(x)\) ו-\(Q(x)\) אנליטיות כדי שהנקודה תהיה רגילה.

\end{remark}
\begin{example}
נסתכל על המשוואה הדיפרנציאלית:
$$y''+x^{2}y-4y=0$$
בנקודה \(x=0\) תהיה נקודה רגולרית. עבור למשל:
$$xy''+y'+y=0$$
נקבל כי:
$$\frac{Q(x)}{P(x)}=\frac{1}{x}\quad \frac{R(x)}{P(x)}=\frac{1}{x}$$
ולכן ב-\(x=0\) נקבל סינגולארית רגילה. כעת עבור:
$$x^{2}y''+y'+y$$
נקבל כי נקודה סינגולארית לא רגולארית.

\end{example}
\begin{example}[משוואת צ'בישב]
$$(1-x^{2})y''-xy'+p^{2}y=0\iff y'' - \frac{x}{1-x^{2}}y'+\frac{p^{2}}{1-x^{2}}$$
נקבל כי ב-\(x=\pm 1\) הנקודה היא סינגולארית רגילה. ולכן ניתן למצוא פתרון בעזרת טור חזקות ב-\(x=0\) עבור \(R\leq 1\).

\end{example}
\begin{theorem}[פרוביניוס]
  \begin{enumerate}
    \item יהי \(y''+P(x)y'+Q(x)y=f(x)\). אם \(x_{0}\) היא נקודה רגילה ו-\(P(x),Q(x)\) ו-\(f(x)\) אנליטיים עם רדיוס התכנסות \(R_{P},R_{Q},R_{f}\) אזי קיים פתרון בעזרת טור חזקות מהצורה: 
$$y(x)=\sum_{n=0}^{N}  a_{n}(x-x_{0})^{n}$$
כאשר רדיוס ההתכנסות יהיה \(R=\min(R_{P},R_{Q},R_{f})\).


    \item אם \(x_{0}\) היא נקודה סינגורית רגולארית ו-\((x-x_{0})P(x),(x-x_{0})^{2}Q(x)\) ו-\(f(x)\) יש רדיוס התכנסות \(R_{P},R_{Q},R_{f}\) בהתאמה אזי יש פתרון בעזרת טור חזקות מוכלל מהצורה: 
$$y(x)=(x-x_{0})^{r}\sum_{n=0}^{\infty}a_{n}(x-x_{0})^{n}$$
כאשר \(r\in \mathbb{R}\) קבוע ורדיוס ההתכנסות יהיה:
$$R=\min (R_{P},R_{Q},R_{f})$$


  \end{enumerate}
\end{theorem}
\begin{example}
נסתכל על המשוואה:
$$2xy''+y'+y=0\iff y''+\frac{1}{2x}y'+\frac{1}{2x}y=0$$
נניח פתרון מהצורה:
\begin{gather*}y(x)=\sum_{n=0}^{\infty} a_{n}x^{n+r}\implies y'=\sum_{n=0}^{\infty} (n+r)a_{n}x^{n+r-1}\implies \\\implies y''(x)=\sum_{n=0}^{\infty} (n+r)(n+r-1)a_{n}x^{n+r-2} 
\end{gather*}
כאשר נציב כעת במשוואה ונקבל:
$$\sum_{n=0}^{\infty}2(n+r)(n+r-1)a_{n}x^{n+r-1}+\sum_{n=0}^{\infty}2(n+r)a_{n}x^{n+r-1}+ \sum_{n=0}^{\infty} a_{n}x^{n+r}=0$$
כאשר ניתן לאחד את הטור הראשון והשני:
$$\sum_{n=0}^{\infty} (2n+2r-1)(n+r)a_{n}x^{n-1}+\sum_{n=0}^{\infty} a_{n}x^{n}$$
כאשר נזיז את הטור הראשון כך שיתחיל מ-\(-1\) כדי שהחזקות יהיו זהות:
$$\sum_{n=-1}^{\infty} (2n+2r+1)(n+r+1)a_{n+1}x^{n}+\sum_{n=0}^{\infty} a_{n}x^{n}=0  $$
הנכתוב בנפרד את הרכיב ה-\(-1\) ונאחד את הטורים:
$$(2r-1)ra_{0}x^{-1}+\sum_{n=0}^{\infty}[(2n+2r+1)(n+r+1)a_{n+1}+a_{n}]x^{n}=0$$
כאשר נדרוש כי המקדמים מתאפסים. כלומר:
$$(2r-1)ra_{0}=0\land (2n+2r+1)(n+r+1)a_{n+1}+a_{n}$$
כאשר מהמשוואה הראשונה נקבל כי \(r=0\) או \(r=\frac{1}{2}\) ומהמשוואה השנייה נקבל את הנוסחאת נסיגה:
$$a_{n+1}=-\frac{1}{(2n+2r+1)(n+r+1)}a_{n}$$
כאשר נשים לב כי בניגוד לשיטה הרגילה, עבור משוואה מסדר שני לא תמיד מקבלים ביטוי שונה עבור הערכים הזוגיים והאי זוגיים! נדרש כעת לפתור בנפרד עבור \(r=0\) ועבור \(r=\frac{1}{2}\). עבור \(r=0\) נקבל:
$$y_{1}(x)=\sum_{n=0}^{\infty}a_{n}x^{n}\quad a_{n+1}= -\frac{1}{(2n+1)(n+1)}a_{n}$$
עבור \(r=\frac{1}{2}\) נקבל:
$$y_{2}(x)=\sum_{n=0}^{\infty} a_{n}x^{n+1/2}\quad a_{n+1}=- \frac{1}{(2n+2)\left( n+\frac{3}{2} \right)}$$
כאשר הפתרון הכללי יהיה מהצורה:
$$y(x)=C_{1}y_{1}(x)+C_{2}y_{2}(x)$$

\end{example}
נסכם את סכמת פתרון - שיטת פרוביניוס:

\begin{enumerate}
  \item נניח פתרון מהצורה: 
$$y(x)=(x-x_{0})^{r}\sum_{k=0}^{\infty}a_{k}(x-x_{0})^{k}=\sum_{k=0}^{\infty}a_{k}(x-x_{0})^{k+r}$$


  \item בעזרת גזירה איבר איבר והזזה נקבל: 


\end{enumerate}
\section{פתרון באמצעות טור}

\begin{definition}[פונקציה אנליטית]
פונקציה אשר ניתן לפתח עבורה טור טיילור בתחום. ניתן לכתוב כל פונקציה אנליטית בצורה:
$$f(x)=\sum_{n=0}^{\infty} a_{n}(x-x_{0})^{n}$$

\end{definition}
\begin{definition}[נקודה רגולרית וסינוגלאית]
משוואה מהצורה:
$$y^{\prime\prime}+p(x)\,y^{\prime}+q(x)\,y=0$$
נקראת רגולרית בנקודה \(x_{0} \in \mathbb{R}\) אם גם \(p(x)\) וגם \(q(x)\) אנליטיות. אחרת נקראת סינגולארית.

\end{definition}
\begin{example}
עבור המשוואה:
$$x\,y^{\prime\prime}+y^{\prime}+x^{2}\,y=0.$$
נרצה להעביר לצורה שבו המקדם המוביל הוא 0:
$$y^{\prime\prime}+\frac{1}{x}\,y^{\prime}+x\,y=0.$$
ונקבל כי עבור \(x\neq 0\) נקבל כי \(x\) רגולרית ועבור \(x=0\) נקבל כי סינגולארית. 

\end{example}
\begin{proposition}[נגזרת של פונקציה אנליטית]
אם \(y\) אנליטית ניתנת לכתיבה בצורה:
$$y=\sum_{n=0}^{\infty} a_{n}(x-x_{0})^{n}$$
כאשר הנגזרת מקיימת:
$$
y'=\sum_{n=0}^\infty na_{n}x^{n-1}=\sum_{n=0}^\infty (n+1)a_{n}x^{n}$$

\end{proposition}
\begin{proof}
כיוון שאנליטית ניתן לגזור את הטור איבר איבר ולקבל:
$$y'=\sum_{n=0} (n+1)a_{n+1}x^{n}$$
כיוון שאנו רוצים שהטור יתחיל מ-0 נגדיר \(m=n-1\) ולכן נציב \(n=m+1\) ונקבל:
$$y'=\sum_{m=0}^\infty (m+1) a_{m+1}x^{m+1-1}=\sum_{n=0} (m+1)a_{m+1}x^{m}$$
כאשר ניתן באותו אופן לקרוא ל-\(m\) עוד פעם \(n\) ולקבל את התוצאה של הטענה.

\end{proof}
\begin{corollary}
עבור הנגזרת השנייה נקבל באופן דומה
$$y''=\sum_{n=0}^\infty n(n-1)a_{n}x^{n-2}=\sum_{n=2}^{\infty}n(n-1)a_{n}x^{n-2}=\sum_{n=0}^{\infty}(n+2)(n+1)a_{n+2}x^n$$
וניתן להמשיך לנגזרות גבוההות.

\end{corollary}
\begin{proposition}
עבור משוואה מהצורה:
$$y^{\prime\prime}+p(x)\,y^{\prime}+q(x)\,y=0$$
אם \(p(x),q(x)\) אנליטיים בסביבה פתוחה \(\left( x_{0}-\varepsilon,x_{0}+\varepsilon \right)\), אזי קיימים שתי פתרונות בלתי תלויים \(y_{1},y_{2}\) אשר יהיו אנליטיים באותה סביבה.

\end{proposition}
\begin{proof}
נציב טור חזקות של הפונקציה \(y\) ושל הנגזרות:
$$y(x)=\sum_{n=0}^{\infty}a_{n}\left(x-x_{0}\right)^{n}.$$$$y^{\prime}(x)=\sum_{n=1}^{\infty}n a_{n}\left(x-x_{0}\right)^{(n-1)}=\sum_{n=0}^{\infty}(n+1)a_{(n+1)}\left(x-x_{0}\right)^{n}$$$$y^{\prime\prime}(x)=\sum_{n=0}^{\infty}(n+2)(n+1)a_{(n+2)}\left(x-x_{0}\right)^{n}$$
כאשר נכפיל את הטורים בפונקציות המתאימות ונקבל:
$$\begin{array}{c}{{q(x)\,y=\left(\sum_{n=0}^{\infty}q_{n}(x-x_{0})^{n}\right)\,\left(\sum_{m=0}^{\infty}a_{m}(x-x_{0})^{m}\right)}}\\ {{=\sum_{n=0}^{\infty}\left(\sum_{k=0}^{n}a_{k}q_{n-k}\right)(x-x_{0})^{n},}}\end{array}$$$$\sum_{n=0}^{\infty}\Bigl[(n+2)(n+1)a_{(n+2)}+\sum_{k=0}^{n}\bigl[(k+1)a_{(k+1)}p_{(n-k)}+a_{k}q_{(n-k)}\bigr]\Bigr]\,(x-x_{0})^{n}=0.$$$$(n+2)(n+1)a_{(n+2)}+\sum_{k=0}^{n}\bigl[(k+1)a_{(k+1)}p_{(n-k)}+a_{k}q_{(n-k)}\bigr]=0$$

\end{proof}
\section{משוואת לג'נדר ופולינומי לג'נדר}

\begin{definition}[משוואת לג'נדר]
$$(1-x^{2})y^{\prime\prime}-2x y^{\prime}+\ell(\ell+1)y=0,$$

\end{definition}
\begin{remark}
שימושי במיוחד במשוואות שכוללות סימטריה ספרית, כאשר משוואה זו מופיעה באופן טבעי.

\end{remark}
\begin{proposition}
יש שלושה נקודות סינגולאיות - \(x=-1,1,\infty\). 

\end{proposition}
\begin{definition}[פונקציות לג'נדר]
פתרון של משוואת לג'נדר.

\end{definition}
\begin{proposition}
פתרון של משוואת לג'נדר יהיה \(y(x)=c_{1}y_{1}(x)+c_{2}y_{2}(x)\) כאשר:
\begin{gather*}y_{1}(x)=1-\ell(\ell+1)\frac{x^{2}}{2!}+(\ell-2)\ell(\ell+1)(\ell+3)\frac{x^{4}}{4!}-\cdots \\y_{2}(x)=x-(\ell-1)(\ell+2)\frac{x^{3}}{3!}+(\ell-3)(\ell-1)(\ell+2)(\ell+4)\frac{x^{5}}{5!}-\cdots
\end{gather*}
כך שהחוקיות בין האיברים נקבע לפי הכלל נסיגה:
$$a_{n+2}=\frac{[n(n+1)-\ell(\ell+1)]}{(n+1)(n+2)}a_{n}$$

\end{proposition}
\begin{proof}
נפתור בעזרת טור חזקות:
$$\sum_{n=0}^{\infty}\left[n(n-1)a_{n}x^{n-2}-n(n-1)a_{n}x^{n}-2n a_{n}x^{n}+\ell(\ell+1)a_{n}x^{n}\right]=0,$$
איחוד משתנים יתן:
$$\sum_{n=0}^{\infty}\left\{(n+2)(n+1)a_{n+2}-[n(n+1)-\ell(\ell+1)]a_{n}\right\}x^{n}=0.$$
כאשר מיחידות הטור נדרוש כי כל המקדמים מתאפסים, מכאן נקבל את הנוסחאת נסיגה:
$$a_{n+2}=\frac{[n(n+1)-\ell(\ell+1)]}{(n+1)(n+2)}a_{n},$$
מספיק למצוא שתי פתרונות בלתי תלויים. עבור \(a_{0}=1\) ו-\(a_{1}=0\) נקבל:
$$y_{1}(x)=1-\ell(\ell+1)\frac{x^{2}}{2!}+(\ell-2)\ell(\ell+1)(\ell+3)\frac{x^{4}}{4!}-\cdots,$$
כאשר עבור \(a_{0}=0\) ו-\(a_{1}=1\) נקבל:
$$y_{2}(x)=x-(\ell-1)(\ell+2)\frac{x^{3}}{3!}+(\ell-3)(\ell-1)(\ell+2)(\ell+4)\frac{x^{5}}{5!}-\cdots$$
כאשר נשים לב כי רדיוס ההתכנסות של הטור יהיה 1(למשל ממבחן דלמבר)
לכן פתרון כללי יהיה מהצורה:
$$y(x)=c_{1}y_{1}(x)+c_{2}y_{2}(x)$$

\end{proof}
\begin{proposition}
כאשר \(\ell\) שלם הנוסחאת נסיגה מתאפסת בשלב מסויים, ולכן נקבל פולינום. בפרט אם \(\ell\) זוגי אז \(y_{1}\) הופך לפולינום ואם \(\ell\) אי זוגי אז \(y_{2}\) הופך לפלונום. 

\end{proposition}
\begin{definition}
הטור מתאפס הבשלב מסויים נקרא פולינומי לג'נדר, או פונקציות בסל מהסוג הראשון ומסומן \(P_{\ell}(x)\). הטור הנותר נקרא פונקציית בסל מהסוג השני, ומסומן \(Q_{\ell}(x)\), כאשר \(Q_{\ell}(x)\) הוא פתרון המתקבל רק במקרה סונגולרי-רגולרי, ופרט מתבדר ב-\(x=\pm 1\).

\end{definition}
\begin{corollary}
עבור \(\ell\) שלם נקבל כי הפתרון הכללי של משוואת בסל מוגדר על ידי:
$$y(x)=c_{1}P_{\ell}(x)+c_{2}Q_{\ell}(x),$$

\end{corollary}
\begin{proposition}[נוסחאת רודריגס]
$$P_{\ell}(x)=\frac{1}{2^{\ell}\ell!}\frac{d^{\ell}}{d x^{\ell}}(x^{2}-1)^{\ell}$$

\end{proposition}
\begin{proof}
נגדיר \(u=(x^{2}-1)^{\ell}\) כך ש-\(u'=2\ell x(x^{2}-1)^{\ell-1}\) וכעת:
$$(x^{2}-1)u^{\prime}-2\ell x u=0.$$
אם נגזור \(\ell+1\) פעמים נקבל:
$$\left[(x^{2}-1)u^{(\ell+2)}+2x(\ell+1)u^{(\ell+1)}+\ell(\ell+1)u^{(\ell)}\right]-2\ell\left[x u^{(\ell+1)}+(\ell+1)u^{(\ell)}\right]=0,$$
אשר מצטמם לביטוי:
$$(x^{2}-1)u^{(\ell+2)}+2x u^{(\ell+1)}-\ell(\ell+1)u^{(\ell)}=0.$$
וקיבלנו את משוואת לג'נדר עבור \(u^{\left( \ell \right)}\).  כיוון ש-\(\ell\) מספר טבעי ו-\(u^{\left( \ell \right)}\) רגולרי נקבל:
$$u^{(\ell)}(x)=c_{\ell}P_{\ell}(x)$$

\end{proof}
\begin{corollary}
$$I_{\ell}=\int_{-1}^{1}P_{\ell}(x)P_{\ell}(x)\,d x={\frac{2}{2\ell+1}}$$

\end{corollary}
\begin{proposition}
משוואת לג'נדר היא משוואת שטרום לוביל, ולכן בפרט ניתן לכתוב כל פונקציה המתנהגת יפה בתחום \(\lvert x \rvert<1\) בצורה הבאה:
$$f(x)=\sum_{\ell=0}^{\infty}a_{\ell}P_{\ell}(x),$$
כאשר המקדמים יהיו שווים ל:
$$a_{\ell}={\frac{2\ell+1}{2}}\int_{-1}^{1}f(x)P_{\ell}(x)\,d x$$

\end{proposition}
\begin{proposition}[יחס נסיגה של נגזרות]
מתקיים:
$$(2n+1)P_{n}=P_{n+1}^{\prime}-P_{n-1}^{\prime}$$

\end{proposition}
\begin{definition}[ההרמוניות הספריות]
$$Y_{\ell}^{m}(\theta,\phi)=(-1)^{m}\left[\frac{2\ell+1}{4\pi}\frac{(\ell-m)!}{(\ell+m)!}\right]^{1/2}P_{\ell}^{m}(\cos\theta)\exp(i m\phi).$$

\end{definition}
כאשר נשים לב ליחס:

\section{משוואת לג'נדר המוכללת}

\begin{definition}[משוואת לג'נדר המוכללת]
משוואה מהצורה:
$$(1-x^{2})y^{\prime\prime}-2x y^{\prime}+\left[\ell(\ell+1)-\frac{m^{2}}{1-x^{2}}\right]y=0,$$
כאשר למשוואה זו תהיה נקודה סינגולארית ב-\(x=-1,1,\infty\).

\end{definition}
\begin{remark}
נשים לב כי עבור \(m=0\) נקבל את משוואת לג'נדר הידועה. כאשר משוואה זו שימושית כשמופיע לפלסיאן בקורדינטות ספריות.

\end{remark}
\begin{proposition}
אם \(u(x)\) הוא פתרון של משוואת לג'נדר, אזי:
$$y(x)=(1-x^{2})^{|m|/2}\frac{d^{|m|}u}{d x^{|m|}}$$
יהיה פתרון של משוואת לג'נדר המוכללת.

\end{proposition}
\begin{proof}
נניח כי \(u\) פותר את משוואת לג'נדר. לכן:
$$(1-x^{2})u^{\prime\prime}-2x u^{\prime}+\ell(\ell+1)u=0,$$
כאשר נגזור \(m\) פעמים לפי נוסחאת לייבניץ ונקבל:
$$(i)\quad (1-x^{2})v^{\prime\prime}-2x(m+1)v^{\prime}+\left( \ell-m \right)\left( \ell+m+1 \right)v=0,$$
כאשר \(v(x)=\frac{\mathrm{d} ^{m}u}{\mathrm{d} x^{m}}\). ניתן לכתוב את הנגזרות \(v',v''\) בצורה הבאה:
$$(1-x^{2})y^{\prime\prime}-2x y^{\prime}+\left[\ell(\ell+1)-\frac{m^{2}}{1-x^{2}}\right]y=0$$$$P_{\ell}^{m}(x)=(1-x^{2})^{m/2}\frac{d^{m}P_{\ell}}{d x^{m}},\qquad Q_{\ell}^{m}(x)=(1-x^{2})^{m/2}\frac{d^{m}Q_{\ell}}{d x^{m}}$$$$y(x)=c_{1}P_{\ell}^{m}(x)+c_{2}Q_{\ell}^{m}(x),$$$$P_{\ell}^{m}(x)=\frac{1}{2^{\ell}\ell!}(1-x^{2})^{m/2}\frac{d^{\ell+m}}{d x^{\ell+m}}(x^{2}-1)^{\ell},$$$$\begin{array}{l l}{{P_{1}^{1}(x)=(1-x^{2})^{1/2},}}&{{\qquad P_{2}^{1}(x)=3x(1-x^{2})^{1/2},}}\\ {{P_{2}^{2}(x)=3(1-x^{2}),}}&{{\qquad P_{3}^{1}(x)=\frac{3}{2}(5x^{2}-1)(1-x^{2})^{1/2},}}\\ {{P_{3}^{2}(x)=15x(1-x^{2}),}}&{{\qquad P_{3}^{3}(x)=15(1-x^{2})^{3/2}.}}\end{array}$$$$\int_{-1}^{1}P_{\ell}^{m}(x)P_{k}^{m}(x)\,d x=0$$$$I_{\ell m}\equiv\int_{-1}^{1}P_{\ell}^{m}(x)P_{\ell}^{m}(x)\,d x=\frac{2}{2\ell+1}\frac{(\ell+m)!}{(\ell-m)!}.$$$$I_{\ell m}=\frac{1}{2^{2\ell}(\ell!)^{2}}\int_{-1}^{1}\left[(1-x^{2})^{m}\,\frac{d^{\ell+m}(x^{2}-1)^{\ell}}{d x^{\prime+m}}\right]\left[\frac{d^{\ell+m}(x^{2}-1)^{\ell}}{d x^{\prime+m}}\right]\,d x,$$$$I_{/m}=\frac{(-1)^{\prime+m}}{2^{2\prime}(\ell!)^{2}}\int_{-1}^{1}(x^{2}-1)^{\prime}\,\frac{d^{\prime+m}}{d x^{\prime+m}}\left[(1-x^{2})^{m}\,\frac{d^{\prime+m}(x^{2}-1)^{\prime}}{d x^{\prime+m}}\right]\,d x.$$$$\frac{d^{\ell+m}}{d x^{\ell+m}}\left[(1-x^{2})^{m}\,\frac{d^{\ell+m}(x^{2}-1)^{\ell}}{d x^{\ell+m}}\right]=\sum_{r=0}^{\ell+m}\frac{(\ell+m)!}{r!(\ell+m-r)!}\,\frac{d^{r}(1-x^{2})^{m}}{d x^{r}}\,\frac{d^{\ell+2m-r}(x^{2}-1)^{r}}{d x^{2r+2m-r}}.$$$$I_{\ell m}=\frac{(-1)^{\ell+m}}{2^{2\ell}(\ell!)^{2}}\frac{(\ell+m)!}{(2m)!(\ell-m)!}\int_{-1}^{1}(1-x^{2})^{\ell}\frac{d^{2m}(1-x^{2})^{m}}{d x^{2m}}\frac{d^{2\ell}(1-x^{2})^{\ell}}{d x^{2\ell}}\,d x.$$$$\frac{\mathrm{d} ^{2\ell}}{\mathrm{d} x^{2\ell}} (1-x^{2})^{\ell}=\frac{(-1)^{\ell}}{\left( 2\ell \right)!}\qquad (-1)^{2\ell+2m}=1$$$$I_{\ell m}=\frac{1}{2^{2\ell}(\ell!)^{2}}\frac{(2\ell)!(\ell+m)!}{(\ell-m)!}\int_{-1}^{1}(1-x^{2})^{\ell}\,d x.$$$$\int_{-1}^{1}(1-x^{2})^{\ell}\,d x=\frac{2^{2\ell+1}(\ell!)^{2}}{(2\ell+1)!},$$$$I_{\ell m}=\frac{2}{2\ell+1}\frac{(\ell+m)!}{(\ell-m)!}$$$$f(x)=\sum_{k=0}^{\infty}a_{m+k}P_{m+k}^{m}(x),$$$$a_{\ell}=\frac{2\ell+1}{2}\frac{(\ell-m)!}{(\ell+m)!}\int_{-1}^{1}f(x)P_{\ell}^{m}(x)\,d x.$$$$G(x,h)=\frac{(2m)!(1-x^{2})^{m/2}}{2^{m}m!(1-2h x+h^{2})^{m+1/2}}=\sum_{n=0}^{\infty}P_{n+m}^{m}(x)h^{n}.$$$$v^{\prime}=(1-x^{2})^{-m/2}\left(y^{\prime}+\frac{m x}{1-x^{2}}y\right),$$$$v^{\prime\prime}=(1-x^{2})^{-m/2}\left[y^{\prime\prime}+\frac{2m x}{1-x^{2}}y^{\prime}+\frac{m}{1-x^{2}}y+\frac{m(m+2)x^{2}}{(1-x^{2})^{2}}y\right]$$כאשר נציב את הנגזרות במשוואה \((i)\) ואחרי פישוט נקבל:שמראה ש-\(y\) פתרון של משוואת לג'נדר המוכללת.\textbf{הגדרה} פונקציות לג'נדר המוכלליםכאשר \(P^{m}_{\ell}(x)\) נקראים פולינומי לג'נדר המוכללים ו-\(Q^{m}_{\ell}(x)\) נקראים פונקציות לג'נדר המוכללים מהסוג השני.\textbf{מסקנה}הפתרון הכללי משוואת לג'נדר המוכללת תהיה:\textbf{טענה}פולינומי לג'נדר המוכללים:פולינומי לג'נדר המוכללים הראשונים יהיו:\textbf{טענה} יחס אורתוגונאליותעבור \(\ell \neq k\) מתקיים:כאשר עבור \(\ell=k\) נקבל:מההגדרות של פונקציות לג'נדר המוכללות ומנוסחאת רודריגס נקבל:נבצע אינטגרציה בחלקים \(\ell+m\) פעמים כאשר נשים לב כי כל האיברי שפה מתאפסים, ונקבל:כאשר מגזירה בעזרת נוסחאת לייבניץ למכפלה נקבל:נסתכל על אגף ימין. נשים לב כי הגורם\(\frac{d^{r}(1-x^{2})^{m}}{d x^{r}}\) לא מתאפס עבור \(r\leq 2m\) כאשר הגורם \(\frac{d^{2\ell+2m-r}(x^{2}-1)^{\ell}}{d x^{2\ell+2m-r}}\) לא מתאפס עבור \(2\ell+2m-r\leq 2\ell\). כלומר כאשר \(r\geq 2m\). מאיחוד שני תנאי אלו נקבל כי האיבר היחיד שלא מתאפס בסכום מתקבל עבור \(r=2m\). לכן ניתן לכתוב:כאשר כיוון שמתקיים:נקבל:כאשר אנו יודעים כי:ולכן:\textbf{הערה}בזבתי את הזמן בלכתוב את ההוכחה הזו כשהיא למעשה נובעת משטרום ליוביל.\textbf{טענה}פולינומי לג'נדר המוכללים מקיימים את שטרום ליוביל, ובפרט מקיימת יחס שלמות, כלומר פונקציה \(f\) המתנהגת יפה מקיימת:כאשר:\textbf{טענה} פונקציה יוצרת\textbf{טענה} נוסחאות נסיגה
$$P_{n}^{m+1}=\frac{2m x}{(1-x^{2})^{1/2}}P_{n}^{m}+[m(m-1)-n(n+1)]P_{n}^{m-1},$$$$(2n+1)xP_{n}^{m}=(n+m)P_{n-1}^{m}+(n-m+1)P_{n+1}^{m},$$$$(2n+1)(1-x^{2})^{1/2}P_{n}^{m}=P_{n+1}^{m+1}-P_{n-1}^{m+1},$$$$2(1-x^{2})^{1/2}(P_{n}^{m})^{\prime}=P_{n}^{m+1}-(n+m)(n-m+1)P_{n}^{m-1}.$$

\end{proof}
\section{שטרום ליוביל}

\begin{reminder}[מרחבי מכפלה פנימית]
\textbf{ססקילינאריות:}\begin{gather*}\left\langle \mathbf{x},\mathbf{x} \right\rangle\geq0 \qquad \left\langle  \mathbf{x},\mathbf{x} \right\rangle=0\iff \mathbf{x}=0 \\ \langle a\mathbf{x}+b\mathbf{y},\mathbf{z}\rangle=a\left\langle\mathbf{x},\mathbf{z}\right\rangle+b\left\langle\mathbf{y},\mathbf{z}\right\rangle \\ \langle\mathbf{x},a\mathbf{y}\rangle=\overline{\langle a\mathbf{y},\mathbf{x}\rangle}=\bar{a}\left\langle\mathbf{x},\mathbf{y}\right\rangle
\end{gather*}\textbf{אי שוויונות:}\begin{gather*}\left|\left\langle \mathbf{x},\mathbf{y} \right\rangle\right|^2\leq\left\langle\mathbf{x},\mathbf{x}\right\rangle\left\langle\mathbf{y},\mathbf{y}\right\rangle=||\mathbf{x}||^2\cdot ||\mathbf{y}||^2\\ \|\mathrm{x}+\mathrm{y}\|\leq\|\mathrm{x}\|+\|\mathrm{y}\|
\end{gather*}

\end{reminder}
\begin{definition}[המכפלה הפנמית של שטרום ליוביל]
מוגדרת ע"י:
$$ u\cdot v \equiv\langle u,v\rangle\equiv\int_a^bu^*(x)v(x)w(x)\:dx$$
כאשר \(w(x)\) נקראת פונקציית המשקל.

\end{definition}
\begin{definition}[אופרטור צמוד]
האופרטור \(L^*\) היחיד שמקיים \(\langle Lu, v \rangle=\langle u, L^*v \rangle\) לכל \(u,v\).

\end{definition}
\begin{definition}[אופרטור צמוד לעצמו]
אופרטור \(L\) המקיים \(\langle Lu, v \rangle=\langle u, Lv \rangle\) נקרא אופרטור צמוד לעצמו

\end{definition}
\begin{proposition}
אם תנאי השפה הומוגנים, ניתן לכתוב את האופרטור הצמוד של אופרטור מסדר שני בצורה הבאה:
$$L^{*}y=\frac{1}{w(x)}\left\{\left[p_{0}(x)y(x)\right]^{\prime\prime}-\left[p_{1}y(x)\right]^{\prime}+p_{2}y(x)\right\}.$$

\end{proposition}
\begin{proof}
יהיו \(u(x),v(x)\) פונקציות שרירותיות. נניח \(L\) אופרטור צמוד לעצמו. לכן \(\langle Lu, v \rangle=\langle v, Lu \rangle\), ומתקיים:
$$\langle Lu, v \rangle \!=\!\int_{a}^b \!\!w(x)v\frac{1}{w(x)}\left[ p_0(x)u''\!+\!p_1(x)u'+p_2(x)u \right]dx\!=\!\int_{a}^b \!\!p_{0}(x)vu''+p_{1}vu'+p_{2}u\;\mathrm{d}x$$
בעזרת אינטגרציה בחלקים נקבל כי ביטוי זה שווה לביטוי:
$${{\displaystyle\langle Lu,v\rangle}}{{=}}{{\biggl[p_{0}v u^{\prime}-(p_{0}v)^{\prime}u+p_{1}v u\biggr]_{a}^{b}}}{{+\displaystyle\int_{a}^{b}w(x)u(x)\left[\frac{1}{w(x)}\left[(p_{0}v)^{\prime\prime}-(p_{1}v)^{\prime}+p_{2}v\right]\right]d x,}}$$
כאשר אם תנאי השפה הומוגנים, נקבל כי הגורם הראשון מתאפס. כאשר מה שנמצא באינטגרל ניתן יהיה \(\langle u, L^*v \rangle=\int_{a}^b w(x)u(x)L^*v\;dx\) ולכן האופרטור בסוגריים זה האופרטור הצמוד.

\end{proof}
\begin{proposition}
אופרטור יהיה צמוד לעצמו אם"ם ניתן לכתוב אותו בצורה:
$$Ly= \frac d{dx}\left[p\left(x\right)\frac{dy}{dx}\right]+\left[\lambda w\left(x\right)-q\left(x\right)\right]y=0$$

\end{proposition}
\begin{proposition}
אופרטור יהיה צמוד לעצמו אם מתקיים \(p_{0}'=p_{1}\)

\end{proposition}
\begin{proof}
אנו יודעים כי הצורה הצמודה תהיה:
$$L^{*}y=\frac{1}{w(x)}\left\{\left[p_{0}(x)y(x)\right]^{\prime\prime}-\left[p_{1}y(x)\right]^{\prime}+p_{2}y(x)\right\}$$
נבצע גזירה לפי כלל המכפלה, נאחד גורמים ונקבל:
$$L^{\ast}y=\frac{1}{w(x)}\left\{p_{0}y^{\prime\prime}+\left[2p_{0}^{\prime}-p_{1}\right]y^{\prime}+\left[p_{0}^{\prime\prime}-p_{1}^{\prime}+p_{2}\right]y\right\}$$
כדי שיהיה צמוד לעצמו, צריך שהמקדמים יהיו שווים למשוואה המקורית, לכן:
$$1)\;\;p_{0}=p_{0}\qquad 2)\;\;p_{1}=2p_{0}'-p_{1}\qquad 3)\;\;p_{2}=p_{0}''-p_{1}'+p_{2}$$
כאשר המשוואה הראשונה תמיד מתקיימת, ואם משוואה 2 מקיימת אז גם כן משוואה 3, לכן מספיק לבדוק אם 2 מתקיים.

\end{proof}
\begin{proposition}[הצמדת אופרטור לעצמו]
ניתן להעביר כל אופרטור מסדר שני לצורה צמודה לעצמה ע"י הצעדים הבאים:

  \begin{enumerate}
    \item נכפיל בפונקציה \(F(x)\) מהצורה: 
$$ \begin{aligned}F(x)=\exp\left[\int_a^x\frac{p_1(x')-p_0'(x')}{p_0(x')}dx'\right]\end{aligned}$$
כאשר בשביל הגבולות פשוט מחשבים אינטגרל לא מסויים ולא מוסיפים \(+C\) כי מספיק פונקציה כלשהי שמקיימת


    \item נגדיר אופרטור חדש \(\tilde{L}=F(x)L\) ונקבל: 
$$\tilde Ly(x)=F(x)Ly(x)=\frac{d}{dx}[F(x)p_0(x)y'(x)]+F(x)p_2(x)y(x)$$
כעת אופרטור זה יהיה צמוד לעצמו. ופנקציית המשקל גם כן תקיים:
$$ \tilde{w}(x)=F(x)w(x)$$


  \end{enumerate}
\end{proposition}
\begin{definition}[משוואת שטרום ליוביל]
זו תהיה משוואה דיפרנציאלית מהצורה:
$${\frac{\mathrm{d}}{\mathrm{d}x}}\biggl[p(x){\frac{\mathrm{d}y}{\mathrm{d}x}}\biggr]+q(x)y=-\lambda\,w(x)y,$$
כאשר \(p(x),q(x),w(x)\) נתונים.

\end{definition}
נשים לב כי זה למעשה איזשהי בעיית ערכים עצמיים עבור אופרטור דיפרנציאלי מהצורה:
$$Ly={\frac{\mathrm{d}}{\mathrm{d}x}}\biggl[p(x){\frac{\mathrm{d}y}{\mathrm{d}x}}\biggr]+q(x)y$$
תחת המכפלה הפנימית \(\langle f, g \rangle=\int f(x)g(x)w(x)dx\).

\begin{definition}[בעיית שטרום ליוביל]
בהנתן משוואת שטרום ליוביל עם תנאי שפה מהצורה:
$$\begin{array}{c}{{\alpha_{1}y(a)+\alpha_{2}y^{\prime}(a)=0}}\qquad {{\beta_{1}y(b)\,+\,\beta_{2}y^{\prime}(b)=0}}\end{array}$$
בעיית שטרום ליוביל זה הבעיית ערכים עצמיים בהם מחפשים:

  \begin{enumerate}
    \item למצוא את כל ה-\(\lambda\) שעבורם קיים פתרון למשוואת שטרום ליוביל 


    \item למצוא את כל הפונקציות העצמיות המקיימות \(Ly=\lambda w(x)y\)


  \end{enumerate}
\end{definition}
\begin{theorem}[שטרום ליוביל]
אם אופרטור דיפרנציאלי מסדר שני מקיים:

  \begin{enumerate}
    \item האופרטור צמוד לעצמו 


    \item המקדמים הם פונקציות רציפות בתחום של המכפלה הפנימית 


    \item מתקיימים תנאי שפה הומוגנים מהצורה: 
$$\begin{array}{c}{{\alpha_{1}y(a)+\alpha_{2}y^{\prime}(a)=0}}\qquad {{\beta_{1}y(b)\,+\,\beta_{2}y^{\prime}(b)=0}}\end{array}$$


  \end{enumerate}
אז האופרטור מקיים את משוואת שטרום ליוביל ולכן:

  \begin{enumerate}
    \item קיים סט של פונקציות עצמיות המקיימות \(Ly = \lambda y\) כך ש-\(y\) ממשי. 


    \item הפונקציות העצמיות הם אוסף שלם של מרחב הפונקציות. כלומר כל פונקציה(כולל פונקציות מוכללות) ניתנות לביטוי ע"י צירוף לינארי של הפונקציות העצמיות. 


    \item הפונקציות העצמיות הם אורתוגונאליות אחת לשנייה ביחס למכפלה הפנימית של שטרום ליוביל 


  \end{enumerate}
\end{theorem}
\begin{proof}
נובע ישירות מהמשפט הספקטרלי עבור פונקציות.

\end{proof}
\chapter{התמרות אינטגרליות}

\section{פונקציות מוכללות}

\begin{definition}[פונקציה מוכללת]
לעיתים נקרא התפלגות. מחלקת כל הפונקציות אשר מתקבלות מסדרות אשר "מתנהגות יפה".

\end{definition}
\begin{definition}[פונקציה יוצרת]
סדרת פונקציות אשר הפונקציה הגבולית שלהם תהיה הפונקציה המוכללת. יהיה יותר מפונקציה אחת כזו.

\end{definition}
\begin{definition}[פונקציית דלתא של דיראק]
הפונקציה מוגדרת בצורה הבאה:
$$ \delta(x)=\left\{\begin{matrix}''\infty''&x=0,\\0&x\ne0.\end{matrix}\right.$$
בצורה שהשטח מתחת הוא 1.
הגדרה יותר פרמלית היא באמצעות גבול של סדרת פונקציות. נדרש סדרה \(\phi_{n}\) של פונקציות ש:

  \begin{enumerate}
    \item מתבדרת לאינסוף ב-0 


    \item בכל מקום אחר מתכנסת ל-0  


    \item השטח מתחת לגרף תמיד נשאר אחד. 


  \end{enumerate}
\end{definition}
יש מספר פונקציות שעונות על דרישות אלו:

\begin{enumerate}
  \item הלורזיאן: 
$$ \phi_1^a(x)=\frac{1}{\pi}\frac{1}{1+x^2}$$


  \item הגאוסיאן: 
$$ \phi_1^b(x)=\frac{1}{\sqrt{\pi}}e^{-x^2}$$


\end{enumerate}
\begin{proposition}[תכונות של פונקציית דלתא]
ניתן להוכיח תכונות אלו בעזרת ההגדרה של הפונקצייה הגבולית של סדרת פונקציות.

  \begin{enumerate}
    \item הזזה: 
$$ \delta(x-a)=\left\{\begin{array}{ll}''\infty''&x=a\\0&x\neq a\end{array}\right.$$


    \item אינטגרל של כפל בפונקציית דלתא: 
$$ I=\displaystyle\int\limits_{-\infty}^{\infty}\delta(x-a)f(x)\:dx=f(a)$$


    \item מתיחה: 
$$ \delta(ax)=\frac{\delta(x)}{|a|}.$$


    \item פונקציית דלתא של פונקציה: 
$$ \delta\left[g\left(x\right)\right]=\sum_{i}\frac{\delta\left(x-x_{i}\right)}{\left|g^{\prime}\left(x_{i}\right)\right|}$$
כאשר \(x_{i}\) הם שורשים של \(g\) כך ש-\(g'(x_{i})\neq 0\). 


  \end{enumerate}
\end{proposition}
\begin{example}
$$ \delta\left(x^2-a^2\right)=\frac1{2\left|a\right|}\left[\delta\left(x+a\right)+\delta\left(x-a\right)\right]$$$$ \delta\left(x^2+x-2\right)=\frac13\delta\left(x-1\right)+\frac13\delta\left(x+2\right)$$

יש למצוא את האינטגרל הבא:
$$ I=\int_{-\infty}^{\infty}\delta\left(e^{|x|}\sin(x)\right)dx.$$\textbf{פתרון:}$$ e^{|x|}\sin(x)=0\implies x=\pi n,$$
ולכן:
$$ \begin{aligned}I&=\int_{-\infty}^{\infty}\sum_{n=-\infty}^{\infty}\frac{\delta\left(x-n\pi\right)}{e^{\pi|n|}}dx=\sum_{n=-\infty}^{\infty}e^{-\pi|n|}=1+2\sum_{n=1}^{\infty}e^{-\pi n}\\&=1+2\frac{e^{-\pi}}{1-e^{-\pi}}=\frac{e^{\pi}+1}{e^{\pi}-1}.\end{aligned}$$

\end{example}
\begin{definition}[פונקציית הביסייד]
מוגדר בצורה הבאה:
$$ \Theta\left(x\right)=\begin{cases}0&\quad x<0\\\frac{1}{2}&\quad x=0\\1&\quad x>0\end{cases}$$
כלומר זוהי סוג של מדרגה. 
\textbf{פונקציה יוצרת:}$$ \Theta(x)=\lim\limits_{m\to\infty}\eta_m(x)=\lim\limits_{m\to\infty}\left(\frac{1}{2}+\frac{\arctan(mx)}{\pi}\right)$$

\end{definition}
נשים לב כי זוהי פונקצייה(כלומר לא רק פונקציה מוכללת כמו דלתא) אך יש לה קשר חשוב עם פונקציית דלתא.

\begin{proposition}[יחס עם פונקציית דלתא]
$$\frac{d}{dx}\Theta(x)=\delta(x) \iff \int\delta(x)\mathrm{d}x=\Theta(x)$$

\end{proposition}
\begin{proposition}[יחס עם פונקציית מלבן]
$$\mathrm{rect}\left( x \right)=\Theta\left( x+\frac{1}{2} \right)-\Theta\left( x-\frac{1}{2} \right)$$

\end{proposition}
\begin{definition}[פונקציית דלתא תלת מימדית]
$$\delta^{3}(\mathbf{r})=\operatorname*{lim}_{\epsilon\to0}\left({\frac{-1}{4\pi}}\nabla^{2}\left({\frac{1}{r+\epsilon}}\right)\right)$$

\end{definition}
\section{התמרת לפלס}

\begin{definition}[התמרה]
מקבל פונקציה ומחזירה פונקציה.

\end{definition}
\begin{reminder}[אינטגרל לא אמיתי]
$$\int_{0}^{\infty} f(x) \, dx =\lim_{ t \to \infty } \int_{0}^{t} f(x) \, dx $$

\end{reminder}
\begin{definition}[התמרת לפלס]
אם \(f\) רציפה למקוטעין נגדיר:$$\mathcal{L}(f)=F(s)=\int_0^\infty f(t)e^{-st} \; dt$$
כאשר \(f\) כך שהאניטגרל מתכנס.

\end{definition}
\begin{proposition}
ההתמרת לפלס מתכנסת עבור \(s_{0}\) כל עוד קיימים \(k,T \in \mathbb{R}\) כך שלכל \(t>T\) מתקיים:
$$\lvert f(t) \rvert \leq k ^{s_{0}t}$$

\end{proposition}
\begin{example}[התמרת לפלס של 1]
לפי ההגדרה:
$${\mathcal{L}}[1]=\int_{0}^{\infty}e^{-s t}\,d t=\operatorname*{lim}_{N\to\infty}\int_{0}^{N}e^{-s t}\,d t.$$
כאשר נשים לב כי עבור \(s\leq 0\) האינטגרל מתבדר. עבור \(s>0\) נקבל:
$$\operatorname*{lim}_{N\to\infty}\int_{0}^{N}e^{-s t}\,d t=\operatorname*{lim}_{N\to\infty}-{\frac{1}{s}}\,e^{-s t}{\Big|}_{0}^{N}=\operatorname*{lim}_{N\to\infty}-{\frac{1}{s}}\,(e^{-s N}-1)=\frac{1}{s}$$
ולכן ההתמרת לפלס של 1 עבור \(s>0\) תהיה \(\frac{1}{s}\).

\end{example}
\begin{example}[התמרה של אקספוננט]
מההגדרה:
$${\mathcal{L}}[e^{a t}]=\int_{0}^{\infty}e^{-s t}(e^{a t})\,d t=\int_{0}^{\infty}e^{-(s-a)t}\,d t.$$
כאשר נשים לב כי עבור \(s\leq a\) האינטגרל מתבדר. עבור \(s>a\) נקבל:
\begin{gather*}{\mathcal{L}}[e^{a t}]=\operatorname*{lim}_{N\to\infty}\int_{0}^{N}e^{-(s-a)t}\,d t=\\{{=\operatorname*{lim}_{N\to\infty}\left[\frac{(-1)}{(s-a)}\,e^{-(s-a)t}\,{\bigg|}_{0}^{N}\right]}}=\\ {{=\operatorname*{lim}_{N\to\infty}\left[\frac{(-1)}{(s-a)}\,(e^{-(s-a)N}-1)\right]}}= \frac{1}{s-a}\\\end{gather*}
כלומר ההתמרה של \(e^{at}\) עבור \(s> a\) תהיה \(\frac{1}{s-a}\).

\end{example}
\begin{proposition}[לינאריות]
אם \(\mathcal{L}f,\mathcal{L}g\) קיימים, אזי לכל \(a,b \in \mathbb{R}\) מתקיים:
$${\mathcal{L}}[a f+b g]=a{\mathcal{L}}[f]+b{\mathcal{L}}[g]$$

\end{proposition}
\begin{proof}
$$\mathcal{L}[af+bg]=\int_{0}^{\infty}e^{-st}\big{[}af(t)+bg(t)\big{]}\,dt$$$$=a\int_{0}^{\infty}e^{-st}f(t)\,dt+b\int_{0}^{\infty}e^{-st}g(t)\,dt$$$$=a\,\mathcal{L}[f]+b\,\mathcal{L}[g].$$\textbf{דוגמא}$$\mathcal{L}[3t^{2}+5\cos(4t)]=3\,\mathcal{L}[t^{2}]+5\,\mathcal{L}[\cos(4t)]$$$$=3\Big{(}\frac{2}{s^{3}}\Big{)}+5\Big{(}\frac{s}{s^{2}+4^{2}}\Big{)},\qquad s>0\,$$$$=\frac{6}{s^{3}}+\frac{5s}{s^{2}+4^{2}}.$$
ולכן:
$${\mathcal{L}}[3t^{2}+5\cos(4t)]={\frac{5s^{4}+6s^{2}+96}{s^{3}(s^{2}+16)}},\qquad s>0$$

\end{proof}
\begin{proposition}[התמרה של נגזרת]
$${\mathcal{L}}[f^{\prime}]=s\,{\mathcal{L}}[f]-f(0)$$

\end{proposition}
\begin{proof}
ראשית מתקיים:
$$\mathcal{L}(f'(t))=0-f(0)+s\mathcal{L}(f(t))=s\mathcal{L}f(t)-f(0)$$$${\mathcal{L}}[f^{(n)}]=s^{n}\,{\mathcal{L}}[f]-s^{(n-1)}\,f(0)-\cdots-f^{(n-1)}(0)$$$$-\frac{\mathrm{d}}{ds}(\mathcal{L}f) = \frac{dF}{ds}=-\mathcal{L}(tf(t))$$$$\int_{0}^{N}e^{-s t}f^{\prime}(t)\,d t=\Big[\Big(e^{-s t}f(t)\Big)\Big|_{0}^{N}-\int_{0}^{N}(-s)e^{-s t}f(t)\,d t\Big]$$$$=e^{-s N}f(N)-f(0)+s\int_{0}^{N}e^{-s t}f(t)\,d t.$$כעת אם ניקח את הגבול \(N\to \infty\) נקבל:\textbf{מסקנה}ניתן לבצע בצורה חוזרת על נגזרות גבוהות ולקבל:\textbf{טענה} נגזרת של התמרה
$$\mathcal{L}[t\,f(t)]=\int_{0}^{\infty}e^{-st}\,t\,f(t)\,dt=\int_{0}^{\infty}\frac{d}{ds}\left(-e^{-st}\right)f(t)\,dt=$$$$=-\frac{d}{ds}\int_{0}^{\infty}e^{-st}\,f(t)\,dt=-\frac{d}{ds}\mathcal{L}[f(t)]=-F^{\prime}(s)$$

\end{proof}
\begin{corollary}
עבור נגזרות מסדר גבוה נקבל:
$${\mathcal{L}}\left[ t^{n}\,f(t) \right]=(-1)^{n}F^{(n)}(s)=(-1)^{n}\frac{\mathrm{d} ^{n}}{\mathrm{d} s ^{n}} F =(-1)^{n}\frac{\mathrm{d} ^{n}}{\mathrm{d} s ^{n}} \left( \mathcal{L}f \right) $$

\end{corollary}
\begin{proposition}[התמרת לפלס של אינטגרל]
$${\mathcal{L}}\left[\int_{0}^{t}f(u)\,d u\right]={\frac{1}{s}}{\mathcal{L}}\left[f\right]$$

\end{proposition}
\begin{proof}
$$\mathcal{L}\left[ e^{ ct }f(t) \right]=\mathcal{L}[f(t)](s-c)$$$$\mathcal{L}\left[ \Theta(t-c)f(t-c) \right]=e^{ -cs }\mathcal{L}[f(t)]$$$$\mathcal{L}\left[ \cos(t) \right]=\frac{s}{s^{2}+1}\quad\Rightarrow\quad\begin{cases}\mathcal{L}\left[ \Theta(t-c)\,\cos(t-c) \right]=e^{-c s}\,\frac{s}{s^{2}+1}\\ \mathcal{L}\left[ e^{c t}\cos(t) \right]=\frac{(s-c)}{(s-c)^{2}+1}\end{cases}$$$$\mathcal{L} \left[\int_{0}^{t}f(u)\,du\right]=\int_{0}^{\infty}dt\,e^{-st}\int_{0}^{t}f(u)\,du$$$$=\left[-\frac{1}{s}e^{-st}\int_{0}^{t}f(u)\,du\right]_{0}^{\infty}+\int_{0}^{\infty}\frac{1}{s}e^{-st}f(t)\,dt$$\textbf{טענה} הזזה של התמרהאם \(\mathcal{L}(f(t))\) קיים עם \(s > a\) 1. אזי עבור \(c \in \mathbb{R}\) ו-\(s>a+c\) נקבל:כלומר אם מחשבים התמרה של אקספוננט כפול פונקציה זה יהיה הזזה של הפונקציה.2. עבור \(c\geq 0\) וכן \(s> a+c\) נקבל:כלומר אם מחשבים את ההתמרה של הזזה ימינה זה יהיה אקספוננט כפול ההתמרה.\textbf{דוגמא}\textbf{טענה}ההתמרת לפלס של פונקציית דלתא תהיה:
$$\mathcal{L}[ \delta(t-c) ]=\begin{cases}
e^{ -cs } & c\geq 0 \

0 & c<0 
\end{cases}$$**הוכחה**
$${\mathcal{L}}[\delta(t-c)]=\int_{0}^{\infty}e^{-s t}\,\delta(t-c)\,d t=\begin{cases}e^{ -cs } & c\geq 0 \
$$f(t)={\mathcal{L}}^{-1}\{F(s)\}(t)={\frac{1}{2\pi i}}\operatorname*{lim}_{T\to\infty}\int_{\gamma-i T}^{\gamma+i T}e^{s t}F(s)\,d s$$$$\sum_{i=0}^{N} a_{n}x^{n}=F(x)$$$$\int_{0}^{\infty} f(t)x^{n} \;\mathrm{d} t=F(x) $$$$x=e^{ \ln x }\implies x^{t}=\left( e^{ \ln x } \right)^{t}$$$$\int_{0}^{\infty} f(t)e^{ -st } \, dt =F(x)=F\left( e^{ -s } \right)=\tilde{F}(s)$$

\end{proof}
\section{לפלס - פתרון משוואות וקונבולוציה}

\begin{proposition}[פתרון משוואות דיפרנציאליות]
ניתן לפתור משוואה דיפרנציאלית בעזרת התמרת לפלס לפי השלבים הבאים:

  \begin{enumerate}
    \item מבצעים התמרת לפלס של שתי צדדי המשוואה. 


    \item משתמשים בזהות של הנגזרת כדי לקבל משוואה ללא נגזרות. 


    \item יש לנו משוואה אלגברית רגילה - נפתור אותה! 


    \item נמצא את ההתמרה ההפוכה של כל אחד מהפתרונות. אלו יהיו הפתרון של המשוואה הדיפרנציאלית. 


  \end{enumerate}
\end{proposition}
\begin{example}
נפתור את המשוואה:
$$y^{\prime\prime}-4y^{\prime}+4y=3\,e^{t},\qquad y(0)=0,\qquad y^{\prime}(0)=0.$$
מלינאריות:
$${\mathcal{L}}[y^{\prime\prime}-4y^{\prime}+4y]={\mathcal{L}}[3\,e^{t}]=3\left({\frac{1}{s-1}}\right)$$$$\left[s^{2}\,{\mathcal{L}}[y]-s\,y(0)-y^{\prime}(0)\right]-4\left[s\,{\mathcal{L}}[y]-y(0)\right]+4\,{\mathcal{L}}[y]={\frac{3}{s-1}}$$
כאשר נציב את תהאי ההתחלה ונקבל:
$$\left(s^{2}-4s+4\right){\mathcal{L}}[y]={\frac{3}{s-1}}\implies{\mathcal{L}}[y]={\frac{3}{(s-1)(s^{2}-4s+4)}}$$
כאשר בעזרת שברים חלקיים ניתן לקבל:
$${\mathcal{L}}[y]={\frac{3}{s-1}}-{\frac{3}{(s-2)}}+{\frac{3}{(s-2)^{2}}}$$
כאשר על ידי שימוש בטבלה נקבל:
$${\mathcal{L}}[y]=3\,{\mathcal{L}}[e^{t}]-3\,{\mathcal{L}}[e^{2t}]+3\,{\mathcal{L}}[t\,e^{2t}]={\mathcal{L}}\big[3\,(e^{t}-e^{2t}+t\,e^{2t})\big]$$
ולכן:
$$y(t)=3\,(e^{t}-\,e^{2t}+t\,e^{2t}).$$

\end{example}
\begin{definition}[קונבולוציה]
$$f * g = \int_0^t f(x) \cdot g(t-x) \; dx$$

\end{definition}
\begin{proposition}[תכונות של קונבולוציה]
  \begin{enumerate}
    \item קומוטטיביות - \(f * g = g * f\). 


    \item פילוגי - \(f * (g + h) = f * g + f * h\). 


    \item אסוצייטבי - \(f*(g*h)=(f*g)*h\). 


    \item איבר זהות - \(f * \delta = f\). 


    \item איבר ניטרלי - \(f * 0 = 0\). 


  \end{enumerate}
\end{proposition}
\begin{proposition}[התמרת לפלס של קונבולוציה]
$$\mathcal{L}(f * g) = \mathcal{L}(f) \cdot \mathcal{L}(g)$$

\end{proposition}
\begin{proof}
\begin{gather*}{{{\mathcal{L}}[f]\,{\mathcal{L}}[g]=\int_{0}^{\infty}g(\tilde{t})\Big(\int_{\tilde{t}}^{\infty}e^{-s\tau}f(\tau-\tilde{t})\,d\tau\Big)\,d\tilde{t}}}\\ {{=\int_{0}^{\infty}\int_{\tilde{t}}^{\infty}e^{-s\tau}\,g(\tilde{t})\,f(\tau-\tilde{t})\,d\tau\,d\tilde{t}.}}\end{gather*}$${\mathcal{L}}[f]\,{\mathcal{L}}[g]=\int_{0}^{\infty}\int_{0}^{\tau}e^{-s\tau}\,g(\tilde{t})\,f(\tau-\tilde{t})\,d\tilde{t}\,d\tau.$$$$\mathcal{L}[f]\,\mathcal{L}[g]=\left[\int_{0}^{\infty}e^{-s t}f(t)\,d t\right]\left[\int_{0}^{\infty}e^{-s\tilde{t}}g(\tilde{t})\,d\tilde{t}\right]$$$$=\int_{0}^{\infty}e^{-s\tilde{t}}g(\tilde{t})\Big{(}\int_{0}^{\infty}e^{-s t}f(t)\,d t\Big{)}\,d\tilde{t}$$$$=\int_{0}^{\infty}g(\tilde{t})\Big{(}\int_{0}^{\infty}e^{-s(t+\tilde{t})}f(t)\,d t\Big{)}\,d\tilde{t},$$כעת נגדיר החלפת משתנה \(\tau=t+\tilde{t}\) ונקבל \(d\tau=dt\) ולכן:ניתן כעת להחליף את הסדר אינטגרציה ולקבל:וכעת:
$$\mathcal{L}[f]\,\mathcal{L}[g]=\int_{0}^{\infty}e^{-s\tau}\,\left(\int_{0}^{\tau}\,g(\hat{t})\,f(\tau-\hat{t})\,d\hat{t}\right)d\tau$$$$=\int_{0}^{\infty}e^{-s\tau}(g*f)(\tau)\,dt$$$$=\mathcal{L}[g*f]\quad\Rightarrow\quad\mathcal{L}[f]\,\mathcal{L}[g]=\mathcal{L}[f*g].$$

\end{proof}
\begin{proposition}[פירוק משוואה עם קינבולוציה]
נסתכל על בעיית התנאי התחלה:
$$y^{\prime\prime}+a_{1}\,y^{\prime}+a_{0}\,y=g(t),\quad y(0)=y_{0},\quad y^{\prime}(0)=y_{1},$$
ניתן לפרק את המשוואה בצורה הבאה קונבולוציה:
$$y(t)=y_{h}(t)+\left( y_{\delta}*g \right)(t)=y_{h}(t)+\int_{0}^{t}y_{\delta}\left( \tau \right)g\left( t-\tau \right)d\tau$$
כאשר \(y_{h}\) זה הפתרון ההומגני, אשר יהיה תנאי התחלה של המשוואה:
$$y_{h}^{\prime\prime}+a_{1}\,y_{h}^{\prime}+a_{0}\,y_{h}=0,\quad y_{h}(0)=y_{0},\quad y_{h}^{\prime}(0)=y_{1},$$
והפתרון \(y_{\delta}\) יהיה הפתרון המאולץ הרגעי, כלומר הפתרון של המשוואה:
$$y_{\delta}^{\prime\prime}+a_{1}\,y_{\delta}^{\prime}+a_{0}\,y_{\delta}=\delta(t),\quad y_{\delta}(0)=0,\quad y_{\delta}^{\prime}(0)=0.$$

\end{proposition}
\begin{proof}
ניתן להפעיל את התמרת לפלס על המשוואה הדיפרנציאלית ולקבל:
$${\mathcal{L}}[y^{\prime\prime}]+a_{1}\,{\mathcal{L}}[y^{\prime}]+a_{0}\,{\mathcal{L}}[y]={\mathcal{L}}[g(t)]$$
כאשר נשתמש בטענה של הנגזרת:
$${\mathcal{L}}[y^{\prime\prime}]=s^{2}\,{\mathcal{L}}[y]-s y_{0}-y_{1},\qquad{\mathcal{L}}[y^{\prime}]=s\,{\mathcal{L}}[y]-y_{0}.$$
ונקבל את המשוואה האלגברית:
$$\left(s^{2}+a_{1}s+a_{0}\right){\mathcal{L}}[y]-s y_{0}-y_{1}-a_{1}y_{0}={\mathcal{L}}[g(t)].$$
כאשר ניתן לבודד את \(\mathcal{L}y\) ולקבל:
$${\mathcal{L}}[y]={\frac{(s+a_{1})y_{0}+y_{1}}{(s^{2}+a_{1}s+a_{0})}}+{\frac{1}{(s^{2}+a_{1}s+a_{0})}}\,{\mathcal{L}}[g(t)].$$
כאשר נשים לב כי הגורם הראשון יפתור את המשוואה ההומגנית(כאשר \(g(t)=0\)):
$${\mathcal{L}}[y_{h}]={\frac{(s+a_{1})y_{0}+y_{1}}{(s^{2}+a_{1}s+a_{0})}}$$
וכן עבור \(g(t)=\delta\) כאשר תנאי ההתחלה \(y_{0},y_{1}=0\) נקבל את האיבר השני ללא \(\mathcal{L}[g(t)]\)$${\mathcal{L}}[y_{\delta}]={\frac{1}{(s^{2}+a_{1}s+a_{0})}}$$
ולכן קיבלנו סה"כ כי:
$${\mathcal{L}}[y]={\mathcal{L}}[y_{h}]+{\mathcal{L}}\left[ y_{\delta} \right]\,{\mathcal{L}}[g(t)]\implies y(t)=y_{h}(t)+{\mathcal{L}}^{-1}\left[{\mathcal{L}}[y_{\delta}]\;{\mathcal{L}}[g(t)]\right]$$
כאשר קיבלנו סה"כ כי:
$$y(t)=y_{h}(t)+(y_{\delta}*g)(t)$$

\end{proof}
היתרון הגדול בטענה הזו היא שמאפשרת למצוא הביטוי עבור תנאי התחלה כללי. 

\begin{proposition}
עבור משוואה דיפרנציאלית עם מקדמים קבועים נקבל:
$$y_{\delta}=\mathcal{L}^{-1}\left[ \frac{e^{ -cs }}{p(s)} \right] \iff \mathcal{L}\left[ y_{\delta} \right]=\frac{e^{ -cs }}{p(s)}$$
כאשר \(p(s)\) זה הפולינום האופייני כפונקציה של \(s\).

\end{proposition}
\begin{example}
ננסה למצוא ביטוי לבעיית תנאי התחלה הבאה:
$$y^{\prime\prime}+2\,y^{\prime}+2\,y=g(t),\quad y(0)=1,\quad y^{\prime}(0)=-1.$$
נמצא את \(y_{\delta}(t)\). הפלינום האופייני כפונקציה של \(s\) יהיה:
$$p(s)=s ^{2}+2s+2= (s+1)^{2}+1$$
ולכן נקבל מהטענה:
$$y_{\delta}(t)=\mathcal{L}^{-1}[(s+1)^{2}+1]=e^{ -t }\sin(t)$$
כעת נמצא את הפתרון ההומוגני:
$$\mathcal{L}\left[ y''_{h}+2y'_{h}+2 \right]=\mathcal{L}\left[ y_{h}^{\prime\prime} \right]+2\,\mathcal{L}\left[ y_{h}^{\prime} \right]+2\,\mathcal{L}[y_{h}]=0$$
כאשר מהנוסחא לנגזרת נקבל:
$$\left(s^{2}\,{\mathcal{L}}[y_{h}]-s\,y_{h}(0)-y_{h}^{\prime}(0)\right)+2\big({\mathcal{L}}[y_{h}^{\prime}]=s\,{\mathcal{L}}[y_{h}]-y_{h}(0)\big)+2{\mathcal{L}}[y_{h}]=0,$$
לאחר הצבת תנאי התחלה והעברת אגפים נקבל:
$${\mathcal{L}}[y_{h}]={\frac{(s+1)}{(s^{2}+2s+2)}}={\frac{(s+1)}{(s+1)^{2}+1}}$$
וזוהי התמרה של קוסינוס מוזז:
$$y_{h}(t)={\mathcal{L}}\Big[e^{-t}\,\cos(t)\Big]$$
ולכן נקבל כי הפתרון הכללי למשוואה תהיה:
$$y(t)=y_{h}(t)+\left( y_{\delta}*g \right)(t)\implies y(t)=e^{-t}\,\cos(t)+\int_{0}^{t}e^{-\tau}\,\sin\left( \tau \right)\,g\left( t-\tau \right)\,d\tau$$

\end{example}
\section{התמרת פורייה}

\begin{theorem}[התמרת פורייה]
אם מתקיים:

  \begin{enumerate}
    \item הפונקציה \(f(t)\) היא רציפה למקוטעין ב-\(-\infty<t<\infty\)


    \item מתכנסת לפי נורמת ממוצע, כלומר: 
$$f(t)=\operatorname*{lim}_{\delta\to0}\frac{1}{2}\bigl[f(t+\delta)+f(t-\delta)\bigr]$$


    \item אינטגרל לא אמיתי חסום, כלומר: 
$$\int_{-\infty}^{\infty}\left|f(t)\right|d t<\infty$$
&אזי מתקיים:
$$f(t)=\frac{1}{2\pi}\int\limits_{-\infty}^{\infty}\int\limits_{-\infty}^{\infty}f(z)e^{\pm i y(z-t)}d z d y$$


  \end{enumerate}
\end{theorem}
\begin{proposition}
$$F\left( \omega \right)\!=\!\frac{1}{\sqrt{2\pi}}\int_{-\infty}^\infty f(t)e^{-i\omega t}dt\;\;\; f(t)\!=\!\frac{1}{\sqrt{2\pi}}\int_{-\infty}^\infty F\left( \omega \right)e^{i\omega t}d\omega$$

\end{proposition}
\section{טבלת התמרות}

\begin{table}[htbp]
  \centering
  \begin{tabular}{|cc|}
    \hline
    פונקציה & התמרה \\ \hline
    \(a\cdot f(x)+b\cdot g(x)\) & \(a\hat{f}(k)+b\hat{g}(k)\) \\ \hline
    \(f(x-a)\) & \(e^{ika}\cdot \hat{f}(k)\) \\ \hline
    \(f(ax)\) & \(\frac{1}{a}\hat{f}\left( \frac{k}{a} \right)\) \\ \hline
    \(\delta(x-a)\) & \(\frac{1}{\sqrt{2\pi}} e^{iat}\) \\ \hline
    \(\sqrt{\frac{\pi}{2}}\left( \frac{1}{i\pi k}+\delta(k)\right)\) & \(\Theta(t)\) \\ \hline
    \((f*g)(x)\) & \(\sqrt{2\pi }\hat{f}(k)\cdot \hat{g}(k)\) \\ \hline
    \(\left( f(x)\cdot g(x) \right)\) & \(\frac{1}{\sqrt{2\pi}}\left( \hat{f}*\hat{g} \right)(k)\) \\ \hline
    \(\mathrm{rect}\left( \frac{x}{a} \right)\) & \(\frac{a}{\sqrt{2\pi}} \mathrm{sinc}{\left(\frac{a\operatorname{k}}{2}\right)}\) \\ \hline
    \(\mathrm{rect}\left( \frac{ax+b}{d} \right)\) & \(\frac{d}{a\sqrt{2\pi}} \mathrm{sinc}{\left(\frac{d\operatorname{k}}{2a}\right)}\cdot e^{-ikb}\) \\ \hline
    \(\mathrm{sinc}(\frac{x}{a})\) & \(\frac{a}{\sqrt{2\pi }} \mathrm{rect}\left( \frac{ak}{2} \right)\) \\ \hline
    \(\sin(ax)\) & \(\sqrt{ \frac{\pi}{2} }i\left[ \delta(k+a)-\delta(k-a)\right]\) \\ \hline
    \(\cos(ax)\) & \(\sqrt{ \frac{\pi}{2} }\left[ \delta(k-a)+\delta(k+a)\right]\) \\ \hline
    \(e^{-\frac{x^2}{2\sigma^2}}\) & \(\sigma e^{-\frac{\sigma^2k^2}{2}}\) \\ \hline
    \(e^{-\alpha\mid t\mid}\) & \(\frac{1}{\sqrt{2\pi}} \frac{2\alpha}{\alpha^2+k^2}\) \\ \hline
    \(f(t)\cos\left( \omega_{0}t \right)\) & \(\frac{1}{2}\left[ \hat{f}\left( \omega-\omega_{0} \right)+\hat{f}\left( \omega+\omega_{0} \right) \right]\) \\ \hline
    \(f(t)\sin\left( \omega_{0} t\right)\) & \(\frac{1}{2}\left[ \hat{f}\left( \omega-\omega_{0} \right)-\hat{f}\left( \omega+\omega_{0} \right) \right]\) \\ \hline
  \end{tabular}
\end{table}
\begin{proposition}[התמרת פורייה בקוטביות]
$$\tilde{f}\left(k_\rho,k_\theta\right)=\frac{1}{2\pi}\int_{-\infty}^\infty\int_{-\infty}^\infty f\left( \rho,\theta \right)e^{-ik_xx-ik_yy}dxdy=\frac{1}{2\pi}\int_0^R\int_0^{2\pi}f\left( \rho,\theta \right)e^{-ik_\rho\rho\cos\left( \theta-k_\theta \right)}\rho d\rho d\theta$$

\end{proposition}
\begin{remark}
פעמים רבות ניתן להשתמש בזהות שימושית:
$$\frac{1}{2\pi}\int_{0}^{2\pi}d\theta e^{\pm ix\cos\left(\theta-\theta'\right)}=J_{0}\left(x\right)\\\int xJ_{0}\left(x\right)dx=xJ_{1}\left(x\right)$$

\end{remark}
\section{פונקציית גרין}

\begin{definition}[פונקציית גרין]
עבור אופרטור לינארי דיפרציאלי \(L=L(x)\) קיים פונקצית גרין \(G(x,s)\) אשר מקיימת:
$$LG(x,s)=\delta(x-s)$$

\end{definition}
\begin{proposition}
בהנתן פונקציית גרין למשוואה לא הומוגנית מהצורה:
$$Lu(x)=f(x)$$
יהיה את הפתרון:
$$u(x)=\int G(x,s)\,f(s)\,d s$$

\end{proposition}
\begin{proof}
ראשית נשים לב כי:
$$\int L G(x,s)\,f(s)\,d s=\int\delta(x-s)\,f(s)\,d s=f(x)$$
כאשר כיוון ש-\(L\) אופרטור דיפרנציאלי לינארי, ניתן להוציא אותו מהאינטגרל ולקבל:
$$L\left(\int G(x,s)\,f(s)\,d s\right)=f(x)$$
ונשים לב שהמשוואה מתקיימת עבור \(u(x)=\int G(x,s)\,f(s)\,d s\).

\end{proof}
\section{קשר לשטרום ליוביל}

אנו יודעים כי עבור אופטור צמוד לעצמו נקבל פונקציות עצמיות מקיימות $${\mathcal L}y_{n}(x)=\lambda_{n}\rho(x)y_{n}(x),$$
כאשר \(\mathcal{L}\) אופרטור צמוד לעצמו תחת תנאי שפה כלשהם, והפונקציות העצמיות \(y_{n}\) אורתוגונליות אחד לשני. כעת נניח כי אנחנו רוצים לפתור את המשוואה הלא הומוגנית \(\mathcal{L}y=f(x)\) תחת התנאי השפה של \(\mathcal{L}\). אז ניתן לכתוב:
$$y(x)=\sum_{n=0}^{\infty}c_{n}y_{n}(x),$$
כאשר מלינאריות נקבל:
$$f(x)={\mathcal{L}}y(x)={\mathcal{L}}\left(\sum_{n=0}^{\infty}c_{n}y_{n}(x)\right)=\sum_{n=0}^{\infty}c_{n}{\mathcal{L}}y_{n}(x)=\sum_{n=0}^{\infty}c_{n}\lambda_{n}\rho(x)y_{n}(x).$$
נכפיל את שתי האגפים ב-\(y_{j}\) פונקציה עצמית אחרת ונטגרל על שני האגפים:
$$\int_{a}^{b}y_{j}^{\ast}(z)f(z)\,d z=\sum_{n=0}^{\infty}\int_{a}^{b}c_{n}\lambda_{n}y_{j}^{\ast}(z)y_{n}(z)\rho(z)\,d z,$$
כאשר האינטגרל על האגף הימיני הוא אפס כל עוד \(j\neq n\)(מאורתוגנאליות). לכן נקבל:
$$c_{n}=\frac{1}{\lambda_{n}}\frac{\int_{a}^{b}y_{n}^{*}(z)f(z)\,d z}{\int_{a}^{b}y_{n}^{*}(z)y_{n}(z)\rho(z)\,d z}\implies y(x)=\sum_{n=0}^{\infty}\frac{1}{\lambda_{n}}\frac{\int_{a}^{b}y_{n}^{\ast}(z)f(z)\,d z}{\int_{a}^{b}y_{n}^{\ast}(z)y_{n}(z)\rho(z)\,d z}\,y_{n}(x).$$
אם ננרמל את הפונקציות העצמיות כך ש:
$$\int_{a}^{b}{\hat{y}}_{n}^{*}(z){\hat{y}}_{n}(z)\rho(z)\,d z=1$$
ונחליף את הסדר של סכימה ואינטגרציה נקבל:
$$y(x)=\int_{a}^{b}\left\{\sum_{n=0}^{\infty}\left[{\frac{1}{\lambda_{n}}}{\hat{y}}_{n}(x){\hat{y}}_{n}^{*}(z)\right]\right\}f(z)\,d z.$$
כאשר הביטוי בתוך הסוגריים נקרא פונקציית גרין, ומסומן \(G(x,z)\). בעזרת סימון זה נקבל:
$$y(x)=\int_{a}^{b}G(x,z)f(z)\,d z \qquad  G(x,z)=\sum_{n=0}^{\infty}\frac{1}{\lambda_{n}}\hat{y}_{n}(x)\hat{y}_{n}^{*}(z).$$
כאשר נשים לב כי פונקציית גרין פונקציית גרין תלוייה בפונקציות העצמיות ובתחום, ו-\(f(z)\) תלוי באיבר האי הומגני, כמו כן נשים לב כי מתקיים:
$$G(x,z)=G^{*}(z,x)$$

\begin{example}
נמצא את הפונקציית גרין של המשוואה $$y^{\prime\prime}+{\textstyle{\frac{1}{4}}}y=f(x),$$
עם התנאי שפה \(y(0)=y\left( \pi \right)=0\)
פתרון:
נשים לב כי האופרטור האופרטור המוגדר \(\mathcal{L}y=y''+\frac{1}{4}y\) הוא צמוד לעצמו. נמצא ע"ע. כלומר נדרש לפתור את המשוואה \(y''+\frac{1}{4}y=\lambda y\) ונקבל:
$$y(x)=A\sin\left(\sqrt{{\frac{1}{4}}-\lambda}\right)x+B\cos\left(\sqrt{{\frac{1}{4}}-\lambda}\right)x.$$
ומהתנאי שפה נקבל:
$$\sqrt{{\textstyle{\frac{1}{4}}}-\lambda}=n \implies y_{n}(x)=A_{n}\sin(nx)\quad \lambda_{n}=\frac{1}{4}-n^2$$
כאשר מהתנאי על הנרמול נקבל:
$$\int_{0}^{\pi}A_{n}^{2}\sin^{2}n x\,d x=1~~~~\Rightarrow~~~~~A_{n}=\left({\frac{2}{\pi}}\right)^{1/2}.$$
ולכן נקבל כעת כי מהפיתוח לעיל פונקציית הגרין תהיה:
$$G(x,z)={\frac{2}{\pi}}\sum_{n=0}^{\infty}{\frac{\sin n x\sin n z}{{\frac{1}{4}}-n^{2}}}.$$

\end{example}
\begin{proposition}
ניתן להכלל לפונקציות מהצורה:
$${\mathcal L}y_{n}(x)=\lambda_{n}\rho(x)y_{n}(x).$$
כאשר במקרה זה פונקציית הגרין תהיה:
$$G(x,z)=\sum_{n=0}^{\infty}{\frac{{\hat{y}}_{n}(x){\hat{y}}_{n}^{*}(z)}{\lambda_{n}-\mu}}.$$

\end{proposition}
\begin{corollary}
$$\frac{1}{2\pi i}\oint_{C} G(x,z)d\lambda=-\frac{\delta(x-z)}{r(z)}$$
כאשר \(C\) היא מספילה סגורה במישור המרוכב אשר מכילה את כל הנקודות הסינגולאריות של \(G(x,z)\).

\end{corollary}
\chapter{משוואות דיפרנציאליות חלקיות}

\section{סיווג משוואות דיפרנציאליות חלקיות}

\begin{definition}[מד"ח]
משוואה דיפרנציאלית חלקית. זוהי משוואה שמכילה פונקציה(שיתכן ומרובת משתנים) והנגזרות החלקיות שלה.

\end{definition}
\begin{definition}[סדר של מד"ח]
המספר המתאים לסדר של הנגזרת החלקית מהמעלה הגבוהה ביותר של המשוואה.

\end{definition}
\begin{definition}[מספר המשתנים הבלתי תלויים]
כמות המשתנים שגוזרים לפיהם במשוואה.

\end{definition}
\begin{example}
המשוואה:
$$\frac{\partial^{2}f}{\partial x^{2}} =\frac{\partial f}{\partial t} $$
תהיה משוואה מסדר 2 עם שתי משתנים בלתי תלויים(\(x\) ו-\(t\)).
המשוואה:
$$\frac{\partial^{3}f}{\partial^{2}x\partial y} =\frac{\partial f}{\partial t} $$
תהיה משוואה מסדר 3 עם שלושה משתנים בלתי תלויים \(x,y,t\).

\end{example}
\begin{definition}[משתנים תלויים]
פונקציה שהאיברים של המשוואה תלויים בהם. המשוואה מופיעה עם הנגזרות שלו. נסמן בדרך כלל ב-\(f\) או \(u\).

\end{definition}
\begin{definition}[מד"ח לינארי]
משוואה מהצורה \(\mathcal{L}f=g\) נקראת לינארית אם האופרטור הדיפרנציאלי \(\mathcal{L}\) הוא אופרטור לינארית. כלומר מקיים:
$$\mathcal{L}(f_{1}+f_{2})=\mathcal{L}(f_{1})+\mathcal{L}(f_{2})\quad\mathrm{and}\quad \mathcal{L}(c f)=c \mathcal{L}(f)$$

\end{definition}
\begin{corollary}
פונקציה היא לינארית אם הפונקציה התלוייה \(f\) מופיעה באופן לינארי(אם למשל יש גורם \(f^{2},\left( \frac{\partial f}{\partial x} \right)^{2}\) או \(f\cdot \frac{\partial f}{\partial x}\) אז לא לינארי). דרך זו בדרך כלל משומשת כי בדרך כלל מידי לראות כך אם המשוואה לינארית או לא.

\end{corollary}
\begin{example}
נסתכל על המשוואה
$$\frac{\partial^{2} f}{\partial x^{2}}=\frac{\partial f}{\partial t} \tan t $$
האופרטור:
$$L(f)={\frac{\partial^{2}f}{\partial x^{2}}}-{\frac{\partial f}{\partial t}}\tan t.$$
מקיים \(\mathcal{L}(f_{1}+cf_{2})=\mathcal{L}f_{1}+c\mathcal{L}f_{2}\) ולכן לינארי. ניתן גם לראות מידית כי \(f\) מופיע תמיד באופן לינארי.

\end{example}
\begin{definition}[מד"ח הומוגני]
כל האיברים במשוואה תלויים בפונקציה \(f\) או בנגזרות שלה. כלומר אין גורם חופשי.

\end{definition}
\begin{definition}[מד"ח עם מקדמים קבועים]
המקדמים של הפונקציה ושל הנגזרות במשוואה יהיו מספרים קבועים.

\end{definition}
\begin{definition}[מד"ח עם מקדמים משתנים]
המקדמים של הפונקציה ושל הנגזרות במשוואה יהיו פונקציה אשר לאו דווקא קבועות.

\end{definition}
\section{תנאי שפה ותנאי התחלה}

\section{תנאי שפה ותנאי התחלה}

\begin{definition}[תנאי שפה של מדחים]
אם אנחנו מועניינים לפתור משוואה דיפרנציאלית חלקית על תחום כלשהו \(\Omega\). ניתן להביא את תנאי שפה הבאים
\textbf{תנאי דיריכלה} - כאשר ניתן ערך הפונקציה על המסילה, כלומר \(u|_{\partial\Omega}\).
\textbf{תנאי נוימן} - כאשר ניתן ערך הנגזרת על המסילה, כלומר \(\dot{u}|_{\partial\Omega}\)\textbf{תנאי קושי} - זוהי בעיית תנאי התחלה, כאשר נתון הערך של הפונקציה והערך של הנגזרת בזמן \(t=0\). תמיד קיים פתרון, ויחיד.

\end{definition}
\begin{remark}
באופן כללי נצטרך שכמות "האילוצים" יהיה שווה לסדר של המשוואה.

\end{remark}
\begin{definition}[תנאי שפה הומגניים]
תנאי שפה אשר תלוים רק בנגזרות ובערך של הפונקציה.

\end{definition}
\begin{example}
עבור קבועים \(\alpha,\beta,\gamma,\delta\) תנאי שפה מהצורה:
$$\begin{cases}\alpha  \frac{\partial u}{\partial x} |_{x=0}+\beta u|_{x=0}=0 \\\gamma \frac{\partial u}{\partial x} |_{x=L}+\delta u|_{x=L}=0
\end{cases}$$
יהיו תנאי שפה הומגניים.

\end{example}
\begin{definition}[משוואה פרידה]
משוואה אשר עבור פתרונה \(u\) קיימים פונקציות \(X,T\) כך ש:
$$u\left( x_{1},\dots,x_{n},t_{1},\dots,t_{m} \right)=X\left( x_{1},\dots,x_{n} \right)T\left( t_{1},\dots,t_{m} \right)$$

\end{definition}
\begin{remark}
בבעיות פיזיקליות בדרך כלל עדיף להעביר את הקבועים לחלק הזמני.

\end{remark}
\begin{definition}[משוואה פרידה לחלוטין]
משוואה אשר הפתרון שלה מתפרק לאוסף משוואות עם משתנה אחד. כלומר:
$$u\left( x_{1},\dots,x_{n} \right)=X_{1}(x_{1})X_{2}(x_{2})\dots X_{n}(x_{n})$$

\end{definition}
\begin{proposition}
ניתן לפתור בעזרת הפרדת משתנים כאשר:

  \begin{enumerate}
    \item המדח הוא לינארי והומגני. 


    \item התנאי שפה הם לינארים והומוגניים. 
זאת כיוון שאנו רוצים שכל משוואה תהיה בעיית שטרום ליוביל אשר דורשת לינאריות והומוגניות של התנאי שפה.


  \end{enumerate}
\end{proposition}
\begin{proposition}
אם \(L_{x}\) ו-\(L_{T}\) הם אופרטור דיפרנציאלים של \(x\) ו-\(t\) בהתאמה המקיימים \(L_{x}=L_{t}\) ו-\(u\) משוואה פרידה אזי מתקיים:
$$L_{x}=\lambda x\quad L_{t}=\lambda t$$
כאשר בפרט אם \(L_{x},L_{t}\) צמודים לעצמם נקבל בעיית ערכים עצמיים.

\end{proposition}
\begin{proof}
נציב \(u(x,t)=X(x)T(t)\) ונקבל:
$$L_{x}(X(x)T(t))=L_{t}(X(x)T(t))\implies L_{x}(X(x))T(t)=X(x)L_{t}(T(t))$$
כאשר כעת ניתן לחלק ב-\(u(x,t)=X(x)T(t)\) ולקבל:
$$\frac{L_{x}(X(x))}{X(x)}=\frac{L_{t}(T(t))}{T(t)}$$
כאשר משוואה זו מתקיימת עבור כל ערך, ו-\(t,x\) משתנים בלתי תלויים, ולכן שווה לקבוע אשר נסמן \(\lambda\). כעת נקבל:
$$L_{x}=\lambda x\quad L_{t}=\lambda t$$

\end{proof}
\begin{corollary}
אם אופרטור הוא צמוד לעצמו וקיבלנו פתרון בעזרת הפרדת משתנים, מהמשפט הספקטרלי זה יהיה כל הפתרונות.

\end{corollary}
\begin{remark}
באופן כללי אין דרך או אלגוריתם וודאי להכריע האם משוואה היא ספרבילית, אבל אם הפתרון הוא יחיד, ומוצאים פתרון ספרבילי, אז נקבל כי הפתרון היחיד הוא הפתרון הזה.

\end{remark}
\begin{definition}[לפלאסיאן]
סכום הנגזרות השניות:
$$\bar{\nabla}^2 f=\sum_{i}^{n}\frac{\partial^{2} f}{\partial x_{i}^{2}} $$

\end{definition}
\begin{proposition}
אופרטור הלפלסיאן הוא אופרטור דיפרנציאלי צמוד לעצמו תחת המכפלה הפנימית:
$$\langle f,g \rangle = \int\int \dots \int f\cdot g\;$$
כאשר כמות האינטגרלים יהיה שווה למימד המרחב(לדוגמה עבור \(\mathbb{R}^{3}\) זה יהיה תחת מימדי)

\end{proposition}
\section{משוואות מסדר ראשון}

\begin{symbolize}
הנגזרת החלקית של \(u\) לפי משתנה \(x\) יסומן \(u_{x}\).

\end{symbolize}
\begin{definition}[לינארית מסדר ראשון]
משוואה מהצורה:
$$a(x,y){\frac{\partial u}{\partial x}}+b(x,y){\frac{\partial u}{\partial y}}=c(x,y,u)$$

\end{definition}
\begin{definition}[משוואה קווזילינארית מסדר ראשון]
משוואה מהצורה:
$$a(x,y,u){\frac{\partial u}{\partial x}}+b(x,y,u){\frac{\partial u}{\partial y}}=c(x,y,u)$$

\end{definition}
\begin{definition}[שיטת הקרקטריסטיקות]
שיטה אלגברית של למצוא החלפת משתנים כך שהמשוואה הדיפרנציאלית החלקית תהיה תלוייה רק בנגזרת חלקית של משתנה אחד, ולכן יהיה למעשה משוואה דיפרנציאלית רגילה. ניתן להשתמש בשיטה זו עבור כל משוואה קווזילינארית.

\end{definition}
נראה שיטה זו עבור משוואה פשוטה.

\begin{proposition}
עבור משוואת מקדמים קבועים מהצורה:
$$a u_{x}+b u_{y}=0,$$
הפתרון יהיה מהצורה:
$$u(x,\,y)=f(b x-a y)$$

\end{proposition}
\begin{proof}
נבצע החלפת משתנים:
$$x^{\prime}=a x+b y\qquad y^{\prime}=b x-a y.$$
כאשר מכלל השרשרת נקבל:
$$u_{x}={\frac{\partial u}{\partial x}}={\frac{\partial u}{\partial x^{\prime}}}{\frac{\partial x^{\prime}}{\partial x}}+{\frac{\partial u}{\partial y^{\prime}}}{\frac{\partial y^{\prime}}{\partial x}}=a u_{x^{\prime}}+b u_{y^{\prime}}$$$$u_{y}={\frac{\partial u}{\partial y}}={\frac{\partial u}{\partial y^{\prime}}}{\frac{\partial y^{\prime}}{\partial y}}+{\frac{\partial u}{\partial x^{\prime}}}{\frac{\partial x^{\prime}}{\partial y}}=b u_{x^{\prime}}-a u_{y^{\prime}}.$$
ולכן:
$$a u_{x}+b u_{y}=a(a u_{x^{\prime}}+b u_{y^{\prime}})+b(b u_{x^{\prime}}-a u_{y^{\prime}})=(a^{2}+b^{2})u_{x^{\prime}}$$
כאשר כיוון ש-\(a^{2}+b^{2}\neq 0\) נקבל כי הפתרון בקורדינטה \(x'\) יהיה:
$$u_{x'}=0\implies u=f(y')=f(bx-ay)$$

\end{proof}
\begin{remark}
ניתן לתת לפתרון הזה פרשנות גאומטרית. ניתן לכתוב את המשוואה בצורה:
$$\begin{pmatrix}a \\ b\end{pmatrix}\cdot \begin{pmatrix}u_{x}\\u_{y}\end{pmatrix}=0\implies \begin{pmatrix} a \\ b
\end{pmatrix}\bar{\nabla}  u =0$$
וכאשר נשים לב כי זו פשוט ההגדרה של הנגזרת הכיוונית בכיוון של \(\vec{v}=\begin{pmatrix}a \\ b\end{pmatrix}\)(עד כדי נרמול שלא משנה שמשוואים ל-0). ולכן \(u(x,y)\) קבוע בכיוון של \(\vec{v}\). כלומר אוסף הפתרונות יהיה אוסף כל הווקטורים שמקבילים ל-\(\vec{v}\), כאשר כיוון שאנחנו ב-\(\mathbb{R}^{2}\) מימד המרחב המאונך הוא 1, ולכן מספק למצוא ווקטור אחד כזה. נשים לב למשל כי:
$$\begin{pmatrix}a \\b\end{pmatrix}\begin{pmatrix}b \\ -a
\end{pmatrix}=0$$
ולכן אוסף הווקטורים שמאונכים לאנך(כלומר מקבילים ל-\(v\)) יהיו:
$$\left\{  \begin{pmatrix}x\\y\end{pmatrix}\mid \begin{pmatrix}b \\ -a\end{pmatrix}\begin{pmatrix}x\\y\end{pmatrix}=0  \right\}=\left\{  \begin{pmatrix}x\\y
\end{pmatrix}\mid bx-ay=0  \right\}=\left\{  bx-ay=0\mid x,y \in \mathbb{R}  \right\}$$
כאשר נרצה כי הפונקציה תהיה קבועה עבור ישרים אלו, כלומר אם נציב את הביטוי עבור הישר הזה בפונקציה הוא יהיה קבוע לכל הערכים של \(x,y \in \mathbb{R}\) ולכן \(f(bx-ay)=c\).

\end{remark}
\begin{example}
נפתור את המשוואה:
$$4u_{x}-3u_{y}=0$$
עם התנאי \(y(0,y)=y^{3}\). ראינו כי הפתרון יהיה מהצורה:
$$f(-3x-4y)=c$$
נציב כעת את התנאי ונקבל:
$$f(-4y)=c$$
כעת נרצה להגדיר משתנה \(w=-4y\) ולבטא את התנאי בעזרתו:
$$f(w)=\left( \frac{w}{-4} \right)^{3}=-\frac{w^{3}}{64}$$
וכעת מצאנו את \(f\). נציב את הביטוי הידוע עבורו:
$$f(-3x-4y)=- \frac{(-3x-4y)^{3}}{64}=\frac{(3x+4y)^{3}}{64}$$
וניתן גם לוודא שזה אכן פתרון.

\end{example}
\begin{remark}
נשים לב כי למעשה בשיטה האלגברית מצאנו עקומה אשר הופכת את הפתרון לתלוי במשתנה אחד, ולכן בפועל המשוואה תחת משתנה זה תהיה מד"ר, אשר אנו יודעים לפתור.

\end{remark}
\begin{definition}[קרקטריסטיקה]
עקומה(או יריעה) שעליה המד"ח מתנהג בפועל כמו מדר - כלומר מכיל נגזרת חלקית מסוג אחד בלבד.

\end{definition}
\begin{proposition}
אם פונקציה קבועה על מסילה \(\gamma:\mathbb{R}\to \mathbb{R}^{n}\) אז בפרט זו תהיה עקומת קרקטריסטיקה. זאת כיוון שמתקיים \(u=u\left( \vec{\gamma}(t) \right)\) ולכן בפרט \(u\) יהיה תלוי על העקומה רק במשתנה אחד - \(t\).

\end{proposition}
\begin{proposition}
עבור משוואה מסדר ראשון מהצורה:
$$A(x,y)\frac{\partial u}{\partial x}+B(x,y)\frac{\partial u}{\partial y}=F(x,y,u).$$
נקבל כי על העקומה \(\vec{r}:\mathbb{R}\to \mathbb{R}^{2}\) תלוי רק במשתנה אחד כאשר מתקיים:
$$\frac{d y}{d x}=\frac{B(x,y)}{A(x,y)}$$

\end{proposition}
\begin{proof}
נסמן את העקומה ע"י \(\gamma:\mathbb{R}\to \mathbb{R}^{2}\) וכן \(x=x(s),y=y(s)\). מתקיים מכלל השרשרת:
$${\frac{d u}{d s}}={\frac{\partial u}{\partial x}}{\frac{d x}{d s}}+{\frac{\partial u}{\partial y}}{\frac{d y}{d s}}=\frac{\mathrm{d} x}{\mathrm{d} s} u_{x}+\frac{\mathrm{d} y}{\mathrm{d} s} u_{y}=\frac{\mathrm{d} \gamma}{\mathrm{d} s} $$
כעת ניתן לתאר את הבעיה בעזרת מטריצה:
$$\begin{pmatrix}\frac{\partial x}{\partial s}  & \frac{\partial y}{\partial s} \\ A & B\end{pmatrix}\begin{pmatrix}u_{x}\\u_{y}\end{pmatrix}=\begin{pmatrix}\gamma(s)\\F(x,y,u)
\end{pmatrix}$$
כאשר נרצה שיהיה תלוי רק במשתנה אחד, ולכן נדרוש שיהיה ניוון, כלומר שתלוי לינארית ובפרט הדטרמיננטה מתאפסת.
$$\left|\begin{array}{c c}{{d x/d s}}&{{d y/d s}}\\ {{A}}&{{B}}\end{array}\right|=0\implies B\frac{d x}{d s}-A\frac{d y}{d s}=0\implies \frac{d y}{d x}=\frac{B(x,y)}{A(x,y)}$$
כאשר נשים לב כי למעשה הדרישה לניוון נבעה מזה שאם לא היה ניוון, אז לא היה ניתן לייצג את המשוואה בעזרת אחת הנגזרות בלבד, ולכן לא היה מצטמצם למדר.

\end{proof}
\begin{example}
נסתכל על המשוואה:
$$u_{x}+y u_{y}=0$$
עקומת הקרקטריסטיקה תהיה:
$$\frac{d y}{d x}=\frac{y}{1}\implies y=Ce^{ x }\implies ye^{ -x }=C$$
ולכן נקבל כי זה עקומה עליה הפונקציה תהיה קבועה, ולכן הפתרון של המשוואה תהיה:
$$u(x,\,y)=f(e^{-x}y)$$

\end{example}
\section{משוואות מסדר שני}

\begin{definition}[משוואה לינארית מסדר שני]
עבור משוואה מהצורה 
$$ A(x,y)\frac{\partial^2u}{\partial x^2}\!+\!2B(x,y)\frac{\partial^2u}{\partial x\partial y}\!+\!C(x,y)\frac{\partial^2u}{\partial y^2}\!=\!\phi\left(\!x,\!y,\!u,\!\!\frac{\partial u}{\partial x},\!\!\frac{\partial u}{\partial y}\right)$$

\end{definition}
\begin{proposition}
משוואת הקרקטרסטיקה תהיה:
$$\frac{dy}{dx}=\frac{B\pm\sqrt{B^2-AC}}{A}$$
כאשר נקבל שתי פתרונות אשר מייצגות שתי עקומות קרקטריסטיקות שונות.

\end{proposition}
\begin{proof}
כעת נדרש שעקומה תלויה במשתנה יחיד \(s\) בשתי כיוונים שונים. נסתכל גם על הכיוון של המשיק וגם על הכיוון של האנך:
$$\begin{array}{l}{{{\frac{\partial u}{\partial s}}\equiv\nabla u\cdot{\frac{d\mathbf{r}}{d s}}={\frac{\partial u}{\partial x}}{\frac{d x}{d s}}+{\frac{\partial u}{\partial y}}{\frac{d y}{d s}}={\frac{d\phi(s)}{d s}}}}\\ {{{\frac{\partial u}{\partial n}}\equiv\nabla u\cdot{\hat{\mathbf{n}}}={\frac{\partial u}{\partial x}}{\frac{d y}{d s}}-{\frac{\partial u}{\partial y}}{\frac{d x}{d s}}=\psi(s).}}\end{array}$$
בעזרת כלל השרשרת ניתן לכתוב:
$$\begin{array}{l}{{\frac{d}{d s}\left(\frac{\partial u}{\partial x}\right)=\frac{d x}{d s}\frac{\partial^{2}u}{\partial x^{2}}+\frac{d y}{d s}\frac{\partial^{2}u}{\partial x\partial y},}}\\ {{\frac{d}{d s}\left(\frac{\partial u}{\partial y}\right)=\frac{d x}{d s}\frac{\partial^{2}u}{\partial x\partial y}+\frac{d y}{d s}\frac{\partial^{2}u}{\partial y^{2}}.}}\end{array}$$
וכעת נדרוש כי קיים ניוון בין כל המשוואות:
$$\left|\begin{array}{c c c}{{A}}&{{B}}&{{C}}\\ {{\frac{d x}{d s}}}&{{\frac{d y}{d s}}}&{{0}}\\ {{0}}&{{\frac{d x}{d s}}}&{{\frac{d y}{d s}}}\end{array}\right|=0.$$
מהדטרמיננטה נקבל:
$$A\left({\frac{d y}{d s}}\right)^{2}-B\left({\frac{d x}{d s}}\right)\left({\frac{d y}{d s}}\right)+C\left({\frac{d x}{d s}}\right)^{2}=0.$$
כאשר ניתן לכפול ב-\(\left( \frac{ds}{dx} \right)^2\) ונקבל:
$$ \begin{aligned}&A\left(\frac{dy}{dx}\right)^2-2B\left(\frac{dy}{dx}\right)+C=0\\&\frac{dy}{dx}=\frac{B\pm\sqrt{B^2-AC}}{A}\end{aligned}$$

\end{proof}
\begin{definition}[סיווג משוואות]
בביטוי עבור משוואת הקרקטריסטיקה הופיע \(\Delta=\sqrt{B^2-AC}\). דבר זה מוביל להגדרות הבאות:
- כאשר יש שתי פתרונות ממשיים, נקבל \(\Delta>0\) משוואה היפרבולית(משוואת הגל)
- כאשר יש פתרון אחד ממשי - \(\Delta=0\) נקבל משוואה פרבולית(משוואת החום)
- כאשר \(\Delta<0\) נקבל שאין פתרון ממשי, ולכן המשוואה אליפטית(משוואת לפלס)

\end{definition}
\begin{proposition}[סדפ מעבר לצורה קנונית]
  \begin{enumerate}
    \item מוצאים את המשוואת הקרקטריסטיות ע"י פתרון המשוואה \(\frac{dy}{dx}=\frac{ B\pm \sqrt{ B^2-AC } }{A}\). משוואת הקרקטריסטיקה תהיה משוואה של \(x,y\) אשר שווה לקבוע. 


    \item אם לא ממשי, זה יהיה מהצורה \(C=y+\left( a\pm bi \right)x\). ניקח את הפתרון הממשי והמדומה בתור משוואת הקרקריסטיקות 


    \item להגדיר שתי קורדינטות \(\xi,\eta\) ששווה לקרקטריסטיות. 


    \item כעת מתקיים: 
$$U_{x}=U_{\xi}\xi_{x}+U_{\eta}\eta_{x}\qquad U_{y}=U_{\xi}\xi_{y}+U_{\eta}\eta_{y}$$


    \item נציב את \(U_{x},U_{y}\) שמצאנו במשוואה ונקבל משוואה של \(U_{\xi},U_{\eta}\). נפשט וזה יהיה הצורה הקנונית.  


  \end{enumerate}
\end{proposition}
\section{פתרון משוואות לא הומוגניות}

\begin{proposition}[פתרון משוואה לא הומוגנית]
דרך 1 - נפרוש את הפתרון הלא הומוגני ע"י פונקציות עצמיות והשוואה מקדמים. נקבל אוסף של משוואות דיפרנציאליות לינאריות.
דרך 2 - נרצה ראשית למצוא פתרון פרטי \(u^P\) של משוואת פואסון. לאחר מכן נקבל:
$$u^H=u-u^P$$
כאשר \(u^H\) משוואה הומוגנית. נדרש למצוא תנאי שפה עבור \(u^H\), לפתור את הבעיה ההומגונית, ולאחר מכן להציב חזרה \(u=u^H+u^P\).

\end{proposition}
\chapter{משוואת הגלים}

\section{תכונות של משוואת הגלים}

\begin{definition}[משוואת הגלים רב מימדית]
משוואה מהצורה:
$$\bar{\nabla}^2 u-c^{2}\frac{\mathrm{d} ^{2}u}{\mathrm{d} t^{2}} =0$$
כאשר הלפלסיאן הוא נגזרת על הרכיב המרחבי בלבד.

\end{definition}
\begin{proposition}[קוסליות]
\end{proposition}
\section{בעיה חד מימדית}

\begin{definition}[משוואת גלים חד מימדית]
משוואה מהצורה:
$$u_{tt}-cu_{x x}=0$$
כאשר הקבוע \(c\) נקרא מהירות הגל. לעיתים קוראים לזה בעיית המיתר.

\end{definition}
\begin{proposition}
פתרון כללי של משוואת הגלים החד מימדית:
$$u_{t t}=c^{2}u_{x x}\qquad\mathrm{for}\ -\infty<x<+\infty.$$
תהיה:
$$u(x,t)=f(x+c t)+g(x-c t)$$

\end{proposition}
\begin{proof}
ניתן לכתוב:
$$u_{t t}-c^{2}u_{x x}=\bigg({\frac{\partial}{\partial t}}-c{\frac{\partial}{\partial x}}\bigg)\bigg({\frac{\partial}{\partial t}}+c{\frac{\partial}{\partial x}}\bigg)u=0$$
נגדיר \(v=u_{t}+cu_{x}\). כעת ניתן להפוך את המשוואה למערכת של שתי משוואות מסדר ראשון:
\begin{gather*}(i)\quad v_{t}-c v_{x}=0 \\(ii)\quad u_{t}+cu_{x}=v \end{gather}
$$
עבור משוואה \((i)\) נקבל:
$$v(x,t)=h(x+c t)$$
כעת המשוואה השנייה תהיה:
$$u_{t}+c u_{x}=h(x+c t)$$
כאשר אם נבדוק פונקציה מהצורה \(f(x+ct)\) נקבל:
$$cf'(x+ct)+cf'(x+ct)=h(x+ct)\implies f'(x+ct)= \frac{1}{2c} h(x+ct)$$
ולכן פתרון אחד זה פונקציה מהצורה הזו אשר מקיימת בנוסף \(f'(s)=\frac{h(s)}{2c}\). וכן עבור המשוואה ההומגנית \(u_{t}+cu_{x}=0\) נקבל פתרון מהצורה \(g(x-ct)\).  ולכן הפתרון הכללי יהיה הסכום של הפתרון הפרטי \(f(x+ct)\) הפתרון ההומוגני \(g(x-ct)\). כלומר:
$$u(x,t)=f(x+ct)+g(x-ct)$$

\end{proof}
דרך יותר פורמלית להראות זאת זה באמצעות שיטת הקרקטריסטיקות כדי להשיג את המשתנים הטבעיים:

\begin{proof}
משוואת תהיה מהצורה:
$$ \frac{\partial^2y}{\partial x^2}=\frac1{c^2}\frac{\partial^2y}{\partial t^2}$$
כאשר זו משוואה שהטרנספורמצייה הקנונית שלה יהיה:
$$ \begin{aligned}\xi&=x-c t,\\\eta&=x+c t\end{aligned}$$
ולאחר הצבה נקבל את הביטוי המפושט:
$$ \frac{\partial^2y}{\partial\xi\text{ }\partial\eta}=0\implies \frac{\partial }{\partial \xi}\left( \frac{\partial y}{\partial \eta} \right)\implies \frac{\partial y}{\partial\eta}=F\left( \eta \right)\implies y=\int  F\left( \eta \right)\, \mathrm{d}\eta=f\left( \eta \right)+g\left( \xi \right) $$
וזה יהיה הפתרון של משוואת הגל. אם נציב בחזרה משתנים נקבל סה"כ:
$$ y=f(x+c t)+g(x-c t)$$

\end{proof}
\begin{remark}
במקרה ש- \(f,g\) שניהם מייצגים גלי סינוס, נקבל:
$$ \begin{aligned}y&=f\left( x+c t \right)+g\left( x-c t \right)\\&=A\sin\left( kx+\omega t \right)+A\sin\left( kx-\omega t \right)\\
&=2A\sin(kx)\cos\left( \omega t \right)\end{aligned}$$
זהו מקרה חשוב כאשר מפרקים לפורייה, כי כל גל יהיה סופרפוזציה של גלי סינוס. זה נותן את התובנה שגל נקבע ע"י התדירות והאורך גל. ואם נניח את הקשר הלינארי \(\omega=kc\) אז נקבל כי גל נקבע ביחידות ע"י התדירות שלו.

\end{remark}
\begin{proposition}[משוואת דלמבר]
בתחום \(\mathbb{R}\) פונקציה \(y=(x,t)\) אשר מקיימת את משוואת הגל כך שמתחילה עם מיקום התחלתי \(I\) ומהירות התחלתית \(V\) תקיים:
$$ y(x,t)=\frac12I(x+ c  t)+\frac12I(x- c  t)+\frac1{2 c }\int_{x- c  t}^{x+ c  t}V(z)dz$$

\end{proposition}
\begin{proof}
נניח כי המיקום ההתחלתי נתון ע"י \(I\), והמהירות התחלתית ב-\(V\). במקרה זה:
$$(*)\quad  y(x,0)=f(x)+g(x)=I(x)$$
כאשר אם נחזור להציב \(\eta=x+vt,\xi=x-vt\) נקבל כי \(\frac{\partial \xi}{\partial t}=-v,\frac{\partial \eta}{\partial t}=v\) ולכן:
$$ \left.\frac{\partial y(x,t)}{\partial t}\right|_{t=0}=\left.\frac{\partial f}{\partial\eta}\frac{\partial\eta}{\partial t}\right|_{t=0}+\left.\frac{\partial g}{\partial\xi}\frac{\partial\xi}{\partial t}\right|_{t=0}\implies  \left.\frac{\partial y(x,t)}{\partial t}\right|_{t=0}=\frac{\partial f}{\partial x}v-\frac{\partial g}{\partial x}v=V(x)$$
כלומר נקבל כי מתקיים:
$$(* *)\quad  \frac{\partial f}{\partial x}-\frac{\partial g}{\partial x}=\frac{1}{v} V(x)\implies  f(x)-g(x)=\frac{1}{v}\int_0^xV(x)dx,$$
כאשר מ-\((*),(* *)\) ניתן להגיע למשוואת דלמבר:
$$ y(x,t)=\frac12I(x+ c  t)+\frac12I(x- c  t)+\frac1{2 c }\int_{x- c  t}^{x+ c  t}V(z)dz,$$
בשיטה זו נוח לפתור בעיות של מיתר אינסופי. אך ניתן גם לפתור בלי בעיה בעיות של מיתר חצי אינסופי עם שיטת הדמויות.

\end{proof}
\begin{remark}
ממשוואה זו רואים תכונה חשובה של משוואת הגלים - קוסאליות - אם פונקצייה מקיימת את משוואת הגלים אז ההתפתחום בזמן של נקודה \(x_{0}\) לא תשפיע על תחום אשר מרחקו ב-\(x_{0}\) גדול מ-\(ct\).

\end{remark}
\begin{proposition}
בתחום סופי ניתן לפתור בעזרת הפרדת משתנים. 

\end{proposition}
\begin{proof}
בהנחה שהפתרון של המשוואה ניתן לכתיבה כמכפלה של פונקציות של משתנה יחיד, ניתן להגדיר \(y(x,t)=X(x)T(t)\). וכעת משוואת הגלים תהיה:
$$ \frac{\partial^2[X(x)T(t)]}{\partial x^2}=\frac1{ c ^2}\frac{\partial^2[X(x)T(t)]}{\partial t^2}\implies T(t)\frac{\partial^2X(x)}{\partial x^2}=\frac{1}{ c ^2}X(x)\frac{\partial^2T(t)}{\partial t^2}$$
כאשר נחלק את שתי נאגפים ב-\(T(t)X(x)\) כדי לקבל בכל אגף איבר שתלוי במשתנה אחד ונקבל:
$$ \frac1{X(x)}\frac{\partial^2X(x)}{\partial x^2}=\frac1{ c ^2}\frac1{T(t)}\frac{\partial^2T(t)}{\partial t^2}.$$
כאשר כיוון שנכון לכל זמן ולכן מקום, שתי האגפים שווים לקבוע. כלומר נקבל שתי משוואות של משתנה יחיד:
$$ \frac{\partial^2X}{\partial x^2}=\alpha X=-k^2X\qquad \frac{\partial^2T}{\partial t^2}=\alpha c ^2T=- c ^2k^2T,$$
כאשר כל אחת מהמשוואת האלה מתאר אוסצילטור הרמוני. נוח לפתור בשיטה זו בעיות של מיתר סופי או תנאי שפה מחזוריים. או באופן כללי בעיות רב מימדיות.

\end{proof}
\begin{proposition}[פתרון בעזרת התמרת פורייה]
הפתרון של משוואת הגלים של מיתר אינסופי עם תנאי התחלה \(\psi(x,0)=f(x)\) ו-\(\dot{\psi}(x,0)=g(x)\) נתון על ידי:
$$ \psi(x,t)=\frac{1}{\sqrt{2\pi}}\int_{-\infty}^{\infty}\left[\tilde{\psi}^{+}\left(k\right)e^{ikct}+\tilde{\psi}^{-}\left(k\right)e^{-ikct}\right]e^{ikx}dk$$
כאשר מתקיים
$$ \begin{aligned}\tilde{\psi}^{+}(k)&=\frac{1}{2}\left[\mathcal{F} [{f}](k)+\frac{1}{ikc}\mathcal{F} [{g}](k)\right]\\ \tilde{\psi}^{-}\left(k\right)&=\frac{1}{2}\left[\mathcal{F[f]} (k)-\frac{1}{ikc}\mathcal{F} [{g}](k)\right]\end{aligned}$$
כאשר \(\omega_{+}(k)\) ו-\(\omega_{-}(k)\) הם יחס הנפיצה של המערכת. כלומר משוואת גלים רגילה נקבל \(\omega_{-}(k)=-kc\) ו-\(\omega_{+}(k)=kc\).

\end{proposition}
\begin{proof}
נתונה משוואת הגלים \(\ddot{\psi}=c^{2}\psi''\). נכתוב את הפתרון הכללי על ידי:
$$\psi(x,t)=\frac{1}{\sqrt{ 2\pi }} \int_{-\infty}^{\infty} \tilde{\psi}^{+}(k) e^{ ik(x-ct) } \, dk + \int_{-\infty}^{\infty} \tilde{\psi}(k) e^{ ik(x+ct) }\, dk  $$
כאשר אם נגזור לפי זמן נקבל:
$$\dot{\psi}(x,t)=\frac{1}{\sqrt{ 2\pi }} \int_{-\infty}^{\infty} -ikc \cdot\tilde{\psi}^{+}(k) e^{ ik(x-ct) } \, dk + \int_{-\infty}^{\infty} ikc\cdot\tilde{\psi}(k) e^{ ik(x+ct) }\, dk  $$
כאשר אם נגדיר \(\psi(x,0)=f(x)\) ו-\(\dot{\psi}(x,0)=g(x)\) נקבל:
\begin{gather*}f(x)=\frac{1}{\sqrt{ 2\pi }}\left[ \int_{-\infty}^{\infty} \tilde{\psi}^{+}(k)e^{ ikx } \, dk + \int_{-\infty}^{\infty} \tilde{\psi}^{-}(x)e^{ ikx } \, dk  \right]  \\g(x)=\frac{1}{\sqrt{ 2\pi }}\left[ \int_{-\infty}^{\infty} -ikc\cdot\tilde{\psi}^{+}(k)e^{ ikx } \, dk + \int_{-\infty}^{\infty} ikc\cdot \tilde{\psi}^{-}(x)e^{ ikx } \, dk  \right] 
\end{gather*}
ולכן:
$$\mathcal{F} [f](k)= \tilde{\psi}^{+}(k)+\tilde{\psi}^{-}(k) \qquad \mathcal{F} [g] (k)=-ikc \cdot\tilde{\psi}^{+}(k) + ikc \cdot \tilde{\psi}^{-}(k)
$$
ומכאן ניתן למצוא את הביטוי עבור \(\tilde{\psi}^{+}\) ו-\(\tilde{\psi}^{-}\) בעזרת ההתמרת פורייה של התנאי התחלה.

\end{proof}
\section{בעיה רב מימדית לא חסומה}

\begin{proposition}
משוואת הגלים מהצורה:
$$u_{tt}-c^{2}\bar{\nabla}^2 u=0$$
היא אינווריאנטית תחת:

  \begin{enumerate}
    \item הזזות במרחב ובזמן. 


    \item סיבובים במרחב. 


    \item טרנספורמציית לורנץ. 


  \end{enumerate}
\end{proposition}
\begin{proposition}
עבור משוואה מהצורה:
$$\frac{\partial ^2u}{\partial t^{2}}-c^{2}u=f $$
תחת התנאים \(u|_{t=0}=g\) ו-\(u_{t}|_{t=0}=h\) נקבל

\end{proposition}
\chapter{משוואת החום}

\section{משוואת החום החד מימדית}

\begin{definition}[משוואת החום החד מימדית]
$$u_{t}=k u_{x x}.$$
כאשר \(k\) נקרא קבוע הדיפוזיה. 

\end{definition}
\begin{remark}
המשוואה אומרת למעשה שאם פונקציה מקיימת את משוואת החום, אז איך שהיא משתנה בזמן פרופוצינלית לקמירות שלה לפי המקום. 

\end{remark}
\begin{proposition}[עקרון המקסימום]
אם \(u(x,t)\) מקיים את משוואת החום בתחום \(0\leq x\leq \ell\) הערך המינימלי או המקסימלי \(u(x,t)\) מושג או בהתחלה או בקצה \(x=0,\ell\), ולא יופיע באמצע כל עוד \(u\) לא קבועה.

\end{proposition}
ניתן לנסח את העקרון באופן זהה עבור המינימום:

\begin{proposition}[עקרון המינימום]
אם \(u\) מתאורת על ידי משוואה מהצורה \(u_{t}=u_{x x}\) בתחום \(D=\left\{  0\leq x\leq L  \right\}\) אזי עבור \(t\geq 0\) המינימום של \(u(x,t)\) מושג או ב-\(t=0\) או בקצוות \(x=0,L\). כאשר אם \(u\) קבועה אז המינימום הוא בכל מקום.

\end{proposition}
\begin{proposition}
אם יש שתי תנאי שפה(אילוצים מרחביים) ותנאי התחלה(אילוץ זמני) אזי קיים פתרון יחיד.

\end{proposition}
\begin{proposition}[פתרון משוואת החום מסדר ראשון]
עבור משוואה מהצורה:
$$u_{x x} = u_{t}$$
עם התנאי שפה \(u|_{x=0}=0\) ו-\(u|_{x=L}=0\) נקבל כי הפתרון הכללי יהיה:
$$u=\sum_{n}C_{2}e^{ -\lambda^{2}t }\sin\left( \frac{\pi n}{L}x \right)$$
כאשר בהנתן תנאי התחלה ניתן למצוא את הקבועים \(C_{2}(n)\) ולקבל יחידות.

\end{proposition}
\begin{example}[פתרון בעזרת הפרדת משתנים]
נניח פתרון מהצורה:
$$u(x,t)=X(x)T(t)$$
כאשר יש הרבה פתרונות כאלה, ופתרון כללי יהיה צירוף לינארי שלהם, כאשר יהיה קיים צירוף לינארי יחיד אשר מקיים את התנאי שפה והתחלה. נציב במשוואה:
$$X\frac{\mathrm{d} T}{\mathrm{d} t} =T\frac{\mathrm{d} ^{2}X}{\mathrm{d} x^{2}} \implies  \frac{1}{T}\frac{\mathrm{d} T}{\mathrm{d} t} =\frac{1}{X}\frac{\mathrm{d} ^{2}X}{\mathrm{d} x^{2}} $$
זה נכון לכל ערך של \(x\) ושל \(t\) ולכן נדרש כי כל אגף שווה לקבוע. נפרק לשלושה מקרים:
אם הקבוע הוא חיובי, נקבל את המערכת:
$$\frac{1}{T}\frac{\mathrm{d} T}{\mathrm{d} t} =\frac{1}{X}\frac{\mathrm{d} ^{2}X}{\mathrm{d} x^{2}} = \lambda^{2}$$
ניתן לפתור כל מדר ולקבל:
$$u(x,t)=X(x)T(t)=Ae^{ \lambda^{2}t }\left[ B\cosh\left( \lambda x \right)+C\sinh\left( \lambda x \right) \right]$$
כאשר זהו פתרון מתבדר ולכן ניתן לפסול אותו.
אם הקבוע הוא אפס נקבל:
$$\frac{1}{T}\frac{\mathrm{d} T}{\mathrm{d} t} =\frac{1}{X}\frac{\mathrm{d} ^{2}X}{\mathrm{d} x^{2}}=0$$
וכעת הפתרון יהיה:
$$u(x,t)=X(x)T(t)=A(Bx+C)$$
ובעזרת התנאי שפה נקבל כי הפתרון היחיד יהיה \(u=0\). ולכן מאלימנציה נקבל כי הקבוע הוא שלילי ולכן ניתן לכתוב אותו כ-\(-\lambda^{2}\). כלומר:
$$\frac{1}{T}\frac{\mathrm{d} T}{\mathrm{d} t} =\frac{1}{X}\frac{\mathrm{d} ^{2}X}{\mathrm{d} x^{2}}=-\lambda^{2}$$
נקבל כי הפתרון יהיה:
$$u=XT= Ae^{ -\lambda^{2} t }\left[ B\cos\left( \lambda x \right)+C\sin\left( \lambda x \right) \right]=e^{ -\lambda^{2}t }\left[ C_{1}\cos\left( \lambda x \right)+C_{2}\sin\left( \lambda x \right) \right]$$
כאשר אם נציב את התנאי שפה \(u(x=0,t)=0\) נקבל \(C_{1}=0\) ואם נציב את התנאי שפה \(u(x=L,t)=0\) נקבל:
$$C_{2}e^{ -\lambda^{2}t }\sin\left( \lambda L \right)=0\implies \sin\left( \lambda L \right)=0\implies \lambda L=\pi n\implies \lambda_{n}=\frac{\pi n}{L}$$
ולכן:
$$u_{n}=C_{2}e^{ -\lambda^{2}t }\sin\left( \lambda_{n}x \right)\implies u=\sum_{n} u_{n}=\sum_{n}C_{2}e^{ -\lambda^{2}t }\sin\left( \frac{\pi n}{L}x \right)$$

\end{example}
\begin{proposition}[פתרון כללי של משוואת החום]
עבור משוואת החום מהצורה:
$$\frac{\partial u}{\partial t} = D\frac{\partial^2 u}{\partial x^2}$$
נגדיר אופרטור \(Lu=u_{x x}\). לפי התנאי השפה, נקבל פונקציות עצמיות \(u_{n}\) וערכים עצמיים \(-\lambda_{n}^2\). כעת משטורם ליוביל נקבל \(u=\sum_{n=0}^{\infty} T(t)u_{n}\). נציב במשוואה ונקבל:
$$\sum_{n=0}^{\infty} \left( \dot{T}(t)+D\lambda^2T(t) \right) u_{n}=0\implies T(t)=A_{n}e^{-D\lambda^2t}$$
ולכן הפתרון הכללי יהיה:
$$u=\sum_{n=0}^{\infty} A_{n}e^{-D\lambda^2t}u_{n}$$
כאשר את המקדמים ניתן למצוא מתנאי התחלה, ופונקציות עצמיות \(u_{n}\) ניתן למצוא מהתנאי שפה.

\end{proposition}
\begin{remark}
בעזרת דרך זו ניתן גם למצוא את הפתרון כאשר יש גורמים נוספים של הנגזרת המרחבית(למשל יש במשוואה את הגורם \(\frac{d}{dx}u\)).

\end{remark}
\begin{proposition}[פתרון משוואה לא הומוגנית]
\textbf{דרך 1}
משוואת חום לא הומוגנית תהיה מהצורה:
$$\frac{\partial u}{\partial t}-D\frac{\partial ^2u}{\partial x^2}=f(t)$$
נגדיר אופרטור \(Lu=u_{x x}\) עם התנאי שפה הנתונים ונמצא פונקציות עצמיות. כלומר \(u=\sum T(t)u_{n}\) נציב במשוואה תוך כדי שנבטא את \(f(t)\) בעזרת הפונקציות העצמיות:
$$\sum \left( \ddot{T}(t)+D\lambda^2 \right) u_{n} =\frac{1}{||u_{n}||}\sum \langle f(t),u_{n}\rangle u_{n}$$
ונשוואה מקדמים כך שנקבל אינסוף משוואת דיפרנציאליות.
\textbf{דרך 2}
ננחש פתרון פרטי לבעיה האי הומגנית. כעת מתקיים:
$$u=u_{h}+u_{p}\implies u_{h}=u-u_{p}$$
וקיבלנו משוואה הומוגנית עם תנאי שפה חדשים. נפתור עבור \(u_{h}\) ונקבל כי הפתרון עבור \(u\) יהיה \(u=u_{h}+u_{p}\).

\end{proposition}
\begin{proposition}[פתרון משוואה עם תנאי שפה לא הומוגניים]
נסתכל על המצב היציב \(u_{ss}\) - פתרון שלא תלוי בזמן. מתקיים:
$$u(x,t)=u_{ss}(x,t)+u_{h}(x,t)$$
עבור משוואת החום מהצורה \(\frac{\partial u}{\partial t}=\frac{\partial ^2u}{\partial x^2}\), עם תנאי התחלה \(u(x,t=0)=\phi(x)\) נקבל:
- עבור משוואה עם תנאי שפה דירכלה נקבל כי \(u_{ss}\) לינארית ונקבל \(u_{h}\) משוואה הומוגנית עם תנאי שפה הומוגנים
- עבור משוואה עם תנאי שפה נוימן נקבל כי \(u_{ss}\) ריבועית(כלומר מהצורה \(Ax^2+Bx+C\)) כאשר נשתמש בפרמטר העודף כדי לדרוש שגם המשוואה תהיה הומוגנית. אם הצלחנו להגיע למצב שרק תנאי השפה הומגניים, נקבל משוואה לא הומוגנית עם תנאי שפה הומגני. אשר ניתן לפתור בעזרת פריסה לפונקציות עצמיות.

\end{proposition}
\section{פתרון בתחום אינסופי חד מימדי}

\begin{definition}[תחום אינסופי]
תחום שבו משתנה מסויים שואף לאינסוף משני הכיוונים

\end{definition}
\begin{proposition}[פתרון בעזרת התמרת פורייה]
ניתן לפתור בתחום אינסופי בעזרת התמרת פורייה עם השלבים הבאים:

  \begin{enumerate}
    \item נתמיר לפי פורייה את אחד המשתנים כדי להפוך את המד"ח למדר. 


    \item נפתור מד"ר פשוט עבור המשתנה המותמר ונציב את התנאי שפה המותמר כדי למצוא את הקבועים. 


    \item נבצע התמרה הפוכה כדי למצוא את הפונקציה המקורית. 


  \end{enumerate}
\end{proposition}
\begin{example}
נניח פתרון מהצורה:
$$\frac{\partial u}{\partial t} =\alpha^{2}\frac{\partial^{2}u}{\partial x^{2}} $$
בתחום \(-\infty \leq x\leq \infty\) ותנאי התחלה \(u(x,t=0)=\phi(x)\). כיוון שהנתון הוא לפי מיקום נבצע התמרת פורייה לפי מיקום:
$$\mathcal{F} _{x}[u_{t}]=\alpha^{2}\mathcal{F} _{x}[u_{x x}]\implies \mathcal{F} _{x}[u]_{t}=-\alpha^{2}\omega^{2}\mathcal{F} _{x}[u]$$
כאשר זהו מדר פריד:
$$\ln\left( \mathcal{F} _{x}[u]\left( \omega,t \right) \right)=-\alpha^{2}\omega^{2}t+C\implies \mathcal{F} _{x}[u]=Ae^{ -\alpha^{2}\omega^{2}t }$$
כאשר מהתנאי התחלה נקבל:
$$\mathcal{F} _{x}[u]\left( \omega,t =0\right)=\mathcal{F} _{x}\left[ \phi \right]\left( \omega \right)=A$$
נרצה לקחת כעת את ההתמרה ההפוכה:
$$u=\mathcal{F} _{x}^{-1}\left[ \mathcal{F} _{x}[u]\left( \omega,t \right) \right]=\mathcal{F}^{-1} _{x}\left[ \mathcal{F} _{x}\left[ \phi \right]e^{ -\alpha^{2}\omega^{2}t } \right]=\mathcal{F} _{x}^{-1}\left[ \mathcal{F} _{x}\left[ \phi \right]\mathcal{F} _{x}\left[ \mathcal{F} _{x}^{-1}\left[ e^{ -\alpha^{2}\omega^{2}t } \right] \right] \right]$$
וניתן כעת להשתמש במשפט הקונבולציה ולקבל:
$$u=\phi * \mathcal{F} _{x}^{-1}\left[ e^{ -\alpha^{2}\omega^{2}t } \right]=\phi(x)* \frac{1}{2\alpha \sqrt{ \pi t}}e^{ -x^{2}/4\alpha^{2}t }= \frac{1}{2\alpha \sqrt{ \pi t }}\int_{-\infty}^{\infty} \phi(y)e^{ -(x-y)^{2}/4\alpha^{2}t } \, dx $$

\end{example}
\begin{remark}
הפונקציה:
$$G(x,t)=\frac{1}{2\alpha \sqrt{ \pi t }}e^{ -(x-y)^{2} /4\alpha^{2}t}$$
תהיה הפונקציית גרין של המשוואה ביחד עם התנאי שפה. 

\end{remark}
\begin{corollary}[פתרון בעזרת פונקציית גרין]
בהנתן משוואת החום החד מימדית \(\frac{\partial u}{\partial t}=D\frac{\partial^2 u}{\partial x^2}\) בתחום אינסופי עם תנאי התחלה \(u(x,t=0)=f(x)\) ניתן לפתור באמצעות הפונקציית גרין:
$$ G\left(x,x^{\prime},t\right)=\frac1{\sqrt{4\pi Dt}}e^{-\frac{\left(x-x^{\prime}\right)^2}{4Dt}}$$
כאשר נקבל כי הפתרון יהיה:
$$ u\left(x,t\right)=\frac{1}{\sqrt{4\pi Dt}}\int_{-\infty}^{\infty}u_{0}\left(\xi\right)e^{-\frac{\left(x-\xi\right)^{2}}{4Dt}}d\xi $$

\end{corollary}
\begin{definition}[תחום חצי אינסופי]
תחום כאשר אחד המשתנים שואף לאינסוף מצד אחד אך לא מהשני.

\end{definition}
\begin{proposition}[פתרון בעזרת שיטת הדמויות]
עבור ישר חצי אינסופי - כמו במשוואת הגלים, נשלים לישר אינסופי:

\end{proposition}
\begin{enumerate}
  \item אם נתון תנאי על הנגזרת נשלים לפונקציה זוגית 


  \item אם נתון תנאי על המיקום נשלים לפונקציה אי זוגית 
ונפתור בעזרת הפונקציית גרין.


\end{enumerate}
עבור תחום חצי אינסופי נוח לפתור בעזרת משוואת לפלס.

\begin{example}[פתרון בעזרת התמרת לפלס]
נתחיל ממשוואת החום החד מימדית:
$$\frac{\partial u}{\partial t} =\frac{\partial^{2}u}{\partial x^{2}} $$
בתחום \(0\leq x \leq \infty\) ו-\(0\leq t\leq \infty\) עם התנאי שפה \(u(x=0,t)=u_{0}\) והתנאי התחלה \(u(x,t=0)=0\). נרצה להפתר מהגזירה לפי הזמן בעזרת התמרת לפלס. נקבל:
$$\mathcal{L}_{t}[u_{t}]=\mathcal{L}_{t}[u_{x x}]\implies s\mathcal{L}_{t}[u]-u(x,t=0)=\mathcal{L}_{t}[u_{x x}]\implies s \mathcal{L}_{t}[u] = \mathcal{L}_{t}[u_{x x}]$$
כאשר נשים לב כי התנאי שפה מקיים \(\mathcal{L}_{t}[u](x=0,s)=\frac{u_{0}}{s}\) כיוון שהתמרת לפלס של קבוע. כעת קיבלנו מדר פשוט אשר פתרונו יהיה:
$$\mathcal{L}_{t}[u_{}]= C_{1}e^{ \sqrt{ s }x }+C_{2}e^{ -\sqrt{ s }x }$$
נרצה כי \(u(x,s)\) יהיה חסום כאשר \(x\to \infty\) ולכן \(C_{1} = 0\). מהתנאי התחלה נקבל:
$$\frac{u_{0}}{s}=C_{1}+C_{2}\implies C_{2}=\frac{u_{0}}{2}\implies \mathcal{L}_{t}[u](x,s)=\frac{u_{0}}{s}e^{ -\sqrt{ s }x }$$
כאשר ניתן למצוא בעזרת טבלה למשל כי ההתמרה ההפוכה תהיה:
$$u(x,t)=u_{0}\mathrm{erfc}\left( \frac{x}{2\sqrt{ t }} \right)\quad \mathrm{where}\quad \mathrm{erfc}(y)=\frac{2}{\sqrt{ \pi }}\int_{y}^{\infty}e^{ -p^{2} }\mathrm{d}p$$

\end{example}
\begin{remark}
אנחנו רוצים בדרך כלל לעשות התמרה לפי המשתנה שאנחנו יודעים עליו את הערכים(כמו תנאי התחלה) בשביל הנוסחא של הנגזרות.

\end{remark}
\section{משוואת החום הרב מימית}

\begin{definition}[משוואת החום/דיפוזיה הרב מימדית]
משווה מהצורה \(\bar{\nabla}^2u+\frac{\mathrm{d} u}{\mathrm{d} t}=0\). נשים לב כי זוהי משוואה פרבולית.

\end{definition}
\begin{proposition}[פתרון של תוף מעגלי]
הפונקציות העצמיות בבעיה זו יהיו:
$$J_{n}\left(k_{nm}r\right)\cos\left(n\theta\right)\quad J_{n}\left(k_{nm}r\right)\sin\left(n\theta\right)$$
כאשר הערכים העצמיים המצאימים הם \(-k_{nm}^2\).
ולכן הפתרון הכללי עבור משוואת החום \(u_t=D\nabla^2 u\) תהיה:
$$u(r,\theta,t)=\sum_{n=0}^{\infty}\sum_{m=1}^{\infty}J_{n}(k_{nm}r)[a_{nm}\cos(n\theta)+b_{nm}\sin(n\theta)]T_{n}(t)$$

\end{proposition}
\section{פתרון בתחום לא חסום}

\begin{definition}[פונקציית גרין של משוואת לפלס]
פונקציה המקיימת:
$$\nabla^{2}G(\mathbf{x},\mathbf{x}^{\prime})=\delta(\mathbf{x}-\mathbf{x}^{\prime})$$

\end{definition}
\begin{proposition}
עבור משוואה דיפרנציאלית מהצורה:
$$\nabla^{2}\Phi=-4\pi g(\mathbf{r})$$
נקבל כי:

\end{proposition}
\chapter{משוואת לפלס}

\section{תכונות של משוואת לפלס}

\begin{definition}[משוואת לפלס]
משוואה מהצורה \(\bar{\nabla}^2u=0\). נשים לב כי זוהי משוואה אליפטית.

\end{definition}
\begin{definition}[פונקציה הרמונית]
פונקציה המקיימת את משוואת לפלס.

\end{definition}
\begin{proposition}
קבוצת הפונקציות ההרמונית יוצרות מרחב ווקטורי.

\end{proposition}
\begin{proof}
הלפלסיאן הוא אופרטור לינארי, ובפרט הגרעין שלו הוא תת מרחב ווקטורי.

\end{proof}
\begin{proposition}[קיום ויחידות]
בהנתן תנאי שפה דיריכלה קיים פתרון ויחיד. בהנתן תנאי שפה נוימן, קיים פתרון יחיד עד כדי קבוע.

\end{proposition}
\begin{proposition}[עקרון המקסימום]
יהי \(D\) תחום פתוח חסום קשיר שבתוכו מתקיים משוואת לפלס \(\bar{\nabla}^2u=0\) כך ש-\(u\) רציפה ב-\(\overline{D}=D\cup \partial D\) אזי:

  \begin{enumerate}
    \item כל עוד הפונקציה לא קבועה, המקסימום והמינימום מתקבלים על השפה \(\partial D\). כלומר קיימים \(x_{M},x_{m}\in \partial D\) כך שמתקיים לכל \(x \in D\): 
$$u(x_{m})\leq u(x)\leq u(x_{M})$$


    \item אם המקסימום מתקבל בפנים \(D\) אזי הפונקציה \(u\) היא קבועה. 


  \end{enumerate}
\end{proposition}
\begin{proposition}
משוואת לפלס אינווריאנטי לסיבובים ולהזזות.

\end{proposition}
\begin{proposition}[עקרון הממוצע]
יהי \(D\) תחום פתוח שעליו מתקיים \(\bar{\nabla}^2u=0\). יהי \(x \in D\) ויהי \(B(x,r)\) כדור ברדיוס \(r\) סביב \(x\). אזי הערך של \(x\) יהיה הערך הממוצע של הערכים על השפה של הכדור.

\end{proposition}
\begin{theorem}[ליוביל]
אם \(u\) מקיימת את משוואת לפלס עבור \(\mathbb{R}^{n}\) כך ש-\(u\) חסומה מלעיל או מלרע, אז \(u\) קבועה.

\end{theorem}
\begin{proposition}[נוסחאת פואסון]
יהי \(D\) תחום פתוח שעליו מתקיים \(\bar{\nabla}^2u=0\). יהי \(x_{0} \in D\) ויהי \(B(x,r)\) כדור ברדיוס \(r\) סביב \(x_{0}\). ניתן לקבל כל נקודה על הכדור בעזרת הערכים על השפה שלו של \(B(x_{0},r)\)$$\begin{array}{c}{{P(x,\xi)=\frac{r^{2}-\left|x-x_{0}\right|^{2}}{\omega_{n}r|x-\xi|^{n}},}}\\ {{x\in B(x_{0},r),\,\xi\in\partial B(x_{0},r),}}\end{array}$$
כאשר \(\omega_{n}\) זה שטח פנים של כדור \(n- 1\) מימדי וכעת משוואת פואסון תהיה:
$$f(x)=\int_{\partial B(x_{0},r)}P(x,\xi)f(\xi)d\sigma(\xi),$$

\end{proposition}
\begin{remark}
נשים לב כי עקרון הממצוע מתקבל מנוסחאת פואסון עבור \(x=0\).

\end{remark}
\begin{proposition}
כל ערך של פונקציה הרמונית תהיה גזירה מכל סדר.

\end{proposition}
\begin{remark}
הכיוון ההפוך לא נכון. לדוגמא \(e^{ -1/x^{2} }\) ב-\(x=0\).

\end{remark}
\section{משוואת לפלס בתחום לא חסום}

\section{פתרון בתחום חסום}

\begin{proposition}[פתרון חד מימדי]
פתרון משוואת לפלס בחד מימד יהיה:
$$\frac{\partial^2 u}{\partial x^2} =0\implies u=Ax+B$$
ומוצאים את הפתרון מהתנאי שפה.

\end{proposition}
\begin{proposition}[פתרון קרטזי דו מימדי]
פתרון משוואת לפלס בקורדינטות קרטזי דו מימדי
$$ \frac{\partial^2u}{\partial x^2}+\frac{\partial^2u}{\partial y^2}=0.$$
נשתמש בהפרדת משתנים מהצורה \(u(x,y)=X(x)Y(y)\) ונקבל:
$$ X^{\prime\prime}=\lambda^2X,\quad Y^{\prime\prime}=-\lambda^2Y.$$
כלומר הפתרון הכללי יהיה:
$$ u(x,y)=(A\cosh\lambda x+B\sinh\lambda x)(C\cos\lambda y+D\sin\lambda y)$$
או לחלופין:
$$ u(x,y)=[A\exp\:\lambda x+B\exp(-\lambda x)](C\cos\lambda y+D\sin\lambda y)$$

\end{proposition}
\begin{example}[משוואת לפלס במלבן]
ראשית נפתור את המקרה שצלע אחת עם תנאי שפה לא הומוגני. כלומר:
$$u(x=0,y)=u(x=L_{x},y)=u(x,y=0)=0\quad u(x,y=L_{y})=f(x)$$
נגדיר אופרטור \(Lu=u_{x x}\). נחפש פונקציות עצמיות מהצורה \(Lu_{n}=-\lambda^2u_{n}\). מתנאי שפה נקבל \(u_{n}=\sin \left( \lambda_{n} x \right)=\sin\left( \frac{\pi n}{L_{x}}x \right)\). נסמן \(u=\sum Y(t)u_{n}\) ונקבל את המשוואה:
\begin{gather*}\sum \left( \ddot{Y}(y) - \lambda_{n}^2Y(y) \right) u_{n} = 0\implies  \ddot{Y}- \left( \frac{\pi n}{L_{x}} \right)^2Y=0 \\Y(y)=A\cosh\left( \frac{\pi n}{L_{x}}y \right)+B\sinh\left( \frac{\pi n}{L_{x}}y \right)
\end{gather*}
מהתנאי שפה \(u(x,0)=0\) נקבל כי \(A=0\). כעת מתקיים:
$$u(x,y)=\sum B_{n}\sinh\left( \frac{\pi n}{L_{x}}y \right)\sin\left( \frac{\pi n}{L_{x}}x \right)$$
וכעת נותר רק להשתמש בתנאי שפה האחרון \(u(x,y=L_{y})=f(x)\).
אחרי שיש לנו את הפתרון הזה. ניתן למצוא כל פתרון בעזרת סופרפוזיציה.

\end{example}
\begin{proposition}[פתרון בקורדינטות פולאריות]
הפתרון הכללי של משוואת לפלס בקורדינטות פולריות יהיה:
$$\!\!\!u(r,\theta)\!=\!A\!+\!B\ln r\!+\!\sum_{n=1}^\infty\left[A_n\cos(n\theta)\!+\!B_n\sin(n\theta)\right](C_{n}r^{-n}\!+\!D_{n}r^n)$$
כאשר אם הנגזרת חסומה \(D_{n}=0\). אם לא מתבדר בראשית, \(C_{n},B=0\), וכן אם הבעיה פנימית כך שאין התבדרות בראשית נקבל:
$$ u(r,\theta)=a_0+\sum_{n=1}^{\infty}\left[a_n\cos(n\theta)+b_n\sin(n\theta)\right]r^{n}$$
ואם הבעיה חיצונית כך שאין התבדרות באינסוף נקבל:
$$ u(r,\theta)=a_0+\sum_{n=1}^{\infty}\left[a_n\cos(n\theta)+b_n\sin(n\theta)\right]r^{-n}$$

\end{proposition}
\begin{proof}
משוואת לפלס בקורדינטות פולאריות תהיה מהצורה:
$$\nabla^{2}u={\frac{1}{r}}{\frac{\partial}{\partial r}}\Big(r{\frac{\partial u}{\partial r}}\Big)+{\frac{1}{r}}{\frac{\partial^{2}u}{\partial\theta^{2}}}=0$$
כאשר נחפש פתרון בהפרדת משתנים \(u\left( r,\theta \right)=R(t)\Theta\left( \theta \right)\). נציב במשוואה ונקבל:
$$R^{\prime\prime}\Theta+\frac{1}{r}R^{\prime}\Theta+\frac{1}{r^{2}}R\Theta^{\prime\prime}=0\implies r^{2}\frac{R^{\prime\prime}}{R}+r\frac{R^{\prime}}{R}=-\frac{\Theta^{\prime\prime}}{\Theta}\equiv\lambda^{2}$$
הפתרון לחלק הזוויתי יהיה:
$$\Theta^{\prime\prime}+\lambda^{2}\Theta=0\implies\Theta=c\cos\left( \lambda\theta \right)+d\sin\left( \lambda\theta \right)$$
הפתרון לחלק המרחבי יהיה:
$$r^{2}\frac{R^{\prime\prime}}{R}+r\frac{R^{\prime}}{R}-\lambda^{2}=0\implies r^{2}R^{\prime\prime}+r R^{\prime}-\lambda^{2}R=0$$
כאשר נשים לב כי זוהי משוואת אויילר, אשר נותנת כאשר \(\lambda \neq 0\):
$$R(r)=ar^{\lambda}+br^{-\lambda}\implies u_{\lambda}=\bigl(a r^{\lambda}+b r^{-\lambda}\bigr)(c\cos(\lambda\theta)+d\sin(\lambda\theta))$$
וכאשר \(\lambda=0\) נקבל ריבוי אלגברי ולכן:
$$R(r)=C_{1}+C_{2}\ln(r)\implies u_{0}=\left( \tilde{C}_{1}+\tilde{C}_{2}\ln(r) \right)$$
כאשר איחדנו את הקבועים שהתקבלו מהמקדם של \(c\cos(0)=c\) עם המקדמים \(C_{1},C_{2}\).
כעת נדרש להציב תנאי שפה. ראשית קיים תנאי שפה מחזורי מהפונקציות הטריגונומטריים, נדרוש כי המחזור יהיה כפולה של \(2\pi\). אנו יודעים כי:
$$f=\frac{\lambda}{2\pi}\implies T=\frac{2\pi}{\lambda}$$
מספיק כי הזמן מחזור יתחלק ב-\(2\pi\)(כי אז בפרט המחזור יהיה \(2\pi\)) כלומר:
$$T\overset{!}{=}  \frac{2\pi}{n}\implies \frac{2\pi}{\lambda}=\frac{2\pi}{n}\implies n=\lambda$$
ולכן הפתרון הכללי יהיה הסכום של כל הפתרונות:
$$\!\!\!u(r,\theta)\!=\!A\!+\!B\ln r\!+\!\sum_{n=1}^\infty\left[A_n\cos(n\theta)\!+\!B_n\sin(n\theta)\right](C_{n}r^{-n}\!+\!D_{n}r^n)$$
כאשר אם הבעיה היא פנימית כך שאין התבדרות בראשית לא ייתכן כי \(C_{n}\neq 0\) או ש-\(B\neq 0\) כיוון שמקרה זה היה התבדרות בראשית.
אם הבעיה היא חיצונית כך שאין התברדות באינסוף לא ייתכן כי \(B\neq 0\) או ש-\(D_{n}\neq 0\) כיוון שאז היה התבדרות באינסוף.

\end{proof}
\begin{remark}
פתרון זה יהיה הפתרון גם במקרה של קורדינטות גליליות כך שקיימת סימטרייה להזזה בציר \(z\). זה כיוון שמקרה זה בפועל שקול למערכת פולארית.

\end{remark}
\begin{proposition}
פתרון משוואת לפלס בקורדינטות גלליות תתן את המשוואה:
$$ \frac{1}{r }\frac{\partial}{\partial r}\left(r \frac{\partial u}{\partial r }\right)+\frac{1}{ r ^{2}}\frac{\partial^{2}u}{\partial\phi^{2}}+\frac{\partial^{2}u}{\partial z^{2}}=0.$$
כאשר אם פותרים בהפרדת משתנים \(u=\Theta\left( \theta \right)R(r)Z(z)\), נקבל:
\begin{gather*}  Z(z)=E\exp(-kz)+F\exp kz\\ \Theta\left( \theta \right)=C\cos \left( m\theta \right)+D\sin \left( m\theta \right) \\R(r)=AJ_m(kr)+BY_m(kr)
\end{gather*}
כאשר \(Y_{m}\) היא פונקציית בסל מהסוג השני ומופיע רק אם יש התבדרות בראשית.

\end{proposition}
\begin{definition}[סימטריה אזיומטלית]
סימטריה גלילית או סימטרייה מסביב לישר.

\end{definition}
\begin{proposition}[פתרון כדורי עם סימטריה אזימוטלית]
פתרון משוואת לפלס \(\bar{\nabla}^2u=0\) כאשר יש סימטריה אזימוטילית.
$$ u\left(r,\theta,\varphi\right)=\sum_{\ell=0}^{\infty}\left(A_\ell r^\ell+\frac{B_\ell}{r^{\ell+1}}\right)P_\ell\left(\cos\theta\right)$$

\end{proposition}
\begin{proof}
משוואת לפלס הכללית תהיה:
$$0=\nabla^{2}u={\frac{1}{r^{2}}}{\frac{\partial}{\partial r}}\left(r^{2}{\frac{\partial u}{\partial r}}\right)+{\frac{1}{r^{2}\sin\theta}}{\frac{\partial}{\partial\theta}}\left(\sin\theta{\frac{\partial u}{\partial\theta}}\right)+{\frac{1}{r^{2}\sin^{2}\theta}}{\frac{\partial^{2}u}{\partial\varphi^{2}}}$$
כאשר כיוון שיש סימטריה אזימוטלית מספיק לפתור:
$$0=\nabla^{2}u={\frac{1}{r^{2}}}{\frac{\partial}{\partial r}}\left(r^{2}{\frac{\partial u}{\partial r}}\right)+{\frac{1}{r^{2}\sin\theta}}{\frac{\partial}{\partial\theta}}\left(\sin\theta{\frac{\partial u}{\partial\theta}}\right)$$
כאשר ניתן כעת לחפש פתרון בהפרדת משתנים מהצורה \(u\left( r,\theta,\varphi \right)=R(r)\Theta\left( \theta \right)\). נציב במשוואת לפלס:
$$0=\frac{\Theta}{r^{2}}\frac{d}{d r}\left(r^{2}\frac{d R}{d r}\right)+\frac{R}{r^{2}\sin\theta}\frac{d}{d\theta}\left(\sin\theta\frac{d\Theta}{d\theta}\right)$$
עבור החלק הזוויתי נקבל:
$$\frac{1}{\sin\theta}\frac{d}{d\theta}\left(\sin\theta\frac{d\Theta}{d\theta}\right)+\Lambda\Theta=0$$
כאשר נגדיר משתנה חדש \(w=\cos \theta\) כך שמתקיים:
$${\frac{d}{d\theta}}={\frac{d w}{d\theta}}{\frac{d}{d w}}=-\sin\theta{\frac{d}{d w}}$$
כאשר נשים לב כי \(\sin ^{2}\theta=1-w^{2}\) ולכן:
$$\frac{d}{d w}\left((1-w^{2})\frac{d\Theta}{d w}\right)+\Lambda\Theta=0.$$
כאשר זוהי משוואת לג'נדר כאשר נדרש לפתור אותה בתחום \(-1\leq w\leq 1\) כאשר נדרוש כי לא תהיה התבדרות ב-\(w=\pm 1\). לכן נקבל \(\Lambda=\ell\left( \ell+1 \right)\) הם הערכים העצמיים כאשר הפונקציות העצמיות המתאימות יהיו פולינומי לג'נדר \(P_{\ell}(w)\). כעת אם נחזור חזרה למשתנה \(\theta\) נקבל כי הפונקציות העצמיות יהיו \(P_{\ell}\left( \cos\left( \theta \right) \right)\) כאשר מתקיים היחס האורתוגונאליות:
$$\int_{0}^{\pi}P_{\ell}\left(\cos\theta\right)P_{\ell^{\prime}}\left(\cos\theta\right)\sin\theta d\theta=\frac{2}{2\ell+1}\delta_{\ell\ell^{\prime}}$$
כעת נפתור עבור החלק הרדיאלי. אחרי פישוט נקבל את המשוואת אוילר הבאה:
$$r^{2}R^{\prime\prime}+2r R^{\prime}-\ell(\ell+1)R=0$$
כאשר נזכור כי הפתרון יהיה מהצורה \(r^{\alpha}\) כאשר ניתן למצוא את \(\alpha\) בעזרת הפולינום האופייני:
$$\alpha\left( \alpha-1 \right)R+2\alpha R-\ell\left( \ell+1 \right)R=0\implies\alpha\left( \alpha+1 \right)=\ell\left( \ell+1 \right)\implies \alpha=\ell,-\left( \ell+1 \right)$$
ולכן הפתרון הכללי יהיה:
$$u\left(r,\theta,\varphi\right)=\sum_{\ell=0}^{\infty}\left(A_{\ell}r^{\ell}+\frac{B_{\ell}}{r^{\ell+1}}\right)P_{\ell}\left(\cos\theta\right)$$

\end{proof}
\begin{reminder}
פולינומי לג'נדר המוכללים:
$$P_{\ell}^{m}\left(x\right)=\left(1-x^{2}\right)^{m/2}\frac{d^{m}P_{\ell}(x)}{d x^{m}}$$
ההרמוניות הספריות:
$$Y_{\ell}^{m}(\theta,\varphi)=(-1)^{m}\sqrt{\frac{2\ell+1}{4\pi}\cdot\frac{(\ell-m)!}{(\ell+m)!}}P_{\ell}^{m}(\cos\theta)e^{i m\varphi}$$

\end{reminder}
\begin{proposition}[פתרון ספרי כללי]
פתרון של משוואת לפלס \(\bar{\nabla}^2u=0\) בקורדינטות ספריות כללית תהיה מהצורה:
$$u(r,\theta,\phi)=R(r)\Theta(\theta)\Phi(\phi)$$
כאשר:
$$ \begin{gather}R(r)=Ar^\ell+Br^{-\left( \ell+1 \right)}\\\Phi\left( \phi \right)= C\cos m\phi+D\sin m\phi = c_{m}e^{im\phi} \\ \Theta\left( \theta \right)= EP_\ell^m\left( \cos\theta \right)+FQ_\ell^m\left( \cos\theta \right)
\end{gather*}

\end{proposition}
\begin{proof}
$$0=\nabla^{2}u={\frac{1}{r^{2}}}{\frac{\partial}{\partial r}}\left(r^{2}{\frac{\partial u}{\partial r}}\right)+{\frac{1}{r^{2}\sin\theta}}{\frac{\partial}{\partial\theta}}\left(\sin\theta{\frac{\partial u}{\partial\theta}}\right)+{\frac{1}{r^{2}\sin^{2}\theta}}{\frac{\partial^{2}u}{\partial\varphi^{2}}}$$
כאשר נפריד לחלק הזוויתי והרדיאלי - \(u\left( r,\theta,\varphi \right)=R(r)Y\left( \theta,\varphi \right)\). נציב את הצורה הזו במשוואת לפלס ונקבל:
$$0=\frac{Y}{r^{2}}\left(r^{2}R^{\prime}\right)^{\prime}+\frac{R}{r^{2}\sin\theta}\frac{\partial}{\partial\theta}\left(\sin\theta\frac{\partial Y}{\partial\theta}\right)+\frac{R}{r^{2}\sin^{2}\theta}\frac{\partial^{2}Y}{\partial\varphi^{2}}$$
כאשר לאחר הפרדת משתנים נקבל:
$${\frac{1}{R}}\left(r^{2}R^{\prime}\right)^{\prime}=-{\frac{1}{Y\sin\theta}}{\frac{\partial}{\partial\theta}}\left(\sin\theta{\frac{\partial Y}{\partial\theta}}\right)-{\frac{1}{Y\sin^{2}\theta}}{\frac{\partial^{2}Y}{\partial\varphi^{2}}}=c o n s t=\Lambda$$
עבור החלק הזוויתי ניתן להגדיר \(Y\left( \theta,\varphi \right)=\Theta\left( \theta \right)\Phi\left( \varphi \right)\) ולקבל:
$$\frac{\sin\theta}{\Theta}\left(\sin\theta\Theta^{\prime}\right)^{\prime}+\Lambda\sin^{2}\theta=-\frac{\Phi^{\prime\prime}}{\Phi}=c o n s t=m^{2}$$
עבור החלק האזימטלי עבור תנאי שפה מחזוריים נקבל:
$$\Phi^{\prime\prime}=-m^{2}\Phi\implies \Phi\left( \varphi \right)=a_{m}\cos\left( m\varphi \right)+b_{m}\sin\left( m\varphi \right) \quad  m \in \mathbb{N}\cup \{ 0 \}$$
כאשר ניתן לכתוב בעזרת אקספוננט מרוכב:
$$\Phi\left( \varphi \right)=c_{m}e^{i m\varphi}\quad m \in \mathbb{Z}$$
כאשר עבור החלק הפולארי נקבל:
$$\frac{1}{\sin\theta}\left(\sin\theta\Theta^{\prime}\right)^{\prime}+\left(\Lambda-\frac{m^{2}}{\sin^{2}\theta}\right)\Theta=0$$
כאשר לאחר הצבה \(w=\cos\left( \theta \right)\) נקבל:
$$\frac{d}{d w}\left((1-w^{2})\frac{d\Theta}{d w}\right)+\left(\Lambda-\frac{m^{2}}{1-w^{2}}\right)\Theta=0$$
כאשר זוהי משוואת לג'נדר המוכללת כאשר הפתרונות שלה הם פונקציות לג'נדר המוכללים כאשר המכפלה של שתי הפתרונות תתן לנו את ההרמוניות הספריות:
$$Y_{\ell}^{m}(\theta,\varphi)=\Theta_{\ell}^{m}(\theta)\Phi_{m}(\varphi)\propto P_{\ell}^{m}(\cos\theta)e^{i m\varphi}$$
עבור החלק הרדיאלי הפתרון זהה למקרה עם סימטריה אזימוטלית.

\end{proof}
\chapter{נושאים מתקדמים של פונקציית גרין}

\section{משוואת הגלים}

\section{נוסחאות גרין}

לפני שנתחיל נזכיר את פונקציות דלתא בקורדינטות שונות:

\begin{table}[htbp]
  \centering
  \begin{tabular}{|cccc|}
    \hline
    קואורדינטות & שלושה מימדים & שני מימדים & מימד אחד \\ \hline
    קרטזיות & \(\delta(x-\xi)\delta(y-\eta)\delta(z-\zeta)\) & \(\delta(x-\xi)\delta(y-\eta)\) & \(\delta(x-\xi)\) \\ \hline
    גליליות & \(\frac{\delta(r-\rho)\delta(\varphi-\varphi^{\prime})\delta(z-\zeta)}{r}\) & \(\frac{\delta(r-\rho)\delta(z-\zeta)}{2\pi r}\) & \(\frac{\delta(r-\rho)}{2\pi r}\) \\ \hline
    כדוריות & \(\frac{\delta(r-\rho)\delta(\theta-\theta^{\prime})\delta(\varphi-\varphi^{\prime})}{r^{2}sin(\theta)}\) & \(\frac{\delta(r-\rho)\delta(\theta-\theta^{\prime})}{2\pi r^{2}sin(\theta)}\) & \(\frac{\delta(r-\rho)}{4\pi r^{2}}\) \\ \hline
  \end{tabular}
\end{table}
\begin{proposition}[הנוסחא הראשונה של גרין]
$$\iiint_{V}\nabla\varphi\cdot\nabla\chi\,d V+\iiint_{V}\varphi\nabla^{2}\chi\,d V={\subset\!\supset} \mathllap{\iint}_{S}\varphi\left( \nabla\chi\cdot\mathbf{n} \right)\,d S,$$

\end{proposition}
\begin{proof}
נובעת מהזהות הווקטורית:
$$\nabla\cdot(\varphi\nabla\chi)=\nabla\varphi\cdot\nabla\chi+\varphi\nabla^{2}\chi.$$
כאשר אם נבצע אינטגרציה מעל נפח סגור נקבל:
$$\iiint_{V}\nabla\cdot(\varphi\nabla\chi)\ d V=\iiint_{V}\left(\nabla\varphi\cdot\nabla\chi+\varphi\nabla^{2}\chi\right)\ d V.$$
כאשר שימוש במשפט הדיברגנץ של גאוס יתן את הנוסחא.

\end{proof}
\begin{proposition}[הנוסחא השנייה של גרין]
$$\iiint_{V}\left(\varphi\nabla^{2}\chi-\chi\nabla^{2}\varphi\right)\,d V={\subset\!\supset} \mathllap{\iint}_{S}\left(\varphi\nabla\chi-\chi\nabla\varphi\right)\cdot\mathbf{n}\,d S,
$$

\end{proposition}
\begin{proof}
נתחיל מהנוסחא הראשונה:
$$\iiint_{V}\nabla\varphi\cdot\nabla\chi\,d V+\iiint_{V}\varphi\nabla^{2}\chi\,d V={\subset\!\supset} \mathllap{\iint}_{S}\varphi(\nabla\chi\cdot\mathbf{n})\,d S,$$
כאשר נכתוב באופן דומה:
$$\iiint_{V}\nabla\chi\cdot\nabla\varphi\,d V+\iiint_{V}\chi\nabla^{2}\varphi\,d V={\subset\!\supset} \mathllap{\iint}_{S}\chi\left( \nabla\varphi\cdot\mathbf{n} \right)\,d S.$$
כעת אם נחסיר את המשוואה השנייה מהראשונה נקבל את הנוסחא השנייה.

\end{proof}
\begin{remark}
נוסחאות גרין למעשה נותנות לנו את הקשר בין מקורות(הלפלאסיאנים) והשטף של שתי שדות סקלארים.

\end{remark}
\begin{corollary}
פונקציה אשר חסומה על ידי משפט חסום \(S\) נתונה על ידי:
$$u\left( \xi,\eta,\zeta \right)={\frac{1}{4\pi}}{\subset\!\supset} \mathllap{\iint}_{S}\left[u\,\nabla\!\left({\frac{1}{r}}\right)-{\frac{1}{r}}\,\nabla u\right]\cdot\mathbf{n}\,d S-{\frac{1}{4\pi}}\iiint_{V}{\frac{1}{r}}\,\nabla^{2}u\,d V.$$

\end{corollary}
\begin{proof}
נשתמש בנוסחא השנייה של גרין על ספרה סגורה \(S\) מסביב לנקודה \((\xi,\eta,\zeta)\) כאשר:
$$\varphi=r^{-1} \quad \varphi=r^{-1}\quad r^{2}=\left( x{-}\xi \right)^{2}{+}\left( y{-}\eta \right)^{2}{+}\left( z{-}\zeta \right)^{2}$$
כיוון שלא ניתן להשתמש בנוסחאות גרין אם הנקודה \(\left( \xi,\eta,\zeta \right)\) כלולה בנפח הסגור נגדיר ספרה קטנה עם שפה \(\Sigma\) ורדיוס \(\varepsilon\) שמרכזו ב-\((\xi,\eta,\zeta)\). נציב את \(\varphi\) ו-\(\chi\) בנוסחאת גרין השנייה:
\begin{gather*}\iiint_{V}{\frac{1}{r}}\,\nabla^{2}u\,d V={\subset\!\supset} \mathllap{\iint}_{S+\Sigma}\left[u\,\nabla\!\left({\frac{1}{r}}\right)-{\frac{1}{r}}\,\nabla u\right]\cdot\mathbf{n}\,d S=\\=\oint_{S}\left[u\,\nabla\!\left({\frac{1}{r}}\right)\,-\,{\frac{1}{r}}\,\nabla u\right]\cdot\mathbf{n}\,d S+\oint_{\Sigma}\left[u\,\nabla\!\left({\frac{1}{r}}\right)-{\frac{1}{r}}\,\nabla u\right]\cdot\mathbf{n}\,d S 
\end{gather*}
כאשר השתמשנו בכך ש-\(\bar{\nabla}^2(r^{-1})=0\) מעל נפח סגור. מתקיים:
$$(*)\quad {\subset\!\supset} \mathllap{\iint}_{\Sigma}u\,\nabla\!\left({\frac{1}{r}}\right)\cdot\mathbf{n}\,d S=-\int_{0}^{2\pi}\int_{0}^{\pi}u\left( \epsilon,\varphi,\theta \right)\,\,{\frac{1}{\epsilon^{2}}}\,\,\epsilon^{2}\sin\left( \theta \right)\,d\theta\,d\varphi,$$
כאשר עבור:
$$(* *)\quad {\subset\!\supset} \mathllap{\iint}_{\Sigma}{\frac{1}{r}}\,\nabla u\cdot\mathbf{n}\,d S=\epsilon\int_{0}^{2\pi}\,\int_{0}^{\pi}\,\nabla u\left( \epsilon,\varphi,\theta \right)\ \sin\left( \theta \right)\,d\theta\,d\varphi.$$
נקבל כי בגבול \(\varepsilon\to 0\) המשוואה \((* *)\) מתאספת כי \(\bar{\nabla}u\) סופי כאשר המשוואה \((*)\) תתן:
$${\subset\!\supset} \mathllap{\iint}_{\Sigma}u\,\nabla\!\left({\frac{1}{r}}\right)\cdot\mathbf{n}\,d S=-4\pi u\left( \xi,\eta,\zeta \right).$$
כך שנקבל סה"כ כי:
$$u\left( \xi,\eta,\zeta \right)={\frac{1}{4\pi}}{\subset\!\supset} \mathllap{\iint}_{S}\left[u\,\nabla\!\left({\frac{1}{r}}\right)-{\frac{1}{r}}\,\nabla u\right]\cdot\mathbf{n}\,d S-{\frac{1}{4\pi}}\iiint_{V}{\frac{1}{r}}\,\nabla^{2}u\,d V.$$

\end{proof}
\begin{corollary}
ניתן לקבל כל ערך בפנים של \(S\) אם נתון:

  \begin{enumerate}
    \item הלפלסיאן בתוך הנפח. 


    \item הערך של \(u\) בכל נקודה על השפה. 


    \item הגרדיאנט של \(u\) לאורך הנורמל של \(S\) בכל נקודה על \(s\). 


  \end{enumerate}
\end{corollary}
\begin{proposition}
מהנוסחא נקבל כי אם \(u\) היא הרמונית אז \(\bar{\nabla}^2u=0\)  ונקבל:
$$u\left( \xi,\eta,\zeta \right)=\frac{1}{4\pi}{\subset\!\supset} \mathllap{\iint}_{S}\left[u\,\nabla\!\left(\frac{1}{r}\right)-\frac{1}{r}\,\nabla u\right]\cdot\mathbf{n}\,d S.$$
ומספיק לדעת את \(u\) על השפה ואת הגרדיאנט של \(u\) על השפה. 

\end{proposition}
\begin{corollary}[יחידות הפתרון של בעיית לפלס]
נניח כי \(u_{1},u_{2}\) שתיהן המשכה אנליטית(פתרונות של משוואת לפלס). לכן \(v=u_{1}-u_{2}\) היא גם הרמונית וכן \(v=0\) על השפה. לכן מהנוסחא הראשונה של גרין עבור \(\varphi=\chi=v\) נקבל:
$$\iiint_{V}\left[\left({\frac{\partial v}{\partial x}}\right)^{2}+\left({\frac{\partial v}{\partial y}}\right)^{2}+\left({\frac{\partial v}{\partial z}}\right)^{2}\right]\,d V={\subset\!\supset} \mathllap{\iint}_{S}v{\frac{\partial v}{\partial n}}\,d S-\iiint_{V}v\nabla^{2}v\,d V.$$
כאשר כיוון שגם \(v=0\) על השפה וגם \(\bar{\nabla}^2v=0\) נקל כי האינטגרנד באגף שמאל מתאפס לכל נפח ולכן:
$${\frac{\partial v}{\partial x}}={\frac{\partial v}{\partial y}}={\frac{\partial v}{\partial z}}=0,$$
ו-\(v\) קבוע. לכן נדרש כי אפס בתוך הנפח ו-\(u_{1}=u_{2}\).

\end{corollary}
\begin{remark}
הוכחה זו תקפה גם עבור תנאי שפה דיריכלה כיוון שהאינטגרל עדיין מתאפס גם אם רק הנגזרת שווה ל-0.

\end{remark}
\end{document}