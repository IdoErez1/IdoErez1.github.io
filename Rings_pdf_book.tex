\documentclass{tstextbook}

\usepackage{amsmath}
\usepackage{amssymb}
\usepackage{graphicx}
\usepackage{hyperref}
\usepackage{xcolor}

\begin{document}

\title{Example Document}
\author{HTML2LaTeX Converter}
\maketitle

\chapter{הגדרת החוג}

\section{חוגים}

\begin{definition}[חוג]
אוסף \(\left( R,+,\cdot \right)\)  כאשר:

  \begin{enumerate}
    \item הקבוצה \(R\) ביחד עם חיבור יוצר חבורה חיבורית, עם איבר ניטרלי אפס. 


    \item הקבוצה \(R\) ביחד עם כפל היא אסוצייטיבית  


    \item מכיל יחידה כפלית 1(לא תנאי הכרחי בחלק מההגדרות) 


    \item מקיים שתי חוקי פילוג: 
$$r(a+b)=r\cdot a+r\cdot b\quad (a+b)\cdot r=a\cdot r+b\cdot r$$


  \end{enumerate}
\end{definition}
נשים לב כי אין בהכרח הופכי כפלי. חוג עם הופכי לכפל נקרא חוג חילוק. יש דבר כזה חוגים ללא יחידה כפלית. אך אנחנו לא מתעניין בהם כעת.

\begin{example}
  \begin{enumerate}
    \item המספרים השלמים \(\mathbb{Z}\) עם החיבור והכפל הרגילים. 


    \item החיבור והכפל המודולרי \(\mathbb{Z} _n\). 


    \item שדה כללי \(\mathbb{F}\). 


    \item פולינומים \(\mathbb{F}[x]\), \(\mathbb{F}[x,y]\). 


    \item פולינום עם חוג אחר \(R[x]\) כש-\(R\) חוג. 


    \item מטריצות \(M_{n\times n}\left( \mathbb{F} \right)\), \(M_{n\times n}(R)\)(כש-\(R\) חוג) 


  \end{enumerate}
\end{example}
\begin{remark}
עבור \(M_{n\times n}(R)\) לא בהכרח אי-הפיכות גורר דטרמיננטה 0.

\end{remark}
\begin{definition}[החוג הטריוויאלי]
החוג המוגדר על ידי:
$$R=\{ 0 \}$$
כאשר \(0+0=0, 0\cdot 0 =0\).

\end{definition}
\begin{definition}
אם \(R\) חוג ו-\(X\) קבוצה אז \(R^X\) (פונקציות \(X\to R\)) יהיה חוג עם הפעולות
$$\begin{gathered}(f+f')(x)=f(x)+f'(x) \\\left( f\cdot f' \right)(x)=f(x)\cdot f'(x)
\end{gathered}$$

\end{definition}
אם \((A,+)\) חבורה אבלית. \(\mathrm{End}(A)\) הוא חוג ביחס ל- 
$$\begin{gathered}(f+f')(a)=f(a)+f(a') \\\left( f\cdot f' \right)(a)= f(f'(a)) \\0=\left( a\mapsto 0_{A} \right)\qquad 1=\mathrm{Id}_{A}
\end{gathered}$$
הפילוג פה יותר מעניין. הפילוג מצריך את האבליות! מושאר כתרגיל להראות פילוג.

\begin{example}
עבור \(A=\mathbb{Z} _p^n\) כאשר \(\mathrm{End}(A)=M_{n\times n}\left( \mathbb{F}_{p} \right)\).

\end{example}
\begin{proposition}[משפט קיילי]
כל חוג איזומורפי לתת חוג של \(\mathrm{End}(A)\) לחבורה אבלית \(A\) כלשהי.

\end{proposition}
\begin{proposition}[תכונות של חוגים]
יהיו \(a,b,c \in R\). אזי מתקיים:

  \begin{enumerate}
    \item \(a 0=0a=0\)


    \item \(a(-b)=(-a)b=-(ab)\)


    \item \((-a)(-b)=ab\)


    \item \(a(b-c)=ab-ac\) וגם \((b-c)a=ba-ca\)


    \item \((-1)a=-a\)


    \item \((-1)(-1)=1\)


  \end{enumerate}
\end{proposition}
\begin{definition}
איבר \(r\in R\) נקרא הפיך אם קיים \(s \in R\) כך ש-\(rs=sr=1\). 

\end{definition}
\begin{proposition}
אם קיים ל-\(r\) הופכי הוא יחיד, ומסומן \(r^{-1}\).

\end{proposition}
\begin{example}
נסתכל על \(End\left( \mathbb{Z} ^{\mathbb{N} } \right)\). סדרות של שלמים זה חבורה אבלית. נסתכל על האנדומורפיזם שלה. עבור \(T\in End\left( \mathbb{Z} ^{\mathbb{N} } \right)\) המוגדר:
$$\begin{gathered}T\left( \left( a_{1},a_{2},\dots \right) \right)=\left( a_{2},a_{3},\dots \right) \\S\left( \left( a_{1},a_{2},\dots \right) \right)=\left( 0,a_{1},a_{2},\dots \right)
\end{gathered}$$
כאשר נשים לב כי \(S\circ T = id\) אך \(T\circ S \neq id\). זה הדוגמא הסטנדרטית ללמה הופכי מימין לא דווקא ההופכי משמאל.

\end{example}
\begin{proposition}
אוסף ההפיכים ב-\(R\) היא חבורה, מסומנת \(R^\times\).

\end{proposition}
\begin{example}
$$M_{n\times n}\left( \mathbb{F} \right)^\times=GL_{n}\left( \mathbb{F} \right)\qquad \mathbb{Z} ^\times=\left\{  \pm 1  \right\}\qquad \mathbb{F}[x]^\times=\mathbb{F}^\times$$

\end{example}
\begin{definition}[הומומורפיזם של חוגים]
פונקציה \(f:R\to S\) שעבורה משמר חיבור, כפל ויחידה: 
$$\begin{gathered}f(x+y)=f(x)+f(y) \\f\left( x\cdot y \right)=f(x)\cdot f(y) \\f(1)=1
\end{gathered}$$

\end{definition}
\begin{example}
קיים הומומורפיזם יחיד \(f:\mathbb{Z} \to R\) עבור כל חוג! זה יהיה
$$f(m)=\begin{cases}\underbrace{ 1+\dots+1 }_{ m }  & m\geq 0\\-\underbrace{ \left( 1+\dots+1 \right) }_{ m  } & m<0
\end{cases}$$

\end{example}
\begin{example}
עבור \(f:\mathbb{R} \to M_{2\times 2}\left( \mathbb{R}  \right)\) המוגדר
$$f\left( \alpha \right)=\begin{pmatrix}\alpha & 0 \\0 & 0
\end{pmatrix}$$
הפונקציה \(f\) משמרת חיבור וכפל אבל לא יחידה!

\end{example}
\begin{example}
הפונקציה \(0:R\to \{ 0 \}\) היא הומומורפיזם ונקראת הומומורפיזם האפס. אם \(S\neq \{ 0 \}\) אזי \(0:R\to S\) היא לא הומומורפיזם(כי לא משמרת יחידה)

\end{example}
\begin{remark}
ברגע שיש הומומורפיזמים, יש גם איזומורפיזים, מונומורפיזם, אפימורפיזם וכו.

\end{remark}
\begin{proposition}
לחוג יש איבר יחידה כפלי יחיד.

\end{proposition}
\begin{definition}[חוג קומוטטיבי]
חוג אשר גם הפעולה של הכפל היא קומוטטיבית. כלומר \(\forall a,b \in R\quad a\cdot b=b\cdot a\)

\end{definition}
\begin{definition}[חוק חילוק]
חוג אשר גם לכל איבר פרט לאיבר הניטרלי החיבורי יש הוכפי כפלי. כלומר \(\forall a \in R\quad a^{-1}\in R\).

\end{definition}
\begin{definition}[שדה]
חוג חילוק קומוטטיבי.

\end{definition}
\section{תת חוג}

\begin{definition}[תת חוג]
קבוצה לא ריקה \(S\) של \(R\) נקראת תת חוג אם \(S\) מהווה תת חבורה חיבורית תחת חיבור(כלומר סגורה לחיבור ולהופכי חיבורי) וכן סגורה תחת כפל, כלומר אם \(a,b\in R\) אז \(a\cdot b \in R\). מסומן \(S\leq R\).

\end{definition}
\begin{example}
$$\{ 0 \}\not\leq \mathbb{Z} \qquad \mathbb{F},\mathbb{F}[x^2]\leq \mathbb{F}[x]$$

\end{example}
\begin{reminder}
כשהיינו בחבורות, דיברנו על קבוצה \(X\) ו-\(S_{X}\) חבורה. פעולה של \(G\) על \(X\) שקול ל-\(\rho:G\to S_{X}\) הומומורפיזם(המבנה).

\end{reminder}
\begin{definition}[פעולה על חוג]
עבור \(A\) חבורה אבלית, \(\mathrm{End}(A)\) חוג. פעולה של חוג \(R\) על \(A\)(\(R \circlearrowright A\)): $$\rho:R\to \mathrm{End}(A)$$ כאשר \(\rho\) זה הומומורפיזם של חוגים. נתרגם לפעולה \(\cdot:R\times A\to A\)$$\begin{gathered}(r+r')a=ra+r'a\qquad (rr')a=r(r'a) \\r(a+a')=ra+ra'\qquad 1a=a
\end{gathered}$$

\end{definition}
עבור שדה \(\mathbb{F}\), מרחב וקטורי מעל \(\mathbb{F}\) שקול לחבורה אבלית עם פעולה של \(\mathbb{F}\). אמנם במרחבים וקטורים היו יותר מ-\(4\) אקסיומות. זאת כיוון שחלק מהאקסימות נדרשו בשביל האבליות.

\begin{remark}
לא קוראים ל-\(A\) עם פעולה של חוג \(R\) מרחב וקטורי מעל \(R\). קוראים להם מודול.

\end{remark}
כעת ניתן להוכיח את משפט קיילי לחוגים

\begin{reminder}[משפט קיילי לחוגים]
כל חוג איזומורפי לתת חוג של \(\mathrm{End}(A)\) לחבורה אבלית \(A\) כלשהי.

\end{reminder}
\begin{proof}
ניקח \(A=R^+\). אזי \(R\circlearrowright A\) ע\"י \(r.a=ra\). נקבל:
$$\rho:R\to \mathrm{End}(R^+)$$
נקבל כי \(\rho\) חח"ע אם"ם:
$$\left\{  r \mid \rho(r)=0  \right\}=\ker\left( \rho \right)=\{ 0 \}$$
כעת:
$$r\in \ker\left( \rho \right):\rho(r)=0\quad r=r\cdot 1=\rho(r)(1)=0(1)=0$$

\end{proof}
תת חוג \(S\leq R\) יקיים \(1\in S\), \(S\pm S, S\cdot S\subseteq S\) אידיאל \(I\trianglelefteq R\) יקיים \(RIR\subseteq I\), \(I+I\).(\(I=R\iff 1\in I\))

\begin{example}
עבור \(m\mathbb{Z} \trianglelefteq \mathbb{Z}\). \(M_{n}\left( \mathbb{F} \right)\trianglelefteq M_{n}\left( \mathbb{F} \right)\), \(\{ 0 \}\) ורק הם.

\end{example}
\section{הומומורפיזם של חוגים}

\begin{definition}[הומומורפיזם של חוגים]
יהיו \(S,R\) חוגים. הומומורפיזם של חוגים זו העתקה \(\varphi:S\to R\) כך שמתקיים:

  \begin{enumerate}
    \item זהו הומומורפיזם של החבורה החיבוריות 


    \item משמר כפל, כלומר \(\varphi\left( s_{1}\cdot s_{2} \right)=\varphi(s_{1})\cdot \varphi(s_{2})\). 


    \item מעביר יחידה ליחידה - \(\varphi(1)=1\). 


  \end{enumerate}
\end{definition}
\begin{definition}[גרעין של הומומורפיזם]
אוסף האברים בחוג \(R\) שההומומורפיזם \(\varphi\) שולח ל-0. כלומר:
$${\mathrm{ker}}\,\left( \varphi \right)=\left\{r\in R\,|\,\varphi\,(r)=0\right\}$$

\end{definition}
\begin{definition}[תמונה של הומומורפיזם]
אוסף כל האיברים שההומומורפיזם מגיע עליהם:
$${\mathrm{Im}}\left( \varphi \right)=\left\{\varphi\left(r\right)\left|\,r\in R\right\}\right.$$

\end{definition}
\begin{proposition}
התמונה של הומומורפיזם היא תת חוג, והגרעין של הומומורפיזם היא אידיאל.

\end{proposition}
\begin{proposition}[משפט האיזומורפיזם הראשון של חוגים ]
נניח \(\phi:R\to S\) הומומורפיזם של חוגים \(R,S\). אז ההעתקה:
$$R / \ker \phi \to \mathrm{Im}\left( \phi \right)$$
היא איזומורפיזם בין חוגים

\end{proposition}
\section{אידיאלים}

\begin{definition}[אידיאל]
תת חוג \(I\) של \(R\) נקרא אידיאל(דו צדדי) אם לכל \(r\in R\) ולכן \(i \in I\) מתקיים \(ri,ir\in I\)

\end{definition}
\begin{proposition}[מבחן האידיאל]
קבוצה לא ריקה \(I\) הוא אידיאל של החוג \(R\) אם:

  \begin{enumerate}
    \item אם \(a,b \in I\) גורר \(a-b\in I\)


    \item כאשר \(i \in I\) ו-\(r \in R\) גורר \(ir,ri \in I\)


  \end{enumerate}
\end{proposition}
נשים לב כי תנאי אחד מראה כי תת חבורה חיבורית, ותנאי 2 מראה שזה אידיאל

\begin{remark}
  \begin{enumerate}
    \item החוג כולו \(R\) והאידיאל הטריוויאלי \(\{ 0 \}\) הם תמיד אידיאלים 


    \item מסומן \(I\trianglelefteq R\). 


    \item האידיאל יכיל את היחידה רק אם הוא החוג כולו. 


  \end{enumerate}
\end{remark}
כעת ננסה להגדיר חוגי מנה כמו בחבורות. 

\begin{proposition}[חוג מנה]
יהי \(R\) חוג, ו-\(A\) תת חוג של \(R\). הקבוצה:
$$\left\{  r+A\mid r\in R  \right\}$$
יוצרת תת חוג ביחד עם הפעולה \((s+A)+(t+A)=s+t+A\) ו-\((s+A)\cdot(t+A)=st+A\) אם"ם החוג \(A\) הוא אידיאל.

\end{proposition}
\begin{definition}[חוג פשוט]
חוג נקרא פשוט אם האידיאלים היחידים בו הם \((0)=0\) ו-\((1)=R\).

\end{definition}
\begin{proposition}
כל חוג חילוק פרט לחוג הטריוויאלי הוא פשוט

\end{proposition}
\begin{proposition}
עבור חוג קומוטטיבי \(R\), \(R\) הוא פשוט אם"ם הוא שדה.

\end{proposition}
\begin{proposition}
לכל שדה \(F\) ו-\(n\) טבעי, חוג המטריצות \(M_{n}(F)\) פשוט.

\end{proposition}
\section{תחום שלמות}

\begin{definition}[מחלק אפס]
איבר \(a\neq 0\) נקרא מחלק אפס של חוג \(R\) אם קיים \(0\neq b \in R\) ומקיים \(a\cdot b =0\)

\end{definition}
\begin{definition}[תחום שלמות]
חוג קומוטטיבי עם יחידה וללא מחלק אפס נקרא חוק שלמות.

\end{definition}
\begin{definition}[פריקות]
יהא \(D\) תחום שלמות. אז \(r \in D\) יקרא פריק אם קיים \(a,b \in \mathbb{R} \setminus \mathbb{R}^\times\) כך ש-\(r=a\cdot b\).

\end{definition}
\begin{example}
החוגים הבאים הם חוגי שלמות:

  \begin{enumerate}
    \item חוג המספרים שלמים 


    \item חוג המספרים הגאוסים \(Z[i]=\{a+b i\mid a,b\in Z\}\). 


    \item חוג הפולינומים עם מקדמים שלמים \(Z[x]\). 


    \item החוג \(Z[{\sqrt{2}}]=\{a+b{\sqrt{2}}\mid a,b\in Z\}\). 


    \item החוג המודולארי \(\mathbb{Z}_{p}\) כאשר \(p\) ראשוני. 


  \end{enumerate}
\end{example}
\begin{proposition}[חוק הביטול]
יהיו \(a,b,c\) בחוג שלמות. אם \(a\neq 0\) ו-\(ab=ac\) אז \(b=c\).

\end{proposition}
\begin{proposition}
תחום שלמות סופי הוא שדה.

\end{proposition}
\chapter{חוג הפולינומים}

\section{פולינומים ופריקות}

\begin{definition}[חוג פולינומים]
חוג המסומן ב-\(R[x]\) כאשר 
$$R[x]=\{a_{n}x^{n}+a_{n-1}x^{n-1}+\cdot\cdot\cdot+a_{1}x+a_{0}\mid a_{i}\in R, x \in \mathbb{F} \} $$

\end{definition}
\begin{definition}[פריקות של פולינום]
יהי \(D\) תחום שלמות. פולינום \(f(x)\in D\) שהוא לא פולינום האפס ולכן פולינום היחידה נקרא פריק אם קיימים \(g(x),h(x)\in D\) כך ש-\(deg(g(x)),deg(h(x))>1\) ומקיימים \(f(x)=g(x)h(x)\).

\end{definition}
\begin{proposition}
אם יש לפולינום \(f(x)\in \mathbb{F}[x]\) שורש, אז הוא פריק. 

\end{proposition}
כאשר נשים לב כי הכיוון השני בדרך כלל לא נכון.

\begin{proposition}
לפולינום \(f(x)\in \mathbb{F}[x]\) מסדר 2 או 3 יש שורש אם"ם הוא פריק.

\end{proposition}
\section{מבחני פריקות}

\begin{proposition}[הלמה של גאוס]
יהא \(p(x)\in \mathbb{Z}[x]\) פולינום ממעלה חיובית. אז \(p(x)\) אי פריק מעל \(\mathbb{Z}[x]\) אם"ם הוא אי פריק מעל \(\mathbb{Q}\)

\end{proposition}
\begin{proposition}[קריטריון אייזנשטיין]
יהא \(\mathbb{Z}[x]\ni q(x)=a_{n}x^n+\dots+a_{0}\), \(n>0\).
אם קיים \(p\) ראשוני כך ש:

  \begin{enumerate}
    \item מתקיים \(p\nmid a_n\)


    \item מתקיים \(p\mid a_{i}\) לכל \(0\leq i<n\). 


    \item מתקיים \(p^2 \nmid a_{0}\) 
אז \(q(x)\) אי פריק


  \end{enumerate}
\end{proposition}
\begin{proposition}[קרטריון השורש הרציונאלי]
נתון פולינום מהצורה $$p(x)=a_{n}x^{n}+\ldots+a_{0}\in\mathbb{Z}[x]$$ אם \(p\left( \frac{a}{b} \right)=0\) ו-\(\frac{a}{b}\in \mathbb{ Q}\) מצומצם אזי \(a\mid a_{0}\) ו-\(b\mid a_{n}\).

\end{proposition}
\begin{proposition}[טרנספורמציה לינארית]
לכל \(a,b \in \mathbb{F}\) כך ש-\(a\neq 0\) אז \(p(x) \in \mathbb{F} [x]\) פריק אם"ם \(f(ax+b)\) פריק.

\end{proposition}
\section{תחום ראשי}

\begin{definition}[תחום ראשי]
תחום שבו כל אידיאל נוצר ע"י איבר יחיד.

\end{definition}
\begin{definition}[איבר ראשוני]
יהא \(R\) תחום שלמות. איבר \(r\in R\) לא הפיך יקרא ראשוני אם \(r \mid a\cdot b\) גורר \(r \mid a\) או \(r \mid b\).

\end{definition}
\begin{proposition}
בתחום ראשי \(R\). איבר הוא ראשוני אם"ם הוא אי פריק. 

\end{proposition}
\section{תחום שלמות}

\begin{definition}[מחלק אפס]
איבר \(a\neq 0\) נקרא מחלק אפס של חוג \(R\) אם קיים \(0\neq b \in R\) ומקיים \(a\cdot b =0\)

\end{definition}
\begin{definition}[תחום שלמות]
חוג קומוטטיבי עם יחידה וללא מחלק אפס נקרא חוק שלמות.

\end{definition}
\begin{example}
החוגים הבאים הם תחום שלמות:

  \begin{enumerate}
    \item חוג המספרים שלמים \(\mathbb{Z}\)


    \item חוג המספרים הגאוסים \(\mathbb{Z}[i]=\left\{ a+b i\mid a,b\in \mathbb{Z} \right\}\)


    \item חוג הפולינומים עם מקדמים שלמים \(\mathbb{Z}[x]\)


    \item החוג \(\mathbb{Z}\left[ {\sqrt{2}} \right]=\left\{ a+b{\sqrt{2}}\mid a,b\in \mathbb{Z} \right\}\)


    \item החוג המודולארי \(\mathbb{Z}_{p}\) כאשר \(p\) ראשוני 


  \end{enumerate}
\end{example}
\begin{example}
החוגים הבאים הם לא חוגי שלמות:

  \begin{enumerate}
    \item החוג המודולארי \(\mathbb{Z}_{n}\) כאשר \(n\) אינו ראשוני 


    \item חוג המטריצות \(M_{2}\left( \mathbb{Z} \right)\)


  \end{enumerate}
\end{example}
\begin{proposition}[חוג הביטול]
יהיו \(a,b,c\) בחוג שלמות. אם \(a\neq 0\) ו-\(ab=ac\) אז \(b=c\)

\end{proposition}
\begin{proof}
מ-\(ab=ac\) נקבל \(a(b-c)=0\). כיוון ש-\(a\neq 0\) בהכרח מתקיים \(b-c=0\) ולכן \(b=c\).

\end{proof}
\begin{proposition}
אם \(R\) תת חוג של שדה \(F\). אזי ב-\(R\) אין מחלקי אפס.

\end{proposition}
\begin{proof}
נניח \(xy=0\) ו-\(x\neq 0\). נראה כי \(y=0\):
$$y=\left(x^{-1}x\right)y=x^{-1}\left(x y\right)=x^{-1}\cdot0=0$$

\end{proof}
\begin{corollary}
כל תת חוג של שדה תחום שלמות.

\end{corollary}
\begin{proposition}
תחום שלמות סופי הוא שדה.

\end{proposition}
\chapter{מודולים}

\section{הגדרת ותכונות}

\begin{definition}[מודול]
יהי \(R\) חוג עם יחידה. מודול מעל R(לעיתים גם נקרא \(R\) מודול) יהיה חבורה אבלית \((M,+)\) ביחד עם כפל סקלארי \(R\times M\to M\) אשר מקיים לכל \(r,s \in R\) ו-\(m,n \in M\):

  \begin{enumerate}
    \item פילוגי מעל \(M\): 
$$r\cdot\left(m+n\right)=r\cdot m+r\cdot n$$


    \item פילוגי מעל \(R\): 
$$(r+s)\cdot m=r\cdot m+s\cdot m$$


    \item אסוצייטיבי: 
$$(rs)\cdot m=r\cdot(s\cdot m)$$


    \item יחידה: 
$$1\cdot m= m$$
מסומן \(R^{M}\).


  \end{enumerate}
\end{definition}
\begin{remark}
זה למעשה הכללה של מרחב ווקטורי, כאשר הווקטורי זה האיבר בחבורה האבלית והסקלר זה איבר בחוג. בפרט אם המודול הוא מעל שדה נקבל מרחב ווקטורי.

\end{remark}
\begin{remark}
זה למעשה נקרא מודול שמאלי. ניתן להגדיר מודול ימיני מוגדר באותו אופן עם סקלר מימין.

\end{remark}
כדי לקבל אינטאיציה על ההגדרה נזכר במשפט קיילי לחוגים.

\begin{proposition}
אם \(A\) חבורה אבלית אזי \(\text{End}(A)\) הוא חוג, כאשר חיבור מוגדר על ידי:
$$(f+g)(m)=f(m)+g(m)$$
וכן כפל מוגדר על ידי הרכבה:
$$(f\cdot g)(m)=f(g(m))$$

\end{proposition}
\begin{reminder}[משפט קיילי לחוגים]
כל חוג איזומורפי לתת חוג של \(\mathrm{End}(A)\) לחבורה אבלית \(A\) כלשהי.

\end{reminder}
\begin{proposition}[תנאי שקול למודול]
יהי \(M\) חוברה אבלית ו-\(\rho:R\to \mathrm{End}(M)\) הומומורפיזם של חוגים כאשר  \(\mathrm{End}(M)\) הוא החוג של כל האנדומורפיזמים של חבורה \(M\).
הפעולה של \(r \in R\) על \(m \in M\) נתונה על ידי \(\rho(r)(m)\).

\end{proposition}
\begin{symbolize}
מסמנים \(R^{M}\) כדי לסמן ש-\(M\) הוא \(R\) מודול.

\end{symbolize}
\begin{example}
המודול \(R^{R}\) זה המודול שהחבורה האבלית היא \(R\) בעצמה, וכפל בסקלר זה מכפלת החוג.

\end{example}
\begin{definition}[סכום ישר של מודולים]
אם \(M\) ו-\(N\) הם \(R\) מודולים, אז הסכום הישר שלהם מוגדר על ידי:
$$M\oplus N=\{(m,n)\mid m\in M,\,n\in N\}$$
כאשר כפל בסקלר מוגדר רכיב רכיב:
$$r\cdot(m,n)=(r\cdot m,\,r\cdot n)$$

\end{definition}
\begin{remark}
הגדרה זו משמרת את ההגדרה הידועה עבור שדות ווקטורים.

\end{remark}
\begin{definition}[תת מודול]
יהי \(R^{M}\) מודול. תת קבוצה \(N\subseteq M\) נקרא תת מודול אם:
- לא ריקה.
- סגורה תחת חיבור. כלומר לכל \(n_{1},n_{2} \in N\) נקבל \(n_{1}+n_{2} \in N\).
- סגורה תחת כפל בסקלר - כלומר לכל \(r \in R\) ו-\(n \in N\) נקבל \(r\cdot n \in N\).

\end{definition}
\begin{remark}
זה בדיוק ההגדרה שהייתה לנו לתת מרחב ווקטורי. 

\end{remark}
\begin{remark}
בפרט מקיים את הנגדי כיוון שמתקיים:
$$(-1)\cdot n+n=(-1)n+1\cdot n=(-1+1)\cdot n=0\cdot n=0$$

\end{remark}
\begin{proposition}
אם \(I\) אידיאלי שמאלי אז בפרט תת חוג ולכן \(R^{I}\) היא מודול.

\end{proposition}
\begin{example}
יהי \(R\) חוג. ניתן לקחת עבור החוג \(M_{3}(R)\) והאידיאל השמאלי \(I=\begin{pmatrix}0& \star &\star\\0& \star &\star\\0& \star &\star\\\end{pmatrix}\) ולקבל את המודול\(M_{3}(R)^{\begin{pmatrix}0& \star &\star\\0& \star &\star\\0& \star &\star\\\end{pmatrix}}\). 

\end{example}
\begin{definition}[\(\mathbb{Z}\) מודול]
מודול של חבורה אבלית \(M\) מעל החוג \(\mathbb{Z}\) אשר כפל בסקלר מוגדר על ידי:
$$n\cdot m=(\underbrace{1+1+\cdot\cdot\cdot+1}_{n\;\mathrm{times}})\cdot m=\underbrace{m+m+\cdot\cdot\cdot+m}_{n\;\mathrm{times}}$$
עבור \(n> 0\) כאשר נגדיר \(0\cdot m = 0\) נקרא \(\mathbb{Z}\) מודול.

\end{definition}
\begin{proposition}
עבור \(\mathbb{Z}\) מודול הכפל מקיים:
$$(-n)\cdot m=-(m+\dots+m)$$

\end{proposition}
\begin{corollary}
לכל חבורה אבלית \(M\) יש מבנה יחיד כ-\(\mathbb{Z}\) מודול, כאשר הכפל מוגדר על ידי:
$$n\cdot m={\left\{\begin{array}{l l}{m+m+\cdot\cdot\cdot+m}&{n>0}\\ {0}&{n=0}\\ {-{\big(}m+m+\cdot\cdot\cdot+m{\big)}}&{n<0}\end{array}\right.}$$

\end{corollary}
\begin{remark}
זה מתקשר לכך שקיים חוג יחיד(כד כדי איזומורפיזם) של הקבוצה \(\mathbb{Z}\).

\end{remark}
\begin{definition}[מודול מנה]
אם \(N\) הוא תת מודול של \(M\) אז החוג מנה מותר בתור קבוצת הקוסטים:
$$M/N=\{m+N\mid m\in M\}$$
כאשר חיבור מוגדר על ידי:
$$(m+N)+(m^{\prime}+N)=(m+m^{\prime})+N$$
וכפל בסקלר מוגדר על ידי:
$$r\cdot(m+N)=(r\cdot m)+N$$

\end{definition}
נוכיח כי מוגדר היטב.

\begin{proof}
אם \(m+N=m'+N\) אז:
$$r m - r m' = r(\underbrace{ m - m' }_{ N })$$
ולכן:
$$r m +N= r m' +N$$
עבור \(m\mathbb{Z}\leq \mathbb{Z}\) מודול מנה מקבלים \(\mathbb{Z} / m\mathbb{Z}\).

\end{proof}
\begin{definition}[הומומורפיזם של מודולים]
פונקציה \(f:M\to N\) בין \(R\) מודולים אשר:
- משמרת חיבור - כלומר לכל \(m,m' \in M\) מתקיים:
$$f(m+m^{\prime})=f(m)+f(m^{\prime})$$
- משמרת כפל בסקלר, כלומר לכל \(r \in R\) ו-\(m \in M\) מתקיים:
$$f(r\cdot m)=r\cdot f(m)$$

\end{definition}
\begin{remark}
זה למעשה מה שקראנו לו טרנספורמצייה לינארית במקרה של מרחבים ווקטורים.

\end{remark}
\begin{proposition}[משפט האיזומורפיזם הראשון]
אם \(f:M\to N\) הוא הומומורפיזם של \(R\) מודולים אזי:
- הגרעין \(\ker f = \{ m \in M\mid f(m)=0 \}\) הוא תת מודול של \(M\).
- התמונה \(\mathrm{Im}f\) היא תת מודול של \(M\).
- קיים איזומורפיזם:
$$M/\ker f\cong\operatorname{Im}f$$
אשר נתון על ידי ההעתקה \(m+\ker f\mapsto f(m)\).

\end{proposition}
\begin{proposition}[משפט האיזומורפיזם השני]
אם \(N\) ו-\(M\) הם תתי מודולים של מודול \(L\) אזי:
$$\frac{N}{N\cap M}\cong\frac{N+M}{M}$$

\end{proposition}
\begin{proposition}[משפט האיזומורפיזם השלישי]
אם \(K\subseteq M \subseteq N\) הם תתי מודולים של מודול \(R\) אזי:
$${\frac{N/K}{M/K}}\cong{\frac{N}{M}}.$$

\end{proposition}
\begin{proposition}[משפט האיזומורפיזם הרביעי / משפט ההתאמה]
קיים התאמה חח"ע בין תתי מודולים של \(M / N\) לבין תתי מודולים של \(M\) אשר מכילות את \(N\).

\end{proposition}
\section{מודולים פשוטים}

\begin{definition}[מודול פשוט]
מגול \(M\) מעל \(R\) הוא פשוט אם התתי מודולים היחידים שלו הם 0 ו\(M\).

\end{definition}
\begin{proposition}
מודול \(R^{M}\) הוא פשוט אם \(M\neq 0\) ולכל תת מודול \(N\leq M\) מתקיים \(N=0\) או \(N=M\)

\end{proposition}
\begin{definition}[מודול ציקלי]
מודול \(M\) נקרא ציקלי אם קיים \(m \in M\) כך ש:
$$M=R\,m=\{\,r\,m:r\in R\}.$$
כאשר במקרה זה כותבים \(M=\langle m \rangle\).

\end{definition}
\begin{example}
החבורה הציקלית \(C_{8}= \mathbb{Z} / 8\mathbb{Z}\) היא \(\mathbb{Z}\)-מודול ציקלי הנוצרת על ידי 1.

\end{example}
\begin{proposition}
כל מודול ציקלי הוא מנה של בחוג. כלומר אם \(M=\langle m \rangle\) ניתן להגדיר:
$$f:R\,\longrightarrow\,M,\quad f(r)=r\,m$$

\end{proposition}
\begin{proof}
מהמשפט האיזומורפיזם הראשון מתקיים:
$$M\,\cong\,R/\ker f.$$
כאשר \(\ker f=\{\,r\in R:r\,m=0\}\) הוא אידיאל שמאלי של \(R\) ולכת תת מודול. לחלופין לכל אידיאל.

\end{proof}
\begin{remark}
זה מקביל לפעולה נאמנה בחבורות.

\end{remark}
\section{מודולים ומרחבים ווקטורים}

\begin{proposition}
אם \(M\) מודול נוצר סופית מעל שדה \(\mathbb{F}\) אז \(M\cong \mathbb{F}^{n}\).

\end{proposition}
\begin{proof}
מודול נוצר סופית מעל שדה הוא מרחב ווקטורי, וניתן תמיד למצוא בסיס \(\mathcal{B}\). ההעתקה \(v\mapsto[v]_{\mathcal{B}}\) היא איזומורפיזם ולכן נקבל כי איזומורפי ל-\(\mathbb{F}^{n}\).

\end{proof}
\begin{remark}
זה לא נכון עבור חוגים כלליים. למשל עבור \(R=\mathbb{Z}\) ו-\(M=\mathbb{Z} / 3\mathbb{Z}\) נקבל כי \(\mathbb{Z}\) לא איזומורפי ל-\(\mathbb{Z}^{m}\) כיוון ש-\(\mathbb{Z}^{n}\) הוא אינסופי אבל \(\mathbb{Z} / 3\mathbb{Z}\) הוא סופי.

\end{remark}
\begin{reminder}
במרחבים ווקטורים אם \(W\leq V\) הוא תת מרחב תמיד קיים משלים \(U\leq V\) כך ש-\(V=W\oplus U\).

\end{reminder}
\begin{remark}
זה לא נכון תמיד לגבי מודולים. למשל אם \(R=\mathbb{Z}\) ו-\(M=C_{4}\) אז עבור תת מודול \(N=\langle 2 \rangle=\{ 0,2 \}\) לא נקבל משלים. זאת כי התתי מודולים של \(M\) יהיו \(0,N,M\) ולא קיים משלים כיוון שאף אחד מהם לא מקיים \(M=N\oplus U\) עבור כל תת מודול \(U\).

\end{remark}
\begin{definition}[חוג פשוט למחצה]
חוג שלכל תת מודול שלו יש משלים.

\end{definition}
\begin{reminder}
במרחבים ווקטורים כל קבוצה פורשת מינימלית הוא בסיס, וכן כל קבוצה בת"ל מקסימלית תהיה בסיס.

\end{reminder}
\begin{remark}
לא תמיד נכון לגבי מודולים. למשל עבור \(R=\mathbb{Z},M=\mathbb{Z}\) הקבוצה \(\{ 2,3 \}\) פורשת את \(\mathbb{Z}\)(כיוון ש-\(3-2=1\)) אבל לא בת"ל:
$$3\cdot 2+(-2)\cdot 3 = 0$$
לעומת זאת הקבוצה \(\{ 2 \}\) היא בת"ל אבל לא פורשת את \(\mathbb{Z}\)(לא ניתן לקבל מספרים אי זוגיים).

\end{remark}
\begin{proposition}
מודול מעל \(\mathbb{F}[x]\) שקול למרחב ווקטורי \(V\) מעל \(\mathbb{F}\) עם אופרטור לינארי \(T:V\to V\) כך שהפעולה על \(x\) מוגדרת על ידי \(x \cdot v = T(v)\) והפעולה של פולינום תהיה מהצורה \(p(x)=a_{0}+a_{1}x+\dots + a_{n}x^{n}\) תהיה:
$$p(x)\cdot v=a_{0}v+a_{1}T(v)+\cdot\cdot\cdot+a_{n}T^{n}(v)$$
כלומר קיים הומומואפיזם \(\rho:\mathbb{F}[x]\to \mathrm{End}(V)\) אשר פועל כנאמר לעיל.

\end{proposition}
\begin{example}
עבור \(V=\mathbb{F}^{2}\) ו-\(T=\begin{pmatrix}0&1\\1& 0\end{pmatrix}\) נקבל:
$$x\cdot\left(v_{1},v_{2}\right)=(v_{2},v_{1}).$$
ועבור פילינום:
$$\left(x^{2}+1\right)\cdot\left(v_{1},v_{2}\right)=T^{2}(v)+v=\left(v_{1},v_{2}\right)+\left(v_{1},v_{2}\right)=\left(2v_{1},2v_{2}\right)$$

\end{example}
\section{מודולים חופשיים}

\begin{definition}[בסיס]
תת קבוצה \(S\subseteq M\) תקרא בסיס אם מקיימת:

  \begin{enumerate}
    \item פורשת - לכל \(m \in M\) ניתן לכתוב בצורה: 
$$m=\sum r_{i}s_{i}$$
כאשר \(s_{i}\in S,r_{i}\in R\).


    \item בת"ל - הצגה זו היא יחודית 


  \end{enumerate}
\end{definition}
\begin{definition}[מודול חופשי]
מודול שיש לו בסיס.

\end{definition}
\begin{definition}[סכום ישר]
עבור קבוצה \(S\) וחוג \(R\) נגדיר:
$$R^{\oplus S}=\{f\in R^{S}\mid f(s)=0\;{\mathrm{for~almost~all}}\;s\in S\}.$$
במילים אחרות כל איבר של \(R^{\oplus S}\) היא פונקציה מ-\(S\) ל-\(R\) שהיא לא אפס רק במספר סופי של ערכים.

\end{definition}
\begin{remark}
ההבדל בין הגדרה זו של סכום ישר ל-\(R^{S}\)(אוסף הפונקציות מ-\(f:S\to R\)) היא ש-\(R^{\oplus S}\) זה אוסף כל הפונקציות עם תומך סופי - כלומר מכיל מספר סופי של ערכים שינם אפס.

\end{remark}
\begin{example}[המודול \(\mathbb{R}^{\oplus \mathbb{N}}\)]
הסכום \(\mathbb{R}^{\oplus \mathbb{N}}\) מכיל את כל הסדרות מהצורה:
$$(a_{0},a_{1},\dots,a_{m}, 0,0,\dots)$$
כאשר הבסיס יהיה \(\{ e_{1} \}\) כאשר \(e_{i}(j)=\delta_{ij}\). כיוון שיש בסיס זה בפרט מודול חופשי.

\end{example}
\begin{example}[המודול \(\mathbb{Z}^{\mathbb{N}}\)]
עבור סופי אנו יודעים כי \(C_{3}=\mathbb{Z} / 3\mathbb{Z}\) - לא \(\mathbb{Z}\) מודול חופשי.
עבור המקרה הכללי הבעיה קצת יותר מסובכת. לפי Baer התשובה היא לא באופן כללי, אך לפי Specker כל תחבורה בת מנייה היא חופשית. בפרט ממשפט Specker מתקיים שאם \(\mathbb{Z}^{\oplus S}\cong \mathbb{Z}^{\mathbb{N}}\) אזי \(\lvert S \rvert> \aleph_{0}\).

\end{example}
\begin{definition}[אנדומורפיזמים של מודולים]
עבור \(R\) מודול \(M\) נגדיר:
$$\operatorname{End}_{R}(M)=\{\operatorname{All}\,R{\mathrm{-linear\;maps}}\;f:M\to M\}.$$
כאשר זה תת חוג של \(\mathrm{End}(M)\), האדומורפיזמים של \(M\) כחבורה אבלית

\end{definition}
\begin{example}[\(\mathrm{End}_{\mathbb{Z}}(\mathbb{Z})\)]
עבור \(\mathrm{End}_{\mathbb{Z}}(\mathbb{Z})\cong \mathbb{Z}\) לכל \(m\) ב-\(\mathbb{Z}\) נגדיר:
$$f_{m}(n)=mn$$
זה אנדומורפיזם:
\begin{gather*}f_{m}(n+n')=m(n+n')=f_{m}(n)+f_{m}(n')\\ f_{m}(m'n)=m(m'n)=(mm')n=m'mn=m'f_{m}(n)
\end{gather*}
מצד שני אם \(f \in \mathrm{End}_{\mathbb{Z}}(\mathbb{Z})\) אזי \(\varphi=f_{\varphi(1)}\) כי:
$$\varphi(n)=\varphi(n\cdot 1)=n\cdot \varphi( 1)=f_{\varphi(1)}(n)$$

\end{example}
\begin{corollary}
עבור חוג כללי \(R\) נקבל:
$$\mathrm{End}_{R}(R)\cong  R$$
לכל חוג קומוטטיבי \(R\).

\end{corollary}
\begin{proof}
זהה למה שעשינו בדוגמא עבור \(R=\mathbb{Z}\). יהי \(f \in \mathrm{End}_{R}(R)\). נגדיר \(f(1)=a\) ואז לכל \(r \in R\) נקבל:
$$f(r)=f(r\cdot1)=r\cdot f(1)=r\cdot a.$$
ולכן \(f\) מתאים לכפל ב-\(a\) ונקבל \(\mathrm{End}_{R}(R)\cong R\).

\end{proof}
\begin{proposition}
$$M_{n}(\mathbb{F} )=\mathrm{End}_{\mathbb{F} }(\mathbb{F} ^{n})$$
וכל \(\mathbb{F}\) מודולים נוצרים סופיים איזומורפיים ל-\(\mathbb{F}^{n}\) כלשהו.

\end{proposition}
\begin{remark}
זה עדין נכון אם מחלפים את \(\mathbb{F}\) עם \(M_{n}(\mathbb{F})\):
$$\mathbb{F} = \mathrm{End}_{M_{n}}(\mathbb{F}^{n} )$$

\end{remark}
\begin{definition}[אנטי הומומרפיזם של חוגים]
הומומורפיזם שבמקום שמשמר כפל הופך את הכפל, כלומר:
$$\Phi(rr')=\Phi(r')\Phi(r)$$

\end{definition}
\begin{reminder}
אם \(f \in \mathrm{End}_{R}(R)\) אז \(f\) נקבע על ידי \(f(1)\). כי:
$$\forall x \in R\quad f(x)=f(x\cdot 1)=xf(1)$$
בנוסף, לכל \(r \in R\) אפשר להגדיר \(f_{r}(x)=xr\). נשים לב כי \(f_{r}\in \mathrm{End}_{R}(R)\). זה כיוון שמתקיים:
$$f_{r}(r'x)=(r'x)r=r'(xr)=r'f_{r}(x)$$
סה"כ:
$$\Phi:R\to \mathrm{End}_{R}(R)\qquad \Phi(r)=f_{r}$$
וכן \(\Phi\) חח"ע ועל.

\end{reminder}
\begin{proposition}
ההעתקה \(\Phi:R\to \mathrm{End}_{R}(R)\) המוגדרת לעיל היא אנטי הומומואפיזם של חוגים.

\end{proposition}
\begin{proof}
$$\Phi(r+r')(x)=f_{r+r'}(x)=xr+xr'=(\Phi(r)+\Phi(r'))(x)$$
עבור הכפל:
$$\Phi(rr')(x)=f_{rr'}(x)=xrr'=(\Phi(r')\circ  \Phi(r))(x)$$
כלומר קיבלנו:
$$\Phi(rr')=\Phi(r')\Phi(r)$$

\end{proof}
\begin{definition}[חוג הפוך]
לחוג \(R\) החוג ההפוך \(R^{\text{op}}\) הוא \(R\) כקבוצה וכחבורה חבורה אבל עם הכפל:
$$x\cdot_{R^{op}}y := y\cdot_{R} x$$

\end{definition}
\begin{corollary}
קיבלנו:
$$\Phi:R^{\text{op}}\xrightarrow{\cong }\mathrm{End}_{R}(R)$$

\end{corollary}
\begin{definition}[מודול נאמן]
מודול \(R^{M}\) נקרא האמן אם הומומורפיזם המבנה \(\rho:R\to \mathrm{End_{Ab}}(B)\) הוא חח"ע. באופן שקול אם לכל \(0\neq r \in R\) קיים \(m \in M\) כך שמתקיים \(r m \neq 0\).

\end{definition}
\begin{lemma}
לחוג חילוק אין אידיאלים שמאליים אם"ם חוג חילוק.

\end{lemma}
\begin{proof}
לכל \(0\neq r \in R\) יש הופכי משמאל. כי \(\langle r \rangle=Rr=R\) וזה אומר שיש הופכי משמאל. נסמן ב-\(\bar{r}\) להופכי משמאל. נדרש רק להראות כי יש הופכי מימין כדי להראות שחוג חילוק. נראה ש-\(r\bar{r}=1\). נסתכל עליו בריבוע:
$$(r\bar{r})^{2}=r\cancel{ \bar{r}r }\bar{r}=r\bar{r}$$
בנוסף \(r\bar{r}\neq 0\) כי \(\bar{r}r\bar{r}r=1\) ולכן ייתכן כי \(r\bar{r}\) מתאפס.
לכן:
$$\overline{r\bar{r}} (r\bar{r})^{2}=\overline{r\bar{r}} r\bar{r}\implies r\bar{r}=1$$

\end{proof}
\begin{lemma}
אפשר להרחיב קבוצה בת"ל שאינה פורשת לבסיס.

\end{lemma}
\begin{proof}
  \begin{itemize}
    \item אם \(m\neq \langle m_{1},\dots,m_{n} \rangle\) אזי \(m_{1},\dots,m_{n},m\) בת"ל כי אם:
$$a_{1}m_{1}+\dots+a_{n}m_{n}+am=0$$
    \item כעת אם \(a=0\) נקבל כי לא אפשרי כי \(a_{1},\dots,a_{n}\) בת"ל.
    \item אם \(a\neq 0\) נקבל:
$$m=-\frac{a_{1}}{a}m_{1}+\dots-\frac{a_{n}}{a}m_{n}$$
סתירה.
  \end{itemize}
\end{proof}
\begin{proposition}
התנאים הבאים שקולים:

  \begin{enumerate}
    \item החוג \(R\) הוא חוג חילוק. 


    \item אין ל-\(R\) אידיאלים שמאליים. 


    \item אין ל-\(R\) אידיאלים ימניים. 


    \item החוג \(R\) פשוט כ-\(R\) מודול. 


    \item כל \(R\) מודול הוא חופשי(כלומר לכל \(R\) מודול יש בסיס) 


  \end{enumerate}
\end{proposition}
\begin{proof}
את 1-4 הראנו. נותר להראות רק ש-5 שקול לקודמים.
נראה חוג חילוק גורר כי בסיס כמו מעל שדה(קבוצה בת"ל שאינה פורשת אפשר להרחיב לקבוצה בת"ל.
אחרי שחוזרים על זה מספיק פעמיים(אינסוף כלשהו - ייתכן וידרש אינדוקציה טרנספיניטית), מגיעים לבסיס).
אם כל \(R\) מודול הוא חופשי, ו-\(0 \neq r\in R\). 
נניח בשלילה שיש \(0\lneq I\lneq R\)(אידיאל שמאלי) ניקח \(M= R / I\). אזי \(1+I\) יוצר אם \(M\) ולכן בסיס. נשלים אולי בהפסקה.
נראה כעת כי אם \(R\) מודול חופשי אז \(R\) חוג חילוק. ניקח \(M\neq 0\) מודול פשוט. (למשל \(M = R / I\) כש-\(I\leq R\) אידיאל שמאלי מקסימלי - קיים מהלמה של צורן).
ניקח בסיס \(B\) ל-\(M\). ניקח \(b\in B\) ונסתכל על:
$$f:R\to M; r\mapsto rb$$
הומומורפיזם של מודולים. \(f\) על כי \(M\) פשוט ו-\(0\neq b \in \mathrm{Im} f\) ולכן \(B=\{ b \}\). בנוסף כיוון ש-\(B\) בסיס אז חופשי ולכן \(f\) חח"ע. לכן \(R\cong M\) כמודול. לכן אין ל-\(R\) תת מודולים - כלומר אין אידיאלים שמאליים ולכן חוג חילוק.

\end{proof}
\section{סופיות}

\begin{definition}[מודול נוצר סופית]
מודל שנוצר על ידי איזשהי קבוצה סופית כלשהי:
$$M=\langle m_{1},\dots,m_{n} \rangle $$
עבור \(m_{1},\dots ,m_{n}\) כלשהם.

\end{definition}
\begin{definition}[מודול נתרי - Noetherian]
מודול \(M\) נקרא נתרי אם כל תת מודול של \(M\) נוצר סופית.

\end{definition}
\begin{example}[מודול לא נתרי]
המודול \(R=M=\mathbb{F}[x_{1},x_{2},\dots]\) הוא לא נתרי. זאת כיוון שלמרות שהחוג כולו מיוצר על ידי 1 האידיאל:
$$I=(x_{1},x_{2},\dots)$$
הוא לא נוצר סופית, כיוון שאם היה אז מספר סופי של פולינומים היה מייצר את כול המונומים, אשר לא אפשרי כי תמיד נתין להוסיף עוד משתנה.

\end{example}
\begin{proposition}
הפסוקים הבאים שקולים:

  \begin{enumerate}
    \item המודול \(M\) נתרי. 


    \item כל סדרה עולה של תתי מודולים של \(M\) מתייצבת, כלומר לא ייתכן: 
$$M_{1} \lneq M_{2}\lneq \dots \lneq \dots \lneq M$$


    \item לכל אוסף של תתי מודולים יש איבר מקסימלי. 


  \end{enumerate}
\end{proposition}
\begin{proof}
מ-2 ל-3 ניתן לעבור מהלמה של צורן.
אם יש סדרה עולה ממש:
$$M_{1} \subsetneq M_{2} \subsetneq \dots$$
ניקח \(N=\bigcup_{i=1}^{\infty}M_{i}\) ונראה שלא נוצר סופית - אם הוא היה ב-\(M_{j}\) כלשהו כל היוצרים כבר מופיעים ולכן הסדרה כבר לא עולה. ואם \(N\leq M\) לא נוצר סופית ניקח \(n_{1},n_{2},n_{3},\dots\) כש-\(n_{j}\not\in \langle n_{1},\dots,n_{j-1} \rangle\) ואז:
$$N_{j}=\langle n_{1},\dots n_{j} \rangle $$
עולה ממש.

\end{proof}
\begin{example}
המודול \(\mathbb{Q}\) הוא לא \(\mathbb{Z}\)-מודול נתרי:
$$\mathbb{Z}\subsetneq \frac{1}{2}\mathbb{Z} \subsetneq \frac{1}{4}\mathbb{Z} \subsetneq \dots$$
כאשר \(\mathbb{Z}\) כן \(\mathbb{Z}\) מודול נתרי:
$$m_{1}\mathbb{Z} \subsetneq m_{2}\mathbb{Z} \subseteq \dots $$
כאשר נדרש \(m_{2}\) כך שמחלק את \(m_{1}\). בסופו של דבר נתייצב כי כל \(m_{i}\) צריך לחלק את כל ה-m-ים הקודמים.

\end{example}
\begin{definition}[מודול ארטיני]
מודול נקרא ארטיני אם אין סדרה אינסופית ירדת ממש של תתי מודולים. כלומר כל שרשרת יורדת מתייצבת.

\end{definition}
\begin{remark}
ארטיני זה תנאי חזק. יותר חזק מנתרי, מאוד מגביל (אומנם זה לא פרמלית חזק יותר כי ייתכן נתרי ולא ארטני).

\end{remark}
\begin{example}
למשל \(\mathbb{Z}^{\mathbb{Z}}\) לא ארטיני:
$$\mathbb{Z}\supset 2\mathbb{Z} \supset 4\mathbb{Z}\supset \dots$$

\end{example}
\begin{example}
נסתכל על \(\mathbb{Z}\left[ \frac{1}{2} \right] / \mathbb{Z}\) כ-\(\mathbb{Z}\) מודול. זה ארטיני ולא נתרי:
$$\left\langle  \frac{1}{2}  \right\rangle \subset \left\langle  \frac{1}{4}  \right\rangle \subset \left\langle  \frac{1}{8}  \right\rangle \subset \dots$$
כל תת מודול הוא \(\left\langle  \frac{1}{2^{n}}  \right\rangle\) ל-\(n\) כלשהו. ואז:
$$\left\langle  \frac{1}{2^{n_{1}}}  \right\rangle \supset \left\langle  \frac{1}{2^{n_{2}}}  \right\rangle\supset $$
כאשר \(n_{2}< n_{1}\) וזה חייב להעצר. ניתן לבנות דוגמא דומה על נתרי ולא ארטני.

\end{example}
\begin{definition}[מאורך סופי]
אם \(M\) גם נתרי וגם ארטיני הוא נקרא מאורך סופי.

\end{definition}
\begin{proposition}
אם \(M\) מאורך סופי אזי יש סדרה:
$$0=M_{n}\leq  \dots \lneq M_{2} \lneq  M_{1}=M$$
כך ש-\(M_{i} / M_{i+1}\) פשוט לכל \(i\).  זה למעשה סדרת הרכב.

\end{proposition}
\begin{remark}
השם מאורך סופי בא מכך שניתן להגדיר אורך:
$$\mathrm{{length}}\left(M\right)=\operatorname*{sup}\left\{\ell\,|\,0=M_{0}\lneq M_{1}\lneq\ldots\lneq M_{\ell}=M\right\}$$
ולקבל מהטענה כי סופי אם"ם ארטני וגם נתרי.

\end{remark}
\begin{theorem}[ג'ורדן הולדר]
אם יש ל-\(M\) סדרת הרכב אזי לכל 2 סדרות הרכב יש את אותן מנות(עד כדי סדר ואיזומורפיזם). 

\end{theorem}
\begin{corollary}
אם \(M\) מאורך סופי אז יש סדרת הרכב.

\end{corollary}
\begin{example}
$$(0)\overset{C_{2}}{<}(6)\overset{C_{3}}{<}  (2)\overset{C_{2}}{<}   C_{12}$$
או לחלופין ניתן לבנות:
$$(0)\overset{C_{2}}{<} (6) \overset{C_{2}}{<} (3)\overset{C_{3}}{<} C_{12}$$

\end{example}
\begin{proposition}
כל מודול נוצר סופית הוא מנה של מודול חופשי עם דרגה סופית.

\end{proposition}
\begin{proof}
אם מודול \(M=\langle m_{1},\dots,m_{n} \rangle\) נוצר סופית ניתן להגדיר הומומואפיזם:
$$f:R^{n}\rightarrow{M},\quad f(r_{1},\ldots,r_{n})=r_{1}m_{1}+\cdot\cdot\cdot+r_{n}m_{n}$$
כאשר ממשפט האיזומורפיזם הראשון כיוון ש-\(f\) היא על נקבל:
$$M\cong R^{n}/\ker f$$

\end{proof}
\begin{lemma}
אם מודול \(M\) הוא נתרי ו-\(f:M\to M\) הומומורפיזם אז קיים \(n \in \mathbb{N}\) כך ש:
$$\ker (f^{n})\cap \mathrm{\mathrm{Im}}(f^{n})=\{ 0 \}$$

\end{lemma}
\begin{corollary}
אם \(M\) נתרי ו-\(f:M\to M\) הומומורפיזם על אז \(f\) חח"ע, ובפרט איזומורפיזם.

\end{corollary}
\begin{proof}
נובע מכך שאם \(M=\mathrm{Im}(f^{n})\) נקבל \(\ker(f^{n})\cap M=\{ 0 \}\) ולכן \(\ker(f^{n})=\{ 0 \}\).

\end{proof}
\begin{lemma}
אם מודול \(M\) הוא ארטני ו-\(f:M\to M\) הומומורפיזם אז קיים \(n \in \mathbb{N}\) כך ש:
$$\ker (f^{n})+\mathrm{Im}(f^{n})=M$$

\end{lemma}
\begin{corollary}
אם \(M\) נתרי ו-\(f:M\to M\) הומומורפיזם חח"ע אז \(f\) על, ובפרט איזומורפיזם.

\end{corollary}
\begin{proposition}[פירוק פיטינג]
אם \(M\) מאורך סופי ו-\(f:M\to M\) הומומורפיזם, אז קיים \(N,I\) כך ש-\(f|_{N}\) יהיה נילפוטנטי(קיים \(n\) כך ש-\(f|_{N}^{n}\) מתאפס) ו-\(f|_{I}\) אוטומורפיזם כך שמתקיים:
$$N\oplus I = M$$

\end{proposition}
\begin{definition}[חוג נתרי/ארטני]
חוג נקרא נתרי / ארטיני אם הוא כזה כמודול מעל עצמו.

\end{definition}
\begin{theorem}[הופקינס לויצקי]
חוג ארטיני הוא נתרי.

\end{theorem}
\begin{proposition}
יהי \(N\leq M\) תת מודול. אזי \(M\) נתרי אם"ם \(N\) ו-\(M / N\) נתרי

\end{proposition}
\begin{proof}
אם \(M\) נטרי ו-\(N\leq M\) אז \(N\) נטרי(כי \(L\leq N \leq M\)). כמו כן, \(M / N\) נתרי:
אם \(L\leq M / N\) אזי ממשפט ההתאמה יש \(\overline{L}\leq M\)(ספציבית \(\overline{L}=\bigcup_{\ell \in L}\ell\)) כך ש-\(L = L / N\). אזי \(\overline{L}\) נוצר סופית על ידי \(\ell_{1},\dots,\ell_{n}\) ואז \(L=\langle \ell_{1}+N, \dots,\ell_{n}+N \rangle\).
אם \(N\) ו-\(M / N\) נטריים:
עבור \(L\leq M\) נקבל מאיזו 2:
$$\frac{L+N}{N}\overset{II}{\cong } \frac{L}{N\cap  L}$$
(כאשר \(\frac{L+N}{N}\) תת מודול של \(\frac{M}{ N}\) לכן נוצר סופית) לכן \(\frac{L}{N \cap L}\) נוצר סופית. כמו כן \(N\cap L\) נוצר סופית(כי \(N\) נטרית) ולכן \(L\) נוצר סופית(על ידי יוצרים של \(N\cap L\) + יוצרים של \(\frac{L}{N\cap L}\)).

\end{proof}
\begin{remark}
נוצרות סופית לא מתנהגת כזה יפה, זה הופך את נתרי לסוג של תיקון של נוצר סופית.

\end{remark}
\begin{corollary}
אם \(R\) נתרי(חוג נתרי = חוג כמודול מעל עצמו = כל אידיאל שמאלי נוצר סופית) אז ל-\(R\) מודולים נ"ס = נתרי.

\end{corollary}
\begin{proof}
תמיד נתרי גורר נוצר סופית. אם \(R\) נתרי גם \(R^{n}\) נתרי לכל \(n\).
$$\frac{R\oplus R}{R\oplus 0}\cong R\cong R\oplus 0 \leq  R\oplus R$$
ואז גם \(R^{n} / N\) נתרי לכל \(N\leq R^{n}\) וכל \(R\) מודול נוצר סופית איזומורפי ל-\(R^{n} / N\) לאיזשהו \(n \in \mathbb{N}\) ו-\(N\leq R^{n}\) כלשהם.

\end{proof}
\begin{proposition}[משפט הבסיס של הילברט]
אם \(R\) חוג נתרי אזי \(R[x]\) נתרי. בפרט \(R[x_{1},\dots,x_{n}]\) נתרי.

\end{proposition}
\begin{remark}
היום מה שנקרה בסיסי Grobner זו הדרך הנכונה לעשות את זה.

\end{remark}
\section{אנדומורפיזמים}

\begin{proposition}
עבור מודול \(R^{M}\) מתקיים האנדומורפיזמים של המודול יהיה תת חוג של האנדומורפיזמים של החוג. כלומר:
$$\operatorname{End}_{R}(M)=\left\{f:M\to M\mid f(r m)=r f(m)\,{\mathrm{for~all}}\,r\in R,m\in M\right\}\leq \mathrm{End}(M)$$

\end{proposition}
\begin{example}[מרחבים ווקטורים]
$${\mathrm{End}}_{\mathbb{F}}{\bigl(}\mathbb{F}^{n}{\bigr)}=M_{n}{\bigl(}\mathbb{F}{\bigr)}$$

\end{example}
\begin{example}[מודולים מעל חוג מטריצות]
$$\operatorname{End}_{M_{n}(\mathbb{F})}(\mathbb{F}^{n})=\mathbb{F}$$

\end{example}
\begin{example}
$$\operatorname{End}_{R}(R)=R^{\mathrm{op}}$$
כש-\(R^{\text{op}}\) היא \(R\) עם כפל הפוך. לכל \(r \in R\) הגדרנו \(m_{r}(x)=xr\), וראינו \(m_{r} \in \mathrm{End}_{R}(R)\) כאשר \(r \mapsto m_{r}:R\to \mathrm{End}_{R}(R)\) חח"ע ועל, מכבדת חיבור:
$$m_{r+r'}=m_{r}+m_{r'}$$
והופכת כפל:
$$m_{r r'}= m_{r'}\circ m_{r}$$

\end{example}
\begin{proposition}[הלמה של שור]
אם \(R^{M}\) מודול פשוט אזי \(\mathrm{End}_{R}(M)\) חוג חילוק.

\end{proposition}
\begin{proof}
אם \(0\neq f \in \mathrm{End}_{R}(M)\) אזי \(\mathrm{\mathrm{Im}}f\neq \{ 0 \}\) ולכן \(\mathrm{Im} f = M\) ולכן \(f\) על.
אם \(\ker f \neq M\) אז \(\ker f = \{ 0 \}\) ואז \(f\) חח"ע.
זה גורר \(f\) הפיך כפונקציה מ-\(M\) ל-\(M\). ואז \(f ^{-1} \in \mathrm{End}_{R}(M)\).
$$f^{-1} (m+n)=f^{-1} (f(f^{-1} (m))+f(f^{-1} (n)))=f^{-1} (f(f^{-1} (m)+f^{-1} (n)))$$

\end{proof}
\section{חוג הפוך}

\begin{definition}[אנטי הומומרפיזם של חוגים]
הומומורפיזם שבמקום שמשמר כפל הופך את הכפל, כלומר:
$$\Phi(rr')=\Phi(r')\Phi(r)$$

\end{definition}
\begin{reminder}
אם \(f \in \mathrm{End}_{R}(R)\) אז \(f\) נקבע על ידי \(f(1)\). כי:
$$\forall x \in R\quad f(x)=f(x\cdot 1)=xf(1)$$
בנוסף, לכל \(r \in R\) אפשר להגדיר \(f_{r}(x)=xr\). נשים לב כי \(f_{r}\in \mathrm{End}_{R}(R)\). זה כיוון שמתקיים:
$$f_{r}(r'x)=(r'x)r=r'(xr)=r'f_{r}(x)$$
סה"כ:
$$\Phi:R\to \mathrm{End}_{R}(R)\qquad \Phi(r)=f_{r}$$
וכן \(\Phi\) חח"ע ועל.

\end{reminder}
\begin{proposition}
ההעתקה \(\Phi:R\to \mathrm{End}_{R}(R)\) המוגדרת לעיל היא אנטי הומומואפיזם של חוגים.

\end{proposition}
\begin{proof}
$$\Phi(r+r')(x)=f_{r+r'}(x)=xr+xr'=(\Phi(r)+\Phi(r'))(x)$$
עבור הכפל:
$$\Phi(rr')(x)=f_{rr'}(x)=xrr'=(\Phi(r')\circ  \Phi(r))(x)$$
כלומר קיבלנו:
$$\Phi(rr')=\Phi(r')\Phi(r)$$

\end{proof}
\begin{definition}[חוג הפוך]
לחוג \(R\) החוג ההפוך \(R^{\text{op}}\) הוא \(R\) כקבוצה וכחבורה חבורה אבל עם הכפל:
$$x\cdot_{R^{\mathrm{op}}}y := y\cdot_{R} x$$

\end{definition}
\begin{corollary}
קיבלנו:
$$\Phi:R^{\text{op}}\xrightarrow{\cong }\mathrm{End}_{R}(R)$$\textbf{דוגמא}
נקבל כי \(f(A)=A^{T}\) הוא איזומורפיזם \(f:M_{R}(\mathbb{F})\to M_{n}(\mathbb{F})^{\mathrm{op}}\).
$$f(I)=I\qquad f(A+B)=fA+fB\qquad f(AB)=f(A)\cdot f(B)=f(B)f(A)$$

\end{corollary}
\begin{remark}
אם \(R\) קומוטטיבי אז \(R\cong R^{\text{op}}\). כמו כן בחבורות תמיד מתקיים \(R\cong R^{\text{op}}\) כאשר באותו אופן \(M_{n}(R)^{\text{op}}\cong M_{n}(R^{\text{op}})\) בעזרת האיזומורפיזם \(A\mapsto A^{T}\). אבל לא לא תמיד מתקיים עבור מודולים.

\end{remark}
\begin{proposition}[אנדומורפיזמים של מודולים חופשיים]
נתון \(R\) מודול חופשי \(M=R^{n}\) מדרגה \(n\). אזי:
$$\operatorname{End}_{R}(R^{n})\cong M_{n}(R^{\mathrm{op}})$$

\end{proposition}
\begin{proof}
כפל ב-\(R\) וב-\(M_{n}(R)\) נסמן ב- \(\cdot\) כאשר כאפל ב-\(R^{\text{op}}\) וב-\(M_{n}(R^{\text{op}})\) יסומן על ידי \(\times\).
המודול \(M_{n}(R)^{\text{op}}\) פועל על \(R^{n}\) כאנדומורפיזם \(\Psi:M_{n}(R^{\text{op}})\to \mathrm{End}_{R}(R^{n})\) על ידי: 
$$A\times v = \begin{pmatrix}v_{1}a_{11}+\dots+v_{1}a_{1n} \\\vdots \\v_{n}a_{n_{1}}+\dots+v_{n}a_{nn}
\end{pmatrix}=(v^{T}A^{T})^{T}$$
כאשר אם החוג היה קומוטטיבי זה היה פשוט \(Av\). כעת נראה כי \(R\)-לינארי:

  \begin{enumerate}
    \item הומוגניות: 
$$A\times(r\,v)=\left((r\,v)^{T}A^{T}\right)^{T}=\left(r\,v^{T}A^{T}\right)^{T}=r\left(v^{T}A^{T}\right)^{T}=r\,(A\times v)$$


    \item אדיטיביות: 
$$A\times(v+w)=\left((v+w)^{T}A^{T}\right)^{T}=\left((v^{T}+w^{T})A^{T}\right)^{T}=\left(v^{T}A^{T}+w^{T}A^{T}\right)^{T}=(A\times v)+(A\times w)$$
ולכן \(\Psi(A)\in \mathrm{End}_{R}(R^{n})\). נראה כי הומומואפיזם:


    \item משמר חיבור: 
$$\Psi(A+B)(v)=(A+B)\times v=A\times v+B\times v=(\Psi(A)+\Psi(B))(v)$$


    \item משמר כפל(עם היפוך): 
$$(A\star B)_{i j}=\sum_{k=1}^{n}a_{i k}\star b_{k j}=\sum_{k=1}^{n}b_{k j}\,a_{i k}$$
כאשר:
$$\Psi(A\star B)(v)=(A\star B)\times v=\left(v^{T}(A\star B)^{T}\right)^{T}=\left(v^{T}(B^{T}A^{T})\right)^{T}=A\times(B\times v)=(\Psi(A)\circ\Psi(B))(v)=(\alpha\star B)$$
ולכן \(\Psi(A\star B)=\Psi(A)\circ\Psi(B).\) כלומר \(\Psi\) הומומורפיזם של חוגים. נראה כי על. יהי \(f\in \mathrm{End}_{R}(R^{n})\). נמצא \(A \in M_{n}(R^{\text{op}})\) עם \(\Psi(A)=f\). יהי \(\{ e_{1},\dots,e_{n}  \}\) הבסיס הסטנדרטי של \(R^{n}\). עבור כל \(j\) נכתוב:
$$f(e_{j})\;=\;\sum_{i=1}^{n}a_{i j}\,e_{i},\quad a_{i j}\in R.$$
נגדיר את המטריצה \(A=(a_{ij})\) וכעת עבור \(v=\sum_{j}v_{j}e_{j}\) נקבל:
$$f(v)=\sum_{j}v_{j}f(e_{j})=\sum_{j}v_{j}\sum_{i}a_{i j}e_{i}=\sum_{i}\Bigl(\sum_{j}v_{j}a_{i j}\Bigr)e_{i}=A\times v$$
ולכן \(f=\Psi(A)\). הייחודיות של \(a_{ij}\) מראה ש-A ייחודי. נראה כי חח"ע. אם \(\Psi(A)=0\) אזי \(A\times e_{j}=0\) לכל \(j\) אבל:
$$A\times e_{j}=\begin{pmatrix}a_{1j}\\ a_{2j}\\ \vdots\\a_{nj}
\end{pmatrix}$$
לכן כל עמודה מתאפסת ו-\(A=0\). כלומר קיבלנו כי הומומואפיזם חח"ע ועל ולכן איזומומורפיזם של חוגים.


  \end{enumerate}
\end{proof}
\chapter{משפט המבנה למודולים מעל PID}

\section{צורת סמית}

\begin{definition}[צורת סמית]
מטריצה \(A \in M_{m \times n}(R)\) עם איברים \(a_{ij}\) נקראת בצורת סמית אם \(a_{ij}=0\) ל-\(i\neq j\) ו-\(a_{i,i}\mid a_{i+1,i+1}\).

\end{definition}
\begin{proposition}[קיום ויחידות צורת סמית]
לכל מטריצה \(A \in M_{m \times n}(R)\) מעל תחום ראשי \(R\), קיימות מטריצות הפיכות \(P \in M_{m}(R)\) ו-\(Q \in M_{n}(R)\) כך ש:
$$PAQ = \begin{pmatrix}d_{1} & & & \\& d_{2} & & \\& & \ddots & \\& & & d_{r} \\& & & \\
\end{pmatrix}$$
כאשר \(d_{1} | d_{2} | \dots | d_{r}\) הם מחלקים ראשיים של \(A\).

\end{proposition}
\begin{remark}
צורה זו היא גם יחידה עד כדי חברות, אך איננו נדרשים ליחידות בשלב זה.

\end{remark}
\begin{proof}
  \begin{enumerate}
    \item נתחיל עם מטריצה \(A \in M_{m \times n}(R)\) מעל תחום ראשי \(R\). המטרה היא באמצעות פעולות שורה ועמודה הפיכות להביא את המטריצה לצורה אלכסונית עם תנאי החלוקה \(d_i | d_{i+1}\).  


    \item נבחר את האיבר \(a_{11}\) בפינה השמאלית העליונה. באמצעות פעולות שורה ועמודה אלמנטריות, נשאף להגיע למצב שבו \(a_{11}\) מחלק את כל שאר האיברים במטריצה. אם זה לא מתקיים, נוכל להקטין את \(a_{11}\) על ידי פעולות דמויות אלגוריתם אוקלידס. 


    \item כאשר \(a_{11}\) מחלק את כל האיברים בשורה הראשונה ובעמודה הראשונה, נוכל לאפס את כל האיברים הללו. לאחר מכן נקבל מטריצה בצורה הבלוקית: 
$$\begin{pmatrix}d_1 & 0 \\0 & A'\end{pmatrix}
$$
כאשר \(A'\) היא תת-מטריצה בגודל \((m-1)\times(n-1)\).


    \item נפעל באותה שיטה על תת-המטריצה \(A'\). בכל שלב נביא את האיבר בפינה השמאלית העליונה של תת-המטריצה הנוכחית למצב שהוא מחלק את כל איברי התת-מטריצה, ונאפס את השורה והעמודה המתאימות. 


    \item לאחר סיום התהליך האינדוקטיבי, נקבל מטריצה אלכסונית. אם מתקיים \(d_i \nmid d_j\) עבור \(i < j\), נוכל להשתמש בפעולות שורה ועמודה כדי לשלב בין האיברים הלולים ולקבל את תנאי החלוקה הרצוי \(d_i | d_{i+1}\). 


    \item התהליך מסתיים לאחר מספר סופי של צעדים כי בכל איטרציה אנחנו מקטינים את "הגודל" של האיבר המוביל (במובן של נורמה או מספר גורמים ראשוניים בחוג ראשי). התוצאה הסופית היא מטריצה בצורת סמית עם האיברים האלכסוניים \(d_1 | d_2 | \cdots | d_r\) ויתר האיבים אפס. 


  \end{enumerate}
\end{proof}
\section{משפט המבנה למודולים נוצרים סופית מעל PID}

\begin{reminder}[תחום ראשי - PID]
תחום שבו כל אידיאל הוא ראשי.

\end{reminder}
\begin{theorem}[משפט המבנה למודולים נוצרים סופית בתחום PID]
$$M\cong  R / (d_{1}) \oplus \dots \oplus R / (d_{n})$$
כש-\(d_{1}|\dots|d_{n}\) לא הפיכים, יחידים עד כדי חברות.

\end{theorem}
נוכיח קיום ויחידות

\begin{proof}
קיום
נוכיח באמצעות צורת סמית. 

  \begin{enumerate}
    \item נניח ש-\(M \cong R^m / \mathrm{Im}(A)\), כלומר \(M\) הוא מודול נוצר סופית שהוא גורם של \(R^m\) לפי תמונה של הומומורפיזם לינארי: 
$$A : R^n \to R^m
$$
כאשר \(A \in M_{m \times n}(R)\).


    \item על פי קיומה של צורת סמית, קיימות מטריצות הפיכות \(P \in GL_m(R)\), \(Q \in GL_n(R)\) כך ש: 
$$PAQ = D =\begin{pmatrix}d_1 & 0 & \cdots & 0 \\0 & d_2 & \cdots & 0 \\\vdots & \vdots & \ddots & \vdots \\0 & 0 & \cdots & d_r \\\vdots & \vdots & & \vdots \\0 & 0 & \cdots & 0 \\\end{pmatrix}
\in M_{m \times n}(R)$$
כאשר \(d_1 \mid d_2 \mid \cdots \mid d_r\), וכל היתר אפסים.


    \item מכיוון שמטריצות \(P\), \(Q\) הפיכות, מתקיים: 
$$\mathrm{Im}(A) \cong \mathrm{Im}(PAQ) = \mathrm{Im}(D)$$
ולכן:
$$M = \frac{R^m}{\mathrm{Im}(A)} \cong \frac{R^m}{\mathrm{Im}(D)}$$


    \item כעת התמונה של \(D\) היא: 
$$\mathrm{Im}(D) = \left\{ (r_1, \dots, r_m) \in R^m \mid d_i \mid r_i \text{ for } 1 \leq i \leq r, \; r_j = 0 \text{ for } j > r \right\}
$$
ולכן:
$$\frac{R^m}{\mathrm{Im}(D)} \cong \frac{R}{(d_1)} \oplus \cdots \oplus \frac{R}{(d_r)} \oplus R^{m - r}
$$
אם \(A\) היא מטריצה שמייצגת מודול ללא חלק חופשי (כלומר מודול נוצר סופית שהוא מוגבל), אז \(r = m\) ואין רכיב חופשי.


  \end{enumerate}
\end{proof}
\begin{proof}
יחידות

  \begin{enumerate}
    \item נגדיר את \textbf{דרגת המודול}\(M\) כמספר היוצרים המינימלי שלו. עבור המודול: 
$$M = \frac{R}{(d_1)} \oplus \frac{R}{(d_2)} \oplus \cdots \oplus \frac{R}{(d_n)}
$$
מתקיים \(\text{rk}(M) = n\).


    \item נבחר אידיאל מקסימלי \(m \lneq R\) המכיל את \(d_1\) (קיים לפי הלמה של צורן). מכיוון ש-\(R\) תחום ראשי, \(R/m\) שדה. נגדיר העתקה: 
$$\varphi: M \to \left( \frac{R}{m} \right)^n
$$
על ידי הטלה טבעית בכל רכיב. ההעתקה מוגדרת היטב כי \((d_i) \subseteq m\) לכל \(i\) (מתנאי החלוקה \(d_1 \mid d_2 \mid \cdots \mid d_n\)).


    \item ההעתקה \(\varphi\) היא על, ולכן פורשת עבור \(\left( \frac{R}{m} \right)^n\) כמרחב וקטורי מעל \(R/m\). מכיוון ש-\(\dim_{R/m} \left( \frac{R}{m} \right)^n = n\), נובע ש-\(\text{rk}(M) = n\). 


    \item לכל \(r \in R\), נגדיר את תת-המודול \(rM = \{ r \cdot x \mid x \in M \}\). מתקיים: 
$$ rM \cong \frac{R}{(e_1)} \oplus \frac{R}{(e_2)} \oplus \cdots \oplus \frac{R}{(e_n)} $$
כאשר \(e_i = \frac{d_i}{\gcd(r, d_i)}\) ו-\(e_1 \mid e_2 \mid \cdots \mid e_n\).


    \item דרגת \(rM\) היא מספר הרכיבים הלא-טריוויאליים: 
$$ \text{rk}(rM) = \#\left\{ i \mid e_i \notin R^\times \right\} = \#\left\{ i \mid r \notin (d_i) \right\} $$


    \item לכל \(1 \leq i \leq n\) מתקיים: 
$$ (d_i) = \left\{ r \in R \mid \text{rk}(rM) \leq i-1 \right\} $$
בפרט, האידיאלים \((d_i)\) נקבעים יחידות מתוך \(M\).


    \item מכיוון שהאידיאלים \((d_i)\) נקבעים יחידות, המחלקים הראשיים \(d_i\) נקבעים עד כדי כפל באיבר הפיך. מכאן נובעת יחידות צורת סמית. 


  \end{enumerate}
\end{proof}
\begin{corollary}
מודול נוצר סופית מעל PID הוא סכום ישר של תתי מודולים ציקלים:
$$M\cong  Rm_{1} \oplus \dots Rm_{R}$$

\end{corollary}
\begin{proof}
נובע מכך ש-\(\frac{R}{(d_{i})}\) נוצר על ידי \(1+(d_{i})\).

\end{proof}
\section{מסקנות ממשפט המיון}

\begin{proposition}[משפט המבנה לחבורות אבליות נוצרות סופית]
יהי \(A\) חבורה אבלית נוצרת סופית. קיים \(r \ge 0\) ואיברים \(d_1,\dots,d_m \in \mathbb{Z}\setminus\{0\}\) כך ש־\(d_1 \mid d_2 \mid \cdots \mid d_m\) ומתקיים:
$$A \;\cong\; \mathbb{Z}/d_1 \;\times\; \mathbb{Z}/d_2 \;\times\; \cdots \times\; \mathbb{Z}/d_m \;\times\; \mathbb{Z}^r
$$

\end{proposition}
\begin{proof}
מסקנה מיידית ממשפט המיון למודולים נוצרים סופית מעל PID עבור \(R=\mathbb{Z}\).
מכיוון ש-\(A\) נוצרת סופית, יש אפימורפיזם \(\;h: \mathbb{Z}^m \twoheadrightarrow A\;\) עבור \(m\) סופי, ולכן \(A \cong \mathbb{Z}^m / \ker(h)\).
לפי משפט המבנה למודולים נוצרים סופית מעל PID, כמודול מעל \(\mathbb{Z}\):
$$\mathbb{Z}^m / \ker(h) \;\cong\; \bigoplus_{i=1}^s \mathbb{Z}/(d_i) \;\oplus\; \mathbb{Z}^t,
$$
כאשר \(d_1 \mid d_2 \mid \cdots \mid d_s\) ו-\(t\ge0\). כאן \(\mathbb{Z}/(d_i)\) מסומן גם \(\mathbb{Z}/d_i\). נקבל:
$$A \;\cong\; \mathbb{Z}/d_1 \times \cdots \times \mathbb{Z}/d_s \times \mathbb{Z}^t
$$

\end{proof}
\begin{proposition}[משפט הפירוק של ג'ורדן]
יהי \(V\) מרחב וקטורי סוף-מימד מעל שדה אלגברי סגור \(\mathbb{F}\), ויהי \(T:V \to V\) טרנספורמציה לינארית. אז קיים בסיס של \(V\) שבו מטריצת \(T\) היא ב-Jordan Normal Form, כלומר בלוק-דיאגונלית של Jordan blocks עם האלכסון \(\lambda\) ו-1 בתת-האלכסון, כאשר כל \(\lambda\) הוא שורש של פולינום האופייני.

\end{proposition}
\begin{proof}
  \begin{enumerate}
    \item נגדיר פעולה של \(x \in \mathbb{F}[x]\) על \(V\) על ידי \\
$$   x \cdot v = T(v)$$\\

כך \(V\) הופך ל-\(\mathbb{F}[x]\)-מודול נוצרים סופית וללא חלק חופשי. אם היה חלק חופשי, היה חלק חופשי מהווה \(\mathbb{F}[x]\)-תת-מודול חופשי, ואז כמרחב וקטורי היה אינסוף-מימד מעל \(\mathbb{F}\), שזה סותר את הסופי-מימדיות של \(V\).


    \item לפי משפט המבנה למודולים נוצרים סופית מעל PID (כאן \(R=\mathbb{F}[x]\)), מתקבל: \\
$$V \cong\bigoplus_{i=1}^m \mathbb{F}[x]/(p_i)$$
כאשר \(p_1 \mid p_2 \mid \cdots \mid p_m\) ו-\(p_i \neq 0\). אלו הגורמים האינווריאנטים של \(T\).


    \item מכיוון ש-\(\mathbb{F}\) סגורה אלגברית, כל פולינום \(p_i(x)\) מתפרק לגורמים ליניאריים: \\
$$p_i(x) = \prod_j (x - \lambda_j)^{e_{i,j}}$$
לפי פירוק ראשוני של מודולים ציקליים:\\
$$\mathbb{F}[x]/(p_i) \;\cong\; \bigoplus_j \mathbb{F}[x]/\bigl((x - \lambda_j)^{e_{i,j}}\bigr)   $$
לכן:
$$   V \;\cong\; \bigoplus_{i=1}^m \bigoplus_j \mathbb{F}[x]/\bigl((x - \lambda_j)^{e_{i,j}}\bigr)$$


    \item עבור המודול \(\mathbb{F}[x]/((x - \lambda)^e)\), הבסיס הטבעי \(\{1, (x-\lambda), \dots, (x-\lambda)^{e-1}\}\) מקיים שפעולת \(T\) (שזה למעשה \(x\cdot\)) מיוצגת על ידי Jordan block בגודל \(e\) עם האלכסון \(\lambda\) ו-1 בתת-האלכסון הראשי. \\

   כלומר מטריצה:\\
$$J_e(\lambda) \;=\;   \begin{pmatrix}   \lambda & 1      & 0      & \cdots & 0 \\   0       & \lambda & 1     & \cdots & 0 \\   \vdots  &        & \ddots & \ddots & \vdots \\   0       & \cdots & 0      & \lambda & 1 \\   0       & \cdots & \cdots & 0       & \lambda   \end{pmatrix}.
   $$


  \end{enumerate}
5.מחיבור כל הבלוקים \(\mathbb{F}[x]/((x - \lambda_j)^{e_{i,j}})\) נקבל בסופו של דבר מטריצה בלוק-דיאגונלית שבה כל בלוק הוא Jordan block מתאים. זוהי בדיוק ה-Jordan Normal Form של \(T\). 

\end{proof}
\begin{definition}[מטריצה מלווה של פולינום מתוקן]
עבור פולינום מתיקון מהצורה:
$$p(x) = x^n + a_{n-1} x^{n-1} + \dots + a_1 x + a_0 \in \mathbb{F}[x]$$
המטריצה מלווה (Companion matrix) \(C_p\) היא מטריצה בגודל \(n\times n\) כך שבבסיס \(\{1,x,\dots,x^{n-1}\}\) של המודול \(\mathbb{F}[x]/(p)\) הפעולה של כפל ב-\(x\) מיוצגת על ידי:\\
$$  C_p \;=\;  \begin{pmatrix}  0 & 0 & \cdots & 0 & -a_0 \\  1 & 0 & \cdots & 0 & -a_1 \\  0 & 1 & \cdots & 0 & -a_2 \\  \vdots & & \ddots & \vdots & \vdots \\  0 & 0 & \cdots & 1 & -a_{n-1}  \end{pmatrix}
  $$
  כך שהפולנום האופייני שלה הוא \(\det(xI - C_p) = p(x)\).

\end{definition}
\begin{theorem}[פורובניוס]
יהי \(V\) מרחב וקטורי סוף-מימדי מעל שדה \(\mathbb{F}\), ויהי \(A:V \to V\) טרנספורמציה לינארית (או מטריצה \(A \in M_n(\mathbb{F})\) ביחס לבסיס כלשהו).\\

אז קיימים פולינומים מתוקנים \(p_1, p_2, \dots, p_r \in \mathbb{F}[x]\) כך ש-\(p_1 \mid p_2 \mid \cdots \mid p_r\) כאשר \(A\) דומה למטריצת בלוקים אלכסונית מהצורה:
$$A \sim \begin{pmatrix}C_{p_1} & & \\& \ddots & \\& & C_{p_r}\end{pmatrix}.
$$
כך שהגורמים האינווריאנטים \(p_{i}\) הם יחידים (עד לכפולות ביחידה) וקובעים את הפירוק ביחודות.
צורה זו נקראת הצורה הקנונית נורמלית, רציונלית, פרובניוס(לבחור 2 מ-4).

\end{theorem}
\begin{proof}
נשתמש במשפט המיון למודולים נוצרים סופית מעל PID כאשר \(R = \mathbb{F}[x]\).

  \begin{enumerate}
    \item נגדיר על \(V\) פעולה של \(\mathbb{F}[x]\) על ידי: \\
$$   \forall f(x)\in \mathbb{F}[x],\quad f(x)\cdot v = f(A)(v)  $$
ובפרט \(x \cdot v = A(v)\).  מכיוון ש-\(V\) סוף-מימד מעל \(\mathbb{F}\), אין בו חלק חופשי כמודול על \(\mathbb{F}[x]\) (כיוון שחלק חופשי היה יוצר מרחב וקטורי אינסופי-מימד מעל \(\mathbb{F}\)).  \(V\) נוצר סופית כמודול על \(\mathbb{F}[x]\), כי אם \(\dim_{\mathbb{F}}V = n\), ניתן לבחור בסיס \(\{v_1,\dots,v_n\}\) מעל \(\mathbb{F}\), ואז כמודול נוצרים על ידי \(\{v_1,\dots,v_n\}\).


    \item כעת לפי משפט המבנה למודולים נוצרים סופית מעל PID (\(\mathbb{F}[x]\) הוא PID), מתקיים: 
$$V \cong \bigoplus_{i=1}^r \mathbb{F}[x] / (p_i)$$
   כאשר כל \(p_i \neq 0\) ו-\(p_1 \mid p_2 \mid \cdots \mid p_r\).  


    \item לכל גורם \(\mathbb{F}[x]/(p_i)\), מכיוון ש-\(p_i\) מתוקן מדרגה \(n_i\), המודול הוא מרחב וקטורי סוף-מימד \(\deg p_i\) עם בסיס \(\{1, x, \dots, x^{n_i-1}\}\). \\

   תחת בסיס זה, פעולת \(x\) (כלומר \(A\) במקור) מיוצגת על ידי המטריצה המלווה \(C_{p_i}\).\\

   כלומר, אם בפרק \(i\) נבחר יוצר \(v\) (תמונת 1 במודול \(\mathbb{F}[x]/(p_i)\)), אז \(x^k \cdot v\) מייצג \(x^k \bmod p_i\).\\

   המטריצה של הפעולה \(x\) בבסיס \(\{v, x v, \dots, x^{n_i-1} v\}\) היא בדיוק \(C_{p_i}\).


    \item מכיוון שהסכום הישיר \(V \cong \bigoplus_{i=1}^r \mathbb{F}[x]/(p_i)\) הוא איזומורפיזם של מודולים, בוחרים בסיס מתאים לכל גורם וצירוף לבסיס של \(V\). בבסיס זה מטריצת \(A\) היא מטריצה בלוקים אלכסונית עם הבלוקים \(C_{p_1}, \dots, C_{p_r}\). \\

   מכאן קיבלנו את הצורה הקנונית הרציונאלית של \(A\):\\
$$   A \sim    \begin{pmatrix}   C_{p_1} & & \\   & \ddots & \\   & & C_{p_r}   \end{pmatrix}.
   $$


  \end{enumerate}
\end{proof}
\begin{enumerate}
  \item לפי משפט המבנה, הפולינומים \(p_i\) (ה-invariant factors) הם יחידים עד כדי חברות, ומכן מגיע היחידות של צורת פרובניוס קנונית. 
\end{enumerate}
זה נותן לנו אלגוריתמים להכריע אם \(A\) דומות ל-\(B\).

\begin{corollary}
אם \(A, B \in M_n(\mathbb{F})\) ו-\(\mathbb{E}/\mathbb{F}\) הרחבת שדות, אז \(A \sim B\) מעל \(\mathbb{F}\) אם ורק אם \(A \sim B\) מעל \(\mathbb{E}\).

\end{corollary}
\begin{proof}
נוכיח את שני הכיוונים:

  \begin{enumerate}
    \item \textbf{כיוון ראשון:}\\

   אם \(A \sim_{\mathbb{F}} B\), קיימת \(P \in \mathrm{GL}_n(\mathbb{F})\) כך ש-\(P^{-1} A P = B\).\\

   מכיוון ש-\(\mathbb{F} \subseteq \mathbb{E}\), מתקיים \(P \in \mathrm{GL}_n(\mathbb{E})\), ולכן \(A \sim_{\mathbb{E}} B\).


    \item \textbf{כיוון שני:}\\

   נניח ש-\(A \sim_{\mathbb{E}} B\). לכל מטריצה מעל שדה יש צורה קנונית רציונלית יחידה (עד כדי סדר בלוקים).  


    \item נסמן ב-\(A'\) את הצורה הקנונית הרציונלית של \(A\) מעל \(\mathbb{F}\) (כלומר \(A' = Q^{-1} A Q\) עבור \(Q \in \mathrm{GL}_n(\mathbb{F})\)).  
    \item נסמן ב-\(B'\) את הצורה הקנונית הרציונלית של \(B\) מעל \(\mathbb{F}\).  
    \item הפולינומים האינווריאנטיים של \(A\) ו-\(B\) מעל \(\mathbb{E}\) זהים לאלו מעל \(\mathbb{F}\), כי הם נקבעים על ידי דטרמיננטות במטריצות מעל \(\mathbb{F}[x]\).  
    \item לכן, הצורה הקנונית הרציונלית של \(A\) (וכן \(B\)) מעל \(\mathbb{E}\) היא \(A'\) (בהתאמה \(B'\)).  
    \item משוויון הצורות הקנוניות מעל \(\mathbb{E}\) נקבל \(A' = B'\).  
    \item מכאן \(A \sim_{\mathbb{F}} A' = B' \sim_{\mathbb{F}} B\), כלומר \(A \sim_{\mathbb{F}} B\).
  \end{enumerate}
\end{proof}
\begin{corollary}
מודול נוצר סופית מעל PID הוא חופשי אם"ם הוא חסר פיתול.
כאשר חסר פיתול פשוט אומר:
$$r m =0 \implies r =0 \text{ or }m=0$$

\end{corollary}
מודול חופשי מעל תחום שלמות יהיה חסר פיתול. 

זה נותן את הכיוון ההפוך:
$$M\cong  \underbrace{ \frac{R}{(d_{1})}\oplus \dots \oplus \frac{R}{(d_{r})}  }_{ d_{i}\neq 0 }\oplus \underbrace{ \frac{R}{(0)}\oplus \dots \oplus \frac{R}{(0)} }_{ \text{mefutal} }$$
אם \(M\) לא חופשי \(d_{1}\neq 0\) ואז:
$$d_{1}\cdot (1,0,0,\dots,0)=0$$

\begin{example}
המודול \(\mathbb{Z}^{\mathbb{Z}}\) חסר פיתול אבל לא חופשי(כ-\(\mathbb{Z}\) מודל)

\end{example}
\chapter{אלגברות חילוק}

\section{חוגי חילוק}

\begin{definition}[חוג חילוק]
חוג שקיים לכל איבר שונה מ-0 הופכי כפלי. כלומר זה שדה ללא הקומוטטיביות.

\end{definition}
\begin{proposition}
יהי \(D\) חוג חילוק. אזי המרכז \(\mathbb{F}:= Z(D)\) של הוא שדה.

\end{proposition}
\begin{proof}
המרכז הוא חוג ובתור תת חוג של חוג חילוק יהיה חוג חילוק. כמו כן כיוון שהמרכז קומוטטיבי נקבל כי המרכז הוא חוג חילוק קומוטטיבי ולכן שדה.

\end{proof}
\begin{proposition}
החוג חילוק \(D\) הוא מרחב ווקטורי מעל \(\mathbb{F}=Z(D)\).

\end{proposition}
\begin{remark}
נתעניין במקרה שבו \(\dim_{\mathbb{F}}(D)<\infty\).

\end{remark}
\begin{definition}
יהי \(d \in D \setminus \mathbb{F}=D\setminus Z(D)\). אזי:
- מה שנוצר על ידי \(\times,+,F,d\) מסומן ב-\(F[d]\) הוא תת תחום קומוטטיבי של \(D\).
- מה שנוצר על ידי \(\div, \times, +, F,d\) מסומן על ידי \(F(d)\) וזה תת שדה של \(D\).

\end{definition}
\begin{proposition}
אם \(\dim_{F}(D)<\infty\) אזי \(F[d]=F(d)\).

\end{proposition}
\begin{proof}
מתישהו נקבל \(1,d,d^{2},\dots,d^{n}\) תלויים לינארית, ולכן:
$$a_{0}+a_{1}d+\dots+a_{n}d^{n}=0$$
אפשר להניח ש-\(a_{0} \neq 0\) כי אחרת נחלק ב-\(d\). ניתן כעת לחלק ב-\(a_{0}\) ולכן:
$$1=\left( -\frac{a_{1}}{a_{0}}-\frac{a_{2}}{a_{0}}d-\dots \right)d$$

\end{proof}
\begin{proposition}
תחום שלמות סוף מימדי מעל תת שדה הוא שדה.

\end{proposition}
\begin{proposition}
אם \(F=Z(D)\) ואם \(\dim_{F}D<\infty\) אזי \(F[d]\) שדה.

\end{proposition}
\section{מרחב ווקטורי מעל חוג חילוק}

\begin{definition}[מרחב וקטורי מעל חוג חילוק]
מרחב וקטורי נוצר סופית מעל חוג חילוק \(D\) הוא בעל בסיס סופי, וכל הבסיסים הם באותו גודל. כלומר, \(V \cong D^n\).

\end{definition}
\begin{proposition}[טרנספורמציות לינאריות]
$$\mathrm{End}(V) = \mathrm{End}_D(D^n) = M_{n \times n}(D^{\mathrm{op}}) \cong M_{n \times n}(D)^{\mathrm{op}}$$

\end{proposition}
מטריצה \(A \in M_n(D)\) פועלת על \(D^n\) לפי \(A \cdot v = vA\), כאשר \(\alpha \in D\) סקלר:
$$A \cdot (\alpha v) = (\alpha v)A = \alpha(vA) = \alpha(A \cdot v).$$

\begin{proposition}[המשפט העצוב של ודרברן]
חוג חילוק סופי הוא שדה.

\end{proposition}
\begin{proof}
  \begin{enumerate}
    \item נניח \(F = Z(D)\) (מרכז \(D\)), שהוא שדה. לכן \(F = \mathbb{F}_q\) עבור \(q = p^m\) ראשוני כלשהו. 


    \item אנו יודעים כי \(D\) הוא מרחב וקטורי מעל \(\mathbb{F}_q\), ולכן \(\lvert D \rvert = q^n\) כאשר \(n = \dim_F(D)\). 


    \item נניח בשלילה כי \(n \geq 2\). עבור \(d \in D \setminus F\), נגדיר את הקומוטטור \(C_D(d) = \{x \in D \mid xd = dx\}\). \(C_D(d)\) הוא תת-חוג חילוק ממש של \(D\) המכיל את \(F\), ולכן \(F \subseteq C_D(d) \subsetneq D\). נסמן \(e_d = \dim_{\mathbb{F}}(C_D(d))\), ולכן \(\lvert C_D(d) \rvert = q^{e_d}\). 


    \item נשתמש במשוואת המחלקה: 
$$q^n - 1 = \lvert D^\times \rvert = \lvert F^\times \rvert + \sum_d \frac{\lvert D^\times \rvert}{\lvert C_D(d)^\times \rvert},$$
כאשר \(d\) מייצג מחלקות צמידות לא טריוויאליות ב-\(D^\times\).


    \item נוכיח שאין פתרון עבור \(n > 2\).נשים לב כי \(e_d \mid n\) כי \(D\) הוא מרחב וקטורי מעל \(C_D(d)\).מתקיים \(\frac{x^n - 1}{x^{e_d} - 1} \in \mathbb{Z}[x]\). 
נשתמש בפולינומים ציקלוטומיים \(\Phi_n(x)\):
      $$x^n - 1 = \prod_{d \mid n} \Phi_d(x)\implies\Phi_n(q) \mid \frac{q^n - 1}{q^{e_d} - 1}$$


    \item ממשוואת המחלקה: 
$$\Phi_n(q) \mid (q - 1)$$
אבל \(\lvert \Phi_n(q) \rvert > q - 1\), סתירה, כיוון ש-\(\Phi_n(q)\) לא יכול לחלק את \((q - 1)\) עבור \(n > 2\).


  \end{enumerate}
\end{proof}
\section{אלגבראות}

\begin{definition}[אלגברה]
יהי \(\mathbb{F}\) שדה. \(\mathbb{F}\) אלגברה הוא חוג \(R\) עם \(\mathbb{F} \subseteq Z(R), \mathbb{F} \hookrightarrow Z(R)\)(זה גורר \(R\) גם מרחב ווקטורי מעל \(\mathbb{F}\)). אנחנו גם נדרוש \(\dim_{\mathbb{F}}R<\infty\).

\end{definition}
\begin{remark}
השיכון \(\varphi : \mathbb{F} \to Z(R)\) מאפשר לנו להגדיר כפל בסקלר בצורה שאנחנו רוצים. כיוון ש-\(\varphi\) הומומורפיזם של חוגים משמר חיבור, כפל, וממפה יחידה \(1_{\mathbb{F}} \mapsto 1_R\). כיוון שהתמונה היא \(Z(R)\) כל \(\varphi(\lambda)\) מתחלף עם כל האיברים של \(R\). לכן ניתן להגדיר כפל בסקלר על ידי:
$$\lambda \cdot r := \varphi\left( \lambda \right)r$$

\end{remark}
\begin{remark}
זוהי למעשה הגדרה של אלגברה מעל שדה. קיימת הגדרה כללית יותר של אלגברה מעל חוג כללי - עבור \(S\) חוג, \(S\)-אלגברה הוא חוג \(R\) עם בחירה של הומומורפיזם \(\iota:S\to Z(R)\). אנחנו לא נשתמש בהגדרה זו - אותנו נתעניין במקרה שבו \(S\) שדה ובמקרה זה הומומורפיזם הוא שיכון.

\end{remark}
\begin{example}
המטריצות הסקלאריות:
$$\mathbb{F} \subseteq M_{n}(\mathbb{F} )$$

\end{example}
\begin{proposition}[הגדרה שקולה לאלגברה]
אלגברה מעל שדה \(K\) היא זוג \(\left( A,\mu \right)\) כאשר \(A\) הוא מרחב ווקטורי מעל \(K\) ו-\(\mu:A\times A\to A\) היא העתקה \(K\) לינארית אוסצייטיבית אשר נקרא כלל המכפלה. 

\end{proposition}
\begin{proof}
נראה שהתכונות של אלגברה מקיימות את התכונות של מרחב ווקטורי:

  \begin{enumerate}
    \item פלוג מעל ווקטורים: 
$$\lambda\cdot(r+s)=\varphi(\lambda)(r+s)=\varphi(\lambda)r+\varphi(\lambda)s=\lambda\cdot r+\lambda\cdot s$$


    \item פילוג מעל סקלרים: 
$$(\lambda+\mu)\cdot r=\varphi(\lambda+\mu)r=(\varphi(\lambda)+\varphi(\mu))r=\varphi(\lambda)r+\varphi(\mu)r=\lambda\cdot r+\mu\cdot r$$


    \item אסוצייטיביות של מכפלה סקלרית: 
$$\lambda\cdot(\mu\cdot r)=\varphi(\lambda)(\varphi(\mu)r)=(\varphi(\lambda)\varphi(\mu))r=\varphi(\lambda\mu)r=(\lambda\mu)\cdot r$$


    \item יחידה כפלית: 
$$1\cdot r=\varphi(1)r=1_{R}r=r$$
ולכן נקבל כי \(R\) אכן מרחב ווקטורי. נראה כי מכפלה היא תבנית בילינארית. נשים לב כי:
$$\mu(\lambda r,s)=(\lambda\cdot r)s=(\varphi(\lambda)r)s=\varphi(\lambda)(r s)=\lambda\cdot(r s)=\lambda\mu(r,s)$$
כלומר לינארי במשתנה הראשון. וכן:
$$\mu(r,\lambda s)=r(\lambda\cdot s)=r(\varphi(\lambda)s)=\varphi(\lambda)(r s)=\lambda\cdot(r s)=\lambda\mu(r,s)$$
ולכן לינארי במשתנה השני.


  \end{enumerate}
\end{proof}
\begin{definition}[אלגברה קומוטטיבית]
אלגברה שבו המכפלה המוגדרת היא קומוטטיבית.

\end{definition}
\begin{example}[חוג הפולינומים]
עבור שדה \(K\) החוג \(K[x]\) היא אלגברה קומוטטיבית מעל שדה \(K\).

\end{example}
\begin{example}
אם \(L / K\) הרחבת שדות אז \(L\) היא אלגברה קומוטטיבית מעל \(K\).

\end{example}
\begin{definition}[מרכז של אלגברה]
$$Z(A)=\{z\in A\mid a z=z a{\mathrm{~for~all~}}a\in A\}$$

\end{definition}
\begin{remark}
כל חוג חילוק הוא אלגברה מעל המרכז שלו, אבל אולי אינסוף מימדית. אנחנו נחקור את הסוף מימדית. לכן כאשר נתייחס לאלגברת חילוק(אלגברה שהיא חוג חילוק) נתכוון למקרה הסוף מימדי.

\end{remark}
\begin{example}
אלגברת המטריצות \(M_{n}(K)\) כאשר \(K\) שדה יהיה עם מרחב \(K\) - כאשר \(K\) מייצג את המטריצות הסקלאריות.

\end{example}
\begin{reminder}
ווברברן הראה שאין חוגי חילוק סופיים מלבד שדות.

\end{reminder}
\begin{proposition}
אם \(\mathbb{F}=\overline{F}\) אז אין אלגבראות חילוק מעל \(\mathbb{F}\) מלבד \(\mathbb{F}\).

\end{proposition}
\begin{proof}
אם \(D\supsetneq \mathbb{F}\) היא \(\mathbb{F}\) אלגברה וחוג חילוק, ניקח \(x \in D \setminus \mathbb{F}\) ואז \(\mathbb{F}(x)=\mathbb{F}[x]\) שדה כי \(\mathbb{F}\subseteq Z(D)\) וזה סתירה לכך ש-\(\mathbb{F}=\overline{\mathbb{F}}\). 

\end{proof}
\begin{definition}[מורפיזם של אלגבראות]
יהי \(A\) אלגברה מעל שדה \(K\). העתקה \(K\) לינארית \(f:A\to B\) המקיימת לכל \(a,a' \in A\):
$$f(aa')=f(a)f(a')$$
כלומר העתקה לינארית אשר משמרת גם כפל.

\end{definition}
\begin{definition}[תת אלגברה]
יהי \(A\) אלגברה. תת מרחב ווקטורי \(B\) של \(A\) אשר סגור למכפלה נקראת תת אלגברה.

\end{definition}
\begin{proposition}
  \begin{enumerate}
    \item חיתוך של קבוצה של תתי אלגברות היא תת אלגברה. 


    \item התמונה של מורפיזם של אלגבראות \(f:A\to B\) היא תת אלגברה. 


  \end{enumerate}
\end{proposition}
\begin{example}
עבור אלגברה \(A\) מעל שדה \(K\) מכלל המכפלה מתקיים לכל \(a \in A\) ו-\(\lambda \in K\) כי:
$$(\lambda{\cdot}1_{A})a=1_{A}(\lambda{\cdot}a)=(\lambda{\cdot}a)1_{A}=a(\lambda{\cdot}1_{A})$$
ולכן \(K\cdot 1_{A}\subseteq Z(A)\) וכן \(K\cdot 1_{A}\) היא תת אלגברה של \(Z(A)\). לכן ההעתקה:
$$K\mapsto Z(A)\qquad \lambda \mapsto \lambda \cdot 1_{A}$$
היא מורפיזם לא טריוויאלי.

\end{example}
\begin{definition}[מרכז - centralizer]
תהי \(A\) אלגברה ו-\(B\subseteq A\) תת קבוצה אז המרכז יהיה:
$$C_{A}(B)=\{a\in A\mid a b=b a{\mathrm{~for~all~}}b\in B\}$$
כאשר זה תת אלגברה של \(A\) ומקיים \(C_{A}(A)=Z(A)\).

\end{definition}
\begin{definition}[אלגברת חילוק]
אלגברה אשר החוג בו הוא חוג חילוק - כלומר לכל איבר פרט לאפס יש הופכי כפלי.

\end{definition}
\begin{definition}[אלגברה מתפצלת]
אלגברה נקראת מתפצלת אם איזומורפית לאלגברת מטריצות.

\end{definition}
\begin{definition}[אלגברה הפוכה]
תהי \(A\) אלגברה. ניתן להגדיר את הקבוצה:
$$A^{o p}=\{a^{o p}\mid a\in A\}$$
ביחד עם הפעולות:
$$k\times A^{o p}\longrightarrow A^{o p}\qquad (\lambda,a^{o p})\longmapsto(\lambda{\cdot}a)^{o p}$$$$A^{o p}\times A^{o p}\longrightarrow A^{o p}\qquad (a_{1}^{o p},a_{2}^{o p})\longmapsto(a_{1}+a_{2})^{o p}$$$$A^{o p}\times A^{o p}\longrightarrow A^{o p}\qquad (a_{1}^{o p},a_{2}^{o p})\longmapsto(a_{2}a_{1})^{o p}$$
וזה יוצר אלגברה.

\end{definition}
\section{קווטרניונים של המילטון}

\begin{definition}[הקווטרניונים של המילטון]
$$\mathbb{H} =\{ r+x i + yj+zk\mid r,x,y,z \in \mathbb{R} \}=\mathbb{R}^{\oplus \{ 1,i,j,k \}}$$
החיבור הוא חיבור קורדינטה קורדינטה ועבור הכפל מספיק להראות איך מכפילים שתי איברי בסיס:
$$i^{2}=j^{2}=k^{2}=-1\qquad ij=k=-jk\quad jk=i=-kj\quad ki=j=-ik$$
כאשר \(\mathbb{R}\subseteq Z(\mathbb{H})\).

\end{definition}
\begin{proposition}
מספיק לכתוב \(i^{2}=j^{2}=k^{2}=ijk=-1\). עבור \(ij\):
$$ij=-ijk^{2}=-k=k$$
כאשר עבור \(jk=i,ki=j\) וכעת:
$$ji=j^{2}k=-k$$
ומזה נובע כל האחרים.

\end{proposition}
\begin{remark}
הקושי בחוג הקווטרניונים זה להראות אסוצייטיביות. 

\end{remark}
\begin{reminder}
ניתן לחשוב על \(\mathbb{C}\) על ידי:
$$\mathbb{C}\cong \left\{ \begin{pmatrix}a & b \\-b & a
\end{pmatrix} \in M_{2}(\mathbb{R})\right\}$$
ואז אסוצייטיביות ב-\(\mathbb{C}\) נובעת מכך שחוג המטריצות אסוצייטביות. 

\end{reminder}
נרצה לבנות מבנה דומה עבור הקווטרניונים. נכתוב:
$$\mathbb{H} = \{ r+x i + yj+zk\mid r,x,y,z \in \mathbb{R} \}=\{ \alpha+\beta j\mid \alpha,\beta \in \mathbb{C} \}$$
זה עובד כי בתור מרוכבים ניתן לכתוב \(\alpha=r+x i,\beta = y+zi\) וכעת \(\beta j=yj+zij=yj+zk\). \(\mathbb{C}\subseteq \mathbb{H}\) אבל \(\mathbb{H}\) הוא לא \(\mathbb{C}\)-אלגברה כי \(\mathbb{C}\not\subseteq Z(\mathbb{H})\). מה כן?
$$\forall \alpha \in \mathbb{C}\quad \alpha j=j\overline{\alpha}$$$$(a+bi)j=aj+bk$$$$j(a-bi)=aj-bji=aj+bk$$$$(\alpha+\beta j)(\gamma+\delta j)=\alpha \gamma+\beta j\gamma+\alpha \delta j+\beta j\delta j=\alpha \gamma+\beta \overline{\gamma } j+\alpha \delta j+\beta \overline{\delta}j^{2}=(\alpha \gamma-\beta \overline{\delta} )+(\beta \overline{\gamma} +\alpha \delta)j $$
נשכן את \(\mathbb{H}\hookrightarrow M_{2}(\mathbb{C})\). וזה יראה אסוצייטיביות. צריך להבין איך \(\mathbb{H}\) פועל על \(\mathbb{H}_{\{ i,j \}}\cong \mathbb{C}^{2}\) כטרנספורמטייה לינארית.
$$\overbrace{ 1 }^{ \begin{pmatrix}1 & 0\end{pmatrix} }\cdot(\alpha+\beta j)=\overbrace{ \alpha+\beta j }^{ \begin{pmatrix}\alpha & \beta
\end{pmatrix} }$$$$\overbrace{ j }^{ \begin{pmatrix}0 & 1\end{pmatrix} }(\alpha+\beta j)=\overbrace{ -\overline{\beta} +\overline{\alpha} j }^{ \begin{pmatrix}-\overline{\beta}  & \overline{\alpha}  
\end{pmatrix} }$$
סה"כ \(\mathbb{H}\hookrightarrow M_{2}(\mathbb{C})\) כאשר השיכון מוגדר על ידי:
$$\alpha+\beta j\mapsto \begin{pmatrix}\alpha & \beta \\-\overline{\beta} & \overline{\alpha} 
\end{pmatrix}$$$$\mathbb{H}' :=\left\{ \begin{pmatrix}\alpha & \beta \\-\overline{\beta} & \overline{\alpha} 
\end{pmatrix} \mid \alpha,\beta \in \mathbb{C}\right\}$$
עם חיבור וכפל מטריצות ב-\(M_{2}(\mathbb{C})\).

\begin{proposition}
הבנייה שעשינו \(\mathbb{H}'\) הוא חוג, ומתקיים \(\mathbb{H}\cong\mathbb{H'}\) כחוגים.

\end{proposition}
\begin{proof}
מתקיים:
$$r+x i + y j + zk=(r+x i+yj+zk)=(r+x i)+(y+z i)j$$
ולכן ניתן לכתוב:
$$\begin{pmatrix}r+x i & y + z i \\-y+zi & r-x i
\end{pmatrix}$$
ולכן נקבל:
$$i'=\begin{pmatrix}i & 0 \\0 & -i\end{pmatrix}\qquad j'=\begin{pmatrix}0 & 1 \\-1 & 0\end{pmatrix}\qquad k'=\begin{pmatrix}0 & i \\i & 0
\end{pmatrix}$$
בודקים שמתקיים \(i^{2}=j^{2}=k^{2}=ijk=-1\) ובפרט \(1,i,j,k\) בת"ל.

\end{proof}
\begin{definition}[צמוד]
$$\overline{r+x i + yj + zk}=r-x i - yj - zk $$

\end{definition}
\begin{proposition}
עבור \(\alpha,\beta \in \mathbb{H}\) מתקיים:
$$\overline{\alpha+\beta} =\overline{\alpha} +\overline{\beta} \qquad \overline{\alpha \beta} =\overline{\beta} \overline{\alpha} $$
צמוד ב-\(\mathbb{H}\) הופך לצימוד + שחלוף ב-\(M_{2}(\mathbb{C})\).

\end{proposition}
\begin{reminder}
עבור \(E / F\) גלואה נגדיר \(N_{E / F}: \mathbb{E}^{\times}\to \mathbb{F}^{\times}\) על ידי:
$$N_{E / F}(\alpha)=\prod_{\sigma \in \mathrm{Gal}(E / F)} \sigma(\alpha)$$

\end{reminder}
\begin{definition}[נורמה ועקבה]
נורמה:
$$N(\alpha):= \alpha \overline{\alpha}\qquad \mathrm{Tr}(\alpha)=\alpha+\overline{\alpha}  $$

\end{definition}
\begin{proposition}[תכונות של נורמה ועקבה]
$$\overline{N(\alpha)} =\overline{\alpha \overline{\alpha} } =\alpha \overline{\alpha} =N(\alpha)\implies N(\alpha)\in \mathbb{R}$$$$\mathrm{Tr}(r+x i + y j + zk)=2r$$$$N(\alpha)=\alpha \overline{\alpha} =\begin{pmatrix}u & w \\ \overline{w}  & \overline{u}  \end{pmatrix}\begin{pmatrix}\overline{u}  & -\overline{w}  \\\overline{w}  & \overline{u} \end{pmatrix}=\begin{pmatrix}u\overline{u} +w\overline{w} & 0  \\0 & u\overline{u}   +w\overline{w}  
\end{pmatrix}$$
כלומר:
$$N(\alpha)=\alpha \overline{\alpha} =\lvert u \rvert_{\mathbb{C}} ^{2}+\lvert w \rvert _{\mathbb{C}}^{2}=r^{2}+x^{2}+y^{2}+z^{2}$$

\end{proposition}
\begin{remark}
זה לא נורמה במובן של אנליזה כי לא הומוגני.

\end{remark}
\begin{corollary}
אם \(\alpha \neq 0\) אז \(N(\alpha)> 0\) ולכן:
$$\forall \alpha \neq 0\quad \alpha \cdot \frac{\overline{\alpha}}{N(\alpha)}=1 $$
ולכן \(\mathbb{H}\) חוג חילוק. כאשר:
$$(r+x i +yj + zk)^{-1}= \frac{r - x i - yj - zk}{r^{2}+x^{2}+y^{2}+z^{2}}$$

\end{corollary}
ראינו \((\mathbb{H},N)\cong(\mathbb{R}^{4},\lVert \cdot \rVert^{2}_{\text{standard}})\). 

\begin{definition}[קווטרניונים טהורים]
ניתן כעת להגדיר:
$$\mathbb{P} =\mathrm{Span}_{\mathbb{R}}(i,j,k)=\{ \alpha \mid \alpha=-\overline{\alpha}  \}=\{ \alpha \mid \mathrm{Tr}(\alpha)=0 \}$$

\end{definition}
\begin{proposition}
$$(\mathbb{P} ,N)\cong (\mathbb{R}^{3},\lVert \cdot \rVert _{\text{standard}})$$

\end{proposition}
\begin{proof}
נסתכל על \(\mathbb{H}^{\times}\circlearrowright \mathbb{P}\) על ידי הצמדה.
$$\alpha \in \mathbb{H} ^{\times}= \mathbb{H} \setminus  \{ 0 \}$$
יהי \(\pi \in \mathbb{P}\). נראה \(\alpha ^{-1}\pi \alpha \in \mathbb{P}\). נשים לב ש-\(\overline{\alpha ^{-1}}=\overline{\alpha}^{-1}\). כעת:
$$\overline{\alpha ^{-1}  \pi \alpha}=\overline{\frac{\overline{\alpha}}{\alpha \overline{\alpha} } \pi \alpha}= \frac{1}{\alpha \overline{\alpha} }\overline{\overline{\alpha} \pi \alpha} =-\frac{1}{\alpha \overline{\alpha} }\overline{\alpha} \pi \alpha=-\alpha ^{-1}  \pi \alpha$$

\end{proof}
\begin{proposition}
$$Z(\mathbb{H} )= \mathbb{R}$$

\end{proposition}
\begin{proof}
$$r\alpha=(r+x i + yj + zk)j=rj+xk-y-zi$$
כאשר אם נכפיל מהצד השני:
$$j\alpha=rj-xk-y+zi$$

\end{proof}
\begin{proposition}
$$\mathbb{H} = \mathbb{R}\oplus \mathbb{P} $$

\end{proposition}
\begin{proof}
כעת נסתכל על \(\alpha^{2}\):
$$\alpha^{2}=\underbrace{ (\dots) }_{ \in \mathbb{R} }+(rx+xr+\cancel{ yz-zy })i+\dots=\underbrace{ (\dots) }_{ \in \mathbb{R} }+2rx i+2ryj+2rzk$$
וזה אומר כי:
$$\alpha^{2}\in \mathbb{R} \iff r = 0 \lor x=y=z=0$$
כלומר:
$$\mathbb{P} = \{ \alpha \mid \alpha^{2}\in Z(\mathbb{H} )\not\ni \alpha  \}\cup  \{  0 \}$$

\end{proof}
\begin{corollary}
$$\alpha=r+\pi \qquad \overline{\alpha} =r-\pi$$

\end{corollary}
\begin{corollary}
ראינו ש-\((\mathbb{H},N)=(\mathbb{R}^{4},\lVert \cdot \rVert_{\text{Euc}}^{2})\). מתקיים:
$$\mathbb{H} ^{1}=\{ \alpha \mid N(\alpha)=1 \}=S^{3}$$
ואנחנו קיבלנו מבנה של חבורה על \(S^{3}\).

\end{corollary}
\begin{proposition}[שיכון הקווטרניונים במטריצות]
ניתן לשכן \(\iota:\mathbb{H}\to M_{2}(\mathbb{C})\) על ידי:
$$\alpha \in \mathbb{H} ^{1}\mapsto A = \begin{pmatrix}u & w \\-\overline{w}  & \overline{u} 
\end{pmatrix}$$
כאשר \(\det A=1\) ו-\(\iota(\overline{\alpha})=A^{*}\).

\end{proposition}
\begin{proposition}[שיכון \(S^{3}\) ב-\(SU(2)\)]
$$\alpha \in \mathbb{H} ^{1}\to A \in SU(2)=\{ A\mid  A^{*}A=\mathbb{1} ,\det A=\mathbb{1}  \}$$
כאשר \(\iota\) היא חח"ע ועל, ולכן:
$$S^{3}\cong  \mathbb{H} ^{1}\cong  SU(2)$$

\end{proposition}
\begin{proposition}[הגאומטריה תלת-ממדית]
$$\mathbb{P} =\mathrm{Span}(i,j,k)=\{ \pi=-\overline{\pi}  \}$$
כאשר \((\mathbb{P} ,N)=(\mathbb{R} ^{3},\lVert \cdot \rVert ^{2}_{\text{Euc}})\).

\end{proposition}
\begin{proposition}[נורמה ותכונות אורתוגונליות]
$$\pi \in \mathbb{P} \qquad N(\pi)=\pi \cdot \overline{\pi} =-\pi^{2}$$
פולריזציה:
$$\langle \pi,\xi \rangle =-\frac{1}{2}\{ \pi,\xi \}$$
בפרט \(\pi \perp \xi\) אם"ם \(\pi \xi=-\xi \pi\).

\end{proposition}
\begin{proposition}[בסיס אורתונורמלי]
יהיו \(\pi_{1},\pi_{2},\pi_{3}\in \mathbb{P}\) בסיס אורתונורמלי ל-\(\mathbb{P}=\mathbb{R}^{3}\) אם"ם הם מקיימים את חוקי הכפל של \(i,j,k\):
$$\pi_{i}\cdot \pi_{j}=\pm \pi_{k}\qquad \pi_{j}^{2}=-1$$

\end{proposition}
\begin{definition}[הצמדה ופונקציה \(\rho\)]
הצמדה \(\mathbb{H}^{\times}\circlearrowright \mathbb{P}\) מוגדרת על ידי:
$$\rho:\mathbb{H} \to O(3)\qquad \alpha \mapsto \gamma_{\alpha}\qquad \gamma_{\alpha}(\pi)=\alpha   \pi \alpha ^{-1}$$\\

כאשר \(\rho\) מצטמצמת ל-\(\rho:\mathbb{H}^{1}\to SO(3)\).

\end{definition}
\begin{proposition}
הגרעין של \(\rho\) נתון על ידי:
$$\ker \rho=\{ \pm 1 \}$$

\end{proposition}
\begin{proof}
מתקיים \(C_{\mathbb{H} }(\mathbb{P} )=\mathbb{R}\), ולכן \(\rho(\alpha)=1\) אם"ם \(\alpha \in \mathbb{R}\).  מכיוון ש-\(\mathbb{R}\cap \mathbb{H}^{1}=\{ \pm 1 \}\), נובע ש-\(\ker \rho=\{ \pm 1 \}\).

\end{proof}
\begin{proposition}
$$SU(2)\cong S^{3}\cong \mathbb{H} ^{1}\xrightarrow{2:1} SO(3)$$
כאשר יש העתקה נוספת \(S^{3}\xrightarrow{v\sim  -v}\mathbb{R}P^{3}\), ו-\(\mathbb{R}P^{3}=SO(3)\).\\

באופן כללי, \(SO(n)\) אינו פשוט קשר ויש לו כיסוי כפול פשוט קשר \(\text{Spin}(n)\), כך ש-\(\text{Spin}(3)=SU(2)\).

\end{proposition}
\section{אלגברה קווטרניונית}

\begin{definition}[אלגברה קווטרניונית]
יהי \(F\) שדה ו-\(a, b \in F^{\times}\). האלגברה הקווטרניונית \(\left( \frac{a,b}{F} \right)\) היא אלגברה אסוציאטיבית מממד \(4\) מעל \(F\), הנוצרת על ידי איברים \(i, j\) עם יחסים:
$$i^2 = a, \quad j^2 = b, \quad ij = -ji$$
נגדיר \(k = ij\), ואז \(k^2 = -ab\). כל איבר באלגברה ניתן לכתיבה יחידה כ-\(x = r + x_1 i + x_2 j + x_3 k\) עם \(r, x_1, x_2, x_3 \in F\).

\end{definition}
\begin{proposition}
המרכז של \(\left( \frac{a,b}{F} \right)\) הוא \(F\).

\end{proposition}
\begin{definition}[צמוד]
הצמוד של \(x\) הוא:
$$\overline{x} = r - x_1 i - x_2 j - x_3 k$$

\end{definition}
\begin{definition}[נורמה]
$$N(x) = x \overline{x} = r^2 - a x_1^2 - b x_2^2 + ab x_3^2$$

\end{definition}
\begin{definition}[עקבה]
$$\mathrm{Tr}(x) = x + \overline{x} = 2r$$

\end{definition}
\begin{proposition}
אלגברת קווטרניונים היא או מתפצלת או אלגברת חילוק.

\end{proposition}
\begin{proposition}
החוג \(\left( \frac{a,b}{F} \right)\) הוא חוג חילוק אם ורק אם \(N(x) = 0\) רק עבור \(x = 0\). כלומר לכל \(x,y,z,t \in F\) נקבל:
$$x^{2}-a y^{2}-b z^{2}+a b t^{2}=0\implies x=y=z=t=0$$

\end{proposition}
\begin{example}
$$\left( \frac{1,1}{F} \right) \cong M_{2}(F)$$
עם הזיהוי \(i = \begin{pmatrix} 1 & 0 \\ 0 & -1 \end{pmatrix}\), \(j = \begin{pmatrix} 0 & 1 \\ 1 & 0 \end{pmatrix}\), \(k = ij\). וכן:
$$\mathbb{H} = \left( \frac{-1,-1}{\mathbb{R}} \right)$$
אלו הם הקווטרניונים של המילטון.

\end{example}
\begin{proposition}
$$\left( \frac{a,b}{F} \right) \cong \left( \frac{b,a}{F} \right)$$

\end{proposition}
\begin{proposition}
$$\forall c \in F^{\times}\qquad \left( \frac{a,b}{F} \right) \cong \left( \frac{c^2 a, b}{F} \right)$$

\end{proposition}
\begin{corollary}
$$\forall u \in F[i]^{\times}\qquad\left( \frac{a,b}{F} \right) \cong \left( \frac{a, b \cdot N_{F[i]/F}(u)}{F} \right)$$

\end{corollary}
\begin{reminder}
אלגברה קווטרניונית  \(A = \left( \frac{a,b}{F} \right)\) נקראת מתפצלת אם \(A \cong M_2(F)\).

\end{reminder}
\begin{proposition}
יהי \(A = \left( \frac{a,b}{F} \right)\). התנאים הבאים שקולים:
- קיימת \(\alpha \neq 0\) עם \(N(\alpha) = 0\).
- החוג \(A\) אינה חוג חילוק.
- החוג \(A \cong M_2(F)\) (מתפצלת).
- האיבר \(b\) היא נורמה מההרחבה \(F[\sqrt{a}]\)(או \(a\) ריבועי ב-\(F\)).
- יש פתרון לא טריוויאלי ב-\(F\) למשוואה:
$$ax^{2}+by^{2}=z^{2}$$

\end{proposition}
\section{פי אדים}

\begin{definition}[חוג המספרים הp אדים]
לכל ראשוני \(p\) מגדירים את חוג המספרים ה-\(p\)-אדיים \(\mathbb{Z}_p\) בתור השלמת \(\mathbb{Z}\) ביחס לנורמת \(p\)-אדית:
$$|x|_p = p^{-v_p(x)},$$
כאשר \(v_p(x)\) הוא המעריך של \(p\) בפירוק של \(x\).

\end{definition}
\begin{proposition}[פיתוח \(p\) אדי]
כל \(x \in \mathbb{Z}_p\) ניתן לכתיבה בצורה:
$$x = a_0 + a_1 p + a_2 p^2 + \dots,$$
כאשר \(a_i \in \{0,1,\dots,p-1\}\).

\end{proposition}
\begin{example}
\(5\) ב-\(7\)-אדי: 
$$5 = 5 + 0\cdot 7 + 0\cdot 7^2 + \dots$$\(-1\) ב-\(3\)-אדי: 
$$-1 = 2 + 2\cdot 3 + 2\cdot 3^2 + \dots$$

\end{example}
\begin{definition}[ולואציה]
לכל \(x \in \mathbb{Q}_p^{\times}\), אפשר לכתוב \(x = p^n u\) כאשר \(u \in \mathbb{Z}_p^{\times}\) הפיך, ומגדירים:
$$v_p(x) = n$$

\end{definition}
\begin{proposition}
החוג המספרים ה-\(p\) אדים \(\mathbb{Z}_p\) הוא תחום ראשי, מקומי, שלם.

\end{proposition}
\begin{definition}[שדה המספרים ה-\(p\) אדים]
ההשלמה \(\mathbb{Q}_p\) הוא שדה המספרים ה-\(p\)-אדיים, כלומר השלמת \(\mathbb{Q}\) ביחס לנורמה \(p\)-אדית.

\end{definition}
\begin{proposition}
ההשלמה \(\mathbb{Q}_{p}\) היא אכן שדה.

\end{proposition}
\begin{proposition}
החוג \(\mathbb{Z}_p\) הוא תת-חוג קומפקטי של \(\mathbb{Q}_p\).

\end{proposition}
\begin{definition}
איבר \(a \in \mathbb{Q}_p^{\times}\) נקרא ריבוע אם קיים \(x \in \mathbb{Q}_p^{\times}\) כך ש-\(x^2 = a\).

\end{definition}
\begin{proposition}
\(-1\) הוא ריבוע ב-\(\mathbb{Q}_p\) אם ורק אם \(p \equiv 1 \pmod{4}\) או \(p=2\).

\end{proposition}
\begin{proof}
ב-\(\mathbb{F}_p\), \(-1\) הוא ריבוע אם ורק אם \(p \equiv 1 \pmod{4}\) או \(p=2\) (משפט האיבר הפרימיטיבי). לכן, גם ב-\(\mathbb{Q}_p\) מתקיים אותו תנאי.

\end{proof}
\begin{proposition}
האלגברת קווטרניונים \(\left(\frac{-1,-1}{\mathbb{Q}_p}\right)\) היא אלגברת חילוק אם ורק אם \(p=2\). עבור \(p \neq 2\), האלגברה מתפצלת: 
$$\left(\frac{-1,-1}{\mathbb{Q}_p}\right) \cong M_2(\mathbb{Q}_p)$$

\end{proposition}
\begin{proposition}
אם \(-1\) הוא ריבוע ב-\(\mathbb{Q}_p\), אז \(\left(\frac{-1,-1}{\mathbb{Q}_p}\right)\) מתפצלת, כאשר אם \(-1\) אינו ריבוע ב-\(\mathbb{Q}_p\), אז \(\left(\frac{-1,-1}{\mathbb{Q}_p}\right)\) היא חילוק.

\end{proposition}
\begin{proposition}
עבור \(p \neq 2\), אם \(a \in \mathbb{F}_p \setminus (\mathbb{F}_p^2)\) (כלומר \(a\) אינו ריבוע בשדה השאריות), אז \(\mathbb{Q}_p[\sqrt{a}]\) היא הרחבת שדה של \(\mathbb{Q}_p\).

\end{proposition}
\begin{proposition}
האלגברה \(\left(\frac{a,b}{\mathbb{Q}_p}\right)\) היא חילוק אם ורק אם \(b\) אינו נורמה מההרחבה \(\mathbb{Q}_p[\sqrt{a}]\). בפרט, אם \(a \notin \mathbb{F}_p^2\), אז \(\left(\frac{a,p}{\mathbb{Q}_p}\right)\) היא אלגברת חילוק.

\end{proposition}
\begin{corollary}
יש בדיוק שתי אלגברות קווטרניונים מעל \(\mathbb{Q}_p\) (עד כדי איזומורפיזם): אלגברת חילוק ו-\(M_2(\mathbb{Q}_p)\) (המתפצלת).

\end{corollary}
\section{ההרחבת שדות \(\mathbb{Q}_{p}[i]\)}

\begin{reminder}
עבור שדה \(F\) ההרחבה \(F[x]\) זו ההוספה של \(x\) לשדה וכל הצירופים הלינארים שלו(חיבור וכפל)

\end{reminder}
\begin{proposition}
ההרחבה \(\mathbb{Q}_p[i]\) היא הרחבה ריבועית של \(\mathbb{Q}_p\) אם ורק אם \(-1\) אינו ריבוע ב-\(\mathbb{Q}_p\).

\end{proposition}
\begin{proposition}
כל איבר ב-\(\mathbb{Q}_p[i]\) ניתן לכתיבה יחידה כ-\(a + b i\) עם \(a, b \in \mathbb{Q}_p\). כלומר:
$$\mathbb{Q}_p[i] = \mathbb{Q}_p \oplus i\mathbb{Q}_p$$

\end{proposition}
\begin{definition}
חוג השלמים של ההרחבה הוא 
$$\mathbb{Z}_p[i] = \{ \alpha \in \mathbb{Q}_p[i] \mid \mathrm{val}(\alpha) \geq 0 \}$$

\end{definition}
\begin{proposition}
איברי היחידה של חוג השלמים הם 
$$\mathbb{Z}_p[i]^{\times} = \{ \alpha \in \mathbb{Q}_p[i] \mid \mathrm{val}(\alpha) = 0 \}$$

\end{proposition}
\begin{proposition}
חבורת האיברים ההפיכים של \(\mathbb{Q}_p[i]\) היא $$\mathbb{Q}_p[i]^{\times} = p^{\mathbb{Z}} \times \mathbb{Z}_p[i]^{\times}$$

שדה השאריות הוא \(\mathbb{F}_p[i]\), שהוא שדה מדרגה \(2\) מעל \(\mathbb{F}_p\) אם \(-1\) אינו ריבוע ב-\(\mathbb{F}_p\) (כלומר \(\mathbb{F}_p[i] = \mathbb{F}_{p^2}\)).

\end{proposition}
\chapter{אלגבראות פשוטות מרכזיות}

\section{אלגברה פשוטה מרכזית}

\begin{definition}[אלגברה פשוטה מרכזית]
אלגברה \(A\) מעל שדה \(\mathbb{F}\) נקראת אלגברה פשוטה מרכזית אם המרכז שלה הוא שדה(כלומר \(Z(A) = \mathbb{F}\)) והאלגברה אינה מכילה אידיאלים נלווים שונים מאפס(פשוטה). אם השדה של המרכז הוא \(\mathbb{F}\) נאמר שהאלגברה היא \(\mathbb{F}-\mathrm{CSA}\).

\end{definition}
\begin{example}
המטריצות \(M_{n}(\mathbb{R})\) ו-\(M_{n}(\mathbb{H})\) הן \(\mathbb{R}-CSA\) כאשר \(\text{char}(F) \neq 2\).
בנוסף, \(\left( \frac{a,b}{F} \right)\) היא \(\mathbb{F}-CSA\) אם היא חוג חילוק או \(M_{2}(F)\) עבור \(a, b \in \mathbb{F}^{\times}\).

\end{example}
\begin{proposition}
כל הומומורפיזם מאלגברה פשוטה מרכזית (CSA) הוא שיכון.

\end{proposition}
\begin{corollary}
בפרט, אם \(A\) היא \(\mathbb{F}-CSA\) ו-\(F \subseteq E \subseteq A\) תת-שדה, אז:

$$\rho: A \hookrightarrow M_{n}(E),$$
כאשר \(\rho(a)\) מוגדר על ידי:
$$\rho(a) = [\mu_{a}]_\mathcal{B}^T$$
כאשר \(\mathcal{B}\) הוא בסיס של \(E\) מעל \(F\), ו-\(\mu_{a}: A \to A\) היא הפעולה על אלמנט \(a \in A\) המוגדרת על ידי הכפל ב-\(a\). כלומר, \(\mu_{a}(x) = ax\) לכל \(x \in A\).

\end{corollary}
\begin{proposition}
אם \(A\) היא \(\mathrm{CSA}\) מעל \(F\), אז:
-  יש ל-\(A\) מודול פשוט יחיד עד כדי איזומורפיזם, נסמנו \(I\).
-  כל \(A\)-מודול נוצר סופית איזומורפי ל-\(I^{m}\) עבור \(m \in \mathbb{N}\), ו-\(I^{m} \cong I^{n}\) אם"ם \(m = n\).
-  אם \(A^{M,M'}\), אז \(M \cong M'\) אם"ם \(\dim_{F}M = \dim_{F}M'\).
-  \(A^{M}\) חופשי אם"ם \(\dim A \mid \dim M\).

\end{proposition}
\begin{theorem}[ארתין ודרברן]
אלגברה פשוטה מרכזית היא \(M_{n}(D)\) כש-\(D\) אלגברת חילוק מרכזית (ולהיפך). להיפך: אם \(D\) היא אלגברת חילוק מרכזית אז \(M_{n}(D)\) אלגברה פשוטה מרכזית.

\end{theorem}
\begin{proof}
אם A הוא CSA אז \(A\cong I^{d_{A}}\) כ-\(A\) מודול(I המודול הפשוט של A. 
$$A^{\text{op}}\cong  \mathrm{End}_{A}(A)\cong  \mathrm{End} _{A}(I^{d_{A}})\cong  M_{d_{A}}(\mathrm{End} _{d_{A}}(I))$$
כאשר \(\mathrm{End}_{A}(I)\) אלגברת חילוק מהלמה של שור. לכן:
$$A\cong  M_{d_{A}}(\mathrm{End} _{A}(I))^{\text{op}}\cong  M_{d_{A}}(\mathrm{End} _{A}(I)^{\text{op}})$$
כאשר \(\mathrm{End}_{A}(I)^{\text{op}}\) אלגברת חילוק.

$$A\cong  M_{d_{A}}(D)$$
כש-\(D\) אלגברת חילוק.

\end{proof}
\begin{corollary}
אם \(\overline{F} = F\) אז אין CSA מלבד \(M_{n}(F)\) גם אם \(F\) סופית.

\end{corollary}
\section{דוגמאות ל-CSA}

\begin{definition}[אלגברה ציקלית]
$$(E,\sigma,a)=E\{ x \} / \langle x\alpha= \sigma(\alpha)x,x^{n}=a \rangle $$

\end{definition}
עבור למשל \(\mathbb{C}\) ו-\(\tau\) הצמוד נקבל:
$$(\mathbb{C},\tau,-1)= \mathbb{C}\{ -1 \} / \langle ju = \overline{u} j\mid  j^{2} = -1 \rangle $$

\begin{definition}[אלגברת סימבול]
יהי \(\mathbb{F}\) שדה, \(a, b \in \mathbb{F}^{\times}\) ו-\(\zeta \in \mathbb{F}\) שורש יחידה פרימיטיבי מסדר \(n\) כך ש-\(\text{char}(\mathbb{F}) \not\mid n\).
נגדיר את אלגברת הסימבול:
$$\left( \frac{a, b}{\mathbb{F}, \zeta} \right) = \mathbb{F}\{i, j\} / \langle i^n - a, j^n - b, ji - \zeta ij \rangle$$
כאשר \(\mathbb{F}\{i, j\}\) היא אלגברת הפולינומים הלא קומוטטיביים מעל \(\mathbb{F}\).

\end{definition}
\begin{proposition}
אלגברת הסימבול \(\left( \frac{a, b}{\mathbb{F}, \zeta} \right)\) היא אלגברה פשוטה מרכזית מממד \(n^2\) מעל \(\mathbb{F}\).
בסיס של האלגברה הוא:
$$\{i^m j^\ell \mid m, \ell = 0, \dots, n-1\}$$

\end{proposition}
\begin{lemma}
ניתן לשכן את \(\left( \frac{a, b}{\mathbb{F}, \zeta} \right)\) בתוך \(M_n(\mathbb{F}[\sqrt[n]{a}])\), כאשר:
$$\mathbb{F}[\sqrt[n]{a}] = \mathbb{F}[x] / (x^n - a).$$
כאשר נשים לב ש-\(\mathbb{F}[\sqrt[n]{a}]\) אינו בהכרח שדה, אלא חוג.

\end{lemma}
\begin{proposition}
אלגברת הסימבול \(\left( \frac{a, b}{\mathbb{F}, \zeta} \right)\) היא חוג חילוק אם ורק אם \(x^n - a\) אי פריק מעל \(\mathbb{F}\), כלומר \(\mathbb{F}[\sqrt[n]{a}]\) הוא שדה.
בנוסף, נדרש ש-\(b, b^2, \dots, b^{n-1}\) אינם נורמות מההרחבה \(\mathbb{F}[\sqrt[n]{a}]\).

\end{proposition}
\begin{proposition}
אם \(E / F\) היא הרחבת גלואה עם:
$$\mathrm{Gal}(E / F) = \langle \sigma \rangle \cong \mathbb{Z} / n,$$
ניתן לבנות אלגברה פשוטה מרכזית ללא שימוש בשורש יחידה \(\zeta\) על ידי שימוש במבנה ההרחבה \(E / F\).

\end{proposition}
\begin{example}
  \begin{itemize}
    \item אם \(\mathbb{F} = \mathbb{Q}\), \(a = 2\), \(b = 3\), ו-\(\zeta = e^{2\pi i / 4}\), אז:
$$\left( \frac{2, 3}{\mathbb{Q}, \zeta} \right) = \mathbb{Q}\{i, j\} / \langle i^4 - 2, j^4 - 3, ji + ij \rangle$$
    \item אם \(\mathbb{F} = \mathbb{F}_p\), \(a = 1\), \(b = 1\), ו-\(\zeta = 1\), אז:
$$\left( \frac{1, 1}{\mathbb{F}_p, 1} \right) \cong M_2(\mathbb{F}_p)$$
  \end{itemize}
\end{example}
\section{מכפלה טנזורית}

\begin{definition}[טנזור של מודולים]
עבור \(M\) מודול ימני ו-\(N\) מודול שמאלי מעל \(R\), נגדיר את המכפלה הטנזורית:
$$M \otimes_{R} N := \mathbb{Z}^{\oplus (M \times N)} / Y,$$
כאשר \(Y\) הוא תת-קבוצה של \(\mathbb{Z}^{\oplus (M \times N)}\) המוגדרת על ידי יחסים מאוזנים:
$$Y = \left\langle \begin{array}{c}(mr, n) - (m, rn), \\(m+m', n) - (m, n) - (m', n), \\(m, n+n') - (m, n) - (m, n')
\end{array} \right\rangle$$
איבר כללי ב-\(M \otimes_{R} N\) נכתב כ:
$$\sum_{i=1}^{r} m_{i} \otimes n_{i} = \sum_{i=1}^{r} (m_{i}, n_{i}) + Y$$

\end{definition}
\begin{definition}[העתקה מאוזנת]
$$\mathrm{Bal}_{R}(M \times N, A) = \{ f : M \times N \to A \mid f \text{ מקיימת יחסים מאוזנים} \}$$
כאשר \(A\) היא חבורה אבלית.

\end{definition}
\begin{proposition}
יש התאמה טבעית לכל חבורה אבלית \(A\):
$$\mathrm{Bil}_{R}(M \times N, A) \leftrightarrow \mathrm{Hom}_{\text{Grp}}(M \otimes_{R} N, A)$$
על ידי כך שלכל \(f \in \mathrm{Bil}_{R}(M \times N, A)\) נגדיר:
$$f\left( \sum_{i=1}^{r} m_{i} \otimes n_{i} \right) = \sum_{i=1}^{r} f(m_{i}, n_{i})$$

\end{proposition}
\begin{proof}
אם \(f\) נגדיר:
$$f\left( \sum_{i=1}^{r} m_{i}\oplus n_{i} \right)=\sum_{i=1}^{r} f(m_{i},n_{i})$$
כאשר נטען כי מוגדרת היטב: \(f\) מוגדרת היטב על \(\mathbb{Z}^{\oplus (M\times N)}\) ומתאפסת על \(Y\). למשל:
$$\hat{f}((m+m',n)-(m,n)-(m',n))=f(m+m',n)-f(m,n)-f(m',n)=0$$
אם \(f \in \mathrm{Hom}(M \otimes_{R} N)\) נגדיר:
$$\hat{f}:M \times N\to A\qquad \hat{f}(m,n)=f(m\oplus n)$$
זו העתקה \(R\) בילינארית:
$$\hat{f}(m+m',n)-\hat{f}(m,n)-\hat{f}(m',n)=f((m+m')\otimes n)-f(m\otimes n)- f (m'\otimes n)=f((m+m')\otimes n)- m\otimes n-m'\otimes n =0$$

\end{proof}
\begin{example}
$$\mathbb{Z}^{m}\otimes_{\mathbb{Z}} \mathbb{Z}^{n}\cong \mathbb{Z}^{mn}$$$${\mathbb{Z}} / {n}\otimes _{\mathbb{Z}} \mathbb{Z} / m= \mathbb{Z} / (\mathrm{gcd}(n,m))$$$$\mathbb{Q} / \mathbb{Z} \otimes_{\mathbb{Z}} \mathbb{Q} / \mathbb{Z} = 0$$
כי:
$$\frac{a}{b}\otimes \frac{c}{d}=\frac{a}{bd}\cdot d \otimes \frac{c}{d}=\frac{a}{bd}\otimes c=\frac{a}{bd}\otimes 0 = 0$$

\end{example}
\section{מכפלות טנזוריות מעל חוגים קומוטטיבים}

\begin{proposition}
אם \(R\) קומוטטיבי אזי \(M \otimes_{R} N\) הוא \(R\) מודול על ידי:
$$r \cdot (m \otimes n) := r m \otimes n = m r \otimes n = m \otimes r n = m \otimes n r$$

\end{proposition}
\begin{definition}[בי-מודול]
אם \(R, S\) חוגים, \(R-S\) בימודול הוא חבורה אבלית \(M\) עם מבנה של \(R^{M}, ^{M}S\) שמקיימים:
$$(r m)s = r(m s)$$
אם \(R^{M}\) מודול ש-\(M\) פועל משמאל ואם \(S^{M}R\) אזי:
$$S^{M \otimes_{R} N}$$
על ידי:
$$s(m \otimes n) = s m \otimes n$$

\end{definition}
\begin{proposition}[מכפלה טנזורית של מרחבים ווקטורים]
אם \(v_{1}, \dots, v_{n}\) בסיס ו-\(w_{1}, \dots, w_{m}\) בסיס אז \(\{ v_{i} \otimes w_{j} \}_{i=1,\dots,n, j=1,\dots,m}\) בסיס ל-\(V \otimes_{F} W\).
בפרט:
$$\dim V \otimes W = \dim V \cdot \dim W$$

\end{proposition}
\begin{proposition}[תכונות של מכפלה טנזורית]
עבור מכפלה טנזורית מעל חוג קומוטטיבי:

  \begin{enumerate}
    \item פירוק לסכום ישר: 
$$(V \oplus V') \otimes_{F} W \cong (V \otimes_{F} W) \oplus (V' \otimes_{F} W)$$
המכפלה הטנזורית מתנהגת היטב עם סכום ישר של מרחבים וקטוריים. כלומר, אם \(V\) ו-\(V'\) הם מרחבים וקטוריים מעל \(F\), אז המכפלה הטנזורית של סכום ישר שלהם עם \(W\) מתפרקת לסכום ישר של המכפלות הטנזוריות.


    \item אסוצייטיביות: 
$$(V \otimes W) \otimes U \cong V \otimes (W \otimes U)$$
המכפלה הטנזורית היא אסוציאטיבית, כלומר אין חשיבות לסדר שבו מבצעים את המכפלה הטנזורית בין שלושה מרחבים וקטוריים.


    \item פעולות של השדה: 
$$F \otimes_{F} V \cong V, \quad \alpha \otimes v \mapsto \alpha v$$
המכפלה הטנזורית של השדה \(F\) עם מרחב וקטורי \(V\) איזומורפית ל-\(V\) עצמו. איבר מהצורה \(\alpha \otimes v\) מתפרש כפעולה של \(\alpha\) על \(v\).


    \item חילופיות: 
$$V \otimes_{F} W \cong W \otimes_{F} V$$
המכפלה הטנזורית היא חילופית, כלומר ניתן להחליף את סדר המרחבים הוקטוריים במכפלה הטנזורית.


    \item מכפלה טנזורית של הבסיסים: 
אם \(v_1, \dots, v_n\) הם בסיס של \(V\) ו-\(w_1, \dots, w_m\) הם בסיס של \(W\), אז:
$$\{v_i \otimes w_j \mid i=1,\dots,n, \; j=1,\dots,m\}$$
מהווה בסיס של \(V \otimes_{F} W\). בפרט:
$$\dim(V \otimes_{F} W) = \dim(V) \cdot \dim(W)$$


    \item מכפלה טנזורית עם אפס:\\
$$V \otimes_{F} 0 = 0, \quad 0 \otimes_{F} W = 0$$
אם אחד מהמרחבים הוקטוריים הוא אפס, אז המכפלה הטנזורית היא אפס.


    \item פונקטוריאליות: אם \(f: V \to V'\) ו-\(g: W \to W'\) הם העתקות לינאריות, אז קיימת העתקה לינארית: 
$$f \otimes g: V \otimes_{F} W \to V' \otimes_{F} W', \quad (v \otimes w) \mapsto f(v) \otimes g(w).$$
כלומר, המכפלה הטנזורית היא פונקטורית ביחס להעתקות לינאריות.


  \end{enumerate}
\end{proposition}
\section{הרחבות סקלאריות}

\begin{definition}[הרחבה סקלראית]
אם \(R \subseteq S\) הם חוגים ו-\(R^{M}\) הוא מודול, ניתן להגדיר הרחבת סקלארים מ-\(R\) ל-\(S\) על ידי:
$$S \otimes_{R} M$$
זהו \(S\) מודול עם הפעולה:
$$s \cdot (s' \otimes m) = (s s') \otimes m$$

\end{definition}
\begin{proposition}[תכונה אוניברסלית של הרחת סקלארים]
לכל \(S\) מודול \(N\), מתקיים:
$$\mathrm{Hom}_{S}(S \otimes_{R} M, N) \cong \mathrm{Hom}_{R}(M, N)$$

\end{proposition}
\begin{example}
אם \(\mathbb{E} / \mathbb{F}\) היא הרחבת שדות ו-\(V\) הוא מרחב וקטורי מעל \(\mathbb{F}\), אז:
$$\mathbb{E} \otimes_{\mathbb{F}} V \cong \bigoplus_{i=1}^{n} (\alpha_{i} \otimes V),$$
כאשר \(\alpha_{1}, \dots, \alpha_{n}\) הם בסיס של \(\mathbb{E}\) מעל \(\mathbb{F}\).

\end{example}
\begin{example}
אם \(V\) הוא מרחב וקטורי מעל \(\mathbb{R}\), אז:
$$\mathbb{C} \otimes_{\mathbb{R}} V \cong (1 \otimes V) \oplus (i \otimes V)$$
עם הפעולות:
$$a(1 \otimes v + i \otimes v') = 1 \otimes (a v) + i \otimes (a v')$$$$i(1 \otimes v + i \otimes v') = 1 \otimes (-v') + i \otimes v$$

\end{example}
\begin{definition}[הרחבת סקלרים לאלגברה ]
אם \(A\) היא \(\mathbb{F}\)-אלגברה, הרחבת סקלארים ל-\(\mathbb{E}\) מוגדרת על ידי:
$$\mathbb{E} \otimes_{\mathbb{F}} A$$
עם הפעולה:
$$(\alpha \otimes a) \cdot (\alpha' \otimes a') = (\alpha \alpha') \otimes (a a')$$

\end{definition}
\begin{example}
$$\mathbb{E} \otimes_{\mathbb{F}} M_{n}(\mathbb{F}) \cong M_{n}(\mathbb{E})$$$$\mathbb{E} \otimes_{\mathbb{F}} \mathbb{F}[x] \cong \mathbb{E}[x]$$$$\mathbb{E} \otimes_{\mathbb{F}} \left( \frac{a, b}{\mathbb{F}} \right) \cong \left( \frac{a, b}{\mathbb{E}} \right)$$

\end{example}
\begin{definition}[קבועי מבנה]
אם \(A\) היא \(\mathbb{F}\)-אלגברה עם בסיס \(b_{1}, \dots, b_{n}\) קבועי המבנה הם קבועים \(\gamma_{k}^{i,j}\) כך ש:
$$b_{i} \cdot b_{j} = \sum_{k=1}^{n} \gamma_{k}^{i,j} b_{k}$$

\end{definition}
\begin{proposition}
המכפלה הטנזורית \(\mathbb{E} \otimes_{\mathbb{F}} A\) היא \(\mathbb{E}\)-אלגברה עם אותם קבועי מבנה ביחס לבסיס \(1 \otimes b_{1}, \dots, 1 \otimes b_{n}\).

\end{proposition}
\begin{proposition}
$$\dim_{\mathbb{E}} (\mathbb{E} \otimes_{\mathbb{F}} V) = \frac{\dim_{\mathbb{F}} (\mathbb{E} \otimes V)}{[\mathbb{E} : \mathbb{F}]} = \dim_{\mathbb{F}} V$$

\end{proposition}
\section{מכפלה טנזורית מעל שדה}

\begin{definition}[מכפלה טנזורית]
יהיו \(V, W\) מרחבים וקטוריים מעל שדה \(F\). המכפלה הטנזורית \(V \otimes_{F} W\) מוגדרת כמרחב הוקטורי מעל \(F\) הנוצר על ידי כל הזוגות \((v, w)\) כאשר \(v \in V\) ו-\(w \in W\), עם יחסים מאוזנים:
$$V \otimes_{F} W = \frac{\langle v \otimes w \mid v \in V, w \in W \rangle}{Y}$$
כאשר \(Y\) הוא תת-קבוצה של היחסים המאוזנים.

\end{definition}
\begin{proposition}[בסיס של המכפלה הטנזורית]
אם \(v_{1}, \dots, v_{n}\) הם בסיס של \(V\) ו-\(w_{1}, \dots, w_{m}\) הם בסיס של \(W\), אזי:
$$\{v_{i} \otimes w_{j} \mid i=1,\dots,n, \; j=1,\dots,m\}$$
מהווה בסיס של \(V \otimes_{F} W\). בפרט:
$$\dim(V \otimes_{F} W) = \dim(V) \cdot \dim(W).$$

\end{proposition}
\begin{definition}[כפל של טרנספורמציות לינאריות]
יהיו \(T: V \to V'\) ו-\(S: W \to W'\) טרנספורמציות לינאריות. ניתן להגדיר את המכפלה הטנזורית שלהן:
$$(T \otimes S)(v \otimes w) = T(v) \otimes S(w)$$
ביחס לבסיסים הרלוונטיים מתקיים:
$$[T \otimes S] = [T] \otimes [S]$$

\end{definition}
\begin{definition}[אורך של טנזור]
עבור \(\alpha \in V \otimes W\), נגדיר את האורך של הטנזור:
$$\mathrm{len}(\alpha) = \min \left\{ \ell \mid \alpha = \sum_{i=1}^{\ell} v_{i} \otimes w_{i} \right\}$$

\end{definition}
\begin{corollary}
לכל \(\alpha\) מתקיים:
$$\mathrm{len}(\alpha) \leq \min(\dim V, \dim W)$$

\end{corollary}
\section{אלגברה טנזורית}

\begin{definition}[אלגברה טנזורית]
אם \(A, B\) הן אלגבראות מעל \(F\), אזי \(A \otimes_{F} B\) היא \(F\)-אלגברה עם פעולות:
- כפל:
$$(a \otimes b)(a' \otimes b') = (aa') \otimes (bb').$$
- יחידה: $$1_{A \otimes_{F} B} = 1 \otimes 1.$$
- פעולה של סקלר:
$$\gamma(a \otimes b) = (\gamma a) \otimes b = a \otimes (\gamma b)$$

\end{definition}
\begin{proposition}
השיכון של אלגבראות מתקיים:
$$A \hookrightarrow A \otimes_{F} B, \quad a \mapsto a \otimes 1$$$$B \hookrightarrow A \otimes_{F} B, \quad b \mapsto 1 \otimes b$$

\end{proposition}
\begin{definition}[הרחבת סקלרים של אלגברה]
עבור \(F\) אלגברה \(A\) ו-\(E / F\) הרחבת שדות, נגדיר את הרחבת הסקלרים:
$$E \otimes_{F} A.$$
זהו אלגברה מממד \(\dim_{F} A \cdot [E : F]\). הפעולה מוגדרת על ידי:
$$e(e' \otimes a) = e e' \otimes a.$$
אם \(a_{1}, \dots, a_{n}\) הם בסיס מעל \(F\) ל-\(A\), אז \(1 \otimes a_{1}, \dots, 1 \otimes a_{n}\) הם בסיס מעל \(E\) ל-\(A\uparrow^{E}\).

\end{definition}
\begin{symbolize}
הסימון \(A\uparrow^{E}\) מתאר את הרחבת הסקלרים של האלגברה \(A\) מ-\(F\) ל-\(E\). כלומר:
$$A\uparrow^{E} := E \otimes_{F} A$$

\end{symbolize}
\begin{proposition}
אם \(A\) היא \(F-\mathrm{CSA}\), אזי \(A\uparrow^{E}\) היא \(E-\mathrm{CSA}\).

\end{proposition}
\begin{corollary}
אם \(A\uparrow^{\overline{F}}\) היא \(\overline{F}-\mathrm{CSA}\), אז מתקיים:
$$M_{n}(\overline{F}) = A\uparrow^{\overline{F}}$$
ומתקיים:
$$\dim_{F} A = \dim_{\overline{F}} (A\uparrow^{\overline{F}}) = n^{2}$$
לכן המימד של CSA הוא ריבוע.

\end{corollary}
\begin{corollary}
$$Z(A\uparrow^{E}) = Z(E) \otimes_{F} Z(A) = E \otimes_{F} Z(A) = E$$
אם"ם \(Z(A) = F\) משיקולי מימד.

\end{corollary}
\begin{corollary}
$$A\uparrow^{\overline{F}} = M_{\deg(A)}(\overline{F})$$
למעשה:
$$A\uparrow^{\overline{F}} = \overline{F} \otimes_{F} A \xrightarrow{\cong} M_{n}(\overline{F}).$$$$a_{1}, \dots, a_{n} \mapsto A_{1}, \dots, A_{n},$$

\end{corollary}
כאשר \(E = F((A_{i})_{j,k})\) כבר יתן \(A\uparrow^{E} = M_{n}(E)\).

\section{מכפלה טנזורת של אלגבראות פשוטות מרכזיות}

\begin{proposition}
אם \(A, B\) הן \(F-\mathrm{CSA}\), אזי \(A \otimes_{F} B\) היא גם \(F-\mathrm{CSA}\).

\end{proposition}
\begin{lemma}
אם \(A, B\) הם \(F\) אלגבראות ויהיו \(A' \subseteq A\) ו-\(B' \subseteq B\) תת-אלגבראות, אזי:
$$C_{A \otimes B}(A' \otimes B') = C_{A}(A') \otimes C_{B}(B')$$

\end{lemma}
\begin{corollary}
$$Z(A \otimes_{F} B) \!=\! C_{A \otimes_{F} B}(A \otimes_{F} B) \!=\! C_{A}(A) \otimes C_{B}(B) \!=\! Z(A) \otimes Z(B)$$
לכן \(Z(A \otimes_{F} B) = F\) אם"ם \(Z(A) = Z(B) = F\) מכפליות המימד.

\end{corollary}
\begin{lemma}
אם \(B\) אלגברה פשוטה (סוף-מימדית) ו-\(A\) אלגברה CSA מעל \(F\), אזי \(A \otimes_{F} B\) פשוטה.

\end{lemma}
\begin{remark}
הכיוון השני לא בהכרח נכון. טנזור של פשוטות לא בהכרח פשוטה. למשל:
$$\mathbb{C} \otimes_{\mathbb{R}} \mathbb{C} = \mathbb{C} \times \mathbb{C}$$

\end{remark}
\section{חבורת בראוור}

\begin{definition}[שקילות בראוור]
יהי \(F\) שדה. נגדיר \(\sim\) על כל ה-\(F\) CSA בצורה הבאה:
$$A\sim  B$$
אם
$$M_{n}(A)\sim  M_{m}(B)$$
ל-\(m,n\) כלשהם.

\end{definition}
\begin{example}
לכל אלגברת חילוק מרכזית \(D\) נקבל
$$D\sim  M_{2}(D)\sim  M_{3}(D)\sim \dots$$
זה אומר שמחלקות השקילות של \(\sim\) הם טיפוסי האיזומורפיזם של אלגבראות חילוק מרכזיות.

\end{example}
\begin{definition}[חבורת בראוור]
חברת בראוור של \(F\) היא מחלקות השקילות של CSA מעל \(F\) עם הפעולה \(\otimes_{F}\). מסומן \(\text{Br}(F)\).

\end{definition}
\begin{proposition}
אם \(A\sim A'\) אז \(A\otimes B\sim A'\otimes B\).

\end{proposition}
\begin{corollary}
$$M_{n}(A\otimes B)\cong  M_{n}(A)\otimes B\cong  M_{m}(A')\otimes B\cong M_{m}(A'\otimes B)$$

\end{corollary}
אסוצייטיביות נובע מאסוצייטיביות של \(\otimes\). כדי למצוא הופכי ראשית נדרש למצוא יחידה:
$$F\otimes_{F}A\cong  A$$
ולכן \(F\) הוא איבר היחידה. כעת ההופכי יהיה:
$$A\otimes _{F}A^{\text{op}}\cong  \mathrm{End} _{F}(A)\cong M_{\dim A}(F)\sim  F$$

\begin{example}
$$\text{Br}(\mathbb{R})=\{ \mathbb{R},\mathbb{H}  \}\cong  \mathbb{Z} / 2$$$$\mathrm{Br}(\mathbb{F} _{p})=1=\mathrm{Br}(\mathbb{C})$$$$\mathrm{Br}(\mathbb{Q} _{p})\cong \mathbb{Q} / \mathbb{Z}$$

\end{example}
\begin{example}
לגבי \(\mathrm{Br}(\mathbb{Q})\) יש את משפט אלברט הסה-נתר-בראוור אשר אם כי אם \(E / F\) ו-\(A\) היא CSA מעל \(F\) אז:
$$E\otimes _{F}A=A\uparrow^{E}$$
היא CSA מעל E
$$\mathrm{Br}(F)\to \mathrm{Br}(E)\qquad A\mapsto E\otimes_{F}A$$$$\mathrm{Br}(\mathbb{Q} )\to \prod_{p\leq  \infty} \mathrm{Br}(\mathbb{Q} _{p})$$$$1\xrightarrow{\text{ABHS}}\mathrm{Br}(\mathbb{Q} )\hookrightarrow \oplus_{p\leq  \infty}\mathrm{ Br}(\mathbb{Q} _{p})\cong (\bigoplus_{p<\infty} \mathbb{Q} / \mathbb{Z}) \oplus \frac{\left[ 0,\frac{1}{2} \right]}{\mathbb{Z}}\xrightarrow{(a_{2},a_{3},a_{5},\dots, a_{\infty})} \mathbb{Q} / \mathbb{Z} \to 0$$
ממשפט אלברט ברוואר הסה נתר
$$A\mapsto(\mathbb{Q}_{2}\otimes A,\mathbb{Q} _{3}\otimes A, \dots,\mathbb{R}\otimes A)$$

\end{example}
\section{משפט סקולם נתר}

\begin{definition}[אוטומורפיזם פנימי]
יהי \(A\) אלגברה פשוטה מרכזית. אם \(a \in A^{\times}\) הוא איבר הפיך של \(A\), ניתן להגדיר אוטומורפיזם על ידי הצמדה:
$$x\mapsto ax a ^{-1}$$
כאשר אוטומורפיזם המתקבל מהצורה הזו נקרא אוטומורפיזם פנימי.

\end{definition}
\begin{theorem}[סקולם נתר]
אם \(A\) הוא CSA מעל \(F\) ו-\(B\) הוא \(F\) אלגברה פשוטה(סוף מימדית) אזי כל שני הומומורפיזמים \(B\to A\) צמודים:
$$\forall f_{1},f_{2} \in \mathrm{Hom}_{F-\text{alg}}(B,A)$$$$\exists a \in A^{\times }\quad af_{1}(b)a^{-1} =f_{2}(b)\quad \forall b \in B$$

\end{theorem}
\begin{corollary}
כל CSA ממימד 4 מעל שדה ממציין שונה מ-2 הוא קווטרניונים.

\end{corollary}
\begin{proof}
תהי A כזו. אם \(A\cong M_{2}(F)\) סיימנו כי אנחנו יודעים שמטרציות הם קווטרניונים. אחרת \(A\) חילוק(מארתין ודרברן כי \(A\cong M_{d}(1)\)). ניקח \(s \in A \setminus F\) ואז \(F[s] / F\) הרחבת שדות ריבועית כי:
$$[A:F]=[A:F[s]][F[s]:F]$$
לכן \(F[s]=F[t]\) עם \(\mathrm{Gal}(F[i] / F)\) עם:
$$\mathrm{Gal}(F[i] / F)= \{ \mathrm{Id}, \tau: t\mapsto-t \}$$
עם איזשהו t(קיים כזה). יש:
$$\text{id}=f_{1}:E\to E$$
וגם יש:
$$f_{2}:E\to A\quad \alpha \mapsto \tau(\alpha)$$
וכעת עבור \(x,y \in F\):
$$f_{2}(x+ty)=x-ty$$
מסקולם נתר קיים \(a \in A^{\times}\) כך ש-
$$a t a ^{-1} = a f_{1}(t)a ^{-1}  = f_{2}(t)=-t\implies at = -ta$$
כעת \(t^{2}\in F\) מבחירת \(t\) ו-\(a^{2} \in F\) כי \(a^{2}\) מתחלף עם \(a\).  ומתקיים:
$$a^{2}t=-ata=ta^{2}$$
ו-\(E[a]=A\).

\end{proof}
נוכחה כעת את משפט סקולר נטר.

\begin{proof}
ניתן ל-A שני מבנים של \(A\otimes_{F} B^{\text{op}}\) מודול = \(A,B\mathrm{\text{op}}\) - \(F\) בימודול. עבור \(i=1,2\) נקבל:
$$(a\otimes b)(x)=axf_{i}(b)$$

\end{proof}
\begin{reminder}
טיפוס של מודול נקבע מהמימד שלו(יש מודול פשוט יחיד וכל מודול נוצר סופית הוא חזקה שלו).
מכפלה טנזורית של אלגברה פשוטה מרכזית עם אלגברה פשוטה היא פשוטה(ולא בהכרח מרכזית).

\end{reminder}
\begin{corollary}
אם \(A\) היא CSA מעל \(F\) מימד \(n^{2}\) וקיים \(F\subseteq E \subseteq A\) עם:
$$  \mathbb{Z} / n\cong \mathrm{Gal}(E / F)=\langle \sigma \rangle $$
אזי קיים \(a \in F^{\times}\) כך ש-\(A=(E,\sigma,a)\) (כאשר נזכור כי זהו האלגברה הציקלית המוגדר על ידי \(\langle x^{n}-a,xe=\sigma(e)x \rangle\)).

\end{corollary}
\begin{proof}
מסולקם נתר קיים \(x \in A^{\times}\) כך ש-\(xex ^{-1} = \sigma(e)\) לכל \(e\) כלומר \(xe=\sigma(e)x\)

\end{proof}
\begin{proposition}
$$\{ 1,x,x^{2},\dots,x^{n-1} \}$$
בתל מעל \(E\). 

\end{proposition}
נובע מכך ש:
$$A=\bigoplus_{i=0}^{n-1}Ex^{i}$$
ולכן גם \(x^{n}\in Z(A)\) כי \(A=E[x]\). 
$$x^{n}\cdot e=\sigma^{n}(e)x^{n}=ex^{n}$$

\section{משפט המרכז}

\begin{theorem}[המרכז]
תהי \(A\) אלגברה פשוטה מרכזית  מעל \(\mathbb{F}\). לכל תת אלגברה פשוטה \(B\subseteq A\) המרכז \(B'=C_{A}(B)\) מקיים:

  \begin{enumerate}
    \item מתקיים \(B'\) פשוטה. 


    \item המרכזים מקיימים \(Z(B')=Z(B)\). 


    \item המימדים מקיימים: 
$$\dim  B\cdot \dim  B'=\dim  A(\implies \dim  B \mid  \dim  A)$$


    \item מתקיים \(B'' = B\). 


    \item אם \(B\) היא אלגברה פשוטה מרכזית \(B\otimes_{F}B' \cong A\). בצורה כללית יותר: 
$$B'\otimes_{Z(B)}B'\cong  Z(B)$$


    \item אם \(E\subseteq A\) שדה \(E\otimes_{F}A\cong M_{\text{dim}}(E')\)


    \item ל-\(B\subseteq A\) פשוטה: 
$$Z(B)\otimes_{F}A\cong M_{\text{dim}_{Z(B)}}(B\otimes_{Z(B)} B')$$


  \end{enumerate}
\end{theorem}
\begin{enumerate}
  \item נסתכל על \(A\) כ-\(B\)-\(A\) בימודול (כלומר, כמודול מעל \(B \otimes_{F} A^{\text{op}}\)). נטען כי: 
$$\mathrm{End}_{B-A}(A) \cong B'$$
לכל \(b \in B'\), נגדיר \(T_b(x) = bx\). מתקיים:
$$T_b(\hat{b} \otimes a)(x) = (\hat{b} \otimes a) T_b(x)$$
כלומר, \(T_b\) הוא הומומורפיזם של \(B\)-\(A\) בימודולים. כל הומומורפיזם כזה נקבע על ידי \(T(1)\), ומתקיים \(T(a) = T(1)a\) לכל \(a \in A\). בנוסף, \(T(1) \in B'\) כי \(b \cdot T(1) = T(b) = T(1) b\) לכל \(b \in B\). לכן:
$$    \mathrm{End}_{B-A}(A) \cong B'    $$


  \item כיוון ש-\(A\) אלגברה פשוטה מרכזית ו-\(B\) פשוטה, גם \(B \otimes_{F} A^{\text{op}}\) פשוטה. לכן \(B'\) אלגברה פשוטה. 


  \item המרכזים: 
$$    Z(B') = Z(B \otimes_{F} A^{\text{op}}) = Z(B) \otimes Z(A^{\text{op}})$$
ובפרט, \(Z(B') = Z(B)\).


  \item ממדים: 
$$    \dim B' = \frac{\dim A}{\dim B}    $$
ולכן \(\dim B \cdot \dim B' = \dim A\).


  \item כעת, \(B \subseteq B''\) (הרכז של הרכז). מהסעיף הקודם: 
$$    \dim B'' = \frac{\dim A}{\dim B'} = \dim B    $$
ולכן \(B'' = B\).


  \item אם \(B\) אלגברה פשוטה מרכזית (CSA), גם \(B'\) ו-\(B \otimes_{F} B'\) פשוטות. ההכללות \(B, B' \hookrightarrow A\) מתחלפות, ולכן יש שיכון \(B \otimes_{F} B' \to A\). זה שיכון כי \(B \otimes B'\) פשוטה, ועל פי חישוב הממדים. במקרה הכללי, נסמן \(E = Z(B)\). אז \(B, B'\) אלגברות פשוטות מרכזיות מעל \(E\), ונקבל: 
$$    B \otimes_{E} B' \cong E'    $$
כנדרש.


  \item עבור שדה \(E \subseteq A\), מתקיים \(E \otimes_{F} A \cong M_{\dim E}(E')\). \(A\) הוא \(A\)-\(E\) בימודול, והומומורפיזם המבנה הוא: 
$$    \rho: A \otimes_{F} E^{\text{op}} \to \mathrm{End}_F(A)    $$
מתקיים \(\mathrm{Im}(\rho) \subseteq \mathrm{End}_{E'^{\text{op}}}(A)\), כאשר \(A\) מודול ימני מעל \(E'\). \(\rho\) הוא איזומורפיזם כי \(A \otimes_{F} E\) פשוטה, ולכן \(\rho\) חד-חד ערכית ועל. חישוב ממדים:
$$
 \dim \mathrm{End}_{E'^{\text{op}}}(A) = \frac{(\dim A)^2}{\dim E'} = \dim A \cdot \dim E = \dim (A \otimes_{F} E) $$
לכן:
$$ \mathrm{End}_{E'^{\text{op}}}(A) \cong M_{\dim E}(E')
 $$


  \item סעיף 7 נובע מיידית מסעיפים 5 ו-6. 


\end{enumerate}
\begin{proposition}
אם \(D\) אלגברת חילוק מרכזית ו-\(E\leq D\) תת שדה. הבאים שקולים:

  \begin{enumerate}
    \item מתקיים \(\dim E = \deg D\). 


    \item מתקיים \(E=E'\)


    \item השדה \(E\) מפצל את \(D\). 


    \item השדה \(E\) מקסימלי מבין תתי השות של \(D\)\(\leftarrow\) בהכרח קיים! 


  \end{enumerate}
\end{proposition}
כעת נוכיח את המשפט העצוב של וודרברן כמו שצריך

\begin{corollary}[המשפט העצוב]
אלגברות חילוק סופיות הן שדות.

\end{corollary}
\begin{proof}
יהי \(D\) אלגברת חילוק מרכזית מעל \(\mathbb{F}=\mathbb{F}_{q}\). לכל \(d \in D\) יש שדה מקסימלי \(\mathbb{F}[d]\subseteq E_{d}\subseteq D\). נקבע \(\alpha\) אחד. לכל \(\beta \in D\) מתקיים:
$$\lvert E_{\alpha} \rvert =\lvert E_{\beta} \rvert=q^{\deg D}$$
ולכן הם איזומורפיים כשדות, וגם כ-\(\mathbb{F}\) אלגבראות: אם \(\varphi:E_{\alpha}\to E_{\beta}\) אוטומורפיזם של שדות, אז \(\varphi(\mathbb{F})=\mathbb{F}\) כי \(\mathbb{F} / \mathbb{F}_{p}\) נורמליות. האם \(\varphi|_{\mathbb{F}}=\mathrm{id}\)? לא בהכרח. אבל כי אוטומורפיזם של \(\mathbb{F}\) מתרחב לאוטומורפיזם של \(E_{\beta}\) (כיוון שכל הרחבה של שדות סופית היא גלואה) ואז אפשר להרכיב את ההרחבה ל-\(E_{\beta}\) של \((\varphi|_{\mathbb{F}})^{-1}\) ולכן איזו \(E_{\alpha}\to E_{\beta}\) שטריוויאלי על \(\mathbb{F}\). סה"כ:
$$E_{\alpha}\xrightarrow[\varphi]{\text{id}}D$$
שיכונים של \(\mathbb{F}\) אלגבראות. מסקולם נתר הם צמודים. ובפרט:
$$\exists d \in D^{\times}\qquad dE_{\alpha}d ^{-1}  = E_{\beta}$$
לכן:
$$\bigcup_{d \in D^{\times}}d E_{\alpha }d ^{-1} = D^{\times}\implies E_{\alpha}^{\times}= D^{\times}$$
כאשר הגרירה נובעת מכך שחבורה סופית היא לא איחוד של הצמדות של תת חבורות ממ"ש(לא נכון עבור חבורות אינסופיות, למשל:
$$\text{GL}_{2}(\mathbb{C})=\bigcup_{A \in \text{GL}_{2}(\mathbb{C})}A\begin{pmatrix}*  &  * \\0 & *
\end{pmatrix}$$
)וזה ומר ש-\(D\) קומוטטיבי ולכן \(D=\mathbb{F}\).

\end{proof}
\chapter{חבורת בראוור}

\section{Untitled}
\end{document}