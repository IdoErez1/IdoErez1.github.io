\documentclass{tstextbook}

\usepackage{amsmath}
\usepackage{amssymb}
\usepackage{graphicx}
\usepackage{hyperref}
\usepackage{xcolor}

\begin{document}

\title{Example Document}
\author{HTML2LaTeX Converter}
\maketitle

\section{הגדרות בסיסיות}

\subsection{פונקציות מצב ומסלול}

\begin{definition}[תכונה אינטנסיבית]
תכונה שאינה משתנה אם משכפלים את המערכת. דוגמאות לגדלים אינטנסיביום יהיו טמפרטורה וצפיפות

\end{definition}
\begin{definition}[תכונה אקסטנסיבי]
תכונה של המערכת אשר פרופרציונאלי לשיכפול המערכת. כלומר אם נשכפל את מערכת אז נצפה כי גודל אקסטנסיבי יגדל פי 2. דוגמאות זה מסה, נפח ואנטרופיה.

\end{definition}
\begin{proposition}
גודל שהוא יחס של גדלים אקסטניביים הוא גודל אינטנסיבי.

\end{proposition}
\begin{definition}[פונקציות מצב]
תכונה אשר תלוייה אך ורק במצב של המערכת ולא איך המערכת הגיעה עליה. דוגמאות לפונקציות מצב הם נפח, אנרגיה פנימית, לחץ ואנטרופיה

\end{definition}
\begin{definition}[פונקציית מסלול]
תכונה אשר תלוייה באיך התהליך התבצעה, ולא רק על ידי המצב שלה ברגע נתון. דוגמאות לפונקציות מסלול הם עבודה וחום.

\end{definition}
התיאור המתמטי של תכונות אלו נעשה באמצעות דיפרנציאלים.

\subsection{דיפרנציאלים}

\begin{definition}[דיפרנציאל]
אופרטור \(d\) המתאר שינוי קטן בגודל.

\end{definition}
\begin{symbolize}
נכתוב \((f)_{y}\) כדי להדגיש שאנחנו משאירים את \(y\) קבוע.

\end{symbolize}
\begin{proposition}
עבור פונקציה \(f=f(x,y)\) ניתן לכתוב את הדיפרנציאל של \(f\) ע"י:
$$d\!f=\left({\frac{\partial f}{\partial x}}\right)_{y}d x+\left({\frac{\partial f}{\partial y}}\right)_{x}d y$$

\end{proposition}
\begin{definition}[דיפרנציאל מדוייק]
ביטוי מהצורה:
$$h(x,y)=M(x,y)dx+N(x,y)dy$$
נקרא דיפרציאל מדוייק אם קיים פונקציה \(f=f(x,y)\) כך ש- \(df=h(x,y)\)

\end{definition}
\begin{proposition}
עבור ביטוי מהצורה ביטוי מהצורה \(h(x,y)=M(x,y)dx+N(x,y)dy\) התנאים הבאים שקולים:

  \begin{enumerate}
    \item הביטוי \(h\) דיפרנציאל מדוייק 


    \item מתקיים בתחום פשוט קשר: 
$$\frac{\partial M}{\partial y} =\frac{\partial N}{\partial x} $$


    \item האינטגרל המסלולי \(\int_{\gamma(a)}^{\gamma(b)}h(x,y)\) יהיה תלוי רק ב-\(a,b\) לכל מסלול \(\gamma\). 


    \item האינטגרל על כל מסלול סגור הוא 0. 


    \item הנגזרות מתחלפות. כלומר: 
$$\frac{\partial^2 f}{\partial x\partial y}=\frac{\partial ^2f}{\partial y\partial x}  $$


  \end{enumerate}
\end{proposition}
\begin{corollary}
פונקציית מצב מתוארת ע"י דיפרנציאל מדוייק

\end{corollary}
\begin{definition}[דיפרנציאל לא מדוייק]
גודל מהצורה \(h(x,y)=M(x,y)dx+N(x,y)dy\) אשר אינו דיפרנציאל מדוייק נקרא דיפרנציאל לא מדוייק.

\end{definition}
\begin{symbolize}
דיפרנציאל לא מדוייק מסומן ב-\(\;\bar{}\mkern-7.5mu df\). כלומר מתקיים \(\;\bar{}\mkern-7.5mu df=h(x,y)\) .

\end{symbolize}
\begin{corollary}
פונקציה המתוארת ע"י דיפרנציאל לא מדוייק בהכרח תלוייה במסלול. לכן פונקציית מסולול מתוארת ע"י דיפרנציאל לא מדוייק.

\end{corollary}
\begin{proposition}[נוסחת המכפלה המשולשת]
עבור פונקציה \(f(x,y,z)=0\) מתקיים:
$$-1=\left({\frac{\partial z}{\partial y}}\right)_{x}\left({\frac{\partial y}{\partial x}}\right)_{z}\left({\frac{\partial x}{\partial z}}\right)_{y}$$

\end{proposition}
\subsection{משאוות מצב}

\begin{definition}[מול]
יחידת מידה חסרת מימדים. שווה למספר אבוגדרו:
$$1\;{\mathrm{mol}}=N_{A}=6.022\times10^{23}$$
זהו מספר האטומים ב-12 גרם של פחמן 12.

\end{definition}
\begin{definition}[קלווין]
יחידה למדידה טמפרטורה. מסומנת \(^\circ K\).

\end{definition}
\begin{definition}[משוואת מצב]
משוואה המקשרת בין לחץ, נפח, טמפרטורה וכמות החלקיקים. כלומר משוואה מהצורה \(f\left(p,V,T,N \right)=0\)

\end{definition}
\begin{theorem}[משוואת גז אידיאלי]
$$pV = Nk_{B}T$$
כאשר \(k\) זה קבוע בולצמן, \(N\) מתאר את מספר החלקיקים.

\end{theorem}
\begin{remark}
משוואה זו נכונה לשיווי משקל. לא ניתן אפילו להגדיר בצורה טובה את המשתנים כשאנחנו לא בשיווי משקל.

\end{remark}
בנוסף, אנחנו מניחים כי אין אינטרקציה בין החלקיקים(או לפחות ניתן להזניח אותה). לכן אם נערבב חלקיקים משני סוגים שונים, השילוב יקיים את משוואת הגז האידיאלי רק אם אין אינטרקציה בינהם, פרט להתנגשויות קשיחות בין החלקיקים. הנחה חשובה זה שלכל חלקיק יש אנרגיה קינטית, אבל אין אנרגיה פוטנציאלית.

ניתן להשתמש בגז אידיאלי על גופים רבים - לא מוגבל לגזים. 

\begin{example}
נתאר את פרופיל הלחץ(כתלות בגובה) של אטמוספרה בטמפרטורה קבועה \(T\), ונניח שהאוויר באטמוספרה ניתן לתיאור ע"י גז אידיאלי.
נדמיין עמודת אוייר בעל שטח חתך \(A\) וגובה \(dz\). מכיוון שהאטמוספרה בשיווי משקל נקבל משוואת כוחות:
$$0=A\left[p\left(z\right)-p\left(z+d z\right)\right]-n\mu g$$
כעת נשתמש במשוואת הגז האידיאלי בשביל \(n\) ובזה ש-\(V=Adz\) ונקבל:
$$p^{\prime}\left(z\right)=\frac{p\left(z+d z\right)-p\left(z\right)}{d z}=-\frac{\mu g}{R T}p\implies p=p_{0}e^{-\mu gz/RT}$$

\end{example}
\begin{definition}[גז וואן דר וואלס]
מודל אלטרנטיבי המתאר גזים מסויימים. משוואת המצב תהיה:
$$\left(p+a\frac{N^{2}}{V^{2}}\right)\left(V-N b\right)\,=\,N k_{B}T$$
כאשר \(T\) זה טמפרטורה(בקלווין), \(k_{B}\) זה קבוע בולצמן, \(N\) זה מספר החלקיקים, \(V\) הנפח, \(p\) הלחץ ו-\(N\) מספר החלקיקים.

\end{definition}
משוואה זו מניחה כי יש התנגשות אלסטית מושלמת בין החלקיקים.

\subsubsection{מוצקים ונוזלים}

נוזלים ומוצקים בניגוד לגזים לא ממלאים את הכלי שבו הם נמצאים, וגדלים באופן פרופורציוני לשינוי בטמפרטורה.

\begin{definition}[מקדם התפשוטות תרמי לינארי]
לנוזל או מוצק יש מקדם התפשטות אורכי \(\alpha\) המוגדר:
$$\alpha\!=\!{\frac{1}{L}}\!\left({\frac{d L}{d T}}\right)$$

\end{definition}
\begin{definition}[מקדם התפשטות תרמי נפחי]
בנוזל או למוצק יש מקדם התפשטות נפחי \(\beta\) המוגדר:
$$\beta\!=\!{\frac{1}{V}}\!\left({\frac{d V}{d T}}\right)$$

\end{definition}
\begin{proposition}
עבור מוצקים, המקדם ההתפשטות הלינארי והנפחי קשורים אחד לשני כך ש:
$${\beta}\approx 3\alpha$$

\end{proposition}
\subsection{חוק האפס של תרמודינמיקה}

\begin{definition}[מערכת תרמודינמית]
איזור שמכיל אנרגיה או חומר 

\end{definition}
\begin{definition}[סביבה]
איזור אשר מבצע אינטרקציה עם המערכת התרמודינמית

\includegraphics[width=0.8\textwidth]{diagrams/svg_1.svg}
\end{definition}
\begin{definition}[מערכת פתוחה]
מערכת תרמודינמית אשר מחליפה של חלקיקים עם הסביבה.

\end{definition}
\begin{definition}[מערכת סגורה]
מערכת תרמודינמית אשר לא מחליפה חלקיקים עם הסביבה.

\end{definition}
\begin{definition}[מערכת מבודדת]
מערכת תרמודינמית אשר לא מושפעת כלל מהסביבה.

\end{definition}
\begin{definition}[גז]
אוסף חלקיקים אשר מתאימים את עצמם לצורה של כלי שבה הם נמצאים.

\end{definition}
\begin{definition}[שיווי משקל]
בצורה מיקרסוקופית החלקיקים עדיין נעים. אבל בצורה המאקרוסקופי נקבל שהרבה גדלים נשמרים בזמן. גודל שנשמר בזמן למערכת נקרא בשיווי משקל.

\end{definition}
\begin{definition}[לחץ]
הכוח הממוצע שפועל ליחידת שטח. מסומן ע"י \(P\) ומתאור ע"י:
$$P= \frac{F}{A}$$
ובעל יחידות של \(\frac{N}{m^2}=\frac{kg}{m\cdot s^2}\equiv Pa\)

\end{definition}
\begin{definition}[צפיפות]
המסה ליחידת נפח. פרופורציונאלי לכמות החלקיקים.

\end{definition}
\begin{definition}[מגע תרמי]
כאשר יש מעבר של חלקיקים בין המערכות.

\end{definition}
\begin{definition}[חום]
האנרגיה המועברת בין המערכות במגע תרמי.

\end{definition}
\begin{definition}[טמפרטורה - זמנית]
גודל המאפיין את המערכת ומשתווה בין שתי מערכות המגיעות לשיווי משקל תרמי.

\end{definition}
\begin{remark}
למעשה חום זה האנרגיה שמתקבל מהבדל טמפרטורה.

\end{remark}
\begin{theorem}[חוק האפס של תרמודינמיקה]
אם \(B\leftrightarrow C\) באותו טמפרטורה ו- \(B\leftrightarrow A\) באותה טמפרטורה אזי \(A\leftrightarrow C\) באותה טמפרטורה.

\end{theorem}
\begin{remark}
ממשפט זה אנו מבינים כי למעשה התפקיד של טמפרטורה כרגע זה להכריע אם מערכת נמצאת בשיווי משקל תרמי, וכמה.

\end{remark}
חוק זה למעשה מגדיר את המושג של שיווי משקל תרמי. העובדה אבל שמערכת נמצאת בשיווי משקל תרמי לא אומר שהמערכת נמצאת בשיווי משקל. נדרש גם לשם כך:

\begin{enumerate}
  \item שיווי משקל מכאני - אין כוחות לא מאוזנים. 


  \item שיווי משקל כימי - אין ראקציות כימיות. 


  \item שיווי משקל דיפוזי - אין זרימה של חומר. 


\end{enumerate}
\begin{proposition}[דרכים לבצע אינטרקציה]
יש 3 דרכים שונות שבהם הסביבה משפיעה על המערכת:

  \begin{enumerate}
    \item מעבר של חומר(חלקיקים) 


    \item מעבר של חום 


    \item שינוי בגודל(נפח) 


  \end{enumerate}
\end{proposition}
\begin{definition}
  \begin{enumerate}
    \item מערכת אשר אין מעבר של חומר נקראת מערכת סגורה. 


    \item מערכת שאין בה מעבר של חום נקראת מערכת אדיאבטית. 


    \item מערכת שאין בה שינוי בגודל נקראת מערכת איזוכורית. 


  \end{enumerate}
\end{definition}
\begin{definition}
  \begin{enumerate}
    \item מערכת שבה הלחץ נשאר קבוע נקראת מערכת איזובארית. 


    \item מערכת שבה החום נשאר קבוע נקרא מערכת איזותרמית. 


  \end{enumerate}
\end{definition}
\section{חוקי התרמודינמיקה}

\subsection{חוק ראשון של תרמודינמיקה}

\begin{definition}[תהליך הפיכה]
תהליך שבו המערכת יכולה לחזור למצב ההתחלתי ללא השפעה חיצונית

\end{definition}
\begin{definition}[תהליך קווזיסטטי]
תהליך שבו בכל שלב המערכת נמצאת בשיווי משקל.

\end{definition}
בדרך כלל מתרחש כאשר המערכת משתנה מאוד לאט - ככה שקצב שינוי המצב של המערכת קטנה משמעותית מהסקלות זמן האופיניות שהמערכת צריכה כדי להגיע לשיווי משקל. 

\begin{remark}
בתהליך שאינו קווזיסטטי המשתנים \(P,T,V\) אינם מוגדרים היטב, כיוון שמוגדרים עבור שיווי משקל.

\end{remark}
\begin{proposition}
תהליך הוא הפיך אם"ם הוא קווזיסטטי.

\end{proposition}
אם המערכת לא קווזיסטטית, אז קצב שינוי במערכת ילא זניח לעומת קצב הזמן שלוקח להגיע לשיווי משקל, והתהליך לא יהיה הפיך.

\begin{proposition}[עבודה של גז]
נניח כי גז עבור ממצב יציב \((P_{1},V_{1})\) למצב יציב חדש \((P_{2},V_{2})\) בתהליך קווזיסטטי. אזי:
$$F=PA\implies dW = PAdx=-PdV\implies dW=-PdV\implies W=-\int_{V_{1}}^{V_{2}}PdV$$

\end{proposition}
\begin{remark}
באופן כללי זה תלוי במשוואת מצב.

\end{remark}
\begin{proposition}
אם \(q>0\) נכנס חום למערכת. 
אם \(q<0\) יוצא חום מהמערכת.
אם \(w>0\) נעשה עבודה על המערכת.
אם \(w<0\) המערכת עושה עבודה.

\end{proposition}
\begin{proposition}[עבודה של גז אדיאלי]
$$W=-\int_{V_{1}}^{V_{2}}PdV=-Nk_{B}T\int_{V_{1}}^{V_{2}} \frac{1}{V}dV=-Nk_{B}T\ln\left( \frac{V_{2}}{V_{1}} \right)$$

\end{proposition}
כאשר בתהילך שאינו קווזיסטטי, נקבל כי \(\;\bar{}\mkern-7.5mu dW\)  לא בהכרח דיפרנציאל מדוייק, ולכן העבודה לא תהיה תלוייה רק בהתחלה ובסוף אלה במסלול כולו.

\begin{proposition}[אנרגיה של גז אידיאלי]
$$U={\frac{N k_{B}T}{2}}f$$
כאשר \(f\) זה מספר דרגות החופש.

\end{proposition}
\begin{proposition}[אנרגיה של גז ואן דר ואלס]
$$U(V,T)=\int C_{V}d T-{\frac{a N^{2}}{V}}+U_{0}$$

\end{proposition}
\begin{theorem}[חוק ראשון של תרמודינמיקה]
האנרגיה נשמרת, כאשר עבודה וחום הם שתי צורות של אנרגיה

\end{theorem}
\begin{corollary}
$$\Delta U=\Delta Q+\Delta W$$
כאשר \(\Delta Q\) זה החום שנכנס למערכת, \(\Delta U\) זה השינוי באנרגיה ו-\(\Delta W\) זה העבודה שמתבצעת. ניתן לכתוב בצורה הדיפרנציאלית:
$$dU=\;\bar{}\mkern-7.5mu dQ+\;\bar{}\mkern-7.5mu dW$$
כאשר הסימון \(\;\bar{}\mkern-7.5mu d\) מדגיש שזהו דיפרנציאל לא מדוייק.

\end{corollary}
\subsubsection{קיבול חום}

\begin{definition}[קיבול חום]
קיבול חום \(C\) זה למעשה כמות האנרגיה חום(\(Q\)) שנדרש כדי לעלות את הטמפרטורה במעלה אחת. מוגדר:
$$C(p,T)=\operatorname*{lim}_{\Delta T\rightarrow0}\frac{\Delta Q}{\Delta T}=\frac{\mathrm{d}Q}{d T}$$

\end{definition}
\begin{definition}[קיבול חום סגולי ]
קיבול חום ליחידת מסה. כלומר:
$$c=\frac{1}{m}\left( \frac{ đ Q}{d T} \right)$$

\end{definition}
כעת נעבור להגדרות קיבול שרלוונטיות בעיקר עבור גזים.
\textbf{הגדרה} קיבול חום בנפח קבוע
אם לא משנים את הנפח, זה למעשה כמות האנרגיה שצריך כדי לעלות את הטמפרטורה ביחידה אחת:
$$C_{V}=\!\frac{\;\bar{}\mkern-7.5mu dQ_{V}}{d T}\!=\!\left(\frac{\partial U}{\partial T}\right)_{V}$$

\begin{definition}[קיבול חום בלחץ קבוע]
אם לא משנים את הלחץ, זה יהיה כמות האנרגיה הנגדרשת כדי לעלות את הטמפרטורה ביחידה אחת.
$$C_{P}={\frac{\mathrm{\;\bar{}\mkern-7.5mu d}Q_{P}}{d T}}$$

\end{definition}
\subsection{חוק שני של תרמודינמיקה}

\begin{theorem}[הניסוח של קלווין של החוק השני של תרמודימיקה]
לא ייתכן תהליך שהתוצאה היחידה שלו היא המרה מלאה של חום לעבודה.

\end{theorem}
\begin{definition}[אנטרופיה]
החום \(\;\bar{}\mkern-7.5mu dQ\) העוברת למערכת ע"י הטמפרטורה. כלומר:
$$dS = \frac{\;\bar{}\mkern-7.5mu dQ}{T}$$

\end{definition}
\begin{proposition}[ניסוח שקול לחוק השני לתרמודינמיקה]
עבור מערכת מבודדת מתקיים:
$$dS \geq \frac{\;\bar{}\mkern-7.5mu dQ}{T}$$
כאשר עבור תהליך הפיך בלבד מתקיים \(dS = \frac{dQ}{T}\)

\end{proposition}
\begin{corollary}
עבור מערכת מבודדת תרמית(איזותרמית) מתקיים \(\;\bar{}\mkern-7.5mu dQ=0\) ולכן החוק השני יהיה:
$$dS \geq 0$$

\end{corollary}
\begin{corollary}
אם מתייחסים ליקום כמערכת מבודדת, ניתן לכתוב את שתי החוקים הראשונים של תרמודינמיקה בצורה הבאה:

  \begin{enumerate}
    \item האנרגיה הפנימית של היקום היא קבועה - \(U_{\mathrm{Universe}}=\mathrm{constant}.\)


    \item האנטרופיה של היקום יכול רק לגדול 


  \end{enumerate}
\end{corollary}
\begin{proposition}[המשוואה היסודית]
$$dU=\;\bar{}\mkern-7.5mu dQ+\;\bar{}\mkern-7.5mu dW=TdS + PdV$$

\end{proposition}
\begin{proof}
נשתמש בחוק הראשון של תרמודינמיקה ונזכור כי \(dQ = TdS\) וכן \(dW = PdV\) ולכן נקבל את המבוקש.

\end{proof}
\begin{remark}
הכתיבה של האנרגיה הפנימית בעזרת המשתנים \(U=U(S,V)\) נקראים המשתנים הטבעיים. 

\end{remark}
\begin{proposition}
עבור מערכת ציקלית נקבל כי סך האנרגיה הפנימית היא 0, ולכן:
$$\oint T\mathrm{d}S = - \oint P \mathrm{d}V$$

\end{proposition}
\subsection{פוטנצילים תרמודינמים}

בפיזיקה תרמית, מערכת מתוארת לחלוטין ע"י אוסף הגדלים \(p,V,N,T,S,U, \dots\). הגדלים האלה מקושרים ע"י המשוואת מצב. ניתן תמיד לבחור משתנה אחד ולתאר את האחרים בעזרת פונקציית מצב.

\begin{definition}[פוטנציאל תרמודינמי]
גודל אשר בעזרתו ניתן לתאר את המערכת בצורה מלאה.

\end{definition}
אנחנו מכירים דוגמא אחד של פוטנציאל תרמודינמי - האנרגיה הפנימית \(U\).

\begin{definition}[המשתנים הטבעיים של פוטנציאל תרמודינמי]
המשתנים עבורו הדיפרנציאל יהיה מורכב מדיפרציאלים מדוייקים, אשר כל אחד מהם אינו תלוי במסלול.

\end{definition}
\begin{proposition}[תכונות של פוטנציאל תרמודינמי]
  \begin{enumerate}
    \item מכיל בתוכו את כל פונקציות המצב 


    \item ערכי קיצון קובעים שיווי משקל תרמודינמי. 


    \item נגזרות כאשר מקבעים גדלים נותנות את פונקציות המצב. 


  \end{enumerate}
\end{proposition}
\begin{example}
נמצא את הפוטנציאל התרמודינמי - האנרגיה הפנימית \(U\) של גז אידיאלי. באופן כללי אנו יודעים כי:
$$dU=\;\bar{}\mkern-7.5mu dQ+\;\bar{}\mkern-7.5mu dW =TdS - pdV$$
כאשר \(p=p(V,S)\) ו-\(T=T(V,S)\). כעת נדרש כבר להשתמש במשוואת מצב. אנו יודעים כי עבור גז אידיאלי:
$$pV=Nk_{B}T \implies p(V,S)= \frac{Nk_{B}T(V,S)}{V}$$
כאשר אם נשתמש באנרגיה של גז אידיאלי נקבל:
$$U=\frac{3}{2}Nk_{B}T\implies T(V,S)= \frac{U(V,S)}{\frac{3}{2}Nk_{B}T}$$
ולכן אם נציב את \(T\) נקבל:
$$dU= TdS-pdV=\frac{U}{\frac{3}{2}Nk_{B}}dS - \frac{2}{3}U \frac{dV}{V}$$
כאשר ניתן כעת לפתור את המדר ולקבל:
$$\ln \left( \frac{U}{U_{0}} \right)=-\frac{2}{3}\ln\left( \frac{V}{V_{0}} \right)+\frac{S-S_{0}}{\frac{3}{2}Nk_{B}} \implies U(S,V)=U_{0}\left( \frac{V_{0}}{V} \right)^{2/3} e^{ \frac{S-S_{0}}{3/2 Nk_{B}T} }$$
כאשר נקבל כי \(\left( \frac{\partial U}{\partial S} \right)_{V}\) יהיה שווה למשוואת מצב של האנרגיה, ועבור \(\left( \frac{\partial U}{\partial V} \right)_{S}\) נקבל את משוואת המצב של גז אידיאלי.

\end{example}
\subsubsection{הפוטנציאלים}

\begin{definition}[אנרגיה פנימית]
פוטנציאל  המקיים:
$$\mathrm{d}U=-p\mathrm{d}V+T\mathrm{d}S$$

\end{definition}
כלומר זהו פוטנציאל על משתנים טבעיים \(V,S\).

\begin{proposition}
עבור תהליך איזוכורי(נפח קבוע) מתקיים:
$$\Delta U = \int_{T_{1}}^{T_{2}} C_{V} \, dT $$

\end{proposition}
\begin{proof}
נזכור כי בלחץ קבוע מתקיים \(dU=TdS\) ועבור תהליך הפיך מתקיים \(dU=\;\bar{}\mkern-7.5mu dQ_{rev}=C_{V}dT\). ומאינטגרציה נקבל את הטענה.

\end{proof}
\begin{proposition}
$$T=\left(\frac{\partial U}{\partial S}\right)_{V}\qquad p=-\left(\frac{\partial U}{\partial V}\right)_{S}$$

\end{proposition}
\begin{definition}[אנטלפיה]
$$H(S,p)=U+PV$$
ולכן:
$$\mathrm{d}H=T\mathrm{d}S-p\mathrm{d}V+p\mathrm{d}V+V\mathrm{d}p=T\mathrm{d}S+V\mathrm{d}p$$

\end{definition}
כלומר זהו פוטנציאל עם משתנים טבעיים \(p,S\).
כאשר עבור תהליך בלחץ קבוע נקבל: 
$$\mathrm{d}H=T\mathrm{d}S\implies \Delta H = \int_{T_{1}}^{T_{2}} C_{p} \, dT$$

\begin{corollary}
בדומה להגדרה בעזרת האנרגיה עם נפח קבוע, מתקיים:
$$T=\left(\frac{\partial H}{\partial S}\right)_{p}\qquad V=\left(\frac{\partial H}{\partial p}\right)_{S}$$

\end{corollary}
\begin{definition}[הפוטנציאל החופשי של הלמהולדס]
$$F(T,V)=U-TS$$
ולכן:
$$\mathrm{d}F=T\mathrm{d}S-p\mathrm{d}V-T\mathrm{d}S-S\mathrm{d}T=-S\mathrm{d}T-p\mathrm{d}V$$

\end{definition}
וקיבלנו פוטנציאל עם המשתנים הטבעיים \(T,V\)

\begin{corollary}
$$S=-\left(\frac{\partial F}{\partial T}\right)_{V}\qquad p=-\left({\frac{\partial F}{\partial V}}\right)_{T}$$

\end{corollary}
\begin{definition}[הפוטנציאל החופשי של גיבס]
$$G(T,p)=H-TS$$
ולכן:
$$\mathrm{d}G=T\mathrm{d}S+V\mathrm{d}p-T\mathrm{d}S-S\mathrm{d}T=-S\mathrm{d}T+V\mathrm{d}p$$

\end{definition}
וקיבלנו פוטנציאל עם המשתנים הטבעיים \(T,p\)

\begin{corollary}
$$S=-\left(\frac{\partial G}{\partial T}\right)_{p}\qquad V=\left(\frac{\partial G}{\partial p}\right)_{T}$$

\end{corollary}
\subsubsection{זמינות}

\begin{definition}[זמינות]
נסתכל על מערכת שנמצאת בתוך סביבה עם טמפרטורה \(T_{0}\) ולחץ \(p_{0}\). נגדיר את הזמינות בתור סך האנרגיה שניתן להוציא מהמערכת. זה יהיה שווה לביטוי הבא:
$$
A=U+p_{0}V-T_{0}S,$$
כלומר זה למעשה האנרגיה הפנימית שיש לגוף, ועוד האנרגיה שניתן לקבל מעבודה שהלחץ החיצוני יכול להפעיל על הנפח, פחות החום שניתן לקבל מהשינוי באנטרופיה בגלל התפשטות תרמית בנפח.

\end{definition}
\begin{proposition}
עבור מערכת מבודדת מכנית, מתקיים \(dA\leq 0\). 

\end{proposition}
\begin{proof}
עבור המערכת שלנו, מתקיים מחוק ראשון של תרמודימיקה:
$${\mathrm{d}}Q=\mathrm{d}U-{\mathrm{d}}W-\left( -p_{0}\,\mathrm{d}V \right)\implies \mathrm{d}W\geq\mathrm{d}U+p_{0}\mathrm{d}V-T_{0}\mathrm{d}S=dA$$
וכן כאשר המערכת מבודדת מכנית, מתקיים \(dW=0\) ולכן \(dA\leq 0\).

\end{proof}
\begin{corollary}
הזמינות יכולה רק לרדת, כאשר מקבלת מינימום בשיווי משקל.

\end{corollary}
נסתכל כעת על המקרים המיחודים של מזעור הזמינות.

\begin{proposition}
כשהלחץ והטמפרטורה קבועה, הזמינות שווה לאנרגיה החופשית של גיבס\((G)\). ובפרט המערכת מגיעה לשיווי משקל כש-\(G\) מינימלי

\end{proposition}
\begin{proof}
$$\Delta\!A\!=\!\Delta(U\!+\!P_{0}V\!-\!T_{0}S)\!=\!\Delta(U\!+\!P V\!+\!T S)\!=\!\Delta G\!\leq\!0$$

\end{proof}
\begin{proposition}
כשהנפח והטמפרטורה קבועים, הזמינות שווה לאנרגיה החופשית של הלמהולדס

\end{proposition}
\begin{proof}
$$\Delta A=\Delta(U+P_{0}V-T_{0}S)=\Delta(U-T_{0}S)=\Delta(U-T S)=\Delta F\leq0$$

\end{proof}
\begin{corollary}
כאשר המערכת מבודדת תרמית עם נפח קבוע נקבל כי \(dU=0\) מחוק ראשון

\end{corollary}
\subsubsection{יחסי מקסוואל}

אנו יודעים כי דיפרנציאל מדוייק יהיה מהצורה:
$$\mathrm{d}f=\overbrace{ \left({\frac{\partial f}{\partial x}}\right)_{y} }^{ F_{x} }\mathrm{d}x+\overbrace{ \left({\frac{\partial f}{\partial y}}\right)_{x} }^{ F_{y} }\mathrm{d}y=F_{y}dx+F_{x}dy$$
כאשר עבור דיפרנציאל מדוייק, נדרש כי יתקיים:
$$\left({\frac{\partial F_{y}}{\partial x}}\right)=\left({\frac{\partial F_{x}}{\partial y}}\right)$$
ניתן לעשות זאת על כל אחד מהיחסים שלנו ולקבל קשר שנקרא קשר מקסוואל. נראה לדוגמא עם הפוטנציאל החופשי של גיבס

\begin{example}
אנו יודעים כי עבור הפוטנציאל החופשי של גיבס מתקיים:
$$\mathrm{d}G=-S\mathrm{d}T+V\mathrm{d}p$$
כאשר ניתן גם לכתוב:
$$\mathrm{d}G=\left({\frac{\partial G}{\partial T}}\right)_{p}\,\mathrm{d}T+\left({\frac{\partial G}{\partial p}}\right)_{T}\,\mathrm{d}p$$
ולכן מהשוואות המקדמים נקבל:
$$\left({\frac{\partial G}{\partial T}}\right)_{p} = -S\qquad \left( \frac{\partial G}{\partial p}  \right)_{T}=V$$
ומזה שדיפרציאל מדוייק:
$$-\left({\frac{\partial S}{\partial p}}\right)_{T}=\left({\frac{\partial V}{\partial T}}\right)_{p}$$

\end{example}
ניתן לבצע תהליך דומה ולקבל את כל יחסי מקסוואל:

\begin{proposition}[יחסי מקסוואל]
$$\begin{gather}{{\left(\frac{\partial T}{\partial V}\right)_{S}}}&{{=}}&{{-\left(\frac{\partial p}{\partial S}\right)_{V}}}\\ {{\left(\frac{\partial T}{\partial p}\right)_{S}}}&{{=}}&{{\left(\frac{\partial V}{\partial S}\right)_{p}}}\\ {{\left(\frac{\partial S}{\partial V}\right)_{T}}}&{{=}}&{{\left(\frac{\partial p}{\partial T}\right)_{V}}}\\ {{\left(\frac{\partial S}{\partial p}\right)_{T}}}&{{=}}&{{-\left(\frac{\partial V}{\partial T}\right)_{p}}}\end{gather}$$

\end{proposition}
\subsubsection{המלבן התרמודינמי}

טריק מגניב לזכור את יחסי מקסוואל. מציירים מלבן וממספרים את הקודקודים החל הימין באמצע בעזרת הביטוי
$$\text{Great Physicists Have Studied Under Very Fine Teachers.}$$
כאשר כדי לזכור את הסימנים נשים חצים בין \(S\) ל-\(T\) ובין \(P\) ל-\(V\). נקבל:

\includegraphics[width=0.8\textwidth]{diagrams/svg_2.svg}
\begin{proposition}
שתי הגדלים שנמצאים ליד כל פוטנציאל תרמודינמים הם המשתנים הטבעיים שלו.

\end{proposition}
\begin{proposition}
כדי למצוא את הדיפרנציאל של פוטנציאל תרמודינמי נצטרך להשתמש בחצים. מסתכלים על שתי המשתנים הטבעיים, מכפילים כל הדיפרנציאל של המשתנה הטבעי בערך שנמצא בצד השני של החץ, כאשר בחסירים אם החץ מצביע על המשתנה הטבע ומחסרים אחרת. לדוגמא אם נסתכל על \(G\) המשתנים הטבעיים יהיו \(P,T\) כאשר נכפיל בערך שנמצא בצד השני של החץ כאשר עבור \(dT\) נדרש סימן מינוס כי החץ מצביע על \(T\) ונקבל:
$$dG=-SdT+PdV$$

\end{proposition}
\begin{proposition}
כדי למצוא את יחסי מקסוואל נצטרך להסתכל על המסלול שאורך הריבוע של שתי משתנים טבעיים של משתנה. נראה בעזרת דוגמא. נסתכל על \(G\). יש ל-\(G\) שתי משתנים טבעיים - \(P,T\). עבור \(T\) נלך ל-\(V\) ואז ל-\(S\). ונקבל \(\left( \frac{\partial T}{\partial V} \right)_{S}\). עבור המסלול השני נקבל \(\left( \frac{\partial P}{\partial S} \right)_{V}\). כיוון שבמסולולים הסגורים שעוברים דרך האלכסון מסלול אחד הוא בכיוון החצים והמסלול השני נגד כיוון החצים ולכן נדרש להוסיף פה סימן מינוס(אם שניהם היו נגד כיווני החצים או שניהם היו בכיוון החצים אז לא היה נדרש).

\end{proposition}
\subsection{החוק השלישי}

\begin{proposition}
מתקיים:
$$S(T)=S(T_{0})+\int_{T_{0}}^{T}{\frac{C_{p}}{T}}\mathrm{d}T,$$

\end{proposition}
\begin{proof}
נזכור כי מתקיים \(C_{p}=T\left({\frac{\partial S}{\partial T}}\right)_{p}\). לכן:
$$S=\int{\frac{C_{p}}{T}}\mathrm{d}T\implies S(T)=S(T_{0})+\int_{T_{0}}^{T}{\frac{C_{p}}{T}}\mathrm{d}T,$$

\end{proof}
נראה שלמעשה אין לנו דרך למצוא את האנטרופיה על ידי שינוי הטמפרטורה, וניתן למצוא רק את השינוי באנטרופיה. החוק השלישי יאפשר לנו למצוא את האנטרופיה של מערכת ע"י הגדרת הערך של האנטרופיה בטמפרטורה ספציפית - 0.

\begin{theorem}[חוק שלישי של תרמודינמיקה]
האנטרופיה 0 מוגדרת בתור הגבול של מערכת כאשר הטמפרטורה שואפת ל-0.

\end{theorem}
\begin{remark}
יש הרבה הגדרות שונות של החוק השלישי, אך כולם שקולים.

\end{remark}
נסתכל כעת על מספר מסקנות מהחוק השלישי.
\textbf{מסקנה}
כל הקיבולי חום שואפים ל-0. זאת כיוון שמתקיים:
$$C=T\left({\frac{\partial S}{\partial T}}\right)=\left({\frac{\partial S}{\partial\ln T}}\right)\rightarrow 0$$

\begin{corollary}
התפשטות תרמית מפסיקה, זאת כי מתקיים \(S,T\to 0\) ולכן \(\left(\frac{\partial S}{\partial p}\right)_{T}\rightarrow0\) אבל מיחס מקסוואל נקבל כי:
$$\frac{1}{V}\left(\frac{\partial V}{\partial T}\right)_{p}\rightarrow0$$

\end{corollary}
\begin{corollary}
לא ניתן לקרר ל-\(T=0\) במספר סופי של צעדים.

\end{corollary}
\section{תהליכים ומנועים}

\subsection{סיווג תהליכים}

\begin{definition}[גרף PV]
גרף של לחץ כתלות בנפח.

\end{definition}
\begin{proposition}
השטח מתחת לגרף PV הוא העבודה:
$$W=\int_{V_{i}}^{V_{f}}P\,d V$$

\end{proposition}
\begin{definition}[גרף ST]
גרף של טמפרטורה כתלות באנטרופיה.

\end{definition}
\begin{proposition}
השטח מתחת לגרף ST הוא המעבר חום:
$$Q=\int_{S_{i}}^{S_{f}}T\,d S$$

\end{proposition}
\subsection{תהליך איזותרמי}

\begin{definition}[תהליך איזותרמי]
תהליך שהטמפרטורה בו לא משתנה. כלומר מתקיים \(\Delta T=0\).

\end{definition}
אפשר לחשוב על תהליך זה בתור תהליך שנמצא ליד מקור חום אשר תמיד מכניס אנרגיה למערכת כך שהטמפרטורה תהיה תמיד שווה, לא משנה איך משתנה הנפח.

\begin{proposition}[אין שינוי באנרגיה]
עבור גז אידיאלי בתהליך איזותרמי, מתקיים \(\Delta U=0\).

\end{proposition}
זאת כיוון שאנו יודעים כי עבור גז אידיאלי,  \(dU=C_{V}dT\).
\textbf{מסקנה}
עבור גז אידיאלי, מהחוק הראשון של תרמודינמיקה מתקיים \(\;\bar{}\mkern-7.5mu dW=\;\bar{}\mkern-7.5mu dQ\).

\includegraphics[width=0.8\textwidth]{diagrams/svg_3.svg}
\begin{proposition}
העבודה של תהליך איזותרמי עבור גז אידאילי יהיה:
$$W=\int_{V_{1}}^{V_{2}} \frac{Nk_{B}T}{V} \mathrm{d}V=Nk_{B}T\ln\left( \frac{V_{2}}{V_{1}} \right)$$

\end{proposition}
\begin{remark}
קיבול חום לא מוגדר בתהליך איזותרמי. זאת כוון שנדרש לחלק ב-\(\Delta T\) אשר אפס.

\end{remark}
\begin{proposition}[שינוי באנטרופיה של תהליך איזותרמי]
$$\Delta S=n R\ln{\frac{V_{f}}{V_{i}}}=n R\ln{\frac{P_{i}}{P_{f}}}$$

\end{proposition}
\subsection{תהליך אדיאבטי}

\begin{definition}[תהליך אדיאבטי]
תהליך אשר אין בו זרימה של חום, כלומר מבודד תרמית. זה אומר שמתקיים:
$$\;\bar{}\mkern-7.5mu dQ=0$$

\end{definition}
\begin{corollary}
מהחוק הראשון של תרמודינמיקה נקבל:
$$dU=\;\bar{}\mkern-7.5mu dW$$

\end{corollary}
כאשר עבור גז אידיאלי אנו יודעים כי \(dU=C_{V}dT\) וכן עבור תהליך הפיך נקבל \(\;\bar{}\mkern-7.5mu dW=-pdV\) ולכן נקבל כי מתקיים:
$$C_{V}\,\mathrm{d}T=-p\,\mathrm{d}V=-{\frac{R T}{V}}\,\mathrm{d}V,$$
ולכן:
$$\ln\frac{T_{2}}{T_{1}}=-\frac{R}{C_{V}}\ln\frac{V_{2}}{V_{1}}.$$
כעת \(C_{p}=C_{V}+R\) כאשר אם נחלק גודל זה ב-\(C_{V}\) נקבל:
$$\gamma=\frac{C_{p}}{C_{V}}=1+\frac{R}{C_{V}}\implies-(R/C_{V})=1-\gamma,$$
ונקבל:
$$T V^{\gamma-1}={\mathrm{constant}} \implies p^{1-\gamma}T^{\gamma}=\mathrm{constant}\implies p V^{\gamma}={\mathrm{constant}},$$

\begin{proposition}[דחיסה אדיבאטית]
$$p_{1}V_{1}^\gamma=p_{2}V_{2}^\gamma \qquad \gamma = \frac{C_{P}}{C_{V}}$$

\end{proposition}
\includegraphics[width=0.8\textwidth]{diagrams/svg_4.svg}
\begin{proposition}
$$TV^{\gamma-1}=\mathrm{const}\qquad T^\gamma P^{1-\gamma}=\mathrm{const}$$

\end{proposition}
\begin{proposition}[שינוי באנטרופיה בתהליך אדיאבטי]
עבור תהליך הפיך נקבל \(\Delta S = 0\). כאשר עבור תהליך שאינו הפיך נקבל \(\Delta S > 0\).

\end{proposition}
\subsection{תהליך איזוכורי}

\begin{definition}[תהליך איזוכורי]
תהליך שבו הנפח נשאר קבוע

\end{definition}
\begin{proposition}[קיבול חום תחת נפח קבוע]
$$C_{V}=\!\frac{\;\bar{}\mkern-7.5mu dQ_{V}}{d T}\!=\!\left(\frac{\partial U}{\partial T}\right)_{V}$$

\end{proposition}
\begin{proposition}[אין עבודה]
$$W=\int_{V_{0}}^{V_{1}} P \;\mathrm{d}V=0$$

\end{proposition}
\begin{proposition}[חום בתהליך איזוכורי]
$$Q=C_{V}\Delta T$$

\end{proposition}
\begin{proposition}[אנרגיה פנימית שווה לחום]
$$U=W+Q = Q=C_{V}\Delta T$$

\end{proposition}
\begin{proposition}[אנטרופיה בתהליך איזוכורי]
$$\Delta S = \int \mathrm{\frac{\;\bar{}\mkern-7.5mu dQ}{T}}=C_{V}\ln \frac{T_{2}}{T_{1}}$$

\end{proposition}
\begin{proposition}[יחסים של גז אידיאלי]
$$V_{1} = V_{2}\qquad  \frac{P_{1}}{T_{1}}=\frac{P_{2}}{T_{2}}$$

\end{proposition}
\includegraphics[width=0.8\textwidth]{diagrams/svg_5.svg}
\begin{proposition}[אנטרופיה בתהליך איזוכורי]
$$\Delta S=n C_{v}\ln{\frac{T_{f}}{T_{i}}}$$

\end{proposition}
\subsection{תהליך איזובארי}

\begin{definition}[תהליך איזובארי]
תהליך שבו הלחץ נשאר קבוע

\end{definition}
\begin{proposition}[עבודה של תהליך איזובארי]
כיוון שהלחץ קבוע נקבל מיידית:
$$W=P\Delta V$$

\end{proposition}
\begin{proposition}[אנטרופיה של תהליך איזובארי]
$$\Delta S =C_{P}\ln \left( \frac{T_{2}}{T_{1}} \right)$$

\end{proposition}
\begin{proposition}[יחסים של גז אידיאלי]
$$\frac{V_{1}}{T_{1}}=\frac{V_{2}}{T_{2}}$$

\end{proposition}
\includegraphics[width=0.8\textwidth]{diagrams/svg_6.svg}
\begin{proposition}[שינוי באנטרופיה בתהליך איזובארי]
$$\Delta S=n C_{p}\ln{\frac{T_{f}}{T_{i}}}$$

\end{proposition}
\subsection{מנועים}

ראינו כי לפי החוק השני של תרמודינמיקה, לא ייתכן תהליך שהתוצאה היחידה שלו היא המרה של חום לעבודה. אם לא ניתן להמיר את כל החום לעבודה, זה מעלה את השאלה מה הכמות המקסימלית שניתן להמיר. לצורך זה נגדיר את המושג של מנוע.
\textbf{הגדרה} מנועה
מערכת הפועלת בצורה מחזורית הממירה אנרגיה לעבודה.

\includegraphics[width=0.8\textwidth]{diagrams/svg_7.svg}
המערכת מורכבת משלושה רכיבים.

\begin{enumerate}
  \item מקור חום בלתי מוגבל(אדום) - כלומר מעביר אנרגיית חום ונשאר תמיד בטמפרטורה \(T_{h}\). 


  \item אמבט קור בלתי מוגבל(כחול) - כלומר מקבל אנרגיית חום ונשאר תמיד בטמפרטורה \(T_{c}\). 


  \item רכיב ביניהם אשר נקרא לו המנוע. המנוע מלא בגז אידיאלי ויכול לשנות את הנפח שלו בהתאם לקשר \(PV=k_{B}NT\) - כאשר השינוי בנפח יכול לגרום לעבודה. 


\end{enumerate}
\textbf{שלב ראשון - התפשטות איזותרמית:} 
המנוע מקבל חום מהאמבט החם בתהליך איזותרמי. כלומר הטמפרטורה נשארת קבועה והנפח גדל בהתאם כדי שהטמפרטורה לא תגדל.

\textbf{השלב השני - התפשטות אדיאבטית:}
נבודד את המנוע מהסביבה בתהליך אדיאבטי. המערכת ממשיכה להתפשט אבל הלחץ יורד.

\textbf{השלב השלישי - דחיסה איזותרמית:}
מחברים את המנוע לאמבט הקר. נקבל כי הנפח קטן, אבל הלחץ גדל.

\textbf{השלב הרביעי - דחיסה אדיובטית:}
מנתקים את המנוע מהאמפט הקר. הלחץ גדל אך הנפח קטן.

\includegraphics[width=0.8\textwidth]{diagrams/svg_8.svg}
\begin{definition}[מנוע קרנו]
רכיב אשר מקיים את מחזור קרנו נקרא מנוע קרנו.

\end{definition}
\includegraphics[width=0.8\textwidth]{diagrams/svg_9.svg}
\begin{definition}[יעילות]
הייעוד של מנוע זה להפוך חום לעבודה. לכן היעילות שלו תהיה היחס בין כמות החום שהתקבל לעבודה שהוציא. כלומר:
$$\eta = \left\lvert  \frac{W_{out}}{Q_{in}}  \right\rvert = \frac{\lvert Q_{in} \rvert -\lvert Q_{out} \rvert}{\lvert Q_{in} \rvert }=1-\left\lvert  \frac{Q_{out}}{Q_{in}}  \right\rvert  $$

\end{definition}
\begin{proposition}[יעילות של מנוע קרנו]
$$\eta_{carnot}=1-\frac{T_{C}}{T_{H}}=\frac{T_{H}-T_{C}}{T_{H}}$$

\end{proposition}
\begin{proof}
בחלק הראשון נקבל כי בין \(A\) ל-\(B\) מתקיים:
$$Q_{H}=T_{H}(S_{B}-S_{A}) \qquad Q_{C}=T_{C}(S_{B}-S_{A})$$
וניתן להציב בביטוי של היעילות ולקבל:
$$\eta = \frac{Q_{H} - Q_{C}}{Q_{H}}= \frac{(T_{H}-T_{C})(S_{B}-S_{A})}{T_{H}(S_{B}-S_{A})}=\frac{T_{H}-T_{C}}{T_{H}}$$

\end{proof}
\begin{remark}
ניתן גם לפתח בדרך יותר מכוערת ללא אנטרופיה. הרעיון הוא להראות שבתליך האיזותרמי מתקיים:
$$\frac{V_{B}}{V_{A}}=\frac{V_{C}}{V_{D}}$$
ואז למצוא את \(Q_{H},Q_{C}\) אשר יכילו את הגורמים \(\frac{V_{B}}{V_{A}}\) ו-\(\frac{V_{C}}{V_{D}}\). אם נציב בביטוי של יעילות נקבל כי הגורמים השווים יצטמצמו וניתן יהיה למצוא את היעילות.

\end{remark}
\begin{theorem}
היעילות של מנוע קרנו היא היעילות המקסימלית האפשרית בשביל מנוע.

\end{theorem}
\section{פוטנציאל כימי ומעברי פאזה}

\subsection{פוטנציאל כימי}

נרצה להסתכל כעת על מערכות פתוחות, כלומר מערכות שיש מעבר של חלקיקים עם הסביבה. דוגמא למשל היא גוש קרח הצף במים ומאבד חלקיקים.

\begin{definition}[פוטנציאל כימי]
זה יהיה כמה שמשתנה האנרגיה הפנימית שמספר החלקיקים משתנה:
$$\mu\!=\!\left({\frac{\partial U}{\partial N}}\right)_{S,V}$$

\end{definition}
ננסה להסביר:
נסתכל על המשוואה היסודית:
$$d U=T d S-P d V$$
כאשר ניתן להוסיף את האפקט של איבוד החלקיקים:
$$d U=T\,d S-P\,d V+\mu\,d N$$
כאשר ניתן גם לנוסיף מספר פוטנציאלים כימים כאשר יש סוגים שונים של חלקיקים:
$$d U=T d S-P\,d V+\sum_{i}\mu_{i}\,d N_{i}$$

\begin{proposition}
ניתן לכתוב בעזרת האנרגיה החופשית של הלמהולדס
$$\mu=\left({\frac{\partial F}{\partial N}}\right)_{V,T}$$

\end{proposition}
\begin{proof}
$${F=U-T S}\implies d F=d U-T\,d S-S\,d T$$
ולכן מהמשוואה של הדיפרנציאל של האנרגיה הפנימית נקבל:
$$d F=-P\,d V-S\,d T+\mu\,d N\implies \mu=\left({\frac{\partial F}{\partial N}}\right)_{V,T}$$

\end{proof}
בנוסף אנו יודעים את הקשר בין פוטנציאל הלמהולדס לפונקציית החלוקה \(Z\):
$$F=-N k_{B}T\ln Z$$
ומזה ניתן לקבל את הקשר בין \(\mu\) ל-\(Z\):
$$\mu=-k_{\mathrm{{B}}}T\left({\frac{\partial}{\partial N}}(N\ln Z)\right)_{V,T}$$

\begin{proposition}
כאשר יש סוג אחד של חלקיק, מתקיים:
$$\mu=\frac{G}{N}$$
כלומר נקבל כי הפוטנציאל הכימי זה למעשה אנרגיית גיבס לחלקיק.

\end{proposition}
\begin{proof}
באופן דומה למה שעשינו עבור פוטנציאל הלמהולדס. נכתוב:
$$G=H-T S=U+P V-T S\implies  d G\,=\,d U\,+\,P\,d V\,+\,V\,d P\,-\,T\,d S\,-\,S\,d T$$
כאשר אם נשתמש בדיפרנציאל של האנרגיה הפנימית נקבל:
$$d G\,=\,V\,d P\,-\,S\,d T\,+\,\mu\,d N\implies \mu=\left({\frac{\partial G}{\partial N}}\right)_{T,P}$$
כיוון ש-\(G\) הוא גודל אקסטנסיבי, הוא פרופרצינאלי למספר חלקיקים, כלומר קיים \(\phi(T,P)\) כך ש:
$$G(T,P,N)\!=\!N\phi(T,P)$$
ולכן אם נגזור לפי \(N\) כאשר נשאיר את \(T,P\) קבועים נקבל:
$$\mu=\phi(T,P)\implies \mu=\frac{G}{N}$$

\end{proof}
נעיר כי לא ניתן לבצע תהליך דומה עבור פוטנציאלים כמו \(U,F\) כיוון שהמשתנים הטבעיים שלהם הם לא שניהם אינטנסיבים ולכן אם נכתוב:
$$U\!=\!U\!(S,V,N)\!=\!N\phi^{\prime}\!\left({\frac{S}{N}},{\frac{V}{N}}\right)$$
נקבל כי גזירה לפי \(N\) תתן:
$$\mu=\phi^{\prime}+N\Biggl(\frac{\partial\phi^{\prime}}{\partial N}\Biggr)_{S,V}=\frac{U}{N}+N\Biggl(\frac{\partial\phi^{\prime}}{\partial N}\Biggr)_{S,V}$$
כאשר מתקבל גורם נוסף.

\begin{corollary}
באופן כללי במערכת אם סוגים שונים של חלקיקים נקבל:
$$G=\sum_{i}\mu_{i}N_{i}$$

\end{corollary}
\begin{example}[שיווי משקל בין שתי מערכות המחליפות חלקיקים]
נסתכל על מערכת בו יש שתי חלקיקים(\(A,B\)) שיכולים להחליף חלקיקים חום ואנרגיה ביניהם אבל המערכת הכוללת סגורה, כלומר מתקיים:
$$\begin{gather}N_{\mathrm{A}}+N_{\mathrm{B}}=N \\V_{\mathrm{{A}}}+V_{\mathrm{{B}}}=V \\U_{\mathrm{{A}}}+U_{\mathrm{{B}}}=U
\end{gather}$$
נשים לב כי האנטרופיה תהיה שווה:
$$\begin{gather}{S\,=S\bigl(U_{\mathrm{A}},V_{\mathrm{A}},N_{\mathrm{A}},U_{\mathrm{B}},V_{\mathrm{B}},N_{\mathrm{B}}\bigr)}{=S_{\mathrm{A}}\bigl(U_{\mathrm{A}},V_{\mathrm{A}},N_{\mathrm{A}}\bigr)\!+\!S_{\mathrm{B}}\bigl(U_{\mathrm{B}},V_{\mathrm{B}},N_{\mathrm{B}}\bigr)}\end{gather}$$
כאשר בשיווי משקל נדרוש כי האנטרופיה תהיה מקסימלית. כלומר מתקיים:
$$dS=dS_{A}+dS_{B}=0$$
נזכור כי מתקיים:
$$d U=T\,d S-P\,d V+\sum_{i}\mu_{i}\,d N_{i}$$
ולכן נבודד את האנטרופיה ונקבל:
$$\begin{gather}d S_{\mathrm{A}}={\frac{1}{T_{\mathrm{A}}}}(d U_{\mathrm{A}}+P_{\mathrm{A}}d V_{\mathrm{A}}-\mu_{\mathrm{A}}\,d N_{\mathrm{A}})  \\  d S_{\mathrm{B}}={\frac{1}{T_{\mathrm{B}}}}\big(d U_{\mathrm{B}}+P_{\mathrm{B}}\,d V_{\mathrm{B}}-\mu_{\mathrm{B}}\,d N_{\mathrm{B}}\big) 
\end{gather}$$
כלומר אחרי שנציב ב-\(dS\) נקבל:
$${\frac{1}{T_{\mathrm{A}}}}(d U_{\mathrm{A}}+P_{\mathrm{A}}\,d V_{\mathrm{A}}-\mu_{\mathrm{A}}\,d N_{\mathrm{A}})+{\frac{1}{T_{\mathrm{B}}}}(d U_{\mathrm{B}}+P_{\mathrm{B}}\,d V_{\mathrm{B}}-\mu_{\mathrm{B}}\,d N_{\mathrm{B}})=0$$
ולכן:
$$\Biggl({\frac{1}{T_{\mathrm{A}}}}-{\frac{1}{T_{\mathrm{B}}}}\Biggr)d U_{\mathrm{A}}+\Biggl({\frac{P_{\mathrm{A}}}{T_{\mathrm{A}}}}-{\frac{P_{\mathrm{B}}}{T_{\mathrm{B}}}}\Biggr)d V_{\mathrm{A}}-\Biggl({\frac{\mu_{\mathrm{A}}}{T_{\mathrm{A}}}}-{\frac{\mu_{\mathrm{B}}}{T_{\mathrm{B}}}}\Biggr)d N_{\mathrm{A}}=0$$
כמו באלגברה לינארית, נדרש לכל אחד מהדיפרנציאלים מתאפס, כלומר נקבל כי:
$$T_{A}=T_{B}\qquad P_{A}=P_{B}\qquad \mu_{A}=\mu_{B}$$

\end{example}
\subsection{מעברי פאזה}

\begin{definition}[פאזה תרמודינמית]
דרך לייחד חומר המתחום במרחב(כלומר נמצא ביחד כגוש) ע"י התכונות הכימיות והפיזיות שלו. לדוגמא, צפיפות, מוליכות חשמלית, צורה, סידור פנימי של אטומים ועוד. מצבי הצבירה מוצק נוזל וגז הם דוגמאות לפאזות.

\end{definition}
כאשר מערכת משנה פאזה תרמודינמית, יש שלב שבה הטמפרטורה נשארת קבוע למרות שמכניסים חום למערכת. זה אומר שהתיאור של קיבול חום כבר לא יהיה נכון. לצורך זה נגדיר מושג חדש.

\begin{proposition}
המעבר פאזה תלוי בלחץ ובטמפרטורה, כאשר מעבר פאזה מתרחש בלחץ וטמפרטורה קבועה.

\end{proposition}
\begin{definition}[חום כמוס]
כמות האנרגיה התרמית הנדרשת כדי להעביר חומר מצב צבירה כאשר הלחץ קבוע.
$${L}=\Delta Q_{\mathrm{rev}}=T_{\mathrm{c}}(S_{2}-S_{1})$$

\end{definition}
\includegraphics[width=0.8\textwidth]{diagrams/svg_10.svg}
\begin{definition}[דיאגרמת פאזה]
זה גרף של לחץ כפונקציה של טמפרטורה. כל תחום קשיר מייצג פאזה.

\end{definition}
נרצה כעת להסתכל על מערכת שמורכבת משתי פאזות ושל אותו החומר ונרצה לראות מה התנאי לשיווי משקל.
\textbf{טענה}
בשיווי משקל נדרש כי לשתי הפאזות יש פוטנציאל כימי שווה - \(\mu_{1}=\mu_{2}\)

\begin{proof}
נכתוב:
$$G=M_{1}\mu_{1}+M_{2}\mu_{2}$$
כאשר אנו יודעים כי בלחץ וטמפרטורה קבועה השיווי משקל הדיפרנציאל יהיה 0. כלומר:
$$dG=0=\mu_{1}dM_{1}+\mu_{2}dM_{2}$$
כאשר אם המערכת סגורה כך ש-\(M\) קבוע נקבל:
$$dM=dM_{1}+dM_{2}=0$$
ומזה ניתן להסיק כי \(\mu_{1}=\mu_{2}\).

\end{proof}
\begin{proposition}[משוואת קלאסיוס קלפירון]
$${\frac{\mathrm{d}p}{\mathrm{d}T}}={\frac{L}{T(V_{2}-V_{1})}}$$
כאשר זה נותן את המביטוי של המעבר פאזה כתלות בחום בכמוס, הטמפרטורה הקריטית וההבדלי נפח.

\end{proposition}
\begin{proof}
נסתכל על מערכת שמורכבת משתי פאזות אשר נמצא על הגבול בין שתי הפאזות. מהטענה הקודמת נקבל:
$$\mu_{1}(p,T)=\mu_{2}(p,T)$$
כאשר אם נבצע הזזה קטנה נקבל:
$$\mu_{1}\left( p+\mathrm{d}p,T+\mathrm{d}T \right)=\mu_{2}\left( p+\mathrm{d}p,T+\mathrm{d}T \right)\implies d\mu_{1}=d\mu_{2}$$
כאשר אם אין מעבר חלקיקים נקבל כי \(dG_{1}=dG_{2}\) ולכן:
$$-S_{1}\mathrm{d}T+v_{1}\mathrm{d}p=-S_{2}\mathrm{d}T+v_{2}\mathrm{d}p \implies \frac{\mathrm{d}p}{\mathrm{d}T}=\frac{S_{2}\,-\,S_{1}}{v_{2}\,-\,v_{1}}$$
כאשר ניתן לכתוב \(L=T\Delta S\) ולקבל את המשוואה.

\end{proof}
\begin{definition}[מעבר פאזה מסדר ראשון]
כאשר הנפח משתנה, יש קפיצה באנטרופיה וקיים חום כמוס, אבל הפוטנציאל החופשי של גיבס נשאר רציף וזהה. כלומר:
$$G_{1}=G_{2}\qquad V_{1}\neq V_{2}\qquad L\neq 0\qquad S_{1}\neq S_{2}$$

\end{definition}
דוגמאות כוללות שינוי מצב צבירה.

\begin{definition}[מעבר פאזה מסדר שני]
הפוטנציאל החופשי של גיבס נשאר רציף, אך אין חום כמוס, אין קפיצה באנטרופיה ואין שינוי בנפח, אבל הנגזרות הם לא רציפות. כלומר:
$$\left( \frac{\partial S}{\partial T}  \right)_{p},\left( \frac{\partial S}{\partial P}  \right)_{T},\left( \frac{\partial V}{\partial T}  \right)_{P},\left( \frac{\partial V}{\partial P}  \right)_{T}$$
הם לא רציפים. זה אומר כי:
$$C_{p}=T\left( \frac{\partial S}{\partial T}  \right)_{p}\qquad \beta=\frac{1}{V}\left( \frac{\partial V}{\partial T}  \right)_{p}\qquad \kappa=\frac{1}{V}\left( \frac{\partial V}{\partial P}  \right)_{T}$$
הם גדלים לא רציפים. 

\end{definition}
מעבר פאזה לדוגמא מסדר שני זה טמפרטורת קירי, חומרים פאראמגנטיים שמגיעים לטמפרטורה מסויימת ומאבדים את הזכרון המגנטי שלהם.

\section{הגישה הסטטיסטית - צברים}

\subsection{אנטרופיה סטטיסטית}

\begin{definition}[מיקרו מצב]
ניתן לחלק את המצבים האפשריים של מערכת לאוסף מצבים שווי הסתברות.

\end{definition}
\begin{definition}[מאקרו מצב]
קונפיגורציה כלשהי של המערכת. לא בהכרח שוות הסתברות.

\end{definition}
\begin{theorem}[העקרון היסודי של פיזיקה סטטיסטית]
כל אחד מהמיקרו מצבים הם שווי הסתברות.

\end{theorem}
כאשר העקרון הזה זה בעצם ההגדרה של מיקרו מצב. ולמעשה נראה כי עקרון זה ניתן לקבל את כל חוקי התרמודינמיקה.

\begin{example}
נסתכל על מערכת עם 100 מטבעות הוגנים. יש \(2^{100}\) מצבים אפשרים עבור כל מטבע. זה יהיה המיקרו מצבים.
כאשר יש 100 קונפיגורציות של המערכת סה"כ - כמה מטבעות הם עץ וכמה הם פאלי. זה יהיה המאקרו מצבים.

\end{example}
\begin{corollary}
  \begin{itemize}
    \item ניתן לתאר מערכת ע"י כמות גדולה של מיקרו מצבים שווי הסתברות.
    \item מה שאנחנו מודדים בפועל זה המאקרו מצבים.
  \end{itemize}
\end{corollary}
\begin{proposition}
התוצאה שתתקבל בפועל תהיה התוצאה המאקרוסקופית עם הכי הרבה מצבים מיקרוסקופים.

\end{proposition}
הרעיון מבוסס על ההנחות הבאות:

\begin{enumerate}
  \item כל אחד מהמיקרו מצבים הם שווי הסתברות. 


  \item הדינמיקה של המערכת פועלת כך שהמערכת משתנה ונמצאת בכל מיקרו מצב כמות שווה של זמן. 


  \item עבור מערכות גדולות, המצב עם הכי הרבה מיקרו מצבים יהיה עם משמעותית הרבה יותר מיקרו מצבים מאשר המערכות האחרות. 


\end{enumerate}
\begin{corollary}
עבור שתי מערכות עם אנרגיה \(E = E_{1}+E_{2}\) החלוקה של אנרגיה שתתקבל תהיה החלוקה שתמקסם את \(\Omega_{1}(E_{1})\Omega_{2}(E_{2})\) כאשר \(\Omega_{1}(E_{1}),\Omega_{2}(E_{2})\) הם כמות המיקרו מצבים המתאימים עבור אנרגיה נתונה.

\end{corollary}
\begin{proposition}
$$\frac{\mathrm{d}\ln\Omega_{1}}{\mathrm{d}E_{1}}=\frac{\mathrm{d}\ln\Omega_{2}}{\mathrm{d}E_{2}}$$

\end{proposition}
\begin{proof}
כיוון שאנו מניחים שאנו נמצאים בגבול התרמודינמי, נמצא מקסימום ע"י גזירה והשוואה ל-0:
$$\frac{\mathrm{d}}{\mathrm{d}E_{1}}\left(\Omega_{1}(E_{1})\Omega_{2}(E_{2})\right)=0 \implies \Omega_{2}(E_{2})\frac{\mathrm{d}\Omega_{1}(E_{1})}{\mathrm{d}E_{1}}+\Omega_{1}(E_{1})\frac{\mathrm{d}\Omega_{2}(E_{2})}{\mathrm{d}E_{2}}\frac{\mathrm{d}E_{2}}{\mathrm{d}E_{1}}=0$$
כאשר נשים לב כי \(dE_{1}=-dE_{2}\) מגזירה של האנרגיה הכוללת. ולכן \(\frac{dE_{2}}{dE_{1}}=-1\) ונקבל:
$$\frac{1}{\Omega_{1}}\frac{\mathrm{d}\Omega_{1}}{\mathrm{d}E_{1}}-\frac{1}{\Omega_{2}}\frac{\mathrm{d}\Omega_{2}}{\mathrm{d}E_{2}}=0 \implies {\frac{\mathrm{d}\ln\Omega_{1}}{\mathrm{d}E_{1}}}={\frac{\mathrm{d}\ln\Omega_{2}}{\mathrm{d}E_{2}}}$$

\end{proof}
כעת נזכור כי הגדרנו את הטמפרטורה של המערכת בתור גודל המאפיין את המערכת ומשתווה בין שתי מערכות המגיעות לשיווי משקל תרמי. כעת ראינו שהגודל שהתקבל הוא גודל כזה, וניתן להגדיר את הטמפרטורה בצורה יותר ריגורוזית:

\begin{definition}[טמפרטורה]
גודל \(T\) המוגדר ע"י המשוואה:
$${\frac{1}{k_{\mathrm{B}}T}}={\frac{\mathrm{d}\ln\Omega}{\mathrm{d}E}}$$
כאשר \(k_{B}\) זה קבוע בולצמן, \(\Omega\) זה מספר המיקרומצבים כתלות באנרגיה, ו-\(E\) זה האנרגיה.

\end{definition}
\begin{proposition}[הגדרה סטטיסטית לאנטרופיה]
$$S=k_{B}\ln \Omega$$
כאשר \(k_{B}\) זה קבוע בולצמן ו-\(\Omega\) זה כמות המיקרו מצבים המתאימים למאקרו מצב הנמדד. 

\end{proposition}
\begin{proof}
נזכור כי:
$$T=\left({\frac{\partial U}{\partial S}}\right)_{V}\implies {\frac{1}{T}}=\left({\frac{\partial S}{\partial U}}\right)_{V}$$
וכן מהגדרת הטמפרטורה נקבל:
$${\frac{1}{k_{\mathrm{B}}T}}={\frac{\mathrm{d}\ln\Omega}{\mathrm{d}E}}\implies \frac{1}{k_{B}T}=\frac{1}{k_{B}}\left( \frac{\partial S}{\partial U} \right)_{V}=\frac{d\ln \Omega}{dE} $$
כאשר עבור \(E=U\) המשוואה מתקיימת עבור \(S=k_{B}\ln \Omega\).

\end{proof}
\begin{example}[מערכת 2 רמות]
נניח מערכת חליקים שהם יכולים להיות או \(\uparrow\) או \(\downarrow\). כאשר אם הם מצביעים למעלה האנרגיה שלהם היא \(u_{\uparrow}=\mathcal{E}\) ואם הם מצביעים למטה הם מקיימים \(u_{\downarrow}=0\) כאשר:
$$U=U(T)=N_{\uparrow}\cdot\mathcal{E}$$
כאשר סך החלקיקים יהיה \(N=N_{\uparrow}+N_{\downarrow}\). כאשר סך המצבים יהיה:
$$\Omega\left(N_{\uparrow};N\right)={\binom{N}{N_{\uparrow}}}=\frac{N!}{N_{\uparrow}!\left(N-N_{\uparrow}\right)!}\;.$$
כאשר אם נשתמש בקירוב סטרילינג \(\ln(n!)\approx n\ln(n)-n\). נקבל:
$$\frac{S}{k_{B}}=\ln \Omega \approx N\ln\left(N\right)-N_{\uparrow}\ln\left(N_{\uparrow}\right)-\left(N-N_{\uparrow}\right)\ln\left(N-N_{\uparrow}\right)$$
נשתמש במשוואה היסודית
$$dS = \frac{1}{T}dU+ \frac{P}{T}dV=\left( \frac{\partial S}{\partial U}  \right)_{V,N}du+\left( \frac{\partial S}{\partial V}  \right)_{U,N}\implies \frac{1}{T}=\left( \frac{\partial S}{\partial U}  \right)_{V,N}\quad \frac{P}{T}=\left( \frac{\partial S}{\partial V}  \right)_{U,N}$$
ונקבל:
$$S\left(U,N\right)=k_{\mathrm{B}}\left[N\ln\left(N\right)-\left(N-\frac{U}{\mathcal{E}}\right)\ln\left(N-\frac{U}{\mathcal{E}}\right)-\frac{U}{\mathcal{E}}\ln\left(\frac{U}{\mathcal{E}}\right)\right]$$
ונשתמש בקשר:
$$
{\frac{1}{T}}=\left({\frac{\partial S}{\partial U}}\right)_{N}={\frac{1}{\mathcal{E}}}\left({\frac{\partial S}{\partial\left(U/\mathcal{E}\right)}}\right)_{N}={\frac{k_{\mathrm{B}}}{\mathcal{E}}}\left[\ln\left(N-{\frac{U}{\mathcal{E}}}\right)-\ln\left({\frac{U}{\mathcal{E}}}\right)\right]={\frac{k_{\mathrm{B}}}{\mathcal{E}}}\ln\left({\frac{N-{\frac{U}{\mathcal{E}}}}{{\frac{U}{\mathcal{E}}}}}\right)_{N}={\frac{1}{\mathcal{E}}}\ln\left({\frac{N-{\frac{U}{\mathcal{E}}}}{{\frac{U}{\mathcal{E}}}}}\right)$$
נבודד את \(U\) ונקבל:
$$U\left(T\right)=\frac{N\mathcal{E}}{1+e^{\frac{\mathcal{E}}{k_{\mathrm{B}}T}}}$$

\end{example}
\subsection{המערכת המבודדת - הצבר המיקרוקנוני}

\begin{definition}[צבר מיקרוקנוני]
המערכת המבודדת. זוהי מערכת שבה האנרגיה, מספר החלקיקים והנפח נשארים קבועים.

\end{definition}
\begin{proposition}
אם אין אינטרקציה בין החלקיקים האנרגיה הפוטנציאלית תהיה אפס.

\end{proposition}
\begin{corollary}
סך האנרגיה של מערכת ללא אינטרקציה בין החלקיקים תהיה:
$$E=\sum_{i=1}^{3N}\frac{p_{i}^{2}}{2m}.$$

\end{corollary}
\begin{reminder}
נפח של כדור \(N\) מימדי יהיה:
$$V_{N}(R)=\frac{\pi^{N/2}}{\Gamma\left(\frac{N}{2}+1\right)}R^{N}$$
כאשר עבור אליפסויד הנפח יהיה:
$$V_{N}=\frac{2}{N}\frac{\pi^{N/2}}{\Gamma(N/2)}(a_{1}a_{2}a_{3}\ldots a_{N})$$
כאשר \(a_{1},\dots,a_{n}\) מייצגות את אורך החציר(חצי ציר ראשי).

\end{reminder}
\begin{definition}[הפונקציה הצוברת של מספר המצבים]
מספר המיקרו מצבים עד האנרגיה \(E\). מסומן ב-\(\Gamma=\Gamma(E)\).

\end{definition}
\begin{proposition}
אם נחלק את המרחב פאזה לנפחים קטנים כך שיש סיכוי שווה שיהיה מצב בכל יחידת נפח נקבל כי כל המיקרומצבים האפשריים יהיו המקירומצבים עם אנרגיה קטנה מ-\(E\). זה יהיה אוסף כל ה-\(p_{i}\) אשר מקיימים:
$$E\,\geq\,\sum_{i=1}^{3N}\frac{p_{i}^{2}}{2m}\implies 2mE \geq \sum_{i=1}^{3N} p_{i}^{2}$$
כלומר זה יהיה כדור \(3N\) מימדי עם רדיוס \(\sqrt{ 2mE }\).

\end{proposition}
\begin{remark}
הגודל של הקוביות שאנחנו מחלקים הוא שרירותי, ואין לו חשיבות במכניקה סטטיסטית קלאסית, אך בדרך כלל כדי שיהיה זהה לתוצאות במכניקה סטטיסטית קוונטית נבחר קוביות קטנות בנפח \(h\) כיוון שזה הנפח הבסיסי לפי העקרון האי וודואות הקוונטי:
$$\Delta x\Delta p_{x}\geq h.$$

\end{remark}
\begin{proposition}
במרחב הפאזה אם אנחנו מניחים שמורכב מקוביות שוות הסתברות נקבל כי:
$$\Gamma\left(E\right)={\frac{\text{Volume in phase space up to energy }E}{\text{Volume in phase space of a single state}}}=\frac{\int_{U\left( \mathbf{p},\mathbf{x} \right)\leq E}\mathrm{d}\mathbf{p}\mathrm{d}\mathbf{x}}{h^{d}} $$
כאשר הנפח של קובייה יחידה תהיה \(h^{d}\). והגבולות נקבעות לפי הדרישה שהאנרגיה המקסילית היא \(E\).

\end{proposition}
\begin{example}[פונקציה צוברת של המצבים עבור גז אידיאלי]
עבור גז אידיאלי נקבל כי \(\mathcal{H}= \frac{p^{2}}{2m}\). נסכום על כל הקוביות במרחב פאזה כאשר הגבולות שלנו נקבעות לפי האנרגיה המקסימלית:
$$E_{\max }=\frac{p_{\max }^{2}}{2m}\implies p_{\max }^{2}=2E_{\max }m\implies p\leq \sqrt{ 2E_{\max }m }$$
נשים לב כי זה לעשה מייצג נפח של כדור תלת מידי במרחב הפאזה. בנוסף אפשר לראות כי האנרגיה אינה תלויה במיקום ולכן אין אילוץ כלל על המקום \(\mathbf{x}\). באופן כללי נדרש לבצע אינטגרל על המרחב כולו אבל כדי שהאינטגרל לא יתבדר נדרוש כי הנפח של המיכל הוא \(V\), ולכן:
$$\Gamma(E)=\frac{1}{h^{3}}\int_{0}^{\sqrt{ 2Em }}  \, d\vec{p} \int_{V}  \, d\vec{x}=  \frac{V}{h^{3}}V_{\text{Sphere}}=\frac{V}{h^{3}} \frac{4\pi}{3}(2Em)^{3/2}$$

\end{example}
\begin{corollary}
מספר המיקרו מצבים המתאימים עבור אנרגיה \(E\) נפח \(V\) ומספר חלקיקים \(N\) יהיה:
$$\Gamma(E,V,N)=\frac{V^{N}}{h^{3N}N!}\frac{\left( 2\pi m E \right)^{3N/2}}{\Gamma\left(\frac{3N}{2}+1\right)}$$
כאשר חלקנו ב-\(N!\) כיוון שהחלקיקים לא מובחנים. כלומר אם נחליף בין שתי חלקיקים זה יהיה עדיין אותו מצב. בלי חלוקה זו האנטרופיה לא תהיה גודל אקסטנסיבי.

\end{corollary}
\begin{proposition}
מספר המצבים בין רמת אנרגיה \(E\) לרמת אנרגיה \(E+\Delta E\) לא תלויה -\(\Delta E\) בגבול התרמודינמי, כלומר כאשר דרגות החופש שואפות לאינסוף.

\end{proposition}
\begin{proof}
השינוי בכמות המצבים כתוצאה בשינוי קטן באנרגיה יהיה שקול לשינוי שנפח של כדור \(N\) מימדי עבור שינוי קטן ברדיוס. אנחנו רוצים להסתכל על הצפיפות של המצבים ולכן נסתכל על היחס של הקלפה הקטנה לעומת הנפח הכולל כאשר מספר המימדים יהיה גדול.
$$\frac{V_{N}(R)-V_{N}\left( R-\Delta R \right)}{V_{N}(R)}=\frac{R^{N}-\left( R-\Delta R \right)^{N}}{R^{N}}= 1-\left( 1-\frac{\Delta R}{R} \right)^{N}\xrightarrow{N\gg 1} 1$$

\end{proof}
\begin{definition}[צפיפות המצבים]
מספר המיקרו מצבים בין אנרגיה \(E\) לאנרגיה \(E+dE\). מקיים:
$$\Omega\left(N,V,E\right)=\Gamma\left(N,V,E\right)-\Gamma\left(N,V,E-d E\right)\equiv g\left(N,V,E\right)d E$$
כאשר צפיפות המצבים היא פונקציה \(g\) כך שמתקיים:
$$g(E)=\frac{\partial \Gamma}{\partial E} $$

\end{definition}
\begin{corollary}
צפיפות המצבים היא תהיה הפונקציה \(g(E)\) אשר מקיימת:
$$\Gamma(E)=\int_{0}^{E}g\left( E^{\prime} \right)d E^{\prime}$$

\end{corollary}
\begin{corollary}
עבור המרחב הפאזה של המיקום ותנע נקבל מהתוצאה הקודמת על הפונקציה הצוברת את הצפיפות מצבים הבאה:
$$g(E,V,N)=\frac{\partial \Gamma}{\partial E} =\frac{V^{N}}{h^{3N}N!}\frac{\left( 2\pi m \right)^{3N/2}}{\Gamma\left( \frac{3N}{2}+1 \right)}\frac{3N}{2}E^{(3N/2)-1}$$

\end{corollary}
\begin{example}
עבור חלקיק יחיד נקבל:
$$g(E,V)=\frac{V}{h^{3}}\frac{\pi}{4}(8m)^{3/2}E^{1/2}$$

\end{example}
\begin{proposition}
בעזרת קירוב סטרלינג נקבל כי האנטרופיה של גז אידיאלי תהיה:
$$S(E,V,N)=N k_{B}\left[\ln\left(\frac{V}{N}\right)+\frac{3}{2}\ln\left(\frac{E}{N}\right)+\frac{3}{2}\ln\left(\frac{4\pi m}{3h^{2}}\right)+\frac{5}{2}\right]$$

\end{proposition}
\begin{proposition}
עבור גז אידיאלי נקבל כי הטמפרטורה תהיה נתונה על ידי:
$$T(E,V,N)={\frac{2E}{3N k_{B}}}$$

\end{proposition}
\begin{proof}
נובע ישירות מהגדרת האנטרופיה:
$$\left({\frac{\partial S}{\partial E}}\right)_{V,N}\!\!\!={\frac{1}{T}}={\frac{3N k_{B}}{2E}}$$
ועל ידי העברת אגפים נקבל את המבוקש.

\end{proof}
\begin{corollary}[חוק החלוקה השווה]
$$E=3N\left({\frac{1}{2}}k_{B}T\right)$$
כאשר כל גורם ריבועי בהמילטוניאן למעשה יתן אנרגיה של \(\frac{k_{B}T}{2}\). וכן בפרט עבור גז אידיאלי יש \(3N\) גורמים ריבועיים.

\end{corollary}
\begin{proposition}
לחץ של מערכת מבודדת של גז אידיאלי תהיה נתונה על ידי:
$$P={\frac{N k_{B}T}{V}} = \frac{3E}{2V}$$

\end{proposition}
\begin{proof}
$$\left(\frac{\partial S}{\partial V}\right)_{E,N}=\frac{P}{T}=\frac{N k_{B}}{V}$$
וכן מחוק החלוקה השוואה ניתן לקבל מזה:
$$P=\frac{3E}{2V}$$

\end{proof}
\begin{example}[מגנט מבודד]
נמצא מה המגנטיזציה כתלות בשדה החיצונית \(\vec{B}\). מומנט מגנטי:
$$E=-\vec{m}\cdot \vec{B} = -m_{z}{B}$$
כאשר \(m_{z}\) נתון על ידי:
$$m_{z}= \pm \frac{1}{2}g \mu$$
כאשר קבוע בור \(\mu= \frac{|e|\hbar}{2m}\)(Bohr magnetism) והפקטור \(g\approx 2\) (g-factor). ולכן \(m_{z}\approx \pm \mu\).
עבור ספין \(\uparrow_{z}\) נקבל \(m_{z}=-\mu\) ויהיה עם אנרגיה \(E=\mu B\) כאשר עבור ספין \(\downarrow_{z}\) נקבל \(m_{z}=\mu\) ויהיה עם אנרגיה \(E=-\mu B\).
נסמן ב-\(N_{\uparrow}\) את כמות הספינים כלפי מעלה, וב-\(N_{\downarrow}\) את כמות הספינים כלי מטה, כך שמתקיים \(N=N_{\uparrow}+N_{\downarrow}\). כעת נגדיר:
$$\tilde{m}=\frac{N_{\uparrow }-N_{\downarrow }}{N_{\uparrow }+N_{\downarrow }}\implies \begin{cases}E=\mu B N \cdot \tilde{m} \\\frac{M}{N}=-\mu \cdot \tilde{m}
\end{cases}$$
לפי בולצמן נדרש לחשב את הכפליות.
$$\Omega\left( N_{\uparrow },N_{\downarrow } \right)=\frac{N!}{N_{\uparrow }!N_{\downarrow }!}$$
כאשר נשתמש כעת בקירוב סטרלינג:
$$\begin{gather}\log\left( \Omega \right)=\log(N!)-\log\left( N_{\uparrow }! \right) - \log\left( N_{\downarrow }! \right)= \\= N \log N - N -N_{\uparrow }\log\left( N_{\uparrow } \right)+N_{\uparrow }-N_{\downarrow }\log\left( N_{\downarrow } \right)+N_{\downarrow }=  \\=\left( N_{\uparrow }+N_{\downarrow } \right) \log N -N_{\uparrow }\log\left( N_{\uparrow } \right)-N_{\downarrow }\log\left( N_{\downarrow } \right)= \\=N_{\uparrow }\log\left( \frac{N}{N_{\uparrow }} \right)+N_{\downarrow }\log\left( \frac{N}{N_{\downarrow }} \right)
\end{gather}$$
כאשר כעת:
$$\tilde{m}= \frac{N_{\uparrow }-N_{\downarrow }}{N_{\uparrow }+N_{\downarrow }}\implies \begin{cases} \tilde{m}N = N_{\uparrow }-N_{\downarrow } \\N= N_{\uparrow }+N_{\downarrow }\end{cases}\implies  \begin{cases}N_{\uparrow }= \frac{\tilde{m} + 1}{2} N \\N_{\downarrow }= \frac{1 - \tilde{m}}{2} N
\end{cases}$$
ולכן נקבל כי כפליות המצב תהיה:
$$\log\left( \Omega \right)=-N_{\uparrow }\log\left( \frac{1+\tilde{m}}{2} \right)-N_{\downarrow }\log\left( \frac{1-\tilde{m}}{2} \right)=-N\left[ \frac{\tilde{m}+1}{2}\log\left( \frac{\tilde{m}+1}{2} \right)+\frac{1-\tilde{m}}{2} \log\left( \frac{1-\tilde{m}}{2} \right)\right]$$
כאשר האנטרופיה לחלקיק תהיה:
$$\frac{S}{N}=\frac{k_{B}\log\left( \Omega \right)}{N}$$
כאשר לדוגמא עבור \(\tilde{m}=0\) נקבל:
$$\frac{S}{N}\left( \tilde{m} = 0 \right)=k_{B}\left( -\frac{1}{2}\log\left( \frac{1}{2} \right)-\frac{1}{2}\log\left( \frac{1}{2} \right) \right)=k_{B}\log(2)$$
כעת:
$$\frac{S\left( \tilde{m} \right)}{Nk_{B}}=-\frac{\tilde{m}+1}{2}\log\left( \frac{1+\tilde{m}}{2} \right)- \frac{1-\tilde{m}}{2}\log\left( \frac{1-\tilde{m}}{2} \right)$$
כאשר כעת:
$$S(E,V,N,B)=N k_{B} f\left( \tilde{m} \right)$$
כאשר:
$$f\left( \tilde{m} \right)= - \frac{1+\tilde{m}}{2}\log\left( \frac{1+\tilde{m}}{2} \right)-\frac{1-\tilde{m}}{2}\log\left( \frac{1-\tilde{m}}{2} \right)$$
ניתן למצוא את הטמפרטורה:
$$\frac{1}{T}=\left.\frac{\partial S}{\partial E}\right|_{V,B} $$
כאשר כיוון ש-\(E=\mu B\left( N_{\uparrow}-N_{\downarrow} \right)=\mu BN\tilde{m}\) נקבל:
$$\frac{1}{T} = \frac{\partial  S}{\partial \tilde{m}} \frac{\partial \tilde{m}}{\partial E} = Nk_{B}\frac{\mathrm{d} f\left( \tilde{m} \right)}{\mathrm{d} \tilde{m}}  \frac{1}{\mu B N}=\frac{k_{B}}{\mu B}\frac{\mathrm{d} f}{\mathrm{d} \tilde{m}} $$
כאשר מתקיים:
$$\frac{\mathrm{d} f}{\mathrm{d} \tilde{m}} =-\frac{\mathrm{d} }{\mathrm{d} \tilde{m}} \left(\frac{ 1-\tilde{m}}{2}\log\left( \frac{1-\tilde{m}}{2} \right)+\frac{1+\tilde{m}}{2}\log\left( \frac{1+\tilde{m}}{2} \right) \right)=\frac{1}{2}\log\left( \frac{1-\tilde{m} }{1+\tilde{m}}\right)$$

\end{example}
\begin{summary}
  \begin{itemize}
    \item הצבר המיקרוקנוני הוא מערכת מבודדת אשר מתאפיינת בכך שהאנרגיה, הנפח ומספר החלקיקים קבועים.
    \item המשתנה המרכזי בצבר זה הוא האנטרופיה, כאשר אם מוצאים את האנטרופיה ניתן למצוא הכל.
    \item מוצאים את האנטריפיה על ידי ספירת מצבים ושימוש באנטרופיית בולצמן עם קירוב סטרלינג.
    \item ניתן לספור את המצבים על ידי מציאת צפיפות המצבים ואז \({\Omega}(E)=\int_{0}^{E}g(E^{\prime})d E^{\prime}\)
    \item האנטרופיה מקיימת \(dS= \frac{1}{T}dE+\frac{p}{T}dV\) ולכן:
$$\frac{1}{T}=\left. \frac{\partial S}{\partial E} \right \rvert_{V} \qquad \frac{p}{T}=\left. \frac{\partial S}{\partial V} \right \rvert_{E}$$
  \end{itemize}
\end{summary}
\subsection{המערכת הסגורה - הצבר הקנוני}

\begin{definition}[הצבר הקנוני]
מאגר חום עם אנרגיה \(E_{R}\) המחובר למערכת עם אנרגיה \(E_{A}\) כך ש-\(E_{R}\gg E_{A}\) והאנרגיה הכוללת \(E_{0}=E_{A}+E_{R}\) היא קבועה.

\end{definition}
\begin{proposition}[התפלגות הההסתברות של הצבר הקנוני]
ההסתברות של מצב להיות באנרגיה \(E_{A}\) יהיה פרופורציונאלי ל-\(e^{ -E_{A}/k_{B}T }\)

\end{proposition}
\begin{remark}
התפלגות זו מכונה התפלגות בולצמן.

\end{remark}
\begin{proof}
אנו יודעים כי \(S(E_{R})=S(E_{0}-E_{A})\) כאשר כיוון ש\(E_{R}\gg E_{A}\) נקבל כי ניתן לפתח בעזרת טור טיילור ולקבל:
$$S(E_{\mathrm{R}})=S(E_{0})-E_{\mathrm{A}}\frac{\partial S}{\partial E}+\cdots$$
כאשר ניתן להזניח גורמים מסדר גבוה ולהציב \(\frac{1}{T}=\frac{\partial S}{\partial E}\). נקבל:
$$S(E_{\mathrm{R}})=S(E_{0})-{\frac{E_{\mathrm{A}}}{T}}$$
כאשר אנו יודעים כי \(S=k_{B}\ln \Omega\) ולכן ההסתברות שיהיה במצב \(E_{R}\) תהיה פרופורציונית לכמות המיקרו מצבים \(\Omega(E_{R})\) ולכן:
$$P(E_{R})\propto e^{S(E_R)/k_{B}}\approx e^{S(E_{0})-E_{A}/k_{B}T}=e^{ S(E_{0}) }e^{ -E_{A}/k_{B}T }$$
כלומר נקבל כי ההסתברות ברופורצינאלית ל-\(e^{ -E_{A}/k_{B}T }\).

\end{proof}
\begin{definition}[ניוון]
כמות המיקרו מצבים עם רמת אנרגיה \(E_{i}\). לכרגע נסמן את זה ב-\(g_{i}\).

\end{definition}
\begin{definition}[פונקצייית החלוקה]
הפקטור נרמול של ההסתברות:
$$Z=\sum_{i}g_{i}e^{-E_{i}/k_{\mathrm{{B}}}T}$$
כאשר:
$$ P(E_{i})={\frac{1}{Z}}g_{i}e^{-E_{i}/k_{\mathrm{B}}T}$$

\end{definition}
\begin{remark}
למעשה פונקציית החלוקה היה תהיה התמרת לפלס של פונקציית מספר המצבים.

\end{remark}
\begin{example}[מערכת שתי רמות]
עבור מערכת שתי רמות, נניח והמערכת יכולה להיות במצב \(-\frac{\Delta}{2}\) או \(\frac{\Delta}{2}\). מתקיים:
$$Z=\sum_{\alpha}\mathrm{e}^{-\beta E_{\alpha}}=\mathrm{e}^{\beta\Delta/2}+\mathrm{e}^{-\beta\Delta/2}=2\cosh\left({\frac{\beta\Delta}{2}}\right)$$

\end{example}
\begin{example}[אוסצילטור הרמוני קוונטי]
האנרגיות האפשריות של המערכת יהיו מהצורה \(\left( n+\frac{1}{2} \right)\hbar \omega\) ולכן:
$$Z=\sum_{\alpha}\mathrm{e}^{-\beta E_{\alpha}}=\sum_{n=0}^{\infty}\mathrm{e}^{-\beta\left( n+\frac{1}{2} \right)\hbar\omega}=\mathrm{e}^{-\beta\frac{1}{2}\hbar\omega}\sum_{n=0}^{\infty}\mathrm{e}^{-n\beta\hbar\omega}=\frac{\mathrm{e}^{-\frac{1}{2}\beta\hbar\omega}}{1-\mathrm{e}^{-\beta\hbar\omega}}=\frac{2}{\sinh\left( \frac{1}{2}\beta\hbar \omega \right)}$$

\end{example}
\begin{example}[מערכת של \(N\) רמות]
נסמן את רמות האנרגיה ב-\(0,\hbar {\omega},\dots,(N-1)\hbar \omega\) ונקבל:
$$Z=\sum_{\alpha}\mathrm{e}^{-\beta E_{\alpha}}=\sum_{j=0}^{N-1}\mathrm{e}^{-j\beta\hbar\omega}=\frac{1-\mathrm{e}^{-N\beta\hbar\omega}}{1-\mathrm{e}^{-\beta\hbar\omega}}$$

\end{example}
\begin{proposition}[אנרגיה פנימית בעזרת התפלגות בולצמן]
$$U=-\frac{\mathrm{d} \ln Z}{\mathrm{d} \beta} =k_{\mathrm{B}}T^{2}{\frac{\mathrm{d}\ln Z}{\mathrm{d}T}}$$

\end{proposition}
\begin{proof}
אם נגדיר \(\beta=\frac{1}{k_{B}T}\) נקבל כי האנרגיה הפנימית תהיה שווה ל:
$$U=\sum_{i}E_{i}P(E_{i})=\frac{\sum_{i}E_{i}\mathrm{e}^{-\beta E_{i}}}{\sum_{i}\mathrm{e}^{-\beta E_{i}}}$$
כאשר המכנה זה פונקציית החלוקה, ולכן מתקיים:
$$-{\frac{\mathrm{d}Z}{\mathrm{d}\beta}}=\sum_{i}E_{i}\mathrm{e}^{-\beta E_{i}}$$
ולכן:
$$U=-\frac{1}{Z}\left( \frac{\mathrm{d} Z}{\mathrm{d} \beta}  \right)\implies U=-\frac{\mathrm{d} \ln Z}{\mathrm{d} \beta} \implies U=k_{\mathrm{B}}T^{2}{\frac{\mathrm{d}\ln Z}{\mathrm{d}T}}$$

\end{proof}
\begin{proposition}[אנטרופיה בעזרת פונקציית החלוקה]
$$S={\frac{U}{T}}+k_{\mathrm{B}}\ln Z$$

\end{proposition}
\begin{proof}
כמו מקודם נגדיר \(\beta = \frac{1}{k_{B}T}\). אנו יודעים כי \(P_{j}=\frac{e^{ -\beta E_{j} }}{Z}\) ולכן:
$$\ln{ P}_{j}=-\beta E_{j}-\ln Z$$
וכן מתקיים \(S=-k_{\mathrm{B}}\sum_{i}P_{i}\ln P_{i}\). ולכן נקבל:
$$\begin{gather}{{S}}={{-k_{\mathrm{B}}\sum_{i}P_{i}\ln P_{i}}}= k_{\mathrm{B}}\sum_{i}P_{i}\left( \beta E_{i}+\ln Z \right)= k_{B}\left( \beta \sum_{i}P_{i}E_{i}+\sum_{i}P_{i}\ln Z \right)
\end{gather}$$
כאשר כיוון ש-הסכום של ההסתברות להיות הרמת אנרגיה כפול האנרגיה ברמה תהיה סך האנרגיה נקבל:
$$k_{\mathrm{{B}}}\left( \beta U+\frac{1}{Z} \ln Z\sum e^{ -\beta_{i}E_{i} } \right) =k_{B}\left( \beta U+ \frac{Z}{Z}\ln Z \right)$$
כאשר כתבנו בסכום השני את \(P_{i}\) מפורשות, הוצאנו את \(Z\) ו-\(\ln Z\) מהסכום וקיבלנו מהסכום את פונקציית החלוקה. נציב חזרה את \(\beta\) ונקבל:
$$S={\frac{U}{T}}+k_{\mathrm{B}}\ln Z$$

\end{proof}
\begin{proposition}[אנרגיה חופשית של הלמהולץ בעזרת פונקציית החלוקה]
$$F=-k_{\mathrm{B}}T\ln Z$$
ולכן גם ניתן לכתוב:
$$Z=e^{ -F/k_{B}T }$$

\end{proposition}
\begin{proof}
נובע מההגדרה \(F=  U-  T  S\) עם הביטוי לאנטרופיה מהטענה הקודמת.

\end{proof}
\begin{corollary}
ניתן בעזרת הגדרה זו להשיג ביטוי של הרבה גדלים אחרים בעזרת הפונקציית חלוקה. למשל:
$$\begin{gather}p=-\left({\frac{\partial F}{\partial V}}\right)_{T}=k_{\mathrm{B}}T\left({\frac{\partial\mathrm{ln}\,Z}{\partial V}}\right)_{T}  \\H=U+p V=k_{\mathrm{{B}}}T\left[T\left({\frac{\partial\mathrm{ln}\,Z}{\partial T}}\right)_{V}+V\left({\frac{\partial\mathrm{ln}\,Z}{\partial V}}\right)_{T}\right] \\G=F+p V=k_{\mathrm{{B}}}T\left[-\ln Z+V\left({\frac{\partial\ln Z}{\partial V}}\right)_{T}\right] \\C_{V}=k_{\mathrm{{B}}}T\left[2\left(\frac{\partial\mathrm{ln}\,Z}{\partial T}\right)_{V}+T\left(\frac{\partial^{2}\mathrm{ln}\,Z}{\partial T^{2}}\right)_{V}\right]
\end{gather}$$

\end{corollary}
\begin{example}
עבור מערכת שתי רמות(עם אנרגיות \(\frac{\Delta}{2}\) ו-\(-\frac{\Delta}{2}\)) ראינו כי פונקציית החלוקה היא מהצורה:
$$Z=2\cosh\left(\frac{\beta\Delta}{2}\right)$$
ולכן האנרגיה הפנימית תהיה:
$$U=-\frac{\mathrm{d}\ln Z}{\mathrm{d}\beta}=-\frac{\Delta}{2}\operatorname{tanh}\left(\frac{\beta\Delta}{2}\right).$$
קיבול החום יהיה:
$$C_{V}=\left({\frac{\partial U}{\partial T}}\right)_{V}=k_{\mathrm{B}}\left({\frac{\beta\Delta}{2}}\right)^{2}\mathrm{sech}^{2}\left({\frac{\beta\Delta}{2}}\right)$$
פונקציית הלמהולדס תהיה:
$$F=-k_{\mathrm{B}}T\ln Z=-k_{\mathrm{B}}T\ln\left[2\cosh\left({\frac{\beta\Delta}{2}}\right)\right]$$
ותן לקבל מזה ישירות את האנטרופיה:
$$S=\frac{U-F}{T}=-\frac{\Delta}{2T}\operatorname{tanh}\left(\frac{\beta\Delta}{2}\right)+k_{\mathrm{B}}\ln\left[2\cosh\left(\frac{\beta\Delta}{2}\right)\right]$$
כלומר מפונקציית החלוקה ניתן לקבל את כל הגדלים של המערכת!

\end{example}
\begin{proposition}[איחוד פונקציות חלוקה]
אם האנרגיה \(E\) היא סכום של שתי תרומות \(E^{(a)}\) ו-\(E^{(b)}\) כך שהרמת אנרגיה \(E_{ij}\) נתונה על ידי:
$$E_{i,j}=E_{i}^{(a)}+E_{j}^{(b)}$$
מתקיים:
$$Z=Z_{a}Z_{b}$$

\end{proposition}
\begin{proof}
$$Z=\sum_{i}\sum_{j}\mathrm{e}^{-\beta(E_{i}^{(a)}+E_{j}^{(b)})}=\sum_{i}\mathrm{e}^{-\beta E_{i}^{(a)}}\sum_{j}\mathrm{e}^{-\beta E_{j}^{(b)}}=Z_{a}Z_{b}$$

\end{proof}
\begin{corollary}
פונקציית החלוקה של רכיבים בלתי תלויים מכפילים אחד בשני, ולכן בפרט \(\ln Z\) עבור רכיבים בלתי תלויים נסכמים אחד עם השני(\(\ln(Z)=\ln(Z_{a})+\ln(Z_{b})\)).

\end{corollary}
\subsection{התפלגות בולצמן במרחב הפאזה}

\begin{proposition}[התפלגות הסתברות במרחב הפאזה]
$$P(p,q)=\frac{1}{\tilde{Z}(T,V,N)}\exp[-\beta H(p,q)]$$
כאשר מתנאי הנרמול:
$$\tilde{Z}(T,V,N)=\int\!\!d q\int\!\!d p\,\exp[-\beta H(q,p)]$$

\end{proposition}
\begin{proof}
מתקיים:
$$P(p,q)=\frac{\Omega_{R}(E_{T}-H(p,q))}{\Omega_{T}(E_{T})}$$
כאשר כיוון ש-\(E_{T}\gg H(p,q)\) ניתן לקחת את הלוגריתם של שתי הצדדים ולפתח את \(\ln\left( \Omega_{R} \right)\) על ידי חזקות של \(\frac{H(p,q)}{E_{T}}\) ולקבל:
$$\ln P(p,q)=\ln\Omega_{R}(E_{T})-H(p,q)\frac{\partial}{\partial E_{T}}\Omega_{R}(E_{T})-\ln\Omega_{T}(E_{T})+\cdots$$
כאשר נזכור כי:
$$\beta=\beta_{R}\equiv\frac{\partial}{\partial E_{T}}\Omega_{R}(E_{T})$$
כאשר רק האיבר השני תלוי ב-\(p,q\) ולכן ניתן לאחד את יתר הגורמים בפונקציה \(\tilde{Z}\):
$$\ln P(p,q)=-\beta H(p,q)-\ln\tilde{Z}(T,V,N)$$
כלומר:
$$P(p,q)=\frac{1}{\tilde{Z}(T,V,N)}\exp[-\beta H(p,q)]$$

\end{proof}
\begin{reminder}
המשוואה הכללית של מספר המצבים במרחב פאזה הוא:
$$\Omega(E,V,N)=\frac{1}{h^{3N}N!}\int\!\!d q\int\!\!d p\,\delta(E-H(q,p))$$

\end{reminder}
\begin{corollary}
$$\widetilde{Z}=\,\frac{1}{h^{3N}N!}\int\!\!d q\int\!\!d p\,\exp[-\beta H(q,p)]$$

\end{corollary}
\begin{proof}
$$\begin{gather}\widetilde{Z}=\int\!\!d E\frac{1}{h^{3N}N!}\int\!\!d q\int\!\!d p\,\delta(E-H(q,p))\exp\left( -\beta E \right)=\\=\frac{1}{h^{3N}N!}\int\!\!d q\int\!\!d p\,\int\!\!d E\,\delta(E-H(q,p))\exp\left( -\beta E \right)=\\=\frac{1}{h^{3N}N!}\int\!\!d q\int\!\!d p\,\exp\left[ -\beta H(q,p) \right] 
\end{gather}$$

\end{proof}
\begin{corollary}
אם נשוואה להתפלגות בולצמן נקבל:
$$\tilde{Z}(T,V,N)=h^{3N}N!\,Z$$
ולכן ההתפלגות במרחב הפאזה נתונה על ידי:
$$P(p,q)=\frac{1}{h^{3N}N!\,Z}\exp[-\beta H(p,q)]$$

\end{corollary}
\begin{summary}
  \begin{itemize}
    \item מערכת שלא מחליפה חלקיקים עם הסביבה אבל כן מחליפה אנרגיה מכונה הצבר הקנוני. כאשר לרוב נבצע אידיאליזציה שבה הסביבה היא אמבט חום.
    \item ההסתברות להיות באנרגיה \(E\) נתונה על ידי התפלגות בולצמן \(P(E)=\frac{1}{Z}\Omega(E)\exp(-\beta E)\) כאשר \(Z\) נקראת פונקציית החלוקה.
    \item בעזרת פונקציית החלוקה ניתן לקבל את האנרגיה החופשית של הלמהולץ, האנטרופיה והאנרגיה הפנימית בצורה הבאה:
$$U=k_{\mathrm{B}}T^{2}{\frac{\mathrm{d}\ln Z}{\mathrm{d}T}} \qquad F=-k_{\mathrm{B}}T\ln Z \qquad S={\frac{U}{T}}+k_{\mathrm{B}}\ln Z
$$
    \item אם יש רכיבים שונים(בלתי תלויים) לאנרגיה הפונקצייה החלוקה הכוללת תהיה המכפלה של כל אחד מהם.
    \item במרחב הפאזה ההסתברות להיות בתנע \(p\) ומיקום \(q\) יהיה נתון על ידי:
$$P(p,q)=\frac{1}{h^{3N}N!\,Z}\exp[-\beta H(p,q)]$$
  \end{itemize}
\end{summary}
\subsection{פיסיקה סטטיסטית של גזים}

\begin{definition}[חלקיקים ניתנים להבחנה]
אם ניתן להחליף שתי חלקיקים ובלי לשנות את המיקרו מצב נקבל כי החלקיקים לא ניתנים להבחנה. כאשר אם החלפה של שתי חליקיקים תשנה את המיקרו מצב חלקיקים אלו יהיו ניתנים להבחנה.

\end{definition}
\begin{proposition}
עבור \(N\) חלקיקים אם כל החלקיקים במערכת ניתנים להבחנה נקבל כי:
$$Z_{N}=(Z_{1})^{N}$$
כאשר \(Z_{N}\) היא פונקציית החלוקה של \(N\) חלקיקים ו-\(Z_{1}\) היא פונקציית החלוקה של חלקיק אחד.
אם לא ניתנים להבחנה, נקבל:
$$Z_{N}=\frac{(Z_{1})^{N}}{N!}$$

\end{proposition}
\begin{proof}
אם כל החלקיקים במערכת ניתנים להבחנה נקבל כי \(E_{tot}=\sum E_{i}\) עבור כל חלקיק. ולכן מהטענה הקודמת עבור \(N\) חלקיקים נקבל:
$$Z_{N}=(Z_{1})^{N}$$
אם לא ניתנים להבחנה, אז יש חפיפה בין האנרגיות! לכן נדרש לחלק ב-\(N!\).

\end{proof}
\begin{proposition}[פונקציית החלוקה של גז אידיאלי של חלקיק יחיד]
$$Z_{1}=\frac{V}{\lambda_{\mathrm{T}}^{3}} \qquad \lambda_{\mathrm{T}}=\frac{h}{\sqrt{2\pi m k_{\mathrm{B}}T}}$$

\end{proposition}
\begin{proof}
נסמן \(k=\frac{p}{\hbar}\) התדר מרחבי. אנו יודעים כי פונקציית החלוקה במרחב התדר המרחבי יהיה:
$$Z_{1}=\int_{0}^{\infty}\mathrm{e}^{-\beta E(k)}\,g(k)\,\mathrm{d}k$$
כאשר אנו יודעים כי האנרגיה תהיה:
$$E(k)=\frac{p}{2m}=\frac{\hbar^{2}k^{2}}{2m}$$
ולכן:
$$Z_{1}=\int_{0}^{\infty}\mathrm{e}^{-\beta\hbar^{2}k^{2}/2m}{\frac{V k^{2}\,\mathrm{d}k}{2\pi^{2}}}={\frac{V}{\hbar^{3}}}\left({\frac{m k_{\mathrm{B}}T}{2\pi}}\right)^{3/2}$$
כאשר עבור \(\lambda_{\mathrm{T}}=\frac{h}{\sqrt{2\pi m k_{\mathrm{B}}T}}\) נקבל \(Z_{1}=\frac{V}{\lambda_{T}^{3}}\)

\end{proof}
\begin{remark}
לעיתים מגדירים גודל הנקרא הצפיפות הקוונטית:
$$n_{\mathrm{Q}}=\frac{1}{\hbar^{3}}\left(\frac{m k_{\mathrm{B}}T}{2\pi}\right)^{3/2}$$
ואז \(Z_{1}=Vn_{Q}\).

\end{remark}
\begin{proposition}[פונקציית החלוקה של גז אידיאלי]
$$Z_{\mathrm{ideal\,gas}}=\frac{1}{N!}\left(\frac{V}{\lambda^{3}}\right)^{N},\;\;\;\;\;\lambda=\frac{h}{\sqrt{2\pi m k_{B}T}}$$
כאשר \(\lambda\) נקרא אורך גל דה ברויי התרמי.

\end{proposition}
\begin{proof}
בגבול שבו \(n\ll n_{Q}\) (או \(n\lambda_{T}^{3}\ll 1\)) האפקטים הקוונטים לא משמעותיים ומערכת של גז אידאלי יהיה אוסף של חלקיקים בלתי מובחנים ולכן:
$$Z_{N}=\frac{1}{N!}\left(\frac{V}{\lambda_{\mathrm{th}}^{3}}\right)^{N}$$

\end{proof}
\begin{example}[מציאת פוטנציאל כימי]
כאשר אנחנו נמצאים בשיווי משקל ואין חילוף חלקיקים ניתן לכתוב:
$$F=-k_{B}T\ln Z_{N}=-k_{B}T\ln \frac{Z_{1}}{N!}\implies F=-k_{B}T\left[ N\ln Z_{1}-N\ln N+N \right]$$
ולכן הפוטנציאל הכימי יהיה:
$$\mu=\frac{\partial F}{\partial N} =-k_{B}T\left[ \ln Z_{1}-\ln N+1 -1 \right]=-k_{B}T\ln \left( \frac{V}{N\lambda^{3}} \right)$$

\end{example}
\begin{summary}
  \begin{itemize}
    \item עבור \(N\) חלקיקים הניתנים להבחנה נקבל \(Z_{N}=(Z_{1})^{N}\) כאשר עבור \(N\) חלקיקים הבלתי ניתנים להבחנה נקבל \(Z_{N}=\frac{1}{N!}(Z_{1})^{N}\).
  \end{itemize}
\end{summary}
\subsection{המערכת הפתוחה - הצבר הגאנד קנוני}

\begin{reminder}[פוטנציאל הכימי]
מדד לכמה "קשה" להכניס חלקיקים למערכת. מסומן ב-\(\mu\).

\end{reminder}
\begin{definition}[הצבר הגראנד קנוני]
מערכת שיכולה להחליף גם אנרגיה וגם חלקיקים עם אמבט. כאשר נניח כי האמבט גדול משמעותית מהמערכת. כיוון שאנו דורשים שיווי משקל נקבל כי הטמפרטורה של המערכת שווה לטמפרטורה של המאגר(\(T=T_{R}\)) וכן הפוטנציאלים הכימים שווים(\(\mu=\mu_{R}\)).

\end{definition}
\begin{proposition}
$$P_{i}=\frac{\mathrm{e}^{\beta\left(\mu N_{i}-E_{i}\right)}}{\mathcal{Z}}\qquad {\mathcal Z}=\sum_{i}\mathrm{e}^{\beta\left( \mu N_{i}-E_{i} \right)}$$

\end{proposition}
\begin{definition}[גראנד פוטנציאל]
ההתמרת לג'נדר של ההאנרגיה הפנימית עם הפוטנציאל הכימי והטמפרטורה:
$$\Phi[T,\mu]=U-T S-\mu N$$

\end{definition}
\begin{remark}
הסיבה שמשתמשים בהתמרת לג'נדר זה כי זה משמר אינפורמציה. כלומר המשוואה הזאת היא זהה מבחינת הפיזיקה אך עם משתנים שונים.

\end{remark}
\begin{proposition}[משוואת איילר]
אם מערכת היא אקסטניבית האנרגיה שלנו היא פונקציה הומוגנית מסדר ראשון, ומקיימת לכל \(\lambda\):
$$\lambda U(S,V,N)=U(\lambda S,\lambda V,\lambda N)$$
וכן מתקיים:
$$U=T S-P V+\mu N$$

\end{proposition}
\begin{proof}
נגזור את המשוואה לפי \(\lambda\) ונקבל:
$$\begin{gather}{{U(S,V,N)=\frac{\partial U\left( \lambda S,\lambda V,\lambda N \right)}{\partial\left( \lambda S \right)}\frac{\partial\left( \lambda S \right)}{\partial\lambda}}} \\{{+\frac{\partial U\left( \lambda S,\lambda V,\lambda N \right)}{\partial\left( \lambda V \right)}\frac{\partial\left( \lambda V \right)}{\partial\lambda}}}{{+\frac{\partial U\left( \lambda S,\lambda V,\lambda N \right)}{\partial\left( \lambda N \right)}\frac{\partial\left( \lambda N \right)}{\partial\lambda}}} 
\end{gather}$$
כאשר עבור \(\lambda=1\) נקבל:
$$U(S,V,N)=\frac{\partial U(S,V,N)}{\partial S}S+\frac{\partial U(S,V,N)}{\partial V}V+\frac{\partial U(S,V,N)}{\partial N}N$$
כאשר אם נציב את ההגדרות של הגדלים המתאימים נקבל:
$$U=T S-P V+\mu N$$

\end{proof}
\begin{remark}
משוואה זו מאוד מזכירה את המשוואה:
$$d U=T d S-P d V+\mu d N$$

\end{remark}
\begin{proposition}
עבור מערכת אקסטנסיבית הפוטנציאל הכימי מקיים:
$$\mu= \frac{U-TS+PV}{N}=\frac{G}{N}$$

\end{proposition}
\begin{proof}
עבור מערכת אקסטנסיבית אם נגדיל את המערכת פי \(\lambda\) נצפה כי כל המשתנים יגדלו פי \(\lambda\), כלומר:
$$U\to\lambda U,\qquad S\to\lambda S,\qquad V\to\lambda V,\qquad N\to\lambda N,$$
כאשר אם נכתוב את האנטרופיה \(S\) בעזרת \(U,V,N\) נקבל:
$$\lambda S(U,V,N)=S(\lambda U,\lambda V,\lambda N)$$
כאשר אם נגזור לפי \(\lambda\) נקבל:
$$S=\frac{\partial S}{\partial(\lambda U)}\frac{\partial(\lambda U)}{\partial\lambda}+\frac{\partial S}{\partial(\lambda V)}\frac{\partial(\lambda V)}{\partial\lambda}+\frac{\partial S}{\partial(\lambda N)}\frac{\partial(\lambda N)}{\partial\lambda}$$
כאשר אם נקבע \(\lambda=1\) ונשתמש ביחסים:
$$\left({\frac{\partial S}{\partial U}}\right)_{N,V}={\frac{1}{T}},\qquad\left({\frac{\partial S}{\partial V}}\right)_{N,U}={\frac{p}{T}},\qquad\left({\frac{\partial S}{\partial N}}\right)_{U,V}=-{\frac{\mu_{0}}{T}}$$
נקבל:
$$U-T S+p V=\mu N$$
כאשר נזהה את פונקציית גיבס באגף שמאל ולכן:
$$G=\mu N\implies \mu=\frac{G}{N}$$

\end{proof}
\begin{corollary}
עבור מערכות אקסטנסיביות, משוואת אוילר מתקיימת, ונקבל כי הגראנד פוטנציאל מקיים:
$$\Phi[T,\mu]=U-T S-\mu N=-P V$$

\end{corollary}
\begin{proposition}[התפלגות ההסתברות של הצבר הגראנד קנוני]
$$P(E,N)=\frac{1}{\mathcal{Z}}\Omega(E,V,N)\exp\left[-\beta E+\beta\mu N\right]$$
כאשר \(\mathcal{Z}\) נקבע על פי תנאי נרמול.

\end{proposition}
\begin{proof}
באופן דומה לצבר הקנוני, נדרוש שההסתברות תקיים:
$$P(E,N)=\frac{\Omega(E,V,N)\Omega_{R}(E_{T}-E,V_{R},N_{T}-N)}{\Omega_{T}(E_{T},V_{T},N_{T})}$$
ניקח לוגוריתם ונפתח את הטור טיילור של \(\ln \Omega_{R}(E_{T}-E, V_{R},N_{T}-N)\) בחזקות של \(\frac{E}{E_{T}},\frac{N}{N_{T}}\):
$$\begin{gather}\ln P(E,N)=\ln\Omega(E,V,N)+\ln\Omega_{R}(E_{T}-E,V_{R},N_{T}-N)-\ln\Omega_{T}(E_{T},V_{T},N_{T})  \\\approx\ln\Omega(E,V,N)+\ln\Omega_{R}(E_{T},V_{R},N_{T})+E\frac{\partial}{\partial E_{T}}\ln\Omega_{R}(E_{T},V_{R},N_{T})+ \\+N\frac{\partial}{\partial N_{T}}\ln\Omega_{R}(E_{T},V_{R},N_{T})-\ln\Omega_{R}(E_{T},V_{T},N_{T})
\end{gather}$$
כאשר כעת נזכור כי:
$$S_{R}=k_{B}\ln\Omega_{R}(E_{T},V_{R},N_{T})$$
לכן נקבל כעת כי \(\beta_{R}=\frac{1}{k_{B}T_{R}}\) יהיה שווה:
$$\beta_{R}\equiv\frac{\partial}{\partial E_{T}}\ln\Omega_{R}(E_{T},V_{R},N_{T})$$
כאשר מזהות נוספת עבור הפוטנציאל הכימי של המאגר נקבל:
$$-\mu_{R}\beta_{R}\equiv\frac{\partial}{\partial N_{T}}\ln\Omega_{R}(E_{T},V_{R},N_{T})$$
כעת כיוון ש-\(\beta=\beta_{R}\) ו-\(\mu=\mu_{R}\) נקבל כי הגדלים \(E_{T},N_{T},V_{R}\) לא תלויים ב-\(E\) או \(N\). ניתן לשלב את \(\ln \Omega_{R}(E_{T},V_{R},N_{T})\) ו-\(\ln\left( \Omega_{R}(E_{T}, V_{T},N_{T}) \right)\) לערך יחיד המסומן ב-\(-\ln \mathcal{Z}\). לכן:
$$\ln P(E,N)\approx\ln\Omega(E,V,N)-\beta E+\beta\mu N-\ln{\mathcal{Z}}$$
כאשר אם ניקח את האקספוננט נקבל:
$$P(E,N)=\frac{1}{\mathcal{Z}}\Omega(E,V,N)\exp\left[-\beta E+\beta\mu N\right]$$

\end{proof}
\begin{corollary}[הפונקציית הגראנד חלוקה]
$${\mathcal{Z}}=\sum_{N=0}^{\infty}\int_{0}^{\infty}d E\,\Omega(E,V,N)\exp\left[-\beta E+\beta\mu N\right]$$
או לחלופין בעזרת הפונקציית חלוקה המוגדרת על ידי:
$$Z(T,V,N)=\int_{0}^{\infty}d E\,\Omega(E,V,N)\exp(-\beta E)d E$$
ניתן לכתוב:
$${\mathcal{Z}}(T,V,\mu)=\sum_{N=0}^{\infty}Z(T,V,N)\exp\left[\beta\mu N\right]$$

\end{corollary}
\begin{proposition}[גראנד פוטנציאל בעזרת הפונקציית הגראנד חלוקה]
הפונקציית הגראנד חלוקה מקיימת:
$$\ln \mathcal{Z} = -\beta \Phi\left( T,\mu \right)$$
כאשר אם המערכת אקסטנסיבית ומקיימת את משוואת אויילר מתקיים בנוסף:
$$\ln \mathcal{Z} = \beta PV$$

\end{proposition}
\begin{example}[פונקציית הגראנד חלוקה עבור גז אידיאלי]
ראינו כי פונקציית החלוקה עבור גז אידיאלי תהיה:
$$Z=\frac{1}{h^{3N}N!}\left(2\pi m k_{B}T\right)^{3N/2}V^{N}$$
כאשר מזה נקבל:
$$\begin{gather}{\mathcal{Z}}\left( T,V,\mu \right)=\sum_{N=0}^{\infty}{\frac{1}{h^{3N}N!}}\left(2\pi m k_{B}T\right)^{3N/2}V^{N}\exp\left[\beta\mu N\right]\\=\sum_{N=0}^{\infty}\frac{1}{N!}\left(\frac{\left(2\pi m k_{B}T\right)^{3/2}}{h^{3}}V e^{\beta\mu}\right)^{N}=\exp\left(\left(2\pi m k_{B}T\right)^{3/2}h^{-3}V e^{\beta\mu}\right) 
\end{gather}$$
כאשר אם ניקח את הלוגוריתם של שתי האגפים נקבל מהטענה הקודמת את האנרגיה:
$$-\beta \Phi[T,\mu]=\left(2\pi m k_{B}T\right)^{3/2}h^{-3}V e^{\beta\mu}=\beta P V$$
כאשר השיוויון האחרון נובע מכך שהגז האידיאלי אקסטנסיבי. על ידי חלוקה ב-\(\beta V\) נקבל:
$$P=k_{B}T\left(2\pi m k_{B}T\right)^{3/2}h^{-3}e^{\beta\mu}$$
כאשר ניתן לכתוב את כל הביטיים בעזרת \(\lambda_{T}=\frac{h}{\sqrt{ 2\pi mk_{B}T }}\) ולקבל:
$${\mathcal{Z}}\left( T,V,\mu \right)=\exp\left({\frac{V}{\lambda^{3}}}e^{\beta\mu}\right)\quad \Phi\left( T,V,\mu \right)=-k_{B}T\frac{V}{\lambda^{3}}e^{\beta\mu}\quad P=\frac{k_{B}T}{\lambda^{3}}e^{\beta\mu}$$

\end{example}
\begin{proposition}
$$\begin{gather}N=\sum_{i}N_{i}P_{i}=k_{\mathrm{B}}T\left({\frac{\partial\mathrm{ln}\;{\mathcal{Z}}}{\partial\mu}}\right)_{\beta}\\ U=\sum_{i}E_{i}P_{i}=-\left(\frac{\partial\mathrm{ln}\;\mathcal{Z}}{\partial\beta}\right)_{\mu}+\mu N\\ S=-k_{\mathrm{B}}\sum_{i}P_{i}\ln P_{i}={\frac{U-\mu N+k_{\mathrm{B}}T\ln{\mathcal{Z}}}{T}} 
\end{gather}$$

\end{proposition}
\begin{proposition}[מציאת הפוטנציאל הכימי]
נעשה בשלבים הבאים:

  \begin{enumerate}
    \item כותבים את \(\mathcal{Z}\) בעזרת \(T,V,\mu\) ורמות האנרגיה. 


    \item נשתמש ביחס \(N=k_{B}T \frac{\partial \ln \mathcal{Z}}{\partial \mu}\). 


    \item נפתור משוואה עבור \(\mu\). 


  \end{enumerate}
\end{proposition}
\begin{example}[מציאת פוטנציאל הכימי]
פונקציית הגראנד חלוקה של גז אידיאלי תהיה:
$${\mathcal{Z}}=\sum_{N=0}^{\infty}{\frac{1}{N!}}\left({\frac{V}{\lambda^{3}}}e^{\beta\mu}\right)^{N}=\exp\left( \frac{V}{\lambda^{3}}e^{ \beta \mu }\right)$$
כאשר כעת ניתן לגזור ולהשתמש ביחס \(\langle N\rangle=\frac{1}{\beta}\frac{\partial\ln{\mathcal{Z}}}{\partial\mu}\) ונקבל:
$$\langle N\rangle=e^{\beta\mu}\frac{V}{\lambda^{3}}$$
וכעת בהנחה ש-\(N\) נתון ניתן לבודד עבור \(\mu\) ולקבל:
$$\mu=k_{B}T\ln\left(\frac{\langle N\rangle\lambda^{3}}{V}\right)$$

\end{example}
\begin{summary}
  \begin{itemize}
    \item הצבר הגאנד קנוני מתאר מערכת שיכול להחליף חלקיקים וגם אנרגיה עם אמבט.
    \item הגראנד פואנציאל הוא ביטוי לאנרגיה במשתנים הטבעיים של הצבר, ומוגדר על ידי:
$$\Phi[T,\mu]=U-T S-\mu N$$
    \item פונקציית הגראנד חלוקה מוגדרת על ידי:
$${\mathcal{Z}}=\sum_{N=0}^{\infty}\int_{0}^{\infty}d E\,\Omega(E,V,N)\exp\left[-\beta E+\beta\mu N\right]$$
    \item הקשר בין הפונקציית הגאנד חלוקה לגראנד פוטנציאל הוא:
$${\mathcal{Z}}=\exp(-\beta \Phi[T,\mu])=\exp(\beta P V)$$
כאשר השיוויון האחרון נכון רק כאשר המערכת אקסטנסיבית.
    \item מהגראנד פוטנציאל ניתן להשיג את הגדלים הבאים:
$$\begin{gather}N=k_{\mathrm{B}}T\left({\frac{\partial\mathrm{ln}\;{\mathcal{Z}}}{\partial\mu}}\right)_{\beta}\quad U=-\left(\frac{\partial\mathrm{ln}\;\mathcal{Z}}{\partial\beta}\right)_{\mu}+\mu N\quad S={\frac{U-\mu N+k_{\mathrm{B}}T\ln{\mathcal{Z}}}{T}} 
\end{gather}$$
  \end{itemize}
\end{summary}
\subsection{משוואת סאהא}

\begin{definition}[יינון]
תהליך שבו אטום או מולקולה מאבדים אלקטרונים, וגורם להתווצרות של יונים. ניתן לכתוב תהליך זה בצורה הבאה:
$$\mathrm{\mathrm{Atom}}\rightleftharpoons\mathrm{\mathrm{Ion}}^{+}+e^{-}$$

\end{definition}
\begin{definition}[צפיפות הפרוטונים]
כמות הפרוטונים ליחידת נפח אשר מיוניים(ללא אלקטרונים). מסומן ב-\(n_{p}\).

\end{definition}
\begin{definition}[צפיפות האלקטרונים]
כמות האלקטרונים ליחידת נפח אשר אינם קשורים לכלום. מסומן ב-\(n_{e}\).

\end{definition}
\begin{definition}[צפיפות האטומים]
צפיפות הפרטונים ליחידת נפח אשר אינם מיוננים(כלומר מכילים אלקטרון). מוסמן ב-\(n_{H}\).

\end{definition}
\begin{definition}[צפיפות הכוללת של גרעיני המימן]
סך גרעיני מימן, כלומר כולל גם את הפרוטונים המיוניים וגם הלא מיונים. מסומן ב-\(n_{0}\).

\end{definition}
\begin{corollary}
נובע מידית מההגדרות כי \(n_{p}+n_{H}=n_{0}\), ו-\(n_{p}=n_{e}\).

\end{corollary}
\begin{proposition}
קיים פונקציה \(x=x(n_{0},T)\) הנקרא השבר היינון אשר מקיים:
$$n_{p}=n_{e}=x n_{0}\;\;\;\;n_{H}=(1-x)\,n_{0}$$

\end{proposition}
\begin{symbolize}
נסמן את הפוטנציאלים הכימים המתאימים על ידי \(\mu_{p},\mu_{H},\mu_{e}\).

\end{symbolize}
\begin{reminder}
כאשר אנחנו בשיווי משקל אנחנו עם טמפרטורה קבועה ועבור תהליך כמו זה שבו הנפח קבוע נקבל כי בשיווי משקל האנרגיה החופשית של הלמהולדס\((F)\) תהיה אקסטרימלית. 

\end{reminder}
\begin{proposition}
הפוטנציאל הכימי מקיים:
$$\sum_{j}\mu_{j}d N_{j}=0\qquad \mu_{j}=\frac{\partial F}{\partial N_{j}}=-\frac{1}{\beta}\frac{\partial \ln Z}{\partial N_{j}}$$

\end{proposition}
\begin{proof}
האנרגיה החופשית של הלמהולדס נתונה על ידי:
$$d F=-S d T-P d V+\sum_{j\in\{ p,e,H \}}\mu_{j}d N_{j}$$
ולכן מצד אחד מתקיים:
$$\mu_{j}=\frac{\partial F}{\partial N_{j}}=-\frac{1}{\beta}\frac{\partial \ln Z}{\partial N_{j}}$$
וכן בנוסף כיוון שהנפח והטמפרטורה קבועות נקבל כי \(\mathrm{d}F=0\) וכן \(\mathrm{d}T=\mathrm{d}V=0\) ונקבל:
$$\sum_{j\in\{ p,e,H \}}\mu_{j}\mathrm{d}N_{j}=0$$

\end{proof}
\begin{proposition}
מהקשר \(H\rightleftarrows p+e\) נקבל כי שינוי המספר חלקיקים מקיים:
$$d N_{p}=d N_{e}=-d N_{H}$$

\end{proposition}
\begin{corollary}
משלוב שתי הטענות הקודמות נקבל:
$$\mu_{p}d N_{p}+\mu_{e}d N_{e}+\mu_{H}d N_{H}=\mu_{p}d N+\mu_{e}d N-\mu_{H}d N=0\implies \mu_{p}+\mu_{e}=\mu_{H}$$

\end{corollary}
\begin{lemma}
פונקציית החלוקה של המערכת שלנו תהיה בקירוב של הרבה חלקיקים:
$$Z\approx\sum_{j}\left(N_{j}l n\left(z_{j}\right)-N_{j}l n N_{j}+N_{j}\right)$$
כאשר \(z_{j}\) היא פונקציה חלוקה חד חלקיקית של כל חלקיק.

\end{lemma}
\begin{proof}
כיוון שיש לנו חלקיקים מסוגים שונים המיקרומצבים של בחלקיקים בלתי תלויים, ולכן הפונקציית החלוקה הכוללת תהיה המכפלה של הפונקציות חלוקה של כל אוכלוסיה:
$$Z=Z_{p}\cdot Z_{e}\cdot Z_{H}$$
כאשר ניתן לפרק את למכפלת פונקציות החלוקה החד חלקיקיות של כל חלקיק \(Z=\prod_{j} \frac{z_{j}^{N_{j}}}{N_{j}!}\). ניקח את הלוגוריתם ונקבל:
$$\ln Z=\sum_{j}\ln\left(\frac{z_{j}^{N_{j}}}{N_{j}!}\right)=\sum_{j}\left(N_{j}\ln\left(z_{j}\right)-\ln\left(N_{j}!\right)\right)\approx\sum_{j}\left(N_{j}\ln\left(z_{j}\right)-N_{j}\ln N_{j}+N_{j}\right)$$
כאשר השתמשנו בקירוב סטרלינג.

\end{proof}
\begin{corollary}
ניתן לקבל את הפוטנציאל הכימי של כל אוכלוסיה על ידי:
$$\mu_{j}=-\frac{1}{\beta}\ln \frac{z_{j}}{N_{j}}$$

\end{corollary}
\begin{proof}
$$\mu_{j}=-\frac{1}{\beta}\frac{\partial}{\partial N_{j}}\sum_{i}\left(N_{i}\ln\left(z_{i}\right)-N_{i}\ln N_{i}+N_{i}\right)=-\frac{1}{\beta}\left(l n\left(z_{j}\right)-\ln N_{j}-N_{j}\cdot\frac{1}{N_{j}}+1\right)=-\frac{1}{\beta}\ln\frac{z_{j}}{N_{j}}$$

\end{proof}
\begin{proposition}
עבור כל חלקיק ב-\(H\)(גרעינים מיוניים - עם אלקטרון) ניתן לפצל את האנרגיה של כל חלקיק לאנרגיה של הגרעין ושל האלקטרון:
$$\mathcal{E} _{H}=\frac{p^{2}_{\text{nucleus}}}{2m_{\text{nucleus}}}+\frac{p^{2}_{e}}{2m_{e}}+\mathcal{E} _{0H}$$
כאשר \(\mathcal{E}_{0H}\) היא אנרגיית הקשר. אך כיוון שמסת האלקטרון קטנה מפאקטור של 4 סדרי גודל נזניח את הגורם של האלקטרון.

\end{proposition}
\begin{corollary}
פונקציית החלוקה החד חלקיקית נתונה על ידי:
$$g_{s j}V\left(\frac{m_{j}k_{B}T}{2\pi\hbar^{2}}\right)^{\frac{3}{2}}e^{-\beta\varepsilon_{0j}}$$
כאשר:
$$\varepsilon_{0,j}=\begin{cases}0&j=p,e\\ -E_{0}&j=H\end{cases};\ g_{s j}=\begin{cases}2&j=p,e\\ 4&j=H\end{cases}$$

\end{corollary}
\begin{proof}
$$z_{j}=\frac{g_{s j}}{h^{3}}\int d^{3}r d^{3}p e^{-\beta\left(\frac{p^{2}}{2m_{j}}+\varepsilon_{0j}\right)}$$

\end{proof}
\begin{corollary}
הפוטנציאל הכימי של כל אוכלוסיה נתונה על ידי:
$$\mu_{j}=-k_{B}T l n\left(\frac{g_{s j}}{n_{j}}\left(\frac{m_{j}k_{B}T}{2\pi\hbar^{2}}\right)^{\frac{3}{2}}\right)+\varepsilon_{0j}$$

\end{corollary}
\begin{proof}
נובע ישירות מ-\(\mu_{j}=-\frac{1}{\beta}\ln \frac{z_{j}}{N_{j}}\) והפונקציית חלוקה שמצאנו.

\end{proof}
\begin{proposition}[משוואת סאהא]
$$\frac{x^{2}}{1-x}=\frac{1}{n_{0}\lambda_{T}^{3}}e^{-\frac{E_{0}}{k_{B}T}}\qquad \lambda_{T}\equiv\left(\frac{2\pi\hbar^{2}}{m_{e}k_{B}T}\right)^{\frac{1}{2}}$$

\end{proposition}
\begin{proof}
מהמשוואה \(\mu_{p}+\mu_{e}=\mu_{H}\) נקבל:
$$-k_{B}T \ln\left(\frac{2}{n_{p}}\left(\frac{m_{p}k_{B}T}{2\pi\hbar^{2}}\right)^{\frac{3}{2}}\right)-k_{B}T \ln\left(\frac{2}{n_{e}}\left(\frac{m_{e}k_{B}T}{2\pi\hbar^{2}}\right)^{\frac{3}{2}}\right)=-k_{B}T \ln\left(\frac{4}{n_{H}}\left(\frac{m_{p}k_{B}T}{2\pi\hbar^{2}}\right)^{\frac{3}{2}}\right)-E_{0}$$
כאשר כיוון שהנחנו \(m_{H}\approx m_{p}\) נחלק ב-\(-k_{B}T\) ונחבר את הלוגריתמים:
$$\ln\left(\frac{2}{n_{p}}\left(\frac{m_{p}k_{B}T}{2\pi\hbar^{2}}\right)^{\frac{3}{2}}\cdot\frac{2}{n_{e}}\left(\frac{m_{e}k_{B}T}{2\pi\hbar^{2}}\right)^{\frac{3}{2}}\right)=\ln\left(\frac{4}{n_{H}}\left(\frac{m_{p}k_{B}T}{2\pi\hbar^{2}}\right)^{\frac{3}{2}}\right)+\frac{E_{0}}{k_{B}T}$$
ניקח אקספוננט לשתי האגפים:
$${\frac{4}{n_{p}n_{e}}}{\bigg(}{\frac{m_{p}k_{B}T}{2\pi\hbar^{2}}}{\bigg)}^{\frac{3}{2}}\left({\frac{m_{e}k_{B}T}{2\pi\hbar^{2}}}\right)^{\frac{3}{2}}={\frac{4}{n_{H}}}{\bigg(}{\frac{m_{p}k_{B}{T}}{2\pi\hbar^{2}}}{\bigg)}^{\frac{3}{2}}\cdot e^{\frac{E_{0}}{k_{B}T}}$$
ולכן:
$$\frac{n_{p}n_{e}}{n_{H}}=\left(\frac{m_{e}k_{B}T}{2\pi\hbar^{2}}\right)^{\frac{3}{2}}e^{-\frac{E_{0}}{k_{B}T}}$$
מהיחסים \(n_{p}=n_{e}=x n_{0},n_{H}=(1-x)\,n_{0}\) נקבל:
$$\frac{x^{2}}{1-x}=\frac{1}{n_{0}}\left(\frac{m_{e}k_{B}T}{2\pi\hbar^{2}}\right)^{\frac{3}{2}}e^{-\frac{E_{0}}{k_{B}T}}$$
כאשר ניתן כעת לזהות את האורך גל התרמי \(\lambda_{T}\) ולקבל את המשוואה.

\end{proof}
\begin{corollary}
עבור המרחק הממוצע בין החלקיקים \(a\) כאשר \(n_{0}=\frac{1}{a^{3}}\) ולקבל:
$$\frac{x^{2}}{1-x}=\left(\frac{a}{\lambda_{T}}\right)^{3}e^{-\frac{E_{0}}{k_{B}T}}$$

\end{corollary}
\begin{example}
נתון גז אידיאלי המורכב מאטומים שיכולים להיות באחד משני מצבים - ניטרלי או מיונן פעם אחת. נרצה למצוא את \(n_{e},n_{p},n_{H}\).
התהליך הנתון הוא \(\mathrm{\mathrm{A}}\rightleftharpoons\mathrm{\mathrm{A}}^{+}+e^{-}\) כמו שראינו. ראשית נשים לב כי בשיווי משקל מתקיים:
$$\sum_{i}\frac{\partial F}{\partial N_{i}}=0\implies \mu^{A^{+}}+\mu^{e^{-}}-\mu^{A}=0$$
ראינו כי הפוטנציאל הכימי של גז אידיאלי נתון על ידי \(\mu=-k_{B}T\left( \frac{V}{N_{A}\lambda_{A}^{3}} \right)\) ולכן:
$$\mu_{A}=-k_{B}T\ln\left( \frac{V}{N_{A}\lambda_{A}^{3}} \right)\qquad \mu_{e^{-}}=-k_{B}T\ln\left( \frac{V}{N_{A}\lambda_{e}^{3}} \right)$$

2016 מועד ב שאלה 1 סעיף ג להשלים. 

\end{example}
\begin{summary}
שלבים למציאת סאהא

  \begin{enumerate}
    \item היחס \(x=x(n_{0},T)\) יקיים: 
$$n_{e}=n_{p}=x\cdot n_{0}\qquad n_{p}+n_{H}=n_{0}=(1-x)n_{0}$$


    \item כותבים את \(\mu_{j}=-\frac{1}{\beta}\frac{\partial \ln Z}{\partial N_{j}}\) כאשר מדרישת שיווי משקל נקבל \(\sum_{j}\mu_{j}\mathrm{d}N_{j}=0\) כאשר \(j \in \{ e,p,H \}\).  


    \item מתקיים \(\mathrm{d}N_{p}=\mathrm{d}N_{e}=-\mathrm{d}N_{H}\). נציב בשלב 2 ונקבל \(\mu_{p}+\mu_{e}=\mu_{H}\). 


    \item נרשום את פונקציית החלוקה של כל המערכת: 
$$Z\approx\sum_{j}\left(N_{j}l n\left(z_{j}\right)-N_{j}l n N_{j}+N_{j}\right)\implies \mu_{j}=-\frac{1}{\beta}\ln \frac{z_{j}}{N_{j}}$$
כאשר \(z_{j}\) היא פונקציית החלוקה עבור כל אוכלוסייה.


    \item מציבים במשוואה \(\mu_{p}+\mu_{e}=\mu_{H}\) כאשר זוכרים כי יש ל-\(z_{H}\) יש ריבוי ספין 4 ולא 2 כמו \(z_{e},z_{H}\) כאשר נציב את היחסים משלב 1. 


  \end{enumerate}
\end{summary}
\section{פיזיקה סטטיטית קוונטית}

\subsection{חילוף חלקיקים}

\begin{definition}[אופרטור החילוף]
עבור שתי חלקיקים זההים אשר נמצאים ב-\(\vec{r}_{1}\) ו-\(\vec{r}_{2}\) בהתאמה ניתן לתאר על ידי פונקציית גל מהצורה \(\psi\left( \vec{r}_{1},\vec{r}_{2} \right)\). אופרטור החילוף יהיה אופרטור \(P\) המקיים:
$$P_{12}\psi\left( \vec{r}_{1},\vec{r}_{2} \right)=\psi\left( \vec{r}_{2},\vec{r}_{1} 
\right)$$

\end{definition}
\begin{proposition}
אופרטורים זההים אינווריאנטים לאופרטור החילוף, כלומר:
$$[H,P_{12}]=0$$

\end{proposition}
\begin{corollary}
כיוון שזו פעולת סימטריה מתקיים:
$$\left\lvert  P_{12}\psi\left( \vec{r}_{1},\vec{r}_{2} \right)  \right\rvert^{2} =\left\lvert  \psi\left( \vec{r}_{1},\vec{r}_{2} \right)  \right\rvert ^{2}\implies\left\lvert  \psi\left( \vec{r}_{2},\vec{r}_{1} \right)  \right\rvert^{2} =\left\lvert  \psi\left( \vec{r}_{1},\vec{r}_{2} \right)  \right\rvert ^{2}$$
כלומר יש שתי אפשרויות:

  \begin{enumerate}
    \item פונקציית הגל היא סימטרית תחת החלפה: 
$$\psi(\mathbf{r}_{2},\mathbf{r}_{1})=\psi(\mathbf{r}_{1},\mathbf{r}_{2}),$$


    \item פונקציית הגל אנטי סימטרית תחת החלפה: 
$$\psi(\mathbf{r}_{2},\mathbf{r}_{1})=-\psi(\mathbf{r}_{1},\mathbf{r}_{2})$$


  \end{enumerate}
\end{corollary}
\begin{remark}
הפיצול לשתי מקרים רלוונטי רק ב-3 מימדים. בשתי מימדים ניתן לקבל גם פאזה:
$$\psi(\,\mathbf{r}_{2},\mathbf{r}_{1}\,)\;=\;\mathrm{e}^{\mathrm{i}\theta}\psi(\,\mathbf{r}_{1},\mathbf{r}_{2}\,)$$

\end{remark}
\begin{proposition}
עבור מערכת עם שתי חלקיקים ניתן לתאר אותם במצב מכפלה. נחלק אותם ל-4 אפשרויות:

  \begin{enumerate}
    \item חלקיקים מובחנים - יהיו 4 אפשרויות: 
$$|0\rangle|0\rangle,\qquad|1\rangle|0\rangle,\qquad|0\rangle|1\rangle,\qquad|1\rangle|1\rangle.$$


    \item חלקיקים לא מובחנים - יהיו 3 אפשרויות: 
$$|0\rangle|0\rangle,\qquad|1\rangle|0\rangle,\qquad|1\rangle|1\rangle.$$


    \item בוזונים - לא מובחנים - יהיו 3 אפשרויות: 
$$\ket{0} \ket{0} \quad \ket{1} \ket{1} \quad \frac{1}{\sqrt{2}}\left(|1\rangle|0\rangle+|0\rangle|1\rangle\right),$$


    \item פרמיונים - לא מובחנים - יהיה רק אפשרות אחת: 
$$\frac{1}{\sqrt{2}}\left(|1\rangle|0\rangle-|0\rangle|1\rangle\right).$$
כאשר עבור הפרמיונים והבוזונים המצב המעורב מתקבל כיוון שנדרש שיהיה ערך עצמי של אופרטור החילוף.


  \end{enumerate}
\end{proposition}
\begin{proposition}[עקרון האיסור של פאולי]
לא ייתכן ויהיו שתי פרמיונים באותו מצב קוונטי

\end{proposition}
\begin{proof}
נניח בשלילה ש-\(\psi=\ket{\varphi}\ket{\varphi}\) הוא מערכת פרמיונית של שתי מצבים באותו מצב קוונטי. תחת אופרטור החילוף נקבל:
$$\hat{P}_{12}|\varphi\rangle|\varphi\rangle=|\varphi\rangle|\varphi\rangle=-|\varphi\rangle|\varphi\rangle,$$
כאשר השתמשנו באנטי סימטריה של פרמיונים. לכן מתקיים:
$$|\varphi\rangle|\varphi\rangle=0$$
כלומר לא ייתכן כי יש שתי מצבים קוונטים זההים. 

\end{proof}
\subsection{פרמיונים ובוזונים}

\begin{reminder}[פונקציית הגראנד חלוקה]
עבור מצב עם אנרגיה נתונה \(E\), הפונקציית הגראנד חלוקה יהיה הסכום על כל קונפיגורציות:
$${\mathcal Z}=\sum_{n}\mathrm{e}^{n\beta(\mu-E)}$$
כאשר כל קונפיגורציה נבדלת למעשה רק על ידי מספר החלקיקים, ולכן נדרש לסכום רק על מספר החלקיקים.

\end{reminder}
\begin{proposition}
מספר הממוצע של חלקיקים במצב יהיה:
$$\langle n\rangle=\frac{\sum_{n}n\mathrm{e}^{n\beta(\mu-E)}}{\sum_{n}\mathrm{e}^{n\beta(\mu-E)}}=-\frac{1}{\beta\mathcal{Z}}\frac{\partial\mathcal{Z}}{\partial E}=-\frac{1}{\beta}\frac{\partial\ln\mathcal{Z}}{\partial E}$$

\end{proposition}
\begin{corollary}
עבור פרמיונים יש רק שתי אפשרויות - \(n=1\) או \(n=0\) ולכן נדרש לסכום רק עליהם ולקבל:
$${\mathcal Z}=\sum_{n=0}^{1}\mathrm{e}^{n\beta(\mu-E)}=1+\mathrm{e}^{\beta(\mu-E)},$$

\end{corollary}
\begin{corollary}
$$\ln{\mathcal{Z}}=\ln(1+\mathrm{e}^{\beta(\mu-E)})$$

\end{corollary}
\begin{proposition}
עבור בוזונים נקבל:
$${\mathcal{Z}}=\sum_{n=0}^{\infty}\mathrm{e}^{n\beta(\mu-E)}={\frac{1}{1-\mathrm{e}^{\beta(\mu-E)}}}$$
ולכן:
$$\ln{\mathcal{Z}}=-\ln(1-\mathrm{e}^{\beta(\mu-E)})$$

\end{proposition}
\begin{proposition}
מספר החלקיקים הממוצע יהיה עבור פרמיונים:
$$\langle n\rangle=\frac{1}{\mathrm{e}^{\beta(E-\mu)}+1},$$
כאשר עבור בוזונים:
$$\langle n\rangle=\frac{1}{\mathrm{e}^{\beta(E-\mu)}-1},$$

\end{proposition}
\begin{proposition}
עבור מערכת כללית שבה מכניסים \(n_{i}\) חלקיקים לרמה \(i\) עם האנרגיה \(E_{i}\) מתקיים:
$${\mathcal Z}=\prod_{i}\sum_{\{n_{i}\}}\mathrm{e}^{n_{i}\beta(\mu-E_{i})}$$

\end{proposition}
\begin{proof}
ניתן לכתוב קונפיגורציה כללית על ידי:
$$\left[\mathrm{e}^{\beta(\mu-E_{1})}\right]^{n_{1}}\times\left[\mathrm{e}^{\beta(\mu-E_{2})}\right]^{n_{2}}\times\cdots=\prod_{i}\mathrm{e}^{n_{i}\beta(\mu-E_{i})}$$
ולכן פונקציית הגראנד חלוקה תהיה:
$${\mathcal{Z}}=\sum_{\left\{n_{i}\right\}}\prod_{i}\mathrm{e}^{n_{i}\beta(\mu-E_{i})}$$
כאשר ניתן לקחת גורם משותף ולהפוך את סדר הסכימה וכפל:
$$\begin{gather}\sum_{\left\{ n_{\epsilon} \right\}}\prod_{\epsilon}\exp\left[ -\beta\left( \epsilon-\mu \right)n_{\epsilon} \right],=\sum_{n_{\epsilon_{1}}}\sum_{n_{\epsilon_{2}}}\cdots e^{-\beta\left( \epsilon_{1}-\mu \right)n_{\epsilon_{1}}}e^{-\beta\left( \epsilon_{2}-\mu \right)n_{\epsilon_{2}}}\ldots=\\=\sum_{n_{\epsilon_{1}}}e^{-\beta\left( \epsilon_{1}-\mu \right)n_{\epsilon_{1}}}\sum_{n_{\epsilon_{2}}}e^{-\beta\left( \epsilon_{2}-\mu \right)n_{\epsilon_{2}}}\ldots =\\=\prod_{\epsilon}\sum_{n_{\epsilon}}\exp\left[ -\beta\left( \epsilon-\mu \right)n_{\epsilon} \right] 
\end{gather}$$

\end{proof}
\begin{remark}
הפיתוח הכללי הזה נותן את הפונקציית החלוקה של פרמיונים כאשר מתאפשר לכל היותר רמה אחת של אנרגיה ושל בוזונים כמות אינסופית של רמות.

\end{remark}
\begin{definition}[פונקציית התפלגות פרמי דיראק]
פונקציה:
$$f(E)=\frac{1}{\mathrm{e}^{\beta(E-\mu)}+1}$$

\end{definition}
\begin{definition}[פונקציית התפלגות בוז-אינשטיין]
$$f(E)=\frac{1}{\mathrm{e}^{\beta(E-\mu)}-1}$$

\end{definition}
\begin{remark}
פונקציות החלוקות מוגדרות כך שמקיימות \(f(E)=\langle n \rangle\). כמו כן נשים לב כי בגבול \(\beta\left( E-\mu \right)\gg 1\) נקבל את התפלגות בולצמן, דבר המתאים לעקרון ההתאמה האומר כי בגבול הקלאסי נצפה כי יתנהג בצורה קלאסית.

\end{remark}
\begin{remark}
נשים לב כי עבור \(\mu=E\) נקבל כי ההתפלגות בוזונית מתבדרת, זאת כיוון שבמצב זה הרמה הנמוכה ביותר תהיה מאוכלסת עם כמות אינסופית של חלקיקים - פתרון לא פיזיקלי, ולכן נדרש \(\mu< E\).

\end{remark}
\begin{summary}
  \begin{itemize}
    \item פונקציית הגראנד חלוקה של פרמיונים ובוזונים תקיים:
$$\ln{\mathcal{Z}}=\pm\ln(1\pm\mathrm{e}^{\beta(\mu-E)}),$$
כאשר ה-\(+\) זה עבור פרמיונים ו-\(-\) עבור בוזונים.
    \item מספר החלקיקים הממוצע יהיה:
$$\langle n_{E}\rangle=\frac{1}{\mathrm{e}^{\beta(E-\mu)}\pm1},$$
כאשר ה-\(+\) זה עבור פרמיונים ו-\(-\) עבור בוזונים.
    \item בוזונים מקיימים סטטיסטיקת בוז-אינשטיין, הנתונה על ידי:
$$f_{BE}(E)=\frac{1}{\mathrm{e}^{\beta(E-\mu)}-1}$$
    \item פרמיונים מקיימים סטטיסטיקת פרמי-דיראק, הנתונה על ידי:
$$f_{FD}(E)=\frac{1}{\mathrm{e}^{\beta(E-\mu)}+1}$$
  \end{itemize}
\end{summary}
\subsection{גז קוונטי}

\begin{proposition}
פונקציית הגראנד חלוקה הכוללת של מערכת תהיה המכפלה של פונקציות החלוקות של כל חלקיק:
$${\mathcal{Z}}_{t o t}=\prod_{i}{\mathcal{Z}}_{i}$$

\end{proposition}
\begin{example}
עבור מערכת עם \(k\) חלקיקים עם ספין \(S\) יש \(2S+1\) מצבים אפשריים ולכן:
$$\mathcal{Z}=\prod_{k}\mathcal{Z}_{k}^{2S+1}$$
כאשר:
$${\mathcal{Z}}_{\mathbf{k}}=\left(1\pm\mathrm{e}^{-\beta(E_{\mathbf{k}}-\mu)}\right)\pm1$$
כאשר \(+\) זה עבור פרמיונים ו-\(-\) זה עבור בוזונים.

\end{example}
\begin{proposition}
פונקציית הגראנד פוטנציאל בגבול הרצף(כלומר \(n\gg 1\)) יקיים:
$$\Phi_{G}= \mp k_{\mathrm{B}}T\int_{0}^{\infty}\ln\left( 1\pm\mathrm{e}^{-\beta\left( E-\mu \right)} \right)\,g(E)\,\mathrm{d}E $$
כאשר \(+\) עבור פרמיונים ו-\(-\) עבור בוזונים.

\end{proposition}
\begin{proof}
נזכור כי בגז קוונטי \(\mathcal{Z}=\prod_{n}\mathcal{Z}_{n}\). כעת נקבל:
$$\begin{gather}\Phi_{G}=-k_{B}T\ln \mathcal{Z} =-k_{B}T\ln \prod_{n}^{\infty}\mathcal{Z} _{n} =-k_{B}T\sum_{n=0}^{N} \ln \mathcal{Z} _{n}
\end{gather}$$
כאשר כיוון שאנו עוברים לגבול הרצף ניתן לכתוב:
$$
\Phi_{G}=\mp k_{\mathrm{B}}T\int_{0}^{\infty}\ln\left( 1\pm\mathrm{e}^{-\beta\left( E-\mu \right)} \right)\,g(E)\,\mathrm{d}E $$

\end{proof}
\begin{proposition}
מספר החלקיקים מקיים:
$$n_{\mathbf{k}}=k_{\mathrm{B}}T{\frac{\partial}{\partial\mu}}\ln{\mathcal{Z}}_{\mathbf{k}}={\frac{1}{\mathrm{e}^{\beta(E_{\mathbf{k}}-\mu)}\pm1}},$$

\end{proposition}
\begin{corollary}
מספר החלקיקים של המערכת נתון על ידי:
$$N=\sum_{k}n_{k}=\int_{0}^{\infty}{\frac{g(E)\,\mathrm{d}E}{\mathrm{e}^{\beta(E-\mu)}\pm1}}$$
כאשר סך האנרגיה נתונה על ידי:
$$U=\sum_{k}n_{k}E_{k}=\int_{0}^{\infty}\frac{E\,g(E)\,\mathrm{d}E}{\mathrm{e}^{\beta(E-\mu)}\pm1}$$

\end{corollary}
\begin{definition}[פוגאסיטי - fugacity]
$$z=\mathrm{e}^{\beta\mu}$$

\end{definition}
\begin{definition}[פולילוגוריתם]
פונקציה המוגדרת על ידי:
$$\operatorname{Li}_{n}(z)=\sum_{k=1}^{\infty}{\frac{z^{k}}{k^{n}}}$$
כאשר \(z\) זה העיגול יחידה הפתוח במישור המרוכב. ההגדרה על המרחב המרוכב כולו נובעת מההמשכה האנליטית.

\end{definition}
\begin{reminder}[פונקציית גאמא]
מוגדר על ידי:
$$\Gamma(z)=\int_{0}^{\infty} t^{z-1}e^{ -t } \, \mathrm{d}t $$
כאשר \(\mathrm{Re}(z)>0\). מקיים \(\Gamma(n)=(n-1)!\).

\end{reminder}
\begin{proposition}
$$\int_{0}^{\infty}\frac{x^{n-1}\,\mathrm{d}x}{z^{-1}\mathrm{e}^{x}\pm1}=\mp\Gamma(n)\mathrm{Li}_{n}(\mp z)$$

\end{proposition}
\begin{proof}
ראשית נשים לב כי:
$${\frac{1}{z^{-1}\mathrm{e}^{x}-1}}={\frac{z\mathrm{e}^{-x}}{1-z\mathrm{e}^{-x}}}=\sum_{m=0}^{\infty}(z\mathrm{e}^{-x})^{m+1}.$$
כאשר ניתן לחשב את האינטגרל הבא:
$$\begin{gather}\int_{0}^{\infty}\,\frac{x^{n-1}\,\mathrm{d}x}{z^{-1}\mathrm{e}^{x}-1}=\sum_{m=0}^{\infty}\int_{0}^{\infty}x^{n-1}\left( \left( z\mathrm{e}^{-x} \right)^{m+1} \right)=\\=\sum_{m=0}^{\infty}z^{m+1}\int_{0}^{\infty}x^{n-1}\mathrm{e}^{-(m+1)x}=\sum_{m=0}^{\infty}{\frac{z^{m+1}}{(m+1)^{n}}}\int_{0}^{\infty}y^{n-1}\mathrm{e}^{-y}=\\=\Gamma(n)\sum_{m=0}^{\infty}\frac{z^{m+1}}{(m+1)^{n}}=\Gamma(n)\sum_{k=1}^{\infty}\frac{z^{k}}{k^{n}} =\Gamma(n)\mathrm{Li}_{n}(z) 
\end{gather}$$
כאשר ניתן באופן דומה להראות כי:
$$\int_{0}^{\infty}\frac{x^{n-1}\,\mathrm{d}x}{z^{-1}\mathrm{e}^{x}+1}=-\Gamma(n)\mathrm{Li}_{n}(-z)$$
ואם נשלב את שתי המשוואות ניתן להראות באופן כללי:
$$\int_{0}^{\infty}\frac{x^{n-1}\,\mathrm{d}x}{z^{-1}\mathrm{e}^{x}\pm1}=\mp\Gamma(n)\mathrm{Li}_{n}(\mp z)$$

\end{proof}
\begin{remark}
נשים לב כי אם \(|z|\ll 1\) נקבל כי רק האיבר הראשון תורם בטור ולכן:
$$\operatorname{Li}_{n}(z)\approx z.$$
בנוסף לכך עבור \(z=1\) נקבל את פונקציית זטא של רימן:
$$\operatorname{Li}_{n}(1)=\sum_{k=1}^{\infty}{\frac{1}{k^{n}}}=\zeta(n),$$

\end{remark}
\begin{example}[מערכת ספינים בקופסא]
נסתכל על מערכת תלת מימדית של פרמיונים ובוזונים עם ספין \(S\)(כלומר לכל ספין יש ניוון אנרגטי של \(2S+1\)). המצבים במרחב ה-\(k\) מתפלגים באופן אחיד ולכן:
$$g(k)\,\mathrm{d}k={\frac{4\pi k^{2}\,\mathrm{d}k}{(2\pi/L)^{3}}}\times(2S+1)={\frac{(2S+1)V k^{2}\,\mathrm{d}k}{2\pi^{2}}}$$
כאשר אם נניח כי \(V=L^{3}\) ונשתמש בקשר \(E=\hbar^{2}k^{2} / 2m\) ניתן לכתוב:
$$g(E)\,\mathrm{d}E={\frac{(2S+1)V E^{1/2}\,\mathrm{d}E}{(2\pi)^{2}}}\left({\frac{2m}{\hbar^{2}}}\right)^{3/2}$$
נקבל כעת:
$$N=\left[\frac{(2S+1)V}{(2\pi)^{2}}\left(\frac{2m}{\hbar^{2}}\right)^{3/2}\right]\int_{0}^{\infty}\frac{E^{1/2}\,\mathrm{d}E}{z^{-1}\mathrm{e}^{\beta E}\pm1}$$
כאשר ניתן לכתוב בעזרת הפונקציה הפולילוגוריתמית ולקבל:
$$N=\frac{(2S+1)V}{\lambda_{\mathrm{th}}^{3}}[\mp\mathrm{Li}_{3/2}(\mp z)]$$
כאשר באופן דומה עבור האנרגיה נקבל:
$$U=\left[\frac{(2S+1)V}{(2\pi)^{2}}\left(\frac{2m}{\hbar^{2}}\right)^{3/2}\right]\int_{0}^{\infty}\frac{E^{3/2}\,\mathrm{d}E}{z^{-1}\mathrm{e}^{\beta E}\pm1}.$$
כאשר ניתן לכתוב:
$$U={{\frac{3}{2}k_{\mathrm{B}}T\frac{(2S+1)V}{\lambda_{\mathrm{th}}^{3}}[\mp\mathrm{Li}_{5/2}(\mp z)]}}=\frac{3}{2}N k_{\mathrm{B}}T\frac{\mathrm{Li}_{5/2}(\mp z)}{\mathrm{Li}_{3/2}(\mp z)}.$$
כמו כן אפשר לכתוב את הפונקציית הגראנד פוטנציאל ולקבל:
$$\Phi_{\mathrm{G}}=\mp k_{\mathrm{B}}T{\frac{(2S+1)V}{(2\pi)^{2}}}\left({\frac{2m}{\hbar^{2}}}\right)^{3/2}\int_{0}^{\infty}\ln(1\pm\mathrm{e}^{-\beta(E-\mu)})\,E^{1/2}\,\mathrm{d}E,$$
כאשר לאחר אינטגרציה בחלקים:
$$\Phi_{\mathrm{G}}=-\frac{2}{3}\frac{(2S+1)V}{(2\pi)^{2}}\left(\frac{2m}{\hbar^{2}}\right)^{3/2}\int_{0}^{\infty}\frac{E^{3/2}\,\mathrm{d}E}{\mathrm{e}^{\beta(E-\mu)}\pm1}$$
וניתן להשוואות למשוואה עבור \(U\) ולקבל את הקשר \(\Phi_{G}=-\frac{2}{3}U\).

\end{example}
\begin{summary}
  \begin{itemize}
    \item חלקיקים קוונטים הם לא מובחנים ולכן פונקציית החלוקה הכוללת תהיה \({\mathcal{Z}}_{t o t}=\prod_{i}{\mathcal{Z}}_{i}\).
    \item ניתן למצוא את מספר החלקיקים הממוצע והאנרגיה על ידי:
$$N\!=\!\sum_{k}n_{k}\!=\!\int_{0}^{\infty}\!\!\!{g(E)f(E)\;\mathrm{d}E}\quad U\!=\!\sum_{k}n_{k}E_{k}\!=\!\!\int_{0}^{\infty}\!\!Eg(E)f(E)\;\mathrm{d}E$$
    \item הגראנד פוטנציאל יהיה:
$$\Phi_{G}= \mp k_{\mathrm{B}}T\int_{0}^{\infty}\ln\left( 1\pm\mathrm{e}^{-\beta\left( E-\mu \right)} \right)\,g(E)\,\mathrm{d}E $$
כאשר \(+\) עבור פרמיונים ו-\(-\) עבור בוזונים.
    \item עבור גז חופשי בנפח \(V\) נקבל את הביטיים:
$$N=\frac{(2S+1)V}{\lambda_{\mathrm{th}}^{3}}\left[ \mp\mathrm{Li}_{3/2}\left( \mp z \right) \right] \qquad U=\frac{3}{2}N k_{\mathrm{B}}T\frac{\mathrm{Li}_{5/2}\left( \mp z \right)}{\mathrm{Li}_{3/2}\left( \mp z \right)}\qquad \Phi_{G}=-\frac{2}{3}U$$
  \end{itemize}
\end{summary}
\subsection{גז פרמי}

\begin{definition}[גז פרמי]
גז של פרמיונים.

\end{definition}
\begin{definition}[אנרגיית פרמי]
רמות האנרגיה מתמלאות בגלל עקרון האיסור של פאולי ללא קשר לטמפרטורה. האנרגיה שהפרמיונים מגיעים עליה באפס המוחלט נקראת אנרגיית פרמי. כלומר:
$$E_{F}=\mu(T=0)$$

\end{definition}
\begin{remark}
הגדרה זו הגיונית כיוון ש:
$$\mu(T=0)=\frac{\partial E}{\partial N} \implies \mu(T=0)=E(N)-E(N-1)=E_{F}$$

\end{remark}
\begin{remark}
זהו לא האנרגיה שיש למערכת באפס המוחלט! זו הרמת האנרגיה הגבוהה ביותר שחלקיק יכול להגיע עליה. האנרגיה הממוצעת באפס המחולט תהיה קטנה יותר כיוון שיש חלקיקים ברמות הנמוכות יותר. כדי למצוא את האנרגיה באפס המוחלט כדי להשתמש בהגדרה \(\langle E \rangle=\sum_{k}n_{k}E_{k}\).

\end{remark}
\begin{proposition}
באפס המוחלט מספר המצבים הרמה \(k\) נתון על ידי פונקציית הביסייד:
$$n_{k}=\Theta(E_{F}-E_{k})$$

\end{proposition}
\begin{proof}
באפס המוחלט נקבל כי \(\beta\to \infty\) ולכן מספר המצבים ברמה \(k\) נתונה על ידי:
$$n_{\mathbf{k}}={\frac{1}{\mathrm{e}^{\beta(E_{\mathbf{k}}-\mu)}+1}}=\Theta(\mu-E_{k})=\Theta(E_{\mathrm{F}}-E_{k})$$

\end{proof}
\begin{corollary}
באפס המולט מספר המצבים יהיה:
$$N=\int_{0}^{k_{\mathrm{F}}}g({\boldsymbol{k}})\,\mathrm{d}^{3}k,$$
כאשר \(k\) זה ווקטור הגל של פרמי(Fermi wave vector) ומוגדר על ידי:
$$E_{\mathrm{F}}={\frac{\hbar^{2}k_{\mathrm{F}}^{2}}{2m}}$$

\end{corollary}
\begin{proposition}[קירוב סומרפלד]
כאשר \(\beta\gg 1\) מתקיים:
$$\int_{-\infty}^{\infty} H\left( \varepsilon \right)f_{FD}\left( \varepsilon \right) \, d\varepsilon = \int_{-\infty}^{\infty}{\frac{H\left( \varepsilon \right)}{e^{\beta\left( \varepsilon-\mu \right)}+1}}\,\mathrm{d}\varepsilon=\int_{-\infty}^{\mu}H\left( \varepsilon \right)\,\mathrm{d}\varepsilon+{\frac{\pi^{2}}{6}}\left({\frac{1}{\beta}}\right)^{2}H^{\prime}\left( \mu \right)+O\left({\frac{1}{\beta\mu}}\right)^{4}$$

\end{proposition}
\begin{proof}
נתחיל מלבצע החלפת משתנים \(\tau x=\varepsilon-\mu\):
$$I=\int_{-\infty}^{\infty}{\frac{H(\varepsilon)}{e^{\beta(\varepsilon-\mu)}+1}}\,\mathrm{d}\varepsilon=\tau\int_{-\infty}^{\infty}{\frac{H(\mu+\tau x)}{e^{x}+1}}\,\mathrm{d}x$$
נחלק את תחום האינטגרציה \(I=I_{1}+I_{2}\) באופן הבא:
$$I=\underbrace{\tau\int_{-\infty}^{0}{\frac{H(\mu+\tau x)}{e^{x}+1}}\,\mathrm{d}x}_{I_{1}}+\underbrace{\tau\int_{0}^{\infty}{\frac{H(\mu+\tau x)}{e^{x}+1}}\,\mathrm{d}x}_{I_{2}}$$
נכתוב את \(I_{1}\) בעזרת החלפת משתנה \(x\mapsto-x\):
$$I_{1}=\tau\int_{-\infty}^{0}{\frac{H(\mu+\tau x)}{e^{x}+1}}\,\mathrm{d}x=\tau\int_{0}^{\infty}{\frac{H(\mu-\tau x)}{e^{-x}+1}}\,\mathrm{d}x$$
כאשר כיוון ש:
$${\frac{1}{e^{-x}+1}}=1-{\frac{1}{e^{x}+1}}$$
נקבל:
$$I_{1}=\tau\int_{0}^{\infty}H(\mu-\tau x)\,\mathrm{d}x-\tau\int_{0}^{\infty}{\frac{H(\mu-\tau x)}{e^{x}+1}}\,\mathrm{d}x$$
נחזור למשתנים המקוריים על \(-\tau \mathrm{d}x=\mathrm{d}\varepsilon\) בגורם הראשון של \(I_{1}\) ונשלב את \(I=I_{1}+I_{2}\) כך שנקבל:
$$I=\int_{-\infty}^{\mu}H(\varepsilon)\,\mathrm{d}\varepsilon+\tau\int_{0}^{\infty}{\frac{H(\mu+\tau x)-H(\mu-\tau x)}{e^{x}+1}}\,\mathrm{d}x$$
כעת ניתן לקרב את המונה בעזרת הנגזרת בהנחה ש-\(\tau\) מספיק קטן:
$$\Delta H=H(\mu+\tau x)-H(\mu-\tau x)\approx2\tau x H^{\prime}(\mu)$$
ונקבל:
$$I=\int_{-\infty}^{\mu}H(\varepsilon)\,\mathrm{d}\varepsilon+2\tau^{2}H^{\prime}(\mu)\int_{0}^{\infty}\frac{x\mathrm{d}x}{e^{x}+1}$$
כאשר האניטגרל האחרון הוא ידוע ושווה ל:
$$\int_{0}^{\infty}{\frac{x\mathrm{d}x}{e^{x}+1}}={\frac{\pi^{2}}{12}}$$
ולכן נקבל סה"כ:
$$I=\int_{-\infty}^{\infty}\frac{H(\varepsilon)}{e^{\beta(\varepsilon-\mu)}+1}\,\mathrm{d}\varepsilon\approx\int_{-\infty}^{\mu}H(\varepsilon)\,\mathrm{d}\varepsilon+\frac{\pi^{2}}{6\beta^{2}}H^{\prime}(\mu)$$

\end{proof}
\begin{example}[מציאת פוטנציאל כימי עם סומפרפלד]
נתון גז פרמיונים עם \(S=\frac{1}{2}\) בתלת מימד. מתקיים:
$$N={{\frac{V}{2\pi^{2}}\left(\frac{2m}{\hbar^{2}}\right)^{3/2}\int_{0}^{\infty}E^{1/2}f(E)\,\mathrm{d}E}}$$
כאשר נרצה לבצע קירוב סומרפלד עם \(H(E)=\sqrt{ E }\). מתקיים:
$$\int_{0}^{\mu}E^{1/2}\,d E={\frac{2}{3}}\mu^{3/2}\qquad H^{\prime}(E)=\frac{d}{d E}
(E^{1/2})=\frac{1}{2}E^{-1/2}$$
ולכן נקבל:
$$N\approx\frac{V}{2\pi^{2}}\left(\frac{2m}{\hbar^{2}}\right)^{3/2}\left[\frac{2}{3}\mu^{3/2}+\frac{\pi^{2}}{12}(k_{B}T)^{2}\mu^{-1/2}\right]$$
כלומר:
$$N\approx \frac{V}{2\pi^{2}}\left(\frac{2m}{\hbar^{2}}\right)^{3/2}\left( \frac{2}{3}\mu^{3/2}+\frac{\pi^{2}}{12}\left( \frac{k_{B}T}{\mu} \right)^{2} \right)$$

\end{example}
\begin{proposition}[חישוב הפוטנציאל הכימי]
משתמשים בנוסחה:
$$N=\int g\left( E \right)f_{FD}\left( E \right) \;\mathrm{d} E $$
ואז מבודדים את \(\mu\) מהתפלגות פרמי דיראק.

\end{proposition}
\begin{proposition}[חישוב אנרגיית פרמי]
יישר מההגדרה:
$$U=\int_{0}^{\infty}  \varepsilon f_{FD}\left( \varepsilon \right)g(E)\, \mathrm{d}\varepsilon= \int_{\varepsilon_{ground}}^{\varepsilon_{FD}}\varepsilon g\left( \varepsilon \right)  \, d\varepsilon  $$
ואז פותרים בשביל \(\varepsilon_{FD}\) בהתאם לאנרגיה.

\end{proposition}
\subsection{גז בוזונים}

\begin{reminder}
עבור גז בוזונים מתקיים:
$$\begin{gather}{N=\int_{0}^{\infty}g\left( \epsilon \right)\left(\exp\left[ \beta\left( \epsilon-\mu \right) \right]-1\right)^{-1}d\epsilon}\\  {U=\int_{0}^{\infty}g\left( \epsilon \right)\,\epsilon\left(\exp\left[ \beta\left( \epsilon-\mu \right) \right]-1\right)^{-1}d\epsilon} 
\end{gather}$$

\end{reminder}
\begin{proposition}
הפוטנציאל הכימי עבור גז בוזונים יהיה קטן או שווה לרמת יסוד

\end{proposition}
\begin{proof}
מספר החלקיקים הממוצע צריך להיות חיובי, ולכן בפרט ההתפלגות בוז אינשטיין צריך להיות חיובי, לכן:
$$\frac{1}{e^{ \beta\left( \varepsilon-\mu \right) }-1}\geq 0\implies e^{ \beta\left( \varepsilon-\mu \right) }-1\geq 0\implies e^{ \beta\left( \varepsilon-\mu \right) }\geq 1$$
ובפרט נכון עבור אנרגיה \(\varepsilon_{0}\) ולכן:
$$e^{ \beta\varepsilon_{0}-\beta \mu }\geq 1\implies \beta \mu \leq \beta\varepsilon_{0}\implies \mu\leq\varepsilon_{0}$$

\end{proof}
\begin{remark}
לרוב מטעמי נוחות קובעים \(\varepsilon_{0}=0\) עבור בוזונים כיוון שקיים אנרגיה אפס של הספינים(בניגוד לפרמיונים).

\end{remark}
\begin{reminder}
צפיפות המצבים של גז אידיאלי הוא:
$$g\left( \epsilon \right)=\frac{V}{4\pi^{2}}\left(\frac{2m}{\hbar^{2}}\right)^{3/2}\epsilon^{1/2}$$

\end{reminder}
\begin{proposition}
מספר החלקיקים עבור גז בוזונים חופשי יהיה:
$$N=\frac{(2S+1)V}{\lambda_{\mathrm{T}}^{3}}\mathrm{Li}_{3/2}(z)\qquad \lambda_{\mathrm{T}}=\frac{h}{\sqrt{2\pi m k_{\mathrm{B}}T}}$$
כאשר \(z=\exp\left( \beta \mu \right)\).

\end{proposition}
\begin{proof}
נציב את הצפופות המצבים של גז אידיאלי בביטוי עבור מספר החלקיקים וכן נבצע החלפת משתנים \(x=\beta\epsilon\) ונקבל:
$$\begin{gather}N=\int_{0}^{\infty}\frac{V}{4\pi^{2}}\left(\frac{2m}{\hbar^{2}}\right)^{3/2}\epsilon^{1/2}\left(\exp\left[ \beta\left( \epsilon-\mu \right) \right]-1\right)^{-1}d\epsilon\\={\frac{V}{4\pi^{2}}}\left({\frac{2m}{\hbar^{2}}}\right)^{3/2}(k_{B}T)^{3/2}\int_{0}^{\infty}x^{1/2}\left(\exp\left( -\beta\mu \right)\exp(x)-1\right)^{-1}d x 
\end{gather}$$
כאשר אם נציב את הפוגסיטי \(z=\exp\left( \beta \mu \right)\) ונקבל:
$$N=\frac{V}{4\pi^{2}}\left(\frac{2m}{\hbar^{2}}\right)^{3/2}\left(k_{B}T\right)^{3/2}\int_{0}^{\infty}x^{1/2}\left(z^{-1}\exp(x)-1\right)^{-1}d x$$
נזהה את הפולילוגוריתם, נכפיל בניוון ספין ונקבל את הטענה.

\end{proof}
\begin{corollary}
כאשר \(T\to 0\) נקבל כי המקדם \(T^{3/2}\to 0\) אבל כיוון שמספר החלקיקים \(N\) נשאר קבוע נדרש כי האינטגרל יתבדר. נשים לב כי אם \(\mu< 0\) נדרש כי \(z^{-1}>1\). לכן האינטגרל חסום על ידי:
$$\int_{0}^{\infty}x^{1/2}\left(\exp(x)-1\right)^{-1}d x={\frac{\sqrt{\pi}}{2}}\zeta\left({\frac{3}{2}}\right)=1.306{\sqrt{\pi}}=2.315$$
ולא מתבדר! קיבלנו סתירה.

\end{corollary}
\begin{corollary}
המשוואה שיפתחנו לא תהיה נכונה עבור טמפרטורות נמוכות מ-\(T_c\) אשר מתקבל על ידי קבועת האינטגרל לערך המקסימלי שלו אשר נקבע על ידי:
$$k_{B}T_c=\left(\frac{2\pi\hbar^{2}}{m}\right)\left(\frac{N}{2.612V}\right)^{2/3}$$
כאשר \(T_c\) נקראת טמפרטורת אינשטיין, ולעיתים מסומנת ב-\(T_{E}\).

\end{corollary}
\begin{proof}
$$N=\frac{V}{4\pi^{2}}\left(\frac{2m}{\hbar^{2}}\right)^{3/2}\left(k_{B}T_c\right)^{3/2}1.306\sqrt{\pi}$$
כאשר ניתן לכתוב גם בצורה הבאה:
$$N=2.315\frac{V}{4\pi^{2}}\left(\frac{2m}{\hbar^{2}}\right)^{3/2}(k_{B}T_c)^{3/2}$$
וכעת ניתן להעביר אגפים ולקבל:
$$k_{B}T_c=\left(\frac{2\pi\hbar^{2}}{m}\right)\left(\frac{N}{2.612V}\right)^{2/3}$$

\end{proof}
\begin{proposition}
התיקון הנדרש למשוואה:
$$N=\int_{0}^{\infty}g\left( \epsilon \right)\left(\exp\left[ \beta\left( \epsilon-\mu \right) \right]-1\right)^{-1}d\epsilon$$
בטמפרטורות נמוכות הוא:
$$N=N_{0}+\int_{0}^{\infty}g(\epsilon)\left(\exp[\beta(\epsilon-\mu)]-1\right)^{-1}d\epsilon$$
כאשר \(N_{0}\) זה מספר החלקיקים ברמת יסוד כך שמתקיים \(N=N_{0}+N_{1}\).

\end{proposition}
\begin{proof}
ננסה להסביר את המקור של הבעיה. מספר החלקיקים ברמת אנרגיה \(\epsilon\) תהיה:
$$f_{B E}\left( \epsilon \right)=\left\langle  n_{\epsilon} \right\rangle=\frac{1}{e^{\beta(\epsilon-\mu)}-1}$$
כאשר ברמת היסוד נקבל:
$$f_{B E}(0)=\langle n_{0}\rangle=N_{0}=\frac{1}{e^{-\beta\mu}-1}$$
ואם \(\mu\to 0\) נקבל כי מספר החלקיקים ברמת יסוד מתבדר. כלומר \(N_{0}\to \infty\). וכדי להתחשב בזה משוואה נגדיר את \(N_{0}\) בתור מספר החלקיקים ברמת יסוד.

\end{proof}
\begin{definition}[עיבוי בוז אינשטיין]
המצב שבה המערכת נמצאת למתחת לטמפרטורה הקריטית ולכן \(N_{0}> 0\). במצב זה נקבל \(\mu=\varepsilon_{0}\) כיוון שלא נדרשת אנרגיה כלל כדי לאכלס עוד חלקיקים ברמת יסוד.

\end{definition}
\begin{corollary}
עבור בוזונים חופשיים נקבל:
$$N=N_{0}+{\frac{V}{4\pi^{2}}}\left({\frac{2m}{\hbar^{2}}}\right)^{3/2}(k_{B}T)^{3/2}\int_{0}^{\infty}x^{1/2}\left(\lambda^{-1}\exp(x)-1\right)^{-1}d x=N_{0}+\frac{V}{\lambda_{T}}\mathrm{Li}_{3 / 2}(z)$$

\end{corollary}
\begin{remark}
אם מוסיפים ריבוי מספין נקבל:
$$N_{1}=\frac{(2S+1)V}{[\lambda_{\mathrm{T}}(T)]^{3}}\mathrm{Li}_{3/2}(z)$$

\end{remark}
\begin{corollary}
עבור גז חופשי ניתן ניתן לכתוב את מספר החלקיקים בעזרת טמפרטורת אינשטיין ולקבל:
$$N=N_{0}+N\left(\frac{T}{T_c}\right)^{3/2}\implies N_{0}=N\left[1-\left(\frac{T}{T_c}\right)^{3/2}\right]$$

\end{corollary}
\begin{remark}
נשים לב כי כאשר \(T\to T_c\) מלמטה אז \(N_{0}\to 0\). כלומר אם \(T>T_c\) נקבל \(N_{0}\ll N\) ולכן ניתן להזניח את האפקט בקירוב טוב בטמפרטורות יותר גבוההות.

\end{remark}
\begin{proposition}[תלות של פוטנציאל כימי בטמפרטורה]
הפוטנציאל הכימי יהיה בקירוב נתון על ידי:
$$\mu\approx-\frac{k_{B}T}{N}\left[1-\left(\frac{T}{T_c}\right)^{3/2}\right]^{-1}$$

\end{proposition}
\begin{proof}
מהמשוואה:
$$\frac{N_{0}}{N}=\left[1-\left(\frac{T}{T_c}\right)^{3/2}\right]$$
כאשר אם נציב את הביטוי:
$$N_{0}=(\exp[-\beta\mu]-1)^{-1}$$
נקבל:
$$N_{0}=\left[\exp(-\beta\mu)-1\right]^{-1}=N\left[1-\left({\frac{T}{T_c}}\right)^{3/2}\right]$$
וכיוון ש-\(\beta \mu\ll 1\) מתחת ל-\(T_c\) ניתן לפתח את \(-\beta \mu\) בטור טיילור ולקבל:
$$\mu\approx-\frac{k_{B}T}{N}\left[1-\left(\frac{T}{T_c}\right)^{3/2}\right]^{-1}$$

\end{proof}
\begin{proposition}[אנרגיה של גז בוזון אידיאלי]
$$U=1.7826\frac{V}{4\pi^{2}}\left(\frac{2m}{\hbar^{2}}\right)^{3/2}(k_{B}T)^{5/2}$$

\end{proposition}
\begin{proof}
הפיתוח דומה לפיתוח עבור מספר החלקיקים. נכתוב:
$$U=U_{0}+\int_{0}^{\infty}f_{B E}(\epsilon)\epsilon\left(\exp[\beta(\epsilon-\mu)]-1\right)^{-1}\epsilon\,d\epsilon$$
כאשר עבור משוואה זו נקבל כי \(U_{0}=0\) כיוון שהאנרגיה של מצב היסוד הוא 0. כעת נגדיר כמו מקודם \(x=\beta\epsilon\) ונכתוב:
$$U=\frac{V}{4\pi^{2}}\left(\frac{2m}{\hbar^{2}}\right)^{3/2}(k_{B}T)^{5/2}\int_{0}^{\infty}x^{3/2}\left(z^{-1}\exp(x)-1\right)^{-1}d x$$
כאשר האינטגרל נתון על ידי:
$$\int_{0}^{\infty}x^{3/2}\left(\exp(x)-1\right)^{-1}d x=\mathrm{Li}_{5 / 2}(1)=\zeta\left({\frac{5}{2}}\right)\Gamma\left({\frac{5}{2}}\right)=1.341\left({\frac{3}{4}}\right)\pi^{1/2}=1.7826$$
ולכן נקבל:
$$U=1.7826\frac{V}{4\pi^{2}}\left(\frac{2m}{\hbar^{2}}\right)^{3/2}(k_{B}T)^{5/2}$$

\end{proof}
\begin{corollary}
האנרגיה לחלקיק תהיה:
$${\frac{U}{N}}=\left({\frac{1.7826}{2.315}}\right)\left({\frac{T}{T_c}}\right)^{3/2}k_{B}T=.7700\,\left({\frac{T}{T_c}}\right)^{3/2}k_{B}T$$

\end{corollary}
\begin{corollary}
קיבול החום סגולי בנפח קבוע מתקבל גזירה האנרגיה לחלקיק לפי טמפרטורה:
$$c_{V}=1.925\,k_{B}\,\left(\frac{T}{T_c}\right)^{3/2}$$

\end{corollary}
\begin{proposition}[מציאת הטמפרטורה הקריטית]
  \begin{enumerate}
    \item נמצא את מספר החלקיקים במצב המעורער: 
$$N_{\mathrm{exc}}=\int_{\epsilon>0}g(\epsilon)f_{\mathrm{BE}}(\epsilon)\,d\epsilon$$


    \item מניחים \(\mu\to 0\).  


    \item מבודדים את הטמפרטורה. אם ניתן לקבל ביטוי גדול מ-0, אז קיים עיבוי בוז אינשטיין. אחרת לא ייתכן 


  \end{enumerate}
\end{proposition}
\begin{example}[עבוי בוז אינשטיין בגז אידיאלי דו מימדי]
\end{example}
\section{מודל איזינג}

\subsection{אינטראקציה אחידה}

\begin{definition}[המילטוניאן של מודל איזינג]
יהיה מהצורה:
$$H=-\sum_{\langle i,j\rangle}J_{ij}s_{i}s_{j}-h\sum_{i}s_{i}$$
כאשר \(\langle i,j \rangle\) מסמל סכימה על זוגות ו-\(J\) זה האינטרקציה בין הספינים.

\end{definition}
\begin{definition}[אינטרקציה אחידה]
ההנחה שבה כל שספין מבצע אינטרקציה פרומגנטית זהה בעוצמתה. כלומר:
$$J_{i j}=-\frac{J}{N}\qquad J> 0$$

\end{definition}
\begin{corollary}
תחת ההנחה של אינטרקציה אחידה ניתן לכתוב את ההמילטוניאן של מודל איזינג על ידי:
$$H=-J\sum_{\langle i,j\rangle}s_{i}s_{j}-h\sum_{i}s_{i}$$
כאשר \(\langle i,j \rangle\) מסמל סכימה על זוגות ו-\(J\) זה האינטרקציה בין הספינים.

\end{corollary}
\begin{definition}[מגנטיזציה ממוצעת]
הערך הממוצע של הספין המערכת, כלומר:
$$m=\frac{1}{N}\sum_{i=1}^{N}s_{i}$$

\end{definition}
\begin{corollary}
מגנטיזציה ממוצעת בצבר הקנוני נתונה על ידי:
$$m=-\frac{1}{N}\frac{\partial F}{\partial h} $$

\end{corollary}
\begin{proof}
$$d F=-P d V-S d T-M d H\implies m=-\frac{1}{N}\frac{\partial F}{\partial h}$$

\end{proof}
\subsection{שכנים קרובים}

\begin{reminder}
ההמילטוניאן של מודל איזינג עם אינטרקציה אחידה יהיה מהצורה:
$$H=-J\sum_{\langle i,j\rangle}s_{i}s_{j}-h\sum_{i}s_{i}$$
כאשר \(\langle i,j \rangle\) מסמל סכימה על זוגות ו-\(J\) זה האינטרקציה בין הספינים.

\end{reminder}
\begin{definition}[מודל איזינג באינטרקציית שכנים קרובים]
האילוץ על מודל איזינג שעבורו סוכמים רק על השכנים הקרובים ולא על כל האינטרקציות בין זוגות בשריג.

\end{definition}
\begin{proposition}
לכל חלקיק יש 2d שכנים קרובים כאשר \(d\) זה המימד.

\end{proposition}
\begin{definition}[קירוב שדה ממוצע]
ההנחה כי לכל ספין יש את אותה מגנטיזציה. כלומר לכל ספין יש מגנטיזציה ממוצעת \(m\).

\end{definition}
\begin{proposition}
סך האינטרקציה המורגשת יהיה \(m\) כפול מספר השכנים הקרובים שלו, כלומר:
$$E_{i}=-\left(2d J m+h\right)s_{i}$$

\end{proposition}
\begin{proposition}
האנרגיה של כל המערכת תהיה:
$$E=-N J d m^{2}-N h m$$

\end{proposition}
\begin{proof}
האנרגיה הכוללת תהיה הסכום של האנרגיה של כל החלקיקים:
$$E=-J\cdot{\frac{1}{2}}\cdot2d m\sum_{i}s_{i}-h\sum_{i}s_{i}=-J d m\cdot N m-h N m=-N J d m^{2}-N h m$$
כאשר הפקטור חצי מונע ספירה כפולה.

\end{proof}
\begin{corollary}
פונקציית החלוקה החד חלקיקית תהיה:
$$Z_{1}=2\cosh\left(\beta\left(2d J m+h\right)\right)$$

\end{corollary}
\begin{proof}
$$Z_{1}=\sum_{s_{i}=\pm1}e^{-\beta E_{i}}=e^{\beta(2d J m+h)}+e^{-\beta(2d J m+h)}=2\cosh\left(\beta\left(2d J m+h\right)\right)$$

\end{proof}
\begin{corollary}
הספין הממוצע יהיה:
$$\langle s_{i}\rangle=\sum_{s_{i}=\pm1}s_{i}p\left(s_{i}\right)=\frac{1}{Z}\sum_{s_{i}\pm1}s_{i}e^{-\beta E_{i}}=\frac{e^{\beta\left(2d J m+h\right)\cdot1}-e^{\beta\left(2d J m+h\right)\cdot\left(-1\right)}}{2\cosh\left(\beta\left(2d J m+h\right)\right)}=\tanh\left(\beta\left(2d J m+h\right)\right)$$

\end{corollary}
\begin{proof}
$$\langle s_{i}\rangle=\sum_{s_{i}=\pm1}s_{i}p\left(s_{i}\right)=\frac{1}{Z}\sum_{s_{i}\pm1}s_{i}e^{-\beta E_{i}}=\frac{e^{\beta\left(2d J m+h\right)\cdot1}-e^{\beta\left(2d J m+h\right)\cdot\left(-1\right)}}{2\cosh\left(\beta\left(2d J m+h\right)\right)}=\tanh\left(\beta\left(2d J m+h\right)\right)$$

\end{proof}
\subsection{שרשרת איזינג}

\begin{reminder}[מודל איזינג]
מודל של סריגים המתואר על ידי האנרגיה הבאה:
$$H=-J\sum_{\langle j,k\rangle}\sigma_{j}\sigma_{k}-h\sum_{j=1}^{N}\sigma_{j}$$
כאשר \(\langle j,k \rangle\) מתאר סכימה על כל הזוגות ו-\(\sigma_{i}\in\{ -1,1 \}\) מתאר את הספין באתר ה-\(i\) של השריג, ו-\(J>0\) קבוע.

\end{reminder}
\begin{definition}[שרשרת איזינג]
מערכת שבה כל ספין מבצע אינטרקציה רק עם האיבר אשר נמצא לידו, ולכן ניתן לכתיבה על ידי:
$$H=-J\sum_{j}\sigma_{j}\sigma_{j+1}-h\sum_{j}\sigma_{j}$$
כאשר ניתן להגדיר שתי סוגי תנאי שפה:

  \begin{enumerate}
    \item תנאי שפה חופשי - הסכום הראשון יהיה מ-\(j=1\) עד \(j=N-1\). 


    \item תנאי שפה מחוזרי - מוגדר \(\sigma_{N+1}=\sigma_{1}\) כאשר הסכום הראשון יהיה מ-\(j=1\) עד \(j=N\). 


  \end{enumerate}
\end{definition}
\begin{proposition}[שרשרת איזינג כאשר \(J=0\)]
פונקציית החלוקה הקנונית תהיה:
$$Z=\prod_{j=1}^{N} Z_{j}\equiv \prod_{j=1}^{N}2\cosh\left( \beta h \right)$$
המגנטיזציה הממוצעת תהיה:
$$m=\left\langle  \sigma_{j}  \right\rangle =\tanh\left( \beta h \right)$$
האנרגיה הממוצעת תהיה:
$$U=\langle H\rangle=-h N\operatorname{tanh}(\beta h)$$

\end{proposition}
\begin{proof}
ניתן לכתוב את ההמילטוניאן על ידי:
$$H=-h\sum_{j=1}^{N}\sigma_{j}$$
כאשר הפונקציית החלוקה הקנונית תהיה נתונה על ידי:
$$Z=\sum_{\{\sigma\}}\exp[-\beta(-h\sum_{j=1}^{N}\sigma_{j})]$$
כאשר מפקטוריזציה של פונקציית החלוקה נקבל:
$$Z=\sum_{\{\sigma\}}\prod_{j=1}^{N}\exp[\beta h\sigma_{j}]=\prod_{j=1}^{N}\sum_{\sigma_{j}}\exp[\beta h\sigma_{j}]=\prod_{j=1}^{N}Z_{j}$$
כאשר:
$$Z_{j}=\sum_{\sigma_{j}}\exp[\beta h\sigma_{j}]$$
זה עושה רידוקציה לבעיית ה-\(N\) ספינים לבעיה של \(N\) ספינים לבעיה של \(N\) ספינים זההים אשר מכילים סכום רק על שתי מצבים. כליוון ש-\(\sigma_{j}=\pm 1\) ניתן לכתוב את הסכומים בצורה מפורשת:
$$Z_{j}=\exp(\beta h)+\exp(-\beta h)=2\cosh(\beta h)$$
כאשר קל להראות שהמגנטיזציה הממוצעת תהיה:
$$m=m_{j}=\langle\sigma_{j}\rangle={\frac{\exp(\beta h)-\exp(-\beta h)}{\exp(\beta h)+\exp(-\beta h)}}=\operatorname{tanh}(\beta h)$$
ולכן האנרגיה הממוצעת תהיה:
$$U_{j}=\langle-h\sigma_{j}\rangle=-h m_{j}=-h m=-h\operatorname{tanh}(\beta h)$$
כאשר האנרגיה של כל השרשרת איזינג תתקבל על ידיד הכפלה ב-\(N\):
$$U=\langle H\rangle=-h\sum_{j=1}^{N}\langle\sigma_{j}\rangle=-h N m=-h N\operatorname{tanh}(\beta h)$$
והקיבול חום הסגולי נתון על ידי גזירה לפי הטמפרטורה:
$$c=\frac{1}{N}\frac{\partial U}{\partial T}=\frac{-1}{N k_{B}T^{2}}\frac{\partial U}{\partial\beta}=k_{B}\beta^{2}h^{2}\mathrm{sech}^{2}(\beta h)$$

\end{proof}
\begin{proposition}[שרשרת איזינג עם \(h=0\)]
\end{proposition}
\begin{proof}
ניתן לכתוב את ההמילטוניאן על ידי:
$$H^{\prime}=-J\sum_{j=1}^{N-1}\sigma_{j}\sigma_{j+1}$$
נגדיר \(\tau_{1}=\sigma_{1}\) ולכל \(2\leq j\leq N\) נגדיר \(\tau_{j}=\sigma_{j-1}\sigma_{j}\). ניתן לכתוב את ההמילטוניאן כעת על ידי:
$$H=-J\sum_{j=2}^{N}\tau_{j}$$
וזהו כמעט זהה למילטוניאן זהה למקרה \(h\neq 0,J=0\) אשר פתרנו בטענה הקודמת. ניתן על ידי איחוד גורמים לקבל:
$$Z=\sum_{\{\tau\}}\prod_{j=1}^{N}\exp[\beta J\tau_{j}]=2\prod_{j=2}^{N}\sum_{\tau_{j}}\exp[\beta J\tau_{j}]=2\prod_{j=2}^{N}Z_{j}^{\prime}$$
כאשר סכמנו גם על \(\sigma_{1}=\pm 1\) ומשם מגיע הגורם של 2. נקבל:
$$Z_{j}=\sum_{\tau_{j}=\pm1}\exp[\beta J\tau_{j}]=\exp[\beta J]-\exp[-\beta J\tau_{j}]=2\cosh(\beta J)$$
וגם:
$$Z=2(Z_{j})^{N-1}=2\left(2\cosh(\beta J)\right)^{N-1}$$
כאשר מלוגוריתם של פונקציית החלוקה נקבל:
$$-\beta F=\ln Z=\ln2+(N-1)\ln Z_{j}=\ln2+(N-1)\ln\left(2\cosh(\beta J)\right)$$
וכן האנרגיה של המערכת תהיה:
$$U=-(N-1)J\operatorname{tanh}(\beta J)$$
כאשר קיבול החום יהיה:
$$c=k_{B}\beta^{2}J^{2}\mathrm{sech}^{2}(\beta J)\left(1-1/N\right)\approx k_{B}\beta^{2}J^{2}\mathrm{sech}^{2}(\beta J)$$

\end{proof}
\end{document}