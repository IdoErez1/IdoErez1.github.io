\documentclass{tstextbook}

\usepackage{amsmath}
\usepackage{amssymb}
\usepackage{graphicx}
\usepackage{hyperref}
\usepackage{xcolor}

\begin{document}

\title{Example Document}
\author{HTML2LaTeX Converter}
\maketitle

\chapter{הגדרות בסיסיות}

\section{פונקציות מצב ומסלול}

\begin{definition}[תכונה אינטנסיבית]
תכונה שאינה משתנה אם משכפלים את המערכת. דוגמאות לגדלים אינטנסיביום יהיו טמפרטורה וצפיפות

\end{definition}
\begin{definition}[תכונה אקסטנסיבי]
תכונה של המערכת אשר פרופרציונאלי לשיכפול המערכת. כלומר אם נשכפל את מערכת אז נצפה כי גודל אקסטנסיבי יגדל פי 2. דוגמאות זה מסה, נפח ואנטרופיה.

\end{definition}
\begin{proposition}
גודל שהוא יחס של גדלים אקסטניביים הוא גודל אינטנסיבי.

\end{proposition}
\begin{definition}[פונקציות מצב]
תכונה אשר תלוייה אך ורק במצב של המערכת ולא איך המערכת הגיעה עליה. דוגמאות לפונקציות מצב הם נפח, אנרגיה פנימית, לחץ ואנטרופיה

\end{definition}
\begin{definition}[פונקציית מסלול]
תכונה אשר תלוייה באיך התהליך התבצעה, ולא רק על ידי המצב שלה ברגע נתון. דוגמאות לפונקציות מסלול הם עבודה וחום.

\end{definition}
התיאור המתמטי של תכונות אלו נעשה באמצעות דיפרנציאלים.

\section{דיפרנציאלים}

\begin{definition}[דיפרנציאל]
אופרטור \(d\) המתאר שינוי קטן בגודל.

\end{definition}
\begin{symbolize}
נכתוב \((f)_{y}\) כדי להדגיש שאנחנו משאירים את \(y\) קבוע.

\end{symbolize}
\begin{proposition}
עבור פונקציה \(f=f(x,y)\) ניתן לכתוב את הדיפרנציאל של \(f\) ע"י:
$$d\!f=\left({\frac{\partial f}{\partial x}}\right)_{y}d x+\left({\frac{\partial f}{\partial y}}\right)_{x}d y$$

\end{proposition}
\begin{definition}[דיפרנציאל מדוייק]
ביטוי מהצורה:
$$h(x,y)=M(x,y)dx+N(x,y)dy$$
נקרא דיפרציאל מדוייק אם קיים פונקציה \(f=f(x,y)\) כך ש- \(df=h(x,y)\)

\end{definition}
\begin{proposition}
עבור ביטוי מהצורה ביטוי מהצורה \(h(x,y)=M(x,y)dx+N(x,y)dy\) התנאים הבאים שקולים:

  \begin{enumerate}
    \item הביטוי \(h\) דיפרנציאל מדוייק 


    \item מתקיים בתחום פשוט קשר: 
$$\frac{\partial M}{\partial y} =\frac{\partial N}{\partial x} $$


    \item האינטגרל המסלולי \(\int_{\gamma(a)}^{\gamma(b)}h(x,y)\) יהיה תלוי רק ב-\(a,b\) לכל מסלול \(\gamma\). 


    \item האינטגרל על כל מסלול סגור הוא 0. 


    \item הנגזרות מתחלפות. כלומר: 
$$\frac{\partial^2 f}{\partial x\partial y}=\frac{\partial ^2f}{\partial y\partial x}  $$


  \end{enumerate}
\end{proposition}
\begin{corollary}
פונקציית מצב מתוארת ע"י דיפרנציאל מדוייק

\end{corollary}
\begin{definition}[דיפרנציאל לא מדוייק]
גודל מהצורה \(h(x,y)=M(x,y)dx+N(x,y)dy\) אשר אינו דיפרנציאל מדוייק נקרא דיפרנציאל לא מדוייק.

\end{definition}
\begin{symbolize}
דיפרנציאל לא מדוייק מסומן ב-\(\;\bar{}\mkern-7.5mu df\). כלומר מתקיים \(\;\bar{}\mkern-7.5mu df=h(x,y)\) .

\end{symbolize}
\begin{corollary}
פונקציה המתוארת ע"י דיפרנציאל לא מדוייק בהכרח תלוייה במסלול. לכן פונקציית מסולול מתוארת ע"י דיפרנציאל לא מדוייק.

\end{corollary}
\begin{proposition}[נוסחת המכפלה המשולשת]
עבור פונקציה \(f(x,y,z)=0\) מתקיים:
$$-1=\left({\frac{\partial z}{\partial y}}\right)_{x}\left({\frac{\partial y}{\partial x}}\right)_{z}\left({\frac{\partial x}{\partial z}}\right)_{y}$$

\end{proposition}
\section{משאוות מצב}

\begin{definition}[מול]
יחידת מידה חסרת מימדים. שווה למספר אבוגדרו:
$$1\;{\mathrm{mol}}=N_{A}=6.022\times10^{23}$$
זהו מספר האטומים ב-12 גרם של פחמן 12.

\end{definition}
\begin{definition}[קלווין]
יחידה למדידה טמפרטורה. מסומנת \(^\circ K\).

\end{definition}
\begin{definition}[משוואת מצב]
משוואה המקשרת בין לחץ, נפח, טמפרטורה וכמות החלקיקים. כלומר משוואה מהצורה \(f\left(p,V,T,N \right)=0\)

\end{definition}
\begin{theorem}[משוואת גז אידיאלי]
$$pV = Nk_{B}T$$
כאשר \(k\) זה קבוע בולצמן, \(N\) מתאר את מספר החלקיקים.

\end{theorem}
\begin{remark}
משוואה זו נכונה לשיווי משקל. לא ניתן אפילו להגדיר בצורה טובה את המשתנים כשאנחנו לא בשיווי משקל.

\end{remark}
בנוסף, אנחנו מניחים כי אין אינטרקציה בין החלקיקים(או לפחות ניתן להזניח אותה). לכן אם נערבב חלקיקים משני סוגים שונים, השילוב יקיים את משוואת הגז האידיאלי רק אם אין אינטרקציה בינהם, פרט להתנגשויות קשיחות בין החלקיקים. הנחה חשובה זה שלכל חלקיק יש אנרגיה קינטית, אבל אין אנרגיה פוטנציאלית.

ניתן להשתמש בגז אידיאלי על גופים רבים - לא מוגבל לגזים. 

\begin{example}
נתאר את פרופיל הלחץ(כתלות בגובה) של אטמוספרה בטמפרטורה קבועה \(T\), ונניח שהאוויר באטמוספרה ניתן לתיאור ע"י גז אידיאלי.
נדמיין עמודת אוייר בעל שטח חתך \(A\) וגובה \(dz\). מכיוון שהאטמוספרה בשיווי משקל נקבל משוואת כוחות:
$$0=A\left[p\left(z\right)-p\left(z+d z\right)\right]-n\mu g$$
כעת נשתמש במשוואת הגז האידיאלי בשביל \(n\) ובזה ש-\(V=Adz\) ונקבל:
$$p^{\prime}\left(z\right)=\frac{p\left(z+d z\right)-p\left(z\right)}{d z}=-\frac{\mu g}{R T}p\implies p=p_{0}e^{-\mu gz/RT}$$

\end{example}
\begin{definition}[גז וואן דר וואלס]
מודל אלטרנטיבי המתאר גזים מסויימים. משוואת המצב תהיה:
$$\left(p+a\frac{N^{2}}{V^{2}}\right)\left(V-N b\right)\,=\,N k_{B}T$$
כאשר \(T\) זה טמפרטורה(בקלווין), \(k_{B}\) זה קבוע בולצמן, \(N\) זה מספר החלקיקים, \(V\) הנפח, \(p\) הלחץ ו-\(N\) מספר החלקיקים.

\end{definition}
משוואה זו מניחה כי יש התנגשות אלסטית מושלמת בין החלקיקים.

\subsection{מוצקים ונוזלים}

נוזלים ומוצקים בניגוד לגזים לא ממלאים את הכלי שבו הם נמצאים, וגדלים באופן פרופורציוני לשינוי בטמפרטורה.

\begin{definition}[מקדם התפשוטות תרמי לינארי]
לנוזל או מוצק יש מקדם התפשטות אורכי \(\alpha\) המוגדר:
$$\alpha\!=\!{\frac{1}{L}}\!\left({\frac{d L}{d T}}\right)$$

\end{definition}
\begin{definition}[מקדם התפשטות תרמי נפחי]
בנוזל או למוצק יש מקדם התפשטות נפחי \(\beta\) המוגדר:
$$\beta\!=\!{\frac{1}{V}}\!\left({\frac{d V}{d T}}\right)$$

\end{definition}
\begin{proposition}
עבור מוצקים, המקדם ההתפשטות הלינארי והנפחי קשורים אחד לשני כך ש:
$${\beta}\approx 3\alpha$$

\end{proposition}
\section{חוק האפס של תרמודינמיקה}

\begin{definition}[מערכת תרמודינמית]
איזור שמכיל אנרגיה או חומר 

\end{definition}
\begin{definition}[סביבה]
איזור אשר מבצע אינטרקציה עם המערכת התרמודינמית

\includegraphics[width=0.8\textwidth]{diagrams/svg_1.svg}
\end{definition}
\begin{definition}[מערכת פתוחה]
מערכת תרמודינמית אשר מחליפה של חלקיקים עם הסביבה.

\end{definition}
\begin{definition}[מערכת סגורה]
מערכת תרמודינמית אשר לא מחליפה חלקיקים עם הסביבה.

\end{definition}
\begin{definition}[מערכת מבודדת]
מערכת תרמודינמית אשר לא מושפעת כלל מהסביבה.

\end{definition}
\begin{definition}[גז]
אוסף חלקיקים אשר מתאימים את עצמם לצורה של כלי שבה הם נמצאים.

\end{definition}
\begin{definition}[שיווי משקל]
בצורה מיקרסוקופית החלקיקים עדיין נעים. אבל בצורה המאקרוסקופי נקבל שהרבה גדלים נשמרים בזמן. גודל שנשמר בזמן למערכת נקרא בשיווי משקל.

\end{definition}
\begin{definition}[לחץ]
הכוח הממוצע שפועל ליחידת שטח. מסומן ע"י \(P\) ומתאור ע"י:
$$P= \frac{F}{A}$$
ובעל יחידות של \(\frac{N}{m^2}=\frac{kg}{m\cdot s^2}\equiv Pa\)

\end{definition}
\begin{definition}[צפיפות]
המסה ליחידת נפח. פרופורציונאלי לכמות החלקיקים.

\end{definition}
\begin{definition}[מגע תרמי]
כאשר יש מעבר של חלקיקים בין המערכות.

\end{definition}
\begin{definition}[חום]
האנרגיה המועברת בין המערכות במגע תרמי.

\end{definition}
\begin{definition}[טמפרטורה - זמנית]
גודל המאפיין את המערכת ומשתווה בין שתי מערכות המגיעות לשיווי משקל תרמי.

\end{definition}
\begin{remark}
למעשה חום זה האנרגיה שמתקבל מהבדל טמפרטורה.

\end{remark}
\begin{theorem}[חוק האפס של תרמודינמיקה]
אם \(B\leftrightarrow C\) באותו טמפרטורה ו- \(B\leftrightarrow A\) באותה טמפרטורה אזי \(A\leftrightarrow C\) באותה טמפרטורה.

\end{theorem}
\begin{remark}
ממשפט זה אנו מבינים כי למעשה התפקיד של טמפרטורה כרגע זה להכריע אם מערכת נמצאת בשיווי משקל תרמי, וכמה.

\end{remark}
חוק זה למעשה מגדיר את המושג של שיווי משקל תרמי. העובדה אבל שמערכת נמצאת בשיווי משקל תרמי לא אומר שהמערכת נמצאת בשיווי משקל. נדרש גם לשם כך:

\begin{enumerate}
  \item שיווי משקל מכאני - אין כוחות לא מאוזנים. 


  \item שיווי משקל כימי - אין ראקציות כימיות. 


  \item שיווי משקל דיפוזי - אין זרימה של חומר. 


\end{enumerate}
\begin{proposition}[דרכים לבצע אינטרקציה]
יש 3 דרכים שונות שבהם הסביבה משפיעה על המערכת:

  \begin{enumerate}
    \item מעבר של חומר(חלקיקים) 


    \item מעבר של חום 


    \item שינוי בגודל(נפח) 


  \end{enumerate}
\end{proposition}
\begin{definition}
  \begin{enumerate}
    \item מערכת אשר אין מעבר של חומר נקראת מערכת סגורה. 


    \item מערכת שאין בה מעבר של חום נקראת מערכת אדיאבטית. 


    \item מערכת שאין בה שינוי בגודל נקראת מערכת איזוכורית. 


  \end{enumerate}
\end{definition}
\begin{definition}
  \begin{enumerate}
    \item מערכת שבה הלחץ נשאר קבוע נקראת מערכת איזובארית. 


    \item מערכת שבה החום נשאר קבוע נקרא מערכת איזותרמית. 


  \end{enumerate}
\end{definition}
\chapter{חוקי התרמודינמיקה}

\section{חוק ראשון של תרמודינמיקה}

\begin{definition}[תהליך הפיכה]
תהליך שבו המערכת יכולה לחזור למצב ההתחלתי ללא השפעה חיצונית

\end{definition}
\begin{definition}[תהליך קווזיסטטי]
תהליך שבו בכל שלב המערכת נמצאת בשיווי משקל.

\end{definition}
בדרך כלל מתרחש כאשר המערכת משתנה מאוד לאט - ככה שקצב שינוי המצב של המערכת קטנה משמעותית מהסקלות זמן האופיניות שהמערכת צריכה כדי להגיע לשיווי משקל. 

\begin{remark}
בתהליך שאינו קווזיסטטי המשתנים \(P,T,V\) אינם מוגדרים היטב, כיוון שמוגדרים עבור שיווי משקל.

\end{remark}
\begin{proposition}
תהליך הוא הפיך אם"ם הוא קווזיסטטי.

\end{proposition}
אם המערכת לא קווזיסטטית, אז קצב שינוי במערכת ילא זניח לעומת קצב הזמן שלוקח להגיע לשיווי משקל, והתהליך לא יהיה הפיך.

\begin{proposition}[עבודה של גז]
נניח כי גז עבור ממצב יציב \((P_{1},V_{1})\) למצב יציב חדש \((P_{2},V_{2})\) בתהליך קווזיסטטי. אזי:
$$F=PA\implies dW = PAdx=-PdV\implies dW=-PdV\implies W=-\int_{V_{1}}^{V_{2}}PdV$$

\end{proposition}
\begin{remark}
באופן כללי זה תלוי במשוואת מצב.

\end{remark}
\begin{proposition}
אם \(q>0\) נכנס חום למערכת. 
אם \(q<0\) יוצא חום מהמערכת.
אם \(w>0\) נעשה עבודה על המערכת.
אם \(w<0\) המערכת עושה עבודה.

\end{proposition}
\begin{proposition}[עבודה של גז אדיאלי]
$$W=-\int_{V_{1}}^{V_{2}}PdV=-Nk_{B}T\int_{V_{1}}^{V_{2}} \frac{1}{V}dV=-Nk_{B}T\ln\left( \frac{V_{2}}{V_{1}} \right)$$

\end{proposition}
כאשר בתהילך שאינו קווזיסטטי, נקבל כי \(\;\bar{}\mkern-7.5mu dW\)  לא בהכרח דיפרנציאל מדוייק, ולכן העבודה לא תהיה תלוייה רק בהתחלה ובסוף אלה במסלול כולו.

\begin{proposition}[אנרגיה של גז אידיאלי]
$$U={\frac{N k_{B}T}{2}}f$$
כאשר \(f\) זה מספר דרגות החופש.

\end{proposition}
\begin{proposition}[אנרגיה של גז ואן דר ואלס]
$$U(V,T)=\int C_{V}d T-{\frac{a N^{2}}{V}}+U_{0}$$

\end{proposition}
\begin{theorem}[חוק ראשון של תרמודינמיקה]
האנרגיה נשמרת, כאשר עבודה וחום הם שתי צורות של אנרגיה

\end{theorem}
\begin{corollary}
$$\Delta U=\Delta Q+\Delta W$$
כאשר \(\Delta Q\) זה החום שנכנס למערכת, \(\Delta U\) זה השינוי באנרגיה ו-\(\Delta W\) זה העבודה שמתבצעת. ניתן לכתוב בצורה הדיפרנציאלית:
$$dU=\;\bar{}\mkern-7.5mu dQ+\;\bar{}\mkern-7.5mu dW$$
כאשר הסימון \(\;\bar{}\mkern-7.5mu d\) מדגיש שזהו דיפרנציאל לא מדוייק.

\end{corollary}
\subsection{קיבול חום}

\begin{definition}[קיבול חום]
קיבול חום \(C\) זה למעשה כמות האנרגיה חום(\(Q\)) שנדרש כדי לעלות את הטמפרטורה במעלה אחת. מוגדר:
$$C(p,T)=\operatorname*{lim}_{\Delta T\rightarrow0}\frac{\Delta Q}{\Delta T}=\frac{\mathrm{d}Q}{d T}$$

\end{definition}
\begin{definition}[קיבול חום סגולי ]
קיבול חום ליחידת מסה. כלומר:
$$c=\frac{1}{m}\left( \frac{ đ Q}{d T} \right)$$

\end{definition}
כעת נעבור להגדרות קיבול שרלוונטיות בעיקר עבור גזים.
\textbf{הגדרה} קיבול חום בנפח קבוע
אם לא משנים את הנפח, זה למעשה כמות האנרגיה שצריך כדי לעלות את הטמפרטורה ביחידה אחת:
$$C_{V}=\!\frac{\;\bar{}\mkern-7.5mu dQ_{V}}{d T}\!=\!\left(\frac{\partial U}{\partial T}\right)_{V}$$

\begin{definition}[קיבול חום בלחץ קבוע]
אם לא משנים את הלחץ, זה יהיה כמות האנרגיה הנדרשת כדי לעלות את הטמפרטורה ביחידה אחת.
$$C_{P}={\frac{\mathrm{\;\bar{}\mkern-7.5mu d}Q_{P}}{d T}}$$

\end{definition}
\section{חוק שני של תרמודינמיקה}

\begin{theorem}[הניסוח של קלווין של החוק השני של תרמודימיקה]
לא ייתכן תהליך שהתוצאה היחידה שלו היא המרה מלאה של חום לעבודה.

\end{theorem}
\begin{definition}[אנטרופיה]
החום \(\;\bar{}\mkern-7.5mu dQ\) העוברת למערכת ע"י הטמפרטורה. כלומר:
$$\mathrm{d}S = \frac{\;\bar{}\mkern-7.5mu dQ}{T}$$

\end{definition}
\begin{definition}[הפרש אנטרופיה בין מצבים]
עבור מצבים \(A,B\) ההפרש אנטרופיה יהיה:
$$\Delta S_{AB}=\int_{A}^{B} \frac{\mathrm{d}Q}{T}  $$
כאשר האינטגרל הוא אינטגרל מסלולי על מסלול הפיך. 

\end{definition}
\begin{proposition}
עבור תהליך כללי שהוא לאו דווקא הפיך נקבל:
$$\int_{A}^{B}  \frac{\;\bar{}\mkern-7.5mu dQ}{T}\leq \int _{A}^{B } \frac{\mathrm{d}Q}{T}\equiv\Delta S_{AB}$$
כאשר עבור תהליך אדיאבטי נקבל \(\;\bar{}\mkern-7.5mu dQ=0\) ולכן:
$$0\leq  \int_{A}^{B}  \frac{\mathrm{d}Q}{T}\leq \Delta S_{AB} $$

\end{proposition}
\begin{proposition}[ניסוח שקול לחוק השני לתרמודינמיקה]
עבור מערכת מבודדת מתקיים:
$$\mathrm{d}S \geq \frac{\;\bar{}\mkern-7.5mu dQ}{T}$$
כאשר עבור תהליך הפיך בלבד מתקיים \(\mathrm{d}S = \frac{\mathrm{d}Q}{T}\).

\end{proposition}
\begin{corollary}
עבור מערכת מבודדת תרמית(איזותרמית) מתקיים \(\;\bar{}\mkern-7.5mu dQ=0\) ולכן החוק השני יהיה:
$$\mathrm{d}S \geq 0$$

\end{corollary}
\begin{definition}[תהליך איזואנטרופי]
תהליך שבו האנטרופיה לא משתנה - \(\Delta S=0\).

\end{definition}
\begin{corollary}
אם מתייחסים ליקום כמערכת מבודדת, ניתן לכתוב את שתי החוקים הראשונים של תרמודינמיקה בצורה הבאה:

  \begin{enumerate}
    \item האנרגיה הפנימית של היקום היא קבועה - \(U_{\mathrm{Universe}}=\mathrm{constant}\)


    \item האנטרופיה של היקום יכול רק לגדול 


  \end{enumerate}
\end{corollary}
\begin{proposition}[המשוואה היסודית]
$$\mathrm{d}U=\;\bar{}\mkern-7.5mu dQ+\;\bar{}\mkern-7.5mu dW=T\mathrm{d}S + P\mathrm{d}V$$

\end{proposition}
\begin{proof}
נשתמש בחוק הראשון של תרמודינמיקה ונזכור כי \(dQ = TdS\) וכן \(dW = PdV\) ולכן נקבל את המבוקש.

\end{proof}
\begin{remark}
הכתיבה של האנרגיה הפנימית בעזרת המשתנים \(U=U(S,V)\) נקראים המשתנים הטבעיים. 

\end{remark}
\begin{proposition}
עבור מערכת ציקלית נקבל כי סך האנרגיה הפנימית היא 0, ולכן:
$$\oint T\mathrm{d}S = - \oint P \mathrm{d}V$$

\end{proposition}
\section{פוטנצילים תרמודינמים}

בפיזיקה תרמית, מערכת מתוארת לחלוטין ע"י אוסף הגדלים \(p,V,N,T,S,U, \dots\). הגדלים האלה מקושרים ע"י המשוואת מצב. ניתן תמיד לבחור משתנה אחד ולתאר את האחרים בעזרת פונקציית מצב.

\begin{definition}[פוטנציאל תרמודינמי]
גודל אשר בעזרתו ניתן לתאר את המערכת בצורה מלאה.

\end{definition}
אנחנו מכירים דוגמא אחד של פוטנציאל תרמודינמי - האנרגיה הפנימית \(U\).

\begin{definition}[המשתנים הטבעיים של פוטנציאל תרמודינמי]
המשתנים עבורו הדיפרנציאל יהיה מורכב מדיפרציאלים מדוייקים, אשר כל אחד מהם אינו תלוי במסלול.

\end{definition}
\begin{proposition}[תכונות של פוטנציאל תרמודינמי]
  \begin{enumerate}
    \item מכיל בתוכו את כל פונקציות המצב 


    \item ערכי קיצון קובעים שיווי משקל תרמודינמי. 


    \item נגזרות כאשר מקבעים גדלים נותנות את פונקציות המצב. 


  \end{enumerate}
\end{proposition}
\begin{example}
נמצא את הפוטנציאל התרמודינמי - האנרגיה הפנימית \(U\) של גז אידיאלי. באופן כללי אנו יודעים כי:
$$dU=\;\bar{}\mkern-7.5mu dQ+\;\bar{}\mkern-7.5mu dW =TdS - pdV$$
כאשר \(p=p(V,S)\) ו-\(T=T(V,S)\). כעת נדרש כבר להשתמש במשוואת מצב. אנו יודעים כי עבור גז אידיאלי:
$$pV=Nk_{B}T \implies p(V,S)= \frac{Nk_{B}T(V,S)}{V}$$
כאשר אם נשתמש באנרגיה של גז אידיאלי נקבל:
$$U=\frac{3}{2}Nk_{B}T\implies T(V,S)= \frac{U(V,S)}{\frac{3}{2}Nk_{B}T}$$
ולכן אם נציב את \(T\) נקבל:
$$dU= TdS-pdV=\frac{U}{\frac{3}{2}Nk_{B}}dS - \frac{2}{3}U \frac{dV}{V}$$
כאשר ניתן כעת לפתור את המדר ולקבל:
$$\ln \left( \frac{U}{U_{0}} \right)=-\frac{2}{3}\ln\left( \frac{V}{V_{0}} \right)+\frac{S-S_{0}}{\frac{3}{2}Nk_{B}} \implies U(S,V)=U_{0}\left( \frac{V_{0}}{V} \right)^{2/3} e^{ \frac{S-S_{0}}{3/2 Nk_{B}T} }$$
כאשר נקבל כי \(\left( \frac{\partial U}{\partial S} \right)_{V}\) יהיה שווה למשוואת מצב של האנרגיה, ועבור \(\left( \frac{\partial U}{\partial V} \right)_{S}\) נקבל את משוואת המצב של גז אידיאלי.

\end{example}
\subsection{הפוטנציאלים}

\begin{definition}[אנרגיה פנימית]
פוטנציאל  המקיים:
$$\mathrm{d}U=-p\mathrm{d}V+T\mathrm{d}S$$

\end{definition}
כלומר זהו פוטנציאל על משתנים טבעיים \(V,S\).

\begin{proposition}
עבור תהליך איזוכורי(נפח קבוע) מתקיים:
$$\Delta U = \int_{T_{1}}^{T_{2}} C_{V} \, dT $$

\end{proposition}
\begin{proof}
נזכור כי בלחץ קבוע מתקיים \(dU=TdS\) ועבור תהליך הפיך מתקיים \(dU=\;\bar{}\mkern-7.5mu dQ_{rev}=C_{V}dT\). ומאינטגרציה נקבל את הטענה.

\end{proof}
\begin{proposition}
$$T=\left(\frac{\partial U}{\partial S}\right)_{V}\qquad p=-\left(\frac{\partial U}{\partial V}\right)_{S}$$

\end{proposition}
\begin{definition}[אנטלפיה]
$$H(S,p)=U+PV$$
ולכן:
$$\mathrm{d}H=T\mathrm{d}S-p\mathrm{d}V+p\mathrm{d}V+V\mathrm{d}p=T\mathrm{d}S+V\mathrm{d}p$$

\end{definition}
כלומר זהו פוטנציאל עם משתנים טבעיים \(p,S\).
כאשר עבור תהליך בלחץ קבוע נקבל: 
$$\mathrm{d}H=T\mathrm{d}S\implies \Delta H = \int_{T_{1}}^{T_{2}} C_{p} \, dT$$

\begin{corollary}
בדומה להגדרה בעזרת האנרגיה עם נפח קבוע, מתקיים:
$$T=\left(\frac{\partial H}{\partial S}\right)_{p}\qquad V=\left(\frac{\partial H}{\partial p}\right)_{S}$$

\end{corollary}
\begin{definition}[הפוטנציאל החופשי של הלמהולדס]
$$F(T,V)=U-TS$$
ולכן:
$$\mathrm{d}F=T\mathrm{d}S-p\mathrm{d}V-T\mathrm{d}S-S\mathrm{d}T=-S\mathrm{d}T-p\mathrm{d}V$$

\end{definition}
וקיבלנו פוטנציאל עם המשתנים הטבעיים \(T,V\)

\begin{corollary}
$$S=-\left(\frac{\partial F}{\partial T}\right)_{V}\qquad p=-\left({\frac{\partial F}{\partial V}}\right)_{T}$$

\end{corollary}
\begin{definition}[הפוטנציאל החופשי של גיבס]
$$G(T,p)=H-TS$$
ולכן:
$$\mathrm{d}G=T\mathrm{d}S+V\mathrm{d}p-T\mathrm{d}S-S\mathrm{d}T=-S\mathrm{d}T+V\mathrm{d}p$$

\end{definition}
וקיבלנו פוטנציאל עם המשתנים הטבעיים \(T,p\)

\begin{corollary}
$$S=-\left(\frac{\partial G}{\partial T}\right)_{p}\qquad V=\left(\frac{\partial G}{\partial p}\right)_{T}$$

\end{corollary}
\subsection{זמינות}

\begin{definition}[זמינות]
נסתכל על מערכת שנמצאת בתוך סביבה עם טמפרטורה \(T_{0}\) ולחץ \(p_{0}\). נגדיר את הזמינות בתור סך האנרגיה שניתן להוציא מהמערכת. זה יהיה שווה לביטוי הבא:
$$A=U+p_{0}V-T_{0}S$$
כלומר זה למעשה האנרגיה הפנימית שיש לגוף, ועוד האנרגיה שניתן לקבל מעבודה שהלחץ החיצוני יכול להפעיל על הנפח, פחות החום שניתן לקבל מהשינוי באנטרופיה בגלל התפשטות תרמית בנפח.

\end{definition}
\begin{proposition}
עבור מערכת מבודדת מכנית, מתקיים \(\mathrm{d}A\leq 0\).

\end{proposition}
\begin{proof}
עבור המערכת שלנו, מתקיים מחוק ראשון של תרמודימיקה:
$${\mathrm{d}}Q=\mathrm{d}U-{\mathrm{d}}W-\left( -p_{0}\,\mathrm{d}V \right)\implies \mathrm{d}W\geq\mathrm{d}U+p_{0}\mathrm{d}V-T_{0}\mathrm{d}S=\mathrm{d}A$$
וכן כאשר המערכת מבודדת מכנית, מתקיים \(\mathrm{d}W=0\) ולכן \(\mathrm{d}A\leq 0\).

\end{proof}
\begin{corollary}
הזמינות יכולה רק לרדת, כאשר מקבלת מינימום בשיווי משקל.

\end{corollary}
נסתכל כעת על המקרים המיחודים של מזעור הזמינות.

\begin{proposition}
כשהלחץ והטמפרטורה קבועה, הזמינות שווה לאנרגיה החופשית של גיבס\((G)\). ובפרט המערכת מגיעה לשיווי משקל כש-\(G\) מינימלי

\end{proposition}
\begin{proof}
$$\Delta\!A\!=\!\Delta(U\!+\!P_{0}V\!-\!T_{0}S)\!=\!\Delta(U\!+\!P V\!+\!T S)\!=\!\Delta G\!\leq\!0$$

\end{proof}
\begin{proposition}
כשהנפח והטמפרטורה קבועים, הזמינות שווה לאנרגיה החופשית של הלמהולדס

\end{proposition}
\begin{proof}
$$\Delta A=\Delta(U+P_{0}V-T_{0}S)=\Delta(U-T_{0}S)=\Delta(U-T S)=\Delta F\leq0$$

\end{proof}
\begin{corollary}
כאשר המערכת מבודדת תרמית עם נפח קבוע נקבל כי \(\mathrm{d}U=0\) מחוק ראשון.

\end{corollary}
\subsection{יחסי מקסוואל}

אנו יודעים כי דיפרנציאל מדוייק יהיה מהצורה:
$$\mathrm{d}f=\overbrace{ \left({\frac{\partial f}{\partial x}}\right)_{y} }^{ F_{x} }\mathrm{d}x+\overbrace{ \left({\frac{\partial f}{\partial y}}\right)_{x} }^{ F_{y} }\mathrm{d}y=F_{y}dx+F_{x}dy$$
כאשר עבור דיפרנציאל מדוייק, נדרש כי יתקיים:
$$\left({\frac{\partial F_{y}}{\partial x}}\right)=\left({\frac{\partial F_{x}}{\partial y}}\right)$$
ניתן לעשות זאת על כל אחד מהיחסים שלנו ולקבל קשר שנקרא קשר מקסוואל. נראה לדוגמא עם הפוטנציאל החופשי של גיבס

\begin{example}
אנו יודעים כי עבור הפוטנציאל החופשי של גיבס מתקיים:
$$\mathrm{d}G=-S\mathrm{d}T+V\mathrm{d}p$$
כאשר ניתן גם לכתוב:
$$\mathrm{d}G=\left({\frac{\partial G}{\partial T}}\right)_{p}\,\mathrm{d}T+\left({\frac{\partial G}{\partial p}}\right)_{T}\,\mathrm{d}p$$
ולכן מהשוואות המקדמים נקבל:
$$\left({\frac{\partial G}{\partial T}}\right)_{p} = -S\qquad \left( \frac{\partial G}{\partial p}  \right)_{T}=V$$
ומזה שדיפרציאל מדוייק:
$$-\left({\frac{\partial S}{\partial p}}\right)_{T}=\left({\frac{\partial V}{\partial T}}\right)_{p}$$

\end{example}
ניתן לבצע תהליך דומה ולקבל את כל יחסי מקסוואל:

\begin{proposition}[יחסי מקסוואל]
\begin{gather*}{{\left(\frac{\partial T}{\partial V}\right)_{S}}}&{{=}}&{{-\left(\frac{\partial p}{\partial S}\right)_{V}}}\\ {{\left(\frac{\partial T}{\partial p}\right)_{S}}}&{{=}}&{{\left(\frac{\partial V}{\partial S}\right)_{p}}}\\ {{\left(\frac{\partial S}{\partial V}\right)_{T}}}&{{=}}&{{\left(\frac{\partial p}{\partial T}\right)_{V}}}\\ {{\left(\frac{\partial S}{\partial p}\right)_{T}}}&{{=}}&{{-\left(\frac{\partial V}{\partial T}\right)_{p}}}\end{gather*}

\end{proposition}
\subsection{המלבן התרמודינמי}

טריק מגניב לזכור את יחסי מקסוואל. מציירים מלבן וממספרים את הקודקודים החל הימין באמצע בעזרת הביטוי
$$\text{Great Physicists Have Studied Under Very Fine Teachers.}$$
כאשר כדי לזכור את הסימנים נשים חצים בין \(S\) ל-\(T\) ובין \(P\) ל-\(V\). נקבל:

\includegraphics[width=0.8\textwidth]{diagrams/svg_2.svg}
\begin{proposition}
שתי הגדלים שנמצאים ליד כל פוטנציאל תרמודינמים הם המשתנים הטבעיים שלו.

\end{proposition}
\begin{proposition}
כדי למצוא את הדיפרנציאל של פוטנציאל תרמודינמי נצטרך להשתמש בחצים. מסתכלים על שתי המשתנים הטבעיים, מכפילים כל הדיפרנציאל של המשתנה הטבעי בערך שנמצא בצד השני של החץ, כאשר בחסירים אם החץ מצביע על המשתנה הטבע ומחסרים אחרת. לדוגמא אם נסתכל על \(G\) המשתנים הטבעיים יהיו \(P,T\) כאשר נכפיל בערך שנמצא בצד השני של החץ כאשר עבור \(dT\) נדרש סימן מינוס כי החץ מצביע על \(T\) ונקבל:
$$dG=-SdT+PdV$$

\end{proposition}
\begin{proposition}
כדי למצוא את יחסי מקסוואל נצטרך להסתכל על המסלול שאורך הריבוע של שתי משתנים טבעיים של משתנה. נראה בעזרת דוגמא. נסתכל על \(G\). יש ל-\(G\) שתי משתנים טבעיים - \(P,T\). עבור \(T\) נלך ל-\(V\) ואז ל-\(S\). ונקבל \(\left( \frac{\partial T}{\partial V} \right)_{S}\). עבור המסלול השני נקבל \(\left( \frac{\partial P}{\partial S} \right)_{V}\). כיוון שבמסולולים הסגורים שעוברים דרך האלכסון מסלול אחד הוא בכיוון החצים והמסלול השני נגד כיוון החצים ולכן נדרש להוסיף פה סימן מינוס(אם שניהם היו נגד כיווני החצים או שניהם היו בכיוון החצים אז לא היה נדרש).

\end{proposition}
\section{פיתוח אלטרנטיבי - התמרת לג'נדר}

\begin{remark}
בפועל במעבדה קל למדוד את \(T,P,V\) וקשה למדוד את \(S\).

\end{remark}
אנו יודעים כי \(U=U(S,V,N)\). נרצה למצוא גודל \(F=F(T,V,N)\) מייצג את אותו המידע ש-\(S\) מייצג. לצורך זה נתמיר בעזרת טרנפורם לג'נדר את \(S\to T\).

\begin{proposition}
התמרת לג'נדר זו ההעתקה היחידה שמקבלת \(f(x)\) ומחזירה \(\tilde{f}(p)\) כך שמתקיים:
$$\forall p\quad \left.\frac{\partial f(x)}{\partial x}  \right\rvert _{x=\frac{\partial \tilde{f}}{\partial p} (p)}=p$$
וגם מקיימת:
$$\tilde{f}(p)=\sup _{x}\{ px-f(x) \} $$
או באופן שקול:
$$\tilde{f}(p)=px_{*}-f(x_{*})$$
כאשר \(x_{* }\) מוגדר כך ש:
$$\left.\frac{\partial f}{\partial x} \right\rvert_{x=x_{*}}=p $$

\end{proposition}
\section{החוק השלישי}

\begin{proposition}
מתקיים:
$$S(T)=S(T_{0})+\int_{T_{0}}^{T}{\frac{C_{p}}{T}}\mathrm{d}T,$$

\end{proposition}
\begin{proof}
נזכור כי מתקיים \(C_{p}=T\left({\frac{\partial S}{\partial T}}\right)_{p}\). לכן:
$$S=\int{\frac{C_{p}}{T}}\mathrm{d}T\implies S(T)=S(T_{0})+\int_{T_{0}}^{T}{\frac{C_{p}}{T}}\mathrm{d}T,$$

\end{proof}
נראה שלמעשה אין לנו דרך למצוא את האנטרופיה על ידי שינוי הטמפרטורה, וניתן למצוא רק את השינוי באנטרופיה. החוק השלישי יאפשר לנו למצוא את האנטרופיה של מערכת ע"י הגדרת הערך של האנטרופיה בטמפרטורה ספציפית - 0.

\begin{theorem}[חוק שלישי של תרמודינמיקה]
האנטרופיה 0 מוגדרת בתור הגבול של מערכת כאשר הטמפרטורה שואפת ל-0.

\end{theorem}
\begin{remark}
יש הרבה הגדרות שונות של החוק השלישי, אך כולם שקולים.

\end{remark}
נסתכל כעת על מספר מסקנות מהחוק השלישי.
\textbf{מסקנה}
כל הקיבולי חום שואפים ל-0. זאת כיוון שמתקיים:
$$C=T\left({\frac{\partial S}{\partial T}}\right)=\left({\frac{\partial S}{\partial\ln T}}\right)\rightarrow 0$$

\begin{corollary}
התפשטות תרמית מפסיקה, זאת כי מתקיים \(S,T\to 0\) ולכן \(\left(\frac{\partial S}{\partial p}\right)_{T}\rightarrow0\) אבל מיחס מקסוואל נקבל כי:
$$\frac{1}{V}\left(\frac{\partial V}{\partial T}\right)_{p}\rightarrow0$$

\end{corollary}
\begin{corollary}
לא ניתן לקרר ל-\(T=0\) במספר סופי של צעדים.

\end{corollary}
\section{משטחים תרמודימים}

\begin{proposition}
ניתן לכתוב את האנטרופיה בעזרת גדלים אקסטנסיבים:
$$S=S(E,x_{i})$$
כאשר \(x_{i}\) גדלים אקסטנסיבים. ניתן כעת להחליף משתנים ולכתוב:
$$E=E\left( S,x_{1},x_{2},\dots \right)$$

\end{proposition}
\begin{definition}[כוח מוכלל]
$$J_{i}\equiv \frac{\partial E(s,x_{i})}{\partial x_{i}} $$

\end{definition}
\begin{theorem}[החוק הראשון המוכלל]
$$\mathrm{d}E=T\mathrm{d}S+\sum_{i}J_{i}\mathrm{d}x_{i}$$

\end{theorem}
\begin{corollary}
$$E(S,x_{i})=T(s,x_{i})S+\sum J_{i}(s,x_{i})x_{i}$$
כאשר זהו הגודל האקסטנסיבי היחיד \(E(\lambda S,\lambda x_{i})=\lambda E(S,x_{i})\) אשר מקיימת:
$$\frac{\partial E}{\partial S} =T\qquad \frac{\partial E}{\partial x_{i}} =J_{i}$$

\end{corollary}
\begin{proposition}[יחס גיבס - דום]
$$S\mathrm{d}T+\sum x_{i}\mathrm{d}J_{i}=0$$

\end{proposition}
\begin{proof}
באופן כללי:
$$E(S,x_{i})=TS+\sum_{i}J_{i}x_{i}$$
ונקבל:
$$\mathrm{d}E=\left[ T\mathrm{d}S+\sum_{i}J_{i}\mathrm{d}x_{i} \right]+\left[ S\mathrm{d}T+\sum_{i}x_{i} \mathrm{d}J_{i} \right]$$
כאשר הסוגריים הראשונים זה \(\mathrm{d}E\) מהחוק הראשון, ולכן הסוגריים השניים מתאפסים ונקבל:
$$S\mathrm{d}T+\sum x_{i}\mathrm{d}J_{i}=0$$

\end{proof}
\chapter{תהליכים ומנועים}

\section{סיווג תהליכים}

\begin{definition}[גרף PV]
גרף של לחץ כתלות בנפח.

\end{definition}
\begin{proposition}
השטח מתחת לגרף PV הוא העבודה:
$$W=\int_{V_{i}}^{V_{f}}P\,d V$$

\end{proposition}
\begin{definition}[גרף ST]
גרף של טמפרטורה כתלות באנטרופיה.

\end{definition}
\begin{proposition}
השטח מתחת לגרף ST הוא המעבר חום:
$$Q=\int_{S_{i}}^{S_{f}}T\,d S$$

\end{proposition}
\section{תהליך איזותרמי}

\begin{definition}[תהליך איזותרמי]
תהליך שהטמפרטורה בו לא משתנה. כלומר מתקיים \(\Delta T=0\).

\end{definition}
אפשר לחשוב על תהליך זה בתור תהליך שנמצא ליד מקור חום אשר תמיד מכניס אנרגיה למערכת כך שהטמפרטורה תהיה תמיד שווה, לא משנה איך משתנה הנפח.

\begin{proposition}[אין שינוי באנרגיה]
עבור גז אידיאלי בתהליך איזותרמי, מתקיים \(\Delta U=0\).

\end{proposition}
זאת כיוון שאנו יודעים כי עבור גז אידיאלי,  \(dU=C_{V}dT\).
\textbf{מסקנה}
עבור גז אידיאלי, מהחוק הראשון של תרמודינמיקה מתקיים \(\;\bar{}\mkern-7.5mu dW=\;\bar{}\mkern-7.5mu dQ\).

\includegraphics[width=0.8\textwidth]{diagrams/svg_3.svg}
\begin{proposition}
העבודה של תהליך איזותרמי עבור גז אידאילי יהיה:
$$W=\int_{V_{1}}^{V_{2}} \frac{Nk_{B}T}{V} \mathrm{d}V=Nk_{B}T\ln\left( \frac{V_{2}}{V_{1}} \right)$$

\end{proposition}
\begin{remark}
קיבול חום לא מוגדר בתהליך איזותרמי. זאת כוון שנדרש לחלק ב-\(\Delta T\) אשר אפס.

\end{remark}
\begin{proposition}[שינוי באנטרופיה של תהליך איזותרמי]
$$\Delta S=n R\ln{\frac{V_{f}}{V_{i}}}=n R\ln{\frac{P_{i}}{P_{f}}}$$

\end{proposition}
\section{תהליך אדיאבטי}

\begin{definition}[תהליך אדיאבטי]
תהליך אשר אין בו זרימה של חום, כלומר מבודד תרמית. זה אומר שמתקיים:
$$\;\bar{}\mkern-7.5mu dQ=0$$

\end{definition}
\begin{corollary}
מהחוק הראשון של תרמודינמיקה נקבל:
$$dU=\;\bar{}\mkern-7.5mu dW$$

\end{corollary}
כאשר עבור גז אידיאלי אנו יודעים כי \(dU=C_{V}dT\) וכן עבור תהליך הפיך נקבל \(\;\bar{}\mkern-7.5mu dW=-pdV\) ולכן נקבל כי מתקיים:
$$C_{V}\,\mathrm{d}T=-p\,\mathrm{d}V=-{\frac{R T}{V}}\,\mathrm{d}V,$$
ולכן:
$$\ln\frac{T_{2}}{T_{1}}=-\frac{R}{C_{V}}\ln\frac{V_{2}}{V_{1}}.$$
כעת \(C_{p}=C_{V}+R\) כאשר אם נחלק גודל זה ב-\(C_{V}\) נקבל:
$$\gamma=\frac{C_{p}}{C_{V}}=1+\frac{R}{C_{V}}\implies-(R/C_{V})=1-\gamma,$$
ונקבל:
$$T V^{\gamma-1}={\mathrm{constant}} \implies p^{1-\gamma}T^{\gamma}=\mathrm{constant}\implies p V^{\gamma}={\mathrm{constant}},$$

\begin{proposition}[דחיסה אדיבאטית]
$$p_{1}V_{1}^\gamma=p_{2}V_{2}^\gamma \qquad \gamma = \frac{C_{P}}{C_{V}}$$

\end{proposition}
\includegraphics[width=0.8\textwidth]{diagrams/svg_4.svg}
\begin{proposition}
$$TV^{\gamma-1}=\mathrm{const}\qquad T^\gamma P^{1-\gamma}=\mathrm{const}$$

\end{proposition}
\begin{proposition}[שינוי באנטרופיה בתהליך אדיאבטי]
עבור תהליך הפיך נקבל \(\Delta S = 0\). כאשר עבור תהליך שאינו הפיך נקבל \(\Delta S > 0\).

\end{proposition}
\section{תהליך איזוכורי}

\begin{definition}[תהליך איזוכורי]
תהליך שבו הנפח נשאר קבוע

\end{definition}
\begin{proposition}[קיבול חום תחת נפח קבוע]
$$C_{V}=\!\frac{\;\bar{}\mkern-7.5mu dQ_{V}}{d T}\!=\!\left(\frac{\partial U}{\partial T}\right)_{V}$$

\end{proposition}
\begin{proposition}[אין עבודה]
$$W=\int_{V_{0}}^{V_{1}} P \;\mathrm{d}V=0$$

\end{proposition}
\begin{proposition}[חום בתהליך איזוכורי]
$$Q=C_{V}\Delta T$$

\end{proposition}
\begin{proposition}[אנרגיה פנימית שווה לחום]
$$U=W+Q = Q=C_{V}\Delta T$$

\end{proposition}
\begin{proposition}[אנטרופיה בתהליך איזוכורי]
$$\Delta S = \int \mathrm{\frac{\;\bar{}\mkern-7.5mu dQ}{T}}=C_{V}\ln \frac{T_{2}}{T_{1}}$$

\end{proposition}
\begin{proposition}[יחסים של גז אידיאלי]
$$V_{1} = V_{2}\qquad  \frac{P_{1}}{T_{1}}=\frac{P_{2}}{T_{2}}$$

\end{proposition}
\includegraphics[width=0.8\textwidth]{diagrams/svg_5.svg}
\begin{proposition}[אנטרופיה בתהליך איזוכורי]
$$\Delta S=n C_{v}\ln{\frac{T_{f}}{T_{i}}}$$

\end{proposition}
\section{תהליך איזובארי}

\begin{definition}[תהליך איזובארי]
תהליך שבו הלחץ נשאר קבוע

\end{definition}
\begin{proposition}[עבודה של תהליך איזובארי]
כיוון שהלחץ קבוע נקבל מיידית:
$$W=P\Delta V$$

\end{proposition}
\begin{proposition}[אנטרופיה של תהליך איזובארי]
$$\Delta S =C_{P}\ln \left( \frac{T_{2}}{T_{1}} \right)$$

\end{proposition}
\begin{proposition}[יחסים של גז אידיאלי]
$$\frac{V_{1}}{T_{1}}=\frac{V_{2}}{T_{2}}$$

\end{proposition}
\includegraphics[width=0.8\textwidth]{diagrams/svg_6.svg}
\begin{proposition}[שינוי באנטרופיה בתהליך איזובארי]
$$\Delta S=n C_{p}\ln{\frac{T_{f}}{T_{i}}}$$

\end{proposition}
\section{מנועים}

ראינו כי לפי החוק השני של תרמודינמיקה, לא ייתכן תהליך שהתוצאה היחידה שלו היא המרה של חום לעבודה. אם לא ניתן להמיר את כל החום לעבודה, זה מעלה את השאלה מה הכמות המקסימלית שניתן להמיר. לצורך זה נגדיר את המושג של מנוע.
\textbf{הגדרה} מנועה
מערכת הפועלת בצורה מחזורית הממירה אנרגיה לעבודה.

\includegraphics[width=0.8\textwidth]{diagrams/svg_7.svg}
המערכת מורכבת משלושה רכיבים.

\begin{enumerate}
  \item מקור חום בלתי מוגבל(אדום) - כלומר מעביר אנרגיית חום ונשאר תמיד בטמפרטורה \(T_{h}\). 


  \item אמבט קור בלתי מוגבל(כחול) - כלומר מקבל אנרגיית חום ונשאר תמיד בטמפרטורה \(T_{c}\). 


  \item רכיב ביניהם אשר נקרא לו המנוע. המנוע מלא בגז אידיאלי ויכול לשנות את הנפח שלו בהתאם לקשר \(PV=k_{B}NT\) - כאשר השינוי בנפח יכול לגרום לעבודה. 


\end{enumerate}
\textbf{שלב ראשון - התפשטות איזותרמית:} 
המנוע מקבל חום מהאמבט החם בתהליך איזותרמי. כלומר הטמפרטורה נשארת קבועה והנפח גדל בהתאם כדי שהטמפרטורה לא תגדל.

\textbf{השלב השני - התפשטות אדיאבטית:}
נבודד את המנוע מהסביבה בתהליך אדיאבטי. המערכת ממשיכה להתפשט אבל הלחץ יורד.

\textbf{השלב השלישי - דחיסה איזותרמית:}
מחברים את המנוע לאמבט הקר. נקבל כי הנפח קטן, אבל הלחץ גדל.

\textbf{השלב הרביעי - דחיסה אדיובטית:}
מנתקים את המנוע מהאמפט הקר. הלחץ גדל אך הנפח קטן.

\includegraphics[width=0.8\textwidth]{diagrams/svg_8.svg}
\begin{definition}[מנוע קרנו]
רכיב אשר מקיים את מחזור קרנו נקרא מנוע קרנו.

\end{definition}
\includegraphics[width=0.8\textwidth]{diagrams/svg_9.svg}
\begin{definition}[יעילות]
הייעוד של מנוע זה להפוך חום לעבודה. לכן היעילות שלו תהיה היחס בין כמות החום שהתקבל לעבודה שהוציא. כלומר:
$$\eta = \left\lvert  \frac{W_{out}}{Q_{in}}  \right\rvert = \frac{\lvert Q_{in} \rvert -\lvert Q_{out} \rvert}{\lvert Q_{in} \rvert }=1-\left\lvert  \frac{Q_{out}}{Q_{in}}  \right\rvert  $$

\end{definition}
\begin{proposition}[יעילות של מנוע קרנו]
$$\eta_{carnot}=1-\frac{T_{C}}{T_{H}}=\frac{T_{H}-T_{C}}{T_{H}}$$

\end{proposition}
\begin{proof}
בחלק הראשון נקבל כי בין \(A\) ל-\(B\) מתקיים:
$$Q_{H}=T_{H}(S_{B}-S_{A}) \qquad Q_{C}=T_{C}(S_{B}-S_{A})$$
וניתן להציב בביטוי של היעילות ולקבל:
$$\eta = \frac{Q_{H} - Q_{C}}{Q_{H}}= \frac{(T_{H}-T_{C})(S_{B}-S_{A})}{T_{H}(S_{B}-S_{A})}=\frac{T_{H}-T_{C}}{T_{H}}$$

\end{proof}
\begin{remark}
ניתן גם לפתח בדרך יותר מכוערת ללא אנטרופיה. הרעיון הוא להראות שבתליך האיזותרמי מתקיים:
$$\frac{V_{B}}{V_{A}}=\frac{V_{C}}{V_{D}}$$
ואז למצוא את \(Q_{H},Q_{C}\) אשר יכילו את הגורמים \(\frac{V_{B}}{V_{A}}\) ו-\(\frac{V_{C}}{V_{D}}\). אם נציב בביטוי של יעילות נקבל כי הגורמים השווים יצטמצמו וניתן יהיה למצוא את היעילות.

\end{remark}
\begin{theorem}
היעילות של מנוע קרנו היא היעילות המקסימלית האפשרית בשביל מנוע.

\end{theorem}
\chapter{פוטנציאל כימי ומעברי פאזה}

\section{פוטנציאל כימי}

נרצה להסתכל כעת על מערכות פתוחות, כלומר מערכות שיש מעבר של חלקיקים עם הסביבה. דוגמא למשל היא גוש קרח הצף במים ומאבד חלקיקים.

\begin{definition}[פוטנציאל כימי]
זה יהיה כמה שמשתנה האנרגיה הפנימית שמספר החלקיקים משתנה:
$$\mu\!=\!\left({\frac{\partial U}{\partial N}}\right)_{S,V}$$

\end{definition}
ננסה להסביר:
נסתכל על המשוואה היסודית:
$$d U=T d S-P d V$$
כאשר ניתן להוסיף את האפקט של איבוד החלקיקים:
$$d U=T\,d S-P\,d V+\mu\,d N$$
כאשר ניתן גם לנוסיף מספר פוטנציאלים כימים כאשר יש סוגים שונים של חלקיקים:
$$d U=T d S-P\,d V+\sum_{i}\mu_{i}\,d N_{i}$$

\begin{proposition}
ניתן לכתוב בעזרת האנרגיה החופשית של הלמהולדס
$$\mu=\left({\frac{\partial F}{\partial N}}\right)_{V,T}$$

\end{proposition}
\begin{proof}
$${F=U-T S}\implies d F=d U-T\,d S-S\,d T$$
ולכן מהמשוואה של הדיפרנציאל של האנרגיה הפנימית נקבל:
$$d F=-P\,d V-S\,d T+\mu\,d N\implies \mu=\left({\frac{\partial F}{\partial N}}\right)_{V,T}$$

\end{proof}
בנוסף אנו יודעים את הקשר בין פוטנציאל הלמהולדס לפונקציית החלוקה \(Z\):
$$F=-N k_{B}T\ln Z$$
ומזה ניתן לקבל את הקשר בין \(\mu\) ל-\(Z\):
$$\mu=-k_{\mathrm{{B}}}T\left({\frac{\partial}{\partial N}}(N\ln Z)\right)_{V,T}$$

\begin{proposition}
כאשר יש סוג אחד של חלקיק, מתקיים:
$$\mu=\frac{G}{N}$$
כלומר נקבל כי הפוטנציאל הכימי זה למעשה אנרגיית גיבס לחלקיק.

\end{proposition}
\begin{proof}
באופן דומה למה שעשינו עבור פוטנציאל הלמהולדס. נכתוב:
$$G=H-T S=U+P V-T S\implies  d G\,=\,d U\,+\,P\,d V\,+\,V\,d P\,-\,T\,d S\,-\,S\,d T$$
כאשר אם נשתמש בדיפרנציאל של האנרגיה הפנימית נקבל:
$$d G\,=\,V\,d P\,-\,S\,d T\,+\,\mu\,d N\implies \mu=\left({\frac{\partial G}{\partial N}}\right)_{T,P}$$
כיוון ש-\(G\) הוא גודל אקסטנסיבי, הוא פרופרצינאלי למספר חלקיקים, כלומר קיים \(\phi(T,P)\) כך ש:
$$G(T,P,N)\!=\!N\phi(T,P)$$
ולכן אם נגזור לפי \(N\) כאשר נשאיר את \(T,P\) קבועים נקבל:
$$\mu=\phi(T,P)\implies \mu=\frac{G}{N}$$

\end{proof}
נעיר כי לא ניתן לבצע תהליך דומה עבור פוטנציאלים כמו \(U,F\) כיוון שהמשתנים הטבעיים שלהם הם לא שניהם אינטנסיבים ולכן אם נכתוב:
$$U\!=\!U\!(S,V,N)\!=\!N\phi^{\prime}\!\left({\frac{S}{N}},{\frac{V}{N}}\right)$$
נקבל כי גזירה לפי \(N\) תתן:
$$\mu=\phi^{\prime}+N\Biggl(\frac{\partial\phi^{\prime}}{\partial N}\Biggr)_{S,V}=\frac{U}{N}+N\Biggl(\frac{\partial\phi^{\prime}}{\partial N}\Biggr)_{S,V}$$
כאשר מתקבל גורם נוסף.

\begin{corollary}
באופן כללי במערכת אם סוגים שונים של חלקיקים נקבל:
$$G=\sum_{i}\mu_{i}N_{i}$$

\end{corollary}
\begin{example}[שיווי משקל בין שתי מערכות המחליפות חלקיקים]
נסתכל על מערכת בו יש שתי חלקיקים(\(A,B\)) שיכולים להחליף חלקיקים חום ואנרגיה ביניהם אבל המערכת הכוללת סגורה, כלומר מתקיים:
\begin{gather*}N_{\mathrm{A}}+N_{\mathrm{B}}=N \\V_{\mathrm{{A}}}+V_{\mathrm{{B}}}=V \\U_{\mathrm{{A}}}+U_{\mathrm{{B}}}=U
\end{gather*}
נשים לב כי האנטרופיה תהיה שווה:
\begin{gather*}{S\,=S\bigl(U_{\mathrm{A}},V_{\mathrm{A}},N_{\mathrm{A}},U_{\mathrm{B}},V_{\mathrm{B}},N_{\mathrm{B}}\bigr)}{=S_{\mathrm{A}}\bigl(U_{\mathrm{A}},V_{\mathrm{A}},N_{\mathrm{A}}\bigr)\!+\!S_{\mathrm{B}}\bigl(U_{\mathrm{B}},V_{\mathrm{B}},N_{\mathrm{B}}\bigr)}\end{gather*}
כאשר בשיווי משקל נדרוש כי האנטרופיה תהיה מקסימלית. כלומר מתקיים:
$$dS=dS_{A}+dS_{B}=0$$
נזכור כי מתקיים:
$$d U=T\,d S-P\,d V+\sum_{i}\mu_{i}\,d N_{i}$$
ולכן נבודד את האנטרופיה ונקבל:
\begin{gather*}d S_{\mathrm{A}}={\frac{1}{T_{\mathrm{A}}}}(d U_{\mathrm{A}}+P_{\mathrm{A}}d V_{\mathrm{A}}-\mu_{\mathrm{A}}\,d N_{\mathrm{A}})  \\  d S_{\mathrm{B}}={\frac{1}{T_{\mathrm{B}}}}\big(d U_{\mathrm{B}}+P_{\mathrm{B}}\,d V_{\mathrm{B}}-\mu_{\mathrm{B}}\,d N_{\mathrm{B}}\big) 
\end{gather*}
כלומר אחרי שנציב ב-\(dS\) נקבל:
$${\frac{1}{T_{\mathrm{A}}}}(d U_{\mathrm{A}}+P_{\mathrm{A}}\,d V_{\mathrm{A}}-\mu_{\mathrm{A}}\,d N_{\mathrm{A}})+{\frac{1}{T_{\mathrm{B}}}}(d U_{\mathrm{B}}+P_{\mathrm{B}}\,d V_{\mathrm{B}}-\mu_{\mathrm{B}}\,d N_{\mathrm{B}})=0$$
ולכן:
$$\Biggl({\frac{1}{T_{\mathrm{A}}}}-{\frac{1}{T_{\mathrm{B}}}}\Biggr)d U_{\mathrm{A}}+\Biggl({\frac{P_{\mathrm{A}}}{T_{\mathrm{A}}}}-{\frac{P_{\mathrm{B}}}{T_{\mathrm{B}}}}\Biggr)d V_{\mathrm{A}}-\Biggl({\frac{\mu_{\mathrm{A}}}{T_{\mathrm{A}}}}-{\frac{\mu_{\mathrm{B}}}{T_{\mathrm{B}}}}\Biggr)d N_{\mathrm{A}}=0$$
כמו באלגברה לינארית, נדרש לכל אחד מהדיפרנציאלים מתאפס, כלומר נקבל כי:
$$T_{A}=T_{B}\qquad P_{A}=P_{B}\qquad \mu_{A}=\mu_{B}$$

\end{example}
\section{מעברי פאזה}

\begin{definition}[פאזה תרמודינמית]
דרך לייחד חומר המתחום במרחב(כלומר נמצא ביחד כגוש) ע"י התכונות הכימיות והפיזיות שלו. לדוגמא, צפיפות, מוליכות חשמלית, צורה, סידור פנימי של אטומים ועוד. מצבי הצבירה מוצק נוזל וגז הם דוגמאות לפאזות.

\end{definition}
כאשר מערכת משנה פאזה תרמודינמית, יש שלב שבה הטמפרטורה נשארת קבוע למרות שמכניסים חום למערכת. זה אומר שהתיאור של קיבול חום כבר לא יהיה נכון. לצורך זה נגדיר מושג חדש.

\begin{proposition}
המעבר פאזה תלוי בלחץ ובטמפרטורה, כאשר מעבר פאזה מתרחש בלחץ וטמפרטורה קבועה.

\end{proposition}
\begin{definition}[חום כמוס]
כמות האנרגיה התרמית הנדרשת כדי להעביר חומר מצב צבירה כאשר הלחץ קבוע.
$${L}=\Delta Q_{\mathrm{rev}}=T_{\mathrm{c}}(S_{2}-S_{1})$$

\end{definition}
\includegraphics[width=0.8\textwidth]{diagrams/svg_10.svg}
\begin{definition}[דיאגרמת פאזה]
זה גרף של לחץ כפונקציה של טמפרטורה. כל תחום קשיר מייצג פאזה.

\end{definition}
נרצה כעת להסתכל על מערכת שמורכבת משתי פאזות ושל אותו החומר ונרצה לראות מה התנאי לשיווי משקל.
\textbf{טענה}
בשיווי משקל נדרש כי לשתי הפאזות יש פוטנציאל כימי שווה - \(\mu_{1}=\mu_{2}\)

\begin{proof}
נכתוב:
$$G=M_{1}\mu_{1}+M_{2}\mu_{2}$$
כאשר אנו יודעים כי בלחץ וטמפרטורה קבועה השיווי משקל הדיפרנציאל יהיה 0. כלומר:
$$dG=0=\mu_{1}dM_{1}+\mu_{2}dM_{2}$$
כאשר אם המערכת סגורה כך ש-\(M\) קבוע נקבל:
$$dM=dM_{1}+dM_{2}=0$$
ומזה ניתן להסיק כי \(\mu_{1}=\mu_{2}\).

\end{proof}
\begin{proposition}[משוואת קלאסיוס קלפירון]
$${\frac{\mathrm{d}p}{\mathrm{d}T}}={\frac{L}{T(V_{2}-V_{1})}}$$
כאשר זה נותן את המביטוי של המעבר פאזה כתלות בחום בכמוס, הטמפרטורה הקריטית וההבדלי נפח.

\end{proposition}
\begin{proof}
נסתכל על מערכת שמורכבת משתי פאזות אשר נמצא על הגבול בין שתי הפאזות. מהטענה הקודמת נקבל:
$$\mu_{1}(p,T)=\mu_{2}(p,T)$$
כאשר אם נבצע הזזה קטנה נקבל:
$$\mu_{1}\left( p+\mathrm{d}p,T+\mathrm{d}T \right)=\mu_{2}\left( p+\mathrm{d}p,T+\mathrm{d}T \right)\implies d\mu_{1}=d\mu_{2}$$
כאשר אם אין מעבר חלקיקים נקבל כי \(dG_{1}=dG_{2}\) ולכן:
$$-S_{1}\mathrm{d}T+v_{1}\mathrm{d}p=-S_{2}\mathrm{d}T+v_{2}\mathrm{d}p \implies \frac{\mathrm{d}p}{\mathrm{d}T}=\frac{S_{2}\,-\,S_{1}}{v_{2}\,-\,v_{1}}$$
כאשר ניתן לכתוב \(L=T\Delta S\) ולקבל את המשוואה.

\end{proof}
\begin{definition}[מעבר פאזה מסדר ראשון]
כאשר הנפח משתנה, יש קפיצה באנטרופיה וקיים חום כמוס, אבל הפוטנציאל החופשי של גיבס נשאר רציף וזהה. כלומר:
$$G_{1}=G_{2}\qquad V_{1}\neq V_{2}\qquad L\neq 0\qquad S_{1}\neq S_{2}$$

\end{definition}
דוגמאות כוללות שינוי מצב צבירה.

\begin{definition}[מעבר פאזה מסדר שני]
הפוטנציאל החופשי של גיבס נשאר רציף, אך אין חום כמוס, אין קפיצה באנטרופיה ואין שינוי בנפח, אבל הנגזרות הם לא רציפות. כלומר:
$$\left( \frac{\partial S}{\partial T}  \right)_{p},\left( \frac{\partial S}{\partial P}  \right)_{T},\left( \frac{\partial V}{\partial T}  \right)_{P},\left( \frac{\partial V}{\partial P}  \right)_{T}$$
הם לא רציפים. זה אומר כי:
$$C_{p}=T\left( \frac{\partial S}{\partial T}  \right)_{p}\qquad \beta=\frac{1}{V}\left( \frac{\partial V}{\partial T}  \right)_{p}\qquad \kappa=\frac{1}{V}\left( \frac{\partial V}{\partial P}  \right)_{T}$$
הם גדלים לא רציפים. 

\end{definition}
מעבר פאזה לדוגמא מסדר שני זה טמפרטורת קירי, חומרים פאראמגנטיים שמגיעים לטמפרטורה מסויימת ומאבדים את הזכרון המגנטי שלהם.

\chapter{הגישה הסטטיסטית - צברים}

\section{אנטרופיה סטטיסטית}

\begin{definition}[מיקרו מצב]
ניתן לחלק את המצבים האפשריים של מערכת לאוסף מצבים שווי הסתברות.

\end{definition}
\begin{definition}[מאקרו מצב]
קונפיגורציה כלשהי של המערכת. לא בהכרח שוות הסתברות.

\end{definition}
\begin{theorem}[העקרון היסודי של פיזיקה סטטיסטית]
כל אחד מהמיקרו מצבים הם שווי הסתברות.

\end{theorem}
כאשר העקרון הזה זה בעצם ההגדרה של מיקרו מצב. ולמעשה נראה כי עקרון זה ניתן לקבל את כל חוקי התרמודינמיקה.

\begin{example}
נסתכל על מערכת עם 100 מטבעות הוגנים. יש \(2^{100}\) מצבים אפשרים עבור כל מטבע. זה יהיה המיקרו מצבים.
כאשר יש 100 קונפיגורציות של המערכת סה"כ - כמה מטבעות הם עץ וכמה הם פאלי. זה יהיה המאקרו מצבים.

\end{example}
\begin{corollary}
  \begin{itemize}
    \item ניתן לתאר מערכת ע"י כמות גדולה של מיקרו מצבים שווי הסתברות.
    \item מה שאנחנו מודדים בפועל זה המאקרו מצבים.
  \end{itemize}
\end{corollary}
\begin{proposition}
התוצאה שתתקבל בפועל תהיה התוצאה המאקרוסקופית עם הכי הרבה מצבים מיקרוסקופים.

\end{proposition}
הרעיון מבוסס על ההנחות הבאות:

\begin{enumerate}
  \item כל אחד מהמיקרו מצבים הם שווי הסתברות. 


  \item הדינמיקה של המערכת פועלת כך שהמערכת משתנה ונמצאת בכל מיקרו מצב כמות שווה של זמן. 


  \item עבור מערכות גדולות, המצב עם הכי הרבה מיקרו מצבים יהיה עם משמעותית הרבה יותר מיקרו מצבים מאשר המערכות האחרות. 


\end{enumerate}
\begin{corollary}
עבור שתי מערכות עם אנרגיה \(E = E_{1}+E_{2}\) החלוקה של אנרגיה שתתקבל תהיה החלוקה שתמקסם את \(\Omega_{1}(E_{1})\Omega_{2}(E_{2})\) כאשר \(\Omega_{1}(E_{1}),\Omega_{2}(E_{2})\) הם כמות המיקרו מצבים המתאימים עבור אנרגיה נתונה.

\end{corollary}
\begin{proposition}
$$\frac{\mathrm{d}\ln\Omega_{1}}{\mathrm{d}E_{1}}=\frac{\mathrm{d}\ln\Omega_{2}}{\mathrm{d}E_{2}}$$

\end{proposition}
\begin{proof}
כיוון שאנו מניחים שאנו נמצאים בגבול התרמודינמי, נמצא מקסימום ע"י גזירה והשוואה ל-0:
$$\frac{\mathrm{d}}{\mathrm{d}E_{1}}\left(\Omega_{1}(E_{1})\Omega_{2}(E_{2})\right)=0 \implies \Omega_{2}(E_{2})\frac{\mathrm{d}\Omega_{1}(E_{1})}{\mathrm{d}E_{1}}+\Omega_{1}(E_{1})\frac{\mathrm{d}\Omega_{2}(E_{2})}{\mathrm{d}E_{2}}\frac{\mathrm{d}E_{2}}{\mathrm{d}E_{1}}=0$$
כאשר נשים לב כי \(dE_{1}=-dE_{2}\) מגזירה של האנרגיה הכוללת. ולכן \(\frac{dE_{2}}{dE_{1}}=-1\) ונקבל:
$$\frac{1}{\Omega_{1}}\frac{\mathrm{d}\Omega_{1}}{\mathrm{d}E_{1}}-\frac{1}{\Omega_{2}}\frac{\mathrm{d}\Omega_{2}}{\mathrm{d}E_{2}}=0 \implies {\frac{\mathrm{d}\ln\Omega_{1}}{\mathrm{d}E_{1}}}={\frac{\mathrm{d}\ln\Omega_{2}}{\mathrm{d}E_{2}}}$$

\end{proof}
כעת נזכור כי הגדרנו את הטמפרטורה של המערכת בתור גודל המאפיין את המערכת ומשתווה בין שתי מערכות המגיעות לשיווי משקל תרמי. כעת ראינו שהגודל שהתקבל הוא גודל כזה, וניתן להגדיר את הטמפרטורה בצורה יותר ריגורוזית:

\begin{definition}[טמפרטורה]
גודל \(T\) המוגדר ע"י המשוואה:
$${\frac{1}{k_{\mathrm{B}}T}}={\frac{\mathrm{d}\ln\Omega}{\mathrm{d}E}}$$
כאשר \(k_{B}\) זה קבוע בולצמן, \(\Omega\) זה מספר המיקרומצבים כתלות באנרגיה, ו-\(E\) זה האנרגיה.

\end{definition}
\begin{proposition}[הגדרה סטטיסטית לאנטרופיה]
$$S=k_{B}\ln \Omega$$
כאשר \(k_{B}\) זה קבוע בולצמן ו-\(\Omega\) זה כמות המיקרו מצבים המתאימים למאקרו מצב הנמדד. 

\end{proposition}
\begin{proof}
נזכור כי:
$$T=\left({\frac{\partial U}{\partial S}}\right)_{V}\implies {\frac{1}{T}}=\left({\frac{\partial S}{\partial U}}\right)_{V}$$
וכן מהגדרת הטמפרטורה נקבל:
$${\frac{1}{k_{\mathrm{B}}T}}={\frac{\mathrm{d}\ln\Omega}{\mathrm{d}E}}\implies \frac{1}{k_{B}T}=\frac{1}{k_{B}}\left( \frac{\partial S}{\partial U} \right)_{V}=\frac{d\ln \Omega}{dE} $$
כאשר עבור \(E=U\) המשוואה מתקיימת עבור \(S=k_{B}\ln \Omega\).

\end{proof}
\begin{example}[מערכת 2 רמות]
נניח מערכת חליקים שהם יכולים להיות או \(\uparrow\) או \(\downarrow\). כאשר אם הם מצביעים למעלה האנרגיה שלהם היא \(u_{\uparrow}=\mathcal{E}\) ואם הם מצביעים למטה הם מקיימים \(u_{\downarrow}=0\) כאשר:
$$U=U(T)=N_{\uparrow}\cdot\mathcal{E}$$
כאשר סך החלקיקים יהיה \(N=N_{\uparrow}+N_{\downarrow}\). כאשר סך המצבים יהיה:
$$\Omega\left(N_{\uparrow};N\right)={\binom{N}{N_{\uparrow}}}=\frac{N!}{N_{\uparrow}!\left(N-N_{\uparrow}\right)!}\;.$$
כאשר אם נשתמש בקירוב סטרילינג \(\ln(n!)\approx n\ln(n)-n\). נקבל:
$$\frac{S}{k_{B}}=\ln \Omega \approx N\ln\left(N\right)-N_{\uparrow}\ln\left(N_{\uparrow}\right)-\left(N-N_{\uparrow}\right)\ln\left(N-N_{\uparrow}\right)$$
נשתמש במשוואה היסודית
$$dS = \frac{1}{T}dU+ \frac{P}{T}dV=\left( \frac{\partial S}{\partial U}  \right)_{V,N}du+\left( \frac{\partial S}{\partial V}  \right)_{U,N}\implies \frac{1}{T}=\left( \frac{\partial S}{\partial U}  \right)_{V,N}\quad \frac{P}{T}=\left( \frac{\partial S}{\partial V}  \right)_{U,N}$$
ונקבל:
$$S\left(U,N\right)=k_{\mathrm{B}}\left[N\ln\left(N\right)-\left(N-\frac{U}{\mathcal{E}}\right)\ln\left(N-\frac{U}{\mathcal{E}}\right)-\frac{U}{\mathcal{E}}\ln\left(\frac{U}{\mathcal{E}}\right)\right]$$
ונשתמש בקשר:
$$
{\frac{1}{T}}=\left({\frac{\partial S}{\partial U}}\right)_{N}={\frac{1}{\mathcal{E}}}\left({\frac{\partial S}{\partial\left(U/\mathcal{E}\right)}}\right)_{N}={\frac{k_{\mathrm{B}}}{\mathcal{E}}}\left[\ln\left(N-{\frac{U}{\mathcal{E}}}\right)-\ln\left({\frac{U}{\mathcal{E}}}\right)\right]={\frac{k_{\mathrm{B}}}{\mathcal{E}}}\ln\left({\frac{N-{\frac{U}{\mathcal{E}}}}{{\frac{U}{\mathcal{E}}}}}\right)_{N}={\frac{1}{\mathcal{E}}}\ln\left({\frac{N-{\frac{U}{\mathcal{E}}}}{{\frac{U}{\mathcal{E}}}}}\right)$$
נבודד את \(U\) ונקבל:
$$U\left(T\right)=\frac{N\mathcal{E}}{1+e^{\frac{\mathcal{E}}{k_{\mathrm{B}}T}}}$$

\end{example}
\section{המערכת המבודדת - הצבר המיקרוקנוני}

\begin{definition}[צבר מיקרוקנוני]
המערכת המבודדת. זוהי מערכת שבה האנרגיה, מספר החלקיקים והנפח נשארים קבועים.

\end{definition}
\begin{proposition}
אם אין אינטרקציה בין החלקיקים האנרגיה הפוטנציאלית תהיה אפס.

\end{proposition}
\begin{corollary}
סך האנרגיה של מערכת ללא אינטרקציה בין החלקיקים תהיה:
$$E=\sum_{i=1}^{3N}\frac{p_{i}^{2}}{2m}.$$

\end{corollary}
\begin{reminder}
נפח של כדור \(N\) מימדי יהיה:
$$V_{N}(R)=\frac{\pi^{N/2}}{\Gamma\left(\frac{N}{2}+1\right)}R^{N}$$
כאשר עבור אליפסויד הנפח יהיה:
$$V_{N}=\frac{2}{N}\frac{\pi^{N/2}}{\Gamma(N/2)}(a_{1}a_{2}a_{3}\ldots a_{N})$$
כאשר \(a_{1},\dots,a_{n}\) מייצגות את אורך החציר(חצי ציר ראשי).

\end{reminder}
\begin{definition}[הפונקציה הצוברת של מספר המצבים]
מספר המיקרו מצבים עד האנרגיה \(E\). מסומן ב-\(\Gamma=\Gamma(E)\).

\end{definition}
\begin{proposition}
אם נחלק את המרחב פאזה לנפחים קטנים כך שיש סיכוי שווה שיהיה מצב בכל יחידת נפח נקבל כי כל המיקרומצבים האפשריים יהיו המקירומצבים עם אנרגיה קטנה מ-\(E\). זה יהיה אוסף כל ה-\(p_{i}\) אשר מקיימים:
$$E\,\geq\,\sum_{i=1}^{3N}\frac{p_{i}^{2}}{2m}\implies 2mE \geq \sum_{i=1}^{3N} p_{i}^{2}$$
כלומר זה יהיה כדור \(3N\) מימדי עם רדיוס \(\sqrt{ 2mE }\).

\end{proposition}
\begin{remark}
הגודל של הקוביות שאנחנו מחלקים הוא שרירותי, ואין לו חשיבות במכניקה סטטיסטית קלאסית, אך בדרך כלל כדי שיהיה זהה לתוצאות במכניקה סטטיסטית קוונטית נבחר קוביות קטנות בנפח \(h\) כיוון שזה הנפח הבסיסי לפי העקרון האי וודואות הקוונטי:
$$\Delta x\Delta p_{x}\geq h.$$

\end{remark}
\begin{proposition}
במרחב הפאזה אם אנחנו מניחים שמורכב מקוביות שוות הסתברות נקבל כי:
$$\Gamma\left(E\right)={\frac{\text{Volume in phase space up to energy }E}{\text{Volume in phase space of a single state}}}=\frac{\int_{U\left( \mathbf{p},\mathbf{x} \right)\leq E}\mathrm{d}\mathbf{p}\mathrm{d}\mathbf{x}}{h^{d}} $$
כאשר הנפח של קובייה יחידה תהיה \(h^{d}\). והגבולות נקבעות לפי הדרישה שהאנרגיה המקסילית היא \(E\).

\end{proposition}
\begin{example}[פונקציה צוברת של המצבים עבור גז אידיאלי]
עבור גז אידיאלי נקבל כי \(\mathcal{H}= \frac{p^{2}}{2m}\). נסכום על כל הקוביות במרחב פאזה כאשר הגבולות שלנו נקבעות לפי האנרגיה המקסימלית:
$$E_{\max }=\frac{p_{\max }^{2}}{2m}\implies p_{\max }^{2}=2E_{\max }m\implies p\leq \sqrt{ 2E_{\max }m }$$
נשים לב כי זה לעשה מייצג נפח של כדור תלת מידי במרחב הפאזה. בנוסף אפשר לראות כי האנרגיה אינה תלויה במיקום ולכן אין אילוץ כלל על המקום \(\mathbf{x}\). באופן כללי נדרש לבצע אינטגרל על המרחב כולו אבל כדי שהאינטגרל לא יתבדר נדרוש כי הנפח של המיכל הוא \(V\), ולכן:
$$\Gamma(E)=\frac{1}{h^{3}}\int_{0}^{\sqrt{ 2Em }}  \, d\vec{p} \int_{V}  \, d\vec{x}=  \frac{V}{h^{3}}V_{\text{Sphere}}=\frac{V}{h^{3}} \frac{4\pi}{3}(2Em)^{3/2}$$

\end{example}
\begin{corollary}
מספר המיקרו מצבים המתאימים עבור אנרגיה \(E\) נפח \(V\) ומספר חלקיקים \(N\) יהיה:
$$\Gamma(E,V,N)=\frac{V^{N}}{h^{3N}N!}\frac{\left( 2\pi m E \right)^{3N/2}}{\Gamma\left(\frac{3N}{2}+1\right)}$$
כאשר חלקנו ב-\(N!\) כיוון שהחלקיקים לא מובחנים. כלומר אם נחליף בין שתי חלקיקים זה יהיה עדיין אותו מצב. בלי חלוקה זו האנטרופיה לא תהיה גודל אקסטנסיבי.

\end{corollary}
\begin{proposition}
מספר המצבים בין רמת אנרגיה \(E\) לרמת אנרגיה \(E+\Delta E\) לא תלויה -\(\Delta E\) בגבול התרמודינמי, כלומר כאשר דרגות החופש שואפות לאינסוף.

\end{proposition}
\begin{proof}
השינוי בכמות המצבים כתוצאה בשינוי קטן באנרגיה יהיה שקול לשינוי שנפח של כדור \(N\) מימדי עבור שינוי קטן ברדיוס. אנחנו רוצים להסתכל על הצפיפות של המצבים ולכן נסתכל על היחס של הקלפה הקטנה לעומת הנפח הכולל כאשר מספר המימדים יהיה גדול.
$$\frac{V_{N}(R)-V_{N}\left( R-\Delta R \right)}{V_{N}(R)}=\frac{R^{N}-\left( R-\Delta R \right)^{N}}{R^{N}}= 1-\left( 1-\frac{\Delta R}{R} \right)^{N}\xrightarrow{N\gg 1} 1$$

\end{proof}
\begin{definition}[צפיפות המצבים]
מספר המיקרו מצבים בין אנרגיה \(E\) לאנרגיה \(E+dE\). מקיים:
$$\Omega\left(N,V,E\right)=\Gamma\left(N,V,E\right)-\Gamma\left(N,V,E-d E\right)\equiv g\left(N,V,E\right)d E$$
כאשר צפיפות המצבים היא פונקציה \(g\) כך שמתקיים:
$$g(E)=\frac{\partial \Gamma}{\partial E} $$

\end{definition}
\begin{corollary}
צפיפות המצבים היא תהיה הפונקציה \(g(E)\) אשר מקיימת:
$$\Gamma(E)=\int_{0}^{E}g\left( E^{\prime} \right)d E^{\prime}$$

\end{corollary}
\begin{corollary}
עבור המרחב הפאזה של המיקום ותנע נקבל מהתוצאה הקודמת על הפונקציה הצוברת את הצפיפות מצבים הבאה:
$$g(E,V,N)=\frac{\partial \Gamma}{\partial E} =\frac{V^{N}}{h^{3N}N!}\frac{\left( 2\pi m \right)^{3N/2}}{\Gamma\left( \frac{3N}{2}+1 \right)}\frac{3N}{2}E^{(3N/2)-1}$$

\end{corollary}
\begin{example}
עבור חלקיק יחיד נקבל:
$$g(E,V)=\frac{V}{h^{3}}\frac{\pi}{4}(8m)^{3/2}E^{1/2}$$

\end{example}
\begin{proposition}
בעזרת קירוב סטרלינג נקבל כי האנטרופיה של גז אידיאלי תהיה:
$$S(E,V,N)=N k_{B}\left[\ln\left(\frac{V}{N}\right)+\frac{3}{2}\ln\left(\frac{E}{N}\right)+\frac{3}{2}\ln\left(\frac{4\pi m}{3h^{2}}\right)+\frac{5}{2}\right]$$

\end{proposition}
\begin{proposition}
עבור גז אידיאלי נקבל כי הטמפרטורה תהיה נתונה על ידי:
$$T(E,V,N)={\frac{2E}{3N k_{B}}}$$

\end{proposition}
\begin{proof}
נובע ישירות מהגדרת האנטרופיה:
$$\left({\frac{\partial S}{\partial E}}\right)_{V,N}\!\!\!={\frac{1}{T}}={\frac{3N k_{B}}{2E}}$$
ועל ידי העברת אגפים נקבל את המבוקש.

\end{proof}
\begin{corollary}[חוק החלוקה השווה]
$$E=3N\left({\frac{1}{2}}k_{B}T\right)$$
כאשר כל גורם ריבועי בהמילטוניאן למעשה יתן אנרגיה של \(\frac{k_{B}T}{2}\). וכן בפרט עבור גז אידיאלי יש \(3N\) גורמים ריבועיים.

\end{corollary}
\begin{proposition}
לחץ של מערכת מבודדת של גז אידיאלי תהיה נתונה על ידי:
$$P={\frac{N k_{B}T}{V}} = \frac{3E}{2V}$$

\end{proposition}
\begin{proof}
$$\left(\frac{\partial S}{\partial V}\right)_{E,N}=\frac{P}{T}=\frac{N k_{B}}{V}$$
וכן מחוק החלוקה השוואה ניתן לקבל מזה:
$$P=\frac{3E}{2V}$$

\end{proof}
\begin{example}[מגנט מבודד]
נמצא מה המגנטיזציה כתלות בשדה החיצונית \(\vec{B}\). מומנט מגנטי:
$$E=-\vec{m}\cdot \vec{B} = -m_{z}{B}$$
כאשר \(m_{z}\) נתון על ידי:
$$m_{z}= \pm \frac{1}{2}g \mu$$
כאשר קבוע בור \(\mu= \frac{|e|\hbar}{2m}\)(Bohr magnetism) והפקטור \(g\approx 2\) (g-factor). ולכן \(m_{z}\approx \pm \mu\).
עבור ספין \(\uparrow_{z}\) נקבל \(m_{z}=-\mu\) ויהיה עם אנרגיה \(E=\mu B\) כאשר עבור ספין \(\downarrow_{z}\) נקבל \(m_{z}=\mu\) ויהיה עם אנרגיה \(E=-\mu B\).
נסמן ב-\(N_{\uparrow}\) את כמות הספינים כלפי מעלה, וב-\(N_{\downarrow}\) את כמות הספינים כלי מטה, כך שמתקיים \(N=N_{\uparrow}+N_{\downarrow}\). כעת נגדיר:
$$\tilde{m}=\frac{N_{\uparrow }-N_{\downarrow }}{N_{\uparrow }+N_{\downarrow }}\implies \begin{cases}E=\mu B N \cdot \tilde{m} \\\frac{M}{N}=-\mu \cdot \tilde{m}
\end{cases}$$
לפי בולצמן נדרש לחשב את הכפליות.
$$\Omega\left( N_{\uparrow },N_{\downarrow } \right)=\frac{N!}{N_{\uparrow }!N_{\downarrow }!}$$
כאשר נשתמש כעת בקירוב סטרלינג:
\begin{gather*}\log\left( \Omega \right)=\log(N!)-\log\left( N_{\uparrow }! \right) - \log\left( N_{\downarrow }! \right)= \\= N \log N - N -N_{\uparrow }\log\left( N_{\uparrow } \right)+N_{\uparrow }-N_{\downarrow }\log\left( N_{\downarrow } \right)+N_{\downarrow }=  \\=\left( N_{\uparrow }+N_{\downarrow } \right) \log N -N_{\uparrow }\log\left( N_{\uparrow } \right)-N_{\downarrow }\log\left( N_{\downarrow } \right)= \\=N_{\uparrow }\log\left( \frac{N}{N_{\uparrow }} \right)+N_{\downarrow }\log\left( \frac{N}{N_{\downarrow }} \right)
\end{gather*}
כאשר כעת:
$$\tilde{m}= \frac{N_{\uparrow }-N_{\downarrow }}{N_{\uparrow }+N_{\downarrow }}\implies \begin{cases} \tilde{m}N = N_{\uparrow }-N_{\downarrow } \\N= N_{\uparrow }+N_{\downarrow }\end{cases}\implies  \begin{cases}N_{\uparrow }= \frac{\tilde{m} + 1}{2} N \\N_{\downarrow }= \frac{1 - \tilde{m}}{2} N
\end{cases}$$
ולכן נקבל כי כפליות המצב תהיה:
$$\log\left( \Omega \right)=-N_{\uparrow }\log\left( \frac{1+\tilde{m}}{2} \right)-N_{\downarrow }\log\left( \frac{1-\tilde{m}}{2} \right)=-N\left[ \frac{\tilde{m}+1}{2}\log\left( \frac{\tilde{m}+1}{2} \right)+\frac{1-\tilde{m}}{2} \log\left( \frac{1-\tilde{m}}{2} \right)\right]$$
כאשר האנטרופיה לחלקיק תהיה:
$$\frac{S}{N}=\frac{k_{B}\log\left( \Omega \right)}{N}$$
כאשר לדוגמא עבור \(\tilde{m}=0\) נקבל:
$$\frac{S}{N}\left( \tilde{m} = 0 \right)=k_{B}\left( -\frac{1}{2}\log\left( \frac{1}{2} \right)-\frac{1}{2}\log\left( \frac{1}{2} \right) \right)=k_{B}\log(2)$$
כעת:
$$\frac{S\left( \tilde{m} \right)}{Nk_{B}}=-\frac{\tilde{m}+1}{2}\log\left( \frac{1+\tilde{m}}{2} \right)- \frac{1-\tilde{m}}{2}\log\left( \frac{1-\tilde{m}}{2} \right)$$
כאשר כעת:
$$S(E,V,N,B)=N k_{B} f\left( \tilde{m} \right)$$
כאשר:
$$f\left( \tilde{m} \right)= - \frac{1+\tilde{m}}{2}\log\left( \frac{1+\tilde{m}}{2} \right)-\frac{1-\tilde{m}}{2}\log\left( \frac{1-\tilde{m}}{2} \right)$$
ניתן למצוא את הטמפרטורה:
$$\frac{1}{T}=\left.\frac{\partial S}{\partial E}\right|_{V,B} $$
כאשר כיוון ש-\(E=\mu B\left( N_{\uparrow}-N_{\downarrow} \right)=\mu BN\tilde{m}\) נקבל:
$$\frac{1}{T} = \frac{\partial  S}{\partial \tilde{m}} \frac{\partial \tilde{m}}{\partial E} = Nk_{B}\frac{\mathrm{d} f\left( \tilde{m} \right)}{\mathrm{d} \tilde{m}}  \frac{1}{\mu B N}=\frac{k_{B}}{\mu B}\frac{\mathrm{d} f}{\mathrm{d} \tilde{m}} $$
כאשר מתקיים:
$$\frac{\mathrm{d} f}{\mathrm{d} \tilde{m}} =-\frac{\mathrm{d} }{\mathrm{d} \tilde{m}} \left(\frac{ 1-\tilde{m}}{2}\log\left( \frac{1-\tilde{m}}{2} \right)+\frac{1+\tilde{m}}{2}\log\left( \frac{1+\tilde{m}}{2} \right) \right)=\frac{1}{2}\log\left( \frac{1-\tilde{m} }{1+\tilde{m}}\right)$$

\end{example}
\begin{summary}
  \begin{itemize}
    \item הצבר המיקרוקנוני הוא מערכת מבודדת אשר מתאפיינת בכך שהאנרגיה, הנפח ומספר החלקיקים קבועים.
    \item המשתנה המרכזי בצבר זה הוא האנטרופיה, כאשר אם מוצאים את האנטרופיה ניתן למצוא הכל.
    \item מוצאים את האנטריפיה על ידי ספירת מצבים ושימוש באנטרופיית בולצמן עם קירוב סטרלינג.
    \item ניתן לספור את המצבים על ידי מציאת צפיפות המצבים ואז \({\Omega}(E)=\int_{0}^{E}g(E^{\prime})d E^{\prime}\)
    \item האנטרופיה מקיימת \(dS= \frac{1}{T}dE+\frac{p}{T}dV\) ולכן:
$$\frac{1}{T}=\left. \frac{\partial S}{\partial E} \right \rvert_{V} \qquad \frac{p}{T}=\left. \frac{\partial S}{\partial V} \right \rvert_{E}$$
  \end{itemize}
\end{summary}
\section{המערכת הסגורה - הצבר הקנוני}

\begin{definition}[הצבר הקנוני]
מאגר חום עם אנרגיה \(E_{R}\) המחובר למערכת עם אנרגיה \(E_{A}\) כך ש-\(E_{R}\gg E_{A}\) והאנרגיה הכוללת \(E_{0}=E_{A}+E_{R}\) היא קבועה.

\end{definition}
\begin{proposition}[התפלגות הההסתברות של הצבר הקנוני]
ההסתברות של מצב להיות באנרגיה \(E_{A}\) יהיה פרופורציונאלי ל-\(e^{ -E_{A}/k_{B}T }\)

\end{proposition}
\begin{remark}
התפלגות זו מכונה התפלגות בולצמן.

\end{remark}
\begin{proof}
אנו יודעים כי \(S(E_{R})=S(E_{0}-E_{A})\) כאשר כיוון ש\(E_{R}\gg E_{A}\) נקבל כי ניתן לפתח בעזרת טור טיילור ולקבל:
$$S(E_{\mathrm{R}})=S(E_{0})-E_{\mathrm{A}}\frac{\partial S}{\partial E}+\cdots$$
כאשר ניתן להזניח גורמים מסדר גבוה ולהציב \(\frac{1}{T}=\frac{\partial S}{\partial E}\). נקבל:
$$S(E_{\mathrm{R}})=S(E_{0})-{\frac{E_{\mathrm{A}}}{T}}$$
כאשר אנו יודעים כי \(S=k_{B}\ln \Omega\) ולכן ההסתברות שיהיה במצב \(E_{R}\) תהיה פרופורציונית לכמות המיקרו מצבים \(\Omega(E_{R})\) ולכן:
$$P(E_{R})\propto e^{S(E_R)/k_{B}}\approx e^{S(E_{0})-E_{A}/k_{B}T}=e^{ S(E_{0}) }e^{ -E_{A}/k_{B}T }$$
כלומר נקבל כי ההסתברות ברופורצינאלית ל-\(e^{ -E_{A}/k_{B}T }\).

\end{proof}
\begin{definition}[ניוון]
כמות המיקרו מצבים עם רמת אנרגיה \(E_{i}\). לכרגע נסמן את זה ב-\(g_{i}\).

\end{definition}
\begin{definition}[פונקצייית החלוקה]
הפקטור נרמול של ההסתברות:
$$Z=\sum_{i}g_{i}e^{-E_{i}/k_{\mathrm{{B}}}T}$$
כאשר:
$$ P(E_{i})={\frac{1}{Z}}g_{i}e^{-E_{i}/k_{\mathrm{B}}T}$$

\end{definition}
\begin{remark}
למעשה פונקציית החלוקה היה תהיה התמרת לפלס של פונקציית מספר המצבים.

\end{remark}
\begin{example}[מערכת שתי רמות]
עבור מערכת שתי רמות, נניח והמערכת יכולה להיות במצב \(-\frac{\Delta}{2}\) או \(\frac{\Delta}{2}\). מתקיים:
$$Z=\sum_{\alpha}\mathrm{e}^{-\beta E_{\alpha}}=\mathrm{e}^{\beta\Delta/2}+\mathrm{e}^{-\beta\Delta/2}=2\cosh\left({\frac{\beta\Delta}{2}}\right)$$

\end{example}
\begin{example}[אוסצילטור הרמוני קוונטי]
האנרגיות האפשריות של המערכת יהיו מהצורה \(\left( n+\frac{1}{2} \right)\hbar \omega\) ולכן:
$$Z=\sum_{\alpha}\mathrm{e}^{-\beta E_{\alpha}}=\sum_{n=0}^{\infty}\mathrm{e}^{-\beta\left( n+\frac{1}{2} \right)\hbar\omega}=\mathrm{e}^{-\beta\frac{1}{2}\hbar\omega}\sum_{n=0}^{\infty}\mathrm{e}^{-n\beta\hbar\omega}=\frac{\mathrm{e}^{-\frac{1}{2}\beta\hbar\omega}}{1-\mathrm{e}^{-\beta\hbar\omega}}=\frac{2}{\sinh\left( \frac{1}{2}\beta\hbar \omega \right)}$$

\end{example}
\begin{example}[מערכת של \(N\) רמות]
נסמן את רמות האנרגיה ב-\(0,\hbar {\omega},\dots,(N-1)\hbar \omega\) ונקבל:
$$Z=\sum_{\alpha}\mathrm{e}^{-\beta E_{\alpha}}=\sum_{j=0}^{N-1}\mathrm{e}^{-j\beta\hbar\omega}=\frac{1-\mathrm{e}^{-N\beta\hbar\omega}}{1-\mathrm{e}^{-\beta\hbar\omega}}$$

\end{example}
\begin{proposition}[אנרגיה פנימית בעזרת התפלגות בולצמן]
$$U=-\frac{\mathrm{d} \ln Z}{\mathrm{d} \beta} =k_{\mathrm{B}}T^{2}{\frac{\mathrm{d}\ln Z}{\mathrm{d}T}}$$

\end{proposition}
\begin{proof}
אם נגדיר \(\beta=\frac{1}{k_{B}T}\) נקבל כי האנרגיה הפנימית תהיה שווה ל:
$$U=\sum_{i}E_{i}P(E_{i})=\frac{\sum_{i}E_{i}\mathrm{e}^{-\beta E_{i}}}{\sum_{i}\mathrm{e}^{-\beta E_{i}}}$$
כאשר המכנה זה פונקציית החלוקה, ולכן מתקיים:
$$-{\frac{\mathrm{d}Z}{\mathrm{d}\beta}}=\sum_{i}E_{i}\mathrm{e}^{-\beta E_{i}}$$
ולכן:
$$U=-\frac{1}{Z}\left( \frac{\mathrm{d} Z}{\mathrm{d} \beta}  \right)\implies U=-\frac{\mathrm{d} \ln Z}{\mathrm{d} \beta} \implies U=k_{\mathrm{B}}T^{2}{\frac{\mathrm{d}\ln Z}{\mathrm{d}T}}$$

\end{proof}
\begin{proposition}[אנטרופיה בעזרת פונקציית החלוקה]
$$S={\frac{U}{T}}+k_{\mathrm{B}}\ln Z$$

\end{proposition}
\begin{proof}
כמו מקודם נגדיר \(\beta = \frac{1}{k_{B}T}\). אנו יודעים כי \(P_{j}=\frac{e^{ -\beta E_{j} }}{Z}\) ולכן:
$$\ln{ P}_{j}=-\beta E_{j}-\ln Z$$
וכן מתקיים \(S=-k_{\mathrm{B}}\sum_{i}P_{i}\ln P_{i}\). ולכן נקבל:
\begin{gather*}{{S}}={{-k_{\mathrm{B}}\sum_{i}P_{i}\ln P_{i}}}= k_{\mathrm{B}}\sum_{i}P_{i}\left( \beta E_{i}+\ln Z \right)= k_{B}\left( \beta \sum_{i}P_{i}E_{i}+\sum_{i}P_{i}\ln Z \right)
\end{gather*}
כאשר כיוון ש-הסכום של ההסתברות להיות הרמת אנרגיה כפול האנרגיה ברמה תהיה סך האנרגיה נקבל:
$$k_{\mathrm{{B}}}\left( \beta U+\frac{1}{Z} \ln Z\sum e^{ -\beta_{i}E_{i} } \right) =k_{B}\left( \beta U+ \frac{Z}{Z}\ln Z \right)$$
כאשר כתבנו בסכום השני את \(P_{i}\) מפורשות, הוצאנו את \(Z\) ו-\(\ln Z\) מהסכום וקיבלנו מהסכום את פונקציית החלוקה. נציב חזרה את \(\beta\) ונקבל:
$$S={\frac{U}{T}}+k_{\mathrm{B}}\ln Z$$

\end{proof}
\begin{proposition}[אנרגיה חופשית של הלמהולץ בעזרת פונקציית החלוקה]
$$F=-k_{\mathrm{B}}T\ln Z$$
ולכן גם ניתן לכתוב:
$$Z=e^{ -F/k_{B}T }$$

\end{proposition}
\begin{proof}
נובע מההגדרה \(F=  U-  T  S\) עם הביטוי לאנטרופיה מהטענה הקודמת.

\end{proof}
\begin{corollary}
ניתן בעזרת הגדרה זו להשיג ביטוי של הרבה גדלים אחרים בעזרת הפונקציית חלוקה. למשל:
\begin{gather*}p=-\left({\frac{\partial F}{\partial V}}\right)_{T}=k_{\mathrm{B}}T\left({\frac{\partial\mathrm{ln}\,Z}{\partial V}}\right)_{T}  \\H=U+p V=k_{\mathrm{{B}}}T\left[T\left({\frac{\partial\mathrm{ln}\,Z}{\partial T}}\right)_{V}+V\left({\frac{\partial\mathrm{ln}\,Z}{\partial V}}\right)_{T}\right] \\G=F+p V=k_{\mathrm{{B}}}T\left[-\ln Z+V\left({\frac{\partial\ln Z}{\partial V}}\right)_{T}\right] \\C_{V}=k_{\mathrm{{B}}}T\left[2\left(\frac{\partial\mathrm{ln}\,Z}{\partial T}\right)_{V}+T\left(\frac{\partial^{2}\mathrm{ln}\,Z}{\partial T^{2}}\right)_{V}\right]
\end{gather*}

\end{corollary}
\begin{example}
עבור מערכת שתי רמות(עם אנרגיות \(\frac{\Delta}{2}\) ו-\(-\frac{\Delta}{2}\)) ראינו כי פונקציית החלוקה היא מהצורה:
$$Z=2\cosh\left(\frac{\beta\Delta}{2}\right)$$
ולכן האנרגיה הפנימית תהיה:
$$U=-\frac{\mathrm{d}\ln Z}{\mathrm{d}\beta}=-\frac{\Delta}{2}\operatorname{tanh}\left(\frac{\beta\Delta}{2}\right).$$
קיבול החום יהיה:
$$C_{V}=\left({\frac{\partial U}{\partial T}}\right)_{V}=k_{\mathrm{B}}\left({\frac{\beta\Delta}{2}}\right)^{2}\mathrm{sech}^{2}\left({\frac{\beta\Delta}{2}}\right)$$
פונקציית הלמהולדס תהיה:
$$F=-k_{\mathrm{B}}T\ln Z=-k_{\mathrm{B}}T\ln\left[2\cosh\left({\frac{\beta\Delta}{2}}\right)\right]$$
ותן לקבל מזה ישירות את האנטרופיה:
$$S=\frac{U-F}{T}=-\frac{\Delta}{2T}\operatorname{tanh}\left(\frac{\beta\Delta}{2}\right)+k_{\mathrm{B}}\ln\left[2\cosh\left(\frac{\beta\Delta}{2}\right)\right]$$
כלומר מפונקציית החלוקה ניתן לקבל את כל הגדלים של המערכת!

\end{example}
\begin{proposition}[איחוד פונקציות חלוקה]
אם האנרגיה \(E\) היא סכום של שתי תרומות \(E^{(a)}\) ו-\(E^{(b)}\) כך שהרמת אנרגיה \(E_{ij}\) נתונה על ידי:
$$E_{i,j}=E_{i}^{(a)}+E_{j}^{(b)}$$
מתקיים:
$$Z=Z_{a}Z_{b}$$

\end{proposition}
\begin{proof}
$$Z=\sum_{i}\sum_{j}\mathrm{e}^{-\beta(E_{i}^{(a)}+E_{j}^{(b)})}=\sum_{i}\mathrm{e}^{-\beta E_{i}^{(a)}}\sum_{j}\mathrm{e}^{-\beta E_{j}^{(b)}}=Z_{a}Z_{b}$$

\end{proof}
\begin{corollary}
פונקציית החלוקה של רכיבים בלתי תלויים מכפילים אחד בשני, ולכן בפרט \(\ln Z\) עבור רכיבים בלתי תלויים נסכמים אחד עם השני(\(\ln(Z)=\ln(Z_{a})+\ln(Z_{b})\)).

\end{corollary}
\section{התפלגות בולצמן במרחב הפאזה}

\begin{proposition}[התפלגות הסתברות במרחב הפאזה]
$$P(p,q)=\frac{1}{\tilde{Z}(T,V,N)}\exp[-\beta H(p,q)]$$
כאשר מתנאי הנרמול:
$$\tilde{Z}(T,V,N)=\int\!\!d q\int\!\!d p\,\exp[-\beta H(q,p)]$$

\end{proposition}
\begin{proof}
מתקיים:
$$P(p,q)=\frac{\Omega_{R}(E_{T}-H(p,q))}{\Omega_{T}(E_{T})}$$
כאשר כיוון ש-\(E_{T}\gg H(p,q)\) ניתן לקחת את הלוגריתם של שתי הצדדים ולפתח את \(\ln\left( \Omega_{R} \right)\) על ידי חזקות של \(\frac{H(p,q)}{E_{T}}\) ולקבל:
$$\ln P(p,q)=\ln\Omega_{R}(E_{T})-H(p,q)\frac{\partial}{\partial E_{T}}\Omega_{R}(E_{T})-\ln\Omega_{T}(E_{T})+\cdots$$
כאשר נזכור כי:
$$\beta=\beta_{R}\equiv\frac{\partial}{\partial E_{T}}\Omega_{R}(E_{T})$$
כאשר רק האיבר השני תלוי ב-\(p,q\) ולכן ניתן לאחד את יתר הגורמים בפונקציה \(\tilde{Z}\):
$$\ln P(p,q)=-\beta H(p,q)-\ln\tilde{Z}(T,V,N)$$
כלומר:
$$P(p,q)=\frac{1}{\tilde{Z}(T,V,N)}\exp[-\beta H(p,q)]$$

\end{proof}
\begin{reminder}
המשוואה הכללית של מספר המצבים במרחב פאזה הוא:
$$\Omega(E,V,N)=\frac{1}{h^{3N}N!}\int\!\!d q\int\!\!d p\,\delta(E-H(q,p))$$

\end{reminder}
\begin{corollary}
$$\widetilde{Z}=\,\frac{1}{h^{3N}N!}\int\!\!d q\int\!\!d p\,\exp[-\beta H(q,p)]$$

\end{corollary}
\begin{proof}
\begin{gather*}\widetilde{Z}=\int\!\!d E\frac{1}{h^{3N}N!}\int\!\!d q\int\!\!d p\,\delta(E-H(q,p))\exp\left( -\beta E \right)=\\=\frac{1}{h^{3N}N!}\int\!\!d q\int\!\!d p\,\int\!\!d E\,\delta(E-H(q,p))\exp\left( -\beta E \right)=\\=\frac{1}{h^{3N}N!}\int\!\!d q\int\!\!d p\,\exp\left[ -\beta H(q,p) \right] 
\end{gather*}

\end{proof}
\begin{corollary}
אם נשוואה להתפלגות בולצמן נקבל:
$$\tilde{Z}(T,V,N)=h^{3N}N!\,Z$$
ולכן ההתפלגות במרחב הפאזה נתונה על ידי:
$$P(p,q)=\frac{1}{h^{3N}N!\,Z}\exp[-\beta H(p,q)]$$

\end{corollary}
\begin{summary}
  \begin{itemize}
    \item מערכת שלא מחליפה חלקיקים עם הסביבה אבל כן מחליפה אנרגיה מכונה הצבר הקנוני. כאשר לרוב נבצע אידיאליזציה שבה הסביבה היא אמבט חום.
    \item ההסתברות להיות באנרגיה \(E\) נתונה על ידי התפלגות בולצמן \(P(E)=\frac{1}{Z}\Omega(E)\exp(-\beta E)\) כאשר \(Z\) נקראת פונקציית החלוקה.
    \item בעזרת פונקציית החלוקה ניתן לקבל את האנרגיה החופשית של הלמהולץ, האנטרופיה והאנרגיה הפנימית בצורה הבאה:
$$U=k_{\mathrm{B}}T^{2}{\frac{\mathrm{d}\ln Z}{\mathrm{d}T}} \qquad F=-k_{\mathrm{B}}T\ln Z \qquad S={\frac{U}{T}}+k_{\mathrm{B}}\ln Z
$$
    \item אם יש רכיבים שונים(בלתי תלויים) לאנרגיה הפונקצייה החלוקה הכוללת תהיה המכפלה של כל אחד מהם.
    \item במרחב הפאזה ההסתברות להיות בתנע \(p\) ומיקום \(q\) יהיה נתון על ידי:
$$P(p,q)=\frac{1}{h^{3N}N!\,Z}\exp[-\beta H(p,q)]$$
  \end{itemize}
\end{summary}
\section{פיסיקה סטטיסטית של גזים}

\begin{definition}[חלקיקים ניתנים להבחנה]
אם ניתן להחליף שתי חלקיקים ובלי לשנות את המיקרו מצב נקבל כי החלקיקים לא ניתנים להבחנה. כאשר אם החלפה של שתי חליקיקים תשנה את המיקרו מצב חלקיקים אלו יהיו ניתנים להבחנה.

\end{definition}
\begin{proposition}
עבור \(N\) חלקיקים אם כל החלקיקים במערכת ניתנים להבחנה נקבל כי:
$$Z_{N}=(Z_{1})^{N}$$
כאשר \(Z_{N}\) היא פונקציית החלוקה של \(N\) חלקיקים ו-\(Z_{1}\) היא פונקציית החלוקה של חלקיק אחד.
אם לא ניתנים להבחנה, נקבל:
$$Z_{N}=\frac{(Z_{1})^{N}}{N!}$$

\end{proposition}
\begin{proof}
אם כל החלקיקים במערכת ניתנים להבחנה נקבל כי \(E_{tot}=\sum E_{i}\) עבור כל חלקיק. ולכן מהטענה הקודמת עבור \(N\) חלקיקים נקבל:
$$Z_{N}=(Z_{1})^{N}$$
אם לא ניתנים להבחנה, אז יש חפיפה בין האנרגיות! לכן נדרש לחלק ב-\(N!\).

\end{proof}
\begin{proposition}[פונקציית החלוקה של גז אידיאלי של חלקיק יחיד]
$$Z_{1}=\frac{V}{\lambda_{\mathrm{T}}^{3}} \qquad \lambda_{\mathrm{T}}=\frac{h}{\sqrt{2\pi m k_{\mathrm{B}}T}}$$

\end{proposition}
\begin{proof}
נסמן \(k=\frac{p}{\hbar}\) התדר מרחבי. אנו יודעים כי פונקציית החלוקה במרחב התדר המרחבי יהיה:
$$Z_{1}=\int_{0}^{\infty}\mathrm{e}^{-\beta E(k)}\,g(k)\,\mathrm{d}k$$
כאשר אנו יודעים כי האנרגיה תהיה:
$$E(k)=\frac{p}{2m}=\frac{\hbar^{2}k^{2}}{2m}$$
ולכן:
$$Z_{1}=\int_{0}^{\infty}\mathrm{e}^{-\beta\hbar^{2}k^{2}/2m}{\frac{V k^{2}\,\mathrm{d}k}{2\pi^{2}}}={\frac{V}{\hbar^{3}}}\left({\frac{m k_{\mathrm{B}}T}{2\pi}}\right)^{3/2}$$
כאשר עבור \(\lambda_{\mathrm{T}}=\frac{h}{\sqrt{2\pi m k_{\mathrm{B}}T}}\) נקבל \(Z_{1}=\frac{V}{\lambda_{T}^{3}}\)

\end{proof}
\begin{remark}
לעיתים מגדירים גודל הנקרא הצפיפות הקוונטית:
$$n_{\mathrm{Q}}=\frac{1}{\hbar^{3}}\left(\frac{m k_{\mathrm{B}}T}{2\pi}\right)^{3/2}$$
ואז \(Z_{1}=Vn_{Q}\).

\end{remark}
\begin{proposition}[פונקציית החלוקה של גז אידיאלי]
$$Z_{\mathrm{ideal\,gas}}=\frac{1}{N!}\left(\frac{V}{\lambda^{3}}\right)^{N},\;\;\;\;\;\lambda=\frac{h}{\sqrt{2\pi m k_{B}T}}$$
כאשר \(\lambda\) נקרא אורך גל דה ברויי התרמי.

\end{proposition}
\begin{proof}
בגבול שבו \(n\ll n_{Q}\) (או \(n\lambda_{T}^{3}\ll 1\)) האפקטים הקוונטים לא משמעותיים ומערכת של גז אידאלי יהיה אוסף של חלקיקים בלתי מובחנים ולכן:
$$Z_{N}=\frac{1}{N!}\left(\frac{V}{\lambda_{\mathrm{th}}^{3}}\right)^{N}$$

\end{proof}
\begin{example}[מציאת פוטנציאל כימי]
כאשר אנחנו נמצאים בשיווי משקל ואין חילוף חלקיקים ניתן לכתוב:
$$F=-k_{B}T\ln Z_{N}=-k_{B}T\ln \frac{Z_{1}}{N!}\implies F=-k_{B}T\left[ N\ln Z_{1}-N\ln N+N \right]$$
ולכן הפוטנציאל הכימי יהיה:
$$\mu=\frac{\partial F}{\partial N} =-k_{B}T\left[ \ln Z_{1}-\ln N+1 -1 \right]=-k_{B}T\ln \left( \frac{V}{N\lambda^{3}} \right)$$

\end{example}
\begin{summary}
  \begin{itemize}
    \item עבור \(N\) חלקיקים הניתנים להבחנה נקבל \(Z_{N}=(Z_{1})^{N}\) כאשר עבור \(N\) חלקיקים הבלתי ניתנים להבחנה נקבל \(Z_{N}=\frac{1}{N!}(Z_{1})^{N}\).
  \end{itemize}
\end{summary}
\section{המערכת הפתוחה - הצבר הגאנד קנוני}

\begin{reminder}[פוטנציאל הכימי]
מדד לכמה "קשה" להכניס חלקיקים למערכת. מסומן ב-\(\mu\).

\end{reminder}
\begin{definition}[הצבר הגראנד קנוני]
מערכת שיכולה להחליף גם אנרגיה וגם חלקיקים עם אמבט. כאשר נניח כי האמבט גדול משמעותית מהמערכת. כיוון שאנו דורשים שיווי משקל נקבל כי הטמפרטורה של המערכת שווה לטמפרטורה של המאגר(\(T=T_{R}\)) וכן הפוטנציאלים הכימים שווים(\(\mu=\mu_{R}\)).

\end{definition}
\begin{proposition}
$$P_{i}=\frac{\mathrm{e}^{\beta\left(\mu N_{i}-E_{i}\right)}}{\mathcal{Z}}\qquad {\mathcal Z}=\sum_{i}\mathrm{e}^{\beta\left( \mu N_{i}-E_{i} \right)}$$

\end{proposition}
\begin{definition}[גראנד פוטנציאל]
ההתמרת לג'נדר של ההאנרגיה הפנימית עם הפוטנציאל הכימי והטמפרטורה:
$$\Phi[T,\mu]=U-T S-\mu N$$

\end{definition}
\begin{remark}
הסיבה שמשתמשים בהתמרת לג'נדר זה כי זה משמר אינפורמציה. כלומר המשוואה הזאת היא זהה מבחינת הפיזיקה אך עם משתנים שונים.

\end{remark}
\begin{proposition}[משוואת איילר]
אם מערכת היא אקסטניבית האנרגיה שלנו היא פונקציה הומוגנית מסדר ראשון, ומקיימת לכל \(\lambda\):
$$\lambda U(S,V,N)=U(\lambda S,\lambda V,\lambda N)$$
וכן מתקיים:
$$U=T S-P V+\mu N$$

\end{proposition}
\begin{proof}
נגזור את המשוואה לפי \(\lambda\) ונקבל:
\begin{gather*}{{U(S,V,N)=\frac{\partial U\left( \lambda S,\lambda V,\lambda N \right)}{\partial\left( \lambda S \right)}\frac{\partial\left( \lambda S \right)}{\partial\lambda}}} \\{{+\frac{\partial U\left( \lambda S,\lambda V,\lambda N \right)}{\partial\left( \lambda V \right)}\frac{\partial\left( \lambda V \right)}{\partial\lambda}}}{{+\frac{\partial U\left( \lambda S,\lambda V,\lambda N \right)}{\partial\left( \lambda N \right)}\frac{\partial\left( \lambda N \right)}{\partial\lambda}}} 
\end{gather*}
כאשר עבור \(\lambda=1\) נקבל:
$$U(S,V,N)=\frac{\partial U(S,V,N)}{\partial S}S+\frac{\partial U(S,V,N)}{\partial V}V+\frac{\partial U(S,V,N)}{\partial N}N$$
כאשר אם נציב את ההגדרות של הגדלים המתאימים נקבל:
$$U=T S-P V+\mu N$$

\end{proof}
\begin{remark}
משוואה זו מאוד מזכירה את המשוואה:
$$d U=T d S-P d V+\mu d N$$

\end{remark}
\begin{proposition}
עבור מערכת אקסטנסיבית הפוטנציאל הכימי מקיים:
$$\mu= \frac{U-TS+PV}{N}=\frac{G}{N}$$

\end{proposition}
\begin{proof}
עבור מערכת אקסטנסיבית אם נגדיל את המערכת פי \(\lambda\) נצפה כי כל המשתנים יגדלו פי \(\lambda\), כלומר:
$$U\to\lambda U,\qquad S\to\lambda S,\qquad V\to\lambda V,\qquad N\to\lambda N,$$
כאשר אם נכתוב את האנטרופיה \(S\) בעזרת \(U,V,N\) נקבל:
$$\lambda S(U,V,N)=S(\lambda U,\lambda V,\lambda N)$$
כאשר אם נגזור לפי \(\lambda\) נקבל:
$$S=\frac{\partial S}{\partial(\lambda U)}\frac{\partial(\lambda U)}{\partial\lambda}+\frac{\partial S}{\partial(\lambda V)}\frac{\partial(\lambda V)}{\partial\lambda}+\frac{\partial S}{\partial(\lambda N)}\frac{\partial(\lambda N)}{\partial\lambda}$$
כאשר אם נקבע \(\lambda=1\) ונשתמש ביחסים:
$$\left({\frac{\partial S}{\partial U}}\right)_{N,V}={\frac{1}{T}},\qquad\left({\frac{\partial S}{\partial V}}\right)_{N,U}={\frac{p}{T}},\qquad\left({\frac{\partial S}{\partial N}}\right)_{U,V}=-{\frac{\mu_{0}}{T}}$$
נקבל:
$$U-T S+p V=\mu N$$
כאשר נזהה את פונקציית גיבס באגף שמאל ולכן:
$$G=\mu N\implies \mu=\frac{G}{N}$$

\end{proof}
\begin{corollary}
עבור מערכות אקסטנסיביות, משוואת אוילר מתקיימת, ונקבל כי הגראנד פוטנציאל מקיים:
$$\Phi[T,\mu]=U-T S-\mu N=-P V$$

\end{corollary}
\begin{proposition}[התפלגות ההסתברות של הצבר הגראנד קנוני]
$$P(E,N)=\frac{1}{\mathcal{Z}}\Omega(E,V,N)\exp\left[-\beta E+\beta\mu N\right]$$
כאשר \(\mathcal{Z}\) נקבע על פי תנאי נרמול.

\end{proposition}
\begin{proof}
באופן דומה לצבר הקנוני, נדרוש שההסתברות תקיים:
$$P(E,N)=\frac{\Omega(E,V,N)\Omega_{R}(E_{T}-E,V_{R},N_{T}-N)}{\Omega_{T}(E_{T},V_{T},N_{T})}$$
ניקח לוגוריתם ונפתח את הטור טיילור של \(\ln \Omega_{R}(E_{T}-E, V_{R},N_{T}-N)\) בחזקות של \(\frac{E}{E_{T}},\frac{N}{N_{T}}\):
\begin{gather*}\ln P(E,N)=\ln\Omega(E,V,N)+\ln\Omega_{R}(E_{T}-E,V_{R},N_{T}-N)-\ln\Omega_{T}(E_{T},V_{T},N_{T})  \\\approx\ln\Omega(E,V,N)+\ln\Omega_{R}(E_{T},V_{R},N_{T})+E\frac{\partial}{\partial E_{T}}\ln\Omega_{R}(E_{T},V_{R},N_{T})+ \\+N\frac{\partial}{\partial N_{T}}\ln\Omega_{R}(E_{T},V_{R},N_{T})-\ln\Omega_{R}(E_{T},V_{T},N_{T})
\end{gather*}
כאשר כעת נזכור כי:
$$S_{R}=k_{B}\ln\Omega_{R}(E_{T},V_{R},N_{T})$$
לכן נקבל כעת כי \(\beta_{R}=\frac{1}{k_{B}T_{R}}\) יהיה שווה:
$$\beta_{R}\equiv\frac{\partial}{\partial E_{T}}\ln\Omega_{R}(E_{T},V_{R},N_{T})$$
כאשר מזהות נוספת עבור הפוטנציאל הכימי של המאגר נקבל:
$$-\mu_{R}\beta_{R}\equiv\frac{\partial}{\partial N_{T}}\ln\Omega_{R}(E_{T},V_{R},N_{T})$$
כעת כיוון ש-\(\beta=\beta_{R}\) ו-\(\mu=\mu_{R}\) נקבל כי הגדלים \(E_{T},N_{T},V_{R}\) לא תלויים ב-\(E\) או \(N\). ניתן לשלב את \(\ln \Omega_{R}(E_{T},V_{R},N_{T})\) ו-\(\ln\left( \Omega_{R}(E_{T}, V_{T},N_{T}) \right)\) לערך יחיד המסומן ב-\(-\ln \mathcal{Z}\). לכן:
$$\ln P(E,N)\approx\ln\Omega(E,V,N)-\beta E+\beta\mu N-\ln{\mathcal{Z}}$$
כאשר אם ניקח את האקספוננט נקבל:
$$P(E,N)=\frac{1}{\mathcal{Z}}\Omega(E,V,N)\exp\left[-\beta E+\beta\mu N\right]$$

\end{proof}
\begin{corollary}[הפונקציית הגראנד חלוקה]
$${\mathcal{Z}}=\sum_{N=0}^{\infty}\int_{0}^{\infty}d E\,\Omega(E,V,N)\exp\left[-\beta E+\beta\mu N\right]$$
או לחלופין בעזרת הפונקציית חלוקה המוגדרת על ידי:
$$Z(T,V,N)=\int_{0}^{\infty}d E\,\Omega(E,V,N)\exp(-\beta E)d E$$
ניתן לכתוב:
$${\mathcal{Z}}(T,V,\mu)=\sum_{N=0}^{\infty}Z(T,V,N)\exp\left[\beta\mu N\right]$$

\end{corollary}
\begin{proposition}[גראנד פוטנציאל בעזרת הפונקציית הגראנד חלוקה]
הפונקציית הגראנד חלוקה מקיימת:
$$\ln \mathcal{Z} = -\beta \Phi\left( T,\mu \right)$$
כאשר אם המערכת אקסטנסיבית ומקיימת את משוואת אויילר מתקיים בנוסף:
$$\ln \mathcal{Z} = \beta PV$$

\end{proposition}
\begin{example}[פונקציית הגראנד חלוקה עבור גז אידיאלי]
ראינו כי פונקציית החלוקה עבור גז אידיאלי תהיה:
$$Z=\frac{1}{h^{3N}N!}\left(2\pi m k_{B}T\right)^{3N/2}V^{N}$$
כאשר מזה נקבל:
\begin{gather*}{\mathcal{Z}}\left( T,V,\mu \right)=\sum_{N=0}^{\infty}{\frac{1}{h^{3N}N!}}\left(2\pi m k_{B}T\right)^{3N/2}V^{N}\exp\left[\beta\mu N\right]\\=\sum_{N=0}^{\infty}\frac{1}{N!}\left(\frac{\left(2\pi m k_{B}T\right)^{3/2}}{h^{3}}V e^{\beta\mu}\right)^{N}=\exp\left(\left(2\pi m k_{B}T\right)^{3/2}h^{-3}V e^{\beta\mu}\right) 
\end{gather*}
כאשר אם ניקח את הלוגוריתם של שתי האגפים נקבל מהטענה הקודמת את האנרגיה:
$$-\beta \Phi[T,\mu]=\left(2\pi m k_{B}T\right)^{3/2}h^{-3}V e^{\beta\mu}=\beta P V$$
כאשר השיוויון האחרון נובע מכך שהגז האידיאלי אקסטנסיבי. על ידי חלוקה ב-\(\beta V\) נקבל:
$$P=k_{B}T\left(2\pi m k_{B}T\right)^{3/2}h^{-3}e^{\beta\mu}$$
כאשר ניתן לכתוב את כל הביטיים בעזרת \(\lambda_{T}=\frac{h}{\sqrt{ 2\pi mk_{B}T }}\) ולקבל:
$${\mathcal{Z}}\left( T,V,\mu \right)=\exp\left({\frac{V}{\lambda^{3}}}e^{\beta\mu}\right)\quad \Phi\left( T,V,\mu \right)=-k_{B}T\frac{V}{\lambda^{3}}e^{\beta\mu}\quad P=\frac{k_{B}T}{\lambda^{3}}e^{\beta\mu}$$

\end{example}
\begin{proposition}
\begin{gather*}N=\sum_{i}N_{i}P_{i}=k_{\mathrm{B}}T\left({\frac{\partial\mathrm{ln}\;{\mathcal{Z}}}{\partial\mu}}\right)_{\beta}\\ U=\sum_{i}E_{i}P_{i}=-\left(\frac{\partial\mathrm{ln}\;\mathcal{Z}}{\partial\beta}\right)_{\mu}+\mu N\\ S=-k_{\mathrm{B}}\sum_{i}P_{i}\ln P_{i}={\frac{U-\mu N+k_{\mathrm{B}}T\ln{\mathcal{Z}}}{T}} 
\end{gather*}

\end{proposition}
\begin{proposition}[מציאת הפוטנציאל הכימי]
נעשה בשלבים הבאים:

  \begin{enumerate}
    \item כותבים את \(\mathcal{Z}\) בעזרת \(T,V,\mu\) ורמות האנרגיה. 


    \item נשתמש ביחס \(N=k_{B}T \frac{\partial \ln \mathcal{Z}}{\partial \mu}\). 


    \item נפתור משוואה עבור \(\mu\). 


  \end{enumerate}
\end{proposition}
\begin{example}[מציאת פוטנציאל הכימי]
פונקציית הגראנד חלוקה של גז אידיאלי תהיה:
$${\mathcal{Z}}=\sum_{N=0}^{\infty}{\frac{1}{N!}}\left({\frac{V}{\lambda^{3}}}e^{\beta\mu}\right)^{N}=\exp\left( \frac{V}{\lambda^{3}}e^{ \beta \mu }\right)$$
כאשר כעת ניתן לגזור ולהשתמש ביחס \(\langle N\rangle=\frac{1}{\beta}\frac{\partial\ln{\mathcal{Z}}}{\partial\mu}\) ונקבל:
$$\langle N\rangle=e^{\beta\mu}\frac{V}{\lambda^{3}}$$
וכעת בהנחה ש-\(N\) נתון ניתן לבודד עבור \(\mu\) ולקבל:
$$\mu=k_{B}T\ln\left(\frac{\langle N\rangle\lambda^{3}}{V}\right)$$

\end{example}
\begin{summary}
  \begin{itemize}
    \item הצבר הגאנד קנוני מתאר מערכת שיכול להחליף חלקיקים וגם אנרגיה עם אמבט.
    \item הגראנד פואנציאל הוא ביטוי לאנרגיה במשתנים הטבעיים של הצבר, ומוגדר על ידי:
$$\Phi[T,\mu]=U-T S-\mu N$$
    \item פונקציית הגראנד חלוקה מוגדרת על ידי:
$${\mathcal{Z}}=\sum_{N=0}^{\infty}\int_{0}^{\infty}d E\,\Omega(E,V,N)\exp\left[-\beta E+\beta\mu N\right]$$
    \item הקשר בין הפונקציית הגאנד חלוקה לגראנד פוטנציאל הוא:
$${\mathcal{Z}}=\exp(-\beta \Phi[T,\mu])=\exp(\beta P V)$$
כאשר השיוויון האחרון נכון רק כאשר המערכת אקסטנסיבית.
    \item מהגראנד פוטנציאל ניתן להשיג את הגדלים הבאים:
\begin{gather*}N=k_{\mathrm{B}}T\left({\frac{\partial\mathrm{ln}\;{\mathcal{Z}}}{\partial\mu}}\right)_{\beta}\quad U=-\left(\frac{\partial\mathrm{ln}\;\mathcal{Z}}{\partial\beta}\right)_{\mu}+\mu N\quad S={\frac{U-\mu N+k_{\mathrm{B}}T\ln{\mathcal{Z}}}{T}} 
\end{gather*}
  \end{itemize}
\end{summary}
\section{משוואת סאהא}

\begin{definition}[יינון]
תהליך שבו אטום או מולקולה מאבדים אלקטרונים, וגורם להתווצרות של יונים. ניתן לכתוב תהליך זה בצורה הבאה:
$$\mathrm{\mathrm{Atom}}\rightleftharpoons\mathrm{\mathrm{Ion}}^{+}+e^{-}$$

\end{definition}
\begin{definition}[צפיפות הפרוטונים]
כמות הפרוטונים ליחידת נפח אשר מיוניים(ללא אלקטרונים). מסומן ב-\(n_{p}\).

\end{definition}
\begin{definition}[צפיפות האלקטרונים]
כמות האלקטרונים ליחידת נפח אשר אינם קשורים לכלום. מסומן ב-\(n_{e}\).

\end{definition}
\begin{definition}[צפיפות האטומים]
צפיפות הפרטונים ליחידת נפח אשר אינם מיוננים(כלומר מכילים אלקטרון). מוסמן ב-\(n_{H}\).

\end{definition}
\begin{definition}[צפיפות הכוללת של גרעיני המימן]
סך גרעיני מימן, כלומר כולל גם את הפרוטונים המיוניים וגם הלא מיונים. מסומן ב-\(n_{0}\).

\end{definition}
\begin{corollary}
נובע מידית מההגדרות כי \(n_{p}+n_{H}=n_{0}\), ו-\(n_{p}=n_{e}\).

\end{corollary}
\begin{proposition}
קיים פונקציה \(x=x(n_{0},T)\) הנקרא השבר היינון אשר מקיים:
$$n_{p}=n_{e}=x n_{0}\;\;\;\;n_{H}=(1-x)\,n_{0}$$

\end{proposition}
\begin{symbolize}
נסמן את הפוטנציאלים הכימים המתאימים על ידי \(\mu_{p},\mu_{H},\mu_{e}\).

\end{symbolize}
\begin{reminder}
כאשר אנחנו בשיווי משקל אנחנו עם טמפרטורה קבועה ועבור תהליך כמו זה שבו הנפח קבוע נקבל כי בשיווי משקל האנרגיה החופשית של הלמהולדס\((F)\) תהיה אקסטרימלית. 

\end{reminder}
\begin{proposition}
הפוטנציאל הכימי מקיים:
$$\sum_{j}\mu_{j}d N_{j}=0\qquad \mu_{j}=\frac{\partial F}{\partial N_{j}}=-\frac{1}{\beta}\frac{\partial \ln Z}{\partial N_{j}}$$

\end{proposition}
\begin{proof}
האנרגיה החופשית של הלמהולדס נתונה על ידי:
$$d F=-S d T-P d V+\sum_{j\in\{ p,e,H \}}\mu_{j}d N_{j}$$
ולכן מצד אחד מתקיים:
$$\mu_{j}=\frac{\partial F}{\partial N_{j}}=-\frac{1}{\beta}\frac{\partial \ln Z}{\partial N_{j}}$$
וכן בנוסף כיוון שהנפח והטמפרטורה קבועות נקבל כי \(\mathrm{d}F=0\) וכן \(\mathrm{d}T=\mathrm{d}V=0\) ונקבל:
$$\sum_{j\in\{ p,e,H \}}\mu_{j}\mathrm{d}N_{j}=0$$

\end{proof}
\begin{proposition}
מהקשר \(H\rightleftarrows p+e\) נקבל כי שינוי המספר חלקיקים מקיים:
$$d N_{p}=d N_{e}=-d N_{H}$$

\end{proposition}
\begin{corollary}
משלוב שתי הטענות הקודמות נקבל:
$$\mu_{p}d N_{p}+\mu_{e}d N_{e}+\mu_{H}d N_{H}=\mu_{p}d N+\mu_{e}d N-\mu_{H}d N=0\implies \mu_{p}+\mu_{e}=\mu_{H}$$

\end{corollary}
\begin{lemma}
פונקציית החלוקה של המערכת שלנו תהיה בקירוב של הרבה חלקיקים:
$$Z\approx\sum_{j}\left(N_{j}l n\left(z_{j}\right)-N_{j}l n N_{j}+N_{j}\right)$$
כאשר \(z_{j}\) היא פונקציה חלוקה חד חלקיקית של כל חלקיק.

\end{lemma}
\begin{proof}
כיוון שיש לנו חלקיקים מסוגים שונים המיקרומצבים של בחלקיקים בלתי תלויים, ולכן הפונקציית החלוקה הכוללת תהיה המכפלה של הפונקציות חלוקה של כל אוכלוסיה:
$$Z=Z_{p}\cdot Z_{e}\cdot Z_{H}$$
כאשר ניתן לפרק את למכפלת פונקציות החלוקה החד חלקיקיות של כל חלקיק \(Z=\prod_{j} \frac{z_{j}^{N_{j}}}{N_{j}!}\). ניקח את הלוגוריתם ונקבל:
$$\ln Z=\sum_{j}\ln\left(\frac{z_{j}^{N_{j}}}{N_{j}!}\right)=\sum_{j}\left(N_{j}\ln\left(z_{j}\right)-\ln\left(N_{j}!\right)\right)\approx\sum_{j}\left(N_{j}\ln\left(z_{j}\right)-N_{j}\ln N_{j}+N_{j}\right)$$
כאשר השתמשנו בקירוב סטרלינג.

\end{proof}
\begin{corollary}
ניתן לקבל את הפוטנציאל הכימי של כל אוכלוסיה על ידי:
$$\mu_{j}=-\frac{1}{\beta}\ln \frac{z_{j}}{N_{j}}$$

\end{corollary}
\begin{proof}
$$\mu_{j}=-\frac{1}{\beta}\frac{\partial}{\partial N_{j}}\sum_{i}\left(N_{i}\ln\left(z_{i}\right)-N_{i}\ln N_{i}+N_{i}\right)=-\frac{1}{\beta}\left(l n\left(z_{j}\right)-\ln N_{j}-N_{j}\cdot\frac{1}{N_{j}}+1\right)=-\frac{1}{\beta}\ln\frac{z_{j}}{N_{j}}$$

\end{proof}
\begin{proposition}
עבור כל חלקיק ב-\(H\)(גרעינים מיוניים - עם אלקטרון) ניתן לפצל את האנרגיה של כל חלקיק לאנרגיה של הגרעין ושל האלקטרון:
$$\mathcal{E} _{H}=\frac{p^{2}_{\text{nucleus}}}{2m_{\text{nucleus}}}+\frac{p^{2}_{e}}{2m_{e}}+\mathcal{E} _{0H}$$
כאשר \(\mathcal{E}_{0H}\) היא אנרגיית הקשר. אך כיוון שמסת האלקטרון קטנה מפאקטור של 4 סדרי גודל נזניח את הגורם של האלקטרון.

\end{proposition}
\begin{corollary}
פונקציית החלוקה החד חלקיקית נתונה על ידי:
$$g_{s j}V\left(\frac{m_{j}k_{B}T}{2\pi\hbar^{2}}\right)^{\frac{3}{2}}e^{-\beta\varepsilon_{0j}}$$
כאשר:
$$\varepsilon_{0,j}=\begin{cases}0&j=p,e\\ -E_{0}&j=H\end{cases};\ g_{s j}=\begin{cases}2&j=p,e\\ 4&j=H\end{cases}$$

\end{corollary}
\begin{proof}
$$z_{j}=\frac{g_{s j}}{h^{3}}\int d^{3}r d^{3}p e^{-\beta\left(\frac{p^{2}}{2m_{j}}+\varepsilon_{0j}\right)}$$

\end{proof}
\begin{corollary}
הפוטנציאל הכימי של כל אוכלוסיה נתונה על ידי:
$$\mu_{j}=-k_{B}T l n\left(\frac{g_{s j}}{n_{j}}\left(\frac{m_{j}k_{B}T}{2\pi\hbar^{2}}\right)^{\frac{3}{2}}\right)+\varepsilon_{0j}$$

\end{corollary}
\begin{proof}
נובע ישירות מ-\(\mu_{j}=-\frac{1}{\beta}\ln \frac{z_{j}}{N_{j}}\) והפונקציית חלוקה שמצאנו.

\end{proof}
\begin{proposition}[משוואת סאהא]
$$\frac{x^{2}}{1-x}=\frac{1}{n_{0}\lambda_{T}^{3}}e^{-\frac{E_{0}}{k_{B}T}}\qquad \lambda_{T}\equiv\left(\frac{2\pi\hbar^{2}}{m_{e}k_{B}T}\right)^{\frac{1}{2}}$$

\end{proposition}
\begin{proof}
מהמשוואה \(\mu_{p}+\mu_{e}=\mu_{H}\) נקבל:
$$-k_{B}T \ln\left(\frac{2}{n_{p}}\left(\frac{m_{p}k_{B}T}{2\pi\hbar^{2}}\right)^{\frac{3}{2}}\right)-k_{B}T \ln\left(\frac{2}{n_{e}}\left(\frac{m_{e}k_{B}T}{2\pi\hbar^{2}}\right)^{\frac{3}{2}}\right)=-k_{B}T \ln\left(\frac{4}{n_{H}}\left(\frac{m_{p}k_{B}T}{2\pi\hbar^{2}}\right)^{\frac{3}{2}}\right)-E_{0}$$
כאשר כיוון שהנחנו \(m_{H}\approx m_{p}\) נחלק ב-\(-k_{B}T\) ונחבר את הלוגריתמים:
$$\ln\left(\frac{2}{n_{p}}\left(\frac{m_{p}k_{B}T}{2\pi\hbar^{2}}\right)^{\frac{3}{2}}\cdot\frac{2}{n_{e}}\left(\frac{m_{e}k_{B}T}{2\pi\hbar^{2}}\right)^{\frac{3}{2}}\right)=\ln\left(\frac{4}{n_{H}}\left(\frac{m_{p}k_{B}T}{2\pi\hbar^{2}}\right)^{\frac{3}{2}}\right)+\frac{E_{0}}{k_{B}T}$$
ניקח אקספוננט לשתי האגפים:
$${\frac{4}{n_{p}n_{e}}}{\bigg(}{\frac{m_{p}k_{B}T}{2\pi\hbar^{2}}}{\bigg)}^{\frac{3}{2}}\left({\frac{m_{e}k_{B}T}{2\pi\hbar^{2}}}\right)^{\frac{3}{2}}={\frac{4}{n_{H}}}{\bigg(}{\frac{m_{p}k_{B}{T}}{2\pi\hbar^{2}}}{\bigg)}^{\frac{3}{2}}\cdot e^{\frac{E_{0}}{k_{B}T}}$$
ולכן:
$$\frac{n_{p}n_{e}}{n_{H}}=\left(\frac{m_{e}k_{B}T}{2\pi\hbar^{2}}\right)^{\frac{3}{2}}e^{-\frac{E_{0}}{k_{B}T}}$$
מהיחסים \(n_{p}=n_{e}=x n_{0},n_{H}=(1-x)\,n_{0}\) נקבל:
$$\frac{x^{2}}{1-x}=\frac{1}{n_{0}}\left(\frac{m_{e}k_{B}T}{2\pi\hbar^{2}}\right)^{\frac{3}{2}}e^{-\frac{E_{0}}{k_{B}T}}$$
כאשר ניתן כעת לזהות את האורך גל התרמי \(\lambda_{T}\) ולקבל את המשוואה.

\end{proof}
\begin{corollary}
עבור המרחק הממוצע בין החלקיקים \(a\) כאשר \(n_{0}=\frac{1}{a^{3}}\) ולקבל:
$$\frac{x^{2}}{1-x}=\left(\frac{a}{\lambda_{T}}\right)^{3}e^{-\frac{E_{0}}{k_{B}T}}$$

\end{corollary}
\begin{example}
נתון גז אידיאלי המורכב מאטומים שיכולים להיות באחד משני מצבים - ניטרלי או מיונן פעם אחת. נרצה למצוא את \(n_{e},n_{p},n_{H}\).
התהליך הנתון הוא \(\mathrm{\mathrm{A}}\rightleftharpoons\mathrm{\mathrm{A}}^{+}+e^{-}\) כמו שראינו. ראשית נשים לב כי בשיווי משקל מתקיים:
$$\sum_{i}\frac{\partial F}{\partial N_{i}}=0\implies \mu^{A^{+}}+\mu^{e^{-}}-\mu^{A}=0$$
ראינו כי הפוטנציאל הכימי של גז אידיאלי נתון על ידי \(\mu=-k_{B}T\left( \frac{V}{N_{A}\lambda_{A}^{3}} \right)\) ולכן:
$$\mu_{A}=-k_{B}T\ln\left( \frac{V}{N_{A}\lambda_{A}^{3}} \right)\qquad \mu_{e^{-}}=-k_{B}T\ln\left( \frac{V}{N_{A}\lambda_{e}^{3}} \right)$$

2016 מועד ב שאלה 1 סעיף ג להשלים. 

\end{example}
\begin{summary}
שלבים למציאת סאהא

  \begin{enumerate}
    \item היחס \(x=x(n_{0},T)\) יקיים: 
$$n_{e}=n_{p}=x\cdot n_{0}\qquad n_{p}+n_{H}=n_{0}=(1-x)n_{0}$$


    \item כותבים את \(\mu_{j}=-\frac{1}{\beta}\frac{\partial \ln Z}{\partial N_{j}}\) כאשר מדרישת שיווי משקל נקבל \(\sum_{j}\mu_{j}\mathrm{d}N_{j}=0\) כאשר \(j \in \{ e,p,H \}\).  


    \item מתקיים \(\mathrm{d}N_{p}=\mathrm{d}N_{e}=-\mathrm{d}N_{H}\). נציב בשלב 2 ונקבל \(\mu_{p}+\mu_{e}=\mu_{H}\). 


    \item נרשום את פונקציית החלוקה של כל המערכת: 
$$Z\approx\sum_{j}\left(N_{j}l n\left(z_{j}\right)-N_{j}l n N_{j}+N_{j}\right)\implies \mu_{j}=-\frac{1}{\beta}\ln \frac{z_{j}}{N_{j}}$$
כאשר \(z_{j}\) היא פונקציית החלוקה עבור כל אוכלוסייה.


    \item מציבים במשוואה \(\mu_{p}+\mu_{e}=\mu_{H}\) כאשר זוכרים כי יש ל-\(z_{H}\) יש ריבוי ספין 4 ולא 2 כמו \(z_{e},z_{H}\) כאשר נציב את היחסים משלב 1. 


  \end{enumerate}
\end{summary}
\section{צבר גיבס}

\begin{definition}[צבר גיבס]
מקבעים את \(T\), ניתן להחליף את האנרגיה \(E\) עם האמבט. כאשר:
$$E^{\text{tot}}=E^{\text{bath}}+E^{a}$$
בנוסף מקבעים את \(P\) כך שניתן להחליף את הנפח:
$$V^{\text{tot}}=V^{\text{bath}}+V^{a}$$

\end{definition}
\begin{proposition}
המיקרו מצב של \(a\) זה מאקרו מצב של \(a\) עם האמבט. אם נזכור את ההגדרה של האנטלפיה \(\overline{H}\) אזי ניתן לכתוב:
$$\Omega^{a+\text{bath}}[\text{macro of }a +\text{bath}]=\Omega_{*}e^{ -\beta \overline{H}  }$$

\end{proposition}
\begin{definition}[פונקציית החלוקה של גיבס]
$$Z_{G}=\sum_{\text{micro}} e^{ -\beta \overline{H} (\text{micro of }a) }$$

\end{definition}
\begin{proposition}
עבור גז ללא אינטראקציה נקבל:
$$Z_{G}=\int  \;\mathrm{d}V e^{ -\beta PV }Z_{F}(T,V,N)$$

\end{proposition}
כאשר קיבלנו שזה פשוט התמרת לפלס של הפונקציית החלוקה הקנונית:

\includegraphics[width=0.8\textwidth]{diagrams/svg_11.svg}
\begin{example}[ספינים ללא אינטראקציה בשדה מגנטי חיצוני \(B\)]
לכל מערכת יש ספין \(s_{i}\in \{ -1,1 \}\). לספינים אין אינטראקציה אחד עם השני ולכן \(H_{\text{internal}}=0\). במקרה שלנו ה-\(B\) הוא המקביל ל-\(-P\)(כלומר ניתן למפות \(B\leftrightarrow -P\)) וה-\(M=\sum_{i=1}^{n}s_{i}\) הוא המקביל ל-\(V\) (כלומר ניתן למפות \(M\leftrightarrow V\)) ולכן \(-BM\leftrightarrow PV\) וכן המגנטיזציה תהיה \(M=\sum_{i=1}^{N}s_{i}\). כעת:
$$-BM(\text{micro})\leftrightarrow  PV(\text{micro})$$
כלומר יש כוח מוכלל של \(-B\sum_{i=1}^{N}s_{i}\) ופונקציית החלוקה תהיה:
$$Z_{G}\left( \beta, B \right)=\sum_{s_{1}\in\{-1,1\}}\sum_{s_{2}\in\{-1,1\}}\cdot\cdot\cdot\sum_{s_{N}\in\{-1,1\}}e^{\beta B\sum_{i}s_{i}}$$
כיוון שללא אינטראקציה ניתן לפשט:
$$Z_{G}=\left(\sum_{s\in\{-1,1\}}e^{\beta B s}\right)^{N}=\left(2\cosh\left(\beta B\right)\right)^{N}$$
כך שנקבל:
$$G\left(\beta,B\right)=-\frac{N}{\beta}\log\left(2\cosh\left(\beta B\right)\right)$$
ומהמיפוי שלנו נקבל:
$$d G=-S d T+V d P\longrightarrow d G=-S d T-M d B$$
כך שנקבל:
$$M=-\frac{\partial G\left(\beta,B\right)}{\partial B}=N\operatorname{tanh}\left(\beta B\right)\qquad \chi=\frac{\partial M\left(\beta,B\right)}{\partial B}=\frac{N\beta}{\cosh^{2}\left(\beta B\right)}$$
ולכן:
$$\chi\mid_{B=0}=N\beta=\frac{N}{k_{B}T}$$

\end{example}
\chapter{פיזיקה סטטיטית קוונטית}

\section{חילוף חלקיקים}

\begin{definition}[אופרטור החילוף]
עבור שתי חלקיקים זההים אשר נמצאים ב-\(\vec{r}_{1}\) ו-\(\vec{r}_{2}\) בהתאמה ניתן לתאר על ידי פונקציית גל מהצורה \(\psi\left( \vec{r}_{1},\vec{r}_{2} \right)\). אופרטור החילוף יהיה אופרטור \(P\) המקיים:
$$P_{12}\psi\left( \vec{r}_{1},\vec{r}_{2} \right)=\psi\left( \vec{r}_{2},\vec{r}_{1} 
\right)$$

\end{definition}
\begin{proposition}
אופרטורים זההים אינווריאנטים לאופרטור החילוף, כלומר:
$$[H,P_{12}]=0$$

\end{proposition}
\begin{corollary}
כיוון שזו פעולת סימטריה מתקיים:
$$\left\lvert  P_{12}\psi\left( \vec{r}_{1},\vec{r}_{2} \right)  \right\rvert^{2} =\left\lvert  \psi\left( \vec{r}_{1},\vec{r}_{2} \right)  \right\rvert ^{2}\implies\left\lvert  \psi\left( \vec{r}_{2},\vec{r}_{1} \right)  \right\rvert^{2} =\left\lvert  \psi\left( \vec{r}_{1},\vec{r}_{2} \right)  \right\rvert ^{2}$$
כלומר יש שתי אפשרויות:

  \begin{enumerate}
    \item פונקציית הגל היא סימטרית תחת החלפה: 
$$\psi(\mathbf{r}_{2},\mathbf{r}_{1})=\psi(\mathbf{r}_{1},\mathbf{r}_{2}),$$


    \item פונקציית הגל אנטי סימטרית תחת החלפה: 
$$\psi(\mathbf{r}_{2},\mathbf{r}_{1})=-\psi(\mathbf{r}_{1},\mathbf{r}_{2})$$


  \end{enumerate}
\end{corollary}
\begin{remark}
הפיצול לשתי מקרים רלוונטי רק ב-3 מימדים. בשתי מימדים ניתן לקבל גם פאזה:
$$\psi(\,\mathbf{r}_{2},\mathbf{r}_{1}\,)\;=\;\mathrm{e}^{\mathrm{i}\theta}\psi(\,\mathbf{r}_{1},\mathbf{r}_{2}\,)$$

\end{remark}
\begin{proposition}
עבור מערכת עם שתי חלקיקים ניתן לתאר אותם במצב מכפלה. נחלק אותם ל-4 אפשרויות:

  \begin{enumerate}
    \item חלקיקים מובחנים - יהיו 4 אפשרויות: 
$$|0\rangle|0\rangle,\qquad|1\rangle|0\rangle,\qquad|0\rangle|1\rangle,\qquad|1\rangle|1\rangle.$$


    \item חלקיקים לא מובחנים - יהיו 3 אפשרויות: 
$$|0\rangle|0\rangle,\qquad|1\rangle|0\rangle,\qquad|1\rangle|1\rangle.$$


    \item בוזונים - לא מובחנים - יהיו 3 אפשרויות: 
$$\ket{0} \ket{0} \quad \ket{1} \ket{1} \quad \frac{1}{\sqrt{2}}\left(|1\rangle|0\rangle+|0\rangle|1\rangle\right),$$


    \item פרמיונים - לא מובחנים - יהיה רק אפשרות אחת: 
$$\frac{1}{\sqrt{2}}\left(|1\rangle|0\rangle-|0\rangle|1\rangle\right).$$
כאשר עבור הפרמיונים והבוזונים המצב המעורב מתקבל כיוון שנדרש שיהיה ערך עצמי של אופרטור החילוף.


  \end{enumerate}
\end{proposition}
\begin{proposition}[עקרון האיסור של פאולי]
לא ייתכן ויהיו שתי פרמיונים באותו מצב קוונטי

\end{proposition}
\begin{proof}
נניח בשלילה ש-\(\psi=\ket{\varphi}\ket{\varphi}\) הוא מערכת פרמיונית של שתי מצבים באותו מצב קוונטי. תחת אופרטור החילוף נקבל:
$$\hat{P}_{12}|\varphi\rangle|\varphi\rangle=|\varphi\rangle|\varphi\rangle=-|\varphi\rangle|\varphi\rangle,$$
כאשר השתמשנו באנטי סימטריה של פרמיונים. לכן מתקיים:
$$|\varphi\rangle|\varphi\rangle=0$$
כלומר לא ייתכן כי יש שתי מצבים קוונטים זההים. 

\end{proof}
\section{פרמיונים ובוזונים}

\begin{reminder}[פונקציית הגראנד חלוקה]
עבור מצב עם אנרגיה נתונה \(E\), הפונקציית הגראנד חלוקה יהיה הסכום על כל קונפיגורציות:
$${\mathcal Z}=\sum_{n}\mathrm{e}^{n\beta(\mu-E)}$$
כאשר כל קונפיגורציה נבדלת למעשה רק על ידי מספר החלקיקים, ולכן נדרש לסכום רק על מספר החלקיקים.

\end{reminder}
\begin{proposition}
מספר הממוצע של חלקיקים במצב יהיה:
$$\langle n\rangle=\frac{\sum_{n}n\mathrm{e}^{n\beta(\mu-E)}}{\sum_{n}\mathrm{e}^{n\beta(\mu-E)}}=-\frac{1}{\beta\mathcal{Z}}\frac{\partial\mathcal{Z}}{\partial E}=-\frac{1}{\beta}\frac{\partial\ln\mathcal{Z}}{\partial E}$$

\end{proposition}
\begin{corollary}
עבור פרמיונים יש רק שתי אפשרויות - \(n=1\) או \(n=0\) ולכן נדרש לסכום רק עליהם ולקבל:
$${\mathcal Z}=\sum_{n=0}^{1}\mathrm{e}^{n\beta(\mu-E)}=1+\mathrm{e}^{\beta(\mu-E)},$$

\end{corollary}
\begin{corollary}
$$\ln{\mathcal{Z}}=\ln(1+\mathrm{e}^{\beta(\mu-E)})$$

\end{corollary}
\begin{proposition}
עבור בוזונים נקבל:
$${\mathcal{Z}}=\sum_{n=0}^{\infty}\mathrm{e}^{n\beta(\mu-E)}={\frac{1}{1-\mathrm{e}^{\beta(\mu-E)}}}$$
ולכן:
$$\ln{\mathcal{Z}}=-\ln(1-\mathrm{e}^{\beta(\mu-E)})$$

\end{proposition}
\begin{proposition}
מספר החלקיקים הממוצע יהיה עבור פרמיונים:
$$\langle n\rangle=\frac{1}{\mathrm{e}^{\beta(E-\mu)}+1},$$
כאשר עבור בוזונים:
$$\langle n\rangle=\frac{1}{\mathrm{e}^{\beta(E-\mu)}-1},$$

\end{proposition}
\begin{proposition}
עבור מערכת כללית שבה מכניסים \(n_{i}\) חלקיקים לרמה \(i\) עם האנרגיה \(E_{i}\) מתקיים:
$${\mathcal Z}=\prod_{i}\sum_{\{n_{i}\}}\mathrm{e}^{n_{i}\beta(\mu-E_{i})}$$

\end{proposition}
\begin{proof}
ניתן לכתוב קונפיגורציה כללית על ידי:
$$\left[\mathrm{e}^{\beta(\mu-E_{1})}\right]^{n_{1}}\times\left[\mathrm{e}^{\beta(\mu-E_{2})}\right]^{n_{2}}\times\cdots=\prod_{i}\mathrm{e}^{n_{i}\beta(\mu-E_{i})}$$
ולכן פונקציית הגראנד חלוקה תהיה:
$${\mathcal{Z}}=\sum_{\left\{n_{i}\right\}}\prod_{i}\mathrm{e}^{n_{i}\beta(\mu-E_{i})}$$
כאשר ניתן לקחת גורם משותף ולהפוך את סדר הסכימה וכפל:
\begin{gather*}\sum_{\left\{ n_{\epsilon} \right\}}\prod_{\epsilon}\exp\left[ -\beta\left( \epsilon-\mu \right)n_{\epsilon} \right],=\sum_{n_{\epsilon_{1}}}\sum_{n_{\epsilon_{2}}}\cdots e^{-\beta\left( \epsilon_{1}-\mu \right)n_{\epsilon_{1}}}e^{-\beta\left( \epsilon_{2}-\mu \right)n_{\epsilon_{2}}}\ldots=\\=\sum_{n_{\epsilon_{1}}}e^{-\beta\left( \epsilon_{1}-\mu \right)n_{\epsilon_{1}}}\sum_{n_{\epsilon_{2}}}e^{-\beta\left( \epsilon_{2}-\mu \right)n_{\epsilon_{2}}}\ldots =\\=\prod_{\epsilon}\sum_{n_{\epsilon}}\exp\left[ -\beta\left( \epsilon-\mu \right)n_{\epsilon} \right] 
\end{gather*}

\end{proof}
\begin{remark}
הפיתוח הכללי הזה נותן את הפונקציית החלוקה של פרמיונים כאשר מתאפשר לכל היותר רמה אחת של אנרגיה ושל בוזונים כמות אינסופית של רמות.

\end{remark}
\begin{definition}[פונקציית התפלגות פרמי דיראק]
פונקציה:
$$f(E)=\frac{1}{\mathrm{e}^{\beta(E-\mu)}+1}$$

\end{definition}
\begin{definition}[פונקציית התפלגות בוז-אינשטיין]
$$f(E)=\frac{1}{\mathrm{e}^{\beta(E-\mu)}-1}$$

\end{definition}
\begin{remark}
פונקציות החלוקות מוגדרות כך שמקיימות \(f(E)=\langle n \rangle\). כמו כן נשים לב כי בגבול \(\beta\left( E-\mu \right)\gg 1\) נקבל את התפלגות בולצמן, דבר המתאים לעקרון ההתאמה האומר כי בגבול הקלאסי נצפה כי יתנהג בצורה קלאסית.

\end{remark}
\begin{remark}
נשים לב כי עבור \(\mu=E\) נקבל כי ההתפלגות בוזונית מתבדרת, זאת כיוון שבמצב זה הרמה הנמוכה ביותר תהיה מאוכלסת עם כמות אינסופית של חלקיקים - פתרון לא פיזיקלי, ולכן נדרש \(\mu< E\).

\end{remark}
\begin{summary}
  \begin{itemize}
    \item פונקציית הגראנד חלוקה של פרמיונים ובוזונים תקיים:
$$\ln{\mathcal{Z}}=\pm\ln(1\pm\mathrm{e}^{\beta(\mu-E)}),$$
כאשר ה-\(+\) זה עבור פרמיונים ו-\(-\) עבור בוזונים.
    \item מספר החלקיקים הממוצע יהיה:
$$\langle n_{E}\rangle=\frac{1}{\mathrm{e}^{\beta(E-\mu)}\pm1},$$
כאשר ה-\(+\) זה עבור פרמיונים ו-\(-\) עבור בוזונים.
    \item בוזונים מקיימים סטטיסטיקת בוז-אינשטיין, הנתונה על ידי:
$$f_{BE}(E)=\frac{1}{\mathrm{e}^{\beta(E-\mu)}-1}$$
    \item פרמיונים מקיימים סטטיסטיקת פרמי-דיראק, הנתונה על ידי:
$$f_{FD}(E)=\frac{1}{\mathrm{e}^{\beta(E-\mu)}+1}$$
  \end{itemize}
\end{summary}
\section{גז קוונטי}

\begin{proposition}
פונקציית הגראנד חלוקה הכוללת של מערכת תהיה המכפלה של פונקציות החלוקות של כל חלקיק:
$${\mathcal{Z}}_{t o t}=\prod_{i}{\mathcal{Z}}_{i}$$

\end{proposition}
\begin{example}
עבור מערכת עם \(k\) חלקיקים עם ספין \(S\) יש \(2S+1\) מצבים אפשריים ולכן:
$$\mathcal{Z}=\prod_{k}\mathcal{Z}_{k}^{2S+1}$$
כאשר:
$${\mathcal{Z}}_{\mathbf{k}}=\left(1\pm\mathrm{e}^{-\beta(E_{\mathbf{k}}-\mu)}\right)\pm1$$
כאשר \(+\) זה עבור פרמיונים ו-\(-\) זה עבור בוזונים.

\end{example}
\begin{proposition}
פונקציית הגראנד פוטנציאל בגבול הרצף(כלומר \(n\gg 1\)) יקיים:
$$\Phi_{G}= \mp k_{\mathrm{B}}T\int_{0}^{\infty}\ln\left( 1\pm\mathrm{e}^{-\beta\left( E-\mu \right)} \right)\,g(E)\,\mathrm{d}E $$
כאשר \(+\) עבור פרמיונים ו-\(-\) עבור בוזונים.

\end{proposition}
\begin{proof}
נזכור כי בגז קוונטי \(\mathcal{Z}=\prod_{n}\mathcal{Z}_{n}\). כעת נקבל:
\begin{gather*}\Phi_{G}=-k_{B}T\ln \mathcal{Z} =-k_{B}T\ln \prod_{n}^{\infty}\mathcal{Z} _{n} =-k_{B}T\sum_{n=0}^{N} \ln \mathcal{Z} _{n}
\end{gather*}
כאשר כיוון שאנו עוברים לגבול הרצף ניתן לכתוב:
$$
\Phi_{G}=\mp k_{\mathrm{B}}T\int_{0}^{\infty}\ln\left( 1\pm\mathrm{e}^{-\beta\left( E-\mu \right)} \right)\,g(E)\,\mathrm{d}E $$

\end{proof}
\begin{proposition}
מספר החלקיקים מקיים:
$$n_{\mathbf{k}}=k_{\mathrm{B}}T{\frac{\partial}{\partial\mu}}\ln{\mathcal{Z}}_{\mathbf{k}}={\frac{1}{\mathrm{e}^{\beta(E_{\mathbf{k}}-\mu)}\pm1}},$$

\end{proposition}
\begin{corollary}
מספר החלקיקים של המערכת נתון על ידי:
$$N=\sum_{k}n_{k}=\int_{0}^{\infty}{\frac{g(E)\,\mathrm{d}E}{\mathrm{e}^{\beta(E-\mu)}\pm1}}$$
כאשר סך האנרגיה נתונה על ידי:
$$U=\sum_{k}n_{k}E_{k}=\int_{0}^{\infty}\frac{E\,g(E)\,\mathrm{d}E}{\mathrm{e}^{\beta(E-\mu)}\pm1}$$

\end{corollary}
\begin{definition}[פוגאסיטי - fugacity]
$$z=\mathrm{e}^{\beta\mu}$$

\end{definition}
\begin{definition}[פולילוגוריתם]
פונקציה המוגדרת על ידי:
$$\operatorname{Li}_{n}(z)=\sum_{k=1}^{\infty}{\frac{z^{k}}{k^{n}}}$$
כאשר \(z\) זה העיגול יחידה הפתוח במישור המרוכב. ההגדרה על המרחב המרוכב כולו נובעת מההמשכה האנליטית.

\end{definition}
\begin{reminder}[פונקציית גאמא]
מוגדר על ידי:
$$\Gamma(z)=\int_{0}^{\infty} t^{z-1}e^{ -t } \, \mathrm{d}t $$
כאשר \(\mathrm{Re}(z)>0\). מקיים \(\Gamma(n)=(n-1)!\).

\end{reminder}
\begin{proposition}
$$\int_{0}^{\infty}\frac{x^{n-1}\,\mathrm{d}x}{z^{-1}\mathrm{e}^{x}\pm1}=\mp\Gamma(n)\mathrm{Li}_{n}(\mp z)$$

\end{proposition}
\begin{proof}
ראשית נשים לב כי:
$${\frac{1}{z^{-1}\mathrm{e}^{x}-1}}={\frac{z\mathrm{e}^{-x}}{1-z\mathrm{e}^{-x}}}=\sum_{m=0}^{\infty}(z\mathrm{e}^{-x})^{m+1}.$$
כאשר ניתן לחשב את האינטגרל הבא:
\begin{gather*}\int_{0}^{\infty}\,\frac{x^{n-1}\,\mathrm{d}x}{z^{-1}\mathrm{e}^{x}-1}=\sum_{m=0}^{\infty}\int_{0}^{\infty}x^{n-1}\left( \left( z\mathrm{e}^{-x} \right)^{m+1} \right)=\\=\sum_{m=0}^{\infty}z^{m+1}\int_{0}^{\infty}x^{n-1}\mathrm{e}^{-(m+1)x}=\sum_{m=0}^{\infty}{\frac{z^{m+1}}{(m+1)^{n}}}\int_{0}^{\infty}y^{n-1}\mathrm{e}^{-y}=\\=\Gamma(n)\sum_{m=0}^{\infty}\frac{z^{m+1}}{(m+1)^{n}}=\Gamma(n)\sum_{k=1}^{\infty}\frac{z^{k}}{k^{n}} =\Gamma(n)\mathrm{Li}_{n}(z) 
\end{gather*}
כאשר ניתן באופן דומה להראות כי:
$$\int_{0}^{\infty}\frac{x^{n-1}\,\mathrm{d}x}{z^{-1}\mathrm{e}^{x}+1}=-\Gamma(n)\mathrm{Li}_{n}(-z)$$
ואם נשלב את שתי המשוואות ניתן להראות באופן כללי:
$$\int_{0}^{\infty}\frac{x^{n-1}\,\mathrm{d}x}{z^{-1}\mathrm{e}^{x}\pm1}=\mp\Gamma(n)\mathrm{Li}_{n}(\mp z)$$

\end{proof}
\begin{remark}
נשים לב כי אם \(|z|\ll 1\) נקבל כי רק האיבר הראשון תורם בטור ולכן:
$$\operatorname{Li}_{n}(z)\approx z.$$
בנוסף לכך עבור \(z=1\) נקבל את פונקציית זטא של רימן:
$$\operatorname{Li}_{n}(1)=\sum_{k=1}^{\infty}{\frac{1}{k^{n}}}=\zeta(n),$$

\end{remark}
\begin{example}[מערכת ספינים בקופסא]
נסתכל על מערכת תלת מימדית של פרמיונים ובוזונים עם ספין \(S\)(כלומר לכל ספין יש ניוון אנרגטי של \(2S+1\)). המצבים במרחב ה-\(k\) מתפלגים באופן אחיד ולכן:
$$g(k)\,\mathrm{d}k={\frac{4\pi k^{2}\,\mathrm{d}k}{(2\pi/L)^{3}}}\times(2S+1)={\frac{(2S+1)V k^{2}\,\mathrm{d}k}{2\pi^{2}}}$$
כאשר אם נניח כי \(V=L^{3}\) ונשתמש בקשר \(E=\hbar^{2}k^{2} / 2m\) ניתן לכתוב:
$$g(E)\,\mathrm{d}E={\frac{(2S+1)V E^{1/2}\,\mathrm{d}E}{(2\pi)^{2}}}\left({\frac{2m}{\hbar^{2}}}\right)^{3/2}$$
נקבל כעת:
$$N=\left[\frac{(2S+1)V}{(2\pi)^{2}}\left(\frac{2m}{\hbar^{2}}\right)^{3/2}\right]\int_{0}^{\infty}\frac{E^{1/2}\,\mathrm{d}E}{z^{-1}\mathrm{e}^{\beta E}\pm1}$$
כאשר ניתן לכתוב בעזרת הפונקציה הפולילוגוריתמית ולקבל:
$$N=\frac{(2S+1)V}{\lambda_{\mathrm{th}}^{3}}[\mp\mathrm{Li}_{3/2}(\mp z)]$$
כאשר באופן דומה עבור האנרגיה נקבל:
$$U=\left[\frac{(2S+1)V}{(2\pi)^{2}}\left(\frac{2m}{\hbar^{2}}\right)^{3/2}\right]\int_{0}^{\infty}\frac{E^{3/2}\,\mathrm{d}E}{z^{-1}\mathrm{e}^{\beta E}\pm1}.$$
כאשר ניתן לכתוב:
$$U={{\frac{3}{2}k_{\mathrm{B}}T\frac{(2S+1)V}{\lambda_{\mathrm{th}}^{3}}[\mp\mathrm{Li}_{5/2}(\mp z)]}}=\frac{3}{2}N k_{\mathrm{B}}T\frac{\mathrm{Li}_{5/2}(\mp z)}{\mathrm{Li}_{3/2}(\mp z)}.$$
כמו כן אפשר לכתוב את הפונקציית הגראנד פוטנציאל ולקבל:
$$\Phi_{\mathrm{G}}=\mp k_{\mathrm{B}}T{\frac{(2S+1)V}{(2\pi)^{2}}}\left({\frac{2m}{\hbar^{2}}}\right)^{3/2}\int_{0}^{\infty}\ln(1\pm\mathrm{e}^{-\beta(E-\mu)})\,E^{1/2}\,\mathrm{d}E,$$
כאשר לאחר אינטגרציה בחלקים:
$$\Phi_{\mathrm{G}}=-\frac{2}{3}\frac{(2S+1)V}{(2\pi)^{2}}\left(\frac{2m}{\hbar^{2}}\right)^{3/2}\int_{0}^{\infty}\frac{E^{3/2}\,\mathrm{d}E}{\mathrm{e}^{\beta(E-\mu)}\pm1}$$
וניתן להשוואות למשוואה עבור \(U\) ולקבל את הקשר \(\Phi_{G}=-\frac{2}{3}U\).

\end{example}
\begin{summary}
  \begin{itemize}
    \item חלקיקים קוונטים הם לא מובחנים ולכן פונקציית החלוקה הכוללת תהיה \({\mathcal{Z}}_{t o t}=\prod_{i}{\mathcal{Z}}_{i}\).
    \item ניתן למצוא את מספר החלקיקים הממוצע והאנרגיה על ידי:
$$N\!=\!\sum_{k}n_{k}\!=\!\int_{0}^{\infty}\!\!\!{g(E)f(E)\;\mathrm{d}E}\quad U\!=\!\sum_{k}n_{k}E_{k}\!=\!\!\int_{0}^{\infty}\!\!Eg(E)f(E)\;\mathrm{d}E$$
    \item הגראנד פוטנציאל יהיה:
$$\Phi_{G}= \mp k_{\mathrm{B}}T\int_{0}^{\infty}\ln\left( 1\pm\mathrm{e}^{-\beta\left( E-\mu \right)} \right)\,g(E)\,\mathrm{d}E $$
כאשר \(+\) עבור פרמיונים ו-\(-\) עבור בוזונים.
    \item עבור גז חופשי בנפח \(V\) נקבל את הביטיים:
$$N=\frac{(2S+1)V}{\lambda_{\mathrm{th}}^{3}}\left[ \mp\mathrm{Li}_{3/2}\left( \mp z \right) \right] \qquad U=\frac{3}{2}N k_{\mathrm{B}}T\frac{\mathrm{Li}_{5/2}\left( \mp z \right)}{\mathrm{Li}_{3/2}\left( \mp z \right)}\qquad \Phi_{G}=-\frac{2}{3}U$$
  \end{itemize}
\end{summary}
\section{גז פרמי}

\begin{definition}[גז פרמי]
גז של פרמיונים.

\end{definition}
\begin{definition}[אנרגיית פרמי]
רמות האנרגיה מתמלאות בגלל עקרון האיסור של פאולי ללא קשר לטמפרטורה. האנרגיה שהפרמיונים מגיעים עליה באפס המוחלט נקראת אנרגיית פרמי. כלומר:
$$E_{F}=\mu(T=0)$$

\end{definition}
\begin{remark}
הגדרה זו הגיונית כיוון ש:
$$\mu(T=0)=\frac{\partial E}{\partial N} \implies \mu(T=0)=E(N)-E(N-1)=E_{F}$$

\end{remark}
\begin{remark}
זהו לא האנרגיה שיש למערכת באפס המוחלט! זו הרמת האנרגיה הגבוהה ביותר שחלקיק יכול להגיע עליה. האנרגיה הממוצעת באפס המחולט תהיה קטנה יותר כיוון שיש חלקיקים ברמות הנמוכות יותר. כדי למצוא את האנרגיה באפס המוחלט כדי להשתמש בהגדרה \(\langle E \rangle=\sum_{k}n_{k}E_{k}\).

\end{remark}
\begin{proposition}
באפס המוחלט מספר המצבים הרמה \(k\) נתון על ידי פונקציית הביסייד:
$$n_{k}=\Theta(E_{F}-E_{k})$$

\end{proposition}
\begin{proof}
באפס המוחלט נקבל כי \(\beta\to \infty\) ולכן מספר המצבים ברמה \(k\) נתונה על ידי:
$$n_{\mathbf{k}}={\frac{1}{\mathrm{e}^{\beta(E_{\mathbf{k}}-\mu)}+1}}=\Theta(\mu-E_{k})=\Theta(E_{\mathrm{F}}-E_{k})$$

\end{proof}
\begin{corollary}
באפס המולט מספר המצבים יהיה:
$$N=\int_{0}^{k_{\mathrm{F}}}g({\boldsymbol{k}})\,\mathrm{d}^{3}k,$$
כאשר \(k\) זה ווקטור הגל של פרמי(Fermi wave vector) ומוגדר על ידי:
$$E_{\mathrm{F}}={\frac{\hbar^{2}k_{\mathrm{F}}^{2}}{2m}}$$

\end{corollary}
\begin{proposition}[קירוב סומרפלד]
כאשר \(\beta\gg 1\) מתקיים:
$$\int_{-\infty}^{\infty} H\left( \varepsilon \right)f_{FD}\left( \varepsilon \right) \, d\varepsilon = \int_{-\infty}^{\infty}{\frac{H\left( \varepsilon \right)}{e^{\beta\left( \varepsilon-\mu \right)}+1}}\,\mathrm{d}\varepsilon=\int_{-\infty}^{\mu}H\left( \varepsilon \right)\,\mathrm{d}\varepsilon+{\frac{\pi^{2}}{6}}\left({\frac{1}{\beta}}\right)^{2}H^{\prime}\left( \mu \right)+O\left({\frac{1}{\beta\mu}}\right)^{4}$$

\end{proposition}
\begin{proof}
נתחיל מלבצע החלפת משתנים \(\tau x=\varepsilon-\mu\):
$$I=\int_{-\infty}^{\infty}{\frac{H(\varepsilon)}{e^{\beta(\varepsilon-\mu)}+1}}\,\mathrm{d}\varepsilon=\tau\int_{-\infty}^{\infty}{\frac{H(\mu+\tau x)}{e^{x}+1}}\,\mathrm{d}x$$
נחלק את תחום האינטגרציה \(I=I_{1}+I_{2}\) באופן הבא:
$$I=\underbrace{\tau\int_{-\infty}^{0}{\frac{H(\mu+\tau x)}{e^{x}+1}}\,\mathrm{d}x}_{I_{1}}+\underbrace{\tau\int_{0}^{\infty}{\frac{H(\mu+\tau x)}{e^{x}+1}}\,\mathrm{d}x}_{I_{2}}$$
נכתוב את \(I_{1}\) בעזרת החלפת משתנה \(x\mapsto-x\):
$$I_{1}=\tau\int_{-\infty}^{0}{\frac{H(\mu+\tau x)}{e^{x}+1}}\,\mathrm{d}x=\tau\int_{0}^{\infty}{\frac{H(\mu-\tau x)}{e^{-x}+1}}\,\mathrm{d}x$$
כאשר כיוון ש:
$${\frac{1}{e^{-x}+1}}=1-{\frac{1}{e^{x}+1}}$$
נקבל:
$$I_{1}=\tau\int_{0}^{\infty}H(\mu-\tau x)\,\mathrm{d}x-\tau\int_{0}^{\infty}{\frac{H(\mu-\tau x)}{e^{x}+1}}\,\mathrm{d}x$$
נחזור למשתנים המקוריים על \(-\tau \mathrm{d}x=\mathrm{d}\varepsilon\) בגורם הראשון של \(I_{1}\) ונשלב את \(I=I_{1}+I_{2}\) כך שנקבל:
$$I=\int_{-\infty}^{\mu}H(\varepsilon)\,\mathrm{d}\varepsilon+\tau\int_{0}^{\infty}{\frac{H(\mu+\tau x)-H(\mu-\tau x)}{e^{x}+1}}\,\mathrm{d}x$$
כעת ניתן לקרב את המונה בעזרת הנגזרת בהנחה ש-\(\tau\) מספיק קטן:
$$\Delta H=H(\mu+\tau x)-H(\mu-\tau x)\approx2\tau x H^{\prime}(\mu)$$
ונקבל:
$$I=\int_{-\infty}^{\mu}H(\varepsilon)\,\mathrm{d}\varepsilon+2\tau^{2}H^{\prime}(\mu)\int_{0}^{\infty}\frac{x\mathrm{d}x}{e^{x}+1}$$
כאשר האניטגרל האחרון הוא ידוע ושווה ל:
$$\int_{0}^{\infty}{\frac{x\mathrm{d}x}{e^{x}+1}}={\frac{\pi^{2}}{12}}$$
ולכן נקבל סה"כ:
$$I=\int_{-\infty}^{\infty}\frac{H(\varepsilon)}{e^{\beta(\varepsilon-\mu)}+1}\,\mathrm{d}\varepsilon\approx\int_{-\infty}^{\mu}H(\varepsilon)\,\mathrm{d}\varepsilon+\frac{\pi^{2}}{6\beta^{2}}H^{\prime}(\mu)$$

\end{proof}
\begin{example}[מציאת פוטנציאל כימי עם סומפרפלד]
נתון גז פרמיונים עם \(S=\frac{1}{2}\) בתלת מימד. מתקיים:
$$N={{\frac{V}{2\pi^{2}}\left(\frac{2m}{\hbar^{2}}\right)^{3/2}\int_{0}^{\infty}E^{1/2}f(E)\,\mathrm{d}E}}$$
כאשר נרצה לבצע קירוב סומרפלד עם \(H(E)=\sqrt{ E }\). מתקיים:
$$\int_{0}^{\mu}E^{1/2}\,d E={\frac{2}{3}}\mu^{3/2}\qquad H^{\prime}(E)=\frac{d}{d E}
(E^{1/2})=\frac{1}{2}E^{-1/2}$$
ולכן נקבל:
$$N\approx\frac{V}{2\pi^{2}}\left(\frac{2m}{\hbar^{2}}\right)^{3/2}\left[\frac{2}{3}\mu^{3/2}+\frac{\pi^{2}}{12}(k_{B}T)^{2}\mu^{-1/2}\right]$$
כלומר:
$$N\approx \frac{V}{2\pi^{2}}\left(\frac{2m}{\hbar^{2}}\right)^{3/2}\left( \frac{2}{3}\mu^{3/2}+\frac{\pi^{2}}{12}\left( \frac{k_{B}T}{\mu} \right)^{2} \right)$$

\end{example}
\begin{proposition}[חישוב הפוטנציאל הכימי]
משתמשים בנוסחה:
$$N=\int g\left( E \right)f_{FD}\left( E \right) \;\mathrm{d} E $$
ואז מבודדים את \(\mu\) מהתפלגות פרמי דיראק.

\end{proposition}
\begin{proposition}[חישוב אנרגיית פרמי]
יישר מההגדרה:
$$U=\int_{0}^{\infty}  \varepsilon f_{FD}\left( \varepsilon \right)g(E)\, \mathrm{d}\varepsilon= \int_{\varepsilon_{ground}}^{\varepsilon_{FD}}\varepsilon g\left( \varepsilon \right)  \, d\varepsilon  $$
ואז פותרים בשביל \(\varepsilon_{FD}\) בהתאם לאנרגיה.

\end{proposition}
\section{גז בוזונים}

\begin{reminder}
עבור גז בוזונים מתקיים:
\begin{gather*}{N=\int_{0}^{\infty}g\left( \epsilon \right)\left(\exp\left[ \beta\left( \epsilon-\mu \right) \right]-1\right)^{-1}d\epsilon}\\  {U=\int_{0}^{\infty}g\left( \epsilon \right)\,\epsilon\left(\exp\left[ \beta\left( \epsilon-\mu \right) \right]-1\right)^{-1}d\epsilon} 
\end{gather*}

\end{reminder}
\begin{proposition}
הפוטנציאל הכימי עבור גז בוזונים יהיה קטן או שווה לרמת יסוד

\end{proposition}
\begin{proof}
מספר החלקיקים הממוצע צריך להיות חיובי, ולכן בפרט ההתפלגות בוז אינשטיין צריך להיות חיובי, לכן:
$$\frac{1}{e^{ \beta\left( \varepsilon-\mu \right) }-1}\geq 0\implies e^{ \beta\left( \varepsilon-\mu \right) }-1\geq 0\implies e^{ \beta\left( \varepsilon-\mu \right) }\geq 1$$
ובפרט נכון עבור אנרגיה \(\varepsilon_{0}\) ולכן:
$$e^{ \beta\varepsilon_{0}-\beta \mu }\geq 1\implies \beta \mu \leq \beta\varepsilon_{0}\implies \mu\leq\varepsilon_{0}$$

\end{proof}
\begin{remark}
לרוב מטעמי נוחות קובעים \(\varepsilon_{0}=0\) עבור בוזונים כיוון שקיים אנרגיה אפס של הספינים(בניגוד לפרמיונים).

\end{remark}
\begin{reminder}
צפיפות המצבים של גז אידיאלי הוא:
$$g\left( \epsilon \right)=\frac{V}{4\pi^{2}}\left(\frac{2m}{\hbar^{2}}\right)^{3/2}\epsilon^{1/2}$$

\end{reminder}
\begin{proposition}
מספר החלקיקים עבור גז בוזונים חופשי יהיה:
$$N=\frac{(2S+1)V}{\lambda_{\mathrm{T}}^{3}}\mathrm{Li}_{3/2}(z)\qquad \lambda_{\mathrm{T}}=\frac{h}{\sqrt{2\pi m k_{\mathrm{B}}T}}$$
כאשר \(z=\exp\left( \beta \mu \right)\).

\end{proposition}
\begin{proof}
נציב את הצפופות המצבים של גז אידיאלי בביטוי עבור מספר החלקיקים וכן נבצע החלפת משתנים \(x=\beta\epsilon\) ונקבל:
\begin{gather*}N=\int_{0}^{\infty}\frac{V}{4\pi^{2}}\left(\frac{2m}{\hbar^{2}}\right)^{3/2}\epsilon^{1/2}\left(\exp\left[ \beta\left( \epsilon-\mu \right) \right]-1\right)^{-1}d\epsilon\\={\frac{V}{4\pi^{2}}}\left({\frac{2m}{\hbar^{2}}}\right)^{3/2}(k_{B}T)^{3/2}\int_{0}^{\infty}x^{1/2}\left(\exp\left( -\beta\mu \right)\exp(x)-1\right)^{-1}d x 
\end{gather*}
כאשר אם נציב את הפוגסיטי \(z=\exp\left( \beta \mu \right)\) ונקבל:
$$N=\frac{V}{4\pi^{2}}\left(\frac{2m}{\hbar^{2}}\right)^{3/2}\left(k_{B}T\right)^{3/2}\int_{0}^{\infty}x^{1/2}\left(z^{-1}\exp(x)-1\right)^{-1}d x$$
נזהה את הפולילוגוריתם, נכפיל בניוון ספין ונקבל את הטענה.

\end{proof}
\begin{corollary}
כאשר \(T\to 0\) נקבל כי המקדם \(T^{3/2}\to 0\) אבל כיוון שמספר החלקיקים \(N\) נשאר קבוע נדרש כי האינטגרל יתבדר. נשים לב כי אם \(\mu< 0\) נדרש כי \(z^{-1}>1\). לכן האינטגרל חסום על ידי:
$$\int_{0}^{\infty}x^{1/2}\left(\exp(x)-1\right)^{-1}d x={\frac{\sqrt{\pi}}{2}}\zeta\left({\frac{3}{2}}\right)=1.306{\sqrt{\pi}}=2.315$$
ולא מתבדר! קיבלנו סתירה.

\end{corollary}
\begin{corollary}
המשוואה שיפתחנו לא תהיה נכונה עבור טמפרטורות נמוכות מ-\(T_c\) אשר מתקבל על ידי קבועת האינטגרל לערך המקסימלי שלו אשר נקבע על ידי:
$$k_{B}T_c=\left(\frac{2\pi\hbar^{2}}{m}\right)\left(\frac{N}{2.612V}\right)^{2/3}$$
כאשר \(T_c\) נקראת טמפרטורת אינשטיין, ולעיתים מסומנת ב-\(T_{E}\).

\end{corollary}
\begin{proof}
$$N=\frac{V}{4\pi^{2}}\left(\frac{2m}{\hbar^{2}}\right)^{3/2}\left(k_{B}T_c\right)^{3/2}1.306\sqrt{\pi}$$
כאשר ניתן לכתוב גם בצורה הבאה:
$$N=2.315\frac{V}{4\pi^{2}}\left(\frac{2m}{\hbar^{2}}\right)^{3/2}(k_{B}T_c)^{3/2}$$
וכעת ניתן להעביר אגפים ולקבל:
$$k_{B}T_c=\left(\frac{2\pi\hbar^{2}}{m}\right)\left(\frac{N}{2.612V}\right)^{2/3}$$

\end{proof}
\begin{proposition}
התיקון הנדרש למשוואה:
$$N=\int_{0}^{\infty}g\left( \epsilon \right)\left(\exp\left[ \beta\left( \epsilon-\mu \right) \right]-1\right)^{-1}d\epsilon$$
בטמפרטורות נמוכות הוא:
$$N=N_{0}+\int_{0}^{\infty}g(\epsilon)\left(\exp[\beta(\epsilon-\mu)]-1\right)^{-1}d\epsilon$$
כאשר \(N_{0}\) זה מספר החלקיקים ברמת יסוד כך שמתקיים \(N=N_{0}+N_{1}\).

\end{proposition}
\begin{proof}
ננסה להסביר את המקור של הבעיה. מספר החלקיקים ברמת אנרגיה \(\epsilon\) תהיה:
$$f_{B E}\left( \epsilon \right)=\left\langle  n_{\epsilon} \right\rangle=\frac{1}{e^{\beta(\epsilon-\mu)}-1}$$
כאשר ברמת היסוד נקבל:
$$f_{B E}(0)=\langle n_{0}\rangle=N_{0}=\frac{1}{e^{-\beta\mu}-1}$$
ואם \(\mu\to 0\) נקבל כי מספר החלקיקים ברמת יסוד מתבדר. כלומר \(N_{0}\to \infty\). וכדי להתחשב בזה משוואה נגדיר את \(N_{0}\) בתור מספר החלקיקים ברמת יסוד.

\end{proof}
\begin{definition}[עיבוי בוז אינשטיין]
המצב שבה המערכת נמצאת למתחת לטמפרטורה הקריטית ולכן \(N_{0}> 0\). במצב זה נקבל \(\mu=\varepsilon_{0}\) כיוון שלא נדרשת אנרגיה כלל כדי לאכלס עוד חלקיקים ברמת יסוד.

\end{definition}
\begin{corollary}
עבור בוזונים חופשיים נקבל:
$$N=N_{0}+{\frac{V}{4\pi^{2}}}\left({\frac{2m}{\hbar^{2}}}\right)^{3/2}(k_{B}T)^{3/2}\int_{0}^{\infty}x^{1/2}\left(\lambda^{-1}\exp(x)-1\right)^{-1}d x=N_{0}+\frac{V}{\lambda_{T}}\mathrm{Li}_{3 / 2}(z)$$

\end{corollary}
\begin{remark}
אם מוסיפים ריבוי מספין נקבל:
$$N_{1}=\frac{(2S+1)V}{[\lambda_{\mathrm{T}}(T)]^{3}}\mathrm{Li}_{3/2}(z)$$

\end{remark}
\begin{corollary}
עבור גז חופשי ניתן ניתן לכתוב את מספר החלקיקים בעזרת טמפרטורת אינשטיין ולקבל:
$$N=N_{0}+N\left(\frac{T}{T_c}\right)^{3/2}\implies N_{0}=N\left[1-\left(\frac{T}{T_c}\right)^{3/2}\right]$$

\end{corollary}
\begin{remark}
נשים לב כי כאשר \(T\to T_c\) מלמטה אז \(N_{0}\to 0\). כלומר אם \(T>T_c\) נקבל \(N_{0}\ll N\) ולכן ניתן להזניח את האפקט בקירוב טוב בטמפרטורות יותר גבוההות.

\end{remark}
\begin{proposition}[תלות של פוטנציאל כימי בטמפרטורה]
הפוטנציאל הכימי יהיה בקירוב נתון על ידי:
$$\mu\approx-\frac{k_{B}T}{N}\left[1-\left(\frac{T}{T_c}\right)^{3/2}\right]^{-1}$$

\end{proposition}
\begin{proof}
מהמשוואה:
$$\frac{N_{0}}{N}=\left[1-\left(\frac{T}{T_c}\right)^{3/2}\right]$$
כאשר אם נציב את הביטוי:
$$N_{0}=(\exp[-\beta\mu]-1)^{-1}$$
נקבל:
$$N_{0}=\left[\exp(-\beta\mu)-1\right]^{-1}=N\left[1-\left({\frac{T}{T_c}}\right)^{3/2}\right]$$
וכיוון ש-\(\beta \mu\ll 1\) מתחת ל-\(T_c\) ניתן לפתח את \(-\beta \mu\) בטור טיילור ולקבל:
$$\mu\approx-\frac{k_{B}T}{N}\left[1-\left(\frac{T}{T_c}\right)^{3/2}\right]^{-1}$$

\end{proof}
\begin{proposition}[אנרגיה של גז בוזון אידיאלי]
$$U=1.7826\frac{V}{4\pi^{2}}\left(\frac{2m}{\hbar^{2}}\right)^{3/2}(k_{B}T)^{5/2}$$

\end{proposition}
\begin{proof}
הפיתוח דומה לפיתוח עבור מספר החלקיקים. נכתוב:
$$U=U_{0}+\int_{0}^{\infty}f_{B E}(\epsilon)\epsilon\left(\exp[\beta(\epsilon-\mu)]-1\right)^{-1}\epsilon\,d\epsilon$$
כאשר עבור משוואה זו נקבל כי \(U_{0}=0\) כיוון שהאנרגיה של מצב היסוד הוא 0. כעת נגדיר כמו מקודם \(x=\beta\epsilon\) ונכתוב:
$$U=\frac{V}{4\pi^{2}}\left(\frac{2m}{\hbar^{2}}\right)^{3/2}(k_{B}T)^{5/2}\int_{0}^{\infty}x^{3/2}\left(z^{-1}\exp(x)-1\right)^{-1}d x$$
כאשר האינטגרל נתון על ידי:
$$\int_{0}^{\infty}x^{3/2}\left(\exp(x)-1\right)^{-1}d x=\mathrm{Li}_{5 / 2}(1)=\zeta\left({\frac{5}{2}}\right)\Gamma\left({\frac{5}{2}}\right)=1.341\left({\frac{3}{4}}\right)\pi^{1/2}=1.7826$$
ולכן נקבל:
$$U=1.7826\frac{V}{4\pi^{2}}\left(\frac{2m}{\hbar^{2}}\right)^{3/2}(k_{B}T)^{5/2}$$

\end{proof}
\begin{corollary}
האנרגיה לחלקיק תהיה:
$${\frac{U}{N}}=\left({\frac{1.7826}{2.315}}\right)\left({\frac{T}{T_c}}\right)^{3/2}k_{B}T=.7700\,\left({\frac{T}{T_c}}\right)^{3/2}k_{B}T$$

\end{corollary}
\begin{corollary}
קיבול החום סגולי בנפח קבוע מתקבל גזירה האנרגיה לחלקיק לפי טמפרטורה:
$$c_{V}=1.925\,k_{B}\,\left(\frac{T}{T_c}\right)^{3/2}$$

\end{corollary}
\begin{proposition}[מציאת הטמפרטורה הקריטית]
  \begin{enumerate}
    \item נמצא את מספר החלקיקים במצב המעורער: 
$$N_{\mathrm{exc}}=\int_{\epsilon>0}g(\epsilon)f_{\mathrm{BE}}(\epsilon)\,d\epsilon$$


    \item מניחים \(\mu\to 0\).  


    \item מבודדים את הטמפרטורה. אם ניתן לקבל ביטוי גדול מ-0, אז קיים עיבוי בוז אינשטיין. אחרת לא ייתכן 


  \end{enumerate}
\end{proposition}
\begin{example}[עבוי בוז אינשטיין בגז אידיאלי דו מימדי]
\end{example}
\chapter{הגישה הסטטיסטית - גיבס שאנון}

\section{התפלגות הסתברות}

\begin{definition}[משתנה מקרי]
פונקציה ממרחב האירועים לממשיים. כלומר פונקציה \(X:\Omega\to \mathbb{R}\) אשר מקבלת אירוע ומחזירה מספר ממשי.

\end{definition}
\begin{definition}[התפלגות הסתברות של משתנה מקרי]
פונקציה \(P_{X}:\mathbb{R}\to [0,1]\) המוגדרת על ידי \(P_{X}(x)=P(X=x)\) כלומר מחזירה את ההסתברות ש-\(X\) יקבל את הערך \(x\).

\end{definition}
\begin{definition}[פונקציונאל לינארי על משתנה מקרי  ]
פונקציונאל לינארי הוא פעולה \(\mathcal{L}\) שמקבלת פונקציה \(F(X)\) של משתנה מקרי \(X\) ומחזירה מספר ממשי, כך שלכל \(a,b\in\mathbb{R}\) ולכל פונקציות \(F_1(X),F_2(X)\) מתקיים:
$$\mathcal{L}\left[a F_1(X) + b F_2(X)\right] = a\,\mathcal{L}\left[F_1(X)\right] + b\,\mathcal{L}\left[F_2(X)\right]
$$
דוגמה לפונקציונאל לינארי כזה היא פעולת הממוצע \(\langle F(X)\rangle\).

\end{definition}
\begin{definition}[ממוצע של התפלגות]
יהי \(X\) משתנה מקרי  עם התפלגות הסתברות \(P_{X}\). יהי \(F(X)\) פונקציונאל לינארי. אז הממוצע של \(F(X)\) מוגדר על ידי:
$$\left\langle F\left(X\right)\right\rangle\equiv\sum_{x}F\left(x\right)P\left(X=x\right)$$

\end{definition}
\begin{proposition}
הממוצע הוא לינארי, כלומר מתקיים:
$$\left\langle a F_{1}\left(X\right)+b F_{2}\left(X\right)\right\rangle=a\left\langle F_{1}\left(X\right)\right\rangle+b\left\langle F_{2}\left(X\right)\right\rangle$$

\end{proposition}
\begin{definition}[מומנט]
יהי \(X\) משתנה מיקרי, אזי ניתן להגדיר את המומנט ה-\(n\) של \(X\) על ידי:
$$\left\langle X^{n}\right\rangle=\sum_{x}x^{n}P_{X}\left(x\right)$$

\end{definition}
\begin{example}
המומנט הראשון מתקבל עבור \(n=1\) והוא יהיה הממוצע:
$$\left\langle X\right\rangle=\sum_{x}x P_{X}\left(x\right)$$

\end{example}
\begin{definition}[מומנט מרכזי]
המומנט ה-\(n\) מוגדר על ידי:
$$\langle(X-\langle X\rangle)^{n}\rangle=\sum_{x}\left(x-\langle X\rangle\right)^{n}P_{X}\left(x\right)$$

\end{definition}
\begin{example}
המומנט המרכזי השני מוגדר על ידי:
$$\left\langle\left(X-\langle X\rangle\right)^{2}\right\rangle\equiv\operatorname{Var}\left[X\right]\equiv\sigma_{X}^{2}$$
כאשר זה השונות. המומנט המרכזי השלישי יהיה:
$$\left\langle\left(X-\langle X\rangle\right)^{3}\right\rangle\equiv\sigma_{X}^{3}\mathrm{Skew}\left[X\right]$$
כאשר המומנט המרכזי הרביעי מוגדר:
$$\left\langle\left(X-\langle X\rangle\right)^{4}\right\rangle\equiv\sigma_{X}^{4}\mathrm{Kurt}\left[X\right]$$

\end{example}
\begin{proposition}
ניתן לכתוב את המומנטים בעזרת המומנטים המרכזיים על ידי:
$$\langle X^{n}\rangle=\sum_{k=0}^{n}{\binom{n}{k}}\left\langle(X-\langle X\rangle)^{k}\right\rangle\langle X\rangle^{n-k}$$
ואת המומנטים המרכזיים בעזרת המומנטים על ידי:
$$\left\langle(X-\langle X\rangle)^{n}\right\rangle=\sum_{k=0}^{n}{(-1)}^{n-k}\left({n\atop k}\right)\left\langle X^{k}\right\rangle\langle X\rangle^{n-k}$$

\end{proposition}
\section{פונקציות יוצרות של התפלגות הסתברות}

\begin{definition}[פונקציה יוצרת]
$$G\left(z\right)\equiv\sum_{k=0}^{\infty}P\left(k\right)z^{k}$$

\end{definition}
\begin{proposition}
ניתן לכתוב בעזרת ההגדרה של ממוצע את הפונקציה היוצרת על ידי:
$$G\left(z\right)=\left\langle z^{X}\right\rangle$$

\end{proposition}
\begin{proposition}
ניתן לקבל את \(P(k)\) על ידי:
$$P\left(k\right)=\frac{1}{k!}G^{\left(k\right)}\left(z=0\right)$$

\end{proposition}
\begin{proposition}
באופן כללי מתקיים:
$$G^{(n)}\left(z=1\right)=\left\langle X\left(X-1\right)\ldots\left(X-n+1\right)\right\rangle=\left\langle{\frac{X!}{(X-n)!}}\right\rangle$$

\end{proposition}
\begin{example}[התפלגות פואסון]
התפלגות פואסון מוגדרת על ידי פרמטר \(\lambda\) בצורה הבאה:
$$P\left(X=k\right)=\frac{\lambda^{k}e^{-\lambda}}{k!}$$
לפי ההגדרה:
$$G\left(z\right)=\sum_{k=0}^{\infty}P\left(k\right)z^{k}=\sum_{k=0}^{\infty}\frac{\lambda^{k}e^{-\lambda}}{k!}z^{k}=e^{-\lambda}\sum_{k=0}^{\infty}\frac{\left(\lambda z\right)^{k}}{k!}=e^{-\lambda}e^{\lambda z}=e^{\lambda(z-1)}$$
וכן אם נגזור \(n\) פעמים נקבל:
$$G^{(n)}\left(z\right)=\lambda^{n}e^{\lambda(z-1)}\implies \left\langle{\frac{X!}{(X-n)!}}\right\rangle=G^{(n)}\left(z=1\right)=\lambda^{n}$$

\end{example}
\begin{definition}[פונקציה יוצרת מומנט]
$$M\left(z\right)\equiv\sum_{n=0}^{\infty}\frac{1}{n!}\left\langle X^{n}\right\rangle z^{n}$$

\end{definition}
\begin{remark}
זה יכול להיות שימושי כאשר יש לנו את המומנטים של ההתפלגות, ולא את ההתפלגות במפורש.

\end{remark}
\begin{proposition}
מההגדרה נקבל:
$$\left\langle X^{n}\right\rangle=M^{(n)}\left(z=0\right)$$

\end{proposition}
\begin{proposition}
$$M\left(z\right)=\sum_{n=0}^{\infty}\frac{1}{n!}\left\langle X^{n}\right\rangle z^{n}=\left\langle\sum_{n=0}^{\infty}\frac{1}{n!}\left(z X\right)^{n}\right\rangle=\left\langle e^{z X}\right\rangle$$

\end{proposition}
\begin{corollary}
$$G\left(e^{z}\right)=\left\langle e^{z X}\right\rangle=M\left(z\right)$$

\end{corollary}
\begin{example}[התפלגות פואסון - המשך]
ראינו כי \({G}\left(z\right)=e^{\lambda(z-1)}\) ולכן:
$$M\left(z\right)=G\left(e^{z}\right)=e^{\lambda\left(e^{z}\!-\!1\right)}$$
כעת ניתן לגזור ולקבל:
$$M^{\prime}\left(z\right)=\lambda e^{z}e^{\lambda\left(e^{z}\!-\!1\right)}\qquad M^{\prime\prime}\left(z\right)=\lambda\left(e^{z}e^{\lambda\left(e^{z}-1\right)}+\lambda e^{2z}e^{\lambda\left(e^{z}-1\right)}\right)$$
ולקבל כי השתי מומנטים הראשונים יהיו:
$$\left\langle X\right\rangle=M^{\prime}\left(z=0\right)=\lambda \qquad \left\langle X^{2}\right\rangle=M^{\prime\prime}\left(z=0\right)=\lambda\left(1+\lambda\right)$$

\end{example}
\section{אנטרופיית שאנון}

\begin{definition}[אנטרופיית שאנון]
יהי \(X\) משתנה מקרי עם התפלגות הסתברות \(P_{X}(x)\). אזי האנטרופיה של ההתפלגות מוגדר על ידי:
$$S(P_{X})=-\sum_{x}P_{X}\left(x\right)\log\left(P_{X}\left(x\right)\right)=-\left\langle\log\left(P_{X}\right)\right\rangle$$

\end{definition}
\begin{remark}
נשים לב כי זה פונקציואל אך זה לא לינארי.

\end{remark}
\begin{remark}
מה שהאנטרופיה באמת מודדת זה האי וודאות של ההתפלגות.

\end{remark}
\begin{example}[אנטרופיה של התפלגות אחידה]
תהי משתנה מקרי \(X\) עם התפלגות אחידה על הערכים \(k=\left\{  1,2,\dots,N  \right\}\) כך ש-\(P_{X}(k)=\frac{1}{N}\) לכל \(k\). אזי האנטרופיה של \(P_{X}\) יהיה:
$$S=-\sum_{k=1}^{N} \frac{1}{N}\log\left( \frac{1}{N} \right)=\log N$$

\end{example}
\begin{example}[אנטרופיה של התפלגות דלתא]
נניח כי \(P_{X}(k)=\delta_{k,1}\). נקבל:
$$S=-1\log 1 - \sum_{k\neq 1} 0 \log 0$$
כאשר עבור הגורם השני ניתן לחשוב על זה כגבול:
$$0\log 0 = \lim_{ x \to 0^{+} } x\log x=0$$
ולכן נקבל \(S=0\).

\end{example}
\begin{example}
נניח כי ההתפלגות ההסתברות נתונה על ידי:
$$P_{X}\left(k\right)=\begin{cases}\frac{1}{2^{k}} & k \in \{ 1,2,\dots,N-1 \} \\\frac{1}{2^{N-1}} & k=N
\end{cases}$$
במקרה זה האנטרופייה תהיה:
$$S=\frac{1}{2^{N-1}}(2^{N}-(N+1))\xrightarrow{N\to \infty}2\log 2$$

\end{example}
\begin{proposition}
לכל התפלגות בסתברות נקבל:
$$0\leq S\leq  \log(N)$$
כלומר ההתפלגות עם מקסימום אנטרופיה היא התפלגות אחידה ועם מינימום אנטרופיה היא התפלגות דלתא.

\end{proposition}
\begin{theorem}[עקרון ג'יינס - Jaynes]
בהנתן אילוצים, מערכות מסדרות את עצמן למאקרו מצב עם אינפורמציה מינימלית. כלומר:
$$P_{\text{nature}}\to P_{\mathrm{extemizeing \;\overline{S}_{SB} }}$$

\end{theorem}
\begin{proposition}
הפונקציונאל שממזער את הפונקציונאל:
$$\overline{S} _{GS}[P(\text{micro})]=-k_{B}\sum_{\text{micro}}P(\text{micro})\log(P(\text{micro}))$$
תחת האילוצים:
$$\sum_{\text{micro}}P(\text{micro})=1\qquad \sum_{\text{micro}}E(\text{micro})P(\text{micro})=\langle E \rangle$$
הוא התפלגות מקסוואל בולצמן.

\end{proposition}
\begin{proof}
נשתמש בכופלי לגרנג'. נמצא את המקסימום של הפונקציונאל:
$$\overline{S} _{\lambda_{1},\lambda_{2}}[P(\text{micro})]=\overline{S} _{GS}[P]+\lambda_{1}\left( \sum_{\text{micro}}P(\text{micro})-1 \right)+\lambda_{2}\left( \sum_{\text{micro}}E_{\text{micro}}P(\text{micro})-\langle E \rangle  \right)$$
נגזור:
\begin{gather*}\frac{\delta \overline{S}_{\lambda_{1}\lambda_{2}} }{\delta \lambda_{1}}=0\implies \sum P = 1 \\\frac{\delta \overline{S}_{\lambda_{1},\lambda_{2}}}{\delta \lambda_{2}}=0\implies \sum EP = 1 \\\frac{\delta \overline{S}_{\lambda_{1}\lambda_{2}}}{\delta P}=0\implies -k_{B}[\log(P[\text{micro}]+1)]+\lambda_{1}+\lambda_{2}E(\text{micro})=0 
\end{gather*}
ונקבל:
$$P(\text{micro})=e^{ c }e^{ \lambda_{2}E(\text{micro})/k_{B} }$$
כאשר \(C\equiv \frac{\lambda_{1}}{k_{B}}-1\). נניח כי:
$$\overline{S} _{GS}(P_{\text{extremal}})=S_{B}$$
עבור \(N\to \infty\) ו-\(S_{B}\to S_{\text{Thermo}}\) נקבל:
$$k_{B}\beta=\frac{1}{T} = \frac{\partial S_{\text{Thermo}}}{\partial E}=\frac{\partial S_{B}}{\partial \langle E \rangle }=\frac{\partial \overline{S}_{GS}[P_{\text{extremum}}]}{\partial \langle E \rangle }=\lambda_{2}$$
כלומר קיבלנו כי ל-\(\lambda_{2}\) יש משמעות פיזיקלית - \(\frac{1}{T}\). כעת:
$$P_{\text{extremum}}\propto e^{ -\beta E(\text{micro}) }$$
כעת מהדרישה:
$$\sum P_{\text{extremum}}(\text{micro})=1\implies C=-\log Z_{F}$$
כלומר:
$$Z_{F}=\sum_{\text{micro}}e^{ -\beta E(\text{micro}) }$$
שזו התפלגות בולצמן. נראה שההנחה \(\overline{S}_{GS}(P_{\text{extremum}})=S_{B}\). מתקיים:
\begin{gather*}\overline{S} _{GS}(P_{\text{Extremum}})=-k_{B}\sum P_{\text{ext}}\log P_{\text{ext}}=-k_{B}\sum P_{\text{ext}}\log P_{\text{ext}}=-k_{B}e^{ C }\sum e^{ -\beta E }(C-\beta E)= \\= k_{B}\log (Z )\cancelto{ 1 }{ e^{ C }Z } +\beta k_{B}e^{ C }\times \sum Ee^{ -\beta E }=\frac{1}{T}\left( \frac{1}{\beta}\log Z+E \right)=\frac{E-F}{T}=S_{B}\leftrightarrow  S_{\text{Thermo}}
\end{gather*}

\end{proof}
\begin{corollary}
יש שתי דרכים להגיע לפונקציית החלוקה

  \begin{enumerate}
    \item בולצמן - מניח ארגודיות - כל המיקרו מצבים הם שווים אבל משתמש באמבט. 


    \item גיבס-שאנון - אין צורך באמבט, אבל מאלץ "אינפורמציה מינימלית". 


  \end{enumerate}
\end{corollary}
\chapter{מערכות עם אינטראקציה}

\section{קירוב שדה ממוצע}

\section{מודל איזינג תלת מימדי}

\begin{symbolize}
אם יש לנו חלקיק עם \(K=\left\{  -k,-k+1,\dots,k-1,k  \right\}\) ספינים אפשריים נכתוב:
$$\text{tr}=\sum_{s_{1} \in K}\dots \sum_{s_{N}\in K}$$

\end{symbolize}
\begin{symbolize}
נסמן את הזוגות הצמודים על ידי:
$$A_{\mathsf{nearest}}=\{(i,j)\mid{\textrm{i~and ~j~connected ~by~an~edge}}\}\equiv\{\langle i,j\rangle\}$$

\end{symbolize}
\begin{reminder}
מודל איזינג הוא צבר גיבס עם התמורות \(M\leftrightarrow V\) ו-\(B\leftrightarrow -P\). נזכור כי זהו צבר גיבס. מתקיים:
$$M=\left\langle\sum s_{i}\right\rangle=-{\frac{\partial G}{\partial B}}$$
כאשר האנרגיה של האינטראקציה היא:
$${H}_{\mathrm{int}}=-J\sum_{\langle i,j\rangle}s_{i}s_{j}$$
כאשר פונקציית החלוקה תהיה:
$$Z_{G}=\sum_{s_{1}\in\{1,-1\}}\sum_{s_{2}\in\{1,-1\}}\cdot\cdot\cdot\sum_{s_{3}\in\{1,-1\}}e^{\beta\left(J\sum_{\langle i,j\rangle}s_{i}s_{j}+B\sum_{i}s_{i}\right)}$$

\end{reminder}
\begin{symbolize}
נסמן את הסטייה של ספין מהממוצע להיות, \(\delta s_{i}=s_{i}-m\). פונקציית החלוקה לפי גיבס כעת תהיה:
$$Z_{G}=\mathrm{tr}\exp\left\{J\beta\sum_{\langle i,j\rangle}\left(\delta s_{i}\delta s_{j}+m\left(\delta s_{i}+\delta s_{j}\right)+m^{2}\right)+\beta B\sum_{i}\delta s_{i}+\beta B N m\right\}$$

\end{symbolize}
\begin{definition}[הנחת קירוב השדה הממוצע]
תחת קירוב השדה הממוצע נניח שאין קורולציה בין \(\delta s_{i}\) שונים. כלומר התוכלת אפס:
$$\left\langle\delta{s_{i}}\delta{s_{j}}\right\rangle=0$$

\end{definition}
\begin{corollary}
תחת הנחת הקירוב השדה הממוצע:
$$\mathrm{tr}\left(\delta s_{i}\delta s_{j}e^{-\beta(E-B m)}\right)\approx 0$$

\end{corollary}
\begin{proof}
נחשב את התוחלת \(\mathrm{tr}\left(\delta s_{i}\delta s_{j}e^{-\beta(E-B m)}\right)\) תחת הנחת קירוב השדה הממוצע. לפי ההנחה, אין קורלציה בין \(\delta s_i\) ו-\(\delta s_j\) עבור \(i \neq j\), כלומר:
$$\langle \delta s_i \delta s_j \rangle = \langle \delta s_i \rangle \langle \delta s_j \rangle = 0
$$
כי \(\langle \delta s_i \rangle = \langle s_i - m \rangle = m - m = 0\).

\end{proof}
בנוסף, פונקציית החלוקה מתפצלת למכפלה של איברים בלתי תלויים, ולכן גם בתוחלת משוקללת נקבל:
$$\mathrm{tr}\left(\delta s_{i}\delta s_{j}e^{-\beta(E-B m)}\right) = \langle \delta s_i \delta s_j \rangle_{MF} \cdot Z_G \approx 0
$$
כלומר, האיבר \(\delta s_i \delta s_j\) מתאפס בתוחלת תחת קירוב השדה הממוצע.

\begin{remark}
הקירוב הזה לא עובד עבור \(d\leq 2\).

\end{remark}
\begin{lemma}
סכום כל הקשרים בין ספינים קרובים יהיה:
$$\sum_{\langle i,j \rangle }{1=\frac{N z}{2}}$$
כאשר החלוקה ב-\(2\) נובעת מסכימה כפולה. כמו כן כמות האינטראקציות יהיה:
$$\sum_{\langle i,j\rangle}\left(\delta s_{i}+\delta s_{j}\right)=z\sum_{i}\delta s_{i}$$

\end{lemma}
\begin{proposition}
פונקציית החלוקה תחת הקירוב שדה הממוצע תהיה:
$$Z_{G}^{\mathsf{M F}}=\exp\left\{-{\frac{\beta J N z m^{2}}{2}}\right\}\cdot(2\cosh\left(\beta B_{\mathsf{M F}}\right))^{N}$$

\end{proposition}
\begin{proof}
נתחיל מהביטוי לפונקציית החלוקה תחת קירוב השדה הממוצע:
$$Z_G = \mathrm{tr} \exp\left\{ J\beta \sum_{\langle i,j\rangle} \left( m(\delta s_i + \delta s_j) + m^2 \right) + \beta B \sum_i \delta s_i + \beta B N m \right\}
$$
מהלמה:
$$\sum_{\langle i,j\rangle} \left( m\left( \delta s_i + \delta s_j \right) \right) = z \sum_i m \delta s_i\qquad \sum_{\langle i,j\rangle} m^2 = \frac{N z}{2} m^2
$$
נציב ונקבל:
$$Z_G = \mathrm{tr} \exp\left\{ J\beta z m \sum_i \delta s_i + J\beta \frac{N z}{2} m^2 + \beta B \sum_i \delta s_i + \beta B N m \right\}
$$
נפשט:
$$Z_G = \mathrm{tr} \exp\left\{ (J\beta z m + \beta B) \sum_i \delta s_i + J\beta \frac{N z}{2} m^2 + \beta B N m \right\}$$
נשים לב ש-\(\sum_i \delta s_i = \sum_i (s_i - m) = \sum_i s_i - N m\) ולכן:
$$(J\beta z m + \beta B) \sum_i \delta s_i = (J\beta z m + \beta B) \sum_i s_i - (J\beta z m + \beta B) N m$$
נאסוף את כל האיברים התלויים ב-\(m\):
$$Z_G = \mathrm{tr} \exp\left\{ (J\beta z m + \beta B) \sum_i s_i - (J\beta z m + \beta B) N m + J\beta \frac{N z}{2} m^2 + \beta B N m \right\}$$
נפשט את האיברים שאינם תלויים ב-\(s_i\):
$$-(J\beta z m + \beta B) N m + J\beta \frac{N z}{2} m^2 + \beta B N m = -J\beta z N m^2 - \beta B N m + J\beta \frac{N z}{2} m^2 + \beta B N m
$$
האיברים עם \(\beta B N m\) מתאפסים, ונשאר:
$$-J\beta z N m^2 + J\beta \frac{N z}{2} m^2 = -J\beta \frac{N z}{2} m^2$$
לכן:
$$Z_G = \exp\left\{ -J\beta \frac{N z}{2} m^2 \right\} \cdot \mathrm{tr} \exp\left\{ (J\beta z m + \beta B) \sum_i s_i \right\}$$
נסמן \(B_{MF} = J z m + B\), ואז:
$$Z_G = \exp\left\{ -J\beta \frac{N z}{2} m^2 \right\} \cdot \mathrm{tr} \exp\left\{ \beta B_{MF} \sum_i s_i \right\}$$
העקיבה (tr) היא מכפלה של \(N\) איברים בלתי תלויים, ולכן:
$$\mathrm{tr} \exp\left\{ \beta B_{MF} \sum_i s_i \right\} = \left( \sum_{s_i = \pm 1} e^{\beta B_{MF} s_i} \right)^N = (2 \cosh(\beta B_{MF}))^N
$$
ולכן:
$$Z_G^{MF} = \exp\left\{ -J\beta \frac{N z}{2} m^2 \right\} \cdot (2 \cosh(\beta B_{MF}))^N$$

\end{proof}
\begin{corollary}
מ-\(G^{\mathsf{M F}}=-\frac{1}{\beta}\log Z_{G}^{\mathsf{M F}}\) נקבל את המגנטיזציה:
$$N m=M^{\mathsf{M F}}=\frac{\partial G^{\mathsf{M F}}}{\partial B}\ \Rightarrow\ m=\frac{1}{\beta N}\frac{\partial\log Z_{G}}{\partial B}=\operatorname{tanh}\left(\beta\left(B+J m z\right)\right)$$

\end{corollary}
\begin{remark}
משוואה מהצורה הזאת כמו \(m=\tanh\left( \beta (B+Jmz) \right)\) נקראת self consistency equation. 

\end{remark}
\section{מודל לאנדאו למעברי פאזה}

נסתכל על הפוטנציאל החופשי לחלקיק \(f=\frac{F}{N}\) ונפתח אותו כטור טיילור סביב הנקודה הקריטית:
$$f(T,m)=f_{0}(T)+{\textstyle{\frac{1}{2}}}a(T)\,m^{2}+{\textstyle{\frac{1}{4}}}b\,m^{4}-h\,m+\cdot\cdot\cdot$$
נניח כי \(b> 0\) וכן \(a(T)\approx a_{0}(T-T_{c})\) עם \(a_{0}> 0\). כדי למצוא את הטמפרטורה קריטית נקבע את השדה המגנטי החיצוני \(h=0\) ונדרוש מינימום של הפוטנציאל(כלומר שהנגזרת מתאפסת):
$$\frac{\partial f}{\partial m}=0\quad\implies a(T)\,m+b\,m^{3}=0$$
כאשר \(m=0\) תמיד פתרון, ויש פתרון נוסף רק אם \(a(T)<0\) כלומר \(T<T_{C}\). הפתרון הלא טריוויאלי יהיה \(m=\pm \sqrt{ -\frac{a(T)}{b} }\). 

כמו כן נשים לב כי סביב הנקודה הקריטית נקבל:
$$m(T)=\pm\sqrt{\frac{a_{0}(T_{c}-T)}{b}}$$
ולכן האקספוננט הקריטי שלנו יהיה \(\beta=\frac{1}{2}\).

 Created with Inkscape (http://www.inkscape.org/) \includegraphics[width=0.8\textwidth]{diagrams/svg_12.svg}
\begin{proof}
נפתח את הטנגנס היפרבולי כטור טיילור:
$$m=\operatorname{tanh}(\beta J z m)\approx\beta J z m-\frac{(\beta J z m)^{3}}{3}+\cdot\cdot\cdot$$
נחסיר \(m\) משתי האגפים:
$$0=(\beta J z-1)m-\frac{(\beta J z)^{3}m^{3}}{3}+\cdot\cdot\cdot$$
וניתן להוציא \(m\) גורם משותף:
$$m\left[(\beta J z-1)-\frac{(\beta J z)^{3}m^{2}}{3}+\cdot\cdot\cdot\right]=0$$
וכן \(m=0\) תמיד פתרון ויש פתרון נוסף אם"ם:
$$\beta Jz-1> 0\implies \beta_{c}Jz=1\implies T_{c}=\frac{Jz}{k_{B}}$$

\end{proof}
\begin{remark}
ניתן לראות זאת גרפית על ידי שרטט הגרף של \(y_{1}(m)=m\) ולראות כי אם \(\beta Jz> 1\) נקבל \(y_{2}(m)=\tanh\left( \beta Jm \right)\) חותך את \(y_{1}\) פעמים כאשר אם \(\beta Jz<1\) חותך את \(y_{1}\) רק פעם אחת.

\end{remark}
\begin{remark}
עבור היפוך ספינים \(P:s_{i}\mapsto -s_{i}\) נקבל כי:
$$Z_{G}^{\mathsf{M F}}={\mathrm{tr}}\exp\left\{\beta\:J\sum_{\langle i j\rangle}s_{i}s_{j}\right\}$$
לא משתנה. כיוון שמתקיים \(P\circ P = \mathrm{Id}\) נקבל ש-\(P\) יוצרת את החבורה \(\mathbb{Z}_{2}\).

\end{remark}
\begin{proposition}
עבור היפוך מגנטיזציה \(P:m\mapsto -m\) נקבל סימטריה עבור \(T> T_{c}\) ולכן \(P\) תתואר על ידי \(\mathbb{Z}_{2}\) ושבירת סימטריה עבור \(T<T_{c}\).

\end{proposition}
\begin{definition}[פרמטר סדר - order paramter]
גודל אשר מאפיין כמה מסודרת המערכת. במצב המסודר נקבל כי הוא לא אפס, כאשר הוא מתאפס במצב המסודר.

\end{definition}
\begin{proposition}
אם נסמן \(y=Jz\beta\) נקבל עבור \(T> T_{c}\):
$$\operatorname*{lim}_{T\rightarrow T_{c}^{+}}m=\operatorname*{lim}_{y\rightarrow1^{+}}\sqrt{\frac{3\left(y-1\right)}{y^{3}}}=0$$
כאשר הסוספטביליות תהיה:
$$\chi=\left.\frac{\partial m\left(B\right)}{\partial B}\right|_{B=0}\implies\chi=\beta\frac{1+J z\chi}{\cosh^{2}\left(J z m\beta\right)}\implies\chi=\frac{\beta}{\cosh^{2}\left(J z m\beta\right)-J z\beta}$$

\end{proposition}
\begin{definition}[מעבר פאזה מסדר ראשון]
כאשר הנגזרת הראשונה של האנרגיה החופשית היא לא רציפה במעבר. לדוגמא:

\end{definition}
\begin{proposition}
באופן דומה עבור \(T<T_{c}\):
$$m=\sqrt{3\left(\frac{T}{T_{c}}\right)^{2}\frac{T_{c}-T}{T_{c}}}$$
ועבור הסוספטביליות:
$$\chi\big|_{T\rightarrow T_{c}^{-}}=\frac{1}{2k_{B}\left(T_{c}-T\right)}+O\left((T_{c}-T)^{0}\right)$$
ובכתיבה חזקתית נקבל \(\chi \sim (T_{c}-T)^{-\gamma}\) כאשר עבור מקרה זה \(\gamma_{MF}=1\).

\end{proposition}
\begin{summary}
  \begin{itemize}
    \item \textbf{סימון}: עבור חלקיק עם \(K=\left\{ -k,-k+1,\dots,k-1,k \right\}\) ספינים אפשריים, \(\text{tr}=\sum_{s_{1} \in K}\dots \sum_{s_{N}\in K}\).
    \item \textbf{סימון}: זוגות צמודים מסומנים כ-\(A_{\mathsf{nearest}}=\{\langle i,j\rangle\}\).
    \item \textbf{תזכורת}: מודל איזינג הוא צבר גיבס עם אנרגיית אינטראקציה \(\mathcal{H}_{\mathrm{int}}=-J\sum_{\langle i,j\rangle}S_{i}S_{j}\) ופונקציית חלוקה \(Z_{G}=\sum e^{\beta\left(J\sum_{\langle i,j\rangle}s_{i}s_{j}+B\sum_{i}s_{i}\right)}\).
    \item \textbf{סימון}: סטיית ספין מהממוצע היא \(\delta s_{i}=s_{i}-m\).
    \item \textbf{הגדרה}\textbf{הנחת קירוב השדה הממוצע}: אין קורלציה בין \(\delta s_{i}\) שונים, כלומר \(\left\langle\delta{s_{i}}\delta{s_{j}}\right\rangle=0\).
    \item \textbf{מסקנה}: תחת הנחת קירוב השדה הממוצע, \(\mathrm{tr}\left(\delta s_{i}\delta s_{j}e^{-\beta(E-B m)}\right)\approx 0\).
    \item \textbf{למה}: סכום כל הקשרים בין ספינים קרובים הוא \(\frac{N z}{2}\), וכמות האינטראקציות היא \(z\sum_{i}\delta s_{i}\).
    \item \textbf{טענה}: פונקציית החלוקה תחת קירוב השדה הממוצע היא \(Z_{G}^{\mathsf{M F}}=\exp\left\{-{\frac{\beta J N z m^{2}}{2}}\right\}\cdot(2\cosh\left(\beta B_{\mathsf{M F}}\right))^{N}\).
    \item \textbf{מסקנה}: המגנטיזציה היא \(m=\operatorname{tanh}\left(\beta\left(B+J m z\right)\right)\).
    \item \textbf{טענה}: כאשר \(B=0\), קיים טמפרטורה קריטית \(T_{c}=\frac{Jz}{k_{B}}\). אם \(T>T_{C}\) הפתרון היחיד עבור \(m\) הוא \(m=0\), ואם \(T<T_{C}\) קיים פתרון נוסף.
    \item \textbf{טענה}: עבור היפוך מגנטיזציה \(P:m\mapsto -m\) קיימת סימטריה עבור \(T> T_{c}\) ושבירת סימטריה עבור \(T<T_{c}\).
    \item \textbf{הגדרה}\textbf{פרמטר סדר}: גודל המאפיין את מידת הסדר במערכת; הוא אינו אפס במצב מסודר ומתאפס במצב לא מסודר.
    \item \textbf{טענה}: כאשר \(y=Jz\beta\), עבור \(T> T_{c}\) מתקיים \(\operatorname*{lim}_{T\rightarrow T_{c}^{+}}m=0\), והסוספטביליות היא \(\chi=\frac{\beta}{\cosh^{2}\left(J z m\beta\right)-J z\beta}\).
    \item \textbf{הגדרה}\textbf{מעבר פאזה מסדר ראשון}: כאשר הנגזרת הראשונה של האנרגיה החופשית אינה רציפה במעבר.
    \item \textbf{טענה}: עבור \(T<T_{c}\), המגנטיזציה היא \(m=\sqrt{3\left(\frac{T}{T_{c}}\right)^{2}\frac{T_{c}-T}{T_{c}}}\), והסוספטביליות היא \(\chi\big|_{T\rightarrow T_{c}^{-}}=\frac{1}{2k_{B}\left(T_{c}-T\right)}+O\left((T_{c}-T)^{0}\right)\), כלומר \(\chi \sim (T_{c}-T)^{-\gamma}\) עם \(\gamma_{MF}=1\).
  \end{itemize}
\end{summary}
\section{מודל איזינג חד מימדי}

\begin{definition}[רכיב חופשי ואינטראקציה]
עבור מערכת עם \(N\) דרגות חופש נגדיר:
$$H(\text{micro})=\sum_{i=1}^{N} H_{\text{free}}^{i}+H_{\text{int}}$$
כך שפונקציית החלוקה הכוללת כעת תהיה:
$$Z_{F}\left( \beta \right)=\sum_{\text{micro}}e^{ -\beta H\left( \text{micro}
 \right) }$$

\end{definition}
\begin{definition}[מודל איזינג חד מימדי עם תנאי שפה מחזורי]
מערכת עם \(N\) אתרי ספינים כאשר את הספין של האתר ה-\(i\) נסמן ב-\(s_{i}\) כך שההמילטוניאן יהיה:
$$H_{\text{free}}^{i}=\sum_{i=1}^{N} s_{i}=M\qquad H_{\text{int}}^{\left( \text{micro} \right)}=-J\sum_{i=1}^{N} s_{i}s_{i+1}$$
כאשר \(J\) הוא קבוע(לא כוח מוכלל). כמו כן נזהה \(s_{N+1}=s_{1}\)(תנאי שפה מחזורי).

\end{definition}
\begin{proposition}
פונקציית החלוקה של מודל איזינג מחזורי חד מימדי עבור חלקיקים ספין חצי תהיה:
$$Z ^{1D-\text{Ising}}_{F, \text{int}}= \overbrace{ \sum_{s_{1} \in \{ -1,1 \}}\sum_{s_{2}\in \{ -1,1 \}}\dots \sum_{s_{N}\in \{ -1,1 \}} }^{ \mathrm{tr} } e^{ \beta J\sum s_{i}s_{i+1} }$$

\end{proposition}
\begin{proof}
פונקציית החלוקה היא סכום על כל המיקרו-מצבים של \(e^{-\beta H(\text{micro})}\). במודל איזינג חד-מימדי עם תנאי שפה מחזוריים, ההמילטוניאן הוא:
$$H(\text{micro}) = -J\sum_{i=1}^{N} s_{i}s_{i+1}$$
כאשר \(s_{N+1} = s_1\). לכן:
$$Z^{1D-\text{Ising}}_{F} = \sum_{\{s_i\}} e^{-\beta H(\text{micro})}= \sum_{s_1 = \pm 1} \cdots \sum_{s_N = \pm 1} e^{-\beta \left(-J\sum_{i=1}^{N} s_{i}s_{i+1}\right)}= \sum_{s_1 = \pm 1} \cdots \sum_{s_N = \pm 1} e^{\beta J\sum_{i=1}^{N} s_{i}s_{i+1}}
$$

\end{proof}
\begin{symbolize}
אם יש לנו חלקיק עם \(K=\left\{  -k,-k+1,\dots,k-1,k  \right\}\) ספינים אפשריים נכתוב:
$$\mathrm{tr}=\sum_{s_{1} \in K}\dots \sum_{s_{N}\in K}$$

\end{symbolize}
\section{פתירת המערכת עם ספין חצי}

\begin{proposition}
ניתן לכתוב את פונקציית החלוקה באופן הבא:
$$Z_{G}=\mathrm{Tr}\begin{pmatrix}e^{ \beta(J-B) } & e^{ -\beta J } \\e^{ -\beta J } & e^{ \beta(J+B) }
\end{pmatrix}^{N}$$

\end{proposition}
\begin{proof}
פונקציה החלוקה תהיה:
\begin{gather*}Z_{G}=\mathrm{tr}\left( \exp \left\{  \beta\left( J\sum_{i=1}^{N} s_{i}s_{i+1} + B\sum_{i=1}^{N} s_{i} \right)  \right\} \right)=\mathrm{tr} \exp\left\{\beta\left(J\sum_{i=1}^{N}s_{i}s_{i+1}+\frac{B}{2}\sum_{i=1}^{N}\left(s_{i}+s_{i+1}\right)\right)\right\}=  \\=\mathrm{tr}\Bigg\{\exp\left(\beta\left(J s_{1}s_{2}+\frac{B}{2}\left(s_{1}+s_{2}\right)\right)\right)\times\exp\left(\beta\left(J s_{2}s_{3}+\frac{B}{2}\left(s_{2}+s_{3}\right)\right)\right)\times\ldots \\\dots \times \exp\left(\beta\left(J s_{N-1}s_{N}+{\frac{B}{2}}\left(s_{N-1}+s_{N}\right)\right)\right)\times\exp\left(\beta\left(J s_{N}s_{1}+{\frac{B}{2}}\left(s_{N}+s_{1}\right)\right)\right)\Bigg\}
\end{gather*}
כאשר ניתן להגדיר \(M_{a b}\equiv\exp\left(\beta\left(J a b+{\textstyle{\frac{B}{2}}}\left(a+b\right)\right)\right)\). ניתן לייצג גודל זה בעזרת מטריצה כיוון שמיוצג על ידי שתי אינדקסים. כאשר עבורנו(ספין חצי) נקבל:
$$M=\begin{pmatrix}M_{-1,-1} & M_{-1,1} \\M_{1,-1} & M_{1,1}\end{pmatrix}=\begin{pmatrix}e^{ \beta(J-B) } & e^{ -\beta J } \\e^{ -\beta J } & e^{ \beta(J+B) }
\end{pmatrix}$$
כעת \(e^{-\beta H(s_{1},\cdot\cdot\cdot,s_{N})}=M_{s_{1}s_{2}}M_{s_{2}s_{3}}\cdot\cdot\cdot M_{s_{N}s_{1}}\) ולכן:
$$Z_{G}=\underbrace{ \sum_{s_{1}\in \{ \pm  1\}} \dots \sum_{s_{N}\in \{ \pm  1\}} }_{ \mathrm{tr} } M_{s_{1}s_{2}}M_{s_{2}s_{3}}\dots M_{s_{N}s_{1}}$$
נזכור כי מהגדרת כפל של שתי מטריצות נקבל \((M^{2})_{a b}=\sum_{x}M_{a x}M_{x b}\) כאשר ניתן להכליל את זה ולהראות באינדוקציה כי:
$$(M^{N})_{a b}=\sum_{s_{1},s_{2},\,\ldots,s_{N-1}}M_{a s_{1}}M_{s_{1}s_{2}}\cdot\cdot\cdot M_{s_{N-2}s_{N-1}}M_{s_{N-1}b}$$
ולכן עבורנו נקבל:
$$Z_{G}=\sum_{s_{1}}(M^{N})_{s_{1}s_{1}}=\mathrm{Tr}(M^{N})$$

\end{proof}
\begin{corollary}
כיוון ש-\(M\) סימטרית קיימים לה ערכים עצמיים \(\lambda_{-},\lambda_{+}\) כך ש-\(M_{d}=O^{T}MO\) ומתקיים:
$$Z_{G}=\mathrm{Tr}(M_{d}^{N})=\lambda_{+}^{N}+\lambda_{-}^{N}$$
ובפרט:
$$G=-{\frac{1}{\beta}}\log Z_{G}=-{\frac{1}{\beta}}\log\left(\lambda_{+}^{N}+\lambda_{-}^{N}\right)$$

\end{corollary}
\begin{proposition}
הערכים העצמיים יהיו:
$$\lambda_{\pm}=e^{\beta J}\cosh(\beta B)\pm\sqrt{e^{2\beta J}\sinh^{2}(\beta B)+e^{-2\beta J}}$$

\end{proposition}
\begin{proof}
עבור מטריצה סימטרית כללית הערכים העצמיים יהיו:
$$\lambda_{\pm}={\frac{a+c}{2}}\pm\sqrt{\left({\frac{a+c}{2}}\right)^{2}-\left(a c-b^{2}\right)}$$
וכעת נציב \(a=e^{ \beta(J-B) },b=e^{ -\beta J },c=e^{ -\beta (J+B) }\) ונקבל:
$$a+c=e^{ \beta(J+B) }+e^{ \beta(J-B) }=2e^{ \beta J }\cosh\left( \beta B \right)$$
וכן \(ac=e^{ 2\beta J },b^{2}=e^{ -2\beta J }\) ולאחר הצבה נקבל את הערכים העצמיים הרצויים.

\end{proof}
\begin{proposition}
במימד אחד אין שבירת סימטריה ספונטנית.

\end{proposition}
\begin{proposition}
אין שבירת סימטריה ספונטנית בדו מימד.

\end{proposition}
\begin{lemma}
$$\sum_{s_{j}\in\{\pm1\}}M_{s_{j-1}s_{j}}s_{j}M_{s_{j}s_{j+1}}=(M\sigma_{d}M)_{s_{j-i}s_{j+1}}$$

\end{lemma}
\begin{proof}
$$\sum_{s_{j}\in\{\pm1\}}M_{s_{j-1}s_{j}}s_{j}M_{s_{j}s_{j+1}}=\sum_{s_{j}\in\{\pm1\}}\sum_{a\in\{\pm1\}}M_{s_{j-1}s_{j}}\left(s_{j}\delta_{s_{j}a}\right)M_{a s_{j+1}}=-\left(M\sigma_{d}M\right)_{s_{j-1}s_{j+1}}$$

\end{proof}
\begin{proposition}[קורלציה בין גורמים]
$$\langle s_{j}s_{k}\rangle=\frac{1}{Z_{G}}\,\mathrm{Tr}(\sigma_{z}M^{k-j}\sigma_{z}M^{N-(k-j)})$$

\end{proposition}
\begin{proof}
מההגדרה של תוחלת נקבל:
$$\langle s_{j}s_{k} \rangle =\frac{1}{Z_{G}}\mathrm{tr}\left\{  s_{j}s_{k}e^{ -\beta H\left( s_{1},\dots,s_{N} \right) }  \right\}$$
כיוון ש-\(e^{-\beta H(s_{1},\cdot\cdot\cdot,s_{N})}=M_{s_{1}s_{2}}M_{s_{2}s_{3}}\cdot\cdot\cdot M_{s_{N}s_{1}}\) נקבל:
$$\langle s_{j}s_{k}\rangle=\frac{1}{Z_{G}}\mathrm{tr}\;s_{j}s_{k}\cdot\prod_{i=1}^{N}{ M}_{s_{i}s_{i+1}}$$
נרצה לכתוב את זה בעזרת עכבה. 

$$\langle s_{j}s_{k}\rangle=\frac{1}{Z_{G}}\sum_{s_{1}\in\{\pm1\}}\cdot\cdot\cdot\sum_{s_{j}\in\{\pm1\}}\cdot\cdot\cdot\sum_{s_{k}\in\{\pm1\}}\cdot\cdot\cdot\sum_{s_{N}\in\{\pm1\}}\left(s_{j}s_{k}e^{\overline{{S}}}\right)$$$$=\frac{1}{Z_{G}}\sum_{s_{j}\in\{\pm1\}}\sum_{s_{k}\in\{\pm1\}}s_{j}\underbrace{(M\cdot M\cdot\cdot\cdot M)}_{k-j\ \mathrm{times}}s_{k}\underbrace{(M\cdot M\cdot\cdot\cdot M)}_{N-k+j\ \mathrm{times}}$$
כאשר אם נגדיר \(\sigma_{z}=\begin{pmatrix}1 & 0 \\ 0 & -1\end{pmatrix}\) נקבל:
$$\left\langle s_{j}s_{k}\right\rangle=\frac1{Z_{G}}{\mathsf{T r}}\left[\sigma_{z}M^{k-j}\sigma_{z}M^{N-k+j}\right]$$

\end{proof}
\begin{corollary}
אם \(\lambda_{+},\lambda_{-}\) הם הערכים העצמיים של מטריצת המעבר נקבל:
$$\langle s_{j}s_{k} \rangle = \left(\frac{\lambda_{-}}{\lambda_{+}}\right)^{k-j}+\left(\frac{\lambda_{-}}{\lambda_{-}}\right)^{N-k+j}$$

\end{corollary}
\begin{proof}
\end{proof}
כעת אם נעבור למטריצה להמלכסנת \(\sigma_{0}=O^{T}\sigma _zO\) נקבל:
$$\langle s_{j}s_{k}\rangle=\frac{1}{Z_{G}}{\mathsf{T r}}\left[\sigma_{z}O M_{d}^{k-j}O^{T}\sigma_{z}O M_{d}^{N-k+j}O^{T}\right]=\frac{1}{Z_{G}}{\mathsf{T r}}\left[\sigma_{o}M_{d}^{k-j}\sigma_{o}M_{d}^{N-k+j}\right]$$
עבור המקרה \(B=0\) נקבל:
$$M={\binom{e^{\beta J}}{e^{-\beta J}}}{e^{-\beta J}}~\Rightarrow~O={\frac{1}{\sqrt{2}}}{\binom{1}{1}}~\Rightarrow~\sigma_{o}=O^{T}\sigma_{z}O={\binom{0}{-1}}~,$$
וכעת נקבל:
$$\langle s_{j}s_{k}\rangle=\frac{1}{\underbrace{\lambda_{+}^{N}}_{Z_{G}}}{\sf T r}\left[\left( \begin{array}{c c c}{{0}}&{{1}}\\ {{-1}}&{{0}}\end{array} \right)\left( \begin{array}{c c c}{{\lambda_{+}}}&{{0}}\\ {{0}}&{{\lambda_{-}}}\end{array} \right)^{k-j}\left( \begin{array}{c c c}{{0}}&{{1}}\\ {{-1}}&{{0}}\end{array} \right)\left( \begin{array}{c c c}{{\lambda_{+}}}&{{0}}\\ {{0}}&{{\lambda_{-}}}\end{array} \right)^{N-k+j}\right]=\left(\frac{\lambda_{-}}{\lambda_{+}}\right)^{k-j}+\left(\frac{\lambda_{-}}{\lambda_{-}}\right)^{N-k+j}$$

\begin{corollary}
אם נניח שהמרחק בין החלקיקים קטן, כלומר \(k-j\ll N\) ונזניח את האיבר השני נקבל:
$$\left\langle s_{j}s_{k}\right\rangle=\left[\operatorname{tanh}\left(\beta J\right)\right]^{k-j}$$
בטמפרטורה \(T\to 0\) נקבל \(\left\langle S_{j}S_{k}\right\rangle\sim e^{-\frac{k-j}{\xi}}\) עם \(\xi=\frac{1}{2}e^{ 2\beta J }\).

\end{corollary}
\section{פתירת המערכת עם ספין כללי}

\begin{definition}[המילטוניאן של מערכת ספין כללית]
$$H\left(\left\{s_{i}\right\}\right)=-\sum_{i=1}^{N}f\left(s_{i},s_{i+1}\right)\qquad s_{i}\in\left\{\sigma_{1},\sigma_{2},...,\sigma_{n}\right\}$$
כאשר \(f\) היא פונקציה המקשרת בין שכנים קרובים. למשל עבור חלקיק עם ספין חצי:
$$f\left(s_{i},s_{i+1}\right)=J s_{i}s_{i+1}+{\frac{h}{2}}\left(s_{i}+s_{i+1}\right)\qquad s_{i}\in\{-1,1\}$$

\end{definition}
\begin{proposition}
הספין הממוצע נתון על ידי:
$$\langle s_{i} \rangle  =\frac{1}{Z}\sum_{s_{1}}...\sum_{s_{N}}s_{i}\prod_{i=1}^{N}g\left(s_{i},s_{i+1}\right)$$
כאשר \(g\left(s_{i},s_{i+1}\right)=e^{\beta f(s_{i},s_{i+1})}\).

\end{proposition}
\begin{proof}
$$\langle s_{i}\rangle=\frac{1}{Z}\sum_{\{s_{i}\}}s_{i}e^{-\beta H(\{s_{i}\})}={\frac{1}{Z}}\sum_{\{s_{i}\}}s_{i}e^{\beta\sum_{i=1}^{N}f(s_{i},s_{i+1})}=\frac{1}{Z}\sum_{s_{1}}...\sum_{s_{N}}s_{i}\prod_{i=1}^{N}g\left(s_{i},s_{i+1}\right)$$

\end{proof}
וניתן לסמן את מטרית המעבר:
$$(P)_{\sigma_{j}\sigma_{k}}=g\left( \sigma_{j},\sigma_{k} \right)$$

\begin{symbolize}[מטריצת ספינים]
נגדיר את המטריצה:
$$S=\begin{pmatrix}\sigma_{1} & 0 & 0 & 0 \\0 & \sigma_{2} & 0 & 0 \\0 & 0 & \ddots & 0 \\0 & 0 & 0 & \sigma_{N}
\end{pmatrix}$$
כך שמתקיים \(S=\sum_{j}|\sigma_{j}\rangle\,\sigma_{j}\,\langle\sigma_{j}|\).

\end{symbolize}
\begin{corollary}
אם ל-\(P\) יש ערכים עצמיים \(\left\{  \lambda_{i}  \right\}\) נקבל:
$$P=\sum_{i}\left|u_{i}\right\rangle\lambda_{i}\left\langle u_{i}\right|\implies P^{R}=\sum_{i}\left|u_{i}\right\rangle\lambda_{i}^{R}\left\langle u_{i}\right|$$
וניתן למצוא את היצוג שלו בבסיס הספינים:
$$\left(P^{R}\right)_{\sigma_{k},\sigma_{\ell}}=\langle\sigma_{k}|\left[\sum_{i}|u_{i}\rangle\,\lambda_{i}^{R}\,\langle u_{i}|\right]|\sigma_{\ell}\rangle=\sum_{i}\left\langle\sigma_{k}|u_{i}\right\rangle\lambda_{i}^{R}\,\langle u_{i}|\sigma_{\ell}\rangle$$
וכעת:
$$\langle s_{1}s_{R+1}\rangle=\frac{1}{Z}\sum_{s_{1}}\sum_{s_{R+1}}s_{1}\left[\sum_{i}\,\langle s_{1}|u_{i}\rangle\,\lambda_{i}^{R}\,\langle u_{i}|s_{R+1}\rangle\right]s_{R+1}\left[\sum_{j}\,\langle s_{R+1}|u_{j}\rangle\,\lambda_{j}^{N-R}\,\langle u_{j}|s_{1}\rangle\right]s_{R+1}^{N},$$$$=\frac{1}{Z}\sum_{i}\sum_{j}\lambda_{i}^{R}\lambda_{j}^{N-R}\left[\sum_{s_{1}}\left\langle u_{j}|s_{1}\right\rangle s_{1}\left\langle s_{1}|u_{i}\right\rangle\right]\left[\sum_{s_{R+1}}\left\langle u_{i}|s_{R+1}\right\rangle s_{R+1}\left\langle s_{R+1}|u_{j}\right\rangle\right]$$$$=\frac{1}{Z}\sum_{i}\sum_{j}\lambda_{i}^{R}\lambda_{j}^{N-R}\left\langle u_{j}\right|S\left|u_{i}\right\rangle\left\langle u_{i}\right|S\left|u_{j}\right\rangle$$
וכיוון ש-\(Z=\sum_{j}\lambda_{j}^{N}\) ניתן לכתוב:
$$\langle s_{1}s_{R+1}\rangle=\frac{\sum_{i}\sum_{j}\lambda_{i}^{R}\lambda_{j}^{N-R}\left|\left\langle u_{j}\right|S\left|u_{i}\right\rangle\right|^{2}}{\sum_{j}\lambda_{j}^{N}}=\frac{\sum_{i}\sum_{j}\left(\frac{\lambda_{i}}{\lambda_{1}}\right)^{R}\left(\frac{\lambda_{j}}{\lambda_{1}}\right)^{N-R}|\langle u_{j}|\,S\,|u_{i}\rangle|^{2}}{\sum_{j}\left(\frac{\lambda_{j}}{\lambda_{1}}\right)^{N}}$$
אם נניח בה"כ כי \(\lambda_{1}\) הערך עצמי הגבוהה ביותר נקבל:
$$\langle s_{1}s_{R+1}\rangle\stackrel{N\rightarrow\infty}{\longrightarrow}\sum_{i}\left(\frac{\lambda_{i}}{\lambda_{1}}\right)^{R}|\langle u_{1}|\,S\,|u_{i}\rangle|^{2}$$

\end{corollary}
\section{פיתוח טמפרטורות נמוכות}

\begin{definition}[המילטוניאן של מערכת חלקיקים בגז מרכזי  ]
נניח מערכת של \(N\) חלקיקים בלתי תלויים במרחב תלת-ממדי, שכל אחד מהם כפוף לפוטנציאל מרכזי \(U(\lvert \mathbf{q} \rvert)\). ההמילטוניאן של המערכת נתון על ידי:\\
$$H(\{\mathbf{q}_i, \mathbf{p}_i\}) = \sum_{i=1}^{N} \left[ \frac{\mathbf{p}_i^2}{2m} + U(\lvert \mathbf{q}_i \rvert) \right]
$$
כאשר \(m\) היא מסת החלקיקים, \(\mathbf{q}_i \in \mathbb{R}^3\) הוא מיקום החלקיק ה-\(i\), \(\mathbf{p}_i \in \mathbb{R}^3\) הוא התנע שלו, ו-\(U\) הוא פונקציה רדיאלית התלויה רק במרחק מהמוצא.

\end{definition}
\begin{definition}[פונקציית החלוקה הקנונית  ]
פונקציית החלוקה הקנונית עבור טמפרטורה \(T\) נתונה בקירוב הסמי-קלאסי על ידי:\\
$$Z_{F} = \frac{1}{N!} \left( \frac{1}{\lambda^3} \cdot I(\beta) \right)^N
$$
כאשר \(\beta = \frac{1}{k_B T}\) הוא ההופכי לטמפרטורה, \(\lambda = V_{*}^{1/3} \sqrt{ \frac{\beta}{2\pi m} }\) הוא אורך הגל התרמי, ו-\(I(\beta)\) הוא אינטגרל על המרחב הקונפיגורציוני:\\
$$I(\beta) = \int_{\mathbb{R}^3} e^{ -\beta U(\lvert \mathbf{q} \rvert) } \, \mathrm{d}^3 q
$$

\end{definition}
\begin{lemma}[רדוקציה רדיאלית של \(I(\beta)\)]
על ידי מעבר לקואורדינטות ספריות מתקבל:\\
$$I(\beta) = 4\pi \int_0^\infty q^2 e^{ -\beta U(q) } \, \mathrm{d}q
$$

\end{lemma}
\begin{remark}
כאשר \(\beta \gg 1\), כלומר בטמפרטורות נמוכות, האינטגרל נשלט על ידי תרומות קרובות לנקודת המינימום של הפוטנציאל. נסמן את נקודת המינימום של \(U\) ב-\(q_*\).

\end{remark}
\begin{definition}[פיתוח טיילור סביב המינימום  ]
מניחים ש-\(U\) גזירה מספיק פעמים, ומבצעים פיתוח טיילור סביב \(q = q_*\):\\
$$U(q) = U_0 + U_2 (q - q_*)^2 + U_3 (q - q_*)^3 + U_4 (q - q_*)^4 + \dots
$$\\

כאשר \(U_0 = U(q_*)\), \(U_2 = \frac{1}{2} U''(q_*)\), \(U_3 = \frac{1}{3!} U^{(3)}(q_*)\) ו-\(U_4 = \frac{1}{4!} U^{(4)}(q_*)\).

\end{definition}
\begin{proposition}[קירוב נקודת האוכף לאינטגרל \(I(\beta)\) ]
נבצע שינוי משתנה \(x = q - q_*\) ונקרב את גבולות האינטגרציה ל-\((-\infty, \infty)\), פעולה שמוצדקת כאשר \(\beta \gg 1\):\\
$$I(\beta) \approx 4\pi e^{ -\beta U_0 } \int_{-\infty}^{\infty} (x + q_*)^2 \exp \left( -\beta U_2 x^2 - \beta U_3 x^3 - \beta U_4 x^4 - \dots \right) \mathrm{d}x
$$

\end{proposition}
\begin{remark}
כדי לחשב את האינטגרל נפתח את הפונקציה האקספוננציאלית בטור טיילור ונשלב כל איבר בנפרד. לשם כך נשתמש בנוסחאות גאוסיות מהצורה:\\
$$\int_{-\infty}^{\infty} x^n e^{-a x^2} \mathrm{d}x =\begin{cases}\frac{(n-1)!!}{(2a)^{n/2}} \sqrt{ \frac{\pi}{a} } & n \text{ even} \\0 & n \text{ odd}\end{cases}
$$

\end{remark}
\begin{proposition}[הצורה הסופית של הפיתוח  ]
לאחר חישוב האינטגרלים מתקבל:\\
$$I(\beta) = 4\pi^{3/2} e^{ -\beta U_0 } q_*^2 \xi \left( 1 + A_2 \xi^2 + A_4 \xi^4 + \dots \right)
$$\\

כאשר \(\xi = \frac{1}{\beta \sqrt{U_2}}\),
$$A_2 = \frac{8U_2^2 - 24q_*U_2 U_3 + 15q_*^2 U_3^2 - 12 q_*^2 U_2 U_4}{16 q_*^2 U_2^2}, \qquad A_4 = \frac{15 U_4}{8 q_*^2 U_2}
$$

\end{proposition}
\begin{definition}[פיתוח פונקציית החלוקה בטמפרטורה נמוכה  ]
נחזיר את \(I(\beta)\) אל ביטוי פונקציית החלוקה ונקבל:\\
$$Z_F = \frac{1}{N!} \left( \frac{1}{\lambda^3} \cdot 4\pi^{3/2} e^{ -\beta U_0 } q_*^2 \xi \left( 1 + A_2 \xi^2 + A_4 \xi^4 + \dots \right) \right)^N
$$

\end{definition}
\begin{example}[פוטנציאל הרמוני  ]
אם \(U(q) = \frac{1}{2} m \omega^2 q^2\), אז \(q_* = 0\), \(U_2 = \frac{1}{2} m \omega^2\), וכל הנגזרות הגבוהות יותר מתאפסות. במקרה זה, מתקבל אינטגרל מדויק:\\
$$I(\beta) = 4\pi \int_0^\infty q^2 e^{ -\frac{1}{2} \beta m \omega^2 q^2 } \mathrm{d}q = 4\pi \cdot \frac{\sqrt{\pi}}{2} \left( \frac{2}{\beta m \omega^2} \right)^{3/2}
$$
תוצאה זו תואמת את הפיתוח הכללי במקרה בו הפוטנציאל קוואדרטי.

\end{example}
\begin{remark}
פיתוח זה מאפשר קירוב של פונקציית החלוקה בטמפרטורות נמוכות ומהווה כלי מרכזי להבנת התנהגות תרמודינמית של מערכות קלאסיות או קוונטיות. הוא מופיע בהקשרים שונים כמו שיטת WKB, אינטגרלים פונקציונליים, ותורת השדה הסטטיסטית.

\end{remark}
\begin{example}[הוכחת נוסחאת סטרלינג עם קירוב טמפרטורות נמוכות]
נשתמש כעת בקירוב בטמפרטורות נמוכות כדי להכיח את נוסחאת סטרלינג.
נגדיר:
$$\tilde{I}(\beta)=\int e^{ -\beta U(q) } \;\mathrm{d}q=e^{ -\beta U_{0} }-\pi^{1/2}\xi\left( 1+ \frac{15U_{3}^{2}-12U_{2}U_{4}}{16 U_{2}^{2}}\xi^{2}+\dots \right)$$
אנו יודעים כי פונקציית גמא מוגדרת על ידי:
$$N! = \Gamma(N+1)=\int _{0}^{\infty}e^{ N\log x-x } \;\mathrm{d}x=$$
נכתוב:
$$x=Nq \qquad U_{\Gamma}(q)=q-\log q$$
וזה מביא אותנו לצורה:
$$N! = Ne^{ N\log N }\int _{0}^{\infty}e^{ -NU_{\Gamma}(q) } \;\mathrm{d}q$$
אנו יודעים כי המינימום הוא 1, לכן \(q_{*}=1\) וכן המקדמים של הפיתוח של \(U\) יהיו:
$$U_{0}^{\Gamma} = 1\quad U_{2}^{\Gamma}=\frac{1}{2}\quad U_{3}^{\Gamma}=-\frac{1}{3}\quad U_{4}^{\Gamma}=\frac{1}{4}$$
כאשר:
$$N! = Ne^{ N\log N }e^{ -N }\sqrt{ \frac{2\pi}{N} }\left( 1+\frac{1}{12N}+\dots \right)$$
ולכן:
$$\log(N!)=N\log N-N+\log \sqrt{ 2\pi N }+\log\left( 1+\frac{1}{12N} \right)+\dots$$

\end{example}
\section{פיתוח בטמפרטורות גבוהות - cluster expansion}

\begin{proposition}
עבור גז של \(N\) חלקיקים לא מובחנים בנפח \(V\), טמפרטורה \(T\) ופוטנציאל אינטראקציה של בין כל שתי החלקיקים הנתון על ידי: \(u_{ij}=u(\lvert r_{i}-r_{j} \rvert)\) פונקציית החלוקה תהיה:
$$Z_{N}=\frac{1}{N!\lambda^{3N}}Q_{N}$$
כאשר:
$$\lambda=\frac{h}{\sqrt{ 2\pi mk_{B}T }}\qquad Q_{N}=\int_{V^{N}}\prod_{1\leq i<j\leq N}e^{-\beta u\left( \mathbf{r}_{i}-\mathbf{r}_{j} \right)}d^{3}\mathbf{r}_{1}\cdot\cdot\cdot d^{3}\mathbf{r}_{N}$$

\end{proposition}
\begin{definition}[פונקציית מייר - Mayer]
מוגדר על ידי:
$$f_{i j}:=e^{-\beta u_{i j}}-1$$
כאשר הרעיון זה שזה שואף לאפס כאשר \(\beta\to 0\) כלומר בטמפרטורות גבוהות.

\end{definition}
\begin{corollary}
מתקיים:
$$\prod_{i<j}e^{-\beta u_{i j}}=\prod_{i<j}(1+f_{i j})$$
כאשר נשים לב כי המכפלה הזו מכילה \({N\choose 2}\) גורמים ולכן ניתן לכתוב כעת את \(Q_{N}\) על ידי:
$$Q_{N}=\int_{V^{N}}\prod_{i<j}(1+f_{i j})d^{3N}{\bf r}$$

\end{corollary}
\begin{lemma}
במקום לסכום על כל הקשרים \(i<j\) ניתן לסכום על כל הגרפים \(G\) של \(N\) קודקודים (המייצגים חלקיקים) והחיבורים שלהם(המייצגים \(f_{ij}\)) כלשהו כך שנקבל:
$$\prod_{i<j}(1+f_{i j})=\sum_{G}\prod_{(i,j)\in G}f_{i j}$$

\end{lemma}
\begin{proof}
אם פותחים את המכפלה \(\prod_{i<j}(1+f_{i j})\) נקבל סכום של כל שילוב אפשרי של מכפלות של \(f_{ij}\) של חלקיקים שונים \(f_{ij}\). אם נייצג את המכפלה על ידי חיבור של הקודקודים בגרף נקבל סכום של כל גרף אפשרי, כאשר עבור כל גרף אנחנו סוכמים על כל הצלעות.

\end{proof}
\begin{corollary}
ניתן לכתוב את \(\prod_{i<j}(1+f_{ij})\) בתור סכום על גרף \(G\) עם \(N\) קודקודים המתאימים ל-\(f_{ij}\) מתאים. הקודקודים מייצגים את החלקיקים והקצוות מייצגות \(f_{ij}\). כלומר:
$$Q_{N}=\sum_{G}\int_{V^{N}}\prod_{(i,j)\in G}f_{i j}\,d^{3N}\mathbf{r}$$
כאשר סכימה על גרף זה סכימה על כל הדרכים שבהם ניתן לבחור איזה \(f_{ij}\) להכליל.

\end{corollary}
\begin{definition}[גוש קשיר]
גוש קשיר הוא הסכום של כל הדרכים לחבר \(s\) נקודות בצורה קשירה.

\end{definition}
\begin{definition}[מקדמי קשירות גושיים]
מקדם הקשירות \(b_{k}\) מייצג את התרומה לפונקציית החלוקה של אוסף של \(k\) חלקיקים. בצורה מפורשת ניתן להגדיר:
$$b_{k}:=\frac{1}{k!V}\int_{V^{k}}\left(\sum_{G\in{\mathcal{C}}_{k}}\prod_{(i,j)\in G}f_{i j}\right)d^{3k}{\bf r}$$
כאשר \(\mathcal{C}_{k}\) זה קבוצת הגרפים הקשירים עם \(k\) קודקודים, ו-\(V\) זה נפח.

\end{definition}
\begin{remark}
כיוון שתלוי בפונקציית מייר הגודל \(b_{k}\) תלוי בפוטנציאל אינטראקציה \(u_{ij}\) ובטמפרטורה.

\end{remark}
\begin{example}[חישוב המקדמים עבור ספרה קשיחה]
ספרה קשיחה היא עם קוטר \(\sigma\) כך שמחוץ לספרה הפוטנציאל מתאפס ובפנים הוא אינסופי. כלומר:
$$u(r)=\left\{{\begin{array}{l l}{\infty}&{r<\sigma}\\ {0}&{r\geq\sigma}\end{array}}\right.\implies f(r)=e^{-\beta u(r)}-1={\left\{\begin{array}{l l}{-1}&{r<\sigma}\\ {0}&{r\geq\sigma}\end{array}\right.}$$

\end{example}
\begin{symbolize}
מספר הדרכים שניתן לפרק \(N\) חלקיקים ל-\(i\) גושים מסומן ב-\(n_{i}\).

\end{symbolize}
\begin{corollary}
$$\sum_{k=1}^{N}k n_{k}=N$$

\end{corollary}
\begin{corollary}
מספר הדרכים לחלק \(N\) חלקיקים לגושים בגדלים נתונים נתון על ידי:
$$\Omega^{N}(n_{1},...,n_{N})=\frac{N!}{\prod_{k=1}^{N}(k!)^{n_{k}}n_{k}!}$$

\end{corollary}
\begin{example}
  \begin{enumerate}
    \item עבור \(\Omega^{3}(1,1,0)\) נקבל גוש 1 מגודל 1 וגוש 1 מודל 2. אכן יש \(3=1+2\) חלקיקים ולכן הביטוי מתקמפל. נציב בנוסחא: 
$$\Omega^{3}(1,1,0)=\frac{3!}{(1!)^{1}\cdot1!\cdot(2!)^{1}\cdot1!}=\frac{6}{1\cdot1\cdot2\cdot1}=\frac{6}{2}=3$$


    \item עבור \(\Omega^{4}(0,2,0,0)\) נקבל 2 גושים מוגדל 2. ואכן מתקיים \(4=2+2\) ונקבל: 
$$\Omega^{4}(0,2,0,0)=\frac{4!}{(2!)^{2}\cdot2!}=\frac{24}{4\cdot2}=\frac{24}{8}=3$$


    \item עבור \(\Omega^{5}(2,0,1,0,0)\) נקבל 2 גושים מודל 1, וגוש 1 מודל 3. אכן \(3+2\cdot 1 = 5\). לכן נקבל: 
$$\Omega^{5}(2,0,1,0,0)=\frac{5!}{(1!)^{2}\cdot2!\cdot(3!)^{1}\cdot1!}=\frac{120}{1\cdot2\cdot6\cdot1}=\frac{120}{12}=10$$


  \end{enumerate}
\end{example}
\begin{corollary}
$$Q_{N}=\sum_{\mathrm{partitions}}\Omega^{N}(n_{1},...,n_{N})\prod_{k=1}^{N}(k!b_{k})^{n_{k}}$$

\end{corollary}
\begin{proof}
נתחיל מהביטוי:
$$Q_{N}=\sum_{G}\int_{V^{N}}\prod_{(i,j)\in G}f_{i j}\,d^{3N}\mathbf{r}$$
נחלק את הגרף לרכיבי הקשירות \(G\cong G_{1}\sqcup G_{2}\sqcup\cdot\cdot\cdot\sqcup G_{m}\) כאשר כל גרף הוא רכיב קשירות על הקודקודים \(S_{j}\subseteq \{ 1,\dots,N \}\) ו-\(S_{j}\) זרות. כעת:
$$\prod_{(i,j)\in G}f_{i j}=\prod_{j=1}^{m}\left(\prod_{(i,j)\in G_{j}}f_{i j}\right)$$
ולכן:
$$\int_{V^{N}}\prod_{(i,j)\in G}f_{i j}\,d^{3N}\mathbf{r}=\prod_{j=1}^{m}\left(\int_{V^{|S_{j}|}}\prod_{(i,j)\in G_{j}}f_{i j}\,d^{3|S_{j}|}\mathbf{r}\right)$$
כאשר נשים לב כי אם הגרף קשיר מתקיים:
$$b_{k}\cdot k!\cdot V =\int_{V^{|S_{j}|}}\prod_{(i,j)\in G_{j}}f_{i j}\,d^{3|S_{j}|}\mathbf{r}$$
כאשר כיוון שיש \(n_{k}\) גושים כאלה, וניתן להתחשב הדרכים השונות לסדר אותם כך שנקבל:
$$Q_{N}=\sum_{\stackrel{n_{1},\ldots,n_{N}}{\sum k n_{k}=N}}\Omega^{N}(n_{1},\ldots,n_{N})\prod_{k=1}^{N}(k!b_{k})^{n_{k}}$$

\end{proof}
\begin{example}
נרצה לחשב את \(Q_{5}\). נמנה את כל הדרכים שניתן לחלק 5 חלקיקים ונחשב את הריבוי שלהם:
\begin{gather*}\Omega_{(0,0,0,0,1)}^5 = 1\qquad \Omega_{(1,0,0,1,0)}^5 = 5\qquad \Omega_{(0,1,1,0,0)}^5 = 10 \qquad\Omega_{(2,0,1,0,0)}^5 = 10 \\ \Omega_{(3,1,0,0,0)}^5 = 10 \qquad\Omega_{(1,2,0,0,0)}^5 = 15 \qquad\Omega_{(5,0,0,0,0)}^5 = 1 
\end{gather*}
וכעת ניתן לחשב:
\begin{gather*}Q_5 = \Omega_{(0,0,0,0,1)}^5 (5!b_5) + \Omega_{(1,0,0,1,0)}^5 (1!b_1)(4!b_4) + \Omega_{(0,1,1,0,0)}^5 (2!b_2)(3!b_3) + \Omega_{(2,0,1,0,0)}^5 (1!b_1)^2 (3!b_3) + \\+\Omega_{(3,1,0,0,0)}^5 (1!b_1)^3 (2!b_2) + \Omega_{(1,2,0,0,0)}^5 (1!b_1)(2!b_2)^2 + \Omega_{(5,0,0,0,0)}^5 (1!b_1)^5 
\end{gather*}
ולאחר הצבה של הריבויים ופישוט נקבל:
$$Q_{5}=120b_{5}+120b_{1}b_{4}+120b_{2}b_{3}+60b_{1}^{2}b_{3}+20b_{1}^{3}b_{2}+60b_{1}b_{2}^{2}+b_{1}^{5}$$

\end{example}
\begin{proposition}
עבור מערכת של \(N\) חלקיקים עם אינטרקציה המפעילה פוטנציאל על זוגות \(v_{ij}\) פונקציית החלוקה \(Z\) נתונה על ידי:
$$Z=\frac{1}{N!}\left(\frac{2\pi m}{\beta h^{2}}\right)^{3N/2}\int\cdot\cdot\cdot\int\prod_{i<j}(1+f_{i j})\,d x_{1}\cdot\cdot\cdot d x_{N}$$
כאשר \(f_{ij}=e^{ -\beta v_{ij} }-1\).

\end{proposition}
\begin{summary}
  \begin{itemize}
    \item עבור מערכת עם אינטראקציה, כאשר האינטראקציה בין החלקיק ה-\(i\) ל-\(j\) נתון על ידי \(u_{ij}=u(\lvert r_{i}-r_{j} \rvert)\) פונקציית החלוקה תהיה:
$$Z_{N}=\frac{1}{N!\lambda^{3N}}Q_{N}$$
כאשר:
$$\lambda=\frac{h}{\sqrt{ 2\pi mk_{B}T }} \qquad Q_{N}=\int_{V^{N}}\prod_{1\leq i<j\leq N}e^{-\beta u\left( \mathbf{r}_{i}-\mathbf{r}_{j} \right)}d^{3}\mathbf{r}_{1}\cdot\cdot\cdot d^{3}\mathbf{r}_{N}$$
    \item נגדיר פונקציית מייר \(f_{i j}:=e^{-\beta u_{i j}}-1\) אשר קטן בטמפרטורות גובההות. בעזרתו ניתן לכתוב את \(Q_{N}\) על ידי סכום על גרפים עם \(N\) קודקודים:
$$Q_{N}=\sum_{G}\int_{V^{N}}\prod_{(i,j)\in G}f_{i j}\,\mathrm{d}^{3N}\mathbf{r}$$
    \item ניתן לחלק את הסכימה לרכיבים קשירים ולכתוב את \(Q_{N}\) על ידי:
$$Q_{N}=\sum_{\mathrm{partitions}}\Omega^{N}(n_{1},...,n_{N})\prod_{k=1}^{N}(k!b_{k})^{n_{k}}$$
כאשר \(b_{k}\) נקראים מקדמי הקשירות הגושיים ונתון על ידי סכימה על כל הגרפים הקשירים עם \(k\) חלקיקים.
$$b_{k}:=\frac{1}{k!V}\int_{V^{k}}\left(\sum_{G\in{\mathcal{C}}_{k}}\prod_{(i,j)\in G}f_{i j}\right)d^{3k}{\bf r}$$
וכן הגודל \(\Omega^{N}(n_{1},n_{2},\ldots,n_{N})\) מייצג את מספר הדרכים לסדר \(N\) חלקיקים לגושים בגדלים שונים. אם \(n_{k}\) זה מספר הגושים בגודל \(k\)(כך שמתקיים \(kn_{k}=N\)) אז נקבל:
$$\Omega^{N}(n_{1},n_{2},\ldots,n_{N})=\frac{N!}{\prod_{k=1}^{N}(k!)^{n_{k}}n_{k}!}$$
  \end{itemize}
\end{summary}
\section{הפיתוח הויראלי}

\begin{reminder}
ראינו מהפיתוח בטמפרטורות גבוהות כי:
$$Z_{N}=\frac{1}{N!\lambda^{3N}}Q_{N}\qquad Q_{N}=\sum_{x=\left(n_{1},\,.\,.\,,n_{N}\right)}\Omega_{x}^{N}\prod_{k=1}^{N}(k!b_{k})^{n_{k}}$$

\end{reminder}
הביטוי:
$$\Omega_{x}^{N}={\frac{N!}{\prod_{k=1}^{N}(k!)^{n_{k}}n_{k}!}},\qquad\sum_{k=1}^{N}k n_{k}=N$$
נהיה קשה לחישוב ולכן נרצה לפשט אותו. לשם כך נעבור לצבר הגראנד קנוני.

\begin{proposition}
הפונקציית חלוקה הגראנד קנונית נתונה על ידי:
$${\mathcal{Z}}_{g}=\exp\left(\sum_{k=1}^{\infty}{\frac{z^{k}b_{k}}{\lambda^{3k}}}\right)$$

\end{proposition}
\begin{proof}
נזכור כי ניתן לקבל את פונקציית החלוקה של הצבר הגראנד קנוני מהפונקציית חלוקה הקנונית על ידי:
$$\mathcal{Z}_{g}=\sum_{N=0}^{\infty}z^{N}Z_{N},\qquad z=e^{\beta\mu}$$
כעת נציב את הביטוי עבור \(Z_{N}\):
$${\mathcal{Z}}_{g}=\sum_{N=0}^{\infty}z^{N}\left({\frac{1}{N!\lambda^{3N}}}Q_{N}\right)=\sum_{N=0}^{\infty}\left({\frac{z}{\lambda^{3}}}\right)^{N}\sum_{x=(n_{1},\ldots,n_{N})}\prod_{k=1}^{N}{\frac{b_{k}^{n_{k}}}{n_{k}!}}$$
נשתמש בזהות קומבינטורית כדי להחליף את סדר הסכימה:
$$\sum_{N=0}^{\infty}\sum_{x=(n_{1},\ldots,n_{N})}\prod_{k=1}^{N}\frac{1}{n_{k}!}\left(\frac{z^{k}b_{k}}{\lambda^{3k}}\right)^{n_{k}}=\prod_{k=1}^{\infty}\sum_{n_{k}=0}^{\infty}\frac{1}{n_{k}!}\left(\frac{z^{k}b_{k}}{\lambda^{3k}}\right)^{n_{k}}=\prod_{k=1}^{\infty}\exp\left(\frac{z^{k}b_{k}}{\lambda^{3k}}\right)^{N_{k}/2}$$
ולכן וכיוון שניתן להפוך מכפלה של אקספוננטים לאקספוננט של סכום נקבל:
$${\mathcal{Z}}_{g}=\exp\left(\sum_{k=1}^{\infty}{\frac{z^{k}b_{k}}{\lambda^{3k}}}\right)$$

\end{proof}
\begin{corollary}
הגראנד פוטנציאל נתון על ידי:
$$\Phi_{G}=-\frac{1}{\beta}\log{\mathcal{Z}}_{g}=-\frac{1}{\beta}\sum_{k=1}^{\infty}\frac{z^{k}b_{k}}{\lambda^{3k}}$$

\end{corollary}
\begin{corollary}
אם נסמן \({\bar{b}}_{k}:=\operatorname*{lim}_{V\to\infty}{\frac{b_{k}}{V}}\) אזי הלחץ והצפיפות חלקיקים מקיימים:
$$P=\frac{1}{\beta}\sum_{k=1}^{\infty}\frac{z^{k}\bar{b}_{k}}{\lambda^{3k}}\quad,\quad n=\sum_{k=1}^{\infty}\frac{k z^{k}\bar{b}_{k}}{\lambda^{3k}}$$

\end{corollary}
\begin{proof}
מיידית מהזהות:
$$P=-\left({\frac{\partial \Phi_{G}}{\partial V}}\right)_{T,\mu}\qquad n={\frac{N}{V}}=-{\frac{1}{V}}\left({\frac{\partial \Phi_{G}}{\partial\mu}}\right)=-{\frac{\beta z}{V}}{\frac{\partial \Phi_{G}}{\partial z}}$$

\end{proof}
\begin{definition}[הפיתוח הוויראלי]
פיתוח בטור חזקות של הצפיפות מהצורה:
$$P\beta=n+B_{2}n^{2}+B_{3}n^{3}+B_{4}n^{4}+\cdot\cdot\cdot$$

\end{definition}
\begin{proposition}
מהשוואה של הרכיבים מהפיתוח ממקודם נקבל כי הגורמים הראשונים יהיו:
$$B_{2}=-\bar{b}_{2},\ \ \ B_{3}=4\bar{b}_{2}^{2}-2\bar{b}_{3},\ \ \ B_{4}=-20\bar{b}_{2}^{3}+18\bar{b}_{2}\bar{b}_{3}-3\bar{b}_{4}$$

\end{proposition}
\begin{proof}
נסמן את המקדמים של הטור של \(n\) על ידי \(a_{k}=\frac{k\bar{b}_{k}}{\lambda^{3k}}\) ושל הטור של \(b\) על ידי \(b_{k}={\frac{\bar{b}_{k}}{\lambda^{3k}}}\)(זהו \(b_{k}\) שונה). נפוך את של הטור של \(n(z)\) לטור של \(z(n)\):
$$z=\alpha_{1}n+\alpha_{2}n^{2}+\alpha_{3}n^{3}+\cdot\cdot\cdot$$
נציב את זה בסדרה המקורית של \(n\):
\begin{gather*}n=a_{1}z+a_{2}z^{2}+a_{3}z^{3}+\cdot\cdot\cdot= \\=a_{1}(\alpha_{1}n+\alpha_{2}n^{2}+\alpha_{3}n^{3})+a_{2}(\alpha_{1}^{2}n^{2}+2\alpha_{1}\alpha_{2}n^{3})+a_{3}(\alpha_{1}^{3}n^{3})+\cdot\cdot\cdot =\\=\mathbf{\sigma}=a_{1}\alpha_{1}n+\left(a_{1}\alpha_{2}+a_{2}\alpha_{1}^{2}\right)n^{2}+\left(a_{1}\alpha_{3}+2a_{2}\alpha_{1}\alpha_{2}+a_{3}\alpha_{1}^{3}\right)n^{3}+\cdot\cdot\cdot
\end{gather*}
ומהדרישה שהמקדם השל \(n\) יהיה 1 וכל שאר המקדמים יהיו 0 נקבל:
$$\alpha_{1}=\frac{1}{a_{1}}\qquad \alpha_{2}=-\frac{a_{2}}{a_{1}^{3}}\qquad \alpha_{3}=-\frac{1}{a_{1}^{4}}(a_{3}a_{1}^{2}-2a_{1}a_{2}^{2})$$
נציב את זה כעת בטור של \(P\beta\) ונקבל:
\begin{gather*}P\beta=b_{1}(\alpha_{1}n+\alpha_{2}n^{2}+\alpha_{3}n^{3})+b_{2}(\alpha_{1}^{2}n^{2}+2\alpha_{1}\alpha_{2}n^{3})+b_{3}(\alpha_{1}^{3}n^{3})+\cdot\cdot\cdot = \\=\left(b_{1}\alpha_{1}\right)n+\left(b_{1}\alpha_{2}+b_{2}\alpha_{1}^{2}\right)n^{2}+\left(b_{1}\alpha_{3}+2b_{2}\alpha_{1}\alpha_{2}+b_{3}\alpha_{1}^{3}\right)n^{3}+\cdot\cdot\cdot
\end{gather*}
ולאחר פישוט והצבה של ה-\(\alpha\) נקבל:
$$P\beta=n+\left( -\bar{b}_{2} \right)n^{2} +\left( 4\bar{b}_{2}^{2}-2\bar{b}_{3} \right)n^{3}+\dots$$
וניתן היה להמשיך באותה דרך להראות את הגורם \(B_{4}\).

\end{proof}
\begin{lemma}[חישוב \(B_{2}\) בסימטריה כדורית דו מימדית]
מההגדרה אנו יודעים כי:
$$B_{2}=-\bar{b}_{2}=-\frac{1}{2V}\int\mathrm{d}^{3}q_{1}\mathrm{d}^{3}q_{2}\left(e^{-\beta U\left(|q_{1}-q_{2}|\right)}-1\right)$$
נעבור לקורדינטות מרכז מסה:
$${\bf q}_{\mathrm{cm}}=\frac{q_{1}+q_{2}}{2},\quad{\bf q}_{-}=q_{1}-q_{2}$$
נציב ונקבל:
$$B_{2}=-\frac12\int{\bf d}^{3}q_{-}\left(e^{-\beta U(|q_{-}|)}-1\right)$$
ובקורדינטות כדוריות:
$$B_{2}=-2\pi\int_{0}^{\infty}q_{-}^{2}\,\left(e^{-\beta U(q_{-})}-1\right){\mathrm d}q_{-}$$

\end{lemma}
\begin{example}[פוטנציאל ואן דר וואלס]
נשתמש בפוטנציאל:
$$U(r)=\left\{\begin{array}{l l}{{\infty}}&{{r<q_{*}}}\\ {{-U_{*}\left({\frac{q_{*}}{r}}\right)^{6}}}&{{r>q_{*}}}\end{array}\right.$$
נפצל את האינטגרל:
$$B_{2}=-2\pi\left[\int_{0}^{q_{*}}(-1)r^{2}\mathrm{d}r+\int_{q_{*}}^{\infty}r^{2}\left(e^{\beta U(r)}-1\right)\mathrm{d}r\right]$$
נפתח בטמפרטורות גבוהות(\(\beta U_{*}\ll 1\)) כלומר \(e^{\beta U(r)}-1\approx\beta U(r)+\cdot\cdot\cdot\) וכעת:
$$B_{2}\approx\frac{2\pi q_{*}^{3}}{3}(1-\beta U_{*})$$
ולכן אם נציב בפיתוח הווראלי נקבל:
$$P\beta=n+\frac{2\pi q_{*}^{3}}{3}(1-\beta U_{*})n^{2}+\cdot\cdot\cdot$$
ניתן להשוואות את זה למשוואה המורכת:
$$\left(P+\frac{N^{2}a}{V^{2}}\right)(V-N b)=N k_{B}T\Rightarrow P\beta=n+(b-\beta a)n^{2}+\cdot\cdot\cdot$$
ואם נתאים מקדמים נקבל:
$$b=\frac{2\pi q_{*}^{3}}{3},\quad a=b U_{*}$$

\end{example}
\begin{example}[כדור קשיר תלת מימדי]
הפוטנציאל של כדור קשיר נתון על ידי:
$$U(r)=\left\{\begin{array}{l l}{{\infty}}&{{r<\sigma}}\\ {{0}}&{{r\ge\sigma}}\end{array}\right.
$$
כלומר חלקיקים לא יכולים להתקרב יותר מ-\(\sigma\). נחשב את \(B_{2}\): 
$$B_{2}=-\bar{b}_{2}=-\frac{1}{2V}\int\mathrm{d}^{3}{\bf r}_{1}\mathrm{d}^{3}{\bf r}_{2}\left(e^{-\beta U\left(|{\bf r}_{1}-{\bf r}_{2}|\right)}-1\right)$$
כיוון שהמערכת אינווריאנטית להזזות ניתן לקבע \(\mathbf{r_{1}}=0\) ולסמן \(\mathbf{r_{2}}=\mathbf{r}\). כעת:
$$B_{2}=-{\frac{1}{2}}\int_{\mathbb{R}^{3}}\left(e^{-\beta U\left(|\mathbf{r}|\right)}-1\right)\mathrm{d}^{3}\mathbf{r}$$
אבל נזכור כי עבור \(r<\sigma\) נקבל \(e^{ -\beta U\left( \mathbf{r} \right) }=0\) ועבור \(r\geq \sigma\) נקבל \(e^{ -\beta U\left( \mathbf{r} \right) }=1\) ולכן:
$$e^{-\beta U(r)}-1=\left\{{\begin{array}{l l}{-1}&{r<\sigma}\\ {0}&{r\geq\sigma}\end{array}}\right.$$
כלומר:
$$B_{2}=-{\frac{1}{2}}\int_{|\mathbf{r}|<\sigma}(-1)\,\mathrm{d}^{3}\mathbf{r}={\frac{1}{2}}\cdot\operatorname{Volume~of~ball~of~radius}~\sigma$$
כאשר הנפח של כדור תלת מימדי נותן:
$$B_{2}=-{\frac{1}{2}}\int_{|\mathbf{r}|<\sigma}(-1)\,\mathrm{d}^{3}\mathbf{r}={\frac{1}{2}}\cdot\operatorname{Volume~of~ball~of~radius}\,\sigma$$
כאשר נזכור הנפח של כדור תלת מימדי נתון על ידי \(\mathrm{Vol}(r<\sigma)=\frac{4\pi}{3}\sigma^{3}\) ולכן:
$$B_{2}=\frac{2\pi}{3}\sigma^{3}$$

\end{example}
\begin{example}[כדור קשיח \(d\) מימדי]
עבור פוטנציאל מהצורה:
$$U(r)=\left\{\begin{array}{l l}{{\infty}}&{{r<q_{*}}}\\ {{0}}&{{r>q_{*}}}\end{array}\right.$$
נקבל:
$$B_{2}^{(d)}=-{\frac{1}{2}}\int_{\mathbb{R}^{d}}\left(e^{-\beta U(r)}-1\right)\mathrm{d}^{d}r={\frac{1}{2}}\int_{|r|<q_{*}}\mathrm{d}^{d}r={\frac{\Omega_{d}}{2}}\int_{0}^{q_{*}}r^{d-1}\mathrm{d}r$$
ולכן:
$$B_{2}^{(d)}=\frac{\Omega_{d}}{2d}q_{*}^{d}\quad,\quad\Omega_{d}=\mathrm{area~of~unit~}S^{d-1}={\frac{2\pi^{d/2}}{\Gamma(d/2)}}$$

\end{example}
\begin{definition}[דיאגרמה 1PI]
דיאגרמות קשירות אשר נשארות קשירות תחת כל הסרה של חלקיק יחיד(1-particle irreducible).

\end{definition}
\begin{symbolize}
יהי \(\mathcal{D}_{\ell}\subseteq \mathcal{C}_{\ell}\) הדיאגרומות \(1PI\). אזי נסמן:
$$\bar{d}_{\ell}=\frac{1}{\ell!V}\int_{V^{\ell}}\left(\sum_{G\in\mathcal{D}_{\ell}}\prod_{(i,j)\in G}f_{i j}\right)\,\mathrm{d}^{3\ell}\mathbf{r}$$
כלומר זהו התרומה לפונקציית החלוקה לכל הגרפים שהם \(1PI\).

\end{symbolize}
\begin{theorem}[מייר - מייר]
יהי \(\overline{d_{\ell}}\) התרומה לפונקציה חלוקה מכל הגרפים אשר הם 1PI. אזי \(B_{\ell}=(1-\ell)\bar{d}_{\ell}\).

\end{theorem}
\begin{example}
עבור \(\ell=2\) יש רק גרף אחד - קשר יחיד בין שני הקודקודים. לכן נקבל:
$${\bar{d}}_{2}={\frac{1}{2!V}}\int_{V^{2}}f_{12}\,\mathrm{d}^{3}\mathbf{r}_{1}\,\mathrm{d}^{3}\mathbf{r}_{2}$$
כאשר מסימטריה להזזות ניתן לקבע \(\mathbf{r_{1}}=0\) ולקבל:
$${\bar{d}}_{2}={\frac{1}{2}}\int_{\mathbb{R}^{3}}f(r)\,\mathrm{d}^{3}\mathbf{r}$$
כך ש:
$$B_{2}=(1-2)\bar{d}_{2}=-\bar{d}_{2}$$

\end{example}
\begin{example}
עבור \(\ell=3\) אם נרצה לחשב את \({\bar{b}_{3}}\) נקבל:
$$\bar{b}_{3}=\frac{1}{3!V}\int_{V^{3}}\left[f_{12}f_{23}+f_{12}f_{13}+f_{13}f_{23}+f_{12}f_{23}f_{13}\right]\mathrm{d}^{3}r_{1}\mathrm{d}^{3}r_{2}\mathrm{d}^{3}r_{3}$$
כאשר רק האיבר האחרון בסכימה נותן גרף \(1PI\) ולכן:
$$\bar{d}_{3}=\frac{1}{3!V}\int_{V^{3}}f_{12}f_{23}f_{13}\,\mathrm{d}^{3}r_{1}\mathrm{d}^{3}r_{2}\mathrm{d}^{3}r_{3}\Rightarrow B_{3}=\left( 1-\ell \right)\bar{d}_{3} =-2\bar{d}_{3}$$

\end{example}
\chapter{מערכות מחוץ לשיווי משקל}

\section{מומנטיים}

\begin{definition}[פונקציית יוצר מומנטים]
יהי \(X\) משתנה מקרי בעל צפיפות \(P_X(x)\). נגדיר את פונקציית יוצר המומנטים:
$$Z(J) \equiv \langle e^{J x}\rangle = \int e^{J x} P_X(x)\,\mathrm{d}x$$
כאשר \(J\) נקרא המקור (source).

\end{definition}
\begin{proposition}[חישוב מומנט מהפונקציה \(Z(J)\)]
המומנט ה-\(n\) של \(X\) ניתן על ידי:
$$\langle x^n \rangle = \frac{1}{Z(0)} \frac{\mathrm{d}^n}{\mathrm{d}J^n} Z(J) \Big|_{J=0}$$

\end{proposition}
\begin{remark}
בפיזיקה סטטיסטית משתמשים ב-\(Z\) גם כפונקציית חלוקה (partition function), שכן היא יוצר מומנטים של משתנים אקראיים.

\end{remark}
\begin{definition}[קומיולנט]
נגדיר:
$$W(J) \equiv \log Z(J)
$$
זה אנלוגי לאנרגיה חופשית בלוגריתם של פונקציית החלוקה.

\end{definition}
\begin{proposition}[מומנטים קומיולנטים]
המומנט ה-\(n\) הקומיולנטי הוא:
$$\langle x^n \rangle_C \equiv \frac{\mathrm{d}^n}{\mathrm{d}J^n} W(J)\Big|_{J=0}
$$
ובפרט לגאוסיאן רק הקומיולנט השני אינו אפס.

\end{proposition}
\begin{example}[דוגמה חד-ממדית: התפלגות גאוסיאנית]
יהי
$$P_X(x) = \frac{1}{\sqrt{2\pi\sigma^2}} \exp\Bigl(-\tfrac{1}{2}\bigl(\tfrac{x-x_0}{\sigma}\bigr)^2\Bigr).
$$
חשבנו:
$$Z(J) = \int_{-\infty}^{\infty} \frac{1}{\sqrt{2\pi\sigma^2}} e^{ - \tfrac{1}{2}(\frac{x-x_0}{\sigma})^2 + Jx }\;\mathrm{d}x= e^{ J x_0 + \tfrac{1}{2}J^2\sigma^2 }.
$$
לכן:
$$\langle x\rangle = x_0\qquad\langle x^2\rangle = x_0^2 + \sigma^2$$
ולכן הקומיולנטים: \(\langle x\rangle_C = x_0\), \(\langle x^2\rangle_C = \sigma^2\), ול-\(n>2\)\(\langle x^n\rangle_C=0\).

\end{example}
\begin{remark}
קומיולנטים מעל השני מתאפסים מאפיינת התפלגות גאוסיאנית.

\end{remark}
\section{אינטגרלים גאוסיים}

\begin{proposition}[טריק אינטגרל גאוסיאני חד-ממדי]
יהי
$$
I(J) = \int_{-\infty}^{\infty} \exp\Bigl\{-\tfrac12 x G^{-1} x + J x\Bigr\}\,\mathrm{d}x\quad G\in\mathbb{R}$$
משלימים לריבוע:
$$I(J) = \int \exp\Bigl\{-\tfrac12 (x - G J) G^{-1} (x - G J) + \tfrac12 G J^2 \Bigr\}\,\mathrm{d}x= e^{\tfrac12 G J^2}\,I(0).
$$

\end{proposition}
\begin{proposition}[הכללה רב-ממדית]
יהי \(G\) מטריצה סימטרית חיובית. ניקח
$$I(\mathbf{J}) = \int \exp\Bigl\{ -\tfrac12 \sum_{j,k} x_j (G^{-1})_{jk} x_k + \sum_j J_j x_j \Bigr\} \prod_{j=1}^n \mathrm{d}x_j.
$$
בהחלפת בסיס אורתוגונלי \(O^T G O = \mathrm{diag}(w_j)\) ומשתנים חדשים \(\tilde J = O^T J\), מתקבל:
$$I(\mathbf{J}) = I(\mathbf{0}) \exp\Bigl\{\tfrac12 \sum_{j,k} J_j G_{jk} J_k \Bigr\}.
$$

\end{proposition}
\begin{remark}
התוצאה \(\exp(\tfrac12 J^T G J)\) מצביעה שקומיולנט שני הוא \(G_{jk}\), ולכן \(G\) מושג כגרין.

\end{remark}
\begin{proposition}[מעבר לגבול הרציף (אינטגרל מסלולי)]
במקרה של משתנה תלוי זמן \(x(t)\) בדיסקרטיזציה \(t_j = j\Delta t\), לוקחים \(n\to\infty,\;\Delta t\to0\) עם \(n\Delta t=T\) קבוע. הקומיולנט השני בדיסקרט:
$$\frac{\partial}{\partial J(t_{i_1})}\frac{\partial}{\partial J(t_{i_2})} \Bigl\{\tfrac12 \sum_{i,j} J(t_i) G(t_i, t_j) J(t_j)\Bigr\}= G(t_{i_1},t_{i_2}),
$$
ובגבול הרציף מתקבל פונקציית גרין \(G(t_1,t_2)\).

\end{proposition}
\begin{remark}
זה הבסיס לגישת אינטגרל מסלולי מחוץ לשיווי משקל, כאשר ה-\(J\) הם שדות מקור.

\end{remark}
\begin{definition}[רעש לבן גאוסיאני (cumulant)]
יהי \(\eta(t)\) רעש המקיים:
$$\langle \eta(t)\rangle_C=0,\quad \langle \eta(t_1)\eta(t_2)\rangle_C = \delta(t_1-t_2),
$$
ובגלל גאוסיאניות, קומולנטים מעל שניים מתאפסים:
$$\langle \eta(t_1)\dots \eta(t_N)\rangle_C = 0,\; N>2.
$$

\end{definition}
\begin{remark}
ידיעת קומולנט שני וקיום גאוסיאני קובעים לחלוטין את ההתפלגות, ומשמשים בתיאור סטוכסטי באמצעות אינטגרל מסלולי.

\end{remark}
\begin{remark}
בחישובים סטוכסטיים נוצרים אינטגרלים על מסלולים \(x(t)\) תחת משקל גאוסיאני שנקבע על-פי גרין \(G\), ומתבססים על הקורלציות ה-\(\delta\) של הרעש לבן.

\end{remark}
\section{הילוך אקראי}

\begin{reminder}[משתנה מקרי]
פונקציה אשר מקבלת אירוע ומחזירה מספר ממשי. בצורה פורמלית זהי פונקציה מדידה מקבוצה \(X\) למרחב מידה הסתברותית.

\end{reminder}
\begin{definition}[משתנים אקראיים בלתי תלויים]
קבוצה \(\{ X_{i} \}_{}\) של משתנים אקראיים אשר מקיימים:

\end{definition}
\begin{definition}[הליכה אקראית בזמן דיסקרטי]
אוסף של \(N\) משתנים מקריים \(\{ \Delta W_i \}_{i=1}^N\) שהם \(I.I.D\) (בלתי תלויים ומתפלגים זהה).\\
\textbf{בלתי תלויים}:
$$P_{\Delta W_1, \dots, \Delta W_N}(\Delta W_1, \dots, \Delta W_N) = \prod_{i=1}^N P_{\Delta W_i}(\Delta W_i)$$\textbf{מתפלגים זהה}:
$$P_{\Delta W_i}(\Delta W) = P_{\Delta W_j}(\Delta W)$$
וקטור השינויים \((\Delta W_1,\dots,\Delta W_N)\) נקרא הליכה אקראית בזמן דיסקרטי.

\end{definition}
\begin{definition}[הליכה אקראית בסריג  ]
הליכה אקראית כך שקיים ערך \(\Delta W\) כך ש-
$$P_{\Delta W_i}(\Delta W) \neq 0$$
כאשר \(\Delta W\) נמצא בסריג כלשהו.

\end{definition}
\begin{definition}[הליכה אקראית סימטרית חד-ממדית (מהלך שיכור)  ]
התפלגות השינויים:\\
$$P(\Delta W = -1) = \frac{1}{2}, \quad P(\Delta W = 1) = \frac{1}{2}$$
כלומר, בכל צעד דיסקרטי, החלקיק נע ימינה או שמאלה בהסתברות \(\frac{1}{2}\).

\end{definition}
\begin{proposition}[תכונות של מהלך שיכור  ]
  \begin{itemize}
    \item ממוצע: 
$$\langle \Delta W_i \rangle = (-1)\cdot\frac{1}{2} + 1\cdot\frac{1}{2} = 0$$
    \item שונות: 
$$\langle \Delta W_i^2 \rangle = (-1)^2\cdot\frac{1}{2} + 1^2\cdot\frac{1}{2} = 1$$
    \item קורולציה:
$$\langle \Delta W_i \Delta W_j \rangle = \delta_{ij}$$
    \item סכום השינויים:
$$\left\langle \sum_{i=1}^N \Delta W_i \right\rangle = 0$$
    \item סטיית תקן של הסכום:
$$\sqrt{ \left\langle \left( \sum_{i=1}^N \Delta W_i \right)^2 \right\rangle } = \sqrt{N}$$
  \end{itemize}
\end{proposition}
\begin{remark}
סטיית התקן המשתנה בסדר גודל של \(\sqrt{N}\) היא גם הסיבה לגדילת שגיאות ניסוי במעבדה בסדר גודל של \(\sqrt{ N }\).

\end{remark}
\begin{definition}[תהליך ווינר - Weiner process]
נגדיר הליכה אקראית \(I.I.D\)\(R_W(\Delta W_1, \dots, \Delta W_N)\) כך ש:\\
$$\langle \Delta W_i \rangle = 0, \qquad \langle \Delta W_i^2 \rangle = 1$$
ונגדיר עבור \(t \in \mathrm{R}_{\geq 0}\):
$$W(t) = \lim_{N \to \infty} \frac{1}{\sqrt{N}} \sum_{j=1}^{\lfloor Nt \rfloor} \Delta W_j$$

\end{definition}
\begin{remark}
זה למעשה הגבול הרציף של הילוך אקראי.

\end{remark}
\begin{proposition}[תכונות של תהליך ווינר]
  \begin{enumerate}
    \item מקיים \(W(0) = 0\) בהסתברות 1. 


    \item אינקרמנטים סטציונריים:  \\
$$W(t+s)-W(t) \overset{d}{=} W(s)$$


    \item אינקרמנטים בלתי תלויים: \\
$$W(t_1)-W(s_1), \dots, W(t_k)-W(s_k)$$
כאשר בלתי תלויים אם:
$$0 \leq s_1 \leq t_1 \leq s_2 \leq t_2 \leq \dots \leq s_k \leq t_k$$


    \item אינקרמנטים גאוסיאניים: \\
$$W(t+\Delta t) - W(t) \sim \mathcal{N}(0, \Delta t)$$
כלומר:\\
$$P_{\Delta W}(x) = \frac{1}{\sqrt{2\pi \Delta t}} \exp\left( -\frac{x^2}{2\Delta t} \right)$$


  \end{enumerate}
\end{proposition}
\begin{definition}[רעש לבן גאוסיאני (GWN)  ]
נגדיר משתנה אקראי \(\eta(t)\) כך ש:\\
$$W(t) = \int_0^t \eta(t') \,\mathrm{d}t'$$
אז \(\eta(t)\) הוא משתנה גאוסיאני עם ממוצע אפס:
$$\langle \eta(t) \rangle = 0$$

\end{definition}
\begin{proposition}[קורולציה של רעש לבן]
נרצה לחשב \(\langle \eta(t)\eta(t') \rangle\) ונשתמש בזהות:\\
$$\langle W^2(t) \rangle = \int_0^t \mathrm{d}t_1 \int_0^t \mathrm{d}t_2 \langle \eta(t_1) \eta(t_2) \rangle = t$$
ומכאן: 
$$\langle \eta(t_1) \eta(t_2) \rangle = \delta(t_1 - t_2)$$

\end{proposition}
\begin{remark}
הרעש נקרא "לבן" כיוון שהתמרת פורייה של פונקציית דלתא היא קבוע — כלומר לכל תדר יש עוצמה שווה.

\end{remark}
\section{משוואת לנג'וון}

\begin{definition}[תנועה בראונית]
תנועה אקראית של חלקיק קטן המוקף בהרבה חלקיקים קטנים יותר אשר מתנגשים בו.

\end{definition}
\begin{remark}
היסטורית משומש לתאר מערכות עם חליקים קטנים בקוטר של בערך מיקרון בתוך נוזל(אשר לפי המודל האטומי מורכב מחלקיקים). למעשה התחזיות של תנועה בראינית עיששו את המודל החלקיקי.

\end{remark}
\begin{proposition}
ניתן לחלק את הכוח של תנועה בראונית לשתי רכיבית
- כוח חיכוח - כוח הפועל על החלקיק על ידי בחלקיקים בסביבה שלו. באופן כללי פרופרציונאלי למהירות של החלקיק \(-\gamma v(t)\).
- כוח אקראי - מסומן על ידי \(\eta(t)\).  זה מושפע מהאפקט של 

\end{proposition}
\begin{proposition}[משוואת לנג'וון]
משוואת לנג'וון של חלקיק חופשי אשר נמצא באיזשהו נוזל המכיל החלקיקים קטנים יותר  נתון על ידי:
$$ m\dot{v} = -\gamma v + g\eta(t) $$

\end{proposition}
\begin{proof}
מהחוק השני של ניוטון ניתן לכתוב:
$$ m\frac{dv_t}{dt} = F_{friction} + F_{random} $$
כעת נסתכל על המקרה שנתון גם פוטנציאל חיצוני מהצורה:
$$ F_{pot}(x) = -\frac{dU(x)}{dx} $$
עבור פוטנציאל שקרוב למינימום שלו ניתן לקרב את הכוח לאוסצילטור הרמוני כאשר \(F_{\text{pot}}=-Kx\) ו-\(K=U''\left( x_{\text{min}} \right)\). נוסיף את המשוואה למשוואת לנג'וון ונקבל:
$$ \begin{cases} \dot{x} = v \\ m\dot{v} = -\gamma v - F_{pot}(x) + g\eta(t) \end{cases} $$

\end{proof}
אם נסתכל על המקרה הפרטי של חלקיק מרונסן(overdamped) ניתן לפשט את המשוואה, נניח \(m\ddot{x} \approx 0\) ונקבל:
$$ \gamma\dot{x} = -Kx + g\eta(t) $$
סידור מחדש נותן את משוואת Smoluchowski:
$$ \dot{x} = -\frac{K}{\gamma}x + \frac{g}{\gamma}\eta(t) $$

\begin{proposition}
הפתרון של משוואת לנג'וון נתון על ידי:
$$v\left(t\right)=e^{-\frac t{\tau_{B}}}v\left(0\right)+\frac g m\int\limits_{0}^{t}d s e^{-\frac{t-s}{\tau_{B}}}\eta\left(s\right)$$
כאשר קבוע הדעיכה של הבעיה בזמן הוא \(\tau_{B}=\frac{m}{\gamma}\).

\end{proposition}
\begin{corollary}
התוחלת תהיה:
$$\left\langle v\left(t\right)\right\rangle=e^{-\frac t{\tau_{B}}}v\left(0\right)+\frac g m\int\limits_{0}^{t}d s e^{-\frac{t-s}{\tau_{B}}}\left\langle\eta\left(s\right)\right\rangle$$
כאשר בפרט מתאפסת ב-\(t\to 0\).

\end{corollary}
\begin{proposition}[קורולציה של המהירות]
הקורלוציה נתונה על ידי:
$$\left\langle v\left(t_{1}\right)v\left(t_{2}\right)\right\rangle=\left(v^{2}\left(0\right)-\frac{g^{2}\tau_{B}}{2m^{2}}\right)e^{-\frac{t_{1}+t_{2}}{\tau_{B}}}+\frac{g^{2}\tau_{B}}{2m^{2}}e^{-\lvert t_{1}-t_{2} \rvert }$$

\end{proposition}
\begin{proof}
$$\left\langle v\left(t_{1}\right)v\left(t_{2}\right)\right\rangle=e^{-{\frac{t_{1}+t_{2}}{\tau_{B}}}}v^{2}\left(0\right)+{\frac{g^{2}}{m^{2}}}\int\limits_{0}^{t_{1}}d s_{1}\int\limits_{0}^{t_{2}}d s_{2}e^{-{\frac{t_{1}+t_{2}-s_{1}-s_{2}}{\tau_{B}}}}\underbrace{\left\langle\eta\left(s_{1}\right)\eta\left(s_{2}\right)\right\rangle}_{\delta\left(s_{1}-s_{2}\right)}$$
כאשר עבור \(t_{2}<t_{1}\) או \(t_{2}<t_{1}\) אין תרומה כיוון שהפונקציית דלתא מאפסת באיזור שאין חפיפה. לכן נסמן \(t_{\min}=\min\{ t_{1},t_{2} \}\) ונקבל:
\begin{gather*}\left\langle v\left(t_{1}\right)v\left(t_{2}\right)\right\rangle=e^{-{\frac{t_{1}+t_{2}}{T_{B}}}}v^{2}\left(0\right)+{\frac{g^{2}}{m^{2}}}\int\limits_{0}^{t_{\mathrm{min}}}\!d s_{1}\int\limits_{0}^{t_{\mathrm{min}}}\!d s_{2}e^{-{\frac{t_{1}+t_{2}-s_{1}-s_{2}}{T_{B}}}}\delta\left(s_{1}-s_{2}\right) = \\=e^{-\frac{t_{1}+t_{2}}{\tau_{B}}}v^{2}\left(0\right)+\frac{g^{2}}{m^{2}}\int\limits_{0}^{t_{\mathrm{min}}}d s e^{-\frac{t_{1}+t_{2}-2s}{\tau_{B}}}= \\=\left(v^{2}\left(0\right)-\frac{g^{2}\tau_{B}}{2m^{2}}\right)e^{-\frac{t_{1}+t_{2}}{\tau_{B}}}+\frac{g^{2}\tau_{B}}{2m^{2}}e^{\frac{2t_{\mathrm{min}}-t_{1}-t_{2}}{\tau_{B}}}
\end{gather*}
כאשר הזמן במערכיך יהיה:
$$2t_{\mathrm{min}}-t_{1}-t_{2}=t_{\min }-\left( t_{1}+t_{2}-t_{\min } \right)=t_{\mathrm{min}}-t_{\mathrm{max}}=-\left|t_{1}-t_{2}\right|$$

\end{proof}
\begin{corollary}
בזמנים ארוכים:
$$\langle v(t_{1})v(t_{2}) \rangle \xrightarrow{t_{1},t_{2}\to \infty}\frac{g^{2}\tau_{B}}{2m^{2}}e^{-\frac{\left|t_{1}-t_{2}\right|}{\tau_{B}}}$$
ובפרט עבור \(t_{1}=t_{2}\equiv t\to \infty\) נקבל:
$$\left\langle v^{2}\left(t\right)\right\rangle=\frac{g^{2}\tau_{B}}{2m^{2}}$$

\end{corollary}
\begin{corollary}
כדי שיתאים לחוק החלוקה השווה נקבל בשיווי משקל:
$$m\left\langle v^{2}\left(t\right)\right\rangle=k_{B}T=\frac{1}{\beta}\implies g^{2}={\frac{2m}{\beta\tau_{B}}}={\frac{2\gamma}{\beta}}$$
ואכן מתקיים:
$$\left\langle{\frac{1}{2}}m v^{2}\right\rangle={\frac{1}{2}}k_{B}T$$

\end{corollary}
\begin{remark}
ניתן לראות כי הגודל של הרעש הגאוסי במשוואה(אשר מייצג על ידי גודל המקדם שלו \(g\)) גדל עם הטמפרטורה.

\end{remark}
\section{פיתוח משוואת הדיפוזיה}

נניח כי יש לנו תנועה בראונית בבור פוטנציאל. נגדיר:
$$\Delta x\left(t\right)\equiv x\left(t\right)-x\left(0\right)=\int_{0}^{t}v\left(t^{\prime}\right)d t^{\prime}$$
ונקבל:
$$\left\langle\Delta x\left(t_{1}\right)\Delta x\left(t_{2}\right)\right\rangle=\left(v^{2}\left(0\right)-g^{2}\frac{\tau_{B}}{2m^{2}}\right)\tau_{B}^{2}\left(e^{-\frac{t_{1}}{t_{B}}}-1\right)\left(e^{-\frac{t_{2}}{t_{B}}}-1\right)+\frac{g^{2}\tau_{B}}{2m^{2}}\int\limits_{0}^{t_{1}}d t_{1}^{\prime}\int\limits_{0}^{t_{2}}d t_{2}^{\prime}e^{\frac{-\left|t_{1}^{\prime}-t_{2}^{\prime}\right|}{t_{B}}}\,d t_{1}^{\prime}\int d t_{2}^{\prime}e^{\frac{-\left|t_{2}^{\prime}-t_{2}^{\prime}\right|}{t_{B}}}$$
מה שמאפיין את התנועה הבראונית תהיה השונות. השונות תהיה:
$$\left\langle\Delta x^{2}\left(t\right)\right\rangle=\left(v^{2}\left(0\right)-g^{2}\frac{\tau_{B}}{2m^{2}}\right)\tau_{B}^{2}\left(e^{-\frac{t}{\tau_{B}}}-1\right)^{2}+\frac{g^{2}\tau_{B}}{2m^{2}}\underset{I(t)}{\underbrace{\int_{0}^{t}d t_{1}^{\prime}\int_{0}^{t}d t_{2}^{\prime}e^{\frac{-\left|t_{1}^{\prime}-t_{2}^{\prime}\right|}{\tau_{B}}}}}$$

\begin{lemma}
$$I(t)=2\tau_{B}\left(t+\tau_{B}\left(e^{-{\frac{t}{\tau_{B}}}}-1\right)\right)$$

\end{lemma}
\begin{proof}
$$I\left(t\right)=\int\limits_{0}^{t}{d t_{1}^{\prime}}\int\limits_{0}^{t_{1}^{\prime}}{d t_{2}^{\prime}}e^{-\frac{t_{1}^{\prime}-t_{2}^{\prime}}{\tau_{B}}}+\int\limits_{0}^{t}{d t_{2}^{\prime}}\int\limits_{0}^{t_{2}^{\prime}}{d t_{1}^{\prime}}e^{-\frac{t_{2}^{\prime}-t_{1}^{\prime}}{\tau_{B}}}=2\tau_{B}\left(t+\tau_{B}\left(e^{-{\frac{t}{\tau_{B}}}}-1\right)\right)$$
כאשר זה למעשה שתי עותקים של אותו אינטרגל בחילופי שמות עבור \(t'_{1},t'_{2}\):
$$=2\int_{0}^{t}d t_{1}^{\prime}\int_{0}^{t_{1}^{\prime}}d t_{2}^{\prime}e^{-\frac{t_{1}^{\prime}-t_{2}^{\prime}}{\tau_{B}}}=2\tau_{B}\int_{0}^{t}d t_{1}^{\prime}e^{-\frac{t_{1}^{\prime}}{\tau_{B}}}\left(e^{-\frac{t_{1}^{\prime}}{\tau_{B}}}-1\right)=2\tau_{B}\left(t+\tau_{B}\left(e^{-{\frac{t}{\tau_{B}}}}-1\right)\right)$$

\end{proof}
ולכן נקבל כי:
$$\left\langle\Delta x^{2}\left(t\right)\right\rangle=\left(v^{2}\left(0\right)-g^{2}\frac{\tau_{B}}{2m^{2}}\right)\tau_{B}^{2}\left(e^{-\frac{t}{\tau_{B}}}-1\right)^{2}+\frac{g^{2}\tau_{B}^{2}}{m^{2}}\left(t+\tau_{B}\left(e^{-\frac{t}{\tau_{B}}}-1\right)\right)$$

כך שעבור \(t\ll \tau_{B}\) נקבל תנועה בליסטית:
$$\left\langle\Delta x^{2}\left(t\right)\right\rangle\approx v^{2}\left(0\right)t^{2}$$
ועבור \(t\gg \tau_{B}\) נקבל תנועה דיפוסיבית:
$$\left\langle\Delta x^{2}\left(t\right)\right\rangle=2\underbrace{\frac{g^{2}\tau_{B}^{2}}{2m^{2}}}_{D}t$$

המקדם של הזמן הוא קבוע הדיפוזיה ולכן נקבל את יחס אינשטיין:
$$g^{2}=\frac{2m}{\beta\tau_{B}}=\frac{2\gamma}{\beta}\implies D=\frac{\tau_{B}}{\beta m}=\frac{1}{\beta\gamma}$$

\section{ממשוואת סמולצ'בסקי}

\begin{proposition}
המשוואה המתקבלת מחלקיק הנמצא בתנועה אקראית קרוב לבור פוטנציאל תהיה מהצורה:
$$m{\ddot{x}}+\gamma{\dot{x}}=-k x+g\eta$$
כאשר זהו מד"ר סטוכסטי מסדר שני.

\end{proposition}
\begin{proof}
עבור חלקיק נמצא בבור פוטנציאל נקבל כי הפוטנציאל הוא בקירוב הרמוני, לכן נקרב את הפוטנציאל על ידי:
$$U\left(x\right)\sim U_{\mathrm{min}}+{\frac{k}{2}}x^{2}\implies F=-k x$$
ונקבל את המשוואת לנג'וון:
$$m{\dot{v}}=-\gamma v+g\eta-k x$$
נציב \(\dot{x}=v\) ונקבל:
$$m{\ddot{x}}+\gamma{\dot{x}}=-k x+g\eta$$

\end{proof}
\begin{definition}[משוואת סמולצ'בסקי]
המשוואה של חלקיק בסמוך לבור פוטנציאל במערכת מאוד מרוסנת:
$$\dot{x}=-{\frac{k}{\gamma}}x+{\frac{g}{\gamma}}\eta$$

\end{definition}
\begin{remark}
משוואת סמולצ'בסקי זהה למשוואת לנג'וון תחת ההחלפות הבאות:
$$v\mapsto x\qquad m\mapsto \gamma \qquad \gamma \mapsto k\qquad g\mapsto g$$

\end{remark}
\section{משאוות לנג'וון כללית}

\begin{proposition}[משוואת לנג'וון בפוטנציאל כללי]
משוואת תנועה סטוכסטית של חלקיק תחת פוטנציאל \(V(x)\), חיכוך ורעש לבן גאוסיאני:
$$m\dot v(t) = -\frac{\partial V}{\partial x}(x(t)) - \gamma v(t) + g\,\eta(t)
$$
כאשר \(\eta(t)\) הוא רעש לבן גאוסיאני המקיים \(\langle \eta(t)\rangle=0,\;\langle \eta(t_1)\eta(t_2)\rangle=\delta(t_1-t_2)\) ו\(\tau_B = \tfrac{m}{\gamma}\).

\end{proposition}
\begin{proposition}[משוואת (Kramers-Chandrasekhar)]
צפיפות ההסתברות \(P(x,p,t)\) בפאזת מיקום \(x\) ותנע \(p=m v\) מקיימת משוואת Fokker-Planck מסוג Kramers:
$$\frac{\partial P}{\partial t} + \frac{p}{m}\frac{\partial P}{\partial x} - \frac{\partial V}{\partial x}\frac{\partial P}{\partial p}= \gamma \frac{\partial}{\partial p}\bigl(p P\bigr) + \frac{g^{2}}{2}\frac{\partial^{2}P}{\partial p^{2}}
$$
כאשר המונחים בצד ימין מתארים חיכוך ודיפוזיה בתנע.

\end{proposition}
\begin{remark}
משוואת Kramers היא הכללה של משוואת לנג'וון לפוטנציאל כללי, ומשמשת לתיאור האבולוציה של צפיפות ההסתברות מחוץ לשיווי משקל.

\end{remark}
\section{משוואת המסטר}

עבור תהליך מרקובי מתקיים לכל \(t_{1}\):
$$P_{v}(v_{2}t_{2}\mid  v_{0}t_{0})= \int P_{v}(v_{2}t_{2}\mid v_{1}t_{1})P(v_{1}t_{1}\mid v_{0}t_{0})\;\mathrm{d}v_{1}$$

\begin{remark}
זה נראה דומה אבל זה שונה מהמשוואה שאנחנו מכירים:
$$P^{t_{1}}(v_{1})=\int P(v_{1}t_{1}\mid v_{0}t_{0})P_{v}^{t_{0}} \;\mathrm{d}v_{0}$$

\end{remark}
\begin{definition}[אינרמנטיים קבועים]
$$P_{V}(v_{1},t+\Delta t\mid  v_{0},t)=T_{\Delta t}(v_{1}\mid v_{0})$$

\end{definition}
\begin{corollary}
ניתן לכתוב את משוואת צ'אפמן קולמוגורוב בצורה הבאה:
$$T_{\Delta t+\Delta t'}(v_{2}\mid v_{0})=\int T_{\Delta t'}(v_{2}\mid v_{1})T_{\Delta t}(v_{1}\mid v_{0}) ;\mathrm{d}v_{1}$$

\end{corollary}
נפתח כטור טיילור סביב \(\Delta t = 0\):
$$T_{\Delta t}(v_{1}\mid v_{0})=A\delta(v_{1}-v_{0})+\Delta t\left[ \frac{\partial}{\partial \Delta t} T_{\Delta t}(v_{1}\mid v_{0})\right]_{\Delta t=0} +O(\Delta t^{2})$$
כאשר \(T_{0}(v_{1}\mid v_{0})=A\delta(v_{1}-v_{0})\). ניתן למצוא את \(A\) על ידי תנאי נרמול:
$$\int T_{\Delta t}\left( v_{1}\mid v_{0} \right) ;\mathrm{d}v_{1}=1\implies A=1-\Delta t a (v_{0})+O\left( \Delta t^{2} \right), \quad \text{where}\;\;  a(v_0) = \int W(v_1 | v_0) dv_1$$
ולאחר פיתוח נוסף נקבל:
$$\frac{\partial }{\partial \Delta t} T_{\Delta t}(v_{2}\mid v_{0})=\int [W(v_{2}\mid v_{1})T_{\Delta t}(v_{1}\mid v_{0})-W(v_{1}\mid v_{2})T_{\Delta t}(v_{2}\mid v_{0})] ;\mathrm{d}x$$

נרצה למצוא משוואה מתארת את ההתפתחות בזמן של צפיפות ההסתברות.
$$\frac{\partial p(x, t)}{\partial t} = \int_{-\infty}^{+\infty} [W_t (x|x' )p(x', t) - W_t (x'|x)p(x, t)] dx'$$

$$P(x, t+\Delta t) = \int dx' P(x, t+\Delta t | x', t) P(x', t)$$$$P(x, t+\Delta t | x', t) = \delta(x-x') + W(x, t | x', t)\Delta t + O(\Delta t^2)$$

\section{תנועה בראונית}

\begin{definition}[תנועה בראונית]
תנועה אקראית של חלקיק קטן המוקף בהרבה חלקיקים קטנים יותר אשר מתנגשים בו.

\end{definition}
\begin{remark}
היסטורית משומש לתאר מערכות עם חליקים קטנים בקוטר של בערך מיקרון בתוך נוזל(אשר לפי המודל האטומי מורכב מחלקיקים). למעשה התחזיות של תנועה בראינית עיששו את המודל החלקיקי.

\end{remark}
\begin{definition}[היררכיית סדרי זמן  ]
הזמנים מוגדרים כך ש-\\
$$\tau_{\text{mol}} \ll \tau_{B} \ll \tau_{\text{diffusion}}$$
כאשר \(\tau_{\text{mol}}\) הוא זמן בין התנגשות מולקולרית, \(\tau_{B}\) הוא הזמן להגיע למהירות הטרמינלית

\end{definition}
\begin{definition}[משוואת לנג'וון  ]
משוואת סטוכסטית לתנועה של חלקיק ב-1D תחת חיכוך ורעש:\\
$$m\dot{v}(t) = -\gamma v(t) + g\,\eta(t)$$\\

כאשר \(\eta(t)\) הוא רעש לבן גאוסיאני, ו-\(\tau_{B} = \frac{m}{\gamma}\) זמן אופייני

\end{definition}
\begin{definition}[רעש לבן גאוסיאני (GWN)  ]
משתנה מקרי \(\eta(t)\) המקיים\\
$$\left\langle  \eta(t)  \right\rangle = 0\left\langle  \eta(t_{1})\qquad  \eta(t_{2})  \right\rangle = \delta(t_{1} - t_{2})$$
ונקרא לבן כי בתורת פורייה הקורולציה היא קבוע  

\end{definition}
\begin{proposition}[פתרון משוואת לנג'וון לתרדמה מתחילת זמן  ]
הפתרון הכללי עבור תנאי התחלה \(v(0)\) הוא:\\
$$v(t) = e^{-t / \tau_{B}}\,v(0) + \frac{g}{m} \int_{0}^{t} e^{-(t - s)/\tau_{B}} \,\eta(s)\,\mathbb{d}s$$

\end{proposition}
\begin{proposition}[תוחלת ומהוות קורלציה של מהירות  ]
  \begin{itemize}
    \item תוחלת:\\
$$\langle v(t) \rangle = 0$$
    \item קורלציית מהירויות:\\
$$\langle v(t_{1}) v(t_{2}) \rangle = \left( v^{2}(0) - \frac{g^{2}\tau_{B}}{2m^{2}} \right) e^{-(t_{1}+t_{2})/\tau_{B}} + \frac{g^{2}\tau_{B}}{2m^{2}}\,e^{-|t_{1} - t_{2}|/\tau_{B}}$$
  \end{itemize}
\end{proposition}
\begin{proposition}[מצב שיווי משקל של מהירות  ]
כאשר \(t \to \infty\), מתקיים\\
$$\langle v^{2}(t) \rangle \to \frac{g^{2}\tau_{B}}{2m^{2}} \equiv \frac{1}{\beta m}$$\\

כאשר \(\beta = \tfrac{1}{k_{B}T}\) לפי עקרון החלוקה השווה:\\
$$\left\langle \tfrac{1}{2} m v^{2} \right\rangle = \tfrac{1}{2} k_{B}T$$

\end{proposition}
\begin{proposition}[קביעת עוצמת הרעש  ]
מעקרון שיווי המשקל:\\
$$g^{2} = \frac{2m}{\beta \tau_{B}} = \frac{2\gamma}{\beta}$$

\end{proposition}
\begin{definition}[תזוזת חלקיק ומיקום  ]
הפרש המיקום מתחילת הזמן:\\
$$\Delta x(t) = x(t) - x(0) = \int_{0}^{t} v(t') \,\mathbb{d}t'$$
בזמנים ארוכים מצפה להתנהגות דיפוזיבית

\end{definition}
\begin{proposition}[קורלציית הזזת חלקיק  ]
לזמנים \(t_{1}, t_{2}\):\\
$$\langle \Delta x(t_{1}) \Delta x(t_{2}) \rangle = \int_{0}^{t_{1}} \mathbb{d}t_{1}' \int_{0}^{t_{2}} \mathbb{d}t_{2}' \langle v(t_{1}') v(t_{2}') \rangle$$
ובחישוב מפורט מניב בביטוי הכולל אקספוננציאליים

\end{proposition}
\begin{proposition}[סטיית ריבועית ממוצעת של מיקום  ]
במקרה \(t_{1}=t_{2}=t\) ו-\(t \to \infty\):\\
$$\langle \Delta x^{2}(t) \rangle = \tau_{B}^{2}\left( v^{2}(0) - \frac{g^{2}\tau_{B}}{2m^{2}} \right)\left(e^{-t/\tau_{B}} - 1\right)^{2} + \frac{g^{2}\tau_{B}^{2}}{m}\left(t + \tau_{B}(e^{-t/\tau_{B}} - 1)\right)$$

\end{proposition}
\begin{proposition}[מצבי יחסיים:  ]
  \begin{itemize}
    \item \textbf{מקרה בליסטי} (\(t \ll \tau_{B}\)):\\
$$\langle \Delta x^{2}(t) \rangle \approx v^{2}(0)\,t^{2}$$
    \item \textbf{מקרה דיפוזי} (\(t \gg \tau_{B}\)):\\
$$\langle \Delta x^{2}(t) \rangle \approx \frac{2g^{2}\tau_{B}^{2}}{2m^{2}}\,t$$
  \end{itemize}
\end{proposition}
\begin{definition}[מקדם הדיפוזיה ויחס אינשטיין  ]
מגדירים:\\
$$D = \frac{g^{2}\tau_{B}^{2}}{2m^{2}}$$\\

ובאמצעות \(g^{2} = \tfrac{2m}{\beta \tau_{B}}\) נקבל:\\
$$D = \frac{\tau_{B}}{\beta m} = \frac{1}{\beta \gamma}$$

\end{definition}
\begin{proposition}[התפלגות שיווי משקל של מהירות  ]
בהסתמך על שאר הקומולנטים נעלמים עבור \(N>2\), מהירות בשיווי משקל היא גאוסיאנית:\\
$$P(v) = \sqrt{\frac{\beta m}{2\pi}} \exp\Bigl(-\tfrac{\beta m}{2} v^{2}\Bigr)$$\\

ומכאן אנרגיה פנימית מפולגת בקטגוריה בולצמן:\\
$$P(E) \propto e^{-\beta E}$$

\end{proposition}
\begin{remark}
היחס \(\tau_{\text{mol}} \ll \tau_{B} \ll \tau_{\text{diffusion}}\) מבטיח שהמשוואת לנג'וון מתאימה לתיאור התאוצה הממוצעת בין התנגשות למעבר דיפוזיבי  

\end{remark}
\begin{remark}
בזמנים קצרים (בליסטיים) התנועה דומיננטית על ידי המהירות ההתחלתית; בזמנים ארוכים (דיפוזיביים) שואפת סטיית הריבועית של המיקום לגדול ליניארית עם הזמן.

\end{remark}
\begin{remark}
יחס אינשטיין \(D = \tfrac{1}{\beta \gamma}\) מקשר בין חיכוך, טמפרטורה ודיפוזיה בסביבה תרמית  

\end{remark}
\begin{remark}
ההתפלגות הגאוסיאנית של מהירות נובעת מאי-תלות קומולנטים מעל ריבעיים, המאפיין התנהגות סטוקסטית לינארית עם רעש גאוסיאני  

\end{remark}
\begin{remark}
גזירת הפתרון והקורלציות נשענת על אינטגרלים של אקספוננציאלים תוך שימוש בקורולציית \(\delta\) של הרעש.

\end{remark}
\begin{proposition}[משוואת לנג'וון בפוטנציאל כללי]
משוואת תנועה סטוכסטית של חלקיק תחת פוטנציאל \(V(x)\), חיכוך ורעש לבן גאוסיאני:
$$m\dot v(t) = -\frac{\partial V}{\partial x}(x(t)) - \gamma v(t) + g\,\eta(t)
$$
כאשר \(\eta(t)\) הוא רעש לבן גאוסיאני המקיים \(\langle \eta(t)\rangle=0,\;\langle \eta(t_1)\eta(t_2)\rangle=\delta(t_1-t_2)\) ו\(\tau_B = \tfrac{m}{\gamma}\).

\end{proposition}
\begin{proposition}[משוואת (Kramers-Chandrasekhar)]
צפיפות ההסתברות \(P(x,p,t)\) בפאזת מיקום \(x\) ותנע \(p=m v\) מקיימת משוואת Fokker-Planck מסוג Kramers:
$$\frac{\partial P}{\partial t} + \frac{p}{m}\frac{\partial P}{\partial x} - \frac{\partial V}{\partial x}\frac{\partial P}{\partial p}= \gamma \frac{\partial}{\partial p}\bigl(p P\bigr) + \frac{g^{2}}{2}\frac{\partial^{2}P}{\partial p^{2}}
$$
כאשר המונחים בצד ימין מתארים חיכוך ודיפוזיה בתנע.

\end{proposition}
\begin{remark}
משוואת Kramers היא הכללה של משוואת לנג'וון לפוטנציאל כללי, ומשמשת לתיאור האבולוציה של צפיפות ההסתברות מחוץ לשיווי משקל.

\end{remark}
\section{אינטגרליים מסלוליים}

\section{מבוא מתמטי}

\begin{definition}[פונקציה יוצרת מומנטיים]
יהי \(X\) משתנה מקרי עם התפלגות הסתברות \(P_{X}\). הפונקציה היוצרת של המומנטים נתונה על ידי:
$$Z\left(J\right)\equiv\left\langle e^{J x}\right\rangle=\int e^{J x}P_{X}\left(x\right)d x$$
כאשר \(J\) נקרא המקור. למעשה זהו התמרת לפלס של \(P_{X}(x)\). 

\end{definition}
\begin{proposition}
המומנט מסדר \(n\) נתון על ידי:
$$\langle X^{n}\rangle=\frac{1}{Z\left[0\right]}\left.\frac{d^{n}}{d J^{n}}\left\{Z\left(J\right)\right\}\right|_{J=0}$$

\end{proposition}
\begin{definition}[פונקציה יוצרת קיומולנטיים]
מוגדר על ידי:
$$W\left(J\right)\equiv\log Z\left(J\right)$$
כך שמקיימת:
$$\left\langle X^{n}\right\rangle_{c}=\left.{\frac{d^{n}}{d J^{n}}}\left\{W\left(J\right)\right\}\right|_{J=0}$$

\end{definition}
\begin{remark}
זה אנלוגי לאנרגיה החופשית.

\end{remark}
\begin{example}[פונקציות יוצרות של גאוסיאן]
יהי
$$P_X(x) = \frac{1}{\sqrt{2\pi\sigma^2}} \exp\Bigl(-\tfrac{1}{2}\bigl(\tfrac{x-x_0}{\sigma}\bigr)^2\Bigr).
$$
חשבנו:
$$Z(J) = \int_{-\infty}^{\infty} \frac{1}{\sqrt{2\pi\sigma^2}} e^{ - \tfrac{1}{2}(\frac{x-x_0}{\sigma})^2 + Jx }\;\mathrm{d}x= e^{ J x_0 + \tfrac{1}{2}J^2\sigma^2 }.
$$
לכן:
$$\left\langle X\right\rangle=\frac{1}{Z\left(0\right)}\left.\frac{d}{d J}Z\left(J\right)\right|_{J=0}=x_{0}$$$$\left\langle X^{2}\right\rangle=\frac{1}{Z\left(0\right)}\left.\frac{d^{2}}{d J^{2}}Z\left(J\right)\right|_{J=0}=\left.\frac{d}{d J}\left[\left(x_{0}+\frac{J\sigma^{2}}{2}\right)e^{J\left(x_{0}+\frac{J\sigma^{2}}{2}\right)}\right]\right|_{J=0}=\sigma^{2}+x_{0}^{2}$$
כאשר השנות של הגאוסיאן אכן \(\sigma^{2}\):
$$\left\langle X^{2}\right\rangle_{c}=\left\langle X^{2}\right\rangle-\left\langle X\right\rangle^{2}=\sigma^{2}$$
ניתן גם להראות את זה באמצעות \(W(J)\). נקבל:
$$W\left(J\right)=\log Z\left(J\right)=J x_{0}+{\frac{J^{2}\sigma^{2}}{2}}$$
ולכן:
$$\langle X\rangle_{c}=\left.\frac{d}{d J}W\left(J\right)\right|_{J=0}=x_{0}$$$$\left\langle X^{2}\right\rangle_{c}=\left.\frac{d^{2}}{d J^{2}}W\left(J\right)\right|_{J=0}=\sigma^{2}$$$$\left\langle X^{n>2}\right\rangle_{c}=\left.\frac{d^{n}}{d J^{n}}W\left(J\right)\right|_{J=0}=0$$

\end{example}
\begin{remark}
קומיולנטים מעל השני מתאפסים מאפיינת התפלגות גאוסיאנית.

\end{remark}
\subsection{טריק כללי עבור גאוסיאנים}

\begin{proposition}
עבור אינטגרלים מהצורה:
$$I\left(J\right)\equiv\int\limits_{-\infty}^{\infty}d x\exp\left[-\frac12x G^{-1}x+J x\right]$$
כאשר \(G\) הוא מספר מתקיים:
$$I(J)=e^{ \frac{1}{2}GJ^{2} }I(0)$$

\end{proposition}
\begin{proof}
הדבר הסטנדרטי לעשות זה להשלים לריבוע:
$$I(J)=\int_{-\infty}^{\infty} \exp \left\{  -\frac{1}{2}(x-GJ) G^{-1}  (x-GJ)-\frac{GJ^{2}}{2} \right\} \, \mathrm{d}x $$
ניתן להגדיר \(a=x-GJ\) ולקבל:
$$I(J)=e^{ \frac{1}{2} GJ^{2} }\int_{-\infty}^{\infty} \exp \left\{  -\frac{1}{2}aG^{-1}  a  \right\} \, \mathrm{d}a =e^{ \frac{1}{2} GJ^{2} }I(J=0)$$

\end{proof}
\begin{proposition}
עבור האינטגרל:
$$I(J_{1},\dots,J_{n})=\int_{-\infty}^{\infty}  \, \mathrm{d}x_{1} \dots \int_{-\infty}^{\infty}  \, \mathrm{d}x_{n}  \exp \left\{  -\frac{1}{2}\sum_{j,k}x_{j}(G^{-1} )_{jk}x_{k} + \sum_{j} J_{j}x_{j}  \right\}$$
כאשר ניתן לבחור את \(G\) כך שסימטרית נקבל:
$$I\left( J_{1},\dots,J_{n} \right)=I\left(J_{1}=0,\ldots,J_{n}=0\right)\exp\left[\frac{1}{2}\sum_{j k}J_{j}G_{j k}J_{k}\right]$$

\end{proposition}
\begin{proof}
נתון האינטגרל:
$$I(J_{1},\dots,J_{n})=\int_{-\infty}^{\infty}  \, \mathrm{d}x_{1} \dots \int_{-\infty}^{\infty}  \, \mathrm{d}x_{n}  \exp \left\{  -\frac{1}{2}\sum_{j,k}x_{j}(G^{-1} )_{jk}x_{k} + \sum_{j} J_{j}x_{j}  \right\}$$
כאשר כיוון ש-\(G\) סימטרית קיים מטריצה \(O\) כך ש-\(O^{T}GO=D=\mathrm{diag}\left( w_{1},\dots,w_{n} \right)\).
נעשה המעבר לבסיס אורתוגונאלית:
$$\mathbf{X}=O\mathbf{V}=\mathbf{V}O^{T}$$
כאשר נשים לב כי כיוון ש-\(O\) אורתוגונאלית היעקוביאן הוא יחידה. לכן:
$$O^{T}G^{-1} O=\text{diag}(w_{1}^{-1} ,\dots w_{n}^{-1} )$$
נגדיר \(\tilde{\mathbf{J}}=\mathbf{J}O\) ונקבל:
$$I(J_{1},\dots,J_{n})=\int_{-\infty}^{\infty}  \, \mathrm{d}v_{1}\dots \int_{-\infty}^{\infty}  \, \mathrm{d}v_{n}\exp \left\{  -\frac{1}{2}\sum_{j} v_{j}w_{j}^{-1} v_{j}+\sum_{j} \tilde{j}_{j}v_{j}  \right\}  $$
כאשר כיוון שהבסיס אורתוגונאלי נקבל \(v_{i}\cdot v_{j}=\delta_{ij}\). כעת:
$$I(J_{1},\dots,J_{n})=I(J_{1}=0,\dots,J_{n}=0)\prod_{j=1}^{n} e^{ \frac{1}{2 \tilde{J}_{j}}w_{j}\tilde{J}_{j}}=I(J_{1}=0,\dots,J_{n}=0)\exp \left\{  \frac{1}{2}\sum_{jn}J_{i}G_{jn}J_{n}  \right\}$$

\end{proof}
\begin{proposition}
בגבול הרצף נקבל אינטגרל מסלולי:
$$I[J]=I[0]\cdot\exp\left({\frac{1}{2}}\int_{0}^{T}\int_{0}^{T}J(t)G(t,s)J(s)\,\mathrm{d}t\,\mathrm{d}s\right)$$

\end{proposition}
\begin{proof}
עבור זמן כולל \(T\) נגדיר צעד זמן בתור \(\Delta t=\frac{T}{n}\) ונקודות ביניים \(t_{j}=j\Delta t\).  כמו כן נסמן \(J_{j}=J(t_{j})\Delta t\) ו-\(G_{jk}=G(t_{j},t_{k})\) כאשר זהו גרעין רציף. זה נותן:
$$\sum_{j,k}J_{j}G_{j k}J_{k}=\sum_{j,k}J(t_{j})\Delta t\cdot G(t_{j},t_{k})\cdot J(t_{k})\Delta t=\Delta t^{2}\sum_{j,k=1}^{n}J(t_{j})G(t_{j},t_{k})J(t_{k})$$
כך שנקבל:
$$I(J_{1},\ldots,J_{n})=I(0)\cdot\exp\left[{\frac{1}{2}}\Delta t^{2}\sum_{j,k=1}^{n}J(t_{j})G(t_{j},t_{k})J(t_{k})\right]$$
עד כה זה היה רק סימונים. כעת ניקח את הגבול \(n\to \infty\) ו-\(\Delta t\to 0\). נשים לב כי \(\sum_{j,k}\Delta t^{2}\to\int_{0}^{T} \int_{0}^{T}  \, dt \, ds\) ולכן:
$$\sum_{j,k=1}^{n}J(t_{j})G(t_{j},t_{k})J(t_{k})\Delta t^{2}\to\int_{0}^{T}\int_{0}^{T}J(t)G(t,s)J(s)\,\mathrm{d}t\mathrm{d}s$$
ולכן בגבול נקבל:
$$I[J]=\operatorname*{lim}_{n\to\infty}I(J_{1},\ldots,J_{n})=I[0]\cdot\exp\left[{\frac{1}{2}}\int_{0}^{T}\int_{0}^{T}J(t)G(t,s)J(s)\,\mathrm{d}t\mathrm{d}s\right]$$

\end{proof}
\begin{remark}
במקרה הזה נקבל כי \(G\) היא פונקציית גרין.

\end{remark}
\begin{symbolize}[Path integral measure]
$$\int Dx(t):= \lim_{ n \to \infty } \int_{-\infty}^{\infty}   \, dx_{1}\dots \int_{-\infty}^{\infty}  \, dx_{n} $$
כך שנקבל כי אם \(Z\) הוא פונקציואל לינארי גאוסי מוגדר על ידי:
$$Z[J]:=\int{\mathcal{D}}x(t)\,\exp\left(-{\frac{1}{2}}\int x G^{-1}x+\int J x\right)=Z[0]\cdot\exp\left({\frac{1}{2}}\int_{0}^{T}\!\!\int_{0}^{T}J(t)G(t,s)J(s)\,\mathrm{d}t\,\mathrm{d}s\right)$$

\end{symbolize}
\begin{proposition}
המומנטים של הפונקציונאל הלינארי הגאוסי \(Z[J(t)]\) יהיו:
$$\langle x(t) \rangle =0\qquad \langle x(t_{1},t_{2}) \rangle =G(t_{1},t_{2})\qquad \left\langle  x(t_{1})x(t_{2})\dots x(t_{n})  \right\rangle _{c}=0$$

\end{proposition}
\begin{proof}
כדי למצוא את המומנטים של הפונקציונאל הלינארי הגאוסי \(Z[J(t)]\) נשתמש בפונקצייה יוצרת מומנטים:
$$\langle x(t) \rangle =\frac{1}{Z[0]} \frac{\delta}{\delta J}Z[J]|_{J=0}=0$$
ובאופן דומה:
$$\langle x(t_{1})x(t_{2}) \rangle =\frac{\delta}{\delta J(t_{1})} \frac{\delta}{\delta J(t_{2})}\log Z[J]|_{J=0}=\frac{\partial}{\partial J\left(t_{1}\right)}\frac{\partial}{\partial J\left(t_{2}\right)}\;\frac{1}{2}\int_{0}^{T}d t\int_{0}^{T}d s\left[J\left(t\right)\left(t,s\right)J\left(s\right)\right]\Biggr|_{J=0}=G(t_{1},t_{2})$$

\end{proof}
\section{פתרון מדר סטוכסטי עם אינטגרל מסלולי}

\begin{definition}[הפעולה של רעש לבן גאוסי]
$$S_{\eta}\equiv \frac{1}{2}\int_{0}^{T} \eta^{2}(t) \, \mathrm{d}t $$

\end{definition}
\begin{corollary}
$$Z_{\eta}\left[ J_{\eta }(t) \right]=\int D\eta(t)e^{ -S_{\eta} }\exp \left\{  \int_{0}^{T} J_{\eta}(t)\eta(t) \, dx   \right\}  $$

\end{corollary}
נניח כי:
$${\dot{v}}\left(t\right)=f\left(v\left(t\right),t\right)+g\left(v\left(t\right),t\right)\eta\left(t\right)$$
כדי לייצר אינטגרל מסלולי נסתכל על כל האפשרויות של \(\eta(t)\) עם משקל \(\exp\left( -S_{\eta} \right)\) כאשר \(v(t)\) משומש כאילוץ. כלומר:
$$Z_{v}=\int _{v(t_{0})=v_{0}}Dv(t)P_{v}[v(t)]  \qquad P_{v}[v(t)]=\int D\eta(t)\; \delta\left( \dot{v}-f-g\eta \right)e^{ -S_{\eta} } $$
ואם נשמתמש בתצוגה הספקטרלית של \(\delta\) כך שלכל \(a(t)\) מתקיים:
$$\delta[a(t)]=\int D\lambda(t)\exp\left( -\int_{0}^{T} \lambda(t)a(t) \, dt  \right)$$
ונקבל:

פיתוח להשלים
\textbf{טענה}$$P_{v}\left[v\left(t\right)\right]=Z_{\eta}\left[J_{\eta}=0\right]Z_{\lambda}\left[J_{\lambda}=0\right]\cdot\exp\left(\int\limits_{0}^{T}\frac{\left({\dot{v}}-f\right)^{2}}{2g^{2}}d t\right)$$

\begin{proof}
להשלים

\end{proof}
\begin{example}
עבור המקרה הספציפי של תנועה בראונית נקבל:
$$P_{v}\left[v\left(t\right)\right]\sim\exp\left\{-\int\limits_{0}^{T}d t\frac{\left(\dot{v}-\frac{\gamma v}{m}\right)^{2}}{2\gamma}\beta m^{2}\right\}$$
ואם נרצה לדמות את המערכת בזמנים ארוכים נניח \(\dot{v}=0\) וניקח את הזמן האופייני להיות הזמן ההאטה של המערכת \(T=\tau_{B}=\frac{m}{\gamma}\) ונקבל:
$$P_{v}\left[v\left(t\right)\right]\sim\exp\left\{-{\frac{1}{2}}\beta m v^{2}\right\}=P_{\mathrm{Maxwell}}\left(v\right)$$

\end{example}
\begin{summary}
  \begin{itemize}
    \item עבור משוואה דיפרנציאלית מהצורה:
$${\dot{v}}(t)=f(v(t),t)+g(v(t),t)\eta(t)$$
ניתן למצוא את ההתפלגות הסתברות על ידי:
$$P_{v}[v(t)]=Z_{\eta}[J_{\eta}=0]Z_{\lambda}[J_{\lambda}=0]\cdot\exp\left(-\int_{0}^{T}{\frac{(\dot{v}-f)^{2}}{2g^{2}}}\mathrm{d}t\right)$$
    \item הגורם של האקספוננט נובע מכך שההסתברות של המסלול \(\eta(t)\) יהיה:
$$P[\eta(t)]\propto\exp\left(-\frac{1}{2}\int_{0}^{T}\eta(t)^{2}\mathrm{d}t\right)$$
וכן אם נבודד את \(\eta(t)\):
$$\eta(t)={\frac{\dot{v}(t)-f(t,v)}{g(t,v)}}$$
נציב ונקבל:
$$P_{v}[v(t)]\propto\exp\left(-\frac{1}{2}\int_{0}^{T}\frac{(\dot{v}(t)-f(t,v))^{2}}{g^{2}(t,v)}\,\mathrm{d}t\right)$$
    \item הקבועים \(Z_{\eta}\left[ J_{\eta}=0 \right],Z_{\lambda}\left[ J_{\lambda}=0 \right]\) הם קבועי נרמול(או פונקציונאלים יוצרים). באופן כללי:
$$Z_{\eta}[J_{\eta}]=\int{\mathcal{D}}\eta\,\exp\left(-{\frac{1}{2}}\int\eta^{2}(t)\mathrm{d}t+\int J_{\eta}(t)\eta(t)\mathrm{d}t\right)$$
כאשר אם אין מקור - \(J_{\eta}=0\) נקבל:
$$Z_{\eta}[0]=\int{\mathcal{D}}\eta\,\exp\left(-{\frac{1}{2}}\int\eta^{2}(t)\mathrm{d}t\right)$$
כאשר זה לא תלוי ב-\(v(t)\). זה מבטיח כי ההסתברות הכוללת היא 1.
  \end{itemize}
\end{summary}
\section{משוואת פוקר פלאנק}

עבור משוואת לנג'וון מהצורה:
$$\gamma\frac{d x}{d t}=F\left(x\right)+\eta\left(t\right)$$
כאשר \(\eta(t)\) רעש לבן המקיים:
$$\left\langle\eta\left(t\right)\right\rangle=0\qquad\left\langle\eta\left(t\right)\eta\left(t^{\prime}\right)\right\rangle=\Lambda\delta\left(t-t^{\prime}\right)$$
משוואת הפוקר פלאנק המתאימה תהיה:
$$\frac{\partial}{\partial t}P\left(x,t\right)=\frac{1}{2}\frac{\Lambda}{\gamma^{2}}\frac{\partial^{2}}{\partial x^{2}}P\left(x,t\right)-\frac{1}{\gamma}\frac{\partial}{\partial x}\left[P\left(x,t\right)F\left(x\right)\right]$$

עבור \(N\) דרגות חופש נקבל:
$$\gamma_{i}\frac{d x_{i}}{d t}=F_{i}\left(\vec{x}\right)+\eta_{i}\left(t\right)$$
כאשר \(\vec{x}=\left( x_{1},x_{2},\dots,x_{N} \right)\) ומקיים את היחס:
$$\left\langle\eta_{i}\left(t\right)\right\rangle=0\qquad\left\langle\eta_{i}\left(t\right)\eta_{j}\left(t^{\prime}\right)\right\rangle=\Lambda_{i}\delta_{i,j}\delta\left(t-t^{\prime}\right)$$
נקבל את המשוואות:
$$\frac{\partial}{\partial t}P\left(\vec{x},t\right)=\sum_{i=1}^{N}\left(\frac{1}{2}\frac{\Lambda_{i}}{\gamma_{i}^{2}}\frac{\partial^{2}}{\partial x_{i}^{2}}P\left(\vec{x},t\right)-\frac{1}{\gamma_{i}}\frac{\partial}{\partial x_{i}}\left[F_{i}\left(\vec{x}\right)P\left(\vec{x},t\right)\right]\right)$$

\begin{example}
עבור המשוואה \(\gamma{\frac{d x}{d t}}=\eta\left(t\right)\) ללא פוטנציאל חיצוני נקבל:
$${\frac{\partial}{\partial t}}P\left(x,t\right)=D{\frac{\partial^{2}}{\partial x^{2}}}P\left(x,t\right)$$
כאשר אם נגדיר \(D=\frac{1}{2}\frac{\Lambda}{\gamma^{2}}\) הפתרון שלה הוא משוואת החום:
$$P\left(x,t\right)={\frac{1}{\sqrt{4\pi D t}}}e^{-{\frac{(x-x_{0})^{2}}{4D t}}}$$

\end{example}
\begin{proposition}[פתרונות סטציונאריים]
\end{proposition}
\begin{proof}
נתחיל מהמשוואה:
$$\frac{\partial}{\partial t}P\left(x,t\right)=\frac{1}{2}\frac{\Lambda}{\gamma^{2}}\frac{\partial^{2}}{\partial x^{2}}P\left(x,t\right)-\frac{1}{\gamma}\frac{\partial}{\partial x}\left[P\left(x,t\right)F\left(x\right)\right]$$
הפתרונות הסטציונארים מקיימים:
$$0=\frac12\frac{\Lambda}{\gamma^{2}}\frac{d^{2}}{d x^{2}}P\left(x\right)-\frac1{\gamma}\frac d{d x}\left[P\left(x\right)F\left(x\right)\right]$$
נסדר מחדש:
$${\frac{d}{d x}}\left[P\left(x\right)F\left(x\right)\right]={\frac{1}{2}}{\frac{\Lambda}{\gamma}}{\frac{d^{2}}{d x^{2}}}P\left(x\right)$$
ניקח אינטגרל:
$$P\left(x\right)F\left(x\right)=\frac{1}{2}\frac{\Lambda}{\gamma}\frac{d}{d x}P\left(x\right)+C$$
כיוון ש-\(P(x)\xrightarrow{x\to \pm \infty}0\) ו-\(P'(x)\xrightarrow{x\to \pm \infty}0\) נקבל כי הקבוע צריך להתאפס ונקבל:
$$P\left(x\right)F\left(x\right)=\frac{1}{2}\frac{\Lambda}{\gamma}\frac{d P}{d x}$$
זוהי משוואה ספרבילית ולכן:
$$\int F\left(x\right)d x=\frac{1}{2}\frac{\Lambda}{\gamma}\int\frac{1}{P}d P=\frac{1}{2}\frac{\Lambda}{\gamma}\ln\left(Z P\right)$$
נסדר מחדש:
$$P\left(x\right)=\frac{1}{Z}e^{\frac{1}{\Lambda/\left(2\gamma\right)}}\int d x\,F(x)$$
אם הכוח מגיע מפוטנציאל נקבל \(F=-\frac{\partial U}{\partial x}\) ולכן:
$$P\left(x\right)=\frac{1}{Z}e^{-\frac{1}{\Lambda/\left(2\gamma\right)}}U(x)$$
שזה כמובן התפלגות בולצמן תחת הזיהוי:
$${\frac{\Lambda}{2\gamma}}=k_{B}T$$

\end{proof}
\begin{proposition}[ממוצעים על התפלגות]
יהי \(A(x)\) איזשהו גודל שאנחנו מתעניינים בו, כך ש:
$$\left\langle A\left(x\right)\right\rangle=\int d x\,A\left(x\right)P\left(x,t\right)$$
אזי:
$$\frac{\partial}{\partial t}\left\langle A\left(x\right)\right\rangle=\frac{1}{2}\frac{\Lambda}{\gamma^{2}}\left\langle\frac{\partial^{2}A}{\partial x^{2}}\right\rangle+\frac{1}{\gamma}\left\langle\frac{\partial A}{\partial x}F\right\rangle$$

\end{proposition}
\chapter{שאלות}

\section{הגדרות - הגדרות בסיסיות}

?
איזור שמכיל אנרגיה או חומר

מה ההגדרה של סביבה?
?
איזור אשר מבצע אינטרקציה עם המערכת התרמודינמית

מה ההגדרה של מערכת פתוחה?
?
מערכת תרמודינמית אשר מחליפה של חלקיקים עם הסביבה.

מה ההגדרה של מערכת סגורה?
?
מערכת תרמודינמית אשר לא מחליפה חלקיקים עם הסביבה.

מה ההגדרה של מערכת מבודדת?
?
מערכת תרמודינמית אשר לא מושפעת כלל מהסביבה.

מה ההגדרה של גז?
?
אוסף חלקיקים אשר מתאימים את עצמם לצורה של כלי שבה הם נמצאים.

מה ההגדרה של שיווי משקל?
?
בצורה מיקרסוקופית החלקיקים עדיין נעים. אבל בצורה המאקרוסקופי נקבל שהרבה גדלים נשמרים בזמן. גודל שנשמר בזמן למערכת נקרא בשיווי משקל.

מה ההגדרה של לחץ?
?
הכוח הממוצע שפועל ליחידת שטח. מסומן ע"י \(P\) ומתאור ע"י:
$$P= \frac{F}{A}$$
ובעל יחידות של \(\frac{N}{m^2}=\frac{kg}{m\cdot s^2}\equiv Pa\)

מה ההגדרה של צפיפות?
?
המסה ליחידת נפח. פרופורציונאלי לכמות החלקיקים.

מה ההגדרה של מגע תרמי?
?
כאשר יש מעבר של חלקיקים בין המערכות.

מה ההגדרה של חום?
?
האנרגיה המועברת בין המערכות במגע תרמי.

מה ההגדרה של טמפרטורה - זמנית?
?
גודל המאפיין את המערכת ומשתווה בין שתי מערכות המגיעות לשיווי משקל תרמי.

מה אומר משפט חוק האפס של תרמודינמיקה?
?
אם \(B\leftrightarrow C\) באותו טמפרטורה ו- \(B\leftrightarrow A\) באותה טמפרטורה אזי \(A\leftrightarrow C\) באותה טמפרטורה.

מה ההגדרה של 1. מערכת אשר אין מעבר של חומר נקראת מערכת סגורה.?
?

\begin{enumerate}
  \item מערכת שאין בה מעבר של חום נקראת מערכת אדיאבטית. 


  \item מערכת שאין בה שינוי בגודל נקראת מערכת איזוכורית. 


\end{enumerate}
מה ההגדרה של 1. מערכת שבה הלחץ נשאר קבוע נקראת מערכת איזובארית.?
?

\begin{enumerate}
  \item מערכת שבה החום נשאר קבוע נקרא מערכת איזותרמית. 
\end{enumerate}
מה ההגדרה של מול?
?
יחידת מידה חסרת מימדים. שווה למספר אבוגדרו:
$$1\;{\mathrm{mol}}=N_{A}=6.022\times10^{23}$$
זהו מספר האטומים ב-12 גרם של פחמן 12.

מה ההגדרה של קלווין?
?
יחידה למדידה טמפרטורה. מסומנת \(^\circ K\).

מה ההגדרה של משוואת מצב?
?
משוואה המקשרת בין לחץ, נפח, טמפרטורה וכמות החלקיקים. כלומר משוואה מהצורה \(f\left(p,V,T,N \right)=0\)

מה אומר משפט משוואת גז אידיאלי?
?
$$pV = Nk_{B}T$$
כאשר \(k\) זה קבוע בולצמן, \(N\) מתאר את מספר החלקיקים.

מה ההגדרה של גז וואן דר וואלס?
?
מודל אלטרנטיבי המתאר גזים מסויימים. משוואת המצב תהיה:
$$\left(p+a\frac{N^{2}}{V^{2}}\right)\left(V-N b\right)\,=\,N k_{B}T$$
כאשר \(T\) זה טמפרטורה(בקלווין), \(k_{B}\) זה קבוע בולצמן, \(N\) זה מספר החלקיקים, \(V\) הנפח, \(p\) הלחץ ו-\(N\) מספר החלקיקים.

מה ההגדרה של מקדם התפשוטות תרמי לינארי?
?
לנוזל או מוצק יש מקדם התפשטות אורכי \(\alpha\) המוגדר:
$$\alpha\!=\!{\frac{1}{L}}\!\left({\frac{d L}{d T}}\right)$$

מה ההגדרה של מקדם התפשטות תרמי נפחי?
?
בנוזל או למוצק יש מקדם התפשטות נפחי \(\beta\) המוגדר:
$$\beta\!=\!{\frac{1}{V}}\!\left({\frac{d V}{d T}}\right)$$

מה ההגדרה של תכונה אינטנסיבית?
?
תכונה שאינה משתנה אם משכפלים את המערכת. דוגמאות לגדלים אינטנסיביום יהיו טמפרטורה וצפיפות

מה ההגדרה של תכונה אקסטנסיבי?
?
תכונה של המערכת אשר פרופרציונאלי לשיכפול המערכת. כלומר אם נשכפל את מערכת אז נצפה כי גודל אקסטנסיבי יגדל פי 2. דוגמאות זה מסה, נפח ואנטרופיה.

מה ההגדרה של פונקציות מצב?
?
תכונה אשר תלוייה אך ורק במצב של המערכת ולא איך המערכת הגיעה עליה. דוגמאות לפונקציות מצב הם נפח, אנרגיה פנימית, לחץ ואנטרופיה

מה ההגדרה של פונקציית מסלול?
?
תכונה אשר תלוייה באיך התהליך התבצעה, ולא רק על ידי המצב שלה ברגע נתון. דוגמאות לפונקציות מסלול הם עבודה וחום.

מה ההגדרה של דיפרנציאל?
?
אופרטור \(d\) המתאר שינוי קטן בגודל.

מה ההגדרה של דיפרנציאל מדוייק?
?
ביטוי מהצורה:
$$h(x,y)=M(x,y)dx+N(x,y)dy$$
נקרא דיפרציאל מדוייק אם קיים פונקציה \(f=f(x,y)\) כך ש- \(df=h(x,y)\)

מה ההגדרה של דיפרנציאל לא מדוייק?
?
גודל מהצורה \(h(x,y)=M(x,y)dx+N(x,y)dy\) אשר אינו דיפרנציאל מדוייק נקרא דיפרנציאל לא מדוייק.

\section{הגדרות - הגישה הסטטיסטית - צברים}

מה ההגדרה של צפיפות המצבים?
?
$$g(E)= \frac{\Gamma(E)}{V}$$

מה ההגדרה של הצבר הקנוני?
?
מאגר חום עם אנרגיה \(E_{R}\) המחובר למערכת עם אנרגיה \(E_{A}\) כך ש-\(E_{R}\gg E_{A}\) והאנרגיה הכוללת \(E_{0}=E_{A}+E_{R}\) היא קבועה.

מה ההגדרה של ניוון?
?
כמות המיקרו מצבים עם רמת אנרגיה \(E_{i}\). לכרגע נסמן את זה ב-\(g_{i}\).

מה ההגדרה של פונקצייית החלוקה?
?
הפקטור נרמול של ההסתברות:
$$Z=\sum_{i}g_{i}e^{-E_{i}/k_{\mathrm{{B}}}T}$$
כאשר:
$$ P(E_{i})={\frac{1}{Z}}g_{i}e^{-E_{i}/k_{\mathrm{B}}T}$$

מה ההגדרה של מיקרו מצב?
?
ניתן לחלק את המצבים האפשריים של מערכת לאוסף מצבים שווי הסתברות.

מה ההגדרה של מאקרו מצב?
?
קונפיגורציה כלשהי של המערכת. לא בהכרח שוות הסתברות.

מה אומר משפט העקרון היסודי של פיזיקה סטטיסטית?
?
כל אחד מהמיקרו מצבים הם שווי הסתברות.

מה ההגדרה של טמפרטורה?
?
גודל \(T\) המוגדר ע"י המשוואה:
$${\frac{1}{k_{\mathrm{B}}T}}={\frac{\mathrm{d}\ln\Omega}{\mathrm{d}E}}$$
כאשר \(k_{B}\) זה קבוע בולצמן, \(\Omega\) זה מספר המיקרומצבים כתלות באנרגיה, ו-\(E\) זה האנרגיה.

מה ההגדרה של חלקיקים ניתנים להבחנה?
?
אם ניתן להחליף שתי חלקיקים ובלי לשנות את המיקרו מצב נקבל כי החלקיקים לא ניתנים להבחנה. כאשר אם החלפה של שתי חליקיקים תשנה את המיקרו מצב חלקיקים אלו יהיו ניתנים להבחנה

מה ההגדרה של צבר מיקרוקנוני?
?
המערכת המבודדת. זוהי מערכת שבה האנרגיה, מספר החלקיקים והנפח נשארים קבועים.

מה ההגדרה של הפונקציה הצוברת של מספר המצבים?
?
מספר המיקרו מצבים עד האנרגיה \(E\). מסומן ב-\(\Gamma=\Gamma(E)\).

מה ההגדרה של צפיפות המצבים?
?
מספר המיקרו מצבים בין אנרגיה \(E\) לאנרגיה \(E+dE\). מקיים:
$$\Omega\left(N,V,E\right)=\Gamma\left(N,V,E\right)-\Gamma\left(N,V,E-d E\right)\equiv g\left(N,V,E\right)d E$$
כאשר צפיפות המצבים היא פונקציה \(g\) כך שמתקיים:
$$g(E)=\frac{\partial \Gamma}{\partial E} $$

מה ההגדרה של הצבר הגראנד קנוני?
?
מערכת שיכולה להחליף גם אנרגיה וגם חלקיקים עם אמבט. כאשר נניח כי האמבט גדול משמעותית מהמערכת. כיוון שאנו דורשים שיווי משקל נקבל כי הטמפרטורה של המערכת שווה לטמפרטורה של המאגר(\(T=T_{R}\)) וכן הפוטנציאלים הכימים שווים(\(\mu=\mu_{R}\)).

מה ההגדרה של גראנד פוטנציאל?
?
ההתמרת לג'נדר של ההאנרגיה הפנימית עם הפוטנציאל הכימי והטמפרטורה:
$$\Phi[T,\mu]=U-T S-\mu N$$

\section{הגדרות - חוקי תרמודינמיקה}

?
תהליך שבו המערכת יכולה לחזור למצב ההתחלתי ללא השפעה חיצונית

מה ההגדרה של תהליך קווזיסטטי?
?
תהליך שבו בכל שלב המערכת נמצאת בשיווי משקל.

מה אומר משפט חוק ראשון של תרמודינמיקה?
?
האנרגיה נשמרת, כאשר עבודה וחום הם שתי צורות של אנרגיה

מה ההגדרה של קיבול חום?
?
קיבול חום \(C\) זה למעשה כמות האנרגיה חום(\(Q\)) שנדרש כדי לעלות את הטמפרטורה במעלה אחת. מוגדר:
$$C(p,T)=\operatorname*{lim}_{\Delta T\rightarrow0}\frac{\Delta Q}{\Delta T}=\frac{\mathrm{d}Q}{d T}$$

מה ההגדרה של קיבול חום סגולי?
?
קיבול חום ליחידת מסה. כלומר:
$$c=\frac{1}{m}\left( \frac{ đ Q}{d T} \right)$$

מה ההגדרה של קיבול חום בנפח קבוע?
?
אם לא משנים את הנפח, זה למעשה כמות האנרגיה שצריך כדי לעלות את הטמפרטורה ביחידה אחת:
$$C_{V}=\!\frac{\;\bar{}\mkern-7.5mu dQ_{V}}{d T}\!=\!\left(\frac{\partial U}{\partial T}\right)_{V}$$

מה ההגדרה של קיבול חום בלחץ קבוע?
?
אם לא משנים את הלחץ, זה יהיה כמות האנרגיה הנגדרשת כדי לעלות את הטמפרטורה ביחידה אחת.
$$C_{P}={\frac{\mathrm{\;\bar{}\mkern-7.5mu d}Q_{P}}{d T}}$$

מה אומר משפט הניסוח של קלווין של החוק השני של תרמודימיקה?
?
לא ייתכן תהליך שהתוצאה היחידה שלו היא המרה מלאה של חום לעבודה.

מה ההגדרה של אנטרופיה?
?
החום \(\;\bar{}\mkern-7.5mu dQ\) העוברת למערכת ע"י הטמפרטורה. כלומר:
$$dS = \frac{\;\bar{}\mkern-7.5mu dQ}{T}$$

מה ההגדרה של פוטנציאל תרמודינמי?
?
גודל אשר בעזרתו ניתן לתאר את המערכת בצורה מלאה.

מה ההגדרה של המשתנים הטבעיים של פוטנציאל תרמודינמי?
?
המשתנים עבורו הדיפרנציאל יהיה מורכב מדיפרציאלים מדוייקים, אשר כל אחד מהם אינו תלוי במסלול.

מה ההגדרה של אנרגיה פנימית?
?
פוטנציאל  המקיים:
$$\mathrm{d}U=-p\mathrm{d}V+T\mathrm{d}S$$

מה ההגדרה של אנטלפיה?
?
$$H(S,p)=U+PV$$
ולכן:
$$\mathrm{d}H=T\mathrm{d}S-p\mathrm{d}V+p\mathrm{d}V+V\mathrm{d}p=T\mathrm{d}S+V\mathrm{d}p$$

מה ההגדרה של הפוטנציאל החופשי של הלמהולדס?
?
$$F(T,V)=U-TS$$
ולכן:
$$\mathrm{d}F=T\mathrm{d}S-p\mathrm{d}V-T\mathrm{d}S-S\mathrm{d}T=-S\mathrm{d}T-p\mathrm{d}V$$

מה ההגדרה של הפוטנציאל החופשי של גיבס?
?
$$G(T,p)=H-TS$$
ולכן:
$$\mathrm{d}G=T\mathrm{d}S+V\mathrm{d}p-T\mathrm{d}S-S\mathrm{d}T=-S\mathrm{d}T+V\mathrm{d}p$$

מה ההגדרה של זמניות?
?
נסתכל על מערכת שנמצאת בתוך סביבה עם טמפרטורה \(T_{0}\) ולחץ \(p_{0}\). נגדיר את הזמינות בתור סך האנרגיה שניתן להוציא מהמערכת. זה יהיה שווה לביטוי הבא:
$$
A=U+p_{0}V-T_{0}S,$$
כלומר זה למעשה האנרגיה הפנימית שיש לגוף, ועוד האנרגיה שניתן לקבל מעבודה שהלחץ החיצוני יכול להפעיל על הנפח, פחות החום שניתן לקבל מהשינוי באנטרופיה בגלל התפשטות תרמית בנפח.

מה אומר משפט חוק שלישי של תרמודינמיקה?
?
האנטרופיה 0 מוגדרת בתור הגבול של מערכת כאשר הטמפרטורה שואפת ל-0.

\section{הגדרות - פוטנציאל כימי ומעברי פאזה}

מה ההגדרה של פאזה תרמודינמית?
?
דרך לייחד חומר המתחום במרחב(כלומר נמצא ביחד כגוש) ע"י התכונות הכימיות והפיזיות שלו. לדוגמא, צפיפות, מוליכות חשמלית, צורה, סידור פנימי של אטומים ועוד. מצבי הצבירה מוצק נוזל וגז הם דוגמאות לפאזות.

מה ההגדרה של חום כמוס?
?
כמות האנרגיה התרמית הנדרשת כדי להעביר חומר מצב צבירה כאשר הלחץ קבוע.
$${L}=\Delta Q_{\mathrm{rev}}=T_{\mathrm{c}}(S_{2}-S_{1})$$

מה ההגדרה של דיאגרמת פאזה?
?
זה גרף של לחץ כפונקציה של טמפרטורה. כל תחום קשיר מייצג פאזה.

מה ההגדרה של מעבר פאזה מסדר ראשון?
?
כאשר הנפח משתנה, יש קפיצה באנטרופיה וקיים חום כמוס, אבל הפוטנציאל החופשי של גיבס נשאר רציף וזהה. כלומר:
$$G_{1}=G_{2}\qquad V_{1}\neq V_{2}\qquad L\neq 0\qquad S_{1}\neq S_{2}$$

מה ההגדרה של מעבר פאזה מסדר שני?
?
הפוטנציאל החופשי של גיבס נשאר רציף, אך אין חום כמוס, אין קפיצה באנטרופיה ואין שינוי בנפח, אבל הנגזרות הם לא רציפות. כלומר:
$$\left( \frac{\partial S}{\partial T}  \right)_{p},\left( \frac{\partial S}{\partial P}  \right)_{T},\left( \frac{\partial V}{\partial T}  \right)_{P},\left( \frac{\partial V}{\partial P}  \right)_{T}$$
הם לא רציפים. זה אומר כי:
$$C_{p}=T\left( \frac{\partial S}{\partial T}  \right)_{p}\qquad \beta=\frac{1}{V}\left( \frac{\partial V}{\partial T}  \right)_{p}\qquad \kappa=\frac{1}{V}\left( \frac{\partial V}{\partial P}  \right)_{T}$$
הם גדלים לא רציפים.

מה ההגדרה של פוטנציאל כימי?
?
זה יהיה כמה שמשתנה האנרגיה הפנימית שמספר החלקיקים משתנה:
$$\mu\!=\!\left({\frac{\partial U}{\partial N}}\right)_{S,V}$$

\section{הגדרות - פיסיקה סטטיסטית קוונטית}

?
$$z=\mathrm{e}^{\beta\mu}$$

מה ההגדרה של פולילוגוריתם?
?
פונקציה המוגדרת על ידי:
$$\operatorname{Li}_{n}(z)=\sum_{k=1}^{\infty}{\frac{z^{k}}{k^{n}}}$$
כאשר \(z\) זה העיגול יחידה הפתוח במישור המרוכב. ההגדרה על המרחב המרוכב כולו נובעת מההמשכה האנליטית.

מה ההגדרה של גז פרמי?
?
גז של פרמיונים.

מה ההגדרה של אנרגיית פרמי?
?
רמות האנרגיה מתמלאות בגלל עקרון האיסור של פאולי ללא קשר לטמפרטורה. האנרגיה שהפרמיונים מגיעים עליה באפס המוחלט נקראת אנרגיית פרמי. כלומר:
$$E_{F}=\mu(T=0)$$

מה ההגדרה של פונקציית התפלגות פרמי דיראק?
?
פונקציה:
$$f(E)=\frac{1}{\mathrm{e}^{\beta(E-\mu)}+1}$$

מה ההגדרה של פונקציית התפלגות בוז-אינשטיין?
?
$$f(E)=\frac{1}{\mathrm{e}^{\beta(E-\mu)}-1}$$

מה ההגדרה של אופרטור החילוף?
?
עבור שתי חלקיקים זההים אשר נמצאים ב-\(\vec{r}_{1}\) ו-\(\vec{r}_{2}\) בהתאמה ניתן לתאר על ידי פונקציית גל מהצורה \(\psi\left( \vec{r}_{1},\vec{r}_{2} \right)\). אופרטור החילוף יהיה אופרטור \(P\) המקיים:
$$P_{12}\psi\left( \vec{r}_{1},\vec{r}_{2} \right)=\psi\left( \vec{r}_{2},\vec{r}_{1} 
\right)$$

\section{הגדרות - תהליכים ומנועים}

?
מערכת הפועלת בצורה מחזורית הממירה אנרגיה לעבודה.

מה ההגדרה של מנוע קרנו?
?
רכיב אשר מקיים את מחזור קרנו נקרא מנוע קרנו.

מה ההגדרה של יעילות?
?
הייעוד של מנוע זה להפוך חום לעבודה. לכן היעילות שלו תהיה היחס בין כמות החום שהתקבל לעבודה שהוציא. כלומר:
$$\eta = \left\lvert  \frac{W_{out}}{Q_{in}}  \right\rvert = \frac{\lvert Q_{in} \rvert -\lvert Q_{out} \rvert}{\lvert Q_{in} \rvert }=1-\left\lvert  \frac{Q_{out}}{Q_{in}}  \right\rvert  $$

מה ההגדרה של גרף PV?
?
גרף של לחץ כתלות בנפח.

מה ההגדרה של גרף ST?
?
גרף של טמפרטורה כתלות באנטרופיה.

מה ההגדרה של תהליך איזותרמי?
?
תהליך שהטמפרטורה בו לא משתנה. כלומר מתקיים \(\Delta T=0\).

מה ההגדרה של תהליך אדיאבטי?
?
תהליך אשר אין בו זרימה של חום, כלומר מבודד תרמית. זה אומר שמתקיים:
$$\;\bar{}\mkern-7.5mu dQ=0$$

מה ההגדרה של תהליך איזוכורי?
?
תהליך שבו הנפח נשאר קבוע

מה ההגדרה של תהליך איזובארי?
?
תהליך שבו הלחץ נשאר קבוע

\section{חישוב - צפיפות מצבים ופונקציות חלוקה}

?
ראשית נחשב את הפונקצייה הצוברת של צפיפות המצבים:
$$\Gamma(E)=\iint\limits_{\left\{  U(x,p)\leq E  \right\}}\mathrm{d^{3}}x\;\mathrm{d}^{3}p$$
כאשר כיוון שהאנרגיה של חלקיק חופשי יהיה:
$$\Gamma$$

\section{פוטנציאלים תרמודינאמיים}

?
$$d U=T d S-P d V$$

מהו ההגדרה של אנטלפיה? מהם במשתנים הטבעיים ומה הביטוי לדיפרנציאל בעזרתם?
?
$$H=U+PV\qquad d H=T d S+V d P$$

בעזרת יחסי מקסוואל למה שווה \(\left( \frac{\partial T}{\partial P} \right)_{S}\)?
?
$$\left( \frac{\partial T}{\partial P} \right)_{S}=\left( \frac{\partial V}{\partial S}  \right)_{P} $$

מהו ההגדרה של האנרגיה החופשית של הלמהולדס? מהם במשתנים הטבעיים ומה הביטוי לדיפרנציאל בעזרתם?
?
$$F(T,V)=U-TS\qquad dF=-SdT-PdV$$

מהי ההגדרה של האנרגיה החופשית של גיבס? מהם המשנים הטבעיים ומה הביטוי לדיפרנציאל בעזרתם?
$$G(P,T)=H-TS\qquad dG=VdP-SdT$$

מהי משוואת האנרגיה? כלומר ביטוי ל-\(\left( \frac{\partial U}{\partial P} \right)_{T}\) ול-\(\left( \frac{\partial U}{\partial V} \right)_{T}\).
?
\begin{gather*}\left({\frac{\partial U}{\partial P}}\right)_{T}=-\Biggl[T\biggl({\frac{\partial V}{\partial T}}\biggr)_{P}+P\biggl({\frac{\partial V}{\partial P}}\biggr)_{T}\Biggr]  \\\Biggl({\frac{\partial U}{\partial V}}\Biggr)_{T}=T\Biggl({\frac{\partial P}{\partial T}}\Biggr)_{V}-P
\end{gather*}
\end{document}