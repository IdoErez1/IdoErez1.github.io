\documentclass{tstextbook}

\usepackage{amsmath}
\usepackage{amssymb}
\usepackage{graphicx}
\usepackage{hyperref}
\usepackage{xcolor}

\begin{document}

\title{Example Document}
\author{HTML2LaTeX Converter}
\maketitle

\Chapter{מכניקה ניוטונית}

\section{הגדרות בסיסות}

\begin{definition}[מערכת צירים]
נקודה במרחב הנקראת ראשית הצירים ביחד עם 3 כיוונים בלתי לתלויים נקרא מערכת צירים.

\end{definition}
זהו למעשה דרך לתאר מיקום במרחב. לרוב נבחר את המערכת צירים להיות אורתוגונאלים, ונסמן אותו ב-\(x,y,z\) כך ששלושה ימינית - כלומר \(\hat{x} \times \hat{y} = \hat{z}\).

כמובן שיש מספר דרכים להגדיר מערכות צירים.

\begin{definition}[טרנספורמציה של מערכת צירים]
פונקציה המעבירה ממערכת צירים אחת למערכת צירים אחרת.

\end{definition}
\begin{proposition}[טרנספורמציית גלילאו]
\end{proposition}
\begin{definition}[ווקטור ההעתק]
ווקטור התאר מיקום במערת צירים. מסומן בדרך כלל ב-\(\vec{x}\) או \(\vec{r}\).

\end{definition}
\begin{definition}[ווקטור המהירות]
ווקטור המתאר את השינוי של ווקטור ההעתק בזמן:
$$\vec{v}=\frac{\mathrm{d} \vec{r}}{\mathrm{d} t} \equiv \dot{\vec{r}}$$

\end{definition}
\begin{definition}[ווקטור התאוצה]
ווקטור המתאר את השינוי בזמן של ווקטור המהירות:
$$\vec{a}=\dot{\vec{v}}=\frac{\mathrm{d} \vec{v}}{\mathrm{d} t} =\frac{\mathrm{d} ^2\vec{r}}{\mathrm{d} t} \equiv \dot{\vec{r}} $$

\end{definition}
\begin{definition}[מסה]
גודל אשר מכריע כמה "קשה" להזיז גוף. אחת ההנחות היסוד זה שמסה נשמרת לאורך זמן. 

\end{definition}
\begin{definition}[תנע]
גודל ווקטורי אשר יהיה שווה למכפלה של המהירות עם המסה. כלומר:
$$\vec{p}=m\vec{v}$$

\end{definition}
\begin{definition}[כוח]
מדד כמותי לאינטראקציה בין שתי גופים. זה גודל ווקטורי, ולכן כאשר יש שתי כוחות שפועלים על גוף, זה יהיה שקול למקרה שבו יש כוח אחד אשר יהיה סכום הווקטורים.

\end{definition}
\begin{definition}[כוחות חיצוניים]
כוחות הפועלים על המערכת שאנחנו מסתכלים עליה על ידי גורמים חיצוניים. זה בשונה מכוחות פנימים שפועלים בתוך במערכת שלנו.

\end{definition}
\section{חוקי ניוטון}

\begin{proposition}[חוק ראשון של ניוטון]
לגוף אשר לא פועלים עליו כוחות חיצוניים, הגוף יהיה בשיווי משקל ולא יאיץ.

\end{proposition}
\begin{remark}
מערכת כזו נקראת מערכת אינרציאלית, ולמעשה חוק ראשון מגדיר מהי מערכת אינרציאלית.

\end{remark}
\begin{proposition}[חוק שני של ניוטון]
התאוצה של גוף פרופורציונאלית לסכום הווקטורי של הכוחות, ופרופרציונאלית להופכי של המסה. כלומר:
$$\vec{a}=\frac{\sum \vec{F}}{m}\iff \sum \vec{F}=m\vec{a}$$

\end{proposition}
\begin{corollary}[תנאי שקול לחוק השני של ניוטון]
הסכום הווקטורי של הכוחות שווה לשינוי בתנע לזמן. כלומר:
$$\sum\vec{F}=\frac{\mathrm{d} \vec{p}}{\mathrm{d} t} =\frac{\mathrm{d} \left( m\vec{v} \right)}{\mathrm{d}t } $$

\end{corollary}
\begin{proposition}[החוק השלישי של ניוטון]
כאשר שתי גופים מבצעים אינטראקציה, הם מפעילים כוחות אחד על השני אשר שווים בגודלם והפוכים בכיוונם.

\end{proposition}
\begin{definition}[כוח כבידה/משקל]
כוח הפועל מעצם היותיינו תחת שדה כבדה. תחת כבידה אחידה הכוח יהיה:
$$\vec{F}=m\vec{g}$$
כלפי מרכז הכובד. 

\end{definition}
\begin{definition}[כוח נורמלי]
כוח אשר משטח מפעיל על הגוף כדי לשמור על מאזן הכוחות.

\end{definition}
\begin{remark}
חייב להיות קיים כוח כזה אחרת הגוף היה מאיץ לתוך המשטח.

\end{remark}
\begin{definition}[כוח חיכוך קינטי]
כוח הפועל על הגוף אשר פרופרציונאלי לקבוע הנקרא מקדם החיכוך הקינטי והכוח הנורמלי הפועל רק כאשר הגוף נמצא בתנועה. כלומר אם הכוח הנורמלי מסומן ב-\(N\) ומקדם החיכוך הקינטי מסומן ב-\(\mu_{k}\) מתקיים:
$$f_{k}=\mu_{k}N$$

\end{definition}
\begin{definition}[כוח חיכוך סטטי]
כוח הפועל אשר פועל למנועת תנועה.

\end{definition}
\begin{proposition}
הכוח החיכוך הסטטי חסום על ידי המקדם החיכוך הסטטי \(\mu_{s}\) כפול הכוח הנורמלי \(N\). כלומר:
$$f_{s}\leq \max f_{s}=\mu_{s}N$$

\end{proposition}
\begin{proposition}[חוק הוק]
הכוח של קפיץ נתון על ידי:
$$\vec{F}=k\cdot \Delta \vec{x}$$
כאשר \(k\) זה קבוע שתלוי בקפיץ.

\end{proposition}
\section{עבודה ואנרגיה}

\begin{definition}[עבודה של כוח קבוע]
מוגדר על ידי:
$$W=\vec{F}\cdot \vec{s}=\left\lVert  \vec{F}  \right\rVert \cdot \left\lVert  \vec{s}  \right\rVert \cdot \cos\left( \theta \right)$$
כאשר \(\vec{s}\) זה ווקטור ההעתק ו-\(\theta\) זה הזווית בין הכוח להזזה.

\end{definition}
\begin{corollary}
אם הכוח יהיה חיובי כאשר מקסימלי כאשר פועל באותו כיוון כמו השינוי במקום, אפס כאשר פועל במאונך לשינוי במקום, ומינימלי(שלילי) כאשר פועל נגד הכיוון של השני במקום.

\end{corollary}
\begin{definition}[אנרגיה קינטית]
בהנתן מסה \(m\) ומהירות \(v\) נגדיר את האנרגיה הקינטית להיות:
$$K=\frac{1}{2}mv^{2}$$

\end{definition}
\begin{proposition}[משפט עבודה אנרגיה]
$$W_{\text{tot}}=\Delta K$$

\end{proposition}
\begin{proof}
כאשר חלקיק עובר מנקודה \(x_{1}\) ל-\(x_{2}\) עובר שינוי במקום \(s=x_{2}-x_{1}\). נסמן את המהירות בנקודה \(x_{2}\) על ידי \(v_{2}\) ובנקודה \(x_{1}\) על ידי \(v_{1}\). ממשוואות תנועה של תאוצה קבועה נקבל:
$$v_{2}^{2\!}=v_{1}^{2}+\,2a_{x}s\implies a_{x}={\frac{v_{2}^{2}-{v_{1}^{2}}}{2s}}$$
נכפיל את הביטוי ב-\(m\) ונקבל את הכוח:
$$F=m a_{x}=m{\frac{v_{2}^{2}-v_{1}^{2}}{2s}}\implies F s\,=\,{{\frac{1}{2}}}m v_{2}^{2}-{\frac{1}{2}}m v_{1}^{2}$$

\end{proof}
\begin{remark}
משפט זה נותן משמעות לאנרגיה קינטית בעזרת עבודה. זהו כמות העבודה שנדרש לבצע כדי להאיץ גוף למהירות \(v\).

\end{remark}
\begin{proposition}[עבודה של כוח משתנה]
העבודה שנעשת על ידי כוח הפועל על עקומה \(\gamma\) נתונה על ידי האינטגרל המסלולי:
$$W=\int_{\gamma} \vec{F} \cdot \mathrm{d}\vec{\ell}=\int _{\gamma}\lVert F \rVert \cos \theta \;\mathrm{d} \ell=\int _{\gamma}F_{||} \;\mathrm{d} \ell   $$

\end{proposition}
\begin{definition}[הספק]
עבודה ליחידת זמן:
$$P=\frac{\mathrm{d} W}{\mathrm{d} t} $$

\end{definition}
\begin{proposition}[הספק בעזרת כוח]
הספק נתון על ידי:
$$P=\vec{F}\cdot \vec{v}$$

\end{proposition}
\begin{definition}[אנרגיה פוטנציאלית]
אנרגיה שמתקשרת למקום.

\end{definition}
\begin{proposition}[עבודה של כוח כבידה]
תחת כבידה אחידה \(g\) העבודה של הכבידה נתונה על ידי:
$$W_{\mathrm{g}}\,=\,F s\,=\,w\bigl(y_{1}\,-\,y_{2}\bigr)\,=\,m g y_{1}\,-\,m g y_{2}$$

\end{proposition}
\begin{definition}[אנרגיה פוטנציאלית כבידתית]
נתונה על ידי:
$$U_{g}=mgy$$

\end{definition}
\begin{corollary}
$$W_{\text{g}}=-\Delta U_{g}$$

\end{corollary}
\begin{proposition}[שימור אנרגיה מכנית]
ניתן לכתוב בעזרת האנרגיה פוטנציאלית הכבידתית:
$$K_{1}+U_{1,g}=K_{2}+U_{2,g}+W_{\text{other}}$$
כאשר \(W_{\text{other}}\) זה העבודה של כוחות שאינם כבידתיים.

\end{proposition}
\begin{proposition}[עבודה של קפיץ]
העבודה שנעשת על מנת למתוח קפיץ מ-\(x_{1}\) ל-\(x_{2}\) נתונה על ידי:
$$W_{\text{el}}=F_{\text{el}}\cdot \Delta \vec{x}=\frac{1}{2}kx_{2}^{2}-\frac{1}{2}kx_{1}^{2}$$
כאשר העבודה הנעשת על ידי הקפיץ כאשר עובר מ-\(x_{1}\) ל-\(x_{2}\) נתונה על ידי:
$$W_{\text{el}}=\frac{1}{2}kx_{1}^{2}-\frac{1}{2}kx_{2}^{2}$$

\end{proposition}
\begin{definition}[אנרגיה פוטנציאלית אלסטית]
מוגדר על ידי:
$$U_{\text{el}}=\frac{1}{2}kx^{2}$$
כאשר \(x>0\) אם החוט נמתח ו-\(x<0\) אם הקפיץ מקווץ.

\end{definition}
\begin{corollary}
$$W_{\text{el}}=-\Delta U_{\text{el}}$$

\end{corollary}
\begin{definition}[כוח משמר]
כוח שניתן להגדיר עבורו אנרגיה פוטנציאלית. 

\end{definition}
\begin{proposition}[תכונות של העבודה של כוח משמר]
  \begin{enumerate}
    \item ניתן לבטא אותו כהפרש של ערך התחלתי וערך סופי של הסכום של פונקציית אנרגיה קינטית ופוטנציאלית. 


    \item הפיך 


    \item לא תלוי במסלול ורק תלוי בערכי ההתחלה והסיום. 


    \item כאשר הסוף וההתחלה זהה, העבודה היא אפס. 


  \end{enumerate}
\end{proposition}
\begin{corollary}[שימור אנרגיה]
כאשר הכוחות היחידים שפועלים הם כוחות משמרים, כך האנרגיה המכנית:
$$E=K+U$$
היא קבועה.

\end{corollary}
\begin{proposition}[כוח משמר בעזרת הפוטנציאל]
נתון על ידי:
$$F_{x}=-\frac{\partial U}{\partial x} $$
או בצורה ווקטורית:
$$F=-\bar{\nabla} U$$

\end{proposition}
\section{תנע והתנגשויות}

\begin{reminder}[תנע]
מוגדר בתור המכפלה של המסה במהירות. כלומר:
$$\vec{p}=m\vec{v}$$

\end{reminder}
\begin{proposition}[חוק שני במונחי תנע]
$${\textstyle\sum}{\vec{F}}=m{\frac{d{\vec{v}}}{d t}}={\frac{d}{d t}}\left( m{\vec{v}} \right)=\frac{\mathrm{d} \vec{p}}{\mathrm{d} t} $$
כאשר ניתן להכניס את המסה לנגזרת כיוון שקבועה.

\end{proposition}
\begin{definition}[מתקף]
עבור כוח קבוע, המתקף של מערכת מוגדר על ידי המכפלה של הכוח על ידי משך הזמן שבו פועל. כלומר:
$$\vec{J}=\left( \sum\vec{F} \right)(t_{2}-t_{1})=\left( \sum\vec{F} \right)\Delta t$$

\end{definition}
\begin{corollary}
עבור כוח לא קבוע נקבל:
$$\vec{J}=\int _{t_{1}}^{t_{2}} \sum\vec{F} \;\mathrm{d} t $$
ניתן לחלופין לכתוב בעזרת הכוח הממצוע:
$$\vec{J}=\vec{F}_{\text{avg}}(t_{2}-t_{1})$$

\end{corollary}
\begin{proposition}[משפט מתקף-תנע]
$$\vec{J}=\Delta \vec{p}$$

\end{proposition}
\begin{proof}
מספיק להוכיח עבור המקרה בו הכוח קבוע. מתקיים:
$${\textstyle\sum}\vec{F}=\frac{\vec{p}_{2}-\vec{p}_{1}}{t_{2}-t_{1}}\implies \left( {\textstyle\sum}\vec{F} \right)\left( t_{2}\,-\,t_{1} \right)=\vec{p}_{2}-\vec{p}_{1}\implies \vec{J}=\Delta \vec{p}$$

\end{proof}
\begin{proposition}[שימור תנע]
סכום התנעים של המערכת נשאר קבוע

\end{proposition}
\begin{proof}
נובע מיידית מהחוק השלישי של ניוטון. כיוון שמתקיים:
$${\vec{F}}_{B\;\mathrm{on}\;A}+{\vec{F}}_{A\;\mathrm{on}\;B}={\frac{d{\vec{P}}}{d t}}=\mathbf{0}$$
נקבל:
$$\vec{P}=\text{const}$$

\end{proof}
\begin{corollary}
תנע נשמר גם עבור כל רכיב.

\end{corollary}
\begin{definition}[התנגשות אלסטית]
התנגשות בה האנרגיה המכנית נשמרת.

\end{definition}
\begin{proposition}
משוואות שימור האנרגיה והתנע עבור התנגשות חד מימדית יהיו:
\begin{gather*}{\frac{1}{2}}m_{A}{v_{A}}_{1}{}^{2}={\frac{1}{2}}m_{A}{v_{A}}_{2}{}^{2}+{\frac{1}{2}}m_{B}{v_{B}}_{2}{}^{2} \\m_{A}v_{A1x}=\,m_{A}v_{A2x}\,+\,m_{B}v_{B2x}
\end{gather*}

\end{proposition}
\begin{corollary}
מתקיים:

\end{corollary}
\Chapter{חשבון וריאציות}

\section{קורדינטות מוכללות}

\begin{definition}[קורדינטה מוכללת]
אוסף של סקלרים שבעזרתם באפשר להציג מיקום של גוף במרחב.
בדרך כלל מסמנים סקלרים אלו ב-\(q_{1},q_{2},\dots\).

\end{definition}
\begin{remark}
הסיבה העיקרית לשימוש בקואורדינטות מוכללות כיוון שיש הרבה בעיות שיותר נוח לפתור בקואורדינטה אחת מבקואורדינטה אחרת.

\end{remark}
\begin{example}
  \begin{itemize}
    \item בקואורדינטות קרטזיות תלת מימדיות ניתן להציג את המיקום של גוף באמצעות \((x,y,z)\).
    \item במערכת פולארית למשל מסמנים \(\left( r,\theta \right)\)
    \item בקואורדינטות כדוריות מסמנים \(\left( r,\theta,\varphi \right)\)
    \item בקורדינטות גליליות מסמנים \(\left( r,\theta,z \right)\).
  \end{itemize}
\end{example}
\begin{definition}[מהירות מוכללת]
הקצב שינוי של קואורדינטה מוכללת. מסומן \(\dot{q}\) כאשר:
$$\dot{q}=\frac{\partial q}{\partial t}$$

\end{definition}
לעיתים נוח לכתוב את הבעיה במערכת קואורדינטות אחת ואז לעבור למערכת קואורדינטות אחרת:

\begin{theorem}[מעבר קואורדינטות]
אם אנחנו רוצים לעבור ממערכת קרטזית \(\left( x_{1},y_{1},z_{1},\dots,x_{N},y_{N},z_{N} \right)\) למערכת \(\left( q_{1},\dots,q_{3N} \right)\) ניתן לכתוב:
$$\begin{array}{c}{{q_{1}=q_{1}\left( x_{1},\,y_{1},\,z_{1},\,x_{2},\,y_{2},\dots,\,z_{N},\,t \right)}}\\ {{q_{2}=q_{2}\left( x_{1},\,y_{1},\,z_{1},\,x_{2},\,y_{2},\dots,\,z_{N},\,t \right)}}\\ {{\vdots}}\\ {{q_{3N}=q_{3N}\left( x_{1},\,y_{1},\,z_{1},\,x_{2},\,y_{2},\dots,z_{N},\,t \right)}}\end{array}$$
כאשר ניתן לעבור חזרה אם"ם היעקוביאן לא מתאפס. כלומר:
$$\frac{\partial(q_{1},q_{2},\ldots,q_{n})}{\partial\left(x_{1},x_{2},\ldots,x_{n}\right)}=\left|\begin{array}{c c c c}{{\partial q_{1}/\partial x_{1}}}&{{\partial q_{1}/\partial x_{2}}}&{{\ldots}}&{{\partial q_{1}/\partial x_{n}}}\\ {{}}&{{}}&{{\vdots}}&{{}}\\ {{\partial q_{n}/\partial x_{1}}}&{{\partial q_{n}/\partial x_{2}}}&{{\ldots}}&{{\partial q_{n}/\partial x_{n}}}\end{array}\right|\neq0$$

\end{theorem}
\begin{remark}
לרוב אנחנו מניחים שהקואורדינטות המוכללות הם בת"ל, כיוון שאחרת לא קיים מעבר הפוך.

\end{remark}
\begin{corollary}[מעבר מהירויות]
\end{corollary}
\begin{definition}[דרגות חופש]
כמות הקואורדינטות המוכללות הבלתי תלויות שנדרשות לתיאור בעיה, כלומר מיקום כל החלקיקים.

\end{definition}
במערכת קרטזית למשל, כדי לתאר מיקום של כל חלקיק, נדרש 3 קואורדינטות - \(x,y,z\). ולכן מספר דרגות החופש עבור \(N\) חלקיקים יהיה \(3N\).

\begin{definition}[אילוץ הולונומי]
זהו אילוץ בצורה של משוואה על הקואורדינטות המוכללות. כלומר מהצורה:
$$f\left( q_{1},\dots,q_{n},t \right)=0$$

\end{definition}
\begin{theorem}
כל אילוץ הולונומי מוריד דרגת חופש. כלומר עבור בעיה עם \(N\) דרגות חופש ו-\(k\) אילוצים הולונומים, מספר דרגות החופש האפקטיביות בבעיה יהיה \(N-k\).

\end{theorem}
\section{אוילר לגרנג'}

\begin{definition}[פונקציונאל]
זהו פונקציה שמקבלת פונקציה ומחזירה מספר. 

\end{definition}
\begin{example}
  \begin{enumerate}
    \item פונקצינאל שמחזיר את השטח מתחת לגרף: 
$$I[f(x)]=\int_{a}^b |f(x)| \, \mathrm{d}x $$


    \item פונקציונאל שמחשב את אורך העקומה שיוצר: 
$$I[f(x)]=\int_{a}^b \sqrt{ 1+f'(x)^2 } \, \mathrm{d}x $$


  \end{enumerate}
\end{example}
אנחנו נסתכל על פונקצינאלים בצורה אינטגרלית כפונקציה של איברים עד הנגזרת השנייה:
$$I[y(x)]=\int F(y',y,x) \, \mathrm{d}x $$
המטרה שלנו היא למצוא מתי הפונקציונאל משיג ערך אקסטרימאלי. כלומר מתי משיג ערך מינימלי/מקסימלי. 

\begin{theorem}[אויילר לגרנג']
אם פונקציונאל מהצורה \(I[y(x)]=\int_{a}^b F(y',y,x) \, \mathrm{d}x\) משיג ערך אקסטרימלי אז:
$${\frac{\partial F}{\partial y}}={\frac{d}{d x}}\left({\frac{\partial F}{\partial y^{\prime}}}\right)$$

\end{theorem}
\begin{proof}
נניח כי עבור \(y(x)\) הפונקצינאל משיג ערך אקסטרימלי. כעת נסתכל על וריאציה קטנה בערך של הפונקצינאל. כלומר נסתכל על
$$\tilde{y}(x)=y(x)+\varepsilon\eta(x)\implies \tilde{y}'(x)=y'(x)+\varepsilon \eta'(x)$$
כאשר \(\eta\) היא פונקציה המקיימת את תנאי השפה \(\eta(a)=\eta (b)=0\), ו-\(\varepsilon>0\). כלומר עבור \(\tilde{y}\) ערך הפונקציונאל שלנו יהיה:
$$I[y(x)]=\int_{a}^b F\left( \tilde{y}',\tilde{y},x \right) \, \mathrm{d}x=\int_{a}^b F\left( y'(x)+\varepsilon \eta'(x),y(x)+\varepsilon\eta(x),x \right) \, \mathrm{d}x $$
כיוון ש-\(y\) נתון ו-\(\eta\) שרירותי, נקבל כי \(I\) תלוי רק ב-\(\varepsilon\). כלומר זהו פונקציה במשתנה אחד, אשר כיוון שמקבל ערך אקסטרימלי מתקיים \(\frac{d I}{d \varepsilon}=0\). נגזור לפי \(\varepsilon\) ונקבל:
$$\frac{d }{d \varepsilon}\int_{a}^b F\left( y'+\varepsilon \eta',y+\varepsilon\eta,x \right) \, \mathrm{d}x=\int_{a}^b \frac{\partial }{\partial \varepsilon}F\left( y'+\varepsilon \eta',y+\varepsilon\eta,x \right) \, \mathrm{d}x $$
כאשר מכלל השרשרת ניתן לכתוב:
$$\frac{\partial }{\partial \varepsilon}\left( F\left( \tilde{y}',\tilde{y},x \right) \right)=\frac{\partial F}{\partial \tilde{y}}\frac{\partial \tilde{y}}{\partial \varepsilon}+\frac{\partial F}{\partial \tilde{y}'}\frac{\partial \tilde{y}'}{\partial \varepsilon}+\cancelto{ 0 }{ \frac{\partial F}{\partial x}\frac{\partial x}{\partial \varepsilon} }=\frac{\partial F}{\partial \tilde{y}'}\eta'+\frac{\partial F}{\partial y}\eta$$
נציב חזרה באינטגרל:
$$\int_{a}^b \frac{\partial }{\partial \varepsilon}F\left( y'+\varepsilon \eta',y+\varepsilon\eta,x \right) \, \mathrm{d}x=\int _{a}^b \frac{\partial F}{\partial \tilde{y}'}\eta'+\frac{\partial F}{\partial \tilde{y}}\eta \, \mathrm{d}x $$
כאשר מאינטגרציה בחלקים נקבל:
$$\int _{a}^b \frac{\partial F}{\partial \tilde{y}'}\eta' \, \mathrm{d}x =\cancelto{ 0 }{ \frac{\partial F}{\partial \tilde{y}}\eta\bigg|_{a}^b- }\int_{a}^b \frac{d}{dt}\left( \frac{\partial F}{\partial \tilde{y}'} \right)\eta  \, \mathrm{d}x $$
כאשר אם נציב חזרה נקבל:
$$\int _{a}^b \frac{\partial F}{\partial \tilde{y}'}\eta'-\frac{\partial F}{\partial \tilde{y}}\eta \, \mathrm{d}x=\int_{a}^b \left(\frac{\partial F}{\partial \tilde{y}}-\frac{d}{dt}\left( \frac{\partial F}{\partial \tilde{y}'} \right) \right)\eta  \, \mathrm{d}x=\int_{a}^b \left(\frac{\partial F}{\partial y}-\frac{d}{dt}\left( \frac{\partial F}{\partial y'} \right) \right)\eta  \, \mathrm{d}x $$
כאשר במעבר האחרון לקחנו את הגבול שבו \(\varepsilon\to 0\), כאשר בגבול זה \(\tilde{y}\to y\). כמו שאמרנו מקודם ביטוי זה שווה ל-\(0\). וכיוון ששווה ל-\(0\) עבור כל \(\eta\) שרירותי, הביטוי בתוך הסוגריים חייב להתאפס. לכן:
$$\frac{\partial F}{\partial y}-\frac{d}{dt}\left( \frac{\partial F}{\partial y'} \right) =0$$

\end{proof}
\begin{theorem}
אם \(F\) מקבל ערך אקסטרימלי לא תלוי מפורשות ב-\(x\), מתקיים:
$$F-y'\frac{\partial F}{\partial y'}=const$$

\end{theorem}
\begin{proof}
נכפיל את משוואת אוילר לגרנג' ב-\(y'\) ונשתמש בזה ש:
$${\frac{d}{d x}}\left(y^{\prime}{\frac{\partial F}{\partial y^{\prime}}}\right)=y^{\prime}{\frac{d}{d x}}\left({\frac{\partial F}{\partial y^{\prime}}}\right)+y^{\prime\prime}{\frac{\partial F}{\partial y^{\prime}}}$$
ונקבל:
$$y^{\prime}{\frac{\partial F}{\partial y}}+y^{\prime\prime}{\frac{\partial F}{\partial y^{\prime}}}={\frac{d}{d x}}\left(y^{\prime}{\frac{\partial F}{\partial y^{\prime}}}\right).$$
כיוון שאגף שמאל לא תלוי מפורשות ב-\(x\) נקבל כי הוא שווה לנגזרת השלמה של \(F\) לפי \(x\):
$$\frac{dF}{dx}=\frac{\partial F}{\partial y}\frac{\partial y}{\partial x}+\frac{\partial F}{\partial y'}\frac{\partial y'}{\partial x}+\cancelto{ 0 }{ \frac{\partial F}{\partial x} }=y'\frac{\partial F}{\partial y}+y''\frac{\partial F}{\partial x}$$
ולכן:
$$\frac{dF}{dx}=\frac{d}{dx}\left( y'\frac{\partial F}{\partial y'} \right)\implies F-y'\frac{\partial F}{\partial y'}=C$$

\end{proof}
\begin{definition}[משתנים תלויים]
במקרה זה 
$$F=F(y_{1},y_{1}^{\prime},y_{2},y_{2}^{\prime},\ldots,y_{n},y_{n}^{\prime},x)$$
כאשר כל \(y_{i}\) הוא פונקציה של \(x\). כלומר \(y_{i}=y_{i}(x)\), נדרש לפתור \(n\) מערכות משוואות עבור כל משתנה, כלומר:
$$\frac{\partial F}{\partial y_{i}}=\frac{d}{d x}\left(\frac{\partial F}{\partial y_{i}^{\prime}}\right)\qquad i=1,2,\ldots,n$$

\end{definition}
\begin{definition}[משתנים בלתי תלויי]
במקרה זה אין משתנה אחד \(x\) שכל המשתנים פונקציות שלו, אבל יש משתפר משתנים \(x\) ומספר נגזרות חלקיות עבור כל משתנה. נצפה לקבל משוואה דיפרציאלית חלקית. זה יהיה פונקציונאל מהצורה:
$$I=\int\int\dots\int F\left(y,{\frac{\partial y}{\partial x_{1}}},{\frac{\partial y}{\partial x_{2}}},\ldots,{\frac{\partial y}{\partial x_{n}}},x_{1},x_{2},\ldots,x_{n}\right)\,d x_{1}\,d x_{2}\cdot\cdot\cdot d x_{n}$$
והפונקציה האסטרימלית נדרשת לקיים:
$${\frac{\partial F}{\partial y}}=\sum_{i=1}^{n}{\frac{\partial}{\partial x_{i}}}\left({\frac{\partial F}{\partial y_{x_{i}}}}\right)$$

\end{definition}
\section{התמרת לג'נדר}

\begin{definition}[התמרת לג'נדר]
בהנתן פונקצייה \(f=f\left( u_{1},u_{2},\dots,u_{n} \right)\) ניתן להגדיר \(v_{i}=\frac{\partial f}{\partial u_{i}}{}\). כעת התמרת לג'נדר על \(f\) תיתן פונקצייה \(g=g\left( v_{1},\dots,v_{n} \right)\) הומוגדרת ע"י:
$$g=\sum_{i=1}^{n}u_{i}v_{i}-f$$

\end{definition}
במבט ראשוני ניתן לחשוב כי \(g\) היא פונקציה גם של \(u\) וגם של \(v\). נראה עכשיו שזה לא המקרה.

\begin{theorem}
כל \(u_i\) שווה לנגזרת של \(g\)(שאינה תלויה ב-\(u\)) לפי \(v_{i}\). כלומר:
 $$u_{i}=\frac{\partial g}{\partial v_{i}}$$

\end{theorem}
\begin{proof}
אם ניקח את הדיפרנציאל של ההתמרת לג'נדר נקבל:
 $$\begin{array}{c}{{d g=\displaystyle\sum_{i=1}^{n}\left(u_{i}d v_{i}+v_{i}d u_{i}\right)-\sum_{i=1}^{n}\frac{\partial f}{\partial u_{i}}d u_{i},}}\\ {{=\displaystyle\sum_{i=1}^{n}u_{i}d v_{i}+\sum_{i=1}^{n}\left(v_{i}-\frac{\partial f}{\partial u_{i}}\right)d u_{i}.}}\end{array}$$
כאשר משים לב כי בסכימה הימנית מתקבל \(v_{i}-v_{i}=0\). ולכן נקבל סה"כ:
 $$d g=\sum_{i=1}^{n}u_{i}d v_{i}$$
כאשר שינויים קטנים ב-\(g\) נובעים רק משינויים ב-\(dv_{i}\). או לחלופין ניתן לכתוב:
 $$d g={\frac{\partial g}{\partial v_{1}}}d v_{1}+{\frac{\partial g}{\partial v_{2}}}d v_{2}+\cdot\cdot\cdot+{\frac{\partial g}{\partial v_{n}}}d v_{n}=\sum_{i=1}^{n}{\frac{\partial g}{\partial v_{i}}}d v_{i}$$
כאשר מהשוואת המקדמים ניתן לקבל
 $$u_{i}={\frac{\partial g}{\partial v_{i}}}$$
 מכאן ניתן לראות כי אכן פונקציה של \(v\) בלבד.

\end{proof}
\begin{remark}
דבר נוסף שנחמד לשים לב עליו, זה שיש איזושהי סימטריה בין הפונקציה \(f\) להתמרת לג'נדר שלה \(g\):
$$g=\sum_{i=1}^{n}u_{i}v_{i}\,-\,f,\quad{\mathrm{and}}\quad f=\sum_{i=1}^{n}u_{i}v_{i}-g$$
כלומר התמרת לג'נדר של \(g\) תהיה \(f\).

\end{remark}
\section{התמרת לג'נדר של שתי משתנים}

אם מגדרים $$f=f(u_{1},u_{2},\,.\,.\,,u_{n};\,w_{1},\,w_{2},.\,.\,,w_{m}).$$
כאשר אנחנו רוצים לבצע התמרת לג'נדר רק עבור \(\{ u_{i} \}\), נגדיר כמו במקרה החד משתני \(v_{i}=\frac{\partial f}{\partial u_{i}}\) ונקבל:
$$g=\sum_{i=1}^{n}u_{i}v_{i}\,-\,f(u,\,w).$$\textbf{משפט}$$\frac{\partial g}{\partial w_{i}}=-\frac{\partial f}{\partial w_{i}}$$

\begin{proof}
כאשר הדיפרנציאל של \(g\) יהיה:
$$\begin{array}{l}{{d g=\displaystyle\sum_{i=1}^{n}\left(u_{i}d v_{i}+v_{i}d u_{i}\right)-d f,}}\\ {{=\displaystyle\sum_{i=1}^{n}\left(u_{i}d v_{i}+v_{i}d u_{i}\right)-\sum_{i=1}^{n}\frac{\partial f}{\partial u_{i}}d u_{i}-\sum_{i=1}^{m}\frac{\partial f}{\partial w_{i}}d w_{i}\,,}}\\ {{=\displaystyle\sum_{i=1}^{n}u_{i}d v_{i}+\sum_{i=1}^{n}\left(v_{i}-\frac{\partial f}{\partial u_{i}}\right)d u_{i}-\sum_{i=1}^{m}\frac{\partial f}{\partial w_{i}}d w_{i}.}}\end{array}$$
ושוב כמו במקרה החד מימדי נשים לב כי \(v_{i}-\frac{\partial f}{\partial u_{i}}=0\) ולכן נקבל:
$$d g=\sum_{i=1}^{n}u_{i}d v_{i}-\sum_{i=1}^{m}\frac{\partial f}{\partial w_{i}}d w_{i}$$
וכיוון ש-\(g=g(v,w)\) ניתן גם לכתוב:
$$d g=\sum_{i=1}^{n}\frac{\partial g}{\partial v_{i}}d v_{i}+\sum_{i=1}^{m}\frac{\partial g}{\partial w_{i}}d w_{i}\,,$$
ולכן מהשוואת המקדמים נקבל:
$${\frac{\partial g}{\partial w_{i}}}=-{\frac{\partial f}{\partial w_{i}}}$$

\end{proof}
\begin{remark}
היתרון התמרת לג'נדר זה שהיא משמרת אינפורמציה. כלומר מאפשרת לעבור משתנים ללא שינוי באינפורמציה של המשוואה. לכן ניתן לקבל ביטוי חדש אשר זהה מבחינת הפיזיקה לביטוי הקודם אך עם משתנה חדש.

\end{remark}
\section{סוגרי פואסון}

\begin{definition}[סוגר פואסון]
מוגדר באופן הבא:
$$\{u,\,v\}_{p,q}\equiv{\frac{\partial u}{\partial q}}{\frac{\partial v}{\partial p}}-{\frac{\partial u}{\partial p}}{\frac{\partial v}{\partial q}}$$
כלומר זהו פעולה שמקבלת שתי פונקציות של \(p,q\) מחזיר פונקציה. אם נכליל למערכת עם \(n\) דרגות חופש נקבל:
$$\{u,v\}=\sum_{i=1}^{n}\left({\frac{\partial u}{\partial q_{i}}}{\frac{\partial v}{\partial p_{i}}}-{\frac{\partial u}{\partial p_{i}}}{\frac{\partial v}{\partial q_{i}}}\right)$$

\end{definition}
\begin{proposition}[תכונות של סוגרי פואסון]
$$\begin{array}{c}{{{\{u,u\}=0}}}\\ {{{\{u,v\}=-\{v,u\}}}}\\ {{{\{a u+b v,\,w\}=a\{u,\,w\}+b\{v,\,w\}}}}\\  {{{\{u v,\,w\}=\{u,\,w\}v+u\{v,\,w\}}}}\end{array}$$\textbf{זהות יעקובי:}$$\{u,\{v,\,w\}\}+\{v,\{w,u\}\}+\{w,\{u,\,v\}\}=0$$

\end{proposition}
\section{סימון לוי צוויטה}

\begin{definition}[סימון לוי צוויטה תלת מימדי]
$$\varepsilon_{i j k}={\left\{\begin{array}{l l}{+1}&{{\mathrm{if}}\;(i,j,k)\;{\mathrm{is}}\;(1,2,3),(2,3,1),\;{\mathrm{or}}\;(3,1,2),}\\ {-1}&{{\mathrm{if}}\;(i,j,k)\;{\mathrm{is}}\;(3,2,1),(1,3,2),\;{\mathrm{or}}\;(2,1,3),}\\ {0}&{{\mathrm{if}}\;i=j,\;{\mathrm{or}}\;j=k,\;{\mathrm{or}}\;k=i}\end{array}\right.}$$

\end{definition}
\begin{proposition}[מכפלה ווקטורית עם לוי צ'יוויטה]
$$({\bf u}\times{\bf v})_{k}=\sum_{i,j=1}^{3}\epsilon_{i j k}u_{i}v_{j}$$

\end{proposition}
\begin{proposition}[דטרמיננטה אם לוי ציוויטה]
$${\left|\begin{array}{l l l}{a_{11}}&{a_{12}}&{a_{13}}\\ {a_{21}}&{a_{22}}&{a_{23}}\\ {a_{31}}&{a_{32}}&{a_{33}}\end{array}\right|}=\sum_{i,j,k=1}^{3}\epsilon_{i j k}a_{1i}a_{2j}a_{3k}$$
כאשר בפועל זה מבוסס על כלל ליבניץ לחישוב דטרמיננטות.

\end{proposition}
\begin{proposition}[מכפלה של לווי ציווטות]
$$\epsilon_{j k i}\epsilon_{j\ell m}=\delta_{k\ell}\delta_{i m}-\delta_{k m}\delta_{i\ell},$$

\end{proposition}
\Chapter{מכניקת לגרנג'}

\section{עקרון המילטון}

\begin{definition}[לגרנג'יאן]
מוגדר בתור \(\mathcal{L} =T-V\) כאשר \(T\) זה האנרגיה הקינטית ו-\(V\) זה האנרגיה הפונטנציאלית.

\end{definition}
זהו תכונה של מערכת בזמן נתון. האנרגיה הקינטית היא פונקציה של המהירויות המוכללות \(\{ \dot{q}_{i} \}\) וזמן \(t\) כאשר האנרגיה הפוטנציאלית היא פונקציה של הקורדטינות המכוללות \(\{ q_{i} \}\) וזמן \(t\). לכן \(\mathcal{L} =f\left( q_1,\dot{q}_{1},\dots,q_{N},\dot{q}_{N},t \right)\).

\begin{definition}[פעולה]
ערך הלגרנג'יאן לאורך מסלול \(\gamma\). כלומר:
$$S=\int _{\gamma}\mathcal{L} \left( q_{i},\dot{q}_{i},t \right)  \, \mathrm{d}x $$
זהו תכונה מסלול של המסלול \(\gamma\), ושל המערכת.

\end{definition}
\begin{theorem}[עקרון המילטון]
המסלול שהמערכת תיקח יהיה המסלול עבור הפעולה אקסטרימלית. 

\end{theorem}
כאשר כפי שאנו יודעים מחשבון וריאציות, אנו יודעים כי מסלול זה יקיים את אויילר לגרנג':
$$\frac{d}{d t}\frac{\partial \mathcal{\mathcal{\mathcal{L}}}}{\partial\dot{q}}-\frac{\partial \mathcal{L}}{\partial q}=0$$
כאשר ב-\(n\) קורדינטות מוכללות נקבל \(n\) מערכות משוואות:
$$\frac{d}{d t}\frac{\partial{\cal L}}{\partial\dot{q}_{i}}-\frac{\partial{\cal L}}{\partial q_{i}}=0\qquad i=1,\dots,n$$

\begin{theorem}[אי יחידות הלגרנג'יאן]
ממספר לגרנג'יאנים שונים ניתן לפתח אותם משוואות תנועה. למעשה, ניתן לבצע את הפעולות הבאות על הלגרנג'יאן בלי לשנות את משוואות התנועה:

  \begin{enumerate}
    \item כפל בסקלר 


    \item הוספה של קבוע 


    \item הוספה של פונקציה שהיא נגזרת שלמה לפי זמן 


  \end{enumerate}
\end{theorem}
\begin{proof}
  \begin{enumerate}
    \item ניתן לחלק את המשוואת אויילר לגרנג' בסקלר ולקבל חזרה את אותה המשוואה. 


    \item כל הקבועים נופלים בגזירה ולכן לא ישפיע על תוצאות המשוואה. 


    \item פונקציה שהיא נגזרת שלמה לפי זמן 


  \end{enumerate}
\end{proof}
\begin{proposition}
אויילר לגרנג' שקול לחוק השני של ניוטון.

\end{proposition}
\begin{proof}
יהי \(L=T-V\) כאשר \(L\) הוא הלגרנג'יאן של המערכת. ניתן לכתוב:
$$L=\frac{1}{2}m\dot{x}^{2}-V(x)$$
כעת כיוון ש-\(\frac{\partial L}{\partial \dot{q}}=m\dot{x}\) ו-\(\frac{\partial L}{\partial q}=-\frac{\partial V}{\partial x}=-\frac{\mathrm{d} V}{\mathrm{d} x}\) נקבל:
\begin{gather*}\frac{\mathrm{d} }{\mathrm{d} t} \left( \frac{\partial L}{\partial \dot{q}}  \right)-\frac{\partial L}{\partial q} =0 \iff m\frac{\mathrm{d} \dot{x}}{\mathrm{d} t} +\frac{\mathrm{d} V}{\mathrm{d} x} =0\iff  \\\iff m\ddot{x}-F(x)=0\iff F(x)=m\ddot{x}
\end{gather*}

\end{proof}
\section{סימטריות}

\begin{definition}[הומוגניות ואיזוטרופיות]
מרחב נקרא \underline{הומוגני} אם הוא נראה אותו דבר מכל מקום, כלומר יש סימטרייה להזזה. ניתן גם להגדיר מרחב \underline{הומוגני זמני} - כלומר סימטרי להזזות בזמן.
מרחב נקרא \underline{איזוטרופי} אם נראה אותו דבר תחת סיבוב, כלומר יש סימטרייה לסיבובים.

\end{definition}
\begin{definition}[תנע מוכלל]
לכל קורדינטה מוכללת ניתן להגדיר תנע מוכלל $$p_{i}={\frac{\partial \mathcal{L}}{\partial{\dot{q}}_{i}}}$$

קורדינטה מוכללת של המערכת שאינה מופיעה בלגרנג'יאן נקראת ציקלית

\end{definition}
\begin{remark}
קורדינטה שאינה מופיעה בלגרנג'יאן אינה משפיעה של משוואות התנועה.

\end{remark}
\begin{theorem}
תנע צמוד של קורדינטה ציקלית הוא שמור(כלומר גודל שנשמר בבעיה)

\end{theorem}
\begin{proof}
נניח \(q_{i}\) קורדינטה ציקלית. מאוילר לגרנג':
$$\frac{d}{dt}\left( \frac{\partial \mathcal{L}  }{\partial \dot{q}_{i}} \right)=\frac{\partial \mathcal{L}  }{\partial q}=0\implies \frac{\partial \mathcal{L}  }{\partial \dot{q}_{i}}=C\implies p_{i}=C$$

\end{proof}
\begin{definition}[פונקציית יעקובי]
$$h=h(q,\dot{q},t)=\sum_{i}\dot{q}_{i}\frac{\partial{\cal L}}{\partial\dot{q}_{i}}-{\cal L}$$

\end{definition}
\begin{theorem}
הנגזרת של הפונקציית יעקובי בזמן היא הנגדית של הנגזרת של הנגרנג'יאן
$$\frac{\partial \mathcal{L} }{\partial t}=-\frac{\partial h}{\partial t}$$

\end{theorem}
\begin{proof}
אנו יודעים כי:
$${\frac{d \mathcal{L}}{d t}}=\sum_{i}{\frac{\partial \cal L}{\partial q_{i}}}{\frac{d q_{i}}{d t}}+\sum_{i}{\frac{\partial \cal L}{\partial{\dot{q}}_{i}}}{\frac{d{\dot{q}}_{i}}{d t}}+{\frac{\partial \cal L}{\partial t}}$$
נשים לב כי מאוילר לגרנג' מתקיים:
$${\frac{d \cal L}{d t}}=\sum_{i}\left[{\dot{q}}_{i}{\frac{d}{d t}}\left({\frac{\partial \cal L}{\partial{\dot{q}}_{i}}}\right)+{\frac{\partial \cal L}{\partial{\dot{q}}_{i}}}{\frac{d{\dot{q}}_{i}}{d t}}\right]+{\frac{\partial \cal L}{\partial t}}$$
כאשר נשים לב כי:
$${\frac{d}{d t}}\sum_{i}{\dot{q}}_{i}{\frac{\partial \cal L}{\partial{\dot{q}}_{i}}}=\sum_{i}{\frac{d}{d t}}\left({\dot{q}}_{i}{\frac{\partial \cal L}{\partial{\dot{q}}_{i}}}\right)=\sum_{i}\left[{\dot{q}}_{i}{\frac{d}{d t}}\left({\frac{\partial \cal L}{\partial{\dot{q}}_{i}}}\right)+{\frac{d{\dot{q}}_{i}}{d t}}{\frac{\partial \cal L}{\partial{\dot{q}}_{i}}}\right]$$
אך הסכימה ניתן להחלפה ע"י \(\frac{d}{d t}\,\sum_{i}\dot{q}_{i}\frac{\partial \cal L}{\partial\dot{q}_{i}}\) ולכן:
$${\frac{d \cal L}{d t}}={\frac{d}{d t}}\sum_{i}{\dot{q}}_{i}{\frac{\partial \cal L}{\partial{\dot{q}}_{i}}}+{\frac{\partial \cal L}{\partial t}}$$
או:
$${\frac{\partial \cal L}{\partial t}}={\frac{d}{d t}}\left(\cal L-\sum_{i}{\dot{q}}_{i}\,{\frac{\partial \cal L}{\partial{\dot{q}}_{i}}}\right)=\frac{dh}{dt}$$

\end{proof}
ולכן אם הלגרנג'יאן לא תלוי בזמן, אז פונקציית יעקובי היא קבועה.

\section{גוף קשיח}

\begin{definition}[גוף קשיח]
גוף אשר אינו יכול להתעוות או להשתנות. המרחק בין כל שתי נקודות בגוף הוא קבוע.

\end{definition}
ההבדל העיקרי כשאנחנו עוסקים בגוף קשיח לעומת גוף נקודתי זה שיש גם אנרגיה מהסיבוב של המערכת.

נסתכל על אוסף חלקיקים עם מסה \(m_{i}\) אשר נמצאים במרחק \(r_{i}\) מנקודת הסיבוב. כיוון שזהו גוף קשיח, כל הלקיקים מסתובבים באותה מהירות זוויתית \(\omega\). נקבל:
$$\omega=\frac{\dot{r}_{i}}{r_{i}}\implies \dot{r}_{i}=\omega r_{i}\implies E_{i}=\frac{1}{2}mr_{i}^2\omega^2$$
ולכן האנרגיה הכוללת תהיה
$$E=\sum_{i}E_{i}=\sum_{i} \frac{1}{2}m_{i}r_{i}^2\omega^2=\frac{1}{2}\omega^2 \sum_{i}m_{i}r_{i}^2 $$
כאשר נסמן את האיבר בסכום ב-\(I\).  זהו גודל שתלוי אך ורק בגאומטריה והצורה של הגוף.

\begin{definition}[מומנט התמד]
כאשר החלקיקים דיסקרטים נגדיר \(I=\sum_{i}m_{i}r_{i}^{2}\). בגבול הרצף נגדיר:
$$I=\int r^2 \, \mathrm{d}m $$

\end{definition}
כאשר נזכור כי במקרה של צפיפות נפחית \(dm=\rho(r)dV\), במקרה של צפיפות משטחית \(dm=\sigma(r)dS\) ומקרה של צפיפות חד מימדית \(dm=\lambda(x)dx\)
המומנט התמד מופיע בכמעט כל הגדלים המעניים שקשורים לגוף קשיח, לדוגמא התנע הזוויתי - \(\vec{L}=I \dot{\vec{\theta}}=I\vec{\omega}\), המומנט - \(\vec{\tau}=I \ddot{\vec{\theta}}=I \dot{\vec{\omega}}\) והגודל שאותנו מעניין במיוחד - האנרגיה הקינטית - \(T =\frac{1}{2}I\omega^2\). הגודל הזה למעשה מתאר את היכולת של הגוף להסתובב. ככל שהוא גדול יותר, נדרש יותר אנרגיה כדי לסובב את הגוף. גודל זה כמובן תלוי בציר הסיבוב. כל עוד אין ציר מקובע. זה יהיה מרכז המסה.

\begin{theorem}[פירוק אנרגיה סביב מרכז מסה]
תמיד ניתן לפרק אנרגיה קינטית של גוף קשיח לאנרגיה של מרכז מסה ולאנרגיה סביב מרכז המסה. כלומר:
$$T=T_{c m}+T_{r o t}=\frac12m v_{c m}^{2}+\frac12I_{c m}{\omega}^{2}$$

\end{theorem}
זה נכון במקרה של סיבוב סביב המרכז מסה, כאשר \(I_{cm}\) זה המומנט התמד סביב המרכז מסה. ניתן לפרק באופן דומה את התנע הזוויתי:
$$\vec{J}=\vec{L}+\vec{S}=M\vec{R}_{c m}\times\vec{v}_{c m}+I_{c m}\vec{\omega}$$
כאשר \(\vec{R}_{cm}\) המיקום של מרכז המסה.
להלן טבלה של מומנטי התמד נפוצים:

\begin{table}[htbp]
  \centering
  \begin{tabular}{|cc|}
    \hline
    גוף & I \\ \hline
    נקודתי & \(mr^2\) \\ \hline
    מוט מהמרכז & \(\frac{1}{12}ML^2\) \\ \hline
    מוט מהקצה & \(\frac{1}{3}ML^2\) \\ \hline
    גליל בציר מרכזי & \(\frac{1}{2}MR^2\) \\ \hline
    גליל עם עובי & \(\frac{1}{2}M(r_1^2+r_2^2)\) \\ \hline
    כדור(מרכז) & \(\frac{2}{5}MR^2\) \\ \hline
    כדור חלול & \(\frac{2}{3}MR^2\) \\ \hline
    דיסקה סביב ציר & \(\frac{1}{4} MR^2\) \\ \hline
    ריבוע & \(\frac{1}{6}MR^2\) \\ \hline
    מלבן סביב הציר & \(\frac{1}{3}Ma^2\) \\ \hline
  \end{tabular}
\end{table}
\begin{theorem}[שטייניר]
אם הסיבוב מתבצע סביב נקודה שהיא לא מרכז המסה, סביב ציר בכיוון \(\hat{n}\) אז מומנט האינרציה יהיה:
$$I_{\hat{n}}=I_{\hat{n},c m}+m d^{2}$$
כאשר \(m\) מרכז מסת הגוף, ו-\(d\) במרחק בין הנקודה לבין מרכז המסה.

\end{theorem}
\section{מומנט התמד בתלת מימד}

במקרה התלת מימדי, לא בהכרח סביב כל ציר הסיבוב היא אותה מהירות זוויתית \(\omega\). ייתכן ו-\(\omega_{x}\neq \omega_{y}\neq \omega_{z}\),  כאשר במקרה זה \(\vec{\omega}=\left( \omega _{x},\omega_{y},\omega_{z} \right)\) יסמן את הסיבוב בכל ציר. המקרה זה נקבל כי:
$$T=T_{c m}+T_{r o t}=\frac{1}{2}M v_{c m}^{2}+\frac{1}{2}\vec{\omega}^{T}{\stackrel{\leftrightarrow}{ I}}_{c m}\vec{\omega}$$

\begin{definition}[טנזור התמד]
זהו מטריצה \(\stackrel{\leftrightarrow}{ I}\) מהמקיימת
$$T_{rot}=\frac{1}{2}\vec{\omega}^t\!\!\stackrel{\leftrightarrow}{I}\!\!\vec{\omega}\qquad \vec{L}=\stackrel{\leftrightarrow}{ I}\!\!\vec{\omega}$$

\end{definition}
במקרה התלת מימדי שאנחנו נשתמש בו, זה יהיה פשוט מטריצה \(3\times 3\).
ניתן להראות כי ערכי מטריצה זו יהיו שווים ל:
$$I_{i j}=\sum_{\alpha}m_{\alpha}\left(r_{\alpha}^{2}\delta_{i j}-r_{\alpha,i}r_{\alpha,j}\right)\to\int d m\left(r^{2}\delta_{i j}-r_{i}r_{j}\right)$$
לדוגמא במקרה הבדיד נקבל:
$$I=\left(\begin{array}{c c c}{{\sum_{\alpha}m_{\alpha}\left(y_{\alpha}^{2}+z_{\alpha}^{2}\right)}}&{{-\sum_{\alpha}m_{\alpha}x_{\alpha}y_{\alpha}}}&{{-\sum_{\alpha}m_{\alpha}x_{\alpha}z_{\alpha}}}\\ {{-\sum_{\alpha}m_{\alpha}x_{\alpha}y_{\alpha}}}&{{\sum_{\alpha}m_{\alpha}\left(x_{\alpha}^{2}+z_{\alpha}^{2}\right)}}&{{-\sum_{\alpha}m_{\alpha}y_{\alpha}z_{\alpha}}}\\ {{-\sum_{\alpha}m_{\alpha}x_{\alpha}z_{\alpha}}}&{{-\sum_{\alpha}m_{\alpha}y_{\alpha}z_{\alpha}}}&{{\sum_{\alpha}m_{\alpha}\left(x_{\alpha}^{2}+y_{\alpha}^{2}\right)}}\end{array}\right)$$
או במקרה הרציף:
$$I=\left(\begin{array}{c c c}{\int (y^2+z^2) \, \mathrm{d}m }&{-\int xy \, \mathrm{d}m }&-\int xz \, \mathrm{d}m \\ -\int xy \, \mathrm{d}m &\int (x^2+z^2) \, \mathrm{d}m &-\int yz \, \mathrm{d}m \\ -\int xz \, \mathrm{d}m &-\int yz \, \mathrm{d}m &\int (x^2+y^2) \, \mathrm{d}m \end{array}\right)$$

כיוון שזוהי מטריצה סימטרית, קיים מערכת צירים שעבורה היא תהיה אלסונית. מערכת זאת נקראת מערכת צירים ראשית. ובמקרה זה אפשר להשתמש בנוסחה הפשוטה \(T=\frac{1}{2}I\omega^2\). כלומר אם אנחנו יודעים שמסתובב סביב ציר, תמיד ניתן לחשב $$I_{\hat{n}}=I_{\hat{n},c m}+m d^{2}$$ ואז להתייחס לבעיה באופן שראינו קודם.

\begin{theorem}[הצירים המאונכים]
עבור גוף דו מימדי, או לפחות גוף שטוח, מתקיים \(I_{x x}+I_{yy}=I_{zz}\).

\end{theorem}
\section{הבעיה הדו גופית}

הבעיה הדו גופית היא בעיה של 2 גופים שמקיימים את הדרישות הבאות:

\begin{enumerate}
  \item פועל כוח "מרכזי" על הציר המחבר ביניהם 


  \item הכוח תלוי אך ורק במרחק בינהם. 


\end{enumerate}
\begin{proposition}[הלגרנג'יאן של הבעיה הדו גופית]
$$ \mathcal{L}=\frac12\left(m_1\left|\dot{\vec{r_1}}\right|^2+m_2\left|\dot{\vec{r_2}}\right|^2\right)-V\left(|\vec{r_1}-\vec{r_2}|\right)$$

\end{proposition}
\section{מציאת המסלול בבעיה הדו גופית}

\begin{definition}[מסה מצומצמת]
גודל בעל יחידות של מסה המוגדר בתור:
$$\mu=\frac{1}{\frac{1}{m_{1}}+\frac{1}{m_{2}}}=\frac{m_1m_2}{m_1+m_2}$$
זהו למעשה המסה האפקטיבית של מערכת של שתי גופים.

\end{definition}
\begin{symbolize}
ניתן לסמן:
$$\vec{r_1}=\vec{R}_{cm}+\frac{\mu}{m_1}\vec{r},\quad\vec{r_2}=\vec{R}_{cm}-\frac{\mu}{m_2}\vec{r}$$
כלומר ניתן לפרק את \(\vec{r}_{1},\vec{r}_{2}\) לתנועה של מרכז המסה וסביב מרכז המסה.

\end{symbolize}
\begin{proposition}
ניתן לכתוב את הלגרנג'יאן האפקטיבי של הבעיה הדו גופית בצורה הבאה:
$$\mathcal{L}_{eff}= \frac\mu2\dot{r}^2-\frac{p_\theta^2}{2\mu r^2}-V(r)$$
כאשר \(V(r)\) זה הפוטנציאל, \(p_{\theta}\) זה התנע הזוויתי השמור, \(\mu\) זה המסה המצומצת ו-\(\dot{r}\) זה המהירות סביב מרכז המסה.

\end{proposition}
\begin{proof}
בהנתן הלגרנג'יאן של בעיה דו גופית:
$$ \mathcal{L}=\frac12\left(m_1\left|\dot{\vec{r_1}}\right|^2+m_2\left|\dot{\vec{r_2}}\right|^2\right)-V\left(|\vec{r_1}-\vec{r_2}|\right)$$
כעת ניתן להציב:
$$ \vec{r_1}=\vec{R}_{cm}+\frac{\mu}{m_1}\vec{r},\quad\vec{r_2}=\vec{R}_{cm}-\frac{\mu}{m_2}\vec{r},\quad\mu=\frac{m_1m_2}{m_1+m_2}$$
כאשר נקבע את ציר ה-\(z\) להיות בכיוון של התנע השמור:
$$ \vec{L}=\vec{r}\times m\dot{\vec{r}}\parallel\hat{z}$$
כאשר נזכור כי \(\dot{\vec{R}}_{cm}=0\), ונקבל בעיה מצומצמת למישור
$$ \mathcal{L}=\frac\mu2\left( \dot{r}^2+r^2\dot{\theta}^2 \right)-V(r)$$
נשים לב כי \(\theta\) היא קואורדינטה ציקלית. ונקבל גודל שמור:
$$ \frac d{dt}\left(\frac12r^2\dot{\theta}\right)=0,\quad\frac{dS}{dt}=\frac12r\left(r\dot{\theta}\right)\implies\frac{dS}{dt}=\frac{p_\theta}{2\mu}=\mathrm{const}$$
גודל שמור זה יהיה שווה לשטח של חלק מהמסלול - הוכחה של חוק שני של קפלר.
לעשה התמרת לג'נדר לפי \(\theta\):
$$ \mathcal{L}_{eff}=\mathcal{L}-p_\theta\dot{\theta}=\boxed{ \frac\mu2\dot{r}^2-\frac{p_\theta^2}{2\mu r^2}-V(r) }$$

\end{proof}
\begin{corollary}
האנרגיה ומשוואות התנועה יתנו:
$$E=\frac{\mu}{2}\dot{r}^2+V_{eff}(r) \qquad \mu\ddot{r}=-\frac{dV}{dr}+\frac{p_\theta^2}{\mu r^3}$$
כאשר הגדרנו את הפוטנציאל האפקטיבי להיות \(V_{eff}\equiv\frac{p_\theta^2}{2\mu r^2}+V(r)\).

\end{corollary}
\begin{proof}
וכיוון שלא תלוי בזמן משימור מפוקציית יעקובי נקבל את האנרגיה:
$$ E=\frac{\partial L_{eff}}{\partial\dot{r}}\dot{r}-L_{eff}=\frac{\mu}{2}\dot{r}^2+\frac{p_\theta^2}{2\mu r^2}+V(r)=\boxed{ \frac{\mu}{2}\dot{r}^2+V_{eff}(r) }$$
בכל אופן, ממשוואת אוילר לגרנג' נקבל:
$$ \frac d{dt}\frac{\partial L_{eff}}{\partial\dot{r}}-\frac{\partial L_{eff}}{\partial r}=0\implies \boxed{ \mu\ddot{r}=-\frac{dV}{dr}+\frac{p_\theta^2}{\mu r^3} }$$

\end{proof}
\begin{proposition}[משוואת בינה]
זוהי המשוואה המתארת את הגאומטריה של תנועה מרכזית כללית:
$$ \frac{p_\theta^2}\mu u^2\frac d{d\theta}(-u^{\prime})=F(1/u)+\frac{p_\theta^2}\mu u^3$$
כאשר \(u'=-\frac{r'}{r^{2}}\).

\end{proposition}
\begin{proof}
כיוון שאנו רוצים לתאר את הגאומטריה של המסלול. נעביר את \(r(t)\to r\left( \theta \right)\).
כיוון שאנו יודעים כי \(\frac{d\theta}{dt}=\frac{p_{\theta}}{\mu r^2}\). נסמן \(\frac{dr}{d\theta}=r'\) ונקבל:
$$ \dot{r}=\frac{dr}{dt}=\frac{dr}{d\theta}\frac{d\theta}{dt}=\frac{p_\theta}{\mu r^2}r^{\prime}$$
נציב במשוואת תנועה ונקבל:
$$ \mu\frac{p_\theta^2}{\mu^2r^2}\frac d{d\theta}\left(\frac{r^{\prime}}{r^2}\right)=-\frac{dV}{dr}+\frac{p_\theta^2}{\mu r^3}$$
נשים לב כי אם נציב במסלול \(u=\frac{1}{r}\implies u'=-\frac{r'}{r^2}\) נקבל את האיבר שבסוגריים. לכן ניתן לעשות החלפת משתנים. נסמן את הכוח \(F(r)= -\frac{dV}{dr}\) ונקבל:
$$ \frac{p_\theta^2}\mu u^2\frac d{d\theta}(-u^{\prime})=F(1/u)+\frac{p_\theta^2}\mu u^3$$
מזה ניתן לפשט ולקבל את משוואת בינה(Binet):
$$\boxed{ u^{\prime\prime}+u=-\frac\mu{p_\theta^2u^2}F(1/u)}
$$

\end{proof}
\section{בעיית קפלר}

\begin{definition}[פוטנציאל כבידתי]
פוטנציאל מהצורה:
$$V(r)=\frac{\alpha}{r}$$

\end{definition}
\begin{definition}[מקדם האקסצנטריות]
מספר \(\varepsilon\) המאפיין צורה קונית ומסווג את הצורה שלו:

  \begin{enumerate}
    \item עבור \(\varepsilon=0\) הצורה היא עיגול. 


    \item עבור \(0<\varepsilon<1\) הצורה היא אליפסה. 


    \item עבור \(\varepsilon=1\) הצורה היא פרבולה. 


    \item עבור \(\varepsilon>1\) הצורה היא היפרבולה. 


    \item עבור \(\varepsilon=\infty\) הצורה היא זוג ישרים. 
זה למעשה הזווית קבוע שקשור לזווית שמישור חותך את בצורה הקונית.


  \end{enumerate}
\end{definition}
\begin{proposition}
אם נציב את הפוטנציאל הכבידתי \(V=\frac{\alpha}{r}\) במשוואת בינה נקבל:
$$ u=\frac1r=\frac1{r_0}(1+\epsilon\cos(\theta-\theta_0)),\quad\quad\frac1{r_0}\equiv\frac{\mu\alpha}{p_\theta^2}$$
כאשר \(\varepsilon\) הוא מקדם האקסצנטריות, \(\mu\) זה מסה מצומצמת, \(r_{0}\) זה קבוע(כאשר נראה בהמשך כי זה המרחק המינימלי).

\end{proposition}
\begin{proof}
אם נניח פוטנציאל מרכזי(בעיית קפלר)כלומר \(V(r)=\frac{\alpha}{r}\) נקבל כי:
$$ F(r)=-\frac{dV}{dr}=-\frac\alpha{r^2}=-\alpha u^2$$
נציב במשוואת בינט ונקבל:
$$ u^{\prime\prime}+u=\frac{\mu\alpha}{p_\theta^2}$$
הפתרון של המדר יהיה:
$$ u(\theta)=\frac{\mu\alpha}{p_\theta^2}+A\cos(\theta-\theta_0)$$
ונעבור לסימונים:
$$ u=\frac1r=\frac1{r_0}(1+\epsilon\cos(\theta-\theta_0)),\quad\quad\frac1{r_0}\equiv\frac{\mu\alpha}{p_\theta^2}$$

\end{proof}
\begin{proposition}
אם \(E\) זה האנרגיה של המערכת, מתקיים:
$$\varepsilon=\sqrt{ 1+ \frac{2Ep_{\theta}^{2}}{\mu \alpha^{2}} }$$

\end{proposition}
\begin{proof}
נבטא את \(\epsilon\) בעזרת האנרגיה:
$$ E=\frac\mu2\dot{r}^2+V_{eff}(r)=\frac\mu2\dot{r}^2-\frac\alpha r+\frac{p_\theta^2}{2\mu r^2} =\frac{p_\theta^2}{2\mu}\left(\frac{r^{\prime2}}{r^4}+\frac1{r^2}\right)-\frac\alpha r$$
כאשר נשים לב כי:
$$ \frac{r^{\prime2}}{r^4}=\left[\left(\frac1r\right)^{\prime}\right]^2=\left(  -\frac\epsilon{r_0}\sin\left( \theta-\theta_0 \right) \right)^2=\frac{\epsilon^2}{r_{0}^2}\sin^2\left( \theta-\theta_{0} \right)$$
נציב באנרגיה ונקבל:
$$ \begin{aligned}E&=\frac{p_\theta^2}{2\mu}\left(\frac{\epsilon^2}{r_0^2}\sin^2(\theta-\theta_0)+\frac{1}{r_0^2}\left(1+\epsilon\cos(\theta-\theta_0)\right)^2\right)-\frac{\alpha}{r_0}\left(1+\epsilon\cos(\theta-\theta_0)\right)\end{aligned}$$
כאשר לאחר פישוט נקבל:
$$ \frac{2\mu E}{p_\theta^2}=\frac{\epsilon^2-1}{r_0^2}\implies  \epsilon=\sqrt{1+\frac{2\mu E}{p_\theta^2}r_0^2}\overset{(*)}{\operatorname*{=}}\sqrt{1+\frac{2\mu E}{p_\theta^2}\left(\frac{p_\theta^2}{\mu\alpha}\right)^2}=\sqrt{1+\frac{2Ep_\theta^2}{\mu\alpha^2}}$$
כאשר ב-\((*)\) השתמשנו בזה ש-\(r_{0}^2=\frac{p_{\theta}^2}{\mu \alpha}\).

\end{proof}
\begin{remark}
נראה כי ייתכן שיש בעיה, כיוון שהאנרגיה \(E\) היא לא תמיד חיובית, נראה כאילו ייתכן כי הביטוי בשורש יכול להיות שלילי. ניתן לפרק את האנרגיה לסכום של אנרגיה קינטית ואנרגיה פוטנציאלית - \(E=E_{k}+V\). מספיק להראות כי האנרגיה הקינטית תמיד חיובית. כלומר \(E-V\geq 0\) או \(E\geq V\). כאשר מהטענה הבאה ניתן לראות כי אכן מתקיים.

\end{remark}
\begin{proposition}
הפוטנציאל המינימלי \(V_{min}\) מתקבל ב-\(r_{min}=r_{0}\), ומקיים \(V_{min}=-\frac{\alpha}{2r_{0}}\)

\end{proposition}
\begin{proof}
$$ \begin{aligned}\frac{dV_{eff}}{dr}&=-\frac{\alpha r_0}{r^3}+\frac\alpha{r^2}=0\implies r_{min}=r_0\\V_{min}&=V_{eff}\left(r_0\right)=-\frac\alpha{2r_0}\leq E\end{aligned}$$

\end{proof}
\begin{proposition}
מתקיים עבור תנועה בפוטנציאל כבידתי:
$$\epsilon=\sqrt{1-\frac E{V_{min}}} \qquad r\left(\theta\right)=\frac{r_0}{1+\epsilon\cos\left(\theta-\theta_0\right)}$$

\end{proposition}
\begin{proof}
נציב את הביטוי \(V_{min}=-\frac{\alpha}{2r_{0}}\) בביטוי שמצאנו עבור \(\varepsilon\) ונקבל:
$$ \epsilon\:=\:\sqrt{1-\frac E{V_{min}}}
$$
כלומר:
$$\boxed{ \epsilon\:=\:\sqrt{ 1+\frac{2Ep_{\theta}^2 }{\mu \alpha^2}}\:=\:\sqrt{1-\frac E{V_{min}}}}
$$
וכן ניתן לבטא את הרדיוס כך:
$$\boxed{ r\left(\theta\right)=\frac{r_0}{1+\epsilon\cos\left(\theta-\theta_0\right)}}
$$

\end{proof}
\section{תנודות קטנות}

אנו יודעים כי כל עוד הפונקציה רציפה \(p\) פעמים ברציפות ניתן לפתח פונקציה רב מימדית \(f\) בטור טיילר סביב \(x\) בנקודה מסויימת \(x+v\) ע"י:
$$f(x+v)=f(x)+(D f)_{x}(v)\,+\frac{1}{2}(D^{2}f)_{x}(v,v)+\cdots+\frac{1}{p!}(D^{(p)}f)_{x}(v,\ldots,v)+R_{p}(v),$$
ולכן עבור פונקציה של מסתפר משתנים \(\mathcal{L}\) בנקודה שיווי משקל \(x_{0}\) עבור ערכים \(x\) קרובים ניתן לכתוב:
$$\mathcal{L}\left( \vec{x} \right) \approx  \mathcal{L}\left( \vec{x}_{0} \right)+D\mathcal{L} |_{x_{0}}\left( \vec{x} \right)+\frac{1}{2}D^2\mathcal{L} |_{x_{0}}\left( \vec{x},\vec{x} \right)  $$
ראשית נשים לב כי \(\mathcal{L}\left( \vec{x}_{0} \right)\) קבועה, לכן אינו משפיע על משוואות התנועה וניתן להתעלם ממנו.
כיוון שזוהי פונקציה מ-\(\mathbb{R} ^n\to \mathbb{R}\) מתקיים \(D\mathcal{L}|_{x=x_{0}}=\vec{\nabla} \mathcal{L}\left( \vec{x}_{0} \right)\). בנקודת שיווי משקל הגרדיאנט מתאפס. ולכן ניתן הפוטנציאל מתאפס. בגלל שאיברים מסדר ראשון במהירות הם נגזרת שלמה של הזמן, ניתן גם כן להתעלם מהם, כי לא ישפיעו על משוואות התנועה.
כאשר נזכור כי \(D^2\mathcal{L}\) היא תבנית בילינארית לכן \(D^2\mathcal{L} \left( \vec{x},\vec{x} \right)=\vec{x}^t \left( D^2\mathcal{L} \right)\vec{x}\). בנוסף ניתן לפצל את המטריצה \(D^2\mathcal{L}\) לשני מטריצות, אחת התלויה במהירות המוכללת \(M\) והשנייה התלויה במיקום המוכלל \(K\). כאשר גם כן נפצל את \(\vec{x}=\vec{q}+\dot{\vec{q}}\) באותה צורה. לכן ניתן לכתוב:
$$\mathcal{L} \left( \vec{x} \right)\approx \frac{1}{2}\dot{\vec{q}}^t M \dot{\vec{q}}+\frac{1}{2}\vec{q}^tK\vec{q}$$
וזהו קירוב סדר שני של הלגרנגיאן סביב הנקודת שיווי משקל. כאשר נזכור כי \(M,K\) הם הנגזרות השניות(ההסיאן) של הקורדינטות המוכללות בנקודת שיווי משקל. כלומר:
$$M_{i j}={\frac{\partial^{2}L}{\partial\dot{q}_{i}\partial\dot{q}_{j}}}\mid_{q=0,\dot{q}=0},\;\;\;K_{i j}=-{\frac{\partial^{2}L}{\partial q_{i}\partial q_{j}}}\mid_{q=0,\dot{q}=0}$$
אלו מטריצות מספריות - לא מכילות את הקורדינטות המוכללות. כמו כן, אלו מטריצות סימטריות כיוון שהסיאן סימטרי כל עוד הנגזרת השנייה גזירה ברציפת.
כדי לפתור נדרש להציב במשוואת אויילר לגרנג':
$$\frac{d}{d t}\left(\frac{\partial L}{\partial{\dot{q}}_{i}}\right)-\frac{\partial L}{\partial q_{i}}=0\rightarrow M_{i j}{\ddot{q}}_{j}+K_{i j}q_{j}=0\rightarrow M_{i j}{\ddot{q}}_{j}=-K_{i j}q_{j}$$
כאשר השתמשנו בזה ש-\(M,K\) מטריצות סימטריות. כדי למצוא תדירות בתנודות קטנות נחפש מתי פתרון מהצורה \(q_{i}(t)=\bar{q}_{i}e^{i\omega t}\) ונקבל:
$$\left(K_{i j}-\omega^{2}M_{i j}\right)\bar{q}_{j}=0$$
כלומר מתי \(\bar{q}_{j}\) נמצא בגרעין. קיים פתרון לא טריוויאלי אם"ם \(K-\omega^2M\) לא הפיכה. כלומר נמצא מתי:
$$\operatorname*{det}\left(K_{i j}-\omega^{2}M_{i j}\right)=0$$
ונקבל משוואה עבור \(\omega^2\). 

\subsection{מטוטלת קפיצה}

מערכת של מטוטלת אשר הציר שלה נע לאורך ציר \(x\) עם כוח מאלץ \(f(x,t)\).
\textbf{פתרון המערכת}
משוואת התנועה הכוללת תהיה:
$$m{\ddot{x}}\left(t\right)=-{\frac{d u\left(x\right)}{d x}}+f\left(x,t\right)$$
כאשר \(u(x)\) זה הפוטנציאל, ו-\(f(x,t)\) כוח אפקטיבי אשר תלוי ב-\(t,x\). כוח זה יכול להיות תלוי אף באופן לא לינארי, ולא נדע בהכרח אם אפילו פתיר. נפרק את הפתרון לזמנים ארוכים ולזמנים קצרים. בזמנים ארוכים נצפה כי התנועה תראה חלקה, כאשר בזמנים קצרים נצפה שיהיה רועש. נפרק את התנועה לרכיב "חלק" \(X(x)\) של התנועה הממוצעת ולרכיב של הרעש \(\xi\) - ההפרש בין התנועה האמיתית לתנועה ההחלקה:
$$x\left(t\right)=X\left(t\right)+\xi\left(t\right)$$
כאשר המטרה שלנו היא לבטא את משוואת התנועה עבור \(X(t)\).
נציב במשוואת התנועה ונקבל:
$$m{\ddot{x}}\left(t\right)=m{\ddot{X}}\left(t\right)+m{\ddot{\xi}}\left(t\right)=-{\frac{d u}{d x}}+f\left(x,t\right)$$
כאשר כיוון ש-\(\xi \ll x\) נקבל כי:
$$u\left(x\right)=u\left(X+\xi\right)=u\left(X\right)+u^{\prime}\left(X\right)\xi+\frac{1}{2}u^{\prime\prime}\left(X\right)\xi^{2}+O\left( \xi^3 \right)$$
ומגזירה איבר איבר נקבל:
$$u'(x)=u'(X)+u''(X)\xi+O\left( \xi^2 \right)$$
ובאופן דומה נקבל:
$$f\left(x,t\right)=f\left(X+\xi,t\right)=f\left(X,t\right)+{\frac{\partial f}{\partial X}}\left(X,t\right)\xi$$
נציב במשוואת התנועה ונקבל:
$$m{\ddot{X}}\left(t\right)+m{\ddot{\xi}}\left(t\right)\approx-u^{\prime}\left(X\right)-u^{\prime\prime}\left(X\right)\xi+f\left(X,t\right)+\partial_{X}f\left(X,t\right)\xi$$
כאשר נשים לב כי אם נסתכל בסקלות זמן קצרות כך ש-\(X\) בקושי משתנה אז גם הפוקנציאל \(u\) בקושי משתנה לכן הנגזרות מתאפסות ונקבל:
$$m{\ddot{\xi}}\left(t\right)\approx f\left(X,t\right)\implies \xi(t)\approx \frac{1}{m}\hat{\hat{f}}(X,t)$$
כאשר כובע מסמל אינטגרציה לפי זמן. כעת נסתכל על זמנים ארוכים. נגדיר כעת בצורה פורמלית:
$${\frac{1}{T}}\int_{t}^{t+T}x\left(t\right)d t=X\left(t\right)$$
וזה כיוון שאנחנו מניחים \(\textstyle{\frac{1}{T}}\int_{t}^{t+T}\xi\left(t\right)d t=0\) נסמן \(\frac{1}{T}\int_{t}^{t+T}\equiv\langle\cdot\rangle\) ונקבל:
$$m\ddot{X}\left(t\right)=-u^{\prime}\left(x\right)+\left\langle\frac{\partial f}{\partial X}\xi\right\rangle$$

\begin{proposition}
$$-\left\langle\frac{\partial f}{\partial X}\xi\right\rangle=\frac{\partial}{\partial X}\left\langle\frac{1}{2}\dot{\xi}^{2}\right\rangle$$

\end{proposition}
\begin{proof}
\begin{gather*}\xi=\frac{1}{m}\hat{\hat{f}}\implies \dot{\xi}=\frac{1}{m}\hat{f} \\\frac{\partial }{\partial X}\left\langle  \frac{1}{2}m\dot{\xi}^2 \right\rangle =\frac{m}{2}\frac{\partial }{\partial X}\left( \frac{1}{T}\int_{t- T / 2}^{t+T / 2} \frac{1}{m^2}\hat{\hat{f}^2 } \, \mathrm{d}t  \right)=\dots=-\left\langle  \xi \frac{\partial f}{\partial X} \right\rangle 
\end{gather*}

\end{proof}
וכעת נקבל:
$$m\ddot{X}\left(t\right)=-u^{\prime}\left(x\right)+\left\langle\frac{\partial f}{\partial X}\xi\right\rangle$$
ולכן:
$${{m{\ddot{X}}\left(t\right)=-u^{\prime}\left(x\right)-{\frac{\partial}{\partial X}}\left\langle{\frac{1}{2}}{\dot{\xi}}^{2}\right\rangle=-\left({\frac{\partial u}{\partial X}}+{\frac{\partial}{\partial X}}\left\langle{\frac{1}{2}}{\dot{\xi}}^{2}\right\rangle\right)=-{\frac{\partial}{\partial X}}\left(\underbrace{u\left(X\right)+\left\langle{\frac{1}{2}}{\dot{\xi}}^{2}\right\rangle}_{\equiv u_{eff}\left(X\right)}\right)}}$$
וקיבלנו את הפתרון:
$$m\ddot{X}\left(t\right)=-\frac{\partial}{\partial X}u_{e f f}\left(X\right)$$

\begin{example}
מטוטלת עם נקודת קצה מתנדנדת בכיוון המאונך לפי \(y=a\cos\left( \gamma t \right)\). 
התדירות הטבעית של המטוטלת תהיה \(\omega_{0}=\sqrt{ \frac{g}{l} }\). נניח \(\omega_{0}\ll \gamma\). הכוח על המטוטלת יתקבל ממשוואת התנועה:
$$m\ell^{2}{\ddot{\varphi}}+m g\ell\sin\varphi=\underbrace{-m\ell a\gamma^{2}\cos\left(\gamma t\right)\sin\varphi}_{\equiv f(\varphi,t)}$$
הפוטנציאל יהיה:
$$u=-mg\ell \cos \varphi$$
כאשר כפי שמצאנו הפוטנציאל האפקטיבי יהיה:
$$u_{eff}=u+\left\langle  \frac{1}{2}m\dot{\xi}^2 \right\rangle $$
נחשב את \(\xi\):
$$\begin{array}{c}{{\xi\approx\displaystyle\frac{1}{m}\hat{\tilde{f}}\left(X,t\right)=\left[-m\ell a\gamma^{2}\sin\left(\gamma t\right)\sin\left(\varphi\right)\right]=\displaystyle\frac{m\ell a\gamma^{2}\ddot{\sin\left(\gamma t\right)}\sin\left(\varphi\right)}{\gamma^{2}}=m\ell a\sin\left(\gamma t\right)\sin\left(\varphi\right)}}\\ {{\implies  \xi=m\ell a\sin\left(\gamma t\right)\sin\left(\varphi\right)}} {{\implies \xi^{2}=\ell^{2}a^{2}\gamma^{2}\cos^{2}\left(\gamma t\right)\sin^{2}\left(\varphi\right)}} {{\implies  \left\langle\dot{\xi}^{2}\right\rangle=\frac{1}{2}\ell^{2}a^{2}\gamma^{2}\sin^{2}\varphi}}\end{array}$$
כאשר השתמשנו בזה ש-\(\left\langle  \cos^2\left( \gamma t \right) \right\rangle=\frac{1}{2}\). נקבל כעת:
$$u_{e f f}\left(x\right)=u\left(X\right)+\left\langle\frac{1}{2}\xi^{2}\right\rangle=-m g\ell\cos\varphi+\frac{1}{4}\ell^{2}a^{2}\gamma^{2}\sin^{2}\varphi$$
וניתן לגזור ולראות כי \(\varphi=0\) נקודת שיווי מקשל יציבה, ו-\(\varphi=\pi\) יציבה אם \(a^2\gamma^2>2gl\)(מהנגזרת השנייה).

\end{example}
\Chapter{מכניקת המילטון}

\section{המשוואות הקנוניות ומרחב הפאזה}

\begin{definition}[המילטוניאן]
פונקציה \(\mathcal{\mathcal{H}}\left( q_{i},\frac{\partial \mathcal{\mathcal{L}} } {\partial \dot{q}_{i}},t \right)\) היא ההתמרת לג'נדר של הלגרנג'יאן לפי המהירות המוכללת. כלומר:
$$\mathcal{\mathcal{H}}(q_{i},\,p_{i},\,t)=\sum_{i=1}^{n}\,p_{i}{\dot{q}}_{i}-\mathcal{L}(q_{i},{\dot{q}}_{i},\,t)$$
ונקבל פונקציה של הקורדינטה המוכללת והתנע המוכלל.

\end{definition}
\begin{theorem}[המשוואות הקנוניות]
$$\begin{array}{c c}{{\dot{q}_{i}=\displaystyle\frac{\partial \mathcal{\mathcal{H}}}{\partial p_{i}}}}& {{\dot{p}_{i}=-\displaystyle\frac{\partial \mathcal{\mathcal{H}}}{\partial q_{i}}}}\end{array}$$

\end{theorem}
\begin{proof}
מההגדרה של התמרת לג'נדר אנו יודעים כי:
$$p_{i}=\frac{\partial \mathcal{L}}{\partial\dot{q}_{i}}\quad\quad\dot{q}_{i}=\frac{\partial \mathcal{\mathcal{H}}}{\partial p_{i}}$$
ולכן כבר קיבלנו את אחת המשוואות. אנו יודעים מהתכונות של התמרת לג'נדר כאשר יש משתנים שאנחנו לא מתמירים איתם נקבל:
$${\frac{\partial \mathcal{\mathcal{H}}}{\partial q_{i}}}=-{\frac{\partial \mathcal{L}}{\partial q_{i}}}\quad\quad{\frac{\partial \mathcal{\mathcal{H}}}{\partial t}}=-{\frac{\partial \mathcal{L}}{\partial t}}$$
אנו יודעים מאילר לגרנג' כי:
$$\frac{d}{d t}\left(\frac{\partial \mathcal{L}}{\partial\dot{q}_{i}}\right)=\frac{\partial \mathcal{L}}{\partial q_{i}}\implies \dot{p}_{i}=\frac{\partial \mathcal{L}}{\partial q_{i}}=-\frac{\partial \mathcal{\mathcal{H}}}{\partial q_{i}}$$

\end{proof}
שתי המשוואות האלו נקראות המשוואות הקנוניות. אפשר גם להוכיח ישירות מעקרון המילטון. בעזרתם אפשר לקבל עבור \(n\) דרגות חופש את משוואות התנועה בעזרת \(2n\) משוואות דיפרציאליות מסדר ראשון. 

\subsection{מרחב הפאזה}

נזכור כי מרחב הקונפיגורציות הוא המרחב שמתאר את המיקום של החלקיק בקורדינטות מוכללות. אם אנחנו מתקדמים בזמן ניתן לראות במרחב זה איך נראה המערכת לאורך זמן. מערכת זו אינה מושלמת, כיוון שבהנתן מרחב קונפיגורציה לא נוכל לדעת איך המערכת תהיה לאחר זמן מסויים. כדי לדעת זאת, דרוש גם התנע המוכלל הצמוד של כל אחת מהקורדינטות המוכללות.

\begin{definition}[מרחב הפאזה]
המרחב שכולל את הערך של הקורדינטה המוכללת וגם התנע התמוד שלה

\end{definition}
אם נסרטט על מרחב הפאזה את המערכת לאורך זמן, נראה כי עבור תנאי התחלה נתון - נקודה במרחב הפאזה - קיים מסלול יחיד שיכול לקחת. השיפוע בכל נקודה נקבע ממשוואות המילטון. מהסיבה הזאת לא ייתכן כי שני מסלולים על מרחב הפאזה נחתכים, ולא שווים. אם ננחתכים אז בנקודה אחת יש שני שיפועים שונים.

\section{טרנספורמצייה קנונית}

ראינו מעבר קורדינטות בקורדינטות מוכללות, כלומר מעבר מקורדינטה מוכללת אחת לאחרת. זה משמר את מרחב הקונפיגורציות. כמו כן ראינו את התמרת לג'נדר אשר מעברה מלגרנג'יאן שתלוייה ב-\(\left( \vec{q},\dot{\vec{q}},t \right)\) להמילטוניאן שתלוייה ב-\(\left( \vec{q},\vec{p},t \right)\). כעת נסתכל על מעבר קורדינטות במרחב הפאזה \(\left( \vec{q},\vec{p} \right)\) לקורדינטות אחרות \(\left( \vec{P},\vec{Q} \right)\). מעברי קורדינטות אלו נקראות טרנספורמציות קנוניות. 

\begin{theorem}
בהנתן קורדיטות \((p,q)\) המגדירות מהילטוניאן \(\mathcal{\mathcal{H}}\) ומעבר קורדינטות \((P,Q)\) המגדירות המילטוניאן \(\mathcal{H}'\) מתקיים:
$$\sum p_{i}\dot{q}_{i}-\mathcal{H}=\sum P_{i}\,\dot{Q}_{i}-\mathcal{H}'+{\frac{d F}{d t}}$$
כאשר \(F\) היא פונקציה של המשתנים של המערכת אשר נקראת פונקצייה יוצרת.

\end{theorem}
\begin{proof}
אנו יודעים כי ממשוואות התנועה הקנוניות מתקיים:
$$\dot{q}_{i}=\frac{\partial \mathcal{H}}{\partial p_{i}}\qquad \dot{p}_{i}=-\frac{\partial \mathcal{H}}{\partial q_{i}}\qquad \dot{Q}_{i}=\frac{\partial \mathcal{H}}{\partial P_{i}}\qquad \dot{P}_{i}=-\frac{\partial \mathcal{H}}{\partial Q_{i}}$$
ממשוואות אלו ניתן לקבל את הלגרנג'יאנים:
\begin{gather*}\mathcal{L} _{1}= \sum_{i}p_{i}{\dot{q}}_{i}\,-\,\mathcal{H}\left( q_{i},\,p_{i},\,t \right)\\ \mathcal{L} _{2}= \sum_{i}P_{i}{\dot{Q}}_{i}\,-\,\mathcal{H}'\left( q_{i},\,p_{i},\,t \right) 
\end{gather*}
כיוון שאנו דירושים שמשוואות התנועה לא ישתנו, נדרוש שיהיו שווים עד כדי פונקציה שהיא נגזרת שלמה לפי זמן. כלומר עד כדי הוספה של נגזרת של פונקציה \(F\) בזמן. נקבל:
$$\sum p_{i}\dot{q}_{i}-\mathcal{H}=\sum P_{i}\,\dot{Q}_{i}-\mathcal{H}'+{\frac{d F}{d t}}$$
כאשר נשמיט את הסכימה מטעמי נוחות.
במקרה זה לפני שנשיג ביטוי יותר פשוט יותר נוח להסתכל על פונקציות \(F\) ספציפיות. 
נסתכל על \(4\) מקרים:
$$\begin{array}{l}{{F=F_{1}(q_{i},\,Q_{i},\,t)}}\\ {{F=F_{2}(q_{i},\,P_{i},\,t)}}\\ {{F=F_{3}(p_{i},\,Q_{i},\,t)}}\\ {{F=F_{4}(p_{i},\,P_{i},\,t)}}\end{array}$$
נסתכל לדוגמא על מקרה \(1\). ניתן לכתוב כעת את המשוואה בעזרת כלל השרשרת בצורה:
$$p\dot{q}-\mathcal{H}=P\dot{Q}-\mathcal{H}'+{\frac{\partial F_{1}}{\partial q}}\dot{q}+{\frac{\partial F_{1}}{\partial Q}}\dot{Q}+{\frac{\partial F_{1}}{\partial t}}$$
כאשר לאחר סידור מחדש נקבל:
$$\left(p-\frac{\partial F_{1}}{\partial q}\right)\dot{q}-\left(P+\frac{\partial F_{1}}{\partial\,Q}\right)\dot{Q}=\mathcal{H}-\mathcal{H}'+\frac{\partial F_{1}}{\partial t}$$
כאשר נשים לב כי המשוואות יתקיימו אם:
$$p=\frac{\partial F_{1}}{\partial q}\qquad P=-\frac{\partial F_{1}}{\partial Q}\qquad \mathcal{H}'=\mathcal{H}+\frac{\partial F_{1}}{\partial t}$$
ולמעשה פתרנו את הבעיה של מעברת קורדינטות. אם יש לנו פונקציה יוצרת \(F_{1}\) אז אנחנו יכולים למצוא \(p=p(q,Q,t)\) לפי מה שמצאנו ומשם לבודד את \(Q\) ולקבל את \(Q=Q(p,q,t)\) ואת \(P=P(p,q,t)\) אשר מקיימים את המשוואות המילטון הקנוניות עם המילטוניאן \(\mathcal{H}'=\mathcal{H}+\frac{\partial F_{1}}{\partial t}\). 
עבור \(F_{2}\) אם נבצע תהליך דומה נקבל:
$$\begin{array}{c}{{p_{i}=\displaystyle\frac{\partial F_{2}}{\partial q_{i}}}}\qquad  {{Q_{i}=\displaystyle\frac{\partial F_{2}}{\partial P_{i}}}}\qquad  {{\mathcal{H}'=\mathcal{H}+\displaystyle\frac{\partial F_{2}}{\partial t}}}\end{array}$$
וכן עבור \(F_{3}\):
$$\begin{array}{c}{{q_{i}=-{\displaystyle\frac{\partial F_{3}}{\partial p_{i}}}}}\qquad  {{P_{i}=-{\displaystyle\frac{\partial F_{3}}{\partial\,Q_{i}}}}}\qquad  {{\mathcal{H}'=\mathcal{H}+{\displaystyle\frac{\partial F_{3}}{\partial t}}}}\end{array}$$
ועבור \(F_{4}\):
$$\begin{array}{c}{{q_{i}=-{\displaystyle\frac{\partial F_{4}}{\partial p_{i}}}}}\qquad  {{Q_{i}={\displaystyle\frac{\partial F_{4}}{\partial P_{i}}}}}\qquad  {{\mathcal{H}'=\mathcal{H}+{\displaystyle\frac{\partial F_{4}}{\partial t}}}}\end{array}$$

\end{proof}
\begin{proposition}[תנאים שקולים לטרנספורמציה קנונית]
התנאים הבאים שקולים לזה שהעתקה \((p,q)\mapsto(P,Q)\) היא טרנספורמציה קנונית:

  \begin{enumerate}
    \item קיימת פונקציה יוצרת \(F\) להעתקה 


    \item המשוואות הקנוניות של ההמילטוניאן נשמרות 


    \item הסוגרי פואסון נשמרים. כלומר לכל פונקציות \(f,g\) מתקיים \(\{ f,g \}_{p,q}=\{ f,g \}_{P,Q}\). 


  \end{enumerate}
\end{proposition}
בפועל עבור תנאי 3 לא צריך להראות על כל הפונקציות, אבל מספיק להראות על \(P_{i},Q_{i}\):
$$\begin{array}{l}{{\{P_{i},P_{j}\}_{P,Q}=\left\{P_{i},P_{j}\right\}_{p,q}=0}}\\ {{\{Q_{i},Q_{j}\}_{P,Q}=\left\{Q_{i},Q_{j}\right\}_{p,q}=0}}\\ {{\{P_{i},Q_{j}\}_{P,Q}=\left\{P_{i},Q_{j}\right\}_{p,q}=\delta_{i j}}}\end{array}$$
כאשר במקרה החד מימדי מספיק לבדוק כי \(\{ P,Q \}=1\). נשים לב כי:
$$\left\{{ P},{ Q}\right\}_{p,q}=\frac{\partial{ P}}{\partial p}\frac{\partial{Q}}{\partial q}-\frac{\partial{ P}}{\partial q}\frac{\partial{ Q}}{\partial p}=\operatorname*{det}\begin{pmatrix}\frac{\partial P}{\partial p} & \frac{\partial P}{\partial q} \\\frac{\partial Q}{\partial p} & \frac{\partial Q}{\partial q}
\end{pmatrix}=|{J}|=1$$
ונשים לב כי זה מראה שההעתקה \((p,q)\mapsto(P,Q)\) משמרת נפח במרחב הפאזה.

\section{משוואות המילטון בעזרת סוגרי פואסון}

נזכור כי סוגרי פואסון של פונקציות \(f(p_{i},q_{i},t),g(p_{i},q_{i},t)\) לפי \(\vec{p},\vec{q}\) מוגדר:
$$\{f,g\}:=\sum_{i=1}^{n}\left({\frac{\partial f}{\partial p_{i}}}{\frac{\partial g}{\partial q_{i}}}-{\frac{\partial f}{\partial q_{i}}}{\frac{\partial g}{\partial p_{i}}}\right)$$

כמו כן נשים לב כי מתקיים:
$$\left\{ q_{i},\,q_{j} \right\}=\left\{ p_{i},\,p_{j} \right\}=0 \qquad \left\{ q_{i},\,p_{j} \right\}=-\{p_{i},q_{j}\}=\delta_{i j}$$
דבר חשוב נוסף זה שסוגר פואסון אינווריאנטי לטרנספורמצייה קנונית. כלומר:
$$\{u,\,v\}_{p,q}=\{u,\,v\}_{{P},{ Q}}$$

אם יש לנו פונקציה \(u=u(q_{i},p_{i},t)\) הנגזרת המלאה לפי זמן מכלל השרשרת תהיה:
$${\frac{d u}{d t}}={\frac{\partial u}{\partial q_{i}}}{\dot{q}}_{i}+{\frac{\partial u}{\partial p_{i}}}{\dot{p}}_{i}+{\frac{\partial u}{\partial t}}$$
כאשר ממשוואות המילטון ניתן להחליף את \(\dot{q}_{i},\dot{p}_{i}\) ולקבל:
$${{\displaystyle{\frac{d u}{d t}}=\frac{\partial u}{\partial q_{i}}\frac{\partial \mathcal{\mathcal{H}}}{\partial p_{i}}-\frac{\partial u}{\partial p_{i}}\frac{\partial \mathcal{\mathcal{H}}}{\partial q_{i}}+\frac{\partial u}{\partial t}}} {{\displaystyle{=\{u,\,\mathcal{\mathcal{H}}\}+\frac{\partial u}{\partial t}}}}$$

כאשר נשים לב כי אם \(u\) קבוע אז הנגזרת לפי זמן היא 0 ונקבל \(\frac{du}{dt}=\frac{\partial u}{\partial t}\). מפה ניתן לגזור ישירות את משוואות התנועה. אם \(u=q\) נקבל 
$$\dot{q}=\{q,\mathcal{\mathcal{H}}\}$$
ואם \(u=p\) נקבל:
$${\dot{p}}=\{p,\mathcal{\mathcal{H}}\}$$

\begin{example}
האם \(\mathcal{\mathcal{\mathcal{H}}}\) שמור?
$$\frac{d\mathcal{\mathcal{\mathcal{H}}}}{dt}=\frac{\partial \mathcal{\mathcal{\mathcal{H}}}}{\partial t}+\left\{  \mathcal{\mathcal{\mathcal{H}}},\mathcal{\mathcal{\mathcal{H}}}  \right\}$$
וקיבלנו \(\mathcal{\mathcal{\mathcal{H}}}\) שמור אם"ם $$\frac{\partial \mathcal{\mathcal{\mathcal{H}}}}{\partial t}=0$$\textbf{תכונות שימושיות}$$\{f(u),g(v)\}=\{u,v\}\,\frac{\partial f}{\partial u}\frac{\partial g}{\partial v}$$$$\{f(\vec{p}),g(\vec{q})\}=\sum_{i=1}^{n}\frac{\partial f}{\partial p_{i}}\frac{\partial g}{\partial q_{i}}$$$$\{f\left(\vec{q},\vec{p}\right),q_{i}\}=\frac{\partial f}{\partial p_{i}}\qquad\{f\left(\vec{q},\vec{p}\right),p_{i}\}=-\frac{\partial f}{\partial q_{i}}$$

\end{example}
\section{המילטון יעקובי}

\begin{reminder}
טרנספורמצייה קנונית מעבירה אותנו מהמילטוניאן \(\mathcal{H}=\mathcal{H}(p,q,t)\) להמילטוניאן מותמר מהצורה \(\mathcal{\mathcal{H}}'=\mathcal{\mathcal{H}}'(P,Q,t)\) כאשר:
$${{\dot{Q}_{i}=\displaystyle\frac{\partial\,\mathcal{\mathcal{H}}'}{\partial\,P_{i}}}}\qquad  {{\dot{P}_{i}=-\displaystyle\frac{\partial\,\mathcal{\mathcal{H}}'}{\partial\,Q_{i}}}}\qquad \mathcal{\mathcal{H}}'=\mathcal{H}+\frac{\partial F}{\partial t}$$

\end{reminder}
\begin{proposition}
קיימות מערכת קורדינטות \(\mathbf{P},\mathbf{Q}\) כך ש-\(\dot{\mathbf{P}},\dot{\mathbf{Q}}=0\). 

\end{proposition}
\begin{remark}
מקרה זה מתקבל בפרט במערכת בו ההמילטוניאן \(\mathcal{H}'\) קבוע, ולכן מספיק לדרוש \(\mathcal{H}'=0\).

\end{remark}
\begin{proposition}
כדי למצוא את היוצר של הטרנספוטרמציה הקונונית אשר מאפסת את ההמילטוניאן מספיק לפתור:
$$\mathcal{H}+\frac{\partial F}{\partial t}=0$$
ולפתור משוואה דיפרנציאלית עבור \(F\). 

\end{proposition}
\begin{corollary}
אם נבחר \(F=F_{2}(q,P,t)\) ונסמן ב-\(S\)(מטעמים היסטוריים כיוון שכפי שנראה זה למעשה הפעולה) נקבל:
$${{p_{i}=\displaystyle\frac{\partial S}{\partial q_{i}}}}\qquad  {{Q_{i}=\displaystyle\frac{\partial S}{\partial P_{i}}}}\qquad {{\mathcal{\mathcal{H}}'=\mathcal{H}+\displaystyle\frac{\partial S}{\partial t}}}$$
כאשר המשוואות הנדרשות יהיו:
$$\mathcal{H}(q_{1},\ldots,q_{n};\,p_{1},\ldots,\,p_{n};\,t)+\frac{\partial S}{\partial t}=0$$

\end{corollary}
\begin{remark}
הבחירה \(F=F_{2}(q,P,t)\). זה בחירה סבירה ביותר כיוון שפונקציה כזה לוקחת אותנו מ-\(p\) ל-\(P\) ומ-\(q\) ל-\(Q\).

\end{remark}
\begin{corollary}[משוואת המילטון יעקובי]
כיוון ש-\(p_{i}=\frac{\partial S}{\partial q_{i}}\) משוואה מהצורה:
$$\mathcal{H}(q_{1},\ldots,q_{n};\,\frac{\partial S}{\partial q_{1}},\ldots,\frac{\partial S}{\partial q_{n}};t)+\frac{\partial S}{\partial t}=0$$
זוהי המשוואה הדיפרנציאלית עבור \(S\) עבורה נקבל פונקציה יוצרת שהופכת את הבעיה לטריוויאלית. זו נקראת משוואת המילטון יעקובי.

\end{corollary}
בהנתן צורה פונקציונאלית של \(\mathcal{H}\) ניתן לפתור בשביל \(F\). נזכור כי \(P,Q\) הם תנאי התחלה קבועים. כדי להדגיש את זה נכתוב \(\alpha_{i}\) בשביל \(P_{i}\) ו-\(\beta_{i}\) בשביל \(Q_{i}\). כעת:
$$S=S(q_{i}\,,\,\alpha_{i}\,,\,t)$$
ולכן:
$$p_{i}=\frac{\partial S}{\partial q_{i}}=\frac{\partial S(q_{i},\alpha_{i},t)}{\partial q_{i}}$$$$\beta_{i}=Q_{i}=\frac{\partial S}{\partial\alpha_{i}}=\frac{\partial S(q_{i},\alpha_{i},t)}{\partial\alpha_{i}}$$
אם המשוואה הפיכה ניתן לבטא את \(q_{i}\) בעזרת \(\alpha_{i},\beta_{i},t\)  כלומר \(q_{i}=q_{i}\left( \alpha_{i},\beta_{i},t \right)\). ולכן נוכל לקבל:
$$p_{i}=\frac{\partial}{\partial q_{i}}S(q_{i}(\alpha_{i},\,\beta_{i},t),\alpha_{i},t)$$
שזה פונקציה של \(\alpha_{i},\beta_{i},t\). 

\begin{proposition}[סד"פ כללי לפתרון בעיות בעזרת המילטון יעקובי]
  \begin{enumerate}
    \item פותרים את משוואת המילטון יעקובי: 
$${\frac{\partial S}{\partial t}}+{\mathcal{H}}\left({\mathbf{q}},{\frac{\partial S}{\partial{\mathbf{q}}}},t\right)=0$$
ע"י הפרדת משתנים. אם לא פתיר ע"י הפרדת משתנים, המילטון יעקובי הוא לא כלי חישוב יעיל.


    \item באמצעות המשוואה \(S=S\left( q,t;\alpha \right)\) מוצאים את \(q_{i}\left( \alpha,\beta,t \right)\)


    \item באמצעות המשוואה \(p=\frac{\partial S}{\partial q}\) מוצאים את \(p_{i}\left( \alpha,\beta,t \right)\)


    \item מוצאים את הקבועים \(\alpha,\beta\) בעזרת התנאי התחלה \(p(0),q(0)\)


  \end{enumerate}
\end{proposition}
\begin{example}[אסצילטור הרמוני]
ניתן לכתוב את ההמילטוניאן של אסצילטור הרמוני בצורה הבאה:
$$\mathcal{H}=\frac{1}{2m}\left(p^{2}+m^{2}\omega^{2}q^{2}\right)$$
כאשר \(\omega=\sqrt{ \frac{k}{m} }\). כיוון ש-\(\mathcal{H}\) לא תלוי מפורשות בזמן, \(\mathcal{H}\) קבוע. משוואת המילטון יעקובי במקרה הזה יחסית פשוט כיוון שלא תלוי בזמן וניתן להפריד משתנים.  נכתוב:
$$\frac{1}{2m}\left[\left(\frac{\partial S}{\partial q}\right)^{2}+m^{2}\omega^{2}q^{2}\right]=-\frac{\partial S}{\partial t}$$

\end{example}
כאשר אם נבצע אינטגרל על שני האגפים נקבל פונקציה מהצורה:
$$S(\alpha\,,\,q\,,\,t)=W(\alpha,\,q)+V(\alpha,\,t)$$
כאשר \(\alpha\) הוא קבוע. כאשר ניתן להפריד בצורה כזו, לפתור משוואה כזו היא בדרך כלל פשוטה. למעשה, אם לא ניתן לבצע הפרדה כזו. בדרך כלל ננסה להציב ונראה אם נקבל תוצאה הגיונית ומקיימת את התנאי שפה. נציב את הפרנקציה המופרדת המשוואה ונקבל:
$$\frac{1}{2m}\left[\left(\frac{\partial{\cal W}}{\partial q}\right)^{2}+m^{2}\omega^{2}q^{2}\right]+\frac{\partial{\cal V}}{\partial t}=0$$
כלומר:
$${\frac{1}{2m}}\left[\left({\frac{\partial W}{\partial q}}\right)^{2}+m^{2}\omega^{2}q^{2}\right]=-{\frac{\partial V}{\partial t}}$$
כאשר שני האגפים הם בלתי תלויים והשיוויון נכון לכל הערכים לכן כל אגף שווה לאותו הקבוע אשר נסמן ב-\(\alpha\). כעת מאגף ימין \(V=-\alpha t\). כאשר מאגף שמאל נקבל:
$${\frac{1}{2m}}\left[\left({\frac{d W}{d q_{i}}}\right)^{2}+m^{2}\omega^{2}q^{2}\right]=\alpha$$
כאשר מיידית נקבל:
$$W=\int d q\sqrt{2m\alpha-m^{2}\omega^{2}q^{2}}$$
וכעת קיבלו את \(S\):
$$S=-\alpha t+\int d q\sqrt{2m\alpha-m^{2}\omega^{2}q^{2}}$$
לפי המשוואה שפיתחנו אנו יודעים כי:
$$\beta=\frac{\partial S}{\partial\alpha}=-t+\int d q\frac{2m}{\sqrt{2m\alpha-m^{2}\omega^{2}q^{2}}}=-t+\frac{1}{\omega}\sin^{-1}\left(q\sqrt{\frac{m\omega^{2}}{2\alpha}}\right)$$
כאשר נזכור כי \(\beta\) זה איך שסימנו את התנאי התחלה \(Q\) הקבוע. ולכן:
$$q={\sqrt{\frac{2\alpha}{m\omega^{2}}}}\sin\omega(t+\beta)$$
כאשר באופן דומה ניתן למצוא את \(p\):
$$p=\frac{\partial S}{\partial q}=\sqrt{2m\alpha-m^{2}\omega^{2}q^{2}}=\sqrt{2m\alpha}\left(\sqrt{1-\sin^{2}\omega(t+\beta)}\,\right)$$
כלומר:
$$p={\sqrt{2m\alpha}}\cos\omega(t+\beta)$$

\subsection{פרשנות לפונקציה הבסיסית \(S\)}

$$S=S(q_{1},\,.\,.\,.\,,q_{n};\alpha_{1},.\,.\,.\,,\alpha_{n};\,t)$$
ולכן:
$${\frac{d S}{d t}}=\sum{\frac{\partial S}{\partial q_{i}}}{\frac{d q_{i}}{d t}}+{\frac{\partial S}{\partial t}}$$
וכיוון ש-\(\frac{\partial S}{\partial q_{i}}=p_{i}\) נקבל:
$$\frac{d S}{d t}=\sum p_{i}{\dot{q}}_{i}+\frac{\partial S}{\partial t}$$
וכאשר לפי הגדרת ההמילטוניאן מתקיים \(\mathcal{\mathcal{H}}=\sum p_{i}\dot{q}_{i}-\mathcal{\mathcal{L}}\) נקבל:

$$\frac{d S}{d t}=\mathcal{H}+\mathcal{L}+\frac{\partial S}{\partial t}$$
כאשר אנו יודעים כי \(\mathcal{H}+\frac{\partial S}{\partial t}=0\) ולכן נקבל:
$$\frac{dS}{dt}=\mathcal{\mathcal{L}} \implies S=\int \mathcal{\mathcal{L}}   \, \mathrm{d}t +C  $$
וניתן לראות שקיבלנו את הפעולה עד כדי קבוע.
\end{document}